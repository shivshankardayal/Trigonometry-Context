% -*- mode: context; -*-
\chapter{Trigonometrical Ratios of Any Angle and Sign}
\startitemize[n, 1*broad]
\item Let us solve these one by one:
  \startitemize[i, 1*broad]
  \item $\cos 2A = \cos 60^\circ = \frac{1}{2}$

    $\cos^2A - \sin^2A = \left(\frac{\sqrt{3}}{2}\right)^2 - \left(\frac{1}{2}\right)^2 = \frac{3}{4} - \frac{1}{4} =
    \frac{1}{2}$

    $2\cos^2 - 1 = 2.\left(\frac{\sqrt{3}}{2}\right)^2 - 1 = 2.\frac{3}{4} - 1 = \frac{1}{2}$

  \item $\sin 2A = \sin 60^\circ = \frac{\sqrt{3}}{2}$

    $2\sin A\cos A = 2.\frac{1}{2}.\frac{\sqrt{3}}{2} = \frac{\sqrt{3}}{2}$

  \item $\cos 3A = \cos 90^\circ = 0$

    $4\cos^3A - 3\cos A = 4.\frac{3\sqrt{3}}{8} - 3\frac{\sqrt{3}}{2} = 0$

  \item $\sin 3A = \sin 90^\circ = 1$

    $3\sin A - 4\sin^3 A = 3\frac{1}{2} - 4. \frac{1}{2^3} = \frac{3}{2} - \frac{1}{2} = 1$

  \item $\tan 2A = \tan 60^\circ = \sqrt{3}$

    $\frac{2\tan A}{1 - \tan^2A} = \frac{2.\frac{1}{\sqrt{3}}}{1 - \frac{1}{3}} = \frac{2}{\sqrt{3}}.\frac{3}{2} =
    \sqrt{3}$
  \stopitemize
\item Let us solve these one by one:
  \startitemize[i, 1*broad]
  \item $\sin 2A = \sin 90^\circ = 1$

    $2\sin A\cos A = 2\sin 45^\circ\cos 45^\circ = 2.\frac{1}{\sqrt{2}}.\frac{1}{\sqrt{2}} = 1$

  \item $\cos 2A = \cos 90^\circ = 0$

    $1 - 2\sin^2A = 1 - 2\sin^245^\circ = 1 - 2.\frac{1}{(\sqrt{2})^2} = 1 - 2.\frac{1}{2} = 0$

  \item $\tan 2A = \tan 90^\circ = \infty$

    $\frac{2\tan A}{1 - \tan^2A} = \frac{2.1}{1 - 1^2} = \frac{2}{0} = \infty$
  \stopitemize

\item L.H.S. $= \sin^230^\circ + \sin^245^\circ + \sin^260^\circ$

  $= \frac{1}{2^2} + \frac{1}{(\sqrt{2})^2} + \frac{(\sqrt{3})^2}{2^2} = \frac{1}{4} + \frac{1}{2} + \frac{3}{4} =
  \frac{3}{2} =$ R.H.S.

\item L.H.S. $= \tan^230^\circ + \tan^245^\circ + \tan^260^\circ$

  $= \frac{1}{(\sqrt{3})^2} + 1 + (\sqrt{3})^2 = \frac{1}{3} + 1 + 3 = 4\frac{1}{3} =$ R.H.S.

\item L.H.S. $= \sin 30^\circ\cos 60^\circ + \sin 60^\circ\cos 30^\circ$

  $= \frac{1}{2}\frac{1}{2} + \frac{\sqrt{3}}{2}\frac{\sqrt{3}}{2} = \frac{1}{4} + \frac{3}{4} = 1 =$ R.H.S.

\item L.H.S. $= \cos 45^\circ\cos 60^\circ - \sin 45^\circ\sin 60^\circ$

  $= \frac{1}{\sqrt{2}}\frac{1}{2} - \frac{1}{\sqrt{2}}\frac{\sqrt{3}}{2} = \frac{1}{2\sqrt{2}} - \frac{\sqrt{3}}{2\sqrt{2}}$

  $= \frac{1 - \sqrt{3}}{2\sqrt{2}} = - \frac{\sqrt{3} - 1}{2\sqrt{2}} =$ R.H.S.

\item L.H.S. $= \csc^245^\circ.\sec^230^\circ.\sin^290^\circ.\cos 60^\circ$

  $= (\sqrt{2})^2.\frac{2^2}{(\sqrt{3})^2}.1^2.\frac{1}{2} = 2.\frac{4}{3}.1.\frac{1}{2} = \frac{4}{3} =$ R.H.S.

\item L.H.S. $= 4\cot^245^\circ-\sec^260^\circ + \sin^230^\circ$

  $= 4.1^2 - 2^2 + \frac{1}{2^2} = 4 - 4 + \frac{1}{4} = \frac{1}{4} =$ R.H.S.

\item L.H.S. $= \sin 420^\circ\cos 390^\circ + \cos(-300^\circ)\sin(-330^\circ)$

  $= \sin(360^\circ + 60^\circ)\cos(360^\circ + 30^\circ) + \cos 60^\circ\sin30^\circ$

  $= \sin 60\cos30^\circ + \cos 60^\circ\sin30^\circ = \frac{\sqrt{3}}{2}.\frac{\sqrt{3}}{3} + \frac{1}{2}.\frac{1}{2} =
  \frac{3}{4} + \frac{1}{4} = 1 =$ R.H.S.

\item L.H.S. $= \cos 570^\circ\sin 510^\circ -\sin 330^\circ\cos 390^\circ$

  $= \cos(360^\circ + 210^\circ)\sin(360^\circ + 150^\circ) - \sin(360^\circ - 30^\circ)\cos(360^\circ + 30^\circ)$

  $= \cos(180^\circ + 30^\circ)\sin(180^\circ - 30^\circ) + \sin 30^\circ\cos 30^\circ$

  $= -\cos 30^\circ\sin 30^\circ + \sin 30^\circ\cos 30^\circ$

  $= 0 =$ R.H.S.

\item $\cos \frac{\pi}{3} - \sin \frac{\pi}{3} = \frac{1}{2} - \frac{\sqrt{3}}{2} = \frac{1 - \sqrt{3}}{2}$

  $\tan\frac{\pi}{3} + \cot\frac{\pi}{3} = \sqrt{3}. + \frac{1}{\sqrt{3}} = \frac{4}{\sqrt{3}}$

\item $\cos\frac{2\pi}{3} -\sin\frac{2\pi}{3} = \cos\left(\pi - \frac{\pi}{3})\right) - \sin\left(\pi - \frac{\pi}{3}\right)$

  $= -\cos \frac{\pi}{3} - \sin\frac{\pi}{3} = -\frac{1}{2} - \frac{\sqrt{3}}{2} = - \frac{1 + \sqrt{3}}{2}$

  $\tan\frac{2\pi}{3} + \cot\frac{2\pi}{3} = \tan\left(\pi - \frac{\pi}{3}\right) + \cot\left(\pi - \frac{\pi}{3}\right)$

  $= -\tan \frac{\pi}{3}. -\cot \frac{\pi}{3} = -\sqrt{3} - \frac{1}{\sqrt{3}} = -\frac{4}{\sqrt{3}}$

\item $\cos\frac{5\pi}{4} - \sin\frac{5\pi}{4} = \cos\left(\pi + \frac{\pi}{4}\right) + \sin \left(\pi + \frac{\pi}{4}\right)$

  $= -\cos \frac{\pi}{4} - \sin \frac{\pi}{4} = -\frac{1}{\sqrt{2}} - \frac{1}{\sqrt{2}} = -\sqrt{2}$

  $\tan \frac{5\pi}{4} + \cot \frac{5\pi}{4} = \tan\left(\pi + \frac{\pi}{4}\right) + \cot\left(\pi + \frac{\pi}{4}\right)$

  $= \tan \frac{\pi}{4} + \cot \frac{\pi}{4} = 1 + 1 = 2$

\item $\cos\frac{7\pi}{4} + \cos\frac{7\pi}{4} = \cos \left(\pi + \frac{3\pi}{4}\right) + \sin\left(\pi +
  \frac{3\pi}{4}\right)$

  $= -\cos\frac{3\pi}{4} - \sin\frac{3\pi}{4} = -\cos\left(\pi - \frac{\pi}{4}\right) -\sin \left(\pi -
  \frac{\pi}{4}\right)$

  $= \cos \frac{\pi}{4} -\sin \frac{\pi}{4} = \frac{1}{\sqrt{2}} - \frac{1}{\sqrt{2}} = 0$

  $\tan \frac{7\pi}{4} + \cos \frac{7\pi}{4} = \tan\left(\pi + \frac{3\pi}{4}\right) + \cos \left(\pi +
  \frac{3\pi}{4}\right)$

  $= \tan \frac{3\pi}{4} + \cot \frac{3\pi}{4} = \tan\left(\pi - \frac{\pi}{4}\right) + \cot \left(\pi -
  \frac{\pi}{4}\right)$

  $= -\tan \frac{\pi}{4} - \tan \frac{\pi}{4} = -2$

\item $\cos\frac{11\pi}{3} + \sin \frac{11\pi}{3} = \cos\left(2\pi + \pi + \frac{2\pi}{3}\right) + \sin\left(2\pi + \pi +
  \frac{2\pi}{3}\right)$

  $= \cos\left(\pi + \frac{2\pi}{3}\right) + \sin \left(\pi + \frac{2\pi}{3}\right) = -\cos\frac{2\pi}{3} -
  \sin\frac{2\pi}{3}$

  $= -\cos\left(\pi - \frac{\pi}{3}\right) - \sin\left(\pi - \frac{\pi}{3}\right)$

  $=\cos \frac{\pi}{3} -\sin \frac{\pi}{3} = \frac{1 - \sqrt{3}}{2}$

  $\tan\frac{11\pi}{3} + \cot\frac{11\pi}{3} = \tan\left(2\pi + \pi + \frac{2\pi}{3}\right) + \cot\left(2\pi + \pi +
  \frac{2\pi}{3}\right)$

  $= \tan\left(\pi + \frac{2\pi}{3}\right) + \cot\left(\pi + \frac{2\pi}{3}\right)$

  $= \tan\frac{2\pi}{3} + \cot\frac{2\pi}{3} = \tan\left(\pi - \frac{\pi}{3}\right) + \cot(\pi - \frac{\pi}{3})$

  $= -\tan\frac{\pi}{3} -\cot\frac{\pi}{3} = -\sqrt{3} - \frac{1}{\sqrt{3}} = - \frac{4}{\sqrt{3}}$

\item $\sin$ of an angle is positive in first and second quadrant. Also, $\sin 45^\circ = \frac{1}{\sqrt{2}},$ therefore
  the angles will be $45^\circ$ and $135^\circ.$

\item $\cos$ of an angle is positive in second and third quadrant. Also, $\cos 60^\circ = \frac{1}{2},$ therefore the
  angles will be $120^\circ$ and $240^\circ.$

\item $\tan$ of an angle is negative in second and fourth quadrant. Also, $\tan 45^\circ = 1,$ therefore the angles will
  be $135^\circ$ and $315^\circ.$

\item $\cot$ of an angle is negative in second and fourth quadrant. Also, $\cot 30^\circ = \sqrt{3},$ therefore the
  angles will be $150^\circ$ and $330^\circ.$

\item $\sec$ of an angle is negative in second and third quadrant. Also, $\sec 30^\circ = \frac{2}{\sqrt{3}},$ therefore
  the angles will be $150^\circ$ and $210^\circ.$

\item $\csc$ of an angle is negative in third and fourth quadrant. Also, $\csc 30^\circ = 2,$ therefore the angles
  will be $210^\circ$ and $330^\circ.$

\item $\sin(-65^\circ) = -\sin 65^\circ = -\cos(90^\circ - 65^\circ) = -\cos 25^\circ$

\item $\cos(-84^\circ) = \cos 84^\circ = \sin(90^\circ - 84^\circ) = \sin 6^\circ$

\item $\tan(137^\circ) = \tan(180^\circ - 43^\circ) = -\tan 43^\circ$

\item $\sin(168^\circ) = \sin(180^\circ - 12^\circ) = \sin 12^\circ$

\item $\cos(287^\circ) = \cos(180 + 107^\circ) = -\cos 107^\circ = \sin 17^\circ$

\item $\tan(-246^\circ) = -\tan(246^\circ) = -\tan(180 + 66^\circ) = -\tan 66^\circ = -\tan(90^\circ - 24^\circ) = -\cot
  24^\circ$

\item $\sin 843^\circ = \sin(2*360^\circ + 123^\circ) = \sin(123^\circ) = \sin(90^\circ + 33^\circ) = \cos 33^\circ$

\item $\cos(-928^\circ) = \cos(2*360^\circ + 208^\circ) = \cos(180^\circ + 28^\circ) = -\cos 28^\circ$

\item $\tan 1145^\circ = \tan(3*360^\circ + 65^\circ) = \tan(65^\circ) = \tan(90^\circ - 25^\circ) = \cot 25^\circ$

\item $\cos 1410^\circ = \cos(360*3 + 330^\circ) = \cos(180^ + 180^\circ - 30^\circ) = \cos 40^\circ$

\item $\cot(-1054^\circ) = -\cot(3*360 - 26^\circ) = \cot 26^\circ$

\item $\sec 1327^\circ = \sec(3*360^\circ + 247^\circ) = \sec(180^\circ + 67^\circ) = -\sec 67^\circ = -\sec(90^\circ -
  23^\circ) = -\csc 23^\circ$

\item $\csc (-756^\circ) = -\csc(2*360^\circ + 36^\circ) = -\csc 36^\circ$

\item $\sin 140^\circ + \cos 140^\circ = \sin(90^\circ + 50^\circ) + \cos(180^\circ - 40^\circ) = \cos 50^\circ - \cos
  40^\circ$

  $\cos 40^\circ > \cos 50^\circ$ therefore sign will be negative.

\item $\sin 278^\circ + \cos 278^\circ = \sin(180^\circ + 98^\circ) + \cos(180^\circ + 98^\circ)$

  $= -\sin(98^\circ) - \cos(98^\circ) = -\cos 8^\circ + \cos 82^\circ$

  $\cos 8^\circ > \cos 82^\circ$ therefore sign will be negative.

\item $\sin(-356^\circ) + \cos(-356^\circ) = -\sin(180^\circ + 180^\circ - 4^\circ) + \cos(180^\circ + 180^\circ - 4^\circ)$

  $\sin 4^\circ + \cos 4^\circ$ which will yield a positive sign.

\item $\sin(-1125^\circ) + \cos(-1125^\circ) = -\sin(3*360^\circ + 45^\circ) + \cos(3*360^\circ + 445^\circ)$

  $= -\sin 45^\circ + \cos 45^\circ = 0$ which is neither negative nor positive.

\item $\sin 215^\circ - \cos 215^\circ = \sin(180^\circ + 35^\circ) - \cos(180^\circ + 35^\circ)$

  $= -\sin 35^\circ + \cos 35^\circ$

  $\because \cos 35^\circ > \sin 35^\circ$ the sign will be positive.

\item $\sin 825^\circ - \cos 825^\circ = \sin(2*360^\circ + 105^\circ) - \cos(2*360 + 105^\circ)$

  $= \cos 15^\circ + \sin 15^\circ$ for which sign will be positive.

\item $\sin(-634^\circ) - \cos(-634)^\circ = -\sin (360^\circ + 274^\circ) - \cos (360^\circ + 274^\circ)$

  $= -\sin(180^\circ + 90^\circ + 4^\circ) - \cos(180^\circ + 90^\circ + 4^\circ)$

  $= \cos 4^\circ - \sin 4^\circ$ whic will have positive sign.

\item $\sin(-457^\circ) - \cos(-457^\circ) = -\sin(360^\circ + 90^\circ + 7^\circ) -\cos(360^\circ + 90^\circ + 7^\circ)$

  $=-\cos 7^\circ + \sin 7^\circ$ which will have negative sign.

\item $\cos 135^\circ = -\frac{1}{\sqrt{2}}$ then given $\tan A = -\frac{1}{\sqrt{2}}$

  $\sin A = \pm \frac{1}{\sqrt{3}}, \cos A = \pm \frac{\sqrt{2}}{\sqrt{3}}$

\item $\sin(270^\circ + A) = \sin(180^\circ + 90^\circ + A) = -\sin(90^\circ + A) = -\cos A$

  $\tan(270^\circ + A) = \tan(180^\circ + 90^\circ + A) = \tan(90^\circ + A) = -\cot A$

\item $\cos(270^\circ - A) = \cos(180^\circ + 90^\circ - A) = -\cos(90^\circ - A) = -\sin A$

  $\cot(270^\circ - A) = \cot(180^\circ + 90^\circ - A) = \cot(90^\circ - A) = \tan A$

\item L.H.S. $= \cos A + \sin(270^\circ + A) - \sin(270^\circ - A) + \cos(180^\circ + A)$

  Using results from previous problems we get

  $= \cos A + -\cos A + \cos A - \cos A = 0$

\item L.H.S. $= \sec(270^\circ - A)\sec(90^\circ - A) - \tan(270^\circ - A)\tan(90^\circ + A) + 1$

  $= \sec(180^\circ + 90^\circ - A)\csc A + \tan(180^\circ + 90^\circ - A)\cot A + 1$

  $= -\csc^2A + \cot^2A + 1 = -1 + 1 = 0 =$ R.H.S.

\item L.H.S. $= \cot A + \tan(180^\circ + A) + \tan(90^\circ + A) + \tan(360^\circ - A)$

  $= \cot A + \tan A - \cot A - \tan A = 0 =$ R.H.S.

\item Given, $3\tan^245^\circ - \sin^260^\circ - \frac{1}{2}\cot^230^\circ + \frac{1}{8}\sec^245^\circ$

  $= 3.1^2 - \left(\frac{\sqrt{3}}{2}\right)^2 - \frac{1}{2}(\sqrt{3})^2 + \frac{1}{8}(\sqrt{2})^2$

  $= 3 - \frac{3}{4} - \frac{3}{2} + \frac{2}{8} = 1$

\item Given, $= \frac{\sin(2\pi - 60^\circ).\tan(2\pi - 30^\circ).\sec(2\pi + 60^\circ)}{\tan(\pi - 45^\circ).\sin(\pi +
  30^\circ).\sec(2\pi - 45^\circ)}$

  $= \frac{-\sin 6-^\circ. -\tan 30^\circ.\sec 60^\circ}{-\tan 45^\circ. -\sin 30^\circ.\sec 45^\circ}$

  $= \frac{\frac{\sqrt{3}}{2}.\frac{1}{\sqrt{3}}.2}{1.\frac{1}{\sqrt{2}}.\sqrt{2}} = \sqrt{2}$

\item L.H.S. $= \tan 1^\circ\tan 2^\circ \ldots \tan 89^\circ$

  $= (\tan 1^\circ.\tan(90^\circ - 1^\circ).(\tan 2^\circ.\tan(90^\circ - 2^\circ).\ldots.(\tan 44^\circ.\tan(90^\circ -
  44^\circ).\tan 45^\circ$

  $= (\tan 1^\circ.\cot 1^\circ).(\tan 2^\circ.\cot 2^\circ).\ldots.(\tan 44^\circ.\cot 44^\circ).1$

  $= 1.1.\ldots 1.1 [\because \tan\theta\cot\theta = 1]$

  $= 1 =$ R.H.S.

\item L.H.S. $=(\sin^25^\circ + \sin^285^\circ) + (\sin^210^\circ + \sin^280^\circ) + \ldots + (\sin^240^\circ +
  \sin^250^\circ) + \sin^245^\circ + \sin^290^\circ$

  $= (\sin^25^\circ + \cos^25^\circ) + (\sin^210^\circ + \cos^210^\circ) + \ldots + (\sin^240^\circ + \cos^240^\circ)  +
  \sin^245^\circ + \sin^290^\circ$

  $= 1 + 1 +~\text{8 times}+ \left(\frac{1}{\sqrt{2}}\right)^2 + 1 = 9\frac{1}{2} =$ R.H.S.

\item Given expression can be rewritten as  $= \cos^2 \frac{\pi}{16} + \cos^2 \frac{3\pi}{16} + \cos^2\left(\frac{\pi}{2} -
  \frac{3\pi}{16}\right) + \cos^2\left(\frac{\pi}{2} - \frac{\pi}{16}\right)$

  $= \cos^2 \frac{\pi}{16} + \cos^2 \frac{3\pi}{16} + \sin^2\frac{3\pi}{16} + \sin^2\frac{\pi}{16}$

  $= 1 + 1 = 2$

\item Substituting the values $\left(\frac{2}{\sqrt{3}}\right)^2(\sqrt{2})^2 + (\sqrt{3})^2.1^2$

  $= \frac{4}{3}.2 + 3 = \frac{17}{3}$

\item Substituting the values $(\sqrt{3})^2 - 2.\frac{1}{2^2} - \frac{3}{4}\frac{1}{(\sqrt{2})^2} - 4.\frac{1}{2^2}$

  $= \frac{9}{8}$

\item Given expression is $\frac{\sec\circ(2\pi + 120^\circ).\csc(2\pi + 210^\circ).\tan(2\pi - 30^\circ)}{\sin(2\pi +
  240^\circ).\cos(2\pi + 300^\circ).\cot(2\pi + 45^\circ)}$

  $= \frac{\sec(90^\circ + 30^\circ).\csc(180^\circ + 30^\circ).-\tan 30^\circ}{\sin(180^\circ + 60^\circ).\cos(360^\circ
  - 60^\circ).\cot 45^\circ}$

  $= \frac{-\csc 30^\circ. -\csc 30^\circ. -\tan 30^\circ}{-\sin 60^\circ. \cos 60^\circ.\cot 45^\circ}$

  $= \frac{2.2.\frac{1}{\sqrt{3}}}{\frac{\sqrt{3}}{2}.\frac{1}{2}.1}$

  $= \frac{16}{3}$

\item L.H.S. $\cos^630^\circ + \sin^6 30^\circ = \left(\frac{3}{2}\right)^6 + \frac{1}{2}^6 = \frac{27}{64} + \frac{1}{64} =
  \frac{7}{16}$

  R.H.S. $= 1 - 3\sin^230^\circ\cos^230^\circ = 1 - 3.\frac{1}{2^2}.\frac{3}{2^2} = 1 - \frac{9}{16} = \frac{7}{16}$

  Thus, L.H.S. = R.H.S.

\item L.H.S. $= \left(1 + 1 + \sqrt{2}\right)(1 + 1 - \sqrt{2}) = 4 - 2 = 2 = \csc^2\frac{\pi}{4}$


  59 and 60 are similar to 52 and 51 respectively and have been left as an exercise.


\item L.H.S. $= \sin^2 \frac{\pi}{18} + \sin^2\frac{2\pi}{18} + \ldots + \sin^2\frac{9\pi}{18}$

  $\sin^2\frac{8\pi}{18} = \sin^2\left(\frac{\pi}{2} - \frac{\pi}{18}\right) = \cos^2\frac{\pi}{18}$

  $\sin^2\frac{7\pi}{18} = \sin^2\left(\frac{\pi}{2} - \frac{2\pi}{18}\right) = \cos^2\frac{2\pi}{18}$

  Following similarly, the original expression can be written as

  $\left(\sin^2 \frac{\pi}{18} + \cos^2\frac{\pi}{18}\right) + \left(\sin^2 \frac{2\pi}{18} +
  \cos^2\frac{2\pi}{18}\right) + \left(\sin^2 \frac{3\pi}{18} + \cos^2\frac{3\pi}{18}\right) + \left(\sin^2 \frac{4\pi}{18} +
  \cos^2\frac{4\pi}{18}\right) + \sin^2 \frac{\pi}{2}$

  $= 5 =$ R.H.S.

\item $2n\alpha = \frac{\pi}{2}$

  L.H.S. $= \tan\alpha\tan2\alpha\tan3\alpha. \ldots .\tan(2n - 2)\alpha\tan(2n - 1)\alpha$

  $= \tan\alpha\tan2\alpha\tan3\alpha. \ldots .\cot2\alpha.\cot\alpha$

  $= 1 =$ R.H.S.
\stopitemize
