% -*- mode: context; -*-
\chapter{Trigonometrical Equations}
An equation involving one or more trigonometric ratios of unknown angle is called trigonometric equation.

Ex. $\cos^2x - 4\sin x = 1$

A trigonometric identity is satisfied for every value of the unknown angle whereas trigonometric equation is satisfied for
only some values of unknown angle. For example, $1 - \cos^2x = \sin^2x$ is a trigonometric identity because it is satisfied
for every value of $x.$

\section{Solution of a Trigonometrical Equation}
A value of the unknown angle which satisfies the given trigonometric equation is called a solution or root of the equation.

For example, $2\sin\theta = 1 \Rightarrow \theta = 30^\circ, 150^\circ$ which are two solutions between $0$ and
$2\pi.$

\section{General Solution}
Some trigonometric functions are periodic functions, therefore, solutions of trigonometric equations can be generalized with
the help of periodicirty of trigonometric functions. The solution consisting of all possible solutions of a trigonometric
equation is called its general solution.

For example, $\sin\theta = 0$ has a genral solution which is $n\pi$ where $n\in I.$

Similarly, for $\cos\theta = 0,$ the general solution is $(2n + 1)\frac{\pi}{2},$ where $n\in I$ and for
$\tan\theta = 0$ the solution is again $n\pi.$

\subsection{General Solution of $\sin\theta = \sin\alpha$}
Given, $\sin\theta = \sin\alpha \Rightarrow \sin\theta - \sin\alpha = 0$

$\Rightarrow 2\cos\frac{\theta + \alpha}{2}\sin\frac{\theta - \alpha}{2} = 0$

{\bf Case I:} $\cos\frac{\theta + \alpha}{2} = 0$

$\Rightarrow \theta + \alpha = (2m + 1)\pi, m\in I$

{\bf Case II:} $\sin\frac{\theta - \alpha}{2} = 0$

$\Rightarrow \theta - \alpha = 2m\pi \Rightarrow \theta = 2m\pi + \alpha$

Thus, $\theta = n\pi + (-1)^n\alpha, n\in I$

\subsection{General Solution of $\cos\theta = \cos\alpha$}
Given, $\cos\theta = \cos\alpha \Rightarrow \cos\theta - \cos\alpha = 0$


$\Rightarrow 2\sin\frac{\alpha + \theta}{2}\sin\frac{\theta - \alpha}{2} = 0$

{\bf Case I:} $\sin\frac{\alpha + \theta}{2} = 0$

$\alpha + \theta = 2n\pi \Rightarrow \theta = 2n\pi - \alpha$

{\bf Case II:} $\sin\frac{\theta - \alpha}{2} = 0 \Rightarrow \theta = 2n\pi + \alpha$

Thus, $\theta = 2n\pi \pm\alpha$

\subsection{General Solution of $\tan\theta = \tan\alpha$}
Given $\tan\theta = \tan\alpha \Rightarrow \frac{\sin\theta}{\cos\theta} = \frac{\sin\alpha}{\cos\alpha}$

$\Rightarrow \sin(\theta - \alpha) = 0 \therefore \theta - \alpha = n\pi$

$\theta = n\pi + \alpha$

\section{Principal Value}
For any equation having multiple solutions, the solution having least numerical value is known as {\it principal value.}

Example: Let $\sin\theta = \frac{1}{2}$ then $\theta = \pi/6, 5\pi/6, 13\pi/6, 17\pi/6, \ldots, -7\pi/6, -11\pi/6,
\ldots$

As $\pi/6$ is the least numerical value so it is the principal value in this case.

\subsection{Method for Finding Principal Value}
For this case we consider $\sin\theta = -\frac{1}{2}.$ Since it is negative, $\theta$ will be in third or fourth
quadrant. We can approach this either using clockwise direction or annticlockwise direction. If we take anticlockwise direction
principal value will be greater than $\pi$ and in case of clockwise direction it will be less than $\pi.$ For principal
value, we have to take numerically smallest angle.

So for principal value:
\startitemize[n]
\item If the angle is in 1st or 2nd quadrant we must select anticlockwise direction i.e. principal value will be positive. If the
  angle is in 3rd or 4th quadrant we must select clockwise direction i.e. principal value will be negative.
\item Principal value is always numerically smaller than $\pi$
\item Principal values always lies in the first circle i.e. first rotation.
\stopitemize

\section{Tips for Finding Complete Solution}
\startitemize[n]
\item There should be no extraneous root.
\item There should be no less root.
\item Squaring should be avoided as far as possible. If it is done then check for extraneous roots.
\item Never cancel equal terms containing *unknown* on two sides which are in product. It may cause root loss.
\item The answer should not contain such values of root which may make any of the terms undefined.
\item Domain should not change. If it changes, necessary correction must be made.
\item Check that denominator is not zero at any stage while solving equations.
\stopitemize

\section{Problems}
Find the most general values of $\theta$ satisfying the equations:
\startitemize[n, 1*broad]
\item $\sin\theta = -1$
\item $\cos\theta = -\frac{1}{2}$
\item $\tan\theta = \sqrt{3}$
\item $\sec\theta = -\sqrt{2}$
\stopitemize
Solve the equations:
\startitemize[n, 1*broad, continue]
\item $\sin9\theta = \sin\theta$
\item $\sin5x = \cos2x$
\item $\sin3x = \sin x$
\item $\sin3x = \cos2x$
\item $\sin ax + \cos bx = 0$
\item $\tan x\tan4x = 1$
\item $\cos\theta = \sin105^\circ + \cos 105^\circ$
\stopitemize
Solve the following:
\startitemize[n, 1*broad, continue]
\item $7\cos^2\theta + 3\sin^2\theta = 4$
\item $3\tan(\theta - 15^\circ) = \tan(\theta + 15^\circ)$
\item $\tan x + \cot x = 2$
\item $\sin^2\theta = \sin^2\alpha$
\item $\tan^2x + \cot^2x = 2$
\item $\tan^2x = 3\rm cosec^2x - 1$
\item $2\sin^2x + \sin^22x = 2$
\item $7\cos^2x + 3\sin^2x = 4$
\item $2\cos2x + \sqrt{2\sin x} = 2$
\item $8\tan^2\frac{x}{2} = 1 + \sec x$
\item $\cos x\cos2x\cos3x = \frac{1}{4}$
\item $\tan x + \tan2x + \tan3x = 0$
\item $\cot x - \tan x - \cos x + \sin x = 0$
\item $2\sin^2x - 5\sin x\cos x - 8\cos^2x = -2$
\item $(1 - \tan x)(1 + \sin2x) = 1 + \tan x$
\item Solve for x,($-\pi \leq x \leq \pi$), the equation $2(\cos x + \cos2x) + \sin2x(1 + 2\cos x) = 2\sin x$
\item Find all the solutions of the equation $4\cos^2x\sin x - 2\sin^2x = 3\sin x$
\item $2 + 7\tan^2x = 3.25\sec^2x$
\item Find all the values of $x$ for which $\cos 2x + \cos 4x = 2\cos x$
\item $3\tan x + \cot x = 5\rm cosec x$
\item Find the value of $x$ between $0$ and $2\pi$ for which $2\sin^2x = 3\cos x$
\item Find the solution of $\sin^2x - \cos x = \frac{1}{4}$ in the interbal $0$ to $2\pi.$
\item Solve $3\tan^2x - 2\sin x = 0$
\item Find all values of $x$ satisfying the equation $\sin x + \sin5x = \sin 3x$ between $0$ and $\pi.$
\item $\sin6x = \sin4x - \sin2x$
\item $\cos6x + \cos 4x + \cos 2x + 1 = 0$
\item $\cos x + \cos 2x + \cos 3x = 0$
\item Find the values of $x$ between $0$ and $2\pi,$ for which $\cos3x + \cos2x = \sin\frac{3x}{2} +
    \sin\frac{x}{2}$
\item $\tan x+ \tan2x + \tan3x = \tan x.\tan2x.\tan3x$
\item $\tan x + \tan 2x + \tan x\tan 2x = 1$
\item $\sin2x + \cos2x + \sin x + \cos x + 1 = 0$
\item $\sin x + \sin 2x + \sin 3x = \cos x + \cos 2x + \cos 3x$
\item $\cos6x + \cos4x = \sin3x + \sin x$
\item $\sec4x - \sec2x = 2$
\item $\cos2x = (\sqrt{2} + 1)\left(\cos x - \frac{1}{\sqrt{2}}\right)$
\item Find all the angles between $-pi$ and $\pi$ for which $5\cos2x + 2\cos^2\frac{x}{2} + 1 = 0$
\item $\cot x - \tan x = \sec x$
\item $1 + \sec x = \cot^2\frac{x}{2}$
\item $\cos3x\cos^3x + \sin3x\sin^3x = 0$
\item $\sin^3x + \sin x\cos x + \cos^3x = 1$
\item Find all the value of $x$ between $0$ and $\frac{\pi}{2},$ for which $\sin 7x + \sin4x + \sin x = 0$
\item $\sin x + \sqrt{3}\cos x = \sqrt{2}$
\item Find the values of $x$ for which $27^{\cos2x}.81^{\sin2x}$ is minimum. Also, find this minimum value.
\item If $32\tan^8x = 2\cos^2y - 3\cos y$ and $3\cos2x = 1,$ then find the general value of $y.$
\item Find all the values of $x$ in the interval $\left(-\frac{\pi}{2}, \frac{\pi}{2}\right)$ for which $(1 - \tan
    x)(1 + \tan x)sec^2x + 2^{\tan^2x} = 0$
\item Solve the equation $e^{\cos x} = e^{-\cos x} + 4.$
\item If $(1 + \tan x)(1 + \tan y) = 2.$ Find all the values of $x + y.$
\item If $\tan(\cot x) = \cot(\tan x),$ prove that $\sin 2x = \frac{4}{(2n + 1)\pi}$
\item If $x$ and $y$ are two distinct roots of the equation $a\tan z+ b\sec z = c.$ Prove that $\tan(x + y) =
    \frac{2ac}{a^2 - c^2}$
\item If $\sin(\pi\cos x) = \cos(\pi\sin x),$ prove that
    1. $\cos\left(x \pm \frac{\pi}{4}\right) = \frac{1}{2\sqrt{2}}$
    2. $\sin2x =  -\frac{3}{4}$
\item Determine the smallest positive values of $x$ for which $\tan(x + 100^\circ) = \tan(x + 50^\circ).\tan x.\tan(x -
    50^\circ)$
\item Find the general value of $x$ for which $\tan^2x + \sec 2x = 1.$
\item Solve the equation $\sec x - \rm cosec x = \frac{4}{3}$
\item Find solutions $x\in[0, 2\pi]$ of equation $\sin2x - 12(\sin x - \cos x) + 12 = 0.$
\item Find the smallest positive number r$p$ for which the equation $\cos(p\sin x) = \sin(p\cos x)$ has a solution for
    $x\in [0, 2\pi].$
\item Solve $\cos x + \sqrt{3}\sin x = 2\cos2x$
\item Solve $\tan x+ \sec x = \sqrt{3}$ for $x\in[0, 2\pi].$
\item Solve $1 + \sin^3x + \cos^3x = \frac{3}{2}\sin2x$
\item Solve the equation $(2 + \sqrt{3})\cos x = 1 - \sin x$
\item Solve the equation $\tan\left(\frac{\pi}{2}\sin x\right) = \cot\left(\frac{\pi}{2}\cos x\right)$
\item Solve $8\cos x\cos2x\cos4x = \frac{\sin6x}{\sin x}$
\item Solve $3 - 2\cos x - 4\sin x -\cos 2x + \sin 2x = 0$
\item Solve $\sin x - 3\sin 2x + \sin 3x = \cos x - 3\cos 2x + \cos 3x$
\item Solve $\sin^2x\tan x + \cos^2x\cot x - \sin 2x = 1 + \tan x + \cot x$
\item Find the most general value of $x$ which satisfies both the equations $\sin x= -\frac{1}{2}$ and $\tan x =
    \frac{1}{\sqrt{3}}$
\item If $\tan(x - y) = 1$ and $\sec(x + y) = \frac{2}{\sqrt{3}},$ find the smallest positive values of $x$ and
    $y$ and their most general value.
\item Find the points of intersection of the curves $y = \cos x$ and $y = \sin 3x$ if $-\frac{\pi}{2}\leq x\leq
    \frac{\pi}{2}.$
\item Find all values of $x\in [0, 2\pi]$ such that $r\sin x = \sqrt{3}$ and $r + 4\sin x = 2(\sqrt{3} + 1)$
\item Find the smallest positive values of $x$ and $y$ satisfying $x - y = \frac{\pi}{4}$ and $\cot x + \cot
    y = 2.$
\item Find the general values of $x$ and $y$ such that $5\sin x\cos y = 1$ and $4\tan x = \tan y.$
\item Find all values of $x$ lying between $0$ and $2\pi,$ such that $r\sin x = 3$ and $r = 4(1 + \sin
    x)$
\item If $\sin x = \sin y$ and $\cos x = \cos y$ then prove that either $x = y$ or $x - y = 2n\pi,$ where
    $n\in I.$
\item If $\cos(x - y) = \frac{1}{2}$ and $\sin(x + y) = \frac{1}{2}$ find the smallest positive values of $x$ and
    $y$ and also their most general values.
\item Find the points of intersection of the curves $y = \cos 2x$ and $y = \sin x$ for, $-\frac{\pi}{2}\leq x\leq
    \frac{\pi}{2}.$
\item Find the most general value of $x$ which satisfies the equations $\cos x = \frac{1}{\sqrt{2}}$ and $\tan x = -1.$
\item Find the most general value of $x$ which satisfies the equations $\tan x = \sqrt{3}$ and $\rm cosec x =
    -\frac{2}{\sqrt{3}}$
\item If $x$ and $y$ be two distinct values of $z$ lying between $0$ and $2\pi,$ satisfying the
    equation $3\cos z + 4\sin z = 2,$ find the value of $\sin(x + y).$
\item Show that the equation $2\cos^2\frac{x}{2}\sin^2x = x^2+ x^{-2}$ for $0<x\leq\frac{\pi}{2}$ has no real solution.
\item Find the real value of $x$ such that $y = \frac{3 + 2i\sin x}{1 - 2i\sin x}$ is either real or purely imaginary.
\item Determine for which values of $a$ the equation $a^2 - 2a + \sec^2\pi(a + x) = 0$ has solutions and find them.
\item Find the values of $x$ in  $(-\pi, \pi)$ which satisfy the equation $8^{1 + |\cos x| + \cos^2x + |\cos^3 x| +
    \ldots \text{~to~}\infty} = 4^3$
\item Solve $|\cos x|^{\sin^2x - \frac{3}{2}\sin x + \frac{1}{2}} = 1.$
\item Solve $3^{\sin2x + 2\cos^2x} + 3^{1 -\sin2x + 2\sin^2x} = 28.$
\item If $A = (x/2\cos^2x + \sin x\leq 2)$ and $B = \left(x/\frac{\pi}{2}\leq x\leq \frac{3\pi}{2}\right)$ find
    $A\cap B$
\item Solve $\sin x + \cos x = 1 + \sin x\cos x.$
\item Solve $\sin6x + \cos4x + 2 = 0.$
\item Prove that the equation $\sin2x + \sin3x + \ldots + \sin nx = n - 1$ has n solution for any arbitrary integer
    $n>2.$
\item Solve $\cos^7x + \sin^4x = 1.$
\item Find the number of solutions of the equation $\sin x + 2\sin2x = 3 + \sin3x$ in the interval $0\leq x\leq \pi.$
\item For what value of $k$ the equation $\sin x + \cos(k + x) + \cos(k - x) = 2$ has real solutions.
\item Solve for $x$ and $y,$ the equation $x\cos^3y + 3x\cos y.\sin^2y = 14$ and $x\sin^3y + 3x\cos^2y\sin y
     = 13$
\item Find all the values of $\alpha$ for which the equation $\sin^4x + \cos^4x + \sin2x + \alpha = 0$ is valid.
\item Solve $\tan\left(x + \frac{\pi}{4}\right) = 2\cot x - 1.$
\item If $x, y$ be two angles both satisfying the equation $a\cos 2z + b\sin2z = c,$ prove that $\cos^2x + \cos^2y = \frac{a^2 + ac + b^2}{a^2 + b^2}$
\item If $x_1, x_2, x_3, x_4$ be roots of the equation $\sin(x + y) = k\sin 2x,$ no two of which differ by a multiple of
     $2\pi,$ prove that $x_1 + x_2 + x_3 + x_4 = (2n + 1)\pi.$
\item Show that the equation $\sec x + \rm cosec x = c$ has two roots between $0$ and $\pi$ if $c^2<8$ and four
     roots if $c^2 > 8.$
\item Let $\lambda$ and $\alpha$ be real. Find the set of all values of $\lambda$ for which the system of linear
     equations $\lambda x + y\sin\alpha + z\cos\alpha = 0, x + y\cos\alpha + z\sin\alpha = 0, -x + y\sin\alpha - z\cos\alpha
     = 0$ has non-trivial solution. For $\lambda = 1,$ find all the values of $\alpha.$
\item Find the values of $x$ and $y, 0<x,y<\frac{\pi}{2},$ satisfying the equation $\cos x \cos y\cos(x + y) =
     -\frac{1}{8}$
\item Find the number of distinct real roots of $\startbmatrix\NC \sin x\NC \cos x \NC \cos x \NR\NC\cos x \NC \sin x &\NC \cos x\NR\NC\cos x \NC
     \cos x \NC \sin x\NR\stopbmatrix = 0$ in the interval $-\frac{\pi}{4}\leq x\leq \frac{\pi}{4}.$
\item Find the number of values of $x$ in the interval $[0, 5\pi]$ satisfying the equation $3\sin^2x - 7\sin x + 2
     = 0.$
\item Find the range of $y$ such that the following equation in $x,$ $y + \cos x = \sin x$ has a real
     solution. For $y = 1,$ find $x$ such that $0\leq x\leq2\pi.$
\item Solve $\sum_{r = 1}^n\sin(rx)\sin(r^2x) = 1$
\item Show that the equation $\sin x(\sin x + \cos x) = a$ has real solutions if $a$ is a real number lying between
     $\frac{1}{2}(1 - \sqrt{2})$ and $\frac{1}{2}(1 + \sqrt{2}).$
\item Find the real solutions of the equation $2\cos^2\frac{x^2 + x}{6} = 2^x + 2^{-x}.$
\item Solve the inequality $\sin x\geq \cos2x.$
\item Find the general solution of the equation $\left(\cos\frac{x}{4} - 2\sin x\right)\sin x + \left(1 + \sin \frac{x}{4}
     -2\cos x\right)\cos x = 0$
\item Find the general solution of the equation $2(\sin x -\cos2x) - \sin2x(1 + 2\sin x) + 2\cos x = 0.$
\item Solve $\frac{\sin2x}{\sin\frac{2x + \pi}{3}} = 0.$
\item Solve the equation $3\tan2x - 4\tan3x = \tan^23x\tan2x$
\item Solve the equation $\sqrt{1 + \sin2x} = \sqrt{2}\cos2x.$
\item Show that $x = 0$ is the only solution satisfying the equation $1 + \sin^2ax = \cos x$ where $a$ is
     irrational.
\item Consider the system of linear equarions in $x, y$ and $z, x\sin3\theta -y + z = 0, x\cos2\theta + 4y + 3z = 0,
     2x + 7y + 7z = 0.$ Find the values of $\theta$ for which the system has non-trivial solutions.
\item Find all the solutions of the equation $\sin x + \sin\frac{\pi}{8}\sqrt{(1 - \cos x)^2 + \sin^2x} = 0$ in the interval
     $\left[\frac{5\pi}{2}, \frac{7\pi}{2}\right]$
\item Let $A = \{x: \tan x -\tan^2x > 0\}$ and $y = \left\{x: |\sin x|<\frac{1}{2}\right\}$. Determine $A\cap B.$
\item If $0\leq x\leq 2\pi,$ then solve $2^{\frac{1}{\sin^2x}}\sqrt{y^2 - 2y + 2}\leq 2$
\item If $|\tan x| = \tan x + \frac{1}{\cos x}(0\leq x\leq 2\pi)$ then prove that $x = \frac{7\pi}{6}$ or
     $\frac{11\pi}{6}$
\item Find the smallest positive solution satisfying $\log_{\cos x}\sin x + \log_{\sin x}\cos x = 2$
\item Solve the inequality $\sin x\cos x + \frac{1}{2}\tan x\geq 1$
\item Solve $\tan x^{\cos^2 x} = \cot x^{\sin x}$
\item If $0\leq \alpha, \beta \leq 3,$ then $x^2 + 4 + 3\cos(\alpha x + \beta) = 2x$ has at least one solution, then
     prove thatt $\alpha + \beta = \pi, 3\pi.$
\item Prove that the equation $2\sin x = |x| + a$ has no solution for $a\in \left(\frac{3\sqrt{3 - \pi}}{3}, \infty\right)$
\stopitemize
