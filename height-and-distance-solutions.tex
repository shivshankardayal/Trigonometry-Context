% -*- mode: context; -*-
\chapter{Height and Distance}
\startitemize[n]
\item The diagram \in{Figure}[fig:28_1] is given below:

  \startplacefigure[reference=fig:28_1]
    \externalfigure[28_1.pdf]
  \stopplacefigure

  Let $BC$ be the tower, $A$ the point of observation and $\theta$ as angle of elevation.

  Since the tower is vertical it forms a right-angle triangle with right angle at $B$. Thus,

  $\tan\theta = \frac{BC}{AB} = \frac{100\sqrt{3}}{100} = \sqrt{3} \Rightarrow \theta = 60^\circ$.

\item The diagram \in{Figure}[fig:28_2] is given below:

  \startplacefigure[reference=fig:28_2]
    \externalfigure[28_2.pdf]
  \stopplacefigure

  Let $BC$ be the tower, $A$ the point of observation and the angle of elevation is $30^\circ$.

  Since the tower is vertical it forms a right-angle triangle with right angle at $B$. Thus,

  $\tan30^\circ = \frac{BC}{AB} \Rightarrow BC = 30.\frac{1}{\sqrt{3}} = 10\sqrt{3}$ m.

\item The diagram \in{Figure}[fig:28_3] is given below:

  \startplacefigure[reference=fig:28_3]
    \externalfigure[28_3.pdf]
  \stopplacefigure

  Let $BC$ be the height of kite, $AC$ be the length of string the angle of elevation is $60^\circ$.

  Since the kite would be vertical it forms a right-angle triangle with right angle at $B$. Thus,

  $\sin60^\circ = \frac{BC}{AC} \Rightarrow AC = \frac{60.2}{\sqrt{3}} = 40\sqrt{3}$ m.

\item The diagram \in{Figure}[fig:28_4] is given below:

  \startplacefigure[reference=fig:28_4]
    \externalfigure[28_4.pdf]
  \stopplacefigure

  Let $BC$ be the height of kite, $AC$ be the length of string the angle of elevation is $60^\circ$.

  Since the kite would be vertical it forms a right-angle triangle with right angle at $B$. Thus,

  $\sin60^\circ = \frac{BC}{AC} \Rightarrow BC = 100\frac{\sqrt{3}}{2}= 50\sqrt{3}$ m.

\item The diagram \in{Figure}[fig:28_5] is given below:

  \startplacefigure[reference=fig:28_5]
    \externalfigure[28_5.pdf]
  \stopplacefigure

  Let $BC$ be the pole, $A$ the point where rope is tied to the ground and the angle of elevation is $30^\circ$.

  Since the pole is vertical it forms a right-angle triangle with right angle at $B$. Thus,

  $\sin30^\circ = \frac{BC}{AC} \Rightarrow AC = \frac{12}{\sin30^\circ} = 24$ m.

  Thus, the acrobat has to climb $24$ m.

\item The diagram \in{Figure}[fig:28_6] is given below:

  \startplacefigure[reference=fig:28_6]
    \externalfigure[28_6.pdf]
  \stopplacefigure

  Let $BC$ be the pole, $A$ the point where rope is tied to the ground and the angle of elevation is $30^\circ$.

  Since the pole is vertical it forms a right-angle triangle with right angle at $B$. Thus,

  $\sin30^\circ = \frac{BC}{AC} \Rightarrow BC = 20.\frac{1}{2} = 10$ m.

\item The diagram \in{Figure}[fig:28_7] is given below:

  \startplacefigure[reference=fig:28_7]
    \externalfigure[28_7.pdf]
  \stopplacefigure

  Let the shaded region represent the river and vertical lines the banks. $AB$ represents the bridge, making an angle of
  $45^\circ$ with the bank. Let $BC$ represent the width of river, which clearly makes a right angle triangle with
  right angle at $C$.

  Clearly, $\sin45^\circ = \frac{BC}{AB} \Rightarrow BC = 150.\frac{1}{\sqrt{2}} = 75\sqrt{2}$ m.

  Thus, width of the river is $74\sqrt{2}$ meters.

\item The diagram \in{Figure}[fig:28_8] is given below:

  \startplacefigure[reference=fig:28_8]
    \externalfigure[28_8.pdf]
  \stopplacefigure

  Let $AB$ be the observer, $1.5$ m tall. $CE$ be the tower. Draw line $BD$ parallel to $AC$ which
  will make $CD = 1.5$ m. In right angle triangle $BDE$ angle of elevation $\angle B = 45^\circ$. Given,
  $BD = 28.5$ m. Thus,

  $\tan45^\circ = \frac{DE}{BD} \Rightarrow DE = 28.5$ m. $\therefore CE = CD + DE = 1.5 + 28.5 = 30$ m.

\item The diagram \in{Figure}[fig:28_9] is given below:

  \startplacefigure[reference=fig:28_9]
    \externalfigure[28_9.pdf]
  \stopplacefigure

  $BD$ is the pole and $AC$ is the ladder. $C$ is the point which the electrician need to reach to repair the
  pole which is $1.3$ m below the top of the pole. Total height of the pole is $4$ m, thus, $BC = 4 - 1.3 =
  2.7$ m.

  We are given than ladder makes an angle of $60^\circ$ with the horizontal.

  $\therefore \sin60^\circ = \frac{BC}{AC} \Rightarrow AC = \frac{BC}{\sin60^\circ} = \frac{2.7\sqrt{3}}{2} = 3.12$ m.

\item The diagram \in{Figure}[fig:28_10] is given below:

  \startplacefigure[reference=fig:28_10]
    \externalfigure[28_10.pdf]
  \stopplacefigure

  $A$ is the point of observation. $B$ is the foot of the tower and $C$ is the top of the tower. $CD$ is
  the height of the water tank above the tower. Given $AB = 40$ m .

  In $\triangle ABC, \tan30^\circ = \frac{BC}{AB} \Rightarrow BC = AB.\tan30^\circ = \frac{40}{\sqrt{3}} = 23.1$ m,
  which is height of the toweer.

  In $\triangle ABD, \tan45\circ = \frac{BD}{AB} \Rightarrow BD = AB.\tan45^\circ = 40.1 = 40$ m which is combined height
  of the tower and water tank. Thus, height or depth of the water tank $= CD = BD - BC = 40 - 23.1 = 16.9$ m.

\item The diagram \in{Figure}[fig:28_11] is given below:

  \startplacefigure[reference=fig:28_11]
    \externalfigure[28_11.pdf]
  \stopplacefigure

  The shaded region is the river. $AB$ is the tree and $C$ is the initial point of the observer. $D$ is the final
  point of observation. Given, $CD = 20$ m. Let $AB = h$ m and $AC = x$ m.

  In $\triangle ABC, \tan60^\circ = \frac{h}{x} \Rightarrow h = \sqrt{3}x$

  In $\triangle ABD \tan30^\circ = \frac{h}{x + 20}\Rightarrow \sqrt{3}h = x + 20$

  $\Rightarrow 3x = x + 20 \Rightarrow x = 10$ m. $\Rightarrow h = 10\sqrt{3}$ m.

\item The diagram \in{Figure}[fig:28_12] is given below:

  \startplacefigure[reference=fig:28_12]
    \externalfigure[28_12.pdf]
  \stopplacefigure

  $AC$ is the tree before breaking. Portion $BC$ has borken and has become $BD$ which makes an angle of
  $60^\circ$ with remaining portion of tree standing. If $AB = x$ m, then $BD = 12 - x$ because original height
  of the tree is given as $12$ m.

  In $\triangle ABD, \sin60^\circ = \frac{x}{12 - x} \Rightarrow \frac{\sqrt{3}}{2} = \frac{x}{12 - x} \Rightarrow x = 5.57$

\item The diagram \in{Figure}[fig:28_13] is given below:

  \startplacefigure[reference=fig:28_13]
    \externalfigure[28_13.pdf]
  \stopplacefigure

  $AC$ is the tree before breaking. Portion $BC$ has borken and has become $BD$ which makes an angle of
  $30^\circ$ with remaining portion of tree standing. If $AB = x$ m, then $BD = l - x$ where $l$ is the
  original height of the tree.

  In $\triangle ABD, \sin30^\circ = \frac{x}{l - x} = \frac{1}{2} \Rightarrow 3x = l$

  $\cos30^\circ = \frac{30}{l - x} = \frac{\sqrt{3}}{2} \Rightarrow x = 17.32 \Rightarrow l = 51.96$ m.

\item The diagram \in{Figure}[fig:28_14] is given below:

  \startplacefigure[reference=fig:28_14]
    \externalfigure[28_14.pdf]
  \stopplacefigure

  $AB$ is the tower. Initial observation point is $D$ where angle of elevation is $\alpha$ such that
  $\tan\alpha = \frac{5}{12}$. $C$ is the second point of observation where angle of elevation is $\beta$ such
  that $\tan\beta = \frac{3}{4}$. Given, $CD = 192$ meters. Let $h$ be the height of the tower and $x$ be
  the distance of $C$ from the foot of the tower i.e. $A$.

  In $\triangle ABC, \tan\beta = \frac{3}{4} = \frac{h}{x}$

  In $\triangle ABD, \tan\alpha = \frac{5}{12} = \frac{h}{x + 192} \Rightarrow h = 180$ meters.

\item The diagram \in{Figure}[fig:28_15] is given below:

  \startplacefigure[reference=fig:28_15]
    \externalfigure[28_15.pdf]
  \stopplacefigure

  $AB$ is the tower. When the sun's altittude is $45^\circ$ the shadow reached $C$. When the shadow reached the
  altitude of sun becomes $30^\circ$. Let $h$ meters be the height and $x$ meters be the distance of of initial
  point of observation from foot of the tower. Given $CD = 10$ meters.

  In $\triangle ABC, \tan45^\circ = 1 = \frac{h}{x}\Rightarrow x = h$

  In $\triangle ABD, \tan30^\circ = \frac{1}{\sqrt{3}} = \frac{h}{x + 10}\Rightarrow h = \frac{10}{\sqrt{3} - 1} = 13.66$
  meters.
  x
\item The diagram \in{Figure}[fig:28_16] is given below:

  \startplacefigure[reference=fig:28_16]
    \externalfigure[28_16.pdf]
  \stopplacefigure

  This problem is same as previous problem, where $10$ m is replaced by $1$ km. Processing similarly, we obtain
  $h = 1.366$ km.

\item The diagram \in{Figure}[fig:28_17] is given below:

  \startplacefigure[reference=fig:28_17]
    \externalfigure[28_17.pdf]
  \stopplacefigure

  This problem is same as two previous problems. The height of the mountain is $5.071$ km.

\item The diagram \in{Figure}[fig:28_18] is given below:

  \startplacefigure[reference=fig:28_18]
    \externalfigure[28_18.pdf]
  \stopplacefigure

  This problem is same as $11$-th. Proceeding similarly, we find width of river as $20$ m and height of the tree as
  $20\sqrt{3}$ m.

\item The diagram \in{Figure}[fig:28_19] is given below:

  \startplacefigure[reference=fig:28_19]
    \externalfigure[28_19.pdf]
  \stopplacefigure

  Height of the plane is $1200$ m which is $AB$. The ships are located at $C$ and $D$. Let $CD = d$
  m and $AC = x$ m.

  In $\triangle ABC, \tan60^\circ = \frac{1200}{x} \Rightarrow x = \frac{1200}{sqrt{3}} = 400\sqrt{3}$ m.

  In $\triangle ABC, \tan30^\circ = \frac{1200}{x + d} \Rightarrow x + d = 1200\sqrt{3} \Rightarrow d = 800\sqrt{3}$ m.

\item The diagram \in{Figure}[fig:28_21] is given below:

  \startplacefigure[reference=fig:28_21]
    \externalfigure[28_21.pdf]
  \stopplacefigure

  Let $AB$ be the flag staff having height $h$ and $AC$ be the shadow when sun's altitude is
  $60^\circ$. Let $AD$ be the shadow when sun's altitude is $\theta^\circ$. If we let $AC = x$ m then
  $AD = 3x \Rightarrow CD = 2x$.

  In $\triangle ABC, \tan60^\circ = \frac{h}{x} \Rightarrow h = \sqrt{3}x$.

  In $\triangle ABD \tan\theta = \frac{h}{3x} = \frac{1}{\sqrt{3}}\Rightarrow \theta = 30^\circ$.

\item The diagram \in{Figure}[fig:28_20] is given below:

  \startplacefigure[reference=fig:28_20]
    \externalfigure[28_20.pdf]
  \stopplacefigure

  Let $AB$ be the height of the plane, equal to $200$ m. Let the shaded region present the river such that width
  $CD = x$ m.

  In $\triangle ABD, \tan45^\circ = \frac{200}{AD} \Rightarrow AD = 200$ m.

  Clearly, $AC = 200 - x$ m. In $\triangle ABC, \tan60^\circ = \frac{200}{200 - x} \Rightarrow x = 84.53$ m.

\item The diagram \in{Figure}[fig:28_22] is given below:

  \startplacefigure[reference=fig:28_22]
    \externalfigure[28_22.pdf]
  \stopplacefigure

  Let $AC$ and $BD$ represent the towers having height $h$. Given the distance between towers is $100$ m
  which is $CD$. Let the point of observation be $E$ which is at distance $x$ from $C$ and $100 - x$
  from $D$. Angle of elevations are given as $30^\circ$ and $60^\circ$.

  In $\triangle ACE, \tan60^\circ = \sqrt{3} = \frac{h}{x}\Rightarrow h = \sqrt{3}x$.

  In $\triangle BDE, \tan30^\circ = \frac{1}{\sqrt{3}} = \frac{h}{100 - x} \Rightarrow x = 25, h = 25\sqrt{3}$ m.

\item The diagram \in{Figure}[fig:28_23] is given below:

  \startplacefigure[reference=fig:28_23]
    \externalfigure[28_23.pdf]
  \stopplacefigure

  Let $AB$ be the light house, $C$ and $D$ are the two locations of the ship. The height of the light house is
  given as $100$ m. The angle of elevations are given as $30^\circ$ and $45^\circ$. Let $AC = y$ m and
  $CD = x$ m.

  In $\triangle ABC, \tan45^\circ = 1 = \frac{100}{y} \Rightarrow y = 100$.

  In $\triangle ABD, \tan30^\circ = \frac{1}{\sqrt{3}} = \frac{100}{x + y} = \frac{100}{100 + x} \Rightarrow x = 73.2$ m.

\item The diagram \in{Figure}[fig:28_24] is given below:

  \startplacefigure[reference=fig:28_24]
    \externalfigure[28_24.pdf]
  \stopplacefigure

  The diagram represents the top $PQ$ and $XY$ as given in the problem. The angle of elevations are also given. Draw
  $YZ$ parallel to $ZQ$ and thus, $PZ = 40$ m. Let $ZQ = x$.

  In $\triangle QYZ, \tan45^\circ = 1 = \frac{ZQ}{YZ} \Rightarrow YZ = x$ m. Thus, $PX = x$ m.

  In $\triangle PQX, \tan60^\circ = \sqrt{3} = \frac{x + 40}{x} \Rightarrow x = \frac{40}{\sqrt{3} - 1}$ m.

  Height of toewr is $x + 40 = \frac{40\sqrt{3}}{\sqrt{3} - 1}$

  In $\triangle PQX, \sin60^\circ = \frac{\sqrt{3}}{2} = \frac{PQ}{XQ} \Rightarrow XQ = \frac{80}{\sqrt{3} - 1}$ m.

\item The diagram \in{Figure}[fig:28_25] is given below:

  \startplacefigure[reference=fig:28_25]
    \externalfigure[28_25.pdf]
  \stopplacefigure

  Let $AB$ and $CD$ are the houses. Given $CD = 15$ m. Let the width of the street is $AC = ED = x$ m. The
  angle of depression and elevation are given as $45^\circ$ and $30^\circ$ respectively. Draw $ED\parallel AC$.

  In $\triangle ACD, \tan45^\circ = 1 = \frac{CD}{AC} \Rightarrow AC = 15$ m. Thus, $ED$ is also $15$ m because
  $ED$ is paralle to $AC$.

  In $\triangle BED, \tan30^\circ = \frac{1}{\sqrt{3}} = \frac{BE}{ED} \Rightarrow BE = 5\sqrt{3}$ m.

  Thus, total height of the house $= 15 + 5\sqrt{3} = 23.66$ m.

\item The diagram \in{Figure}[fig:28_26] is given below:

  \startplacefigure[reference=fig:28_26]
    \externalfigure[28_26.pdf]
  \stopplacefigure

  Let $AB$ represent the building and $CD$ the tower. Let $CD = h$ m and given $AB = 60$ m. Also, let
  $AC = x$ m. Draw $DE\parallel AC$, thus $CE = x$ m and $AE = h$ m.

  The angles of depression are given which would be same as angle of elevation from top and bottom of tower.

  In $\triangle ABC, \tan60^\circ = \sqrt{3} = \frac{60}{x} \Rightarrow x = 20\sqrt{3}$ m.

  In $\triangle ADE, \tan30^\circ = \frac{1}{\sqrt{3}} = \frac{BE}{x}\Rightarrow BE = 20$ m.

  $\therefore$ Height of the building $CD = AE = AB - BE = 60 - 20 = 40$ m.

\item The diagram \in{Figure}[fig:28_27] is given below:

  \startplacefigure[reference=fig:28_27]
    \externalfigure[28_27.pdf]
  \stopplacefigure

  Let $CD$ represent the deck of the ship with height $10$ m and $AB$ the hill. The water level is
  $AC$. Draw $DE||AC$ and let $AC = DE = x$ m.

  The angle of elevation are shown as given in the question.

  In $\triangle ACD, \tan30^\circ = \frac{1}{\sqrt{3}} = \frac{CD}{x} \Rightarrow x = 10\sqrt{3}$ m.

  In $\triangle BDC, \tan60^\circ = \sqrt{3} = \frac{BE}{x} \Rightarrow BE = 30$ m.

  Thus, height of the hill $= AE + BE = 10 + 30 = 40$ m.

\item The diagram \in{Figure}[fig:28_28] is given below:

  \startplacefigure[reference=fig:28_28]
    \externalfigure[28_28.pdf]
  \stopplacefigure

  Let $CE$ be the line in which plane is flying and $ABD$ be the horizontal ground. Since the plane is flying at a
  constant height of $3600\sqrt{3}$ m, we have $BC = DE = 3600\sqrt{3}$ m. Let $AB = x$ m and $BD = y$ m.

  In $\triangle ABC, \tan60^\circ = \sqrt{3} = \frac{3600\sqrt{3}}{x} \Rightarrow x = 3600$ m.

  In $\triangle ADE, \tan40^\circ = \frac{1}{\sqrt{3}} = \frac{3600\sqrt{3}}{x + y} \Rightarrow y = 7200$ m.

  Thus, the plane flies $7200$ m in $30$ s. Speed of plane $= \frac{7200}{30}.\frac{3600}{1000} = 284$ km/hr.

\item The diagram \in{Figure}[fig:28_29] is given below:

  \startplacefigure[reference=fig:28_29]
    \externalfigure[28_29.pdf]
  \stopplacefigure

  Let $AC$ be the river and $BD$ be the tree on the island in the river. Given wdith of the river $AC$ as
  $100$ m. Let $BC = x$ m $\Rightarrow AB = 100 - x$ m. The angles of elevation are shown as given in the
  question. Let $BD = h$ m be the height of the tower.

  In $\triangle BCD, \tan45^\circ = 1 = \frac{h}{x} \Rightarrow h = x$ m.

  In $\triangle ABC, \tan30^\circ = \frac{1}{\sqrt{3}} = \frac{h}{100 - x} \Rightarrow x = \frac{100}{\sqrt{3} + 1} = h$ m.

\item The diagram \in{Figure}[fig:28_30] is given below:

  \startplacefigure[reference=fig:28_30]
    \externalfigure[28_30.pdf]
  \stopplacefigure

  Let $AB$ be the first tower and $CD$ be the second tower. Given $AC = 140$ m and $CD = 40$ m. Let
  $AC$ be the horizontal plane. Draw $DE\parallel AC \Rightarrow DE = 140$ m and $AE = 60$ m. Angle of elevation
  is shown as given in the question from top of second tower to top of first tower to be $30^\circ$.

  In $\triangle BDE, \tan30^\circ = \frac{1}{\sqrt{3}} = \frac{BE}{140} \Rightarrow BE = \frac{140}{\sqrt{3}}$ m.

  Thus, total height of first tower is $\frac{140}{\sqrt{3}} + 60$ m.

\item The diagram \in{Figure}[fig:28_31] is given below:

  \startplacefigure[reference=fig:28_31]
    \externalfigure[28_31.pdf]
  \stopplacefigure

  Let $AD$ be the horizontal ground. Let $AB$ and $AC$ be the heights at which planes are flying. Given
  $AC = 4000$ m. Also, given are angles of elevation of the two aeroplanes. Let point of observation be $D$ and
  $AD = b$ m.

  In $\triangle ACD, \tan60^\circ = \sqrt{3} = \frac{AC}{AD}\Rightarrow b = \frac{4000}{\sqrt{3}}$ m.

  In $\triangle ABD, \tan45^\circ = 1 = \frac{AB}{AD} \Rightarrow AB = b = \frac{4000}{\sqrt{3}}$ m.

  Therefore, distance between heights of two planes $= 4000.\frac{\sqrt{3} - 1}{\sqrt{3}}$ m.

\item The diagram \in{Figure}[fig:28_32] is given below:

  \startplacefigure[reference=fig:28_32]
    \externalfigure[28_32.pdf]
  \stopplacefigure

  Let $BC$ be the tower where $B$ is the foot of the toewr. Let $A$ be the point of observation. Given
  $\angle BAC = 60^\circ$.

  In $\triangle ABC, \tan60^\circ = \sqrt{3} = \frac{BC}{AB} \Rightarrow BC = 20\sqrt{3}$ m.

\item The diagram \in{Figure}[fig:28_33] is given below:

  \startplacefigure[reference=fig:28_33]
    \externalfigure[28_33.pdf]
  \stopplacefigure

  Let $BC$ be the wall and $AC$ the ladder. Given distance of the foot of the ladder is $9.5$ m away from the
  wall i.e. $AB = 9.5$ m. The angle of elevation is given as $\angle BAC = 60^\circ$.

  In $\triangle ABC, \cos60^\circ = \frac{1}{2} = \frac{AB}{AC} \Rightarrow AC = 19$ m.

\item The diagram \in{Figure}[fig:28_34] is given below:

  \startplacefigure[reference=fig:28_34]
    \externalfigure[28_34.pdf]
  \stopplacefigure

  Let $BC$ be the wall and $AC$ the ladder. Given distance of the foot of the ladder is $2$ m away from the
  wall i.e. $AB = 2$ m. The angle of elevation is given as $\angle BAC = 60^\circ$.

  In $\triangle ABC, \tan60^\circ = \sqrt{3} = \frac{BC}{AC} \Rightarrow BC = 2\sqrt{3}$ m.

\item The diagram \in{Figure}[fig:28_35] is given below:

  \startplacefigure[reference=fig:28_35]
    \externalfigure[28_35.pdf]
  \stopplacefigure

  Let $BC$ be the electric pole, having a height of $10$ m. Let $AC$ be the length of wire. The angle of
  elevation is given as $\angle BAC=45^\circ$.

  In $\triangle ABC, \sin45^\circ = \frac{1}{\sqrt{2}} = \frac{BC}{AC} \Rightarrow AC = 10\sqrt{2}$ m.

\item The diagram \in{Figure}[fig:28_36] is given below:

  \startplacefigure[reference=fig:28_36]
    \externalfigure[28_36.pdf]
  \stopplacefigure

  Let $BC$ represent the height of kite. Given $BC = 75$ m. Let $AC$ represent the length of the string. The
  angle of elevation is given as $60^\circ$.

  In $\triangle ABC, \sin60^\circ = \frac{\sqrt{3}}{2} = \frac{BC}{AC} \Rightarrow AC = 50\sqrt{3}$ m.

\item The diagram \in{Figure}[fig:28_37] is given below:

  \startplacefigure[reference=fig:28_37]
    \externalfigure[28_37.pdf]
  \stopplacefigure

  Let $BC$ represent the wall and $AC$ the ladder. Given that the length of ladder is $15$ m. The angle of
  elevation of the wall from foot of the tower is given as $60^\circ \Rightarrow \angle BAC = 60^\circ$.

  In $\triangle ABC, \sin60^\circ = \frac{\sqrt{3}}{2} = \frac{BC}{AC} \Rightarrow BC = \frac{15\sqrt{3}}{2}$ m.

\item The diagram \in{Figure}[fig:28_38] is given below:

  \startplacefigure[reference=fig:28_38]
    \externalfigure[28_38.pdf]
  \stopplacefigure

  Let $BC$ be the tower and $CD$ be the flag staff, the heights of which are to be found. Let $A$ be the point
  of obsevation. Given that $AB = 70$ m. The angle of elevation of the foot and the top of flag staff are given as
  $45^\circ$ and $60^\circ$ i.e. $\angle BAC = 45^\circ$ and $\angle BAD = 60^\circ$.

  In $\triangle ABC, \tan45^\circ = 1 = \frac{BC}{AB} \Rightarrow BC = 70$ m, which is height of the tower.

  In $\triangle ABD, \tan60^\circ = \sqrt{3} = \frac{BD}{AB} \Rightarrow BD = 70\sqrt{3}$ m, which is combined height of
  tower and flag staff. Thus, $CD = 70(\sqrt{3} - 1)$ m, which is height of flag staff.

\item This problem is same as 12. Put $15$ instead of $12$.

\item The diagram \in{Figure}[fig:28_40] is given below:

  \startplacefigure[reference=fig:28_40]
    \externalfigure[28_40.pdf]
  \stopplacefigure

  Let $AB$ be the tower and $BC$ the flag staff, whose height is $5$ m. Let $D$ be the point of
  observation. Given that angle of elevation of the foot of the flag staff is $30^\circ$ and that of top is
  $60^\circ$ i.e. $\angle ADB = 30^\circ$ and $\angle ADC = 60^\circ$. Let $AB = h$ m and $AD =
  x$ m.

  In $\triangle ABD, \tan30^\circ = \frac{1}{\sqrt{3}} = \frac{h}{x} \Rightarrow x = \sqrt{3}h$ m.

  In $\triangle ACD, \tan60^\circ = \sqrt{3} = \frac{h + 5}{x} \Rightarrow h = 2.5~{\rm m}, x = 2.5\sqrt{3}$ m.

\item This problem is same as 15. Put $50$ m instead of $10$ m and $60^\circ$ instead of $45^\circ$.

\item This problem is similar to 15. Put $45^\circ$ instead of $30^\circ$ and $60^\circ$ instead of
  $30^\circ$.

\item The diagram \in{Figure}[fig:28_43] is given below:

  \startplacefigure[reference=fig:28_43]
    \externalfigure[28_43.pdf]
  \stopplacefigure

  Let $AB$ be the current height of the skydiver as $h$ m. $C$ and $D$ are two points observed at angle of
  depression $45^\circ$ and $60^\circ$ which woule be equal to angle of elevation from these points. Given that
  $CD = 100$ m. Let $AC = x$ m.

  In $\triangle ABD, \tan45^\circ = 1 = \frac{AB}{AD} = \frac{h}{x + 100} \Rightarrow h = x + 100$ m.

  In $\triangle ABC, \tan60^\circ = \sqrt{3} = \frac{AB}{AC} = \frac{h}{x} \Rightarrow h = \sqrt{3}x$ m.

  $\Rightarrow x = \frac{100}{\sqrt{3} - 1}, h = \frac{100\sqrt{3}}{\sqrt{3} - 1}$ m.

\item The diagram \in{Figure}[fig:28_44] is given below:

  \startplacefigure[reference=fig:28_44]
    \externalfigure[28_44.pdf]
  \stopplacefigure

  Let $AB$ be the tower having a height of $150$ m. Let $C$ and $D$ be the points observed such that
  $\angle ADB = 45^\circ$ and $\angle ACB = 60^\circ$. Let $AC = y$ m and $CD = x$ m. We have to find
  $x$.

  In $\triangle ABC, \tan60^\circ = \sqrt{3} = \frac{AB}{AC} = \frac{150}{y} \Rightarrow y = 50\sqrt{3}$ m.

  In $\triangle ABD, \tan45^\circ = 1 = \frac{AB}{AD} = \frac{150}{x + y}\Rightarrow x = 150 - 50\sqrt{3}$ m.

\item The diagram \in{Figure}[fig:28_45] is given below:

  \startplacefigure[reference=fig:28_45]
    \externalfigure[28_45.pdf]
  \stopplacefigure

  Let $AB$ be the towerr having a height of $h$ m. Let $C$ and $D$ be the points observed such that
  $\angle ADB = 30^\circ$ and $\angle ACB = 60^\circ$. Let $AC = x$ m. Given $CD = 150$ m.

  In $\triangle ABC, \tan60^\circ = \sqrt{3} = \frac{AB}{BC} = \frac{h}{x} \Rightarrow h = \sqrt{3}x$ m.

  In $\triangle ABD, \tan30^\circ = \frac{1}{\sqrt{3}} = \frac{AB}{AD} = \frac{150}{x + 150} \Rightarrow h = 75\sqrt{3}$ m.

\item The diagram \in{Figure}[fig:28_46] is given below:

  \startplacefigure[reference=fig:28_46]
    \externalfigure[28_46.pdf]
  \stopplacefigure

  Let $AB$ be the towerr having a height of $h$ m. Let $C$ and $D$ be the points observed such that
  $\angle ADB = 30^\circ$ and $\angle ACB = 60^\circ$. Let $AC = x$ m. Given $CD = 100$ m.

  In $\triangle ABC, \tan60^\circ = \sqrt{3} = \frac{AB}{BC} = \frac{h}{x} \Rightarrow h = \sqrt{3}x$ m.

  In $\triangle ABD, \tan30^\circ = \frac{1}{\sqrt{3}} = \frac{AB}{AD} = \frac{h}{x + 100} \Rightarrow x = 50$ m.

  Thus, $h = 50\sqrt{3}$ m. Distance of initial point $= x + 100 = 150$ m.

\item The diagram \in{Figure}[fig:28_47] is given below:

  \startplacefigure[reference=fig:28_47]
    \externalfigure[28_47.pdf]
  \stopplacefigure

  Let $AB$ be the tower and $CD$ be the building. Given $CD = 15$ m. $AC$ is the horizontal plane joining
  foot of the building and foot of the tower having width $x$ m. Draw $DE||AC$ then $DE = x$ m and $AE =
  15$ m.

  In $\triangle BDE, \tan30^\circ = \frac{h}{x}\Rightarrow x = \sqrt{3}h$ m.

  In $\triangle ABC, \tan60^\circ = \frac{h}{x + 15}\Rightarrow h = 7.5$ m and $x = 7.5\sqrt{3}$ m.

\item The diagram \in{Figure}[fig:28_48] is given below:

  \startplacefigure[reference=fig:28_48]
    \externalfigure[28_48.pdf]
  \stopplacefigure

  Let $AB$ be the tower and $BC$ be the flag staff having heights $x$ and $y$ m respectively. The distance
  of foot of tower from the point of observation $9$ m. The angles of elevation of the foot and the top of the flag staff
  are $30^\circ$ and $60^\circ$ as given in the question.

  In $\triangle ABD, \tan30^\circ = \frac{1}{\sqrt{3}} = \frac{x}{9}\Rightarrow x = 3\sqrt{3}$ m.

  In $\triangle ACD, \tan60^\circ = \sqrt{3} = \frac{x + y}{9} \Rightarrow y = 6\sqrt{3}$ m.

\item The diagram \in{Figure}[fig:28_49] is given below:

  \startplacefigure[reference=fig:28_49]
    \externalfigure[28_49.pdf]
  \stopplacefigure

  Let $AC$ be the full tree and $BC$ is the portion which has fallen. $BC$ becomes $BD$ after falling and
  angle of elevation is $30^\circ$. Let the height of remaining portion of tree be $AB = x$ m. Given, $AD = 8$
  m and $BC = BD$, which is broken part of tree.

  In $\triangle ABC, \tan30^\circ = \frac{1}{\sqrt{3}} = \frac{AB}{AD} \Rightarrow AB = \frac{8}{\sqrt{3}}$ m.

  Also, $\cos30^\circ = \frac{\sqrt{3}}{2} = \frac{AD}{BD} \Rightarrow BD = 4\sqrt{3}$ m.

  Thus, height of the tree $AC = AB + BC = AB + BD = \frac{20}{\sqrt{3}}$ m.

\item The diagram \in{Figure}[fig:28_50] is given below:

  \startplacefigure[reference=fig:28_50]
    \externalfigure[28_50.pdf]
  \stopplacefigure

  Let $AB$ be the building with height $10$ m. Let $BC$ be the flag with height $h$ m. Also, let distance
  between $P$ and foot of the building as $AP = x$ m. The angle of elevation of top of the building is
  $30^\circ$ and that of the flag is $45^\circ$.

  In $\triangle ABP, \tan30^\circ = \frac{1}{\sqrt{3}} = \frac{AB}{AP} = \frac{10}{x} \Rightarrow x = 10\sqrt{3}$ m.

  In $\triangle ACP, \tan45^\circ = 1 = \frac{AC}{AP} = \frac{10 + h}{x} \Rightarrow h = 10(\sqrt{3} - 1)$ m.
\item The diagram \in{Figure}[fig:28_51] is given below:

  \startplacefigure[reference=fig:28_51]
    \externalfigure[28_51.pdf]
  \stopplacefigure

  Let $AB$ be the lamp post having height $h$ m, and $BD$ be the girl having height $1.6$ m. The distance
  of the grl from the lamp post is $AC = 3.2$ m. $CE$ is the langeth of the shadow given as $4.8$ m. In the
  $\triangle ABE$ and $\triangle CDE, \angle E$ is common, $\angle A = \angle C = 90^\circ$ so third angle will
  be also equal. This makes the triangles similar.

  $\therefore \frac{AB}{CD} = \frac{AE}{CE} \Rightarrow h = \frac{8}{3}$ m.

\item The diagram \in{Figure}[fig:28_52] is given below:

  \startplacefigure[reference=fig:28_52]
    \externalfigure[28_52.pdf]
  \stopplacefigure

  Let $AC$ be the building having a height of $30$ m. Let $E$ and $G$ point of observations where angles
  of elevation are $60^\circ$ and $30^\circ$ respectively. Let $AEF$ be the line of foot of the building and
  foot of the observer which is a horizontal line. Let $DE$ and $FE$ are the heights of the
  observer. Draw $BEG\parallel ADF$ so that $AB = DE = FG = 1.5$ m. Thus, $BC = 28.5$ m. We have to find
  $DF = EG$.

  In $\triangle BCE, \tan60^\circ = \sqrt{3} = \frac{BC}{CE} \Rightarrow CE = \frac{28.5}{\sqrt{3}}$ m.

  In $\triangle BCG, \tan30^\circ = \frac{1}{\sqrt{3}} = \frac{BC}{CG} \Rightarrow CG = 28.5\sqrt{3}$ m.

  Thus, $DF = EG = CG - CF = \frac{57}{\sqrt{3}}$ m, which is the distance walked by the observer.

\item The diagram \in{Figure}[fig:28_53] is given below:

  \startplacefigure[reference=fig:28_53]
    \externalfigure[28_53.pdf]
  \stopplacefigure

  Let the height of the tower $AB$ is $h$ m. When the altitude of the sun is $60^\circ$ let the length of the
  shadown be $AC = x$ m. Then according to question length of shadow when the sun's altitude i $30^\circ$ the length
  of shadow will be $AD, 40$ m longer i.e. $AD = x + 40$.

  In $\triangle ABC, \tan60^\circ = \sqrt{3} = \frac{AB}{AC} = \frac{h}{x} \Rightarrow x = \sqrt{3}h$ m.

  In $\triangle ABC, \tan30^\circ = \frac{1}{\sqrt{3}} = \frac{AB}{AC} = \frac{h}{x + 40} \Rightarrow x = 20$ m and $h
  = 20\sqrt{3}$ m.

\item The diagram \in{Figure}[fig:28_54] is given below:

  \startplacefigure[reference=fig:28_54]
    \externalfigure[28_54.pdf]
  \stopplacefigure

  Let $AB$ be the building with $20$ m height. Let the height of tower be $h$ m represented by $BC$ in the
  figure. Let $D$ be the point of observation at a distance $x$ from the foot of the building $AB$.

  In $\triangle ABD, \tan30^\circ = \frac{1}{\sqrt{3}} = \frac{AB}{AD} = \frac{20}{x} \Rightarrow x = 20\sqrt{3}$ m.

  In $\triangle ACD, \tan60^\circ = \sqrt{3} = \frac{AC}{AD} = \frac{h + 20}{x}\Rightarrow h = 40$ m.

\item The diagram \in{Figure}[fig:28_55] is given below:

  \startplacefigure[reference=fig:28_55]
    \externalfigure[28_55.pdf]
  \stopplacefigure

  Let $DE$ be the building having a height of $8$ m. Let $AC$ be the multistoried building having height
  $h + 8$ m. Foot of both the buildings are joined on horizontal plane i.e. $AD$. Draw a line parallel to $AD$
  which is $BE$. So $BE$ is equal to $AD$ which we have let as $x$ m. Clearly, $AB = 8$ m. Let
  height of $BC$ to be $h$ m.

  In $\triangle CBE, \tan30^\circ = \frac{1}{\sqrt{3}} = \frac{BC}{BE} = \frac{h}{x} \Rightarrow \sqrt{3}h = x$.

  In $\triangle ACD, \tan30^\circ = \frac{1}{\sqrt{3}} = \frac{AC}{AD} = \frac{h + 8}{x}\Rightarrow h =
  \frac{8}{\sqrt{3} - 1} \Rightarrow h + 8 = \frac{8\sqrt{3}}{\sqrt{3} - 1}$ m.

\item The diagram \in{Figure}[fig:28_56] is given below:

  \startplacefigure[reference=fig:28_56]
    \externalfigure[28_56.pdf]
  \stopplacefigure

  Let $AB$ be the pedestal having height $h$ m and $BC$ be the statue having height $1.6$ m on top of
  pedestal. Let $D$ be the point of observation from where the angles of elevation as given in the question are
  $45^\circ$ and $60^\circ$.

  In $\triangle ABD, \tan45^\circ = 1 = \frac{AB}{BD} = \frac{h}{x} \Rightarrow h = x$.

  In $\triangle ACD, \tan60^\circ = \sqrt{3} = \frac{AC}{CD} = \frac{h + 1.6}{x} \Rightarrow h = \frac{1.6}{\sqrt{3} -
    1}$ m.

\item This problem is similar to 55 and has been left as an exercise.

\item The diagram \in{Figure}[fig:28_58] is given below:

  \startplacefigure[reference=fig:28_58]
    \externalfigure[28_58.pdf]
  \stopplacefigure

  Let $AB$ be the tower having height $75$ m. Let $C$ and $D$ be the position of two ships and angles of
  elevation are as given in the question. Let foor of the tower be in line with ships such that $AC = x$ m and distance
  between the ships as $d$ m.

  In $\triangle ABC, \tan45^\circ = 1 = \frac{AB}{AC} = \frac{75}{x} \Rightarrow x = 75$ m.

  In $\triangle ABC, \tan30^\circ = \frac{1}{\sqrt{3}} = \frac{75}{x + d} \Rightarrow d = 75(\sqrt{3} - 1)$ m.

\item The diagram \in{Figure}[fig:28_59] is given below:

  \startplacefigure[reference=fig:28_59]
    \externalfigure[28_59.pdf]
  \stopplacefigure

  Let $AB$ be the building and $CD$ be thw tower having height $50$ m. The angles of elevation are shown as
  given in the question. Let distance between the foot of the tower and the building be $d$ m and height of the building be
  $h$ m.

  In $\triangle ABC, \tan30^\circ = \frac{1}{\sqrt{3}} = \frac{h}{d} \Rightarrow d = \sqrt{3}h$.

  In $\triangle ABD, \tan60^\circ = \sqrt{3} = \frac{50}{d} \Rightarrow 3h = 50 \Rightarrow h = \frac{50}{3}$ m.

\item The diagram \in{Figure}[fig:28_60] is given below:

  \startplacefigure[reference=fig:28_60]
    \externalfigure[28_60.pdf]
  \stopplacefigure

  Let $DE$ represent the banks of river and $BC$ the  bridge. Given that height of the bridge is
  $30$ m. $\therefore BD = CE = 30$ m. The angles of depression from point $A$ is shown as given in the
  question. We have to find $DE = BC$ i.e. width of the river.

  In $\triangle ACE, \tan45^\circ = 1 = \frac{CE}{AC} \Rightarrow AC = 30$ m.

  In $\triangle ABD, \tan30^\circ = \frac{1}{\sqrt{3}} = \frac{BD}{AB} \Rightarrow AB = 30\sqrt{3}$ m.

  Thus, width of river $= 30 + 30\sqrt{3} = 30(\sqrt{3} + 1)$ m

\item The diagram \in{Figure}[fig:28_61] is given below:

  \startplacefigure[reference=fig:28_61]
    \externalfigure[28_61.pdf]
  \stopplacefigure

  Let $BC$ and $DE$ be the two poles. Let $A$ be the point between them such that $AB = x$ m and, thus
  $AD = 80 - x$ m. Let the elevation from $A$ to $C$ is $60^\circ$ and to $E$ is
  $30^\circ$. Let the height of poles be $h$ m.

  In $\triangle ABC, \tan60^\circ = \sqrt{3} = \frac{BC}{AB} = \frac{h}{x} \Rightarrow h = \sqrt{3}x$ m.

  In $\triangle ABD, \tan30^\circ = \frac{1}{\sqrt{3}} = \frac{DE}{AD} = \frac{h}{80 - x} \Rightarrow 3x = 80 -x
  \Rightarrow x = 20$ m. $\Rightarrow h = 20\sqrt{3}$ m.

\item The diagram \in{Figure}[fig:28_62] is given below:

  \startplacefigure[reference=fig:28_62]
    \externalfigure[28_62.pdf]
  \stopplacefigure

  Let $BD$ and $CE$ be the poles and $AJ$ be the tree. Given, $AJ = 20$ m and angles of depression to
  base of poles are $60^\circ$ and $30^\circ$. Let $\angle DAB = 6-00^\circ$ and $\angle EAC = 30^\circ$.

  Clearly, $AB = DJ = y$ m(say) and $AC = EJ = x$ m(say).

  In $\triangle AEJ, \tan60^\circ = \sqrt{3} = \frac{AJ}{EJ} \Rightarrow x = \frac{20}{\sqrt{3}}$ m.

  Similarly, $y = 20\sqrt{3}$ m.

  Thus, width of river $x + y = \frac{80}{\sqrt{3}}$ m.

\item This problem is similar to 56 and has been left as an exercise.

\item This problem is similar to 58 and has been left as an exercise.

\item This problem is similar to 49 and has been left as an exercise.

\item The diagram \in{Figure}[fig:28_66] is given below:

  \startplacefigure[reference=fig:28_66]
    \externalfigure[28_66.pdf]
  \stopplacefigure

  Let $A$ be the point on the ground, $AC$ be the string and $BC$ the height of balloon. Then given, angle of
  elevation $\angle BAC = 60^\circ$.

  In $\triangle ABC, \sin60^\circ = \frac{\sqrt{3}}{2} = \frac{BC}{AC} = \frac{BC}{215} = 107.5\sqrt{3}$ m.

\item The diagram \in{Figure}[fig:28_67] is given below:

  \startplacefigure[reference=fig:28_67]
    \externalfigure[28_67.pdf]
  \stopplacefigure

  Let $AB$ be the cliff having a height of $80$ m. Let $C$ and $D$ be two points on eihter side of the
  cliff from where angle of elevations are $60^\circ$ and $30^\circ$ respectively.

  In $\triangle ABC, \tan60^\circ = \frac{AB}{AC} \Rightarrow AC = \frac{80}{\sqrt{3}}$ m.

  In $\triangle ABD, \tan30^\circ = \frac{AB}{AD}\Rightarrow AD = 80\sqrt{3}$ m.

  Distance bettwen points of observation $CD = AC + AD = \frac{320}{\sqrt{3}}$ m.

\item Since the length of shadow is equal to height of pole the angle of elevation would be $45^\circ$ as $\tan45^\circ =
  1$.

\item This problem is similar to 62 and has been left as an exercise.

\item This problem is similar to 25 and has been left as an exercise.

\item The diagram \in{Figure}[fig:28_71] is given below:

  \startplacefigure[reference=fig:28_71]
    \externalfigure[28_71.pdf]
  \stopplacefigure

  Let $AB$ be the lighthouse having a height of $200$ m. Let $C$ and $D$ be the ships. The angles of
  depression are converted to angles of elevation.

  In $\triangle ABC, \tan45^\circ = 1 = \frac{AB}{AC}\Rightarrow AC = 200$ m.

  In $\triangle ABD, \tan60^\circ = \sqrt{3} = \frac{AB}{AD} \Rightarrow AD = \frac{200}{\sqrt{3}}$ m.

  Thus distance between ships $CD = AC + AD = \frac{200(\sqrt{3} + 1)}{\sqrt{3}}$ m.

\item The diagram \in{Figure}[fig:28_72] is given below:

  \startplacefigure[reference=fig:28_72]
    \externalfigure[28_72.pdf]
  \stopplacefigure

  Let $AB$ be the first pole and $CD$ be the second pole. Given, $CD = 24$ m and $AC = 15$ m. Draw
  $BE || AC \Rightarrow BE = 15$ m. Angle of depression is converted to angle of elevation.

  In $\triangle BDE, \tan30^\circ = \frac{1}{\sqrt{3}} = \frac{ED}{BE} \Rightarrow ED = \frac{15}{\sqrt{3}} = 5\sqrt{3}$ m.

  $\Rightarrow CE = BD - ED = 24 - 5\sqrt{3} = AB$ which is height of the first pole.

\item This problem is similar to 71 and has been left as an exercise.

\item The diagram \in{Figure}[fig:28_74] is given below:

  \startplacefigure[reference=fig:28_74]
    \externalfigure[28_74.pdf]
  \stopplacefigure

  xLet $AB$ be the tower and $C$ and $D$ are two points at a distance of $4$ m and $9$ m
  respectively. Because it is given that angles of elevations are complementary we have chosen and angle of $\theta$ for
  $C$ and $90^\circ - \theta$ for $D$.

  In $\triangle ABC, \tan\theta = \frac{AB}{AC} = \frac{h}{4}$

  In $\triangle ABD, \tan(90^\circ - \theta) = \cot\theta = \frac{AB}{AD} = \frac{h}{9}$

  Substituting for $\cot\theta$, we get

  $\frac{4}{h} = \frac{h}{9}\Rightarrow h^2 = 36 \Rightarrow h = 6$ m.

\item This problem is similar to 72 and has been left as an exercise.

\item This problem is similar to 56 and has been left as an exercise.

\item This problem is similar to 55 and has been left as an exercise.

\item This problem is similar to 71 and has been left as an exercise

\item This problem is similar to 55 and has been left as an exercise.

\item This problem is similar to 58 and has been left as an exercise.

\item This problem is similar to 26 annd has been left as an exercise.

\item This problem is similar to 71 and has been left as an exercise.

\item This problem is similar to 26 annd has been left as an exercise.

\item This problem is similar to 28 annd has been left as an exercise.

\item This problem is similar to 71 and has been left as an exercise

\item This problem is similar to 23 and has been left as an exercise

\item The diagram \in{Figure}[fig:28_87] is given below:

  \startplacefigure[reference=fig:28_87]
    \externalfigure[28_87.pdf]
  \stopplacefigure

  Let $AB$ be the tower and $BC$ be the flag-staff having a height of $h$ m. Let $D$ be the point of
  observation having angle of elevations $\alpha$ and $\beta$ as given in the question.

  In $\triangle ABC, \tan\alpha = \frac{AB}{AD} \Rightarrow AB = AD\tan\alpha$

  In $\triangle ABD, \tan\beta = \frac{AC}{AD} = \frac{AB + BC}{AD}$

  $\Rightarrow \frac{AB\tan\beta}{\tan\alpha} = AB + h \Rightarrow AB = \frac{h\tan\alpha}{\tan\beta - \tan\alpha}.$

\item This proble is similar to 74 and has been left as an exercise.

\item The diagram \in{Figure}[fig:28_89] is given below:

  \startplacefigure[reference=fig:28_89]
    \externalfigure[28_89.pdf]
  \stopplacefigure

  Let $BE$ be the tower leaning northwards and $AB$ be the vertical height of tower taken as $h$. Let $C$
  and $D$ be the points of observation. Given that angle of leaning is $\theta$ and angles of elevation are
  $\alpha$ at $C$ and $\beta$ at $D$. Let $AB = x$. Given $BC = a$ and $BD = b$.

  In $\triangle ABE, \cot\theta = \frac{x}{h}$, in $\triangle ACE, \cot\alpha = \frac{x + a}{h}$ and in
  $\triangle ADE, \cot\beta = \frac{x + b}{h}$.

  $\Rightarrow b\cot\alpha = \frac{bx + ab}{h}, a\cot\beta = \frac{ax + ab}{h}$

  $\Rightarrow b\cot\alpha - a\cot\beta = \frac{bx - ax}{h}\Rightarrow \frac{x}{h} = \cot\theta = \frac{b\cot\alpha - a\cot\beta}{b - a}$.

\item The diagram \in{Figure}[fig:28_90] is given below:

  \startplacefigure[reference=fig:28_90]
    \externalfigure[28_90.pdf]
  \stopplacefigure

  Let $AE$ be the plane of lake and $AC$ be the height of the cloud. $F$ is the point of observation at a
  height $h$ from lake. $AD$ is the reflection of cloud in the lake. Clearly, $AC = AD$. Draw $AE || BF$
  and let $BF = x$. $\alpha$ and $\beta$ are angles of elevation and depression as given.

  In $\triangle BCF, \tan\alpha = \frac{BC}{BF} = \frac{BC}{x}\Rightarrow BC = x\tan\alpha$

  $AC = AD = AB + BC = h + x\tan\alpha$

  In $\triangle BDF, \tan\beta = \frac{AB + AD}{BF} = \frac{h + h + x\tan\alpha}{x} \Rightarrow x = \frac{2h}{\tan\beta -
    \tan\alpha}$

  $AC = AB + BC = h + x\tan\alpha = \frac{h(\tan\alpha + \tan\beta)}{\tan\beta - \tan\alpha}$.

\item The diagram \in{Figure}[fig:28_91] is given below:

  \startplacefigure[reference=fig:28_91]
    \externalfigure[28_91.pdf]
  \stopplacefigure

  Let the cicle represent round balloon centered at $O$ having radius $r$. $B$ is the point of observation from
  where angle of elevation to the center of the balloon is given as $\beta$. $BL$ and $BM$ are tangents to the
  balloon and $OL$ and $OM$ are perpendiculars. Clearly $OL = OM = r$. GIven $\angle LBM = \alpha$ and
  $\angle OBL = \angle OBM = \alpha/2$.

  In $\triangle OBL, \sin\alpha/2 = \frac{OL}{OB} \Rightarrow OB = r\csc\alpha/2$.

  In $\triangle ABO, \sin\beta = \frac{AO}{OB}\Rightarrow AO = r\sin\beta\csc\alpha/2$.

\item The diagram \in{Figure}[fig:28_92] is given below:

  \startplacefigure[reference=fig:28_92]
    \externalfigure[28_92.pdf]
  \stopplacefigure

  Let $AB$ be the cliff having a height $h$ and $F$ be the initial point of observation from where the angle of
  elevation is $\theta$. Let $D$ be the point reached after walking a distance $k$ towards the top at an angle
  $\phi$. The angle of elevation at $D$ is $\alpha$.

  In $\triangle DEF, \sin\phi = \frac{DE}{DF} \Rightarrow DE = k\sin\phi, \cos\phi = \frac{EF}{DF} \Rightarrow EF =
  k\cos\phi$.

  In $\triangle ABF, \tan\theta = \frac{AB}{BF} \Rightarrow \frac{x}{k\cos\phi + (x - k\sin\phi)\cot\alpha}$

  $\Rightarrow x\cot\theta = k\cos\phi + x\cot\alpha - k\sin\phi\cot\alpha \Rightarrow x(\cot\theta - \cot\alpha) =
  k(\cos\phi - \sin\phi\cot\alpha)$

  $\Rightarrow x = \frac{k(\cos\phi - \sin\phi\cot\alpha)}{\cot\theta - \cot\alpha}$.

\item The diagram \in{Figure}[fig:28_93] is given below:

  \startplacefigure[reference=fig:28_93]
    \externalfigure[28_93.pdf]
  \stopplacefigure

  Let $CD$ be the tower having a height $h$. Point $A$ is due south of $A$ making an angle of elevation
  $\alpha$ and $B$ is due east of tower making an angle of elevation $\beta$. Clearly, $\angle ACB =
  90^\circ$. Given that $AB = d$.

  In $\triangle ACD, \tan\alpha = \frac{CD}{AC} \Rightarrow AC  = h\cot\alpha$ and in $\triangle BCD, \tan\beta =
  \frac{CD}{BC} \Rightarrow BC = h\cot\beta$.

  In $\triangle ABC, AB^2 = AC^2 + AD^2 \Rightarrow d^2 = h^2\cot^2\alpha + h^2\cot^2\beta \Rightarrow h =
  \frac{d}{\sqrt{\cot^2\alpha + \cot^2\beta}}$.

\item This problem is similar to 93 and has been left as an exercise.

\item The diagram \in{Figure}[fig:28_95] is given below:

  \startplacefigure[reference=fig:28_95]
    \externalfigure[28_95.pdf]
  \stopplacefigure

  Let $AB$ be the girl having a height of $1.2$ m, $C$ and $F$ be the two places of balloon for which
  angle of elevations are $60^\circ$ and $30^\circ$ respectively. Height of ballon above ground level is given as
  $88.2$ m and thus height of balloon above the girl's eye-level is $88.2 - 1.2 = 87$ m.

  In $\triangle ACD, \tan60^\circ = \frac{CD}{AD} \Rightarrow AD = 87/\sqrt{3}$ m.

  In $\triangle AFG, \tan30^\circ = \frac{FG}{AG} \Rightarrow AG = 87\sqrt{3}$ m.

  Thus distance trarvelled by the ballon $= 87\sqrt{3} - 87/\sqrt{3} = 174/\sqrt{3}$

\item The diagram \in{Figure}[fig:28_96] is given below:

  \startplacefigure[reference=fig:28_96]
    \externalfigure[28_96.pdf]
  \stopplacefigure

  Let $AB$ represent the tower with a height $h$. Let $C$ and $D$ be the points to which angles of
  depression are given as $60^\circ$ and $30^\circ$ which are shown as angles of elevation at these points.

  In $\triangle ABC, \tan60^\circ = \sqrt{3} = \frac{AB}{AC} \Rightarrow AC = h/\sqrt{3}$

  In $\triangle ABD, \tan30^\circ = \frac{1}{\sqrt{3}} = \frac{AB}{AD} \Rightarrow AD = h\sqrt{3}$

  $CD = AD - AC = 2h/\sqrt{3}$

  The car covers the distance $CD$ in six seconds. Thus speed of the car if $2h/(6\sqrt{3}) = h/3\sqrt{3}$

  Time taken to cover $AC$ to reach the foot of the tower is $\frac{h}{\sqrt{3}}\times\frac{3\sqrt{3}}{h} = 3$
  seconds.

\item Proceeding like previous problem the answer would be three minutes.

\item This problem is similar to 96 and has been left as an exercise.

\item The diagram \in{Figure}[fig:28_99] is given below:

  \startplacefigure[reference=fig:28_99]
    \externalfigure[28_99.pdf]
  \stopplacefigure

  Let $AB$ be the building having height $h$ m. Let $C$ and $D$ be the fire stations from which the
  angles of elevation are $60^\circ$ and $45^\circ$ separated by $20,000$ m.

  In $\triangle ABC, \tan60^\circ = \sqrt{3} = \frac{AB}{AC}\Rightarrow AC = h/\sqrt{3}$ m.

  In $\triangle ABD, \tan45^\circ = h = \frac{AB}{AD}\Rightarrow AD = h$ m.

  Since $AD < AD$ so the fire station at $C$ will reach the building faster.

  $AD = AC + CD \Rightarrow h = h/\sqrt{3} + 20000 \Rightarrow h = \frac{20000\sqrt{3}}{\sqrt{3} - 1}$

  $\therefore AC = \frac{2000}{\sqrt{3} - 1}$ m.

\item The diagram \in{Figure}[fig:28_100] is given below:

  \startplacefigure[reference=fig:28_100]
    \externalfigure[28_100.pdf]
  \stopplacefigure

  Let $AB$ be the deck of the ship with given height of $10$ m. Let $CE$ be the cliff with base at
  $C$. Let the height of portion $DE$ be $x$ m. The angles of elevation of the top and of the bottom of the
  cliff are shown as given in the question.

  In $\triangle BDE, \tan45^\circ = DE/BD \Rightarrow BD = x$ m.

  In $\triangle, \tan30^circ = CD/BD \Rightarrow BD = 10\sqrt{3} = x$

  Thus, $CE = 10 + 10\sqrt{3} = 27.32$ m.

  So height of the cliff is $27.32$ m and distance of cliff from the ship is $10$ m.
\item The diagram \in{Figure}[fig:28_101] is given below:

  \startplacefigure[reference=fig:28_101]
    \externalfigure[28_101.pdf]
  \stopplacefigure

  Let $AB$ and $CD$ be the two temples and $AC$ be the river. Let the height of temple $AB$ be
  $50$ m. $AC$ is the river. The angles are depression are shown as corresponding angles of elevation. Let the
  height of $CD$ be $x$ m and width of river be $w$ m. Thus, $CD = x$ and $AC = w$.

  In $\triangle ABC, \tan60^\circ = \frac{AB}{AC} \Rightarrow w = \frac{50}{\sqrt{3}} = DE[\because DE||AC]$

  In $\triangle BDE, \tan30^\circ = \frac{BD}{DE} \Rightarrow \frac{1}{\sqrt{3}} = \frac{BD}{50/\sqrt{3}}\Rightarrow BD =
  \frac{50}{3}$.

  Thus, height of the second temple $CD = AB - BD = \frac{100}{3}$ m.

\item This problem is similar to 95 and has been left as an exercise.

\item This problem is similar to 95 and has been left as an exercise.

\item The diagram \in{Figure}[fig:28_104] is given below:

  \startplacefigure[reference=fig:28_104]
    \externalfigure[28_104.pdf]
  \stopplacefigure

  Let $BE$ be the tower leaning due east where $B$ is the foot of the tower and $E$ is the top. $AB$ is
  the vertical height of the tower taken as $h$. The angles of elevation are shown from tow points as given in the
  question.

  In $\triangle ACE, \tan\alpha = \frac{h}{x + a}$ and in $\triangle ADE, \tan\beta = \frac{h}{x + b}$.

  $\Rightarrow \frac{b - a}{h} = \frac{1}{\tan\beta} - \frac{1}{\tan\alpha}$

  $\Rightarrow h = \frac{(b - a)\tan\alpha\tan\beta}{\tan\alpha - \tan\beta}$.

\item The diagram \in{Figure}[fig:28_105] is given below:

  \startplacefigure[reference=fig:28_105]
    \externalfigure[28_105.pdf]
  \stopplacefigure

  Let $AC$ be the lake and $B$ be the point of observation $2500$ m above lake. Let $E$ be the cloud and
  $F$ be its reflection in the lake. If we take height of the cloud above lake as $h$ then $CD = 2500$ m where
  $BD||AC$ . $DE = h - 2500$ m and $CD = 2500$ m. The angle of elevation and angle of depression of cloud and
  its reflection are shown as given in the problem.

  In $\triangle BDF, \tan45^\circ = \frac{DF}{BD} \Rightarrow BD = h + 2500$

  In $\triangle BDE, \tan15^\circ = \frac{DE}{BD} \Rightarrow DE = 1830.6$ m.

  $\Rightarrow CE = CD + DE = h = 2500 + 1830.6 = 4330.6$ m.

\item The diagram \in{Figure}[fig:28_106] is given below:

  \startplacefigure[reference=fig:28_106]
    \externalfigure[28_106.pdf]
  \stopplacefigure

  This is a problem similar to previous problem with $2500$ replaced by $h$ and angles are replaced by
  $\alpha$ and $\beta$. So the diagram is similar in nature. Let the height of the cloud above lake be
  $h'$ m. So $DE = h' - h$ and $DF = h + h'$.

  In $\triangle BDE, \tan\alpha = \frac{DE}{BD}\Rightarrow BD = (h' - h)/\tan\alpha$

  In $\triangle BDF, \tan\beta = \frac{DF}{BD}\Rightarrow BD = (h' + h)/\tan\beta$

  $\Rightarrow (h' - h)\tan\beta = (h' + h)\tan\alpha \Rightarrow h' = \frac{h(\tan\alpha + \tan\beta)}{\tan\beta -
    \tan\alpha}$

  $\therefore BD = \frac{2h}{(\tan\alpha - \tan\beta)}$

  Also, $\sec\alpha = \frac{BE}{BD} \Rightarrow BE = \frac{2h\sec\alpha}{\tan\alpha - \tan\beta}$ which is the distance of
  the cloud from the point of observation.

\item The diagram \in{Figure}[fig:28_107] is given below:

  \startplacefigure[reference=fig:28_107]
    \externalfigure[28_107.pdf]
  \stopplacefigure

  Let $AB$ be the height of plane above horizontal ground as $h$ miles. $C$ and $D$ are two consecutive
  milestones so $CD = 1$ mile. Let $BC = x$ mile. The angles of depression are represented as angles of elevation.

  In $\triangle ABC, \tan\alpha = \frac{AB}{BC} \Rightarrow h = x\tan\alpha$

  In $\triangle ABD, \tan\beta = \frac{AB}{BD} \Rightarrow h = (x + 1)\tan\beta$

  $\Rightarrow x = \frac{\tan\beta}{\tan\alpha - \tan\beta}\Rightarrow h = \frac{\tan\alpha}{\tan\alpha - \tan\beta}$.

\item The diagram \in{Figure}[fig:28_108] is given below:

  \startplacefigure[reference=fig:28_108]
    \externalfigure[28_108.pdf]
  \stopplacefigure

  Let $PQ$ be the post with height $h$ and $AB$ be the tower. Given that the angles of elevation of $B$
  at $P$ and $Q$ are $\alpha$ and $\beta$ respectively. Draw $CQ||PA$ such that $PQ = AC =
  h$ and A$AP = QC = x$. Also, let $BC = h'$ so that $AB = AC + BC = h + h'$.

  In $\triangle ABP, \tan\alpha = \frac{h + h'}{x}\Rightarrow x = \frac{h + h'}{\tan\alpha}$

  In $\triangle BCQ, \tan\beta = \frac{h'}{x}\Rightarrow x = \frac{h'}{\tan\beta}$

  $\Rightarrow \frac{h + h'}{\tan\alpha} = \frac{h'}{\tan\beta}$

  $h' = \frac{h\tan\beta}{\tan\alpha - \tan\beta} \Rightarrow x = \frac{h}{\tan\alpha - \tan\beta}$

  $\therefore PQ = h + h' = \frac{h\tan\alpha}{\tan\alpha - \tan\beta}$.

\item The diagram \in{Figure}[fig:28_109] is given below:

  \startplacefigure[reference=fig:28_109]
    \externalfigure[28_109.pdf]
  \stopplacefigure

  Let $AD$ be the wall, $BD$ and $CE$ are two positions of the ladder. Then according to question $BC =
  a, DE = b$ and angles of elevations at $B$ and $C$ are $\alpha$ and $\beta$. Let $AB = x$ and
  $AE = y$. Also, let length of ladder be $l$ i.e $BD = CE = l$.

  In $\triangle ABD, \sin\alpha = \frac{AD}{BD} = \frac{y + b}{l}, \cos\alpha = \frac{AB}{BD} = \frac{x}{l}$

  In $\triangle ACE, \sin\alpha = \frac{AE}{CE} = \frac{y}{l}, \cos\beta = \frac{AC}{CE} = \frac{a + x}{l}$.

  $\Rightarrow \cos\beta - \cos\alpha = \frac{a}{l}$ and $\sin\alpha - \sin\beta = \frac{b}{l}$

  $\Rightarrow \frac{a}{b} = \frac{\cos\alpha - \cos\beta}{\sin\beta - \sin\alpha}$.

\item The diagram \in{Figure}[fig:28_110] is given below:

  \startplacefigure[reference=fig:28_110]
    \externalfigure[28_110.pdf]
  \stopplacefigure

  Let $CD$ be the tower subtending angle $\alpha$ at $A$. Let $B$ be the point $b$ m above
  $A$ from which angle of depression to foot of tower at $C$ is $\beta$ which is shown as angle of elevation.
  Let $AC = x$ and $CD = h$.

  In $\triangle ACD, \tan\alpha = \frac{h}{x} \Rightarrow x = h\cot\alpha$

  In $\triangle ABC, \tan\beta = \frac{b}{x} \Rightarrow x = b\cot\beta$

  $\Rightarrow h\cot\alpha = b\cot\beta \Rightarrow h = b\tan\alpha\cot\beta$
\item The diagram \in{Figure}[fig:28_111] is given below:

  \startplacefigure[reference=fig:28_111]
    \externalfigure[28_111.pdf]
  \stopplacefigure

  Let $AB$ be the observer with a height of $1.5$ m, $28.5$ m i.e. $AD$ from tower $DE, 30$ m
  high. Draw $BC||AD$ such that $AB = CD = 1.5$ m and thus $CE = 28.5$ m. Let the angle of elevation from
  observer's eye to the top of the tower be $\alpha$.

  In $\triangle BCE, \tan\alpha = \frac{CE}{BC} = \frac{28.5}{28.5} = 1\Rightarrow \alpha = 45^\circ$.

\item The diagram \in{Figure}[fig:28_112] is given below:

  \startplacefigure[reference=fig:28_112]
    \externalfigure[28_112.pdf]
  \stopplacefigure

  Let $AB$ be the tower havin a height of $h$ and $C$ and $D$ are two objects at a distance of $x$
  and $x + y$ such that angles of depression shown as angles of elevatin are $\beta$ and $\alpha$ respectively.

  In $\triangle ABC, \tan\beta = \frac{h}{x} \Rightarrow x = h\cot\beta$

  In $\triangle ABD, \tan\alpha = \frac{h}{x + y} \Rightarrow x + y = h\cot\alpha$

  Distance between $C$ and $D = y = h(\cot\alpha - \cot\beta)$.

\item The diagram \in{Figure}[fig:28_113] is given below:

  \startplacefigure[reference=fig:28_113]
    \externalfigure[28_113.pdf]
  \stopplacefigure

  Let $AB$ be the height of the window at a height $h$ and $DE$ be the house opposite to it. Let the distance
  between the houses be $AD = x$. Draw $BC||AD$ such that $BC = x$ and $CD = h$. The angles are shown as
  given in the problem. Let $CE = y$

  In $\triangle BCD, \tan\beta = \frac{h}{x} \Rightarrow x = h\cot\beta$

  In $\triangle BCE, \tan\alpha = \frac{y}{x} \Rightarrow x = y\cot\alpha$

  $\Rightarrow y = h\tan\alpha\cot\beta$

  Total height of the second house $DE = CD + DE = y + h = h(1 + \tan\alpha\cot\beta)$

\item The diagram \in{Figure}[fig:28_114] is given below:

  \startplacefigure[reference=fig:28_114]
    \externalfigure[28_114.pdf]
  \stopplacefigure

  Let $AD$ be the ground, $B$ be the lower window at a height of $2$ m, $C$ be the upper window at a
  height of $4$ m above lower window and $G$ be the balloon at a height of $x + 2 + 4$ m above ground. Draw
  $DG||AC, BE||AD$ and $CF||AD$ so that $DE = 2$ m, $EF = 4$ m and $FG = x$ m. Also, let $BE
  = CF = d$ m. The angles of elevation are shown as given in the problem.

  In $\triangle BEG, \tan60^\circ = \sqrt{3} = \frac{EG}{BE} = \frac{x + 4}{d}\Rightarrow d = \frac{x + 4}{\sqrt{3}}$

  In $\triangle CFG, \tan30^\circ = \frac{1}{\sqrt{3}} = \frac{FG}{CF} = \frac{x}{d}\Rightarrow d = \sqrt{3}x$

  $\Rightarrow \sqrt{3}x = \frac{x + 4}{\sqrt{3}}\Rightarrow x = 2$

  $\therefore$ the height of the balloon $= 2 + 4 + 2 = 8$ m.

\item The diagram \in{Figure}[fig:28_115] is given below:

  \startplacefigure[reference=fig:28_115]
    \externalfigure[28_115.pdf]
  \stopplacefigure

  Let $AB$ be the lamp post, $EF$ and $GH$ be the two positions of the man having
  height $6$ ft. Let the shdows be $EC$ and $GD$ of lengths $24$ ft. and
  $30$ ft. for initial and final position. Since the man moves eastward from his initial position
  $\therefore \angle ACD = 90^\circ$.

  Let $AB = h, AE = x$ and $AG = y$.

  From similar triangles $CEF$ and $ABC$

  $\frac{h}{6} = \frac{24 + x}{24}$

  From similar triangles $ABD$ and $DGH$

  $\frac{h}{6} = \frac{30 + y}{30}$

  Thus, $1 + \frac{x}{24} = 1 + \frac{y}{30} \Rightarrow y = \frac{5x}{4}$.

  From right angles $\triangle ACD$,

  $y^2 = x^2 + 300^2 \Rightarrow x = 400$ ft.

  $\Rightarrow h = 106$ ft.

\item The diagram \in{Figure}[fig:28_116] is given below:

  \startplacefigure[reference=fig:28_116]
    \externalfigure[28_116.pdf]
  \stopplacefigure

  Let $AB$ be the tower having a height of $h$ m, $AC$ be the final length of shadow taken as $x$ m,
  $AD$ is the initial length of shadow which is $5$ m more than finla length i.e. $CD = 5$ m. The angles of
  elevation are shown as given in the problem.

  In $\triangle ABC, \tan60^\circ = \sqrt{3} = \frac{h}{x} \Rightarrow x = \frac{h}{\sqrt{3}}$

  In $\triangle ABD, \tan30^\circ = \frac{1}{\sqrt{3}} = \frac{h}{x + 5} \Rightarrow \frac{2h}{\sqrt{3}} = 5$

  $\Rightarrow h = \frac{5\sqrt{3}}{2} = 4.33$ m.

\item The diagram \in{Figure}[fig:28_117] is given below:

  \startplacefigure[reference=fig:28_117]
    \externalfigure[28_117.pdf]
  \stopplacefigure

  Let $A$ be the initial position of the man and $D$ and $E$ be the objects in the west. Let $DE = x, AD
  = y, \angle ADB=\theta, \angle AEB = \phi$ and $\angle ADC=\psi$. $\alpha$ and $\beta$ are the angles made
  by objects on the two positions of the man as given in the problem.

  $\Rightarrow \tan\theta = \frac{c}{y}$ and $\tan\phi = \frac{c}{x + y}$

  Now $\theta - \phi = \alpha \Rightarrow \tan(\theta - \phi) = \tan\alpha$

  $\Rightarrow \frac{\tan\theta - \tan\phi}{1 + \tan\theta\tan\phi} = \tan\alpha$

  $\Rightarrow \frac{\frac{c}{y} - \frac{c}{x + y}}{1 + \frac{c}{y}\frac{c}{x + y}} = \tan\alpha$

  $\Rightarrow cx\cot\alpha = xy + y^2 + c^2$

  Similarly, substituting $2c$ for $x$ and $\psi$ for $\phi$, we get

  $2cx\cot\beta = xy + y^2 + 4c^2 \Rightarrow x= \frac{3c}{2\cot\beta - \cot\alpha}$.

\item The diagram \in{Figure}[fig:28_118] is given below:

  \startplacefigure[reference=fig:28_118]
    \externalfigure[28_118.pdf]
  \stopplacefigure

  Let $P$ be the object and $OA$ be the straight line on which $B$ and $C$ lie underneath the object.
  Let $OP = h$. According to question the angles of elevation made are $\alpha, 2\alpha$ and $3\alpha$ from
  $A, B$ and $C$ i.e. $\angle PCO = 3\alpha, \angle PBO = 2\alpha$ and $\angle PAO = \alpha$. Given that
  $AB = \alpha$ and $BC = b$.

  $\angle APB = 2\alpha - \alpha = \alpha$ and $\angle BPC = 3\alpha - 2\alpha = \alpha$

  $\therefore AB = BP = a$

  In $\triangle PBC, \frac{BC}{\sin\alpha} = \frac{PB}{\sin(180^\circ - 3\alpha)} \Rightarrow \frac{b}{\sin\alpha} =
  \frac{a}{\
    sin3\alpha}$

  $\Rightarrow \frac{a}{b} = \frac{\sin3\alpha}{\sin\alpha} = 3 - 4\sin^2\alpha \Rightarrow \sin\alpha = \sqrt{\frac{3b -
      a}{4b}}$

  In $\triangle OPB, OP = BP\sin2\alpha = 2a\sin\alpha\cos\alpha = \frac{a}{2b}\sqrt{(a + b)(3b - a)}$

\item This problem is similar to 92 and has been left as an exercise.

\item The diagram \in{Figure}[fig:28_120] is given below:

  \startplacefigure[reference=fig:28_120]
    \externalfigure[28_120.pdf]
  \stopplacefigure

  Let $\theta$ be the angle of inclination of the inclines plane $AC$. Let $AB = c$ and
  $BC = c$. Let the object be at $D$. Now $\angle DBA = \theta - \alpha$ and
  $\angle DCA = \theta - \beta$.

  Using sine rule in $\triangle DAB, \frac{c}{\sin\alpha} = \frac{AD}{\sin(\theta - \alpha)}$

  $\Rightarrow AD = \frac{c\sin(\theta - \alpha)}{\sin\alpha}$

  Applying sine rule in $\triangle DAC, \frac{2c}{\sin\beta} = \frac{AD}{\sin(\theta - \beta)}$

  $\Rightarrow AD = \frac{2\sin(\theta - \beta)}{\sin\beta}$

  $\Rightarrow \frac{c\sin(\theta - \alpha)}{\sin\alpha} = \frac{2c\sin(\beta - \beta)}{\sin\beta}$

  $\Rightarrow \frac{\sin\theta\cos\alpha - \cos\theta\sin\alpha}{\sin\theta\sin\alpha} =
  \frac{2[\sin\theta\cos\beta - \cos\theta\sin\beta]}{\sin\theta\sin\beta}$

  $\Rightarrow \cot\alpha - \cot\theta = 2(\cot\beta - \cot\beta)$

  $\Rightarrow \cot\theta = 2\cot\beta - \cot\alpha$.

\item The question is same as 109 just that we have a different relation to prove. From 109
  we have

  $\frac{a}{b} = \frac{\cos\beta - \cos\alpha}{\sin\alpha - \sin\beta}$

  $= \frac{2\sin\frac{\alpha + \beta}{2}\sin\frac{\alpha - \beta}{2}}{2\cos\frac{\alpha +
      \beta}{2}\sin\frac{\alpha - \beta}{2}}$

  $= \tan\frac{\alpha + \beta}{2}\Rightarrow a = b\tan\frac{\alpha + \beta}{2}$.

\item The diagram \in{Figure}[fig:28_122] is given below:

  \startplacefigure[reference=fig:28_122]
    \externalfigure[28_122.pdf]
  \stopplacefigure

  Given that $A$ and $B$ are two points of observation on ground $1000$ m apart. Let
  $C$ be the point where the balloon will hit the ground at a distance $x$ m from
  $B$. Also, let $D$ and $E$ be the points above $A$ and $B$ respectively
  such that $\angle BAE= 30^\circ$ and $\angle DBA = 60^\circ$.

  In $\triangle ABD, \tan60^\circ = \sqrt{3} = \frac{AD}{AB}\Rightarrow AD = 1000\sqrt{3}$ m.

  In $\triangle ABE, \tan30^\circ = \frac{1}{\sqrt{3}} = \frac{BE}{AB}\Rightarrow BE =
  \frac{1000}{\sqrt{3}}$ m.

  Clearly, $\triangle BCE$ and $ACD$ are similar. Therefore,

  $\frac{BC}{AC} = \frac{BE}{AD} \Rightarrow \frac{x}{x + 1000} =
  \frac{1000}{1000\sqrt{3}.\sqrt{3}}$

  $\Rightarrow x = 500 \Rightarrow AC = 1500$ m.

\item The diagram \in{Figure}[fig:28_123] is given below:

  \startplacefigure[reference=fig:28_123]
    \externalfigure[28_123.pdf]
  \stopplacefigure

  Let $AB$ be tree having height $h$ m and $BC$ be the width of the river having width
  $w$ m. According to question angle of elevation of the tree from the opposite bank is
  $60^\circ$. Also, let $D$ be the point when the man retires $40$ m from where the
  angle of elevation of the tree is $30^\circ$.

  In $\triangle ABC, \tan60^\circ = \sqrt{3} = \frac{h}{w} \Rightarrow h = w\sqrt{3}$ m.

  In $\triangle ABD, \tan30^\circ = \frac{h}{w + 40}\Rightarrow 3w = w + 40 \Rightarrow w = 20$ m.

  $\Rightarrow h = 20\sqrt{3}$ m.

  Thus, width of the river is $20$ m and height of the tree is $20\sqrt{3}$ m.

\item The diagram \in{Figure}[fig:28_124] is given below:

  \startplacefigure[reference=fig:28_124]
    \externalfigure[28_124.pdf]
  \stopplacefigure

  Let $O$ be the point of observation. The bird is flying in the horizontal line $WXYZ$. The
  angles of elevation of the bird is given at equal intervals of time. Since the speed of the bird is
  constant $WX = XY = YZ = y$ (let). From question $\angle AOW = \alpha, \angle BOX = \beta,
  \angle COY = \gamma$ and $\angle DOZ = \delta$. Let $OA = x$ and $AW = h$.

  In $\triangle AOW, \cot\alpha = \frac{x}{h}$

  In $\triangle BOX, \cot\beta = \frac{x + y}{h}$

  In $\triangle COY, \cot\gamma = \frac{x + 2y}{h}$

  In $\triangle DOZ, \cot\delta = \frac{x + 3y}{h}$

  L.H.S. $= \cot^2\alpha - \cot^2\delta = \frac{-6xy - 9y^2}{h^2}$

  R.H.S. $= \cot^2\beta - \cot^2\gamma = \frac{-6xy - 9y^2}{h^2}$

  $\therefore$ L.H.S. = R.H.S.

\item The diagram \in{Figure}[fig:28_125] is given below:

  \startplacefigure[reference=fig:28_125]
    \externalfigure[28_125.pdf]
  \stopplacefigure

  Let $AB$ be the tower, $BC$ be the pole and $D$ be the point of observation where the
  tower and the pole make angles $\alpha$ and $\beta$ respectively. Let the height of the
  tower be $h'$ and $AD = d$. Given that the height of the pole is $h$.

  In $\triangle ABD, \cot\alpha = \frac{AD}{AB} = \frac{d}{h'} \Rightarrow d = h'\cot\alpha$

  In $\triangle ACD, \tan(\alpha + \beta) = \frac{AC}{AB} = \frac{h + h'}{d}$

  $\Rightarrow h + h' = h'\cot\alpha\tan(\alpha + \beta)$

  $h' = \frac{h}{\cot\alpha\tan(\alpha + \beta) - 1} = \frac{h\sin\alpha\cos(\alpha +
    \beta)}{\sin(\alpha + \beta)\cos\alpha - \cos(\alpha + \beta)\sin\alpha}$

  $= \frac{h\sin\alpha\cos(\alpha + beta)}{\sin(\alpha + \beta - \alpha)} =
  h\sin\alpha\csc\beta\cos(\alpha + \beta)$

\item The diagram \in{Figure}[fig:28_126] is given below:

  \startplacefigure[reference=fig:28_126]
    \externalfigure[28_126.pdf]
  \stopplacefigure

  Given $AC = BC = x$ (let) and $\angle BPC = \beta$.

  Let $\angle BPA = \theta$ then $\angle CPA = \theta - \beta$

  In $\triangle APC, \tan(\theta - \beta) = \frac{x}{AP}$

  In $\triangle APB, \tan\theta = \frac{2x}{AP}$

  $\Rightarrow \tan\theta = 2\tan(\theta - \beta) = \frac{2(\tan\theta - \tan\beta)}{1 +
    \tan\theta\tan\beta}$

  $\tan\theta = \frac{AB}{AP} = \frac{1}{n}$ (from question)

  $\Rightarrow \frac{1}{n} = \frac{2\left(\frac{1}{n} - \tan\beta\right)}{1 + \frac{\tan\beta}{n}}$

  $\Rightarrow \tan\beta = \frac{n}{2n^2 + 1}$.

\item The diagram \in{Figure}[fig:28_127] is given below:

  \startplacefigure[reference=fig:28_127]
    \externalfigure[28_127.pdf]
  \stopplacefigure

  Let $AB$ be the first chimney and $CD$ be the second chimney. The angles of elevation are
  shown as angles of elevation as given in the problem. Draw $BE||AC$ and let $AC = BE = d$ m
  and $AB = CE = h$ m. Given $CD = 150$ m. Clearly, $DE = 150 - h$ m.

  In $\triangle BED, \tan\theta = \frac{4}{3} = \frac{150 - h}{d} \Rightarrow 4d = 450 - 3h$

  In $\triangle ACD, \tan\phi = \frac{5}{2} = \frac{150}{d} \Rightarrow d = 60$ m.

  $\Rightarrow h = 70$ m. $\Rightarrow BE = 60$ m and $ED = 150 - 70 = 80$ m.

  $BD^2 = BE^2 + DE^2 = 80^2 + 60^2 \Rightarrow BD = 100$ m, which, is the distance between the tops
  of two chimneys.

\item The diagram \in{Figure}[fig:28_128] is given below:

  \startplacefigure[reference=fig:28_128]
    \externalfigure[28_128.pdf]
  \stopplacefigure

  Let $CD$ be the tower of height $h$ having an elevation of $30^\circ$ from $A$
  which is southward of it. Let $B$ be eastward of $A$ at a distance of $a$ from it
  from where the angle of elevation is $18^\circ$. Since $B$ is eastward of $A
  \angle CAB = 90^\circ$.

  In $\triangle ACD, \tan 30^\circ = \frac{h}{AC} \Rightarrow AC = h\sqrt{3}$

  In $\triangle BCD,, \tan18^\circ = \frac{h}{BC} \Rightarrow BC = h\cot18^\circ$

  In $\triangle ABC, BC^2 = a^2 + AC^2 \Rightarrow h^2\cot^218^\circ = a^2 + 3h^2$

  $\therefore h = \frac{a}{\sqrt{\cot^218^\circ - 3}}$

  Now $\cot^218^\circ = 5 + 2\sqrt{5} \therefore h = \frac{a}{\sqrt{2 + 2\sqrt{5}}}$.

\item The diagram \in{Figure}[fig:28_129] is given below:

  \startplacefigure[reference=fig:28_129]
    \externalfigure[28_129.pdf]
  \stopplacefigure

  Let $AB$ be the tower having height $h$. Given that $P$ is north of the tower and
  $Q$ is due west of $P\therefore \angle APQ= 90^\circ$.

  In $\triangle ABP, \tan\theta = \frac{h}{AP} \Rightarrow AP = h\cot\theta$

  In $\triangle ABQ, \tan\phi = \frac{h}{AQ} \Rightarrow AQ = h\cot\phi$

  In $\triangle APQ, AQ^2 = AP^2 + PQ^2$

  $\Rightarrow PQ^2 = h^2[\cot^2\phi - \cot^2\theta]$

  $\Rightarrow h = \frac{PQ}{\sqrt{\cot^2\phi - \cot^2\theta}}$

  $= \frac{PQ\sin\theta\sin\phi}{\sqrt{\sin^2\theta\cos^2\phi - \sin^2\phi\cos^2\theta}}$

  $= \frac{PQ\sin\theta\sin\phi}{\sqrt{(\sin^2\theta(1 - \sin^2\phi)) - \sin^2\phi(1 -
      \sin^2\theta)}}$

  $= \frac{PQ\sin\theta\sin\phi}{\sqrt{\sin^2\theta - \sin^2\phi}}$.

\item The diagram \in{Figure}[fig:28_130] is given below:

  \startplacefigure[reference=fig:28_130]
    \externalfigure[28_130.pdf]
  \stopplacefigure

  Let $B$ be the peak having a height of $h$ with base $A$. Let $PQ$ is the
  horizontal base having a length $2a$ making angle of elevation of $\theta$ from each
  end. Let $R$ be the mid-point of $PQ$ from where the angle of elevation of $B$ is
  $\phi$ as given in the question.

  Thus, $\angle APB = \angle AQB = \theta$ and $\angle ARB = \phi$.

  In $\triangle APB, \tan\theta = \frac{h}{AP} \Rightarrow AP = h\cot\theta$

  Similarly, $AQ = h\cot\theta$ and $AR = h\cot\phi$

  $\because AR$ is the median of the $\triangle APQ$

  $\therefore AP^2 + AQ^2 = 2PR^2 + 2AR^2$

  $\Rightarrow 2h^2\cot^2\theta = 2a^2 + 2h^2\cot^2\phi$

  $\Rightarrow h^(\cot^2\theta - \cot^2\phi) = a^2$

  $\Rightarrow h = \frac{a}{\sqrt{\frac{\cos^2\theta}{\sin^2\theta} -
      \frac{\cos^2\phi}{\sin^2\phi}}}$

  $= \frac{a\sin\theta\sin\phi}{\sqrt{(\sin\phi\cos\theta +
      \cos\phi\sin\theta)(\sin\phi\cos\theta - \cos\phi\sin\theta)}}$

  $= \frac{a\sin\theta\sin\phi}{\sqrt{\sin(\theta + \phi)\sin(\phi - \theta)}}$

\item The diagram \in{Figure}[fig:28_131] is given below:

  \startplacefigure[reference=fig:28_131]
    \externalfigure[28_131.pdf]
  \stopplacefigure

  Let $B$ be the top of the hill such that height of the hill $AB$ is $h$ and $P,
  R, Q$ be the three consecutive milestones. Given, $\angle APB = \alpha, \angle ARB = \beta,
  \angle AQB = \gamma$.

  In $\triangle APB, \tan\alpha = \frac{h}{AP}\Rightarrow AP = h\cot\alpha$

  Similalrly, $AR = h\cot\beta$ and $AQ = h\cot\gamma$

  Also, $PR = QR = 1$ mile.

  $\because PR = QR, AR$ is the median of the triangle $APR$.

  $\therefore AP^2 + AR^2 = 2PR^2 + 2AQ^2 \Rightarrow h^2(\cot^2\alpha + \cot^2\gamma) = 2 +
  2h^2\cot^2\beta$

  $\Rightarrow h = \sqrt{\frac{2}{\cot^2\alpha + \cot^2\gamma - 2\cot^2\beta}}$ miles.

\item The diagram \in{Figure}[fig:28_132] is given below:

  \startplacefigure[reference=fig:28_132]
    \externalfigure[28_132.pdf]
  \stopplacefigure

  Let $OP$ be the tower haing a height of $h$ which is to be found. Let $ABC$ be the
  equilateral triangle. Given that $OP$ subtends angles of $\alpha, \beta, \gamma$ at
  $A, B, C$ respectively. Given that $\tan\alpha = \sqrt{3} + 1$ and $\tan\beta =
  \tan\gamma = \sqrt{2}$. It is given that $OP$ is perpendicular to the plane of $\triangle
  ABC$.

  In $\triangle AOP, \tan\alpha = \frac{h}{OA} \Rightarrow \sqrt{3} + 1 = \frac{h}{OA} \Rightarrow
  OA = \frac{h}{\sqrt{3} + 1}$

  Similarly, $OB = OC = \frac{h}{\sqrt{2}}$

  In $\triangle AOB$ and $AOC$, $AB = AC, OB = OC, OA$ is common. $\therefore
  \triangle AOB$ and $\triangle AOC$ are equal.

  $\therefore \angle OAB = \angle OAC$ but $\angle BAC = 60^\circ$

  $\Rightarrow \angle OAB = \angle OAC = 30^\circ$

  Using sine rule in the $\triangle OAB, \frac{OB}{\sin30^\circ} = \frac{OA}{\sin\theta}$ (let
  $\angle ABO = \theta$)

  $\Rightarrow \frac{\frac{h}{\sqrt{2}}}{\frac{1}{2}} = \frac{\frac{h}{\sqrt{3} + 1}}{\sin\theta}$

  $\Rightarrow \sin\theta = \frac{\sqrt{3} - 1}{2\sqrt{2}} = \sin15^\circ$

  $\Rightarrow \theta = 15^\circ$

  $\Rightarrow \angle OBD = \angle ABC - \theta = 45^\circ$

  In $\triangle BOC, OB=OC, OD\perp BC \therefore BD = DC = 40'$

  In $\triangle OBD, \cos45^\circ = \frac{BD}{OB} \Rightarrow \frac{1}{\sqrt{2}} =
  \frac{40}{h\sqrt{2}}\Rightarrow h = 80'$

\item The diagram \in{Figure}[fig:28_133] is given below:

  \startplacefigure[reference=fig:28_133]
    \externalfigure[28_133.pdf]
  \stopplacefigure

  In the diagram we have shown only one tower instead of three. We will apply cyclic formula to this one
  tower relationships. Let $P$ be the position of the eye and height of $PQ = x$. Let
  $AB$ be the tower having a height of $a$ as given in the question and let the angle
  subtended by $AB$ at $P$ is $\theta$.

  Thus, $\angle APB = \theta, \angle PAQ = \alpha \Rightarrow \angle ABP = 180^\circ - \theta -
  (90^\circ - \alpha) = 90^\circ + (\alpha - \theta)$

  By sine rule in $\triangle APB$,

  $\frac{a}{\sin\theta} = \frac{AP}{\sin[90^\circ + (\alpha - \theta)]} = \frac{AP}{\cos(\alpha -
    \theta)}$

  In $\triangle APQ, \sin\alpha = \frac{x}{AP} \Rightarrow AP = \frac{x}{\sin\alpha}$

  $\Rightarrow \frac{a}{\sin\theta} = \frac{x}{\sin\alpha\cos(\alpha - \theta)}$

  $\Rightarrow x\sin\theta = a\sin\alpha\cos(\alpha - \theta)\Rightarrow \cos(\alpha - \theta) =
  \frac{x\sin\theta}{a\sin\alpha}$

  If we consider the other two towers we will have similar relations i.e.

  $\Rightarrow \cos(\beta - \theta) = \frac{x\sin\theta}{b\sin\beta}$ and $\cos(\gamma -
  \theta) = \frac{x\sin\theta}{c\sin\gamma}$

  Now, $\frac{\sin(\beta - \gamma)}{a\sin\alpha} + \frac{\sin(\gamma - \alpha)}{b\sin\beta} +
  \frac{\sin(\alpha - \beta)}{c\sin\gamma}$

  $=\displaystyle\sum\frac{\sin(\alpha - \beta)}{c\sin\gamma} =
  \displaystyle\sum\frac{\sin(\alpha - \theta + \theta - \beta)}{c\sin\gamma}$

  $= \displaystyle\sum\frac{\sin(\alpha - \theta)\cos(\theta - \beta) + \cos(\alpha -
    \theta)\sin(\theta - \beta)}{c\sin\gamma}$

  $=\displaystyle\sum\frac{1}{c\sin\gamma}\left[\frac{\sin(\alpha - \theta)x\sin\theta}{b\sin\beta}
    + \frac{\sin(\theta - beta)x\sin\theta}{a\sin\alpha}\right]$

  $= \displaystyle\sum\frac{x\sin\theta}{abc\sin\alpha\sin\beta\sin\gamma}\left[a\sin(\alpha -
    \theta)\sin\alpha - b\sin(\beta - \theta)sin\beta\right]$

  $= \frac{x\sin\theta}{abc\sin\alpha\sin\beta\sin\gamma}.0 = 0$

\item The diagram \in{Figure}[fig:28_134] is given below:

  \startplacefigure[reference=fig:28_134]
    \externalfigure[28_134.pdf]
  \stopplacefigure

  Let $S$ be the initial position of the man and $P$ and $Q$ be the poosition of the
  objects. Since $PQ$ subtends greatest angle at $R$, a circle will pass through $P, Q$
  and $R$ and $RS$ will be a tangent to this circle at $R$.

  Also, $\angle PQR = \angle PRS = \theta$ (let). Let $PQ = x$.

  Clearly $\angle SRQ = \theta + \beta$

  Using sine law in $\triangle PRQ, \frac{x}{\sin\beta} = \frac{PR}{\sin\theta} \Rightarrow x =
  \frac{PR\sin\beta}{\sin\theta}$

  Using sine law in $\triangle PRS, \frac{PR}{\sin\alpha} = \frac{c}{\sin(\theta + \beta)}
  \Rightarrow PR = \frac{c\sin\alpha}{\sin(\theta + \beta)}$

  $\Rightarrow x = \frac{c\sin\alpha\sin\beta}{\sin(\theta + \beta)\sin(\theta)} =
  \frac{2c\sin\alpha\sin\beta}{2\sin(\theta + \beta)\sin(\theta)}$

  $= \frac{2c\sin\alpha\sin\beta}{\cos\beta - \cos(2\theta + \beta)}$

  In $\triangle QRS, \alpha + \beta + 2\theta = 180^\circ \Rightarrow 2\theta + \beta = 180^\circ -
  \alpha$

  $\Rightarrow \cos(2\theta + \beta) = -\cos\alpha$

  $\Rightarrow x = \frac{2c\sin\alpha\sin\beta}{\cos\alpha + \cos\beta}$.

\item The diagram \in{Figure}[fig:28_135] is given below:

  \startplacefigure[reference=fig:28_135]
    \externalfigure[28_135.pdf]
  \stopplacefigure

  Let $OP$ be the tower having a height of $h$ and $PQ$ be the flag-staff having a
  height of $x$. $A$ and $B$ are the two points on the horizontal line $OA$. Let
  $OB = y$. Given, $AB = d, \angle QAP = \angle QBP = \alpha$.

  Since $\angle QAP = \angle QBP$, a circle will pass through the points $A, B, P$ and
  $Q$ because angles in the same segment of a circle are equal.

  Thus, $\angle BAP = \angle BQP = \beta$ (angles on the same segment $BP$)

  $\Rightarrow \angle BPO = \angle QAO = \alpha + \beta$

  In $\triangle AOP, \tan\beta = \frac{h}{y + d}$

  In $\triangle BOP, \tan(\alpha + \beta) = \frac{y}{h} \Rightarrow y = h\tan(\alpha + \beta)$

  $\Rightarrow h = y\tan\beta + d\tan\beta\Rightarrow h = \frac{d\tan\beta}{1 - \tan(\alpha +
    \beta)\tan\beta}$

  In $\triangle BOQ, \tan\beta = \frac{y}{x + h}\Rightarrow x\tan\beta + h\tan\beta =
  h\tan(\alpha + \beta)$

  $\Rightarrow x = \frac{d[\tan(\alpha + \beta) - \tan\beta]}{1 - \tan(\alpha + \beta) + \tan\beta}$

\item This question is same as 92 with $\alpha$ replaced by $\theta$ and $\beta$
  replaced by $\phi$.

  Referring to diagram of 92, $AC = \frac{h(\tan\theta + \tan\phi)}{\tan\phi - \tan\theta}$

  $= \frac{h(\sin\theta\cos\phi + \sin\phi\cos\theta)}{\sin\phi\cos\theta - \sin\theta\cos\phi}$

  $= \frac{h\sin(\theta + \phi)}{\sin(\phi - \theta)}$.

\item The diagram \in{Figure}[fig:28_137] is given below:

  \startplacefigure[reference=fig:28_137]
    \externalfigure[28_137.pdf]
  \stopplacefigure

  Let $BC$ represent the road inclined at $10^\circ$ to the vertical towards sun and
  $AB = 2.05$ m represents the shadow where the elevation of the sun is $\angle BAC =
  38^\circ$. Thus, $\angle BCA = 180^\circ - (10^\circ + 90^\circ + 38^\circ) = 42^\circ$.

  Using sine rule in $\triangle ABC$,

  $\frac{BC}{\sin38^\circ} = \frac{AB}{\sin426\circ}\Rightarrow BC =
  \frac{2.05\sin38^\circ}{\sin42^\circ}$.

\item The diagram \in{Figure}[fig:28_138] is given below:

  \startplacefigure[reference=fig:28_138]
    \externalfigure[28_138.pdf]
  \stopplacefigure

  Let $CD$ be the tower having a height of $h$ m. Let $BC$ be its shadow when altitude
  of the sun is $60^\circ$ and $AC$ be its shadow when altitude of the sun is
  $30^\circ$.

  Given that shadow decreases by $30$ m when altitude changes from $30^\circ$ to
  $60^\circ$ i.e. $AB = 30$ m. Let $BC = x$ m.

  In $\triangle BCD, \tan60^\circ = \frac{h}{x} \Rightarrow h = \sqrt{3}x$

  In $\triangle ACD, \tan30^\circ = \frac{h}{x + 30} \Rightarrow h = 15\sqrt{3}$ m.

\item This problem is similar to 138 and has been left as an exercise.

\item This problem is similar to 96 and has been left as an exercise. The answer is $90$ seconds.

\item The diagram \in{Figure}[fig:28_141] is given below:

  \startplacefigure[reference=fig:28_141]
    \externalfigure[28_141.pdf]
  \stopplacefigure

  Let $C$ be the position of the aeroplane flying $3000$ m above ground and $D$ be the
  aeroplane below it. Given that the angles of elevation of these aeroplanes are $45^\circ$ and
  $60^\circ$ respectively. Let the height of $D$ is $h$ m and $AB = d$ m.

  In $\triangle ABC, \tan60^\circ = \sqrt{3} = \frac{3000}{d} \Rightarrow d = 1000\sqrt{3}$ m.

  In $\triangle ABD, \tan45^\circ = 1 = \frac{h}{d} \Rightarrow h = 1000\sqrt{3}$ m.

  $\therefore$ Distance between heights of the aeroplanes $= CD = 3000 - 1000\sqrt{3} =
  1268$ m.

\item The diagram \in{Figure}[fig:28_142] is given below:

  \startplacefigure[reference=fig:28_142]
    \externalfigure[28_142.pdf]
  \stopplacefigure

  Let $C$ and $D$ be two consecutive milestones so that $CD = 1$ mile. Let $D$ be
  position of aeroplane having a height $h$ above $A$, to which angles of elevation are
  $\alpha$ and $\beta$ from $C$ and $D$ respectively. Let $AC = x
  \Rightarrow AD = 1 - x$.

  In $\triangle ABC, \tan\alpha = \frac{h}{x} \Rightarrow h = x\tan\alpha$

  In $\triangle ABD, \tan\beta = \frac{h}{1 - x} \Rightarrow h =
  \frac{\tan\alpha\tan\beta}{\tan\alpha + \tan\beta}$

\item This problem is similar to 119 and has been left as an exercise.

\item The diagram \in{Figure}[fig:28_144] is given below:

  \startplacefigure[reference=fig:28_144]
    \externalfigure[28_144.pdf]
  \stopplacefigure

  Using $m:n$ theorem,

  $2c\cot(\theta - 30^\circ) = c\cot15^\circ - c\cot30^\circ$

  $\Rightarrow \cot(\theta - 30^\circ) = 1 = \cot45^\circ$

  $\Rightarrow \theta = 75^\circ$.

\item This problem is simmilar to 138, and has been left as an exercise.

\item The diagram \in{Figure}[fig:28_146] is given below:

  \startplacefigure[reference=fig:28_146]
    \externalfigure[28_146.pdf]
  \stopplacefigure

  Let $AB$ be the height of air-pilot which has height of $h$. Let $CD$ be the tower
  whose angles of depression of top and bottom of tower be $30^\circ$ and $60^\circ$
  respectively. Draw $DE||AC$ such that $DE = AC$. Let the height of tower $CD$ be
  $x$.

  In $\triangle ABC, \tan60^\circ = \sqrt{3} = \frac{h}{x} \Rightarrow h = x\sqrt{3}$

  In $\triangle ADE, \tan30^\circ = \frac{1}{\sqrt{3}} = \frac{BE}{x}\Rightarrow BE =
  \frac{x}{\sqrt{3}} = \frac{h}{3}$

  $\therefore$ Height of the tower $CD = h - \frac{h}{3} = \frac{2h}{3}$

\item This problem is similar to 146, and has been left as an exercise.

\item The diagram \in{Figure}[fig:28_148] is given below:

  \startplacefigure[reference=fig:28_148]
    \externalfigure[28_148.pdf]
  \stopplacefigure

  As the diagram shows there are two possible solutions. Let $BD$ and $QS$ be the tower of
  height $h$. According to question $BC:CD = 1:2, QR:RS = 1:2$ and $\tan\phi =
  \frac{1}{2}$.

  In $\triangle ABC, \tan\theta = \frac{h}{60}$

  In $\triangle ABD, \tan(\theta + \phi) = \frac{h}{20} = 3\tan\theta$

  $\Rightarrow \frac{\tan\theta + \tan\phi}{1 - \tan\theta\tan\phi} = 3\tan\theta$

  $\Rightarrow \tan\theta = 1, \frac{1}{3}$

  $\Rightarrow h = 20, 60$

\item The diagram \in{Figure}[fig:28_149] is given below:

  \startplacefigure[reference=fig:28_149]
    \externalfigure[28_149.pdf]
  \stopplacefigure

  Let $AB$ be the man given a height of $2$ m making an angle of $\theta$ on the
  opposite side of the bank at $O$. Let $AC$ be the tower having a height $64$ m
  making an angle of $\phi$ at $O$. Let $CD$ be the statue having a height of $8$
  m at the top of tower making the angle $\theta$, which is equal to the angle made by the man at
  $O$. Let the width of the river be $AO = x$ m.

  In $\triangle ABO, \tan\theta = \frac{2}{x}$

  In $\triangle ACO, \tan(\theta + \phi) = \frac{64}{x}$

  In $\triangle ADO, \tan(2\theta + \phi) = \frac{72}{x}$

  $\Rightarrow \frac{\tan(\theta + \phi) + \tan\theta}{1 - \tan(\theta + \phi)\tan\theta} =
  \frac{72}{x}$

  $\Rightarrow \frac{\frac{64}{x} + \frac{2}{x}}{1 - \frac{64}{x}.\frac{2}{x}} = \frac{72}{x}$

  $\Rightarrow x = 16\sqrt{6}$ m.

\item The diagram \in{Figure}[fig:28_150] is given below:

  \startplacefigure[reference=fig:28_150]
    \externalfigure[28_150.pdf]
  \stopplacefigure

  Let $BC$ be the statue given a height of $a$ placed over the column $AB$ given a
  height of $b$. Let both of these make an angle of $\theta$ at $Q$ the top of the
  observer $PQ$ given a height of $h$. Let the distance $AP = d$.

  Clearly, $BQ$ is the bisector of $\angle AQC$ and hence it will divide the opposite side in
  in the ratios of the sides of the angle.

  $\Rightarrow \frac{AB}{BC} = \frac{b}{a} = \frac{AQ}{CQ} \Rightarrow \frac{b^2}{a^2} =
  \frac{d^2 + h^2}{d^2 + (a + b - h)^2}$

  $\Rightarrow (a - b)d^2 = (a + b)b^2 - 2b^2h - (a - b)h^2$
\item The diagram \in{Figure}[fig:28_151] is given below:

  \startplacefigure[reference=fig:28_151]
    \externalfigure[28_151.pdf]
  \stopplacefigure

  We follow the question $74$ and have a similar question. Following the same method we find that
  $h = \sqrt{ab}. BC^2 = AC^2 + AB^2 = a^2 + ab$. Let $CD$ subtend an angle of $\alpha$
  at $B$.

  In $\triangle BCD$, using sine rule

  $\frac{CD}{\sin\alpha} = \frac{BC}{\sin(90^\circ - \theta)}$

  $\Rightarrow \sin\alpha = \frac{CD}{BC}\cos\theta = \frac{CD}{BC}.\frac{AC}{BC} = \frac{(b -
    a)a}{a^2 + ab} = \frac{b - a}{a + b}$.

\item The diagram \in{Figure}[fig:28_152] is given below:

  \startplacefigure[reference=fig:28_152]
    \externalfigure[28_152.pdf]
  \stopplacefigure

  Let $BC$ be the pillar given a height of $h$ and $CD$ be the statue having a height
  of $x$. Both the statue and the pillar make the same angle at $A$ which we have let to be
  $\theta$.

  In $\triangle ABC, \tan\theta = \frac{h}{d}$

  In $\triangle ABD, \tan2\theta = \frac{2\tan\theta}{1 - \tan^2\theta} = \frac{h + x}{d}$

  $\Rightarrow \frac{\frac{2h}{d}}{1 - \frac{h^2}{d^2}} = \frac{h + x}{d} \Rightarrow h + x =
  \frac{2hd^2}{d^2 - h^2}$

  $\Rightarrow x = \frac{h(d^2 + h^2)}{d^2 - h^2}$.

\item The diagram \in{Figure}[fig:28_153] is given below:

  \startplacefigure[reference=fig:28_153]
    \externalfigure[28_153.pdf]
  \stopplacefigure

  Let $BE$ be the tower and $CD$ be the pole such that base of the tower is at half the
  height of the pole. Given height of the tower is $50'$. Aangles of depression of the top and the
  foot of the pole from top of the tower are given as $15^\circ$ and $45^\circ$. Let the
  distance between the pole and the tower be $d'$.

  In $\triangle ABC, \tan45^\circ = 1 = \frac{50 + \frac{h}{2}}{d}\Rightarrow d = 50 + \frac{h}{2}$

  In $\triangle BGD, \tan15^\circ = \frac{\sqrt{3} - 1}{\sqrt{3 + 1}} = \frac{50 - \frac{h}{2}}{d}$

  $\Rightarrow h = 100/\sqrt{3}$ ft.

\item The diagram \in{Figure}[fig:28_154] is given below:

  \startplacefigure[reference=fig:28_154]
    \externalfigure[28_154.pdf]
  \stopplacefigure

  Given $A$ is the initial point of observation and $D$ is the second point of observation
  which is $4$ km south of $A$. Let $P$ be the point in air where the plane is flying
  and $Q$ be the point directly beneath it. Given that $Q$ is directly east of $A$ and
  angles of elevation from $A$ and $D$ are respectively $60^\circ$ and
  $30^\circ$. Let $PQ = h$ km be the height of the airplane. Clearly, $\angle DAQ$ is a
  right angle.

  In $\triangle APQ, \tan60^\circ = \frac{h}{AQ} \Rightarrow AQ = h\cot60^\circ$

  In $\triangle DPQ, \tan30^\circ = \frac{h}{DQ} \Rightarrow DQ = h\cot30^\circ$

  In $\triangle ADQ, DQ^2 = AB^2 + AQ^2 \Rightarrow h^2\cot^230^\circ = h^2\cot^260^\circ + 4^2$

  $\Rightarrow h = \sqrt{6}$ km.

\item The diagram \in{Figure}[fig:28_155] is given below:

  \startplacefigure[reference=fig:28_155]
    \externalfigure[28_155.pdf]
  \stopplacefigure

  Let $PN$ be the flag-staff having a height of $h$. $AB$ is perpendicular to
  $AN$. Let $AN = x$ and $BN = y$. Given angles of elevation from $A$ and
  $B$ to $P$ are $\alpha$ and $\beta$ respectively.

  In $\triangle APN, x = h\cot\alpha$. In $\triangle BPN, y = h\cot\beta$

  In $\triangle ABN, AB^2 + h^2\cot^2\alpha = h^2\cot^2\beta$

  $\Rightarrow h = \frac{AB}{\sqrt{\cot^2\beta - \cot^2\alpha}} =
  \frac{AB\sin\alpha\sin\beta}{\sin(\alpha + \beta)\sin(\alpha - \beta)}$

\item The diagram \in{Figure}[fig:28_156] is given below:

  \startplacefigure[reference=fig:28_156]
    \externalfigure[28_156.pdf]
  \stopplacefigure

  Let $AC$ be the tower having a height of $h$ such that $AB:BC::1:9$. Given the point
  at a distance of $20$ m is where both $AB$ and $BC$ subtend equal angle which we have
  let to be $\theta$.

  In $\triangle ABD, \tan\theta = \frac{h}{10*20} \Rightarrow h = 200\tan\theta$

  In $\triangle ACD, \tan2\theta = \frac{h}{20} = 10\tan\theta$

  $\Rightarrow \frac{2\tan\theta}{1 - \tan^2\theta} = 10\tan\theta$

  $\Rightarrow 10\tan^2\theta -8 = 0 \Rightarrow \tan\theta = \frac{2}{\sqrt{5}}$

  $\Rightarrow h = 80\sqrt{5}$ m.

\item The diagram \in{Figure}[fig:28_157] is given below:

  \startplacefigure[reference=fig:28_157]
    \externalfigure[28_157.pdf]
  \stopplacefigure

  Let $BC$ be the tower inclined at angle an angle $\theta$ from horizontal having a vertical
  height of $h$. Let $A$ and $D$ be two equidistant points from base $B$ of the
  tower from where the angles of elevation to the top of the tower is $\alpha$ and $\beta$
  respectively. Let $AB = BD = d$. Let $BE = x$.

  In $\triangle BCE, \tan\theta = \frac{h}{x} \Rightarrow x = h\cot\theta$

  Clearly, $AE = d - h\cot\theta$ and $BE = d + h\cot\theta$

  In $\triangle ACE, \tan\alpha = \frac{h}{d - h\cot\theta}$ and in $\triangle BCE, \tan\beta
  = \frac{h}{d + h\cot\theta}$

  $\Rightarrow \frac{1}{\cot\alpha + \cot\theta} = \frac{1}{\cot\beta - \cot\theta}$

  $\Rightarrow \theta = \tan^{-1}\frac{\sin(\alpha - \beta)}{2\sin\alpha\sin\beta}$.

\item The diagram \in{Figure}[fig:28_158] is given below:

  \startplacefigure[reference=fig:28_158]
    \externalfigure[28_158.pdf]
  \stopplacefigure

  Let $ABC$ be the triangle in horizontal plane and $PQ$ be the $10$ m high flag staff
  at the center of the $\triangle ABC$. Given that each side subtends an angle of $60^\circ$
  at the top of flag staff i.e. $Q$.

  $\therefore  \angle AQC = 60^\circ \Rightarrow AQ = QC$ making $\triangle AQC$ equilateral.

  Let $AQ = QC = AC = 2a$. We know that centroid is a point on median from where the top of the
  vertex is at a distance of $\frac{2}{3}$ rd times length of a side. We also know that median of
  an equilateral triangle is perpendicular bisector of the opposite side.

  $\Rightarrow AP = \frac{2}{3}.2a.\sin60^\circ = \frac{2a}{\sqrt{3}}$

  $\triangle APQ$ is also a right angle triangle with right angle at $P$.

  $\Rightarrow AQ^2 = AP^2 + PQ^2 \Rightarrow a = 5\sqrt{\frac{3}{2}}$

  $\therefore$ Length of a side $= 2a = 5\sqrt{6}$ m.

\item The diagram \in{Figure}[fig:28_159] is given below:

  \startplacefigure[reference=fig:28_159]
    \externalfigure[28_159.pdf]
  \stopplacefigure

  Let $AB$ be the pole having a height of $h$ then the height of the second pole $CD$
  would be $2h$. $O$ is the point of observation situated at mid-point between the poles
  i.e. at a distance of $60$ m from each pole. Let $\angle AOB = \theta$ and therefore
  $\angle COD = 90^\circ - \theta$.

  In $\triangle AOB, \tan\theta = \frac{h}{60}$

  In $\triangle COD, \tan(90^\circ - \theta) = \cot\theta = \frac{2h}{60} = 2\tan\theta \Rightarrow
  \tan\theta = \frac{1}{\sqrt{2}}$

  $\Rightarrow h = 30\sqrt{2}$ m and $2h = 60\sqrt{2}$ m.

\item This problem is similar to 158, and has been left as an exercise.

\item This problem is similar to 134, and has been left as an exercise.

\item The diagram \in{Figure}[fig:28_162] is given below:

  \startplacefigure[reference=fig:28_162]
    \externalfigure[28_162.pdf]
  \stopplacefigure

  Since $AB$ and $CD$ are two banks of a straight river they would be parallel. We have shown
  alternate angles for $\beta$ and $\gamma$ in the diagram other than given angles. In
  $\triangle ABC, \angle ACB = \pi - (\alpha + \beta + \gamma) \Rightarrow \sin ACB = \sin(\alpha +
  \beta + \gamma)$.

  Using sine formula in $\triangle ABC$,

  $\frac{AB}{\sin ACB} = \frac{AC}{\sin ABC} \Rightarrow AC = \frac{a\sin\gamma}{\sin(\alpha +
    \beta + \gamma)}$

  Using sine formula in $\triangle ACD$,

  $\frac{CD}{\sin\alpha} = \frac{AC}{\sin\beta} \Rightarrow CD =
  \frac{a\sin\alpha\sin\gamma}{\sin\beta\sin(\alpha + \beta + \gamma)}$

\item The diagram \in{Figure}[fig:28_163] is given below:

  \startplacefigure[reference=fig:28_163]
    \externalfigure[28_163.pdf]
  \stopplacefigure

  Let $PQ$ be the bank of river having a width of $b$ and $R$ be the point in line with
  $PQ$ at a distance of $a$ from $Q$. $QS$ is the distance of $100$ m to
  which the person walks at right angle from initial line.

  In $\triangle PRS, \tan40^\circ = \frac{a + b}{100}$

  In $\triangle QRS, \tan25^\circ = \frac{b}{100}$

  $\Rightarrow b = 100(\tan40^\circ - \tan25^\circ)$.

\item This problem is similar to 96, and has been left as an exercise.

\item This problem is similar to 96, and has been left as an exercise.

\item The diagram \in{Figure}[fig:28_166] is given below:

  \startplacefigure[reference=fig:28_166]
    \externalfigure[28_166.pdf]
  \stopplacefigure

  Let $P$ and $Q$ be the tops of two spires, $P'$ and $Q'$ be their
  reflections. From question $OA = h$. Let $BP = BP' = h1, CQ = CQ' = h$

  Let the distance between spires be $x = MN = OM - ON$.

  In $\triangle OMP', \tan\beta = \frac{h + h1}{OM} \Rightarrow OM\tan\beta = h + h1$

  In $\triangle OMP, \tan\alpha = \frac{h1 - h}{OM} \Rightarrow OM\tan\alpha = h1 - h$

  $\Rightarrow OM(\tan\beta - \tan\alpha) = 2h \Rightarrow OM = \frac{2h}{\tan\beta - \tan\alpha}$

  Similarly, $ON = \frac{2h}{\tan\gamma - \tan\alpha}$

  $\Rightarrow x = OM - ON = 2h\left[\frac{1}{\tan\beta - \tan\alpha} - \frac{1}{\tan\gamma
      -\tan\alpha}\right]$

  On simplification we arrive at the desired result.

\item The diagram \in{Figure}[fig:28_167] is given below:

  \startplacefigure[reference=fig:28_167]
    \externalfigure[28_167.pdf]
  \stopplacefigure

  Let $O$ be the center of the square and $OP$ be the pole having a height of $h$. Let
  $OQ$ be the shdow of the pole. Given $CO = x$ and $BQ = y$. Then $BC = x +
  y$. Let $OR\perp BC$.

  $\therefore OR = BR = \frac{x + y}{2}$ and $QR = \frac{x - y}{2}$

  In $\triangle POQ, \tan\alpha = \frac{h}{OQ}\Rightarrow OQ = h\cot\alpha$

  In $\triangle ORQ, OQ^2 = OR^2 + QR^2 \Rightarrow h^2\cot^2\alpha = \left(\frac{x +
    y}{2}\right)^2 + \left(\frac{x - y}{2}\right)^2$

  $\Rightarrow h = \sqrt{\frac{x^2 + y^2}{2}}\tan\alpha$

\item The diagram \in{Figure}[fig:28_168] is given below:

  \startplacefigure[reference=fig:28_168]
    \externalfigure[28_168.pdf]
  \stopplacefigure

  Let $OP$ be the vertical height $c$ of the candle. $O'$ is the point vertically below
  $O$ therefore $OO' = b$ as given in the question. Let $EF$ represent the line of
  intersection of the wall and the horizontal ground. Draw $O'D\perp EF$ then $O'D = a$.

  Clearly, $EF = 2DE$ as shadow is symmetrical about line $O'D$,

  In similar triangles $AOP$ and $PO'E$,

  $\frac{OA}{O'E} = \frac{OP}{O'P} \Rightarrow \frac{a}{O'E} = \frac{c}{b + c} \Rightarrow O'E =
  \frac{a(b + c)}{c}$

  In $\triangle O'DE$,

  $O'E^2 = a^2 + DE^2 \Rightarrow DE = \frac{a}{c}\sqrt{b^2 + 2bc}$

  $\Rightarrow EF = 2DE \frac{2a}{c}\sqrt{b^2 + 2bc}$

\item The diagram \in{Figure}[fig:28_169] is given below:

  \startplacefigure[reference=fig:28_169]
    \externalfigure[28_169.pdf]
  \stopplacefigure

  Let $PABCD$ be the pyramid, $PQ$ the flag-staff having a height of $6$ m. Let
  $OP = h$ and the shadow touches the side at $L$.

  Proceeding like problem $167$, we have in $\triangle OML$,

  $OL^2 = OM^2 + LM^2 \Rightarrow (h + 6)^2\cot^2\alpha = \left(\frac{x + y}{2}\right)^2 +
  \left(\frac{x - y}{2}\right)^2$

  $\Rightarrow h = \sqrt{\frac{x^2 + y^2}{2}}\tan\alpha - 6$

\item The diagram \in{Figure}[fig:28_170] is given below:

  \startplacefigure[reference=fig:28_170]
    \externalfigure[28_170.pdf]
  \stopplacefigure

  Let $PQ$ be the tower with given height $h, C$ be the initial point of observation from
  where angle of elevation is $\theta$. When the man moves a distance $d$ let him reach point
  $B$ from where angle of elevation is $2\theta$ and then final point be $A$ which is
  at a distance of $\frac{3}{4}d$ from $B$, having an angle of elevation $3\theta$.

  $\angle QCB = \angle CQB = \theta \therefore BC = BQ = d$

  In $\triangle PQB, \sin2\theta = \frac{h}{d} \Rightarrow h = 2d\sin\theta\cos\theta$

  Using sine rulel in $\triangle ABQ, \frac{3d}{4\sin\theta} = \frac{d}{\sin(180^\circ - \theta)}$

  $\Rightarrow \frac{3}{4\sin\theta} = \frac{1}{3\sin\theta - 4\sin^3\theta}$

  $\Rightarrow \sin^2\theta = \frac{5}{12} \therefore \cos^2\theta = \frac{7}{12}$

  $\Rightarrow h^2 = 4d^2\sin^2\theta\cos^2\theta \Rightarrow 36h^2 = 35d^2$

\item The diagram \in{Figure}[fig:28_171] is given below:

  \startplacefigure[reference=fig:28_171]
    \externalfigure[28_171.pdf]
  \stopplacefigure

  Let $O$ be the mid-point of $AB$ having a measure of $8$ m. Let $OP$ be the
  $2$ m long object, $PQ$ be its position after $1$ second and $RS$ be the
  position after $2$ seconds.

  $\angle PAQ = \alpha, \angle RAS = \beta$ as given in the problem. Also given,

  $\frac{ds}{dt} = 2t + 1 \Rightarrow \int ds = \int(2t + 1)dt \Rightarrow s = t^2 + t + k$
  where $k$ is the constant of acceleration. At $t = 0, s = 0 \Rightarrow k = 0$

  At $t = 1, s = 2$ and $t = 2, s = 6 \therefore OP = PQ = QR = RS = 2$ m.

  Let $\angle OAP = \theta_1, \angle OAQ = \theta_2, \angle OAR = \theta_3$ and $\angle OAS =
  \theta_4$

  $\Rightarrow \tan\alpha = \tan(\theta_2 - \theta_1) = \frac{\tan\theta_2 - \theta_1}{1 +
    \tan\theta_1\tan\theta_2} = \frac{1 - \frac{1}{2}}{1 + \frac{1}{2}} = \frac{1}{3}$

  Similarly, $\tan\beta = \frac{1}{8}$

  $\Rightarrow \cos(\alpha - \beta) = \frac{5}{\sqrt{26}}$

\item The diagram \in{Figure}[fig:28_172] is given below:

  \startplacefigure[reference=fig:28_172]
    \externalfigure[28_172.pdf]
  \stopplacefigure

  Let $OD$ be the pole having a height of $h$. Given that $\triangle ABC$ is isosceles
  and $B$ and $C$ subtend same angle at $P$ which is feet of the observer, therefore
  $AB = AC$. Let $BD = DC = x$. Given $\angle APO = \beta, \angle CPQ = \alpha$ and
  $OP = d$.

  $\Delta ABC = \frac{1}{2}BC.AD = x.AD$

  In $\triangle AOP, \tan\beta = \frac{h + AD}{d} \Rightarrow AD = d\tan\beta - h$

  In $\triangle CQP, \tan\alpha = \frac{h}{PQ} \Rightarrow PQ = h\cot\alpha$

  In $\triangle OPQ, OQ^2 = PQ^2 - OP^2 \Rightarrow OQ = \sqrt{h^2\cot^2\alpha - d^2}$

  $\Rightarrow \Delta ABC = (d\tan\beta - h)\sqrt{h^2\cot^2\alpha - d^2}$

\item The diagram \in{Figure}[fig:28_173] is given below:

  \startplacefigure[reference=fig:28_173]
    \externalfigure[28_173.pdf]
  \stopplacefigure

  In the diagram $A, Q, B$ are in the plane of paper and $PQ$ is perpedicular to the plane
  of paper.

  In $\triangle APQ, \tan(90^\circ - \theta) = \frac{h}{AQ}$

  In $\triangle BPQ, \tan\theta = \frac{h}{BQ}$

  $\Rightarrow h = \sqrt{AQ.BQ}$

  Since $BQ$ is north-west $\therefore \angle AQB = 45^\circ = \angle QBA\Rightarrow AQ = AB
  = 100$ m.

  In $\triangle ABQ, OB = \sqrt{AQ^2 + AB^2} = 100\sqrt{2}$ m.

  $\Rightarrow h = 100\sqrt[4]{2}$ m.

\item The diagram \in{Figure}[fig:28_174] is given below:

  \startplacefigure[reference=fig:28_174]
    \externalfigure[28_174.pdf]
  \stopplacefigure

  Let $AB$ and $CD$ be the vertical poles having heights of $a$ and $b$
  respectively and angle of elevation $\alpha$ from $O$ which is same for both of them. Also,
  the angles of elevation from $P$ are $\beta$ and $\gamma$ along with $\angle
  APC = 90^\circ$.

  In $\triangle ABQ, \tan\alpha = \frac{a}{AQ} \Rightarrow AQ = a\cot\alpha$

  In $\triangle CDQ, \tan\alpha = \frac{b}{CQ} \Rightarrow CQ = b\cot\alpha$

  $\Rightarrow AC = AQ + CQ = (a + b)\cot\alpha$

  In $\triangle ABP, \tan\beta = \frac{a}{AP} \Rightarrow AP = a\cot\beta$

  In $\triangle CDP, \tan\gamma = \frac{a}{CP}\Rightarrow CP = b\cot\gamma$

  In $\triangle APC, AC^2 = AP^2 + CP^2$

  $\Rightarrow (a + b)^2\cot^2\alpha = a^2\cot^2\beta + b^2\cot^2\gamma$

\item The diagram \in{Figure}[fig:28_175] is given below:

  \startplacefigure[reference=fig:28_175]
    \externalfigure[28_175.pdf]
  \stopplacefigure

  Given the pole is $PQ$, let $h$ be its height. $PQ$ is perpedicular to the plane of
  paper i.e $ABC. \therefore \angle QPA = \angle QPB = \angle QPC = 90^\circ$

  In $\triangle APQ, \tan\theta = \frac{h}{PA} \Rightarrow PA = h\cot\theta$

  Similarly, $PB = PC = h\cot\theta = PA$

  Hence, $P$ is the circumcenter of the $\triangle ABC$ and $PA$ is circum-radius of
  the circumcircle.

  $\therefore h = PA\tan\theta = \frac{abc}{4\Delta}\tan\theta$

\item The diagram \in{Figure}[fig:28_176] is given below:

  \startplacefigure[reference=fig:28_176]
    \externalfigure[28_176.pdf]
  \stopplacefigure

  Let $PQ$ be the tower having a height of $h$ and $\angle AOP = \theta$. Given that
  $\tan\theta = \frac{1}{\sqrt{2}}$

  In $\triangle AOP, \tan\theta = \frac{AP}{AO} \Rightarrow AP = 150\sqrt{2}$ m.

  In $\triangle POQ, \tan30^\circ = \frac{h}{OP} \Rightarrow OP = h\sqrt{3}$

  In $\triangle AOP, OP^2 = OA^2 + AP^2 \Rightarrow 3h^2 = 300^2 + (150\sqrt{2})^2$

  $\Rightarrow h = 150\sqrt{2}$ m,

  $\therefore \tan\phi = \frac{h}{AP} = 1 \Rightarrow \phi = 45^\circ$.

\item The diagram \in{Figure}[fig:28_177] is given below:

  \startplacefigure[reference=fig:28_177]
    \externalfigure[28_177.pdf]
  \stopplacefigure

  Let $OB = h, OA = x$. In $\triangle AOB, \tan\alpha =\frac{x}{h}$

  $\Rightarrow x = h\tan\alpha$

  In $\triangle BOC, \tan\beta = \frac{h}{d - x} \Rightarrow d = h\tan\alpha + h\cot\beta$

  In $\triangle BOD, \tan\gamma = \frac{h}{d + x}\Rightarrow d = h\cot\gamma - h\tan\alpha$

  $\therefore h\tan\alpha + h\cot\beta = h\cot\gamma - h\tan\alpha$

  $\Rightarrow 2\tan\alpha = \cot\gamma - \cot\beta$.

\item The diagram \in{Figure}[fig:28_178] is given below:

  \startplacefigure[reference=fig:28_178]
    \externalfigure[28_178.pdf]
  \stopplacefigure

  Let $M$ be the mid-point of $ES$ such that $SM = ME = x$ and $OP$ be the tower
  having a height of $h$.

  In $\triangle EOP, \tan\alpha = \frac{h}{OE} \Rightarrow OE = h\cot\alpha$

  Similarly in $\triangle OPS, OS = h\cot\beta$ and in $\triangle MOP, OM = h\cot\theta$

  Since $OE$ is eastward and $OS$ is southward $\Rightarrow EOS = 90^\circ$

  $\Rightarrow ES^2 = OS^2 + OE^2 \Rightarrow 4x^2 = h^2(\cot^2\beta + cot^2\alpha)$

  Since $M$ is mid-point of $ES, OM$ would be the median.

  $\Rightarrow OS^2 + OE^2 = 2MS^2 + 2OM^2$

  $\Rightarrow h^2\cot^2\beta + h^2\cot^2\alpha = \frac{h^2(\cot^2\beta + \cot^2\alpha)}{2} +
  2h^2\cot^2\theta$

  $\Rightarrow \cot^2\beta + \cot^2\alpha = 4\cot^2\theta$

\item The diagram \in{Figure}[fig:28_179] is given below:

  \startplacefigure[reference=fig:28_179]
    \externalfigure[28_179.pdf]
  \stopplacefigure

  Let $AP$ be the tree having a height of $h$ and $AB$ be the width of canal equal to
  $x$. Given, $BC = 20$ m and $\angle BAC = 120^\circ$.

  In $\triangle ABP, \tan60^\circ = \frac{h}{AB} \Rightarrow AB = \frac{h}{\sqrt{3}}$

  In $\triangle ACP, \tan30^\circ = \frac{h}{AC}\Rightarrow AC = \sqrt{3}h$

  Using cosine rule in $\triangle ABC,$

  $\cos120^\circ = \frac{AB^2 + BC^2 - AC^2}{2.AB.BC}\Rightarrow -\frac{1}{2} = \frac{\frac{h^2}{3}
    + 20^2 - 3h^2}{2.20.\frac{h}{\sqrt{3}}}$

  $\Rightarrow 2h^2 - 5\sqrt{3}h - 300 = 0\Rightarrow h = \frac{5\sqrt{3} + 15\sqrt{11}}{4}$ m.

  $\Rightarrow AB = \frac{5 + 5\sqrt{33}}{4}$ m.

\item The diagram \in{Figure}[fig:28_180] is given below:

  \startplacefigure[reference=fig:28_180]
    \externalfigure[28_180.pdf]
  \stopplacefigure

  Let $OP$ be the tower with $P$ being the top having a height of $h$. According to
  question $S_1S_2 = S_2S_3, \angle PS_2S_1 = \gamma_1, \angle PS_3S_2 = \gamma_2, \angle S_1PS_2 =
  \delta_1, \angle S_2PS_3 = \delta_2, \angle PS_1O = \beta_1$ and $\angle PS_2O = \beta_2$.

  In $\triangle OPS1, \sin\beta_1 = \frac{h}{PS_1} \Rightarrow PS_1 = \frac{h}{\sin\beta_1}$

  In $\triangle OPS_2, PS_2 = \frac{h}{\sin\beta_2}$

  Using sine rule in $PS_1S_2, \frac{S_1S_2}{\sin\delta_1} = \frac{PS_1}{\sin\gamma_1}$

  $\Rightarrow \frac{h}{S_1S_2} = \frac{\sin\beta_1\sin\gamma_1}{\sin\delta_1}$

  Similarly in $PS_2S_3, \frac{h}{S_2S_3} = \frac{\sin\beta_2\sin\gamma_2}{\sin\delta_2}$

  Equalting last two results we have desired equality.

\item The diagram \in{Figure}[fig:28_181] is given below:

  \startplacefigure[reference=fig:28_181]
    \externalfigure[28_181.pdf]
  \stopplacefigure

  Let $PQ$ be the vertical pillar having a height of $h$. According to question,
  $\tan\alpha = 2, AN = 20$ m and that $\triangle PAM$ is equilateral. Let $\angle QAP
  =\beta, \angle QBP = \gamma$

  In $\triangle NPQ, \tan\alpha = \frac{h}{PN} = 2 \Rightarrow PN = \frac{h}{2}$

  In $\triangle ANP, \tan60^\circ = \sqrt{3} = \frac{PN}{AN} \Rightarrow PN = 20\sqrt{3}
  \Rightarrow h = 40\sqrt{3}$ m.

  $\cos60^\circ = \frac{1}{2} = \frac{AN}{PA} \Rightarrow PA = 40$ m.

  $\triangle PAM$ is equilateral and $PN\perp AM \therefore AN = MN = 20$ m
  $\Rightarrow AM = 40$ m, $\Rightarrow AB = 80$ m.

  $\therefore PB = \sqrt{AB^2 - PA^2} = 40\sqrt{3}$ m.

  $\Rightarrow \beta = 60^\circ$ and $\gamma = 45^\circ$.

\item The diagram \in{Figure}[fig:28_182] is given below:

  \startplacefigure[reference=fig:28_182]
    \externalfigure[28_182.pdf]
  \stopplacefigure

  Let $ABC$ be the triangular park, $O$ be the mid-point of $BC$ and $OP$ be the
  television tower(out of the plane of paper). Given that, $\angle PAO = 45^\circ, \angle PBO =
  60^\circ, \angle PCO = 60^\circ, AB = AC = 100$ m. Also, let $OP = h$ m.

  Clearly, $\angle POA = \angle POB = \angle POC = 90^\circ$.

  In $\triangle POA, \tan45^\circ = \frac{h}{OA}\Rightarrow OA = h$

  In $\triangle POB, \tan60^\circ = \frac{h}{OB}\Rightarrow OB = \frac{h}{\sqrt{3}}$

  Similarly $OC$ would be $\frac{h}{\sqrt{3}}$.

  $\because \triangle ABC$ is an isosceles triangle and $O$ is the mid-point of
  $BC. \therefore AO\perp BC$.

  In $\triangle AOB, AB^2 = OA^2 + OB^2 \Rightarrow h = 50\sqrt{3}$ m.

\item The diagram \in{Figure}[fig:28_183] is given below:

  \startplacefigure[reference=fig:28_183]
    \externalfigure[28_183.pdf]
  \stopplacefigure

  Let $ABCD$ be the base of the square tower whose upper corners are $A', B', C', D'$
  respectively. From a point $O$ on the diagonal $AC$ the three upper corners $A', B'$
  and $D'$ are visible.

  According to question $\angle AOA' = 60^\circ, \angle BOB' = \angle DOD' = 45^\circ$

  Also, $AA' = BB' = h$ and $AB = a$

  In $\triangle AA'O, \tan60^\circ = \frac{h}{AO}\Rightarrow AO = \frac{h}{\sqrt{3}}$

  In $\triangle BB'O, \tan45^\circ = 1= \frac{h}{BO} \Rightarrow BO = h$

  Using cosine rule in $\triangle AOB,$

  $\cos135^\circ = \frac{AO^2 + AB^2 - BO^2}{2AO.AB}$

  $\Rightarrow -\frac{1}{\sqrt{2}} = \frac{\frac{h^2}{3} + a^2 - h^2}{2.\frac{h}{\sqrt{3}}.a}$

  Considering $h > 0$, on simplification we arrive at $\frac{h}{a} = \frac{\sqrt{6}(1 +
    \sqrt{5})}{4}$.

\item The diagram \in{Figure}[fig:28_184] is given below:

  \startplacefigure[reference=fig:28_184]
    \externalfigure[28_184.pdf]
  \stopplacefigure

  In the diagram $PP'R'R$ is a plane perpendicular to the plane of the paper. Let $C$ be the
  center of top of the cylindrical tower. Since $A$ is the point on the horizontal plane nearest to
  $Q$, hence $A$ will be on the line $Q'A$ where $Q'A\perp QQ'$. According to
  question $QQ' = h, C'Q' = r, \angle QAQ' = 60^\circ$ and $\angle PAP' = 45^\circ$.

  In $\triangle AQQ', \tan60^\circ = \sqrt{3} = \frac{h}{AQ'} \Rightarrow AQ' = \frac{h}{\sqrt{3}}$

  In $\triangle APP', \tan45^\circ = 1 = \frac{h}{AP'} \Rightarrow AP' = h$

  $AC' = AQ' + C'Q' = \frac{h}{\sqrt{3}} + r$

  In $\triangle AC'P', AP'^2 = AC'^2 + C'P'^2 \Rightarrow h^2 + \left(\frac{h}{\sqrt{3}} +
  r\right)^2 + r^2$

  Taking into account that $h > 0$, on simplification we arrive at

  $\frac{h}{r} = \frac{\sqrt{3}(1 + \sqrt{5})}{2}$.

\item The diagram \in{Figure}[fig:28_185] is given below:

  \startplacefigure[reference=fig:28_185]
    \externalfigure[28_185.pdf]
  \stopplacefigure

  Let $AP$ be the pole having a height of $h$ m. Let $\angle PCA = \theta, \angle ADB =
  \alpha$ and $\angle BDC = \beta$. Then $\angle PBA = 2\theta$ and $\angle BPC =
  \theta$.

  $\Rightarrow \angle BPC = \angle BCP \Rightarrow BP = BC = 20$ m.

  From question $\tan\alpha = \frac{1}{5}, CD = 30$ m and $BC = 20$ m.

  In $\triangle BCD, \tan\beta = \frac{BC}{CD} = \frac{20}{30} = \frac{2}{3}$

  Now $\tan(\alpha + \beta) = \frac{\tan\alpha + \tan\beta}{1 - \tan\alpha\tan\beta} = 1$

  $\Rightarrow \alpha + \beta = 45^\circ \Rightarrow \angle ADC = \angle DAC = 45^\circ$

  $\Rightarrow AC = CD = 30$ m. $\Rightarrow AB = AC - BC = 30 - 20 = 10$ m.

  In $\triangle PAB, h^2 = PB^2 - AB^2 = 20^2 - 10^2 \Rightarrow h = 10\sqrt{3}$ m.

\item The diagram \in{Figure}[fig:28_186] is given below:

  \startplacefigure[reference=fig:28_186]
    \externalfigure[28_186.pdf]
  \stopplacefigure

  Let $OP$ be the tower having a height of $h, A$ be the initial position of the man,
  $B$ be the second position of the man at a distance $a$ from $A$ and $C$ be the
  final position of the man at a distance of $\frac{5a}{3}$ from $B$. Given that angles of
  elevation from $A, B$ and $C$ of the top of the tower are $30^\circ, 30^\circ$ and
  $60^\circ$ respectively. $OC\perp AB$ and $DN\perp OC$.

  In $\triangle POA, \tan30^\circ = \frac{1}{\sqrt{3}} = \frac{h}{OA} \Rightarrow OA = \sqrt{3}h$

  Similarly in $\triangle POB, OB = \sqrt{3}h$ and in $\triangle POD, \tan60^\circ =
  \frac{h}{OD} \Rightarrow OD = \frac{h}{\sqrt{3}}$

  $\because OA = OA \Rightarrow AC = BC = \frac{a}{2}$

  $OC = \sqrt{OA^2 - AC^2} = \sqrt{3h^2 - \frac{a^2}{4}}, ON = \sqrt{OD^2 - DN^2} =
  \sqrt{\frac{h^2}{3} - \frac{a^2}{4}}$

  $BD = CN = OC - CN \Rightarrow \frac{5a}{3} = \sqrt{3h^2 - \frac{a^2}{4}} - \sqrt{\frac{h^2}{3} -
    \frac{a^2}{4}}$

  On simplification, we get $h = \sqrt{\frac{85}{48}}a$ or $h = \sqrt{\frac{5}{6}}a$.

\item The diagram \in{Figure}[fig:28_187] is given below:

  \startplacefigure[reference=fig:28_187]
    \externalfigure[28_187.pdf]
  \stopplacefigure

  Let $OP$ be the tower having a height of $h$. Given $ABC$ is an equilateral
  triangle. Let the angle subtended by $OP$ at $A, B, C$ be $\alpha, \beta, \gamma$
  respectively. According to question $\tan\alpha = \sqrt{3} + 1, \tan\beta = \sqrt{2}$ and
  $\tan\gamma = \sqrt{2}$. $OP$ is perpedicular to the plane of $\triangle ABC$.

  In $\triangle AOP, \tan\alpha = \frac{h}{OA}\Rightarrow OA = \frac{h}{\sqrt{3} + 1}$.

  Similarly, $OB = \frac{h}{\sqrt{2}}$ and $OC = \frac{h}{\sqrt{2}}$.

  In $\triangle AOB$ and $AOC, AB = AC, OB = OC$ and $OA$ is common. So
  $\triangle AOB$ and $\triangle AOC$ are equal. $\therefore \angle OAB = \angle OAC$.

  But $\angle BAC = 60^\circ \therefore \angle OAB = \angle OAC = 30^\circ$

  Let $\angle OBA = \theta$

  Using sine rule in $\triangle OAB, \frac{OB}{\sin30^\circ} = \frac{OA}{\sin\theta}$

  $\Rightarrow \sin\theta = \frac{\sqrt{3} - 1}{2\sqrt{2}} = \sin15^\circ$

  $\Rightarrow \theta = 15^\circ. \Rightarrow \angle OBD = \angle ABC - \theta = 45^\circ$

  In $\triangle BOC, OB = OC, OD\perp BC \therefore BD = DC = 40'$

  In $\triangle BOD, \cos45^\circ = \frac{BD}{OB} = \frac{40}{h/\sqrt{2}}\Rightarrow h = 80'$

\item The diagram \in{Figure}[fig:28_188] is given below:

  \startplacefigure[reference=fig:28_188]
    \externalfigure[28_188.pdf]
  \stopplacefigure

  Let $OP$ be the tower having a height of $h$ and $PQ$ be the flag-staff having a
  height of $x$. Since $PQ$ subtends equal angle $\alpha$ at $A$ and $B$ so
  a circle will pass through $A, B, P$ and $Q$. Since $C$ is the mid-point of $AB
  \therefore AC = BC = a$.

  Let $OA = d$ and $\angle PAO = \theta$. In $\triangle AOP, \tan\theta = \frac{h}{d}$

  In $\triangle AOQ, \tan(\theta + \alpha) = \frac{h + x}{d} \Rightarrow \frac{\tan\theta +
    \tan\alpha}{1 - \tan\theta\tan\alpha} = \frac{\frac{h}{d} + \tan\alpha}{1 - \frac{h}{d}\tan\alpha}$

  $\Rightarrow \frac{h + d\tan\alpha}{d - h\tan\alpha} = \frac{h + x}{d}$

  $\Rightarrow d^2 + h(x + h) = xd\cot\alpha$

  Similarly, $(d + a)^2 + h(x + h) = x(d + a)\cot\beta$

  As the points $A, B, P$ and $Q$ are concyclic $\therefore OA.OB = OP.OQ$

  $d(d + 2a) = h(h + x)$

  $\Rightarrow d^2 + d(d + 2a) = xd\cot\alpha \Rightarrow d + a = \frac{x}{2}\cot\alpha$

  Similarly, $(d + a)^2 + (d + a)^2 - a^2 = x(d + a)\cot\beta$

  Solving the above two equations

  $\frac{x^2}{4}\cot^2\alpha + \frac{x^2}{4}\cot^2\alpha - a^2 = x.\frac{x}{2}\cot\alpha\cot\beta$

  $\Rightarrow \frac{x^2}{2}(\cot^2\alpha - \cot\alpha\cot\beta) = a^2$

  $x = a\sin\alpha\sqrt{\frac{2\sin\beta}{\cos\alpha\sin(\beta - \alpha)}}$

\item The diagram \in{Figure}[fig:28_189] is given below:

  \startplacefigure[reference=fig:28_189]
    \externalfigure[28_189.pdf]
  \stopplacefigure

  Let $A_1, A_2, \ldots, A_{10}, \ldots, A_{17}$ be the feet of the first, second, ..., tenth, and
  seventeenth pillars respectively and $h$ be the height of each of these pillars. Given that these
  pillars are equidistant, therefore $A_1A_2 = A_2A_3 = \cdots = A_{16}A_{17} = x$ (let).

  Clearly, $A_1A_{10} = 9x$ and $A_1A_{17} = 16x$. We have let $O$ as the position of
  the observer and $\angle A_2A_1O = \theta$.

  In $\triangle A_{10}OP, \tan\alpha = \frac{h}{OA_{10}}\Rightarrow OA_{10} = h\cot\alpha$

  Similarly, $OA_{17} = h\cot\beta$

  From question $OA_1 = \frac{h\cot\alpha}{2}$ and $OA_1 = \frac{h\cot\beta}{3}$

  $\Rightarrow 2OA1 = OA_{10}$ and $3OA_1 = OA_{17}$. Let $OA_1 = y$ then

  $OA_{10} = 2y$ and $OA_{17} = 3y$

  Using cosine rule in $\triangle OA_1A_{10}, \cos\theta = \frac{81x^2 + y^2 - 4y^2}{2.9x.y}$

  $\Rightarrow y^2 = 27x^2 - 6xy\cos\theta$

  Similarly in $\triangle OA_1A_{17}, y^2 = 32x^2 - 4xy\cos\theta$

  $\Rightarrow y^2 = 42x^2 \Rightarrow \frac{y}{x} = \sqrt{42}$

  $\Rightarrow \sec\theta = -\frac{2\sqrt{42}}{5}$

  Acute angle will be given by $\sec\theta = \left|-\frac{2\sqrt{42}}{5}\right| = 2.6$ (approximately).

\item The diagram \in{Figure}[fig:28_190] is given below:

  \startplacefigure[reference=fig:28_190]
    \externalfigure[28_190.pdf]
  \stopplacefigure

  Let $DP$ be the tower having a height of $h$ with foot at $D$ and $A, B, C$ be
  the three points on the ciircular lake. According to question $\angle PAD = \alpha, \angle PBD =
  \beta$ and $\angle PCD = \gamma$. Also, $\angle BAC = \theta$ and $\angle ACB =
  \theta$. We know that angles on the same segement of a circle are equal. $\therefore \angle ADB =
  \angle ACB = \theta$ and $\angle BDC = \angle BAC = \theta$.

  In $\triangle PDA, \tan\alpha = \frac{h}{AD} \Rightarrow AD = h\cot\alpha$

  Similarly, $BD = h\cot\beta$ and $CD = h\cot\gamma$

  In $\triangle ABC, \angle BAC = \angle ACB \Rightarrow AB = BC \Rightarrow AB^2 = bC^2$

  Using cosine rule in $\triangle ABD, \cos\theta = \frac{AD^2 + BD^2 - AB^2}{2.AD.BD}$

  $\Rightarrow AB^2 = AD^2 + BD^2 - 2.AD.BD.\cos\theta$

  Similarly in $\triangle BDC, BC^2 = BD^2 + CD^2 - 2.BD.CD.\cos\theta$

  $\Rightarrow AD^2 + BD^2 - 2.AD.BD.\cos\theta = BD^2 + CD^2 - 2.BD.CD.\cos\theta$

  $\Rightarrow 2.BD.\cos\theta[CD - AD] = CD^2 - AD^2$

  $\Rightarrow 2.BD.\cos\theta = CD + AD \Rightarrow 2\cos\theta\cot\beta = \cot\alpha +
  \cot\gamma$.

\item The diagram \in{Figure}[fig:28_191] is given below:

  \startplacefigure[reference=fig:28_191]
    \externalfigure[28_191.pdf]
  \stopplacefigure

  Let $DP$ be the pole of height $h$ and $R$ be the radius of the circular pond.
  According to question, $\angle PAD = \angle PBD = 30^\circ$ and $\angle PCD = 45^\circ$.

  Clearly, $\angle PDA = \angle PDB = \angle PDC = 90^\circ$

  Also arc $AB = 40$ m and arc $BC = 20$ m.

  Now $\frac{2\pi R}{40} = \frac{2\pi}{\angle AOB}\Rightarrow \angle AOB = \frac{40}{R}$

  Similarly, $\angle BOC = \frac{20}{R} \therefore \angle AOB = 2\angle BOC$

  $\Rightarrow \angle ADB = 2.\angle BDC$ [$\because$ angle subtended by a segment at the
    center is double the angle subtended at circumference.]

  Let $\angle BDC = \theta$, then $\angle ADB = 2\theta$.

  In $\triangle PDA, \tan30^\circ = \frac{h}{AD} \therefore AD = \sqrt{3}h$.

  Similarly, in $\triangle PDB, BD = \sqrt{3}h$ and in $\triangle PDC, CD = h$.

  Now $\because AD = BD \therefore \angle DAB = \angle DBA = 90^\circ - \theta$

  Also, $\angle BAC = \angle BDC = \theta$ and $\angle ACB = \angle ADB = 2\theta$.

  Now $\angle ABC = 180^\circ - 3\theta \therefore \angle DBC = \angle ABC - \angle ABD$

  $= (180^\circ - 3\theta) - (90^\circ - \theta) = 90^\circ - 2\theta$

  $\angle BCD = 90^\circ + \theta$

  Using sine rule in $\triangle BCD, \frac{BD}{sin\angle BCD} = \frac{CD}{sin\angle DBC}$

  $\Rightarrow \frac{\sqrt{3}h}{\sin(90^\circ + \theta)} = \frac{h}{\sin(90^\circ - 2\theta)}$

  $\Rightarrow \frac{\sqrt{3}}{\cos\theta} = \frac{1}{\cos2\theta}$

  $\Rightarrow \cos\theta = \frac{\sqrt{3}}{2}, -\frac{1}{\sqrt{3}}$ (rejected because
  $\theta \ngtr 90^\circ$)

  $\theta = 30^\circ \Rightarrow \angle ADC = 3\theta = 90^\circ$

  $\therefore AC$ will be the diameter. arc $ABC =$ semiperimeter $= 60$ m.

  $\pi R = 60 \Rightarrow R = 19.09$ m.

  In $\triangle ADC, AC^2 = AD^2 + CD^2 \Rightarrow 4R^2 = 3h^2 + h^2 \Rightarrow h = R = 19.09$ m.

\item The diagram \in{Figure}[fig:28_192] is given below:

  \startplacefigure[reference=fig:28_192]
    \externalfigure[28_192.pdf]
  \stopplacefigure

  Let the man start at $O$ on the straight sea shore $OAB, P$ and $Q$ be the
  buoys. According to question, $OA = a, OB = b, \angle POA = \alpha - \angle PAQ = \angle PBQ$.

  $\because\angle PAQ = \angle PBQ = \alpha \therefore$ a circle will pass through the points
  $A, B, P$ and $Q$.

  Let $\angle OAQ = \theta \angle QAB = \pi - \theta$

  Also, $\angle OQA = \pi - (\angle QOA + \angle OAQ) = \pi - (\alpha + \theta)$

  $\therefore \angle APQ = \pi - (\angle PAQ + \angle PQA) = \pi - [\alpha + \pi - (\alpha +
    \theta)] = \theta$

  Since $ABPQ$ is concyclic $\therefore \angle ABQ = \pi - \angle APQ = \pi - \theta = \angle
  QAB \Rightarrow QA = QA \therefore \triangle QAB$ is an isosceles triangle.

  Draw $OD\perp AB \therefore D$ is the mid-point of $AB$.

  $\Rightarrow AD = BD = \frac{b}{2}\Rightarrow OD = OA + AD = a + \frac{b}{2}$

  In $\triangle ODQ, \cos\alpha = \frac{OD}{OQ} \Rightarrow OQ = \left(a +
  \frac{b}{2}\right)\sec\alpha$

  From the properties of a circle, $OA.OB = OP.OQ$

  $\Rightarrow a.(a + b) = OP.\left(a + \frac{b}{2}\right)\sec\alpha$

  $\Rightarrow OP = \frac{2a(a + b)\cos\alpha}{2a + b}$

  $\Rightarrow PQ = OQ - OP = \left(a + \frac{b}{2}\right)\sec\alpha - \frac{2a(a + b)}{2a +
    b}\cos\alpha$.

\item The diagram \in{Figure}[fig:28_193] is given below:

  \startplacefigure[reference=fig:28_193]
    \externalfigure[28_193.pdf]
  \stopplacefigure

  Let $A_1OA_n$ be the railway curve in the shape of a quadrant, the telegraph posts be
  represented by $A_1, A_2, \ldots, A_n$ and the man be stationed at $C$. From question
  $CPQ$  is a straight line. Also, $A_1C = a$. Let $OA_1$ be the radius of the quadrant
  and $O$ its center. Clearly, $A_1OA_n = \frac{\pi}{2}$.

  As there are $n$ telegraph posts from $A_1$ to $A_n$ at equal distances,
  arc $A_1A_N$ is divided in $n - 1$ equal parts.

  $\therefore \angle A_1OA_2 = \angle A_2OA_3 = \cdots = A_{n - 1}OA_n = \frac{\pi}{2(n - 1)} =
  \theta$

  According to question, $\phi = \frac{\pi}{4(n - 1)} \Rightarrow \theta = 2\phi$.

  Let $P$ and $Q$ be the $p^{th}$ and $q^{th}$ posts as seen from $A_1$.

  $\therefore \angle A_1OP = p\theta = 2p\phi$ and $\angle A_1OQ = q\theta = 2q\phi$

  $\angle POQ = (q - p)\theta = 2(q - p)\phi$. Draw $OD\perp PQ$

  $\because OP = OQ =$ radius of the circular quadrant. $\therefore \triangle POQ$ is an
  isosceles triangle.

  Clearly, $OD$ bisects the $\angle POQ \therefore \angle POD = \angle QOD = (q - p)\phi$

  $\angle COD = \angle A_1OD = \angle A_1OP + \angle POD = 2p\phi + (q - p)\phi = (p + q)\phi$

  In $\triangle ODC, \cos\angle COD = \frac{OD}{OC} \Rightarrow \cos(p + q)\phi = \frac{OD}{r + a}$

  $\Rightarrow OD = (r + a)\cos(p + q)\phi$

  In $\triangle ODP, \cos\angle POD = \frac{OD}{OP} \Rightarrow \cos(p - q)\phi =
  \frac{OD}{r}\Rightarrow OD = r\cos(q - p)\phi$

  $\Rightarrow (r + a)\cos(q + p)\phi = r\cos(q - p)\phi$

  $\Rightarrow -a\cos(q + p)\phi = r[\cos(q + p)\phi - \cos(q - p)\phi]$

  $\Rightarrow r = \frac{a}{2}\cos(q + p)\phi.\csc p\phi.\csc q\phi$.

\item The diagram \in{Figure}[fig:28_194] is given below:

  \startplacefigure[reference=fig:28_194]
    \externalfigure[28_194.pdf]
  \stopplacefigure

  Let $r$ be the radius of the wheel and $x$ be the length of the rod. Clearly, $AC =
  2r + x$. According to question $\angle APC = \alpha$.

  In $\triangle PAC, \tan\alpha = \frac{AC}{AP} = \frac{2r + x}{d} \Rightarrow x = d\tan\alpha -
  2r$.

  After rotation of the wheel, let $C'$ be the new position of $C$ as shown in the figure. In
  this case angle of elevation of $C'$ is $\beta$. Since $C'$ is the position of
  $C$ when it is about to disappear, so $PC'$ will be tangent to the wheel. Let it touch the
  wheel at $Q$.

  In $\triangle OPQ$ and $APO, OQ = OA = r, OP$ is common.

  $\angle OQP = \angle OAP = 90^\circ \therefore$ triangle are equal.

  $\Rightarrow \angle OPQ = \angle OPA = \frac{\beta}{2}$

  In $\triangle OAP, \tan\frac{\beta}{2} = \frac{OA}{AP} = \frac{r}{d} \Rightarrow r =
  d\tan\frac{\beta}{2}$

  $\Rightarrow x = d\left(\tan\alpha - 2\tan\frac{\beta}{2}\right)$

  $PA$ and $PQ$ are tangents to the same circle $\therefore PQ = PA = d$

  $\therefore \angle OQC' = 90^\circ$

  In $\triangle OQC', QC' = \sqrt{OC'^2 - OQ^2} = \sqrt{(x + r)^2 - r^2} = \sqrt{x(x + 2r)}$

  $= d\sqrt{\tan^2\alpha - 2\tan\alpha\tan\frac{\beta}{2}}$

  $\therefore PC' = PQ + QC' = d + d\sqrt{\tan^2\alpha - 2\tan\alpha\tan\frac{\beta}{2}}$

\item The diagram \in{Figure}[fig:28_195] is given below:

  \startplacefigure[reference=fig:28_195]
    \externalfigure[28_195.pdf]
  \stopplacefigure

  Let $PQ$ be the tower having a height of $h, ADB$ be the arc having the given
  length of $2L$ and $AC$ be the part of arc with length $\frac{L}{2}$. Clearly, line
  $PC$ will be tangent to the arch as the man at $C$ just sees the topmost point $P$ of
  the tower. $D$ is the topmost point of the semi-circullar arch.

  Let $r$ be the radius of the arch. According to question $\angle PDT = \theta$ where
  $DT\perp PQ$.

  Let $\angle COA = \phi$. Here $O$ is the center of the arch. Clearly, $2L$ length
  represents semi-cicular arch which means $AC$ which is of length $\frac{L}{2}$ will make an
  angle of $45^\circ$ at center i.e. $\phi = 45^\circ$.

  In $\triangle ONC, CN = OC\sin\phi$ and $ON = OC\cos\phi$

  $\Rightarrow CN = \frac{r}{\sqrt{2}}$ and $ON = \frac{r}{\sqrt{2}}$

  Let $CR\perp PQ$ then $CD\parallel NO \therefore \angle OCM = \angle CON = 45^\circ$

  Also, $\angle OCP = 90^\circ$ beccause $OC$ is normal at $C$.

  $\therefore \angle PCR = \angle PCO - \angle OCR = 90^\circ - 45^\circ = 45^\circ$

  In $\triangle PRC, \tan45^\circ = \frac{PR}{CR} = \frac{PQ - QR}{CR} = \frac{h -
    \frac{r}{\sqrt{2}}}{CR}$

  $\Rightarrow CR = h - \frac{r}{\sqrt{2}}$

  In $\triangle DPT, \tan\theta = \frac{PT}{DT} = \frac{PQ - QT}{MR} = \frac{PQ - OD}{CR - CM}$

  $\Rightarrow \tan\theta = \frac{h - r}{h- \frac{r}{\sqrt{2}} - \frac{h}{\sqrt{2}}}$

  $\Rightarrow h = \frac{r(\sqrt{2}\tan\theta - 1)}{\tan\theta - 1} =
  \frac{2L}{\pi}.\frac{\sqrt{2}\tan\theta - 1}{\tan\theta - 1}$.

\item The diagram \in{Figure}[fig:28_196] is given below:

  \startplacefigure[reference=fig:28_196]
    \externalfigure[28_196.pdf]
  \stopplacefigure

  According to question, $\angle DAB = \alpha, \angle CAB = \beta$

  $\therefore \angle CAD = \beta - \alpha.\because AC$ is the diameter. $\therefore \angle
  ABC = 90^\circ$. Let $O$ be the ceter of the circle and $r$ be its radius then $AC =
  2r$.

  $\because E$ is the mid-point of $CD. \therefore CE = ED = x$ (let)

  $\because \angle ADC$ is the exterior angle of $\triangle ABC. \therefore \angle ADC =
  90^\circ + \alpha$

  Using sine rule in $\triangle ADC, \frac{2r}{\sin(90^\circ + \alpha)} = \frac{2x}{\sin(\beta -
    \alpha)} \Rightarrow x = \frac{r\sin(\beta - \alpha)}{\cos\alpha}$

  In $\triangle ABC, \cos\alpha = \frac{AB}{AD} = \frac{2r\cos\beta}{AD}\Rightarrow AD =
  \frac{2r\cos\beta}{\cos\alpha}$

  $\because AE$ is the median of the $\triangle CAD. \therefore AC^2 + AD^2 = 2(AE^2 + CE^2)$

  $\Rightarrow 4r^2 + \frac{4r^2\cos^2\beta}{\cos^2\alpha} = 2d^2 + 2x^2$

  $\Rightarrow \frac{4r^2(\cos^2\alpha + \cos^2\beta)}{\cos^2\alpha} = 2d^2 + \frac{r^2\sin^2(\beta
    - \alpha)}{\cos^2\alpha}$

  $\Rightarrow r^2 = \frac{d^2\cos^2\alpha}{2\cos^2\alpha + 2\cos^2\beta - \sin^2(\beta - \alpha)}$

  $\Rightarrow = \frac{d^2\cos^2\alpha}{\cos^2\alpha + \cos^2\beta + \cos(\alpha + \beta)\cos(\beta
    - \alpha) + \cos^2(\beta - \alpha)}$

  $= \frac{d^2\cos^2\alpha}{\cos^2\alpha + \cos^2\beta + 2\cos\alpha\cos\beta\cos(\beta - \alpha)}$

  Thus are of the triangle can be found which is equal to desired result.

\item The diagram \in{Figure}[fig:28_197] is given below:

  \startplacefigure[reference=fig:28_197]
    \externalfigure[28_197.pdf]
  \stopplacefigure

  Let $AB$ be the surface of the lake and $C$ be the point of observation such that $AC
  = h$ m. Let $E$ be the position of the cloud and $E'$ be its reflection then $BE =
  BE'$.

  In $\triangle CDE, \tan\alpha = \frac{DE}{CD} = \frac{H}{CD} \Rightarrow CD =
  \frac{H}{\tan\alpha}$

  In $\triangle CDE', \tan\beta = \frac{DE'}{CD} = \frac{2h + H}{CD} \Rightarrow CD = \frac{2h +
    H}{\tan\beta}$

  $\Rightarrow \frac{H}{\tan\alpha} = \frac{2h + H}{\tan\beta} \Rightarrow H(\tan\beta -
  \tan\alpha) = 2h\tan\alpha \Rightarrow H = \frac{2h\tan\alpha}{\tan\beta - \tan\alpha}$

  In $\triangle CDE, \sin\alpha = \frac{DE}{CE} \Rightarrow CE = \frac{H}{\sin\alpha}$

  $\Rightarrow CD = \frac{2h\tan\alpha}{(\tan\beta - \tan\alpha)\sin\alpha} =
  \frac{2h\sec\alpha}{\tan\beta - \tan\alpha}$

  $= \frac{2h\sec\alpha.\cos\alpha.\cos\beta}{\sin\beta\cos\alpha - \sin\alpha\cos\beta}$

  $= \frac{2h\cos\beta}{\sin(\beta - \alpha)}$.

\item Front view \in{Figure}[fig:28_198] and side view \in{Figure}[fig:28_198_1] are given below:

  \startplacefigure[reference=fig:28_198]
    \externalfigure[28_198.pdf]
  \stopplacefigure
  \startplacefigure[reference=fig:28_198_1]
    \externalfigure[28_198_1.pdf]
  \stopplacefigure

  In $\triangle ADE, \tan30^\circ = \frac{AD}{DE} \Rightarrow DE = \sqrt{3}h$

  In $\triangle BDE, \tan\alpha = \frac{BD}{DE} = \frac{a}{\sqrt{3}h}$

  Tangent of apex of shadow $= \tan2\alpha = \frac{2\tan\alpha}{1 - \tan^2\alpha}$

  $= \frac{\frac{2a}{\sqrt{3}h}}{1 - \frac{a^2}{3h^2}} = \frac{2ah\sqrt{3}}{3h^2 - a^2}$.

\item Let $ABCD$ be the target and $ABC'D'$ be its shadow then $\angle DAD' = \beta$ and
  $\angle BAD'= 90^\circ - \beta$. Area of the target $= AB.AD$ and area of the shadow
  $= AB.AD'.\cos\beta$

  $\frac{\text{Area of target}}{\text{Area of shadow}} = \frac{AB.AD}{AB.AD'.\cos\beta} =
  \tan\alpha.\sec\beta$

\item This question is similar to 169, and has been left as an exercise.
\item The diagram \in{Figure}[fig:28_201] is given below:

  \startplacefigure[reference=fig:28_201]
    \externalfigure[28_201.pdf]
  \stopplacefigure

  Let $BC$ be the tower having a height of $h$. According to question $AB = h$ and $BD = h/2$.

  In $\triangle BCD, \tan\alpha = \frac{h}{\frac{h}{2}} = 2$

  In $\triangle ABC, \tan\beta = \frac{h}{h} \Rightarrow \alpha = 45^\circ$

  $L\tan\alpha = 10 + \log 2 = 10.30103$ (given)

  $L\tan\alpha - L\tan63^\circ26'= 10.30103 - 10.30094$

  If the difference is $x^2$, then $x^2 = \frac{60\times306}{3152} = 5.81^2$

  $\therefore \alpha = 63^\circ26'6''$

  Change in sun's altitude $= 63^\circ26'6'' - 45^\circ = 18^\circ26'6''$

\item The diagram \in{Figure}[fig:28_206] is given below:

  \startplacefigure[reference=fig:28_206]
    \externalfigure[28_206.pdf]
  \stopplacefigure

  Let $ABCD$ be the vertical cross-section of the tower through the middle, let the side of the
  square is $AB$ having length $a$ and height of the tower be $OP$ equal to
  $h$. Let the height of flag-staff $PQ$ be $b$. $M$ and $N$ are points of
  observation such that $AN = MN = 100$ m. Let $\alpha$ and $\beta$ be angles of
  elevation from $M$ of $D$ and $Q$ such that $\tan\alpha = \frac{5}{9}$ and
  $\tan\beta = \frac{1}{2}$. At $N$ the man just sees the flag.

  $\tan\beta = \frac{1}{2} = \frac{AD}{AM} = \frac{AD}{100} \Rightarrow AD = 100 = OP = h$.

  $\therefore AD = AN = 100 \Rightarrow \angle AND = \angle NDA = 45^\circ \Rightarrow PD = PQ
  \Rightarrow \frac{a}{2} = b \Rightarrow a = 2b$

  $\tan\alpha = \frac{5}{9} = \frac{OQ}{OM} = \frac{b + h}{200 + \frac{a}{2}} = \frac{b + h}{200 +
    b}$

  $\Rightarrow b = 25 \Rightarrow a = 50$.

\item The diagram \in{Figure}[fig:28_207] is given below:

  \startplacefigure[reference=fig:28_207]
    \externalfigure[28_207.pdf]
  \stopplacefigure

  Let $OC$ be the vertical pole having a height of $h$. $A$ and $B$ are given
  points in the question from where anglea of elevations of $C$ are $\alpha$ and
  $\beta$ respectively. Angle subtended by $AB$ at $O$ is $\gamma$ as shown in
  the diagram. Let $OB = x$ and $OA = y$. Given that $AB = d$

  In $\triangle OAC, \tan\alpha = \frac{h}{y}\Rightarrow y = h\cot\alpha$

  Similarly $x = h\cot\beta$

  In $\triangle OAB, d^2 = x^2 + y^2 - 2xy\cos\gamma$

  $d^2 = h^2\cot^2\alpha + h^2\cot^2\beta - 2h^2\cot\alpha\cot\beta\cos\gamma$

  $\Rightarrow h = \frac{d}{\sqrt{\cot^2\alpha + \cot^2\beta - 2\cot\alpha\cot\beta\cos\gamma}}$

\item The diagram \in{Figure}[fig:28_208] is given below:

  \startplacefigure[reference=fig:28_208]
    \externalfigure[28_208.pdf]
  \stopplacefigure

  Let $OP$ be the tree having a height of $h$ and $OAB$ is the hill inclines at angle
  $\alpha$ with the horizontal. Let $A$ be the point from where angle of elevation of the top
  of the tree be $\beta$ and $B$ be the point from where the angle of depression of the top
  of the tee be $\gamma$. Given $AB = m$.

  $\angle POA = 90^\circ - \alpha, \angle OAP = \alpha + \beta, \angle = \alpha - \gamma$ and
  $\angle ABP = (\alpha + \beta) - (\alpha - \gamma) = \beta + \gamma$

  In $\triangle OAP, \frac{AP}{\sin(90^\circ - \alpha)} = \frac{OP}{\sin(\alpha + \beta)}$

  $\Rightarrow \frac{AP}{\cos\alpha} = \frac{h}{\sin(\alpha + \beta)}$

  In $\triangle PAB, \frac{AB}{\sin(\beta + \gamma) = \frac{AP}{\sin(\alpha - \gamma)}}$

  $\Rightarrow h = \frac{m\sin(\alpha + \beta)\sin(\alpha - \gamma)}{\cos\alpha\sin(\beta + \gamma)}$

\item The diagram is \in{Figure}[fig:28_209] given below:

  \startplacefigure[reference=fig:28_209]
    \externalfigure[28_209.pdf]
  \stopplacefigure

  Let $ABCDA'B'C'D'$ be the vertical tower having a height of $h$ i.e. $AA' = BB' = CC'
  = DD' = h$ with side equal to $b$ and $O$ be the point of observation on the diagonal
  $AC$ extended at a distance $2a$ from $A$. Clearly, $\angle OAB =
  135^\circ$. Given that $\angle AOA' = 45^\circ$ and $\angle BOB' = 30^\circ$.

  In $\triangle AOA', \tan45^\circ = 1 = \frac{AA'}{OA} = \frac{h}{2a}\Rightarrow h = 2a$

  In $\triangle BOB', \tan30^\circ = \frac{1}{\sqrt{3}} = \frac{BB'}{OB} \Rightarrow OB =
  2\sqrt{3}a$

  In $\triangle OAB, \cos135^\circ = -\frac{1}{\sqrt{2}} = \frac{4a^2 + b^2 - 12a^2}{2.2a.b}$

  This is a quadratic equation in b wiht two roots $a(-\sqrt{2}\pm\sqrt{10})$. Clearly, $b$
  cannot be negaitive so $b = a(\sqrt{10} - \sqrt{2})$.

\item The diagram is \in{Figure}[fig:28_210] given below:

  \startplacefigure[reference=fig:28_210]
    \externalfigure[28_210.pdf]
  \stopplacefigure

  Let $AS$ be the steeple having a height of $h, B$ is the point due south having an angle of
  elevation of $45^\circ$ to the top of the tower and $C$ is the point due south of
  $B$, at a distance of $a$ from $B$, having an angle of elevation of $15^\circ$
  to the top of the tower. $AS$ is perpendicular to the plane of paper.

  In $\triangle ABS, \tan45^\circ = 1 = \frac{AS}{AB}\Rightarrow AB = h$

  In $\triangle ACS, \tan15^\circ = 2 - \sqrt{3} = \frac{AS}{AC} \Rightarrow AC = h(2 + \sqrt{3})$

  In $\triangle ABC, AC^2 = AB^2 + BC^2 \Rightarrow h^2(2 + \sqrt{3})^2 = h^2 + a^2$

  $\Rightarrow a = \frac{h}{\sqrt{6 + 2\sqrt{3}}}$

\item The diagram \in{Figure}[fig:28_211] is given below:

  \startplacefigure[reference=fig:28_211]
    \externalfigure[28_211.pdf]
  \stopplacefigure

  Let $CD$ be the given tower with a given height of $c$, $FG$ be the mountain behind
  the spire and tower at a distance $x$ having a height of $h$ and $A$ and $B$
  are the points of observation. Let the angle of elevation from $A$ is $\alpha$ such that
  the mountain is just visible behind the tower. Let $DE$ be the spire which subtends equal angle
  of $\beta$ at $A$ and $B$. Since it subtends equal angles at $A$ and $B$
  the points $A, B, D$ and $E$ will be concyclic. $a$ and $b$ are shown as given
  in the question.

  $\angle AEC = 90^\circ - (\alpha + \beta)$

  Segment $AD$ will also subtend equal angles at $B$ and $E$ $\therefore \angle
  AED = \angle ABD = 90^\circ - (\alpha + \beta) \Rightarrow \angle CBE = 90^\circ - \alpha$

  In $\triangle ACD$ and $AFG, \tan\alpha = \frac{c}{a} = \frac{h}{x + a} \Rightarrow x =
  \frac{ah - ac}{c}$

  In $\triangle BFG, \tan(90^\circ - \alpha) = \frac{h}{x + a + b}$

  $\Rightarrow \frac{a}{c} = \frac{h}{\frac{ah - ac}{c} + a + b}$

  $\Rightarrow h = \frac{abc}{c^2 - a^2}$

\item The diagram The diagram \in{Figure}[fig:28_212] is given below:

  \startplacefigure[reference=fig:28_212]
    \externalfigure[28_212.pdf]
  \stopplacefigure

  Let $AB$ be the pole having height $h$ and $CD$ be the tower having a height of
  $h + x$ as shown in the diagram. The angles $\alpha$ and $\beta$ are shown as given
  in the question. Let $d$ be the distance between the pole and tower. Clearly, $\angle ADC =
  90^\circ - \alpha \Rightarrow \angle BDC = 90^\circ - (\alpha - \beta)$. Let $h + x = H$

  In $\triangle ACD, \tan\alpha = \frac{h + x}{d} \Rightarrow d = (h + x)\cot\alpha = H\cot\alpha$

  In $\triangle BDE, \tan(\alpha - \beta) = \frac{H - h}{d} \Rightarrow d = (H - h)\cot(\alpha -
  \beta)$

  $\Rightarrow H\cot\alpha = (H - h)\cot(\alpha - \beta)$

  $\Rightarrow H = \frac{h\cot(\alpha - \beta)}{\cot(\alpha- \beta) - \cot\alpha}$

\item The diagram \in{Figure}[fig:28_213] is given below:

  \startplacefigure[reference=fig:28_213]
    \externalfigure[28_213.pdf]
  \stopplacefigure

  Let $A, B, C$ and $D$ be the points on one bank such that $AB = 6d, AC = 2d, AD = BD
  = 3d$ and $PQ$ be the tower on the other bank perpendicular to the plane of the paper having a
  height of $h$. Given that $\angle PBQ = \angle PAQ = \alpha$ and $\angle PCQ =
  \beta$.

  In $\triangle PBQ, \tan\alpha = \frac{PQ}{PB} \Rightarrow PB = h\cot\alpha$

  Similarly, $PA = h\cot\alpha$ and $PC = h\cot\beta$. Since $PA = PA$ the
  $\triangle PAB$ is an isosceles triangle. As $D$ is the mid-point of $AB$ so
  $\triangle PBD, \triangle PCD$ and $\triangle PAD$ will be right angled triangles.

  In $\triangle PAD, PA^2 = PD^2 + AD^2$ and in $\triangle PCD, PC^2 = PD^2 + CD^2$

  Subtracting, we get $PA^2 - PC^2 = AD^2 - CD^2$

  $\Rightarrow h^2(\cot^2\alpha - \cot^2\beta) = 9d^2 - d^2 = 8d^2$

  $\Rightarrow h = \frac{2\sqrt{2}d}{\sqrt{\cot^2\alpha - \cot^2\beta}}$

  $PD$ represents the width of the canal. $\Rightarrow PD^2 = PA^2 - AD^2 = h^2\cot^2\alpha -
  9d^2$

  $\Rightarrow PD = d\sqrt{\frac{9\cot^2\beta - \cot^2\alpha}{\cot^2\alpha - \cot^2\beta}}$

\item The diagram \in{Figure}[fig:28_214] is given below:

  \startplacefigure[reference=fig:28_214]
    \externalfigure[28_214.pdf]
  \stopplacefigure

  Let $PQ$ be the tower having a height of $h$ and points $A, B$ are the two stations
  at a distance of $2$ km having angles of elevation of $60^\circ$ and $30^\circ$
  respectively. $C$ is the mid-point between $A$ and $B$ from where the angle of
  elevation is $45^\circ$.

  In $\triangle PBQ, \tan60^\circ = \frac{h}{PB}\Rightarrow PB = \frac{h}{\sqrt{3}}$

  Similarly, $PA = \sqrt{3}h$ and $PC = h$.

  Now since $C$ is the mid-point of $AB$ therefore $PC$ is the median of the triangle
  $PAB$.

  $\Rightarrow PA^2 + PB^2 = 2(PC^2 + AC^2)$

  $\Rightarrow \frac{h^2}{3} + 3h^2 = 2(h^2 + 1)$

  $\Rightarrow h = \frac{\sqrt{3}}{\sqrt{2}}$ km $= 500\sqrt{6}$ m.

\item The diagram \in{Figure}[fig:28_215] is given below:

  \startplacefigure[reference=fig:28_215]
    \externalfigure[28_215.pdf]
  \stopplacefigure

  Let $PQ$ be the flag-staff standing inside equilateral $\triangle ABC$ and since all sides
  subtend an angle of $60^\circ$ it is guaranteed that $P$ will be centroid of the
  $\triangle ABC$. Given that the height of the flag-staff is $10$ m. Also, according to
  question $\angle AQB = \angle BQC = \angle CQA = 60^\circ \therefore AQ = BQ$. Let each side of
  the triangle has length of $2a$ m.

  Thus, $\triangle AQB$ is an equilateral triangle. $\therefore AQ = BQ = AB = 2a = CQ$

  We know from geometry that $AP = \frac{2}{3}AD$. We also know that median of an equilateral
  triangle is perpendicular bisector. $\therefore \triangle ABD$ is a right-angle triangle where
  $D$ is the point where $AP$ would meet $BC$.

  $\Rightarrow \sin60^\circ = \frac{AD}{AB} \Rightarrow AD = 2a\sin60^\circ$

  $\Rightarrow AP = \frac{2a}{\sqrt{3}}$

  $\triangle APQ$ is also a right angle triangle.

  $\Rightarrow AQ^2 = AP^2 + PQ^2 \Rightarrow 4a^2 - \frac{4a^2}{3} = 10$

  $\Rightarrow a = 5\sqrt{\frac{3}{2}}$

  $\Rightarrow 2a = 5\sqrt{6}$ m.

\item The diagram \in{Figure}[fig:28_216] is given below:

  \startplacefigure[reference=fig:28_216]
    \externalfigure[28_216.pdf]
  \stopplacefigure

  Let $CD$ be the cliff having a height of $H, DE$ be the tower on the cliff having a height
  of $h$ and $A, B$ are two points on horizontal level where the tower subtends the equal
  angle $\beta$ at a distance of $a, b$ from the cliff's foot. Let $\alpha$ be the
  angle of elevation from $A$ of the cliff's top.

  Since $DE$ subtends equal angles at $A, B$ therefore a circle will pass through these four
  points and thus chord $AD$ will also subtend equal angles $\angle AEC$ and $\angle
  ABD$ equal to $90^\circ - (\alpha + \beta)$.

  In $\triangle ACD, \tan\alpha = \frac{H}{a}$ and $\tan(\alpha + \beta) = \frac{H + h}{a} =
  \frac{b}{H}$

  In $\triangle BCE, \tan(90^\circ - \alpha) = \cot\alpha = \frac{H + h}{b}$

  We have $\frac{H + h}{a} = \frac{b}{H} \Rightarrow ab - H^2 = Hh$

  We have $\tan(\alpha + \beta) = \frac{b}{H}$

  $\Rightarrow \frac{\tan\alpha + \tan\beta}{1 - \tan\alpha\tan\beta} = \frac{b}{H}$

  $\Rightarrow \frac{\frac{H}{a} + \tan\beta}{1 - \frac{H}{a}\tan\beta} = \frac{b}{H}$

  $\Rightarrow \left(H + \frac{bH}{a}\right)\tan\beta = \frac{ab - H^2}{a}$

  $\Rightarrow h = (a + b)\tan\beta$

\item The diagram \in{Figure}[fig:28_217] is given below:

  \startplacefigure[reference=fig:28_217]
    \externalfigure[28_217.pdf]
  \stopplacefigure

  Let $AB$ be the tower and $BC$ be the flag-staff having heights of $x$ and $y$
  respectively. According to question $BC$ makes an angle of $\alpha$ at $E$ which is
  $c$ distance from the tower. Let the angle of elevation from $E$ to the top of tower
  $B$ is $\beta$.

  In $\triangle ABE, \tan\beta = \frac{x}{c}$

  In $\triangle ACE, \tan(\alpha + \beta) = \frac{x + y}{c}$

  $\Rightarrow \frac{\tan\alpha + \tan\beta}{1 - \tan\alpha\tan\beta} = \frac{x + y}{c}$

  $\Rightarrow \frac{x + c\tan\alpha}{c - x\tan\alpha} = \frac{x + y}{c}$

  $\Rightarrow \tan\alpha = \frac{cy}{x^2 + c^2 + xy}$

  Given that $\alpha$ is the greatest angle made which means $\tan\alpha$ will be greatest. So
  equating the derivative w.r.t to $c$ to zero, we get

  $\frac{d}{dc}\left[\frac{cy}{x^2 + c^2 + xy}\right] = \frac{c[x(x + c) - c^2]}{[x^2 + c^2 +
      xy]^2} = 0$

  $\Rightarrow c^2 = x(x + y)$

  $\Rightarrow \tan\alpha = \frac{cy}{2c^2} = \frac{y}{2c} \Rightarrow y = 2c\tan\alpha$

  We had $x^2 + xy - c^2 = 0 \Rightarrow x^2 + 2cx\tan\alpha - c^2 = 0$

  Neglecting the negative root we have $\Rightarrow x = -c\tan\alpha + c\sec\alpha$

  $\Rightarrow = c\left(\frac{1 - \sin\alpha}{\cos\alpha}\right) = 2d\left(\frac{1 +
    \tan^2\frac{\alpha}{2} - 2\tan\frac{\alpha}{2}}{1 - \tan^2\frac{\alpha}{2}}\right)$

  $= c\left(\frac{1 - \tan\frac{\alpha}{2}}{1 + \tan\frac{\alpha}{2}}\right)$

  $= c\tan\left(\frac{\pi}{4} - \frac{\alpha}{2}\right)$

\item The diagram \in{Figure}[fig:28_218] is given below:

  \startplacefigure[reference=fig:28_218]
    \externalfigure[28_218.pdf]
  \stopplacefigure

  We know that $B$ is due north of $D$ at a distance of $2$ km and $D$ is due
  west of $C$ such that $\angle BCD = 25^\circ$ we can plot $B, C, D$ as shown in the
  diagram. It is given that $B$ lies on $AC$ such that $\angle BDA = 40^\circ$. From
  figure it is clear that $\angle ACD = \angle CAD = 25^\circ$ thus $\triangle ACD$ is an
  isoscelels triangle. Let $AD = CD = x$.

  In $\triangle BCD, \tan25^\circ = \frac{2}{x} \Rightarrow x = 2\cot25^\circ = 4.28$ km.

\item The diagram \in{Figure}[fig:28_219] is given below:

  \startplacefigure[reference=fig:28_219]
    \externalfigure[28_219.pdf]
  \stopplacefigure

  Let the train move along the line $PQ$. The train is at $O$ at some instant. $A$ is
  the observation point. Ten minutes earlier let the train position be $P$ and ten minutes
  afterwards let the train be at $Q$.

  According to question, $\angle OAP = \alpha_1, \angle OAQ = \alpha_2, \angle NOQ = \theta$. Hence
  $\angle POA = \theta$.

  Since the speed of the train is constant $\therefore OP = OQ$

  Applying $m:n$ rule in $\triangle PAQ, (1 + 1)\cot\theta = \cot\alpha_2 - \cot\alpha_1$

  $\Rightarrow \cot\theta = \frac{\cot\alpha_2 - \cot\alpha_1}{2}$

  $\Rightarrow \tan\theta = \frac{2\sin\alpha_1\sin\alpha_2}{\sin(\alpha_1 - \alpha_2)}$

\item The diagram \in{Figure}[fig:28_220] is given below:

  \startplacefigure[reference=fig:28_220]
    \externalfigure[28_220.pdf]
  \stopplacefigure

  Let $OP$ be the flag-staff and that the man walk along the horizontal circle. Clearly, the
  flag-staff will subtend the greatest and least angles when the man is at $A$ and $B$
  respectively. Let $C$ be the mid-point of the arc $ACB$. According to question,
  $\angle PAO = \alpha, \angle PBO = \beta, \angle PCO = \theta$. Clearly, $\angle POC =
  90^\circ$. Let $OP = h, \angle POD = \phi$.

  Also, $OA = OB = OC = r$ where $r$ is the radius of the circle.

  In $\triangle PDO, \sin\phi = \frac{PD}{OP} \Rightarrow PD = h\sin\phi$

  $\cos\phi = \frac{OD}{OP} \Rightarrow OD = h\cos\phi$

  $\therefore BD = r + h\cos\phi$ and $AD = r - h\cos\phi$

  In $\triangle POC, \tan\theta = \frac{h}{r} \Rightarrow h = r\tan\theta$

  In $\triangle PDA, \tan\alpha = \frac{PD}{AD} = \frac{h\sin\phi}{r - h\cos\phi} =
  \frac{r\tan\theta\sin\phi}{r - r\tan\theta\cos\phi}$

  $\Rightarrow \tan\alpha = \frac{\tan\theta\sin\phi}{1 - \tan\theta\cos\phi}$

  $\Rightarrow \tan\theta\sin\phi = \tan\alpha - \tan\alpha\tan\theta\cos\phi$

  In $\triangle PDB, \tan\beta = \frac{h\sin\phi}{r + h\cos\phi} = \frac{r\tan\theta\sin\phi}{r +
    r\tan\theta\cos\phi}$

  $\Rightarrow \tan\beta = \frac{\tan\theta\sin\phi}{1 + \tan\theta\cos\phi}$

  $\Rightarrow \tan\theta\sin\phi = \tan\beta + \tan\beta\tan\theta\cos\phi$

  $\Rightarrow \tan\alpha - \tan\alpha\tan\theta\cos\phi = \tan\beta + \tan\beta\tan\theta\cos\phi$

  $\Rightarrow \tan\alpha - \tan\beta = \tan\theta\cos\phi(\tan\alpha + \tan\beta)$

  Also, $1 - \tan\theta\cos\phi = \frac{\tan\theta\sin\phi}{\tan\alpha}$

  and $1 + \tan\theta\cos\phi = \frac{\tan\theta\sin\phi}{\tan\beta}$

  $\Rightarrow 2 = \tan\theta\sin\phi\left(\frac{1}{\tan\alpha} + \frac{1}{\tan\beta}\right)$

  $\Rightarrow 2\tan\alpha\tan\beta = \tan\theta\sin\phi(\tan\alpha + \tan\beta)$

  $\Rightarrow (\tan\alpha - \tan\beta)^2 + 4\tan^2\alpha\tan^2\beta = \tan^2\theta(\tan\alpha +
  \tan\beta)^2$

  $\Rightarrow \tan^2\theta\left[\frac{\sin\alpha}{\cos\alpha} +
    \frac{\sin\beta}{\cos\beta}\right]^2 = \left(\frac{\sin\alpha}{\cos\alpha} -
  \frac{\sin\beta}{\cos\beta}\right)^2 + \frac{4\sin^2\alpha\sin^2\beta}{\cos^2\alpha\cos^2\beta}$

  $\Rightarrow \tan^2\theta\frac{\sin^2(\alpha + \beta)}{\cos^2\alpha\cos^2\beta} =
  \frac{\sin^2(\alpha - \beta) + 4\sin^2\alpha\sin^2\beta}{\cos^2\alpha\cos^2\beta}$

  $\Rightarrow \tan\theta = \frac{\sqrt{\sin^2(\alpha - \beta) +
      4\sin^2\alpha\sin^2\beta}}{\sin(\alpha + \beta)}$

\item The diagram \in{Figure}[fig:28_221] is given below:

  \startplacefigure[reference=fig:28_221]
    \externalfigure[28_221.pdf]
  \stopplacefigure

  Let $O$ be the position of the observer. $PRQS$ is the horizontal circle in which the bird
  is flying. $P$ and $Q$ are the two extremes and $R$ is the mid point of the arc of
  the circle. $P'R'Q'S$ is the vertical projection of the ground. $C$ is the center of the
  circle $PRQS$.

  According to the question, $\angle POP'' = 60^\circ, \angle QOQ' = 30^\circ, \angle ROP'=
  \theta$.

  Also, let $PP' = QQ' = RR' = h, r$ be the radius of the horizontal circle and $OP' = z$.

  In $\triangle PP'O, \tan60^\circ = \sqrt{3} = \frac{PP'}{OP'} = \frac{h}{z} \Rightarrow h =
  \sqrt{3}z$

  In $\triangle QOQ', \tan30^\circ = \frac{1}{\sqrt{3}} = \frac{h}{z + 2r} \Rightarrow z + 2r =
  \sqrt{3}h$

  $\Rightarrow z + 2r = \sqrt{3}\sqrt{3}z \Rightarrow z = r$

  In $\triangle ROR', \tan\theta = \frac{h}{OR'} = \frac{h}{\sqrt{OC^2 + C'R'^2}} =
  \frac{h}{\sqrt{(z + r)^2 + r^2}} = \frac{\sqrt{3}r}{\sqrt{(r + r)^2 + r^2}} = \sqrt{\frac{3}{5}}$

  $\Rightarrow \tan^2\theta = \frac{3}{5}$

\item The diagram \in{Figure}[fig:28_222]  is given below:

  \startplacefigure[reference=fig:28_222]
    \externalfigure[28_222.pdf]
  \stopplacefigure

  Let $O$ be the position of the observer and $OPQ$ be the horizontal line through $O$
  meeting the hill at $P$ and the vertical through the center $C$ of the sphere at $Q$.

  Let $OA$ be the tangent to the sphere from $O$ touching it at $A$. According to
  question, $\angle AOQ = \beta, \angle QPC = 90^\circ - \alpha, \angle ACN = \beta$. Let $r$
  be the radius of the hill. Draw $AM\perp OQ$ and $AN\perp CR$.

  In $\triangle AMO, \tan\beta = \frac{AM}{OM} = \frac{QN}{OP + PM} = \frac{QN}{OP + PQ - MQ}$

  $= \frac{CN - CQ}{OP + PQ - MQ}$

  $\Rightarrow \frac{\sin\beta}{\cos\beta} = \frac{r\cos\beta - r\cos\alpha}{a + r\sin\alpha -
    r\sin\beta} [\because MQ = AN]$

  $\Rightarrow a\sin\beta + r\sin\beta(\sin\alpha - \sin\beta) = r\cos\beta(\cos\beta -
  \cos\alpha)$

  $\Rightarrow a\sin\beta = r[1 - \cos(\alpha - \beta)]$

  $\Rightarrow r = \frac{a\sin\beta}{2\sin^2\frac{\alpha - \beta}{2}}$

  Height of the hill above the plane $= OR = CR _ CQ = r - r\cos\alpha = 2r\sin^2\frac{\alpha}{2}$

  $= \frac{a\sin\beta\sin^2\frac{\alpha}{2}}{\sin^2\frac{\alpha - \beta}{2}}$

\item The diagram \in{Figure}[fig:28_223] is given below:

  \startplacefigure[reference=fig:28_223]
    \externalfigure[28_223.pdf]
  \stopplacefigure

  Let $O$ be the center of the sphere and $r$ be its radius. Given, $\angle PAM =
  \theta, \angle PBM = \phi, CA = a, CB = b$

  Let $\angle DOC = \beta$

  In $\triangle PMA, \tan\theta = \frac{PM}{AM} = \frac{DN}{AC + CM} = \frac{ON _ OD}{AC + DC -
    DM}$

  $\Rightarrow \frac{\sin\theta}{\cos\theta} = \frac{r\cos\theta - r\cos\beta}{a + r\sin\beta -
    r\sin\theta} [\because DM = NP]$

  Proceeding like previous problem $a\sin\beta = r[1 - \cos(\theta - \beta)]$

  $\Rightarrow 2r\sin^2\frac{\theta - \beta}{2} = 2a\sin\frac{\theta}{2}\cos\frac{\theta}{2}$

  $\Rightarrow \sqrt{r}\sin\frac{\theta - \beta}{2} =
  \sqrt{a\sin\frac{\theta}{2}\cos\frac{\theta}{2}}$

  $\Rightarrow \sqrt{r}\frac{\left[\sin\frac{\theta}{2}\cos\frac{\beta}{2} -
      \cos\frac{\theta}{2}\sin\frac{\beta}{2}\right]}{\sin\frac{\theta}{2}} =
  \frac{\sqrt{a\sin\frac{\theta}{2}\cos\frac{\theta}{2}}}{\sin\frac{\theta}{2}}$

  $\Rightarrow \sqrt{r}\left[\cos\frac{\beta}{2} - \cot\frac{\theta}{2}\sin\frac{\beta}{2}\right] =
  \sqrt{a\cot\frac{\theta}{2}}$

  Similarly, $\sqrt{r}\left[\cos\frac{\beta}{2} - \cot\frac{\phi}{2}\sin\frac{\beta}{2}\right] =
  \sqrt{b\cos\frac{\phi}{2}}$

  Subtracting, we get $\sqrt{r}\sin\frac{\beta}{2}\left[\cot\frac{\theta}{2} -
    \cot\frac{\phi}{2}\right] = \sqrt{b\cos\frac{\phi}{2}} - \sqrt{a\cot\frac{\theta}{2}}$

  Height of the hill $DR = OR - OD = r - r\cos\beta = 2r\sin^2\frac{\beta}{2}$

  $= 2\left[\frac{\sqrt{b\cos\frac{\phi}{2}} - \sqrt{a\cot\frac{\theta}{2}}}{\cot\frac{\theta}{2} -
      \cot\frac{\phi}{2}}\right]^2$

\item The diagram \in{Figure}[fig:28_224] is given below:

  \startplacefigure[reference=fig:28_224]
    \externalfigure[28_224.pdf]
  \stopplacefigure

  Let $O$ be the center of the hemisphere and $PQ$ is the flag-staff. Given, $OP = OR =
  r, AB = d$.

  In $\triangle ORB, \cos45^\circ = \frac{r}{OB} \Rightarrow OB = \sqrt{2}r$

  In $\triangle QOA, \tan30^\circ = \frac{1}{\sqrt{3}} = \frac{OQ}{OA} = \frac{h + r}{\sqrt{2}r + d}$

  $\Rightarrow h + r = \frac{\sqrt{2}r + d}{\sqrt{3}}$

  In $\triangle QOB, \tan45^\circ = 1 = \frac{OQ}{OB} \Rightarrow h + r = \sqrt{2}r$

  $\Rightarrow \sqrt{2}r = \frac{\sqrt{2}r + d}{\sqrt{3}}$

  $\Rightarrow (\sqrt{6} - \sqrt{2})r = d \Rightarrow r = \frac{\sqrt{3} + 1}{2\sqrt{2}}d$

  $\Rightarrow h = (\sqrt{2} - 1)r = \frac{(\sqrt{2} - 1)(\sqrt{3} + 1)}{2\sqrt{2}}d$

\item The diagram \in{Figure}[fig:28_225] is given below:

  \startplacefigure[reference=fig:28_225]
    \externalfigure[28_225.pdf]
  \stopplacefigure

  Let the direction in which man starts walking be the $x$-axis. From question $OA = AB = BC
  = a$.

  Let coordinate of last point be $(X, Y)$ then $X = a + a\cos\alpha + a\cos2\alpha + \cdots$
  up to $n$ terms

  $= a.\frac{\cos(n - 1)\frac{\alpha}{2}\sin\frac{n\alpha}{2}}{\sin\frac{\alpha}{2}}$

  $Y = 0 + a\sin\alpha + a\sin2\alpha + \cdots$ up to $n$ terms

  $= a\left[\frac{\sin(n - 1)\frac{\alpha}{2}\sin\frac{n\alpha}{2}}{\sin\frac{\alpha}{2}}\right]$

  Distance from the starting point $= \sqrt{X^2 + Y^2} =
  \frac{a\sin\frac{n\alpha}{2}}{\sin\frac{\alpha}{2}}$

  Let $\theta$ be the angle which this distance makes with $x$-axis. Then

  $\tan\theta = \frac{Y}{X} = \tan(n - 1)\frac{\alpha}{2} \Rightarrow \theta = (n -
  1)\frac{\alpha}{2}$

\item The diagram \in{Figure}[fig:28_226] is given below:

  \startplacefigure[reference=fig:28_226]
    \externalfigure[28_226.pdf]
  \stopplacefigure

  Let $ABC$ be the horizontal triangle. $A'B'$ represents the stratum of coal. Suppose this
  startus meets the horizontal plane in line $DD'$. Let $\theta$ be the angle between the
  horizontal and the stratum of the coal.

  Clearly, $\angle ADA' = \theta$. According to question, $AA' = x, BB' = x + y$ and
  $CC' = x + z$.

  In $\triangle AA'D, \tan\theta = \frac{AA'}{AD} \Rightarrow x = AD\tan\theta$

  In $\triangle BB'D, \tan\theta = \frac{BB'}{BD} \Rightarrow x + y = (AD + AB)\tan\theta$

  $\Rightarrow x + z = (AD + c)\tan\theta$

  In $\triangle CC'D, \tan\theta = \frac{CC'}{CD'} \Rightarrow x + z = (AD + b\cos A)\tan\theta$

  $\Rightarrow y = c\tan\theta \Rightarrow \frac{y}{c} = \tan\theta$ and $z = b\cos
  A\tan\theta \Rightarrow \frac{z}{b} = \cos A\tan\theta$

  Now, $\frac{y^2}{c^2} + \frac{z^2}{b^2} - \frac{2yz}{bc}\cos A = \frac{y^2}{c_2}\sin^2A +
  \frac{y^2}{c^2}\cos^2A + \frac{z^2}{b^2} - \frac{2yz}{bc}\cos A$

  $= \frac{y^2}{c^2}\sin^2A + \left(\frac{y}{c}\cos A - \frac{z}{b}\right)^2$

  $= \tan^2\theta\sin^2A + (\tan\theta\cos A - \cos A\tan\theta)^2 = \tan^2\theta\sin^2A$

  $\Rightarrow \tan\theta\sin A = \sqrt{\frac{y^2}{c^2} + \frac{z^2}{b^2} - \frac{2yz}{bc}\cos A}$

\stopitemize
