% -*- mode: context; -*-
\chapter{Summation of Trigonometric Series}
\section{Problems}
\startitemize[n, 1*broad]
\item Multiplying both sides by $2\sin\frac{\beta}{2}$ and writing termwise, we get
  \startformula\startalign
    \NC 2\sin\alpha\sin\frac{\beta}{2} \NC = \cos\left(\alpha - \frac{\beta}{2}\right) - \cos\left(\alpha +
    \frac{\beta}{2}\right)\NR
    \NC 2\sin(\alpha + \beta)\sin\frac{\beta}{2} \NC = \cos\left(\alpha + \frac{\beta}{2}\right) -
    \cos\left(\alpha + \frac{3\beta}{2}\right)\NR
    \NC 2\sin(\alpha + 2\beta)\sin\frac{\beta}{2} \NC = \cos\left(\alpha + \frac{3\beta}{2}\right) -
    \cos\left(\alpha + \frac{5\beta}{2}\right)\NR
    \NC \NC \cdots\NR
    \NC 2\sin(\alpha + (n - 1)\beta)\sin\frac{\beta}{2} \NC = \cos\left(\alpha + (2n - 3)\frac{\beta}{2}\right)
    - \cos\left(\alpha + (2n - 1)\frac{\beta}{2}\right)\NR
  \stopalign\stopformula

  Adding we get $2\left[\sin\frac{\beta}{2}\sin\alpha + \sin(\alpha + \beta) + \sin(\alpha + 2\beta) + \cdots +
  \sin(\alpha + (n - 1)\beta)\right] = \cos\left(\alpha - \frac{\beta}{2}\right) - \cos\left[\alpha + (2n -
    1)\frac{\beta}{2}\right]$

  $\Rightarrow \sin\alpha + \sin(\alpha + \beta) + \sin(\alpha + 2\beta) + \cdots + \sin(\alpha + (n -
  1)\beta) = \frac{\sin\frac{n\beta}{2}}{\sin\frac{\beta}{2}}\sin\left[\alpha + (n -
    1)\frac{\beta}{2}\right]$
\item Let $S = cos\alpha + \cos(\alpha + \beta) + \cos(\alpha + 2\beta) + \cdots + \cos(\alpha + (n -
  1)\beta)$

  Multiplyinf both sides by $2\sin\frac{\beta}{2}$ like previous problems and writing termwise

  \startformula\startalign
    \NC 2\cos\alpha\sin\frac{\beta}{2} \NC = \sin\left(\alpha + \frac{\beta}{2}\right) - \sin\left(\alpha -
    \frac{\beta}{2}\right)\NR
    \NC 2\cos(\alpha + \beta)\sin\frac{\beta}{2} \NC = \sin\left(\alpha + \frac{3\beta}{2}\right) -
    \sin\left(\alpha + \frac{\beta}{2}\right)\NR
    \NC 2\cos(\alpha + 2\beta)\sin\frac{\beta}{2} \NC = \sin\left(\alpha + \frac{5\beta}{2}\right) -
    \sin\left(\alpha + \frac{3\beta}{2}\right)\NR
    \NC\NC \cdots\NR
    \NC 2\cos(\alpha+ (n - 1)\beta)\sin\frac{\beta}{2} \NC = \sin\left(\alpha + (2n - 1)\frac{\beta}{2}\right)
    - \sin\left(\alpha + (2n - 3)\frac{\beta}{2}\right)\NR
  \stopalign\stopformula

  Adding we get $2S\sin\frac{\beta}{2} = \sin\left(\alpha + (2n - 1)\frac{\beta}{2}\right) -
  \sin\left(\alpha - \frac{\beta}{2}\right)$

  $\Rightarrow S = \frac{\sin\frac{n\beta}{2}}{\sin\frac{\beta}{2}}\cos\left[\alpha + (n -
    1)\frac{\beta}{2}\right]$.
\item Let $S = \sin\theta + \sin3\theta + \sin5\theta + \cdots + \sin(2n - 1)\theta$

  Multiplying both sides by $2\sin\theta$ and writing in termwise fashion

  \startformula\startalign
    \NC 2\sin\theta\sin\theta \NC = 1 - \cos2\theta\NR
    \NC 2\sin\theta\sin3\theta \NC = \cos2\theta - \cos4\theta\NR
    \NC 2\sin\theta\sin5\theta \NC = \cos4\theta - \cos6\theta\NR
    \NC \NC \cdots\NR
    \NC 2\sin\theta\sin(2n - 1)\theta \NC = \cos(2n - 2)\theta - \cos2n\theta\NR
  \stopalign\stopformula

  Adding gives us $2S\sin\theta = 1 - \cos2n\theta = 2\sin^2n\theta$

  $\Rightarrow S = \frac{\sin^2n\theta}{\sin\theta}$.
\item The terms are of the form $\sin^2x$ which we can write as $\frac{1 - \cos2x}{2}$. Let the sum be $S$.

  Then $S = \frac{n}{2} - \frac{1}{2}\left[\cos2\alpha + \cos2(\alpha + \beta) + \cos2(\alpha + 2\beta) +
    \cdots + \cos2\left\{\alpha + (2n - 1)\beta\right\}\right]$

  Now we invoke the sum we obtained in second problem to arrive at

  $S = \frac{n}{2} - \frac{1}{2}\frac{\sin n\beta}{\sin\beta}\cos[2\alpha + (n - 1)\beta]$.
\item Since $B$ is the exterior angle of a regular polygon of $n$-sides. Thus, $B = \frac{2\pi}{n}$.

  L.H.S.\ $= \cos A + \cos\left(A + \frac{2\pi}{n}\right) + \cos\left(A + \frac{4\pi}{n}\right) + \cdots$ to
  $n$ terms

  $= \frac{\sin\pi}{\sin\frac{\pi}{n}}\cos\left[A + (n - 1)\frac{\pi}{n}\right]$.
\item $\because \sin(\pi + \theta) = -\sin\theta$ and $\sin(2\pi + \theta) = \sin\theta$

  $\Rightarrow -\sin(\alpha + \beta) = \sin(\pi + \alpha + \beta), \sin(\alpha + 2\beta) = \sin(2\pi +
  \alpha + 2\beta)$

  and $-\sin(\alpha + 3\beta) = \sin(3\pi + \alpha + 3\beta)$

  So we apply the formula obtained in first problem to get sum as $=
  \frac{\sin\frac{n\beta}{2}}{\sin\frac{\beta}{2}}\sin[\alpha + (n - 1)\beta]$.
\item Let $S = \sin\theta\sin2\theta + \sin2\theta\sin3\theta + \sin3\theta\sin4\theta + \cdots$

  $= \frac{1}{2}[(\cos\theta - \cos3\theta) + (\cos\theta - \cos5\theta) + (\cos\theta - \cos7\theta) +
  \cdots]$

  $= \frac{1}{2}[n\cos\theta - (\cos3\theta + \cos5\theta + \cos7\theta + \cdots)]$

  $= \frac{n}{2}\cos\theta - \frac{1}{2}\frac{\sin n\theta}{\sin\theta}\cos(n + 2)\theta$.
\item We have proven that $\cos\alpha + \cos(\alpha + \beta) + \cos(\alpha + 2\beta) + \cdots + \cos(\alpha
  + (n - 1)\beta) = \frac{\sin\frac{n\beta}{2}}{\sin\frac{\beta}{2}}\cos\left[\alpha + (n -
    1)\frac{\beta}{2}\right]$

  Comparing we have $\alpha = \frac{\pi}{11}, \beta = \frac{2\pi}{11}, n = 5$

  Thus, sum is $\frac{\sin\frac{5\pi}{11}}{\sin\frac{\pi}{11}}\cos\left[\frac{\pi}{11} +
    \frac{4\pi}{11}\right]$

  $= \frac{\sin\frac{5\pi}{11}}{\sin\frac{\pi}{11}}\cos\frac{5\pi}{11} =
  \frac{1}{2}\frac{\sin\frac{10\pi}{11}}{\sin\frac{\pi}{11}} = \frac{1}{2}\frac{\sin\left(\pi -
      \frac{\pi}{11}\right)}{\sin\frac{\pi}{11}} = \frac{1}{2}$.
\item Let $t_n$ denote the $n$th term. Then $t_1 = \frac{\sin x}{\sin2x\sin3x} = \frac{\sin(3x -
    2x)}{\sin2x\sin3x} = \cot2x - \cot3x$

  \startformula\startalign
    \NC t_1 \NC = \cot2x - \cot3x\NR
    \NC t_2 \NC = \cot3x - \cot4x\NR
    \NC t_3 \NC = \cot4x - \cot5x\NR
    \NC \NC \cdots\NR
    \NC t_n \NC = \cot(n + 1)x - \cot(n + 2)x\NR
  \stopalign\stopformula

  Adding and represeenting the sum as $S$, we have $S = \cot2x - \cot(n + 2)x$.
\item Let $t_n$ denote the $n$th term. Then $t_1 = \csc\theta\csc2\theta =
  \frac{1}{\sin\theta}.\frac{\sin\theta}{\sin\theta.\sin2\theta}$

  $= \frac{1}{\sin\theta}.\frac{\sin(2\theta - \theta)}{\sin\theta\sin2\theta} =
  \frac{1}{\sin\theta}[\cot\theta - \cot2\theta]$

  Now we proceed like previous problems to obtains the sum as $S = \frac{1}{\sin\theta}[\cot\theta - \cot(n
  + 1)\theta]$.
\item Let $t_n$ denote the $n$th term. Then $t_1 = \frac{1}{\cos\theta + \cos3\theta} =
  \frac{1}{2\cos2\theta\cos\theta}$

  $= \frac{1}{2}\frac{\sin\theta}{\cos2\theta\cos\theta\sin\theta} =
  \frac{1}{2\sin\theta}.\frac{\sin(2\theta - \theta)}{\cos2\theta\cos\theta} =
  \frac{1}{2\sin\theta}[\tan2\theta - \tan\theta]$

  \startformula\startalign
    \NC t_1\NC = \frac{1}{\sin\theta}[\tan2\theta - \tan\theta]\NR
    \NC t_2\NC = \frac{1}{\sin\theta}[\tan3\theta - \tan2\theta]\NR
    \NC t_3\NC = \frac{1}{\sin\theta}[\tan4\theta - \tan3\theta]\NR
    \NC \NC \cdots\NR
    \NC t_n\NC = \frac{1}{\sin\theta}[\tan(n + 1)\theta - \tan n\theta]\NR
  \stopalign\stopformula

  Adding and representing the sum as $S$, we have $S = \frac{1}{\sin\theta}[\tan(n + 1)\theta -
  \tan\theta]$.
\item Let $t_n$ denote the $n$th term. Then $t_1 = \frac{\sin x}{\cos x + \cos2x}$

  $= \frac{1}{2\sin\frac{x}{2}}.\frac{2\sin\frac{x}{2}\sin x}{\cos x + \cos 2x} =
  \frac{1}{2\sin\frac{x}{2}}.\frac{\cos\frac{x}{2} - \cos\frac{3x}{2}}{2\cos\frac{3x}{2}\cos\frac{x}{2}}$

  $= \frac{1}{4\sin\frac{x}{2}}\left[\sec\frac{3x}{2} - \sec\frac{x}{2}\right]$

  \startformula\startalign
    \NC t_1\NC = \frac{1}{4\sin\frac{x}{2}}\left[\sec\frac{3x}{2} - \sec\frac{x}{2}\right]\NR
    \NC t_2\NC = \frac{1}{4\sin\frac{x}{2}}\left[\sec\frac{5x}{2} - \sec\frac{3x}{2}\right]\NR
    \NC t_3\NC = \frac{1}{4\sin\frac{x}{2}}\left[\sec\frac{7x}{2} - \sec\frac{5x}{2}\right]\NR
    \NC \NC \cdots\NR
    \NC t_n\NC = \frac{1}{4\sin\frac{x}{2}}\left[\sec(2n + 1)\frac{x}{2} - \sec(2n - 1)\frac{x}{2}\right]\NR
  \stopalign\stopformula

  Adding and representing the sum as $S$, we have $S = \frac{\csc\frac{x}{2}}{4}\left[\sec(2n +
    1)\frac{x}{2} - \sec\frac{x}{2}\right]$.
\item We know that $\sin3\theta = 3\sin\theta - 4\sin^3\theta \Rightarrow 4\sin^3\theta = 3\sin\theta -
  \sin3\theta$

  \startformula\startalign
    \NC \sin^3\theta \NC = \frac{1}{4}[3\sin\theta - \sin3\theta]\NR
    \NC \frac{1}{3}\sin^33\theta \NC = \frac{1}{4}\left[\sin3\theta - \frac{1}{3}\sin9\theta\right]\NR
    \NC \frac{1}{3^2}\sin^39\theta \NC = \frac{1}{4}\left[\frac{1}{3}\sin9\theta - \frac{1}{3^2}\sin27\theta\right]\NR
    \NC \NC \cdots\NR
    \NC \frac{1}{3^{n - 1}}\sin\left(3^{n - 1}\theta\right) \NC = \frac{1}{4}\left[\frac{1}{3^{n -
          1}}\sin\left(3^{n - 1}\theta\right) - \frac{1}{3^{n - 1}}\sin3^n\theta\right]\NR
  \stopalign\stopformula

  Adding and representing the sum as $S$, we have $S = \frac{1}{4}\left[3\sin\theta - \frac{1}{3^{n -
        1}}\sin3^n\theta\right]$.
\item $\csc\theta = \frac{1}{\sin\theta} = \frac{1}{\sin\theta}.\frac{\sin\left(\theta -
  \frac{\theta}{2}\right)}{\sin\frac{\theta}{2}}$

  $= \frac{\sin\theta\cos\frac{\theta}{2} - \cos\theta\sin\frac{\theta}{2}}{\sin\theta\sin\frac{\theta}{2}}
  = \cot\frac{\theta}{2} - \cot\theta$

  \startformula\startalign
    \NC \csc\theta \NC = \cot\frac{\theta}{2} - \cot\theta\NR
    \NC \csc2\theta \NC = \cot\theta - \cot2\theta\NR
    \NC \csc2^2\theta \NC = \cot2\theta - \cot2^2\theta\NR
    \NC \NC \cdots\NR
    \NC \csc2^{n- 1}\theta \NC = \cot2^{n - 2}\theta - \cot2^{n - 1}\theta\NR
  \stopalign\stopformula

  Adding and representing the sum as $S$, we have $S = \cot\frac{\theta}{2} - \cot2^{n - 1}\theta$.
\item From previous problem we can say that

  \startformula\startalign
    \NC \csc\theta \NC = \cot\frac{\theta}{2} - \cot\theta\NR
    \NC \csc\frac{\theta}{2} \NC = \cot\frac{\theta}{2^2} - \cot\frac{\theta}{2}\NR
    \NC \csc\frac{\theta}{2^2} \NC = \cot\frac{\theta}{2^3} - \cot\frac{\theta}{2^2}\NR
    \NC \NC \cdots\NR
    \NC \csc\frac{\theta}{2^{n - 1}} \NC = \cot\frac{\theta}{2^n} - \cot\frac{\theta}{2^{n - 1}}\NR
  \stopalign\stopformula

  Adding and representing the sum as $S$, we have $S = \cot\frac{\theta}{2^n} - \cot\theta$.
\item $\tan(A - B) = \frac{\tan A - \tan B}{1 + \tan A\tan B}\Rightarrow \tan A\tan B = \cot(A - B)[\tan A -
  \tan B] - 1$. Thus, we can write

  \startformula\startalign
    \NC \tan x\tan2x\NC = \cot x[\tan2x - \tan x]- 1\NR
    \NC \tan2x\tan3x\NC = \cot x[\tan3x - \tan2x]- 1\NR
    \NC \tan3x\tan4x\NC = \cot x[\tan4x - \tan3x]- 1\NR
    \NC \NC \cdots\NR
    \NC \tan n\tan(n + 1)x\NC = \cot x[\tan(n + 1)x - \tan nx]- 1\NR
  \stopalign\stopformula

  Adding and representing the sum as $S$, we have $S = \cot x[\tan(n + 1)x - \tan nx] - n$.
\item $\tan x = \frac{\sin x}{\cos x} = \frac{\sin^2x}{\sin x\cos x} = \frac{\cos^2x - \left(\cos^2x -
  \sin^2x\right)}{\sin x\cos x}$

  $= \frac{\cos^2x}{\sin x\cos x} - \frac{2\cos2x}{\sin2x} = \cot x - 2\cot2x$

  Thus, we can write(we let $t_n$ represent the $n$th term)

  \startformula\startalign
    \NC t_1 \NC = \cot x - 2\cot2x\NR
    \NC t_2 \NC = 2\cot2x - 2^2\cot2^2x\NR
    \NC t_3 \NC = 2^2\cot2^2x - 2^3\cot2^3x\NR
    \NC \NC \cdots\NR
    \NC t_n \NC = 2^{n - 1}\cot2^{n - 1}x - 2^n\cot2^nx
  \stopalign\stopformula

  Adding and representing the sum as $S$, we have $S = \cot x - 2^n\cot2^nx$.
\item We can reuse the identity derives and write similarly

  \startformula\startalign
    \NC t_1 \NC = \cot\theta - 2\cot2\theta\NR
    \NC t_2 \NC = \frac{1}{2}\cot\frac{\theta}{2} - \cot\theta\NR
    \NC t_3 \NC = \frac{1}{2^2}\cot\frac{\theta}{262} - \frac{1}{2}\cot\frac{\theta}{2}\NR
    \NC \NC \cdots\NR
    \NC t_n \NC = \frac{1}{2^{n - 1}}\cot\frac{\theta}{2^{n - 1}} - \frac{1}{2^{n -
        2}}\cot\frac{\theta}{2^{n - 2}}\NR
  \stopalign\stopformula

  Adding and representing the sum as $S_n$, we have $S_n = \frac{1}{2^{n - 1}}\cot\frac{\theta}{2^{n - 1}} -
  2\cot2\theta$.

  $\displaystyle S_\infty = \lim_{n\to\infty}\left(\frac{1}{2^{n - 1}}\cot\frac{\theta}{2^{n - 1}} -
  2\cot2\theta\right)$

  $= \lim_{n\to\infty}\left[\frac{1}{\theta}.\frac{\theta}{2^{n - 1}}.\frac{1}{\tan\frac{\theta}{2^{n - 1}}}
    - 2\cot2\theta\right]$

  $= \frac{1}{\theta} - 2\cot2\theta$.
\item Let $S_n = \frac{1}{2}\sec x + \frac{1}{2^2}\sec x\sec2x + \frac{1}{2^3}\sec x\sec2x\sec2^2x +
  \cdots$ to $n$ terms

  $= \frac{1}{2\cos x} + \frac{1}{2^2\cos x\cos2x} + \frac{1}{2^3\cos x\cos2x\cos2^2x} + \cdots$ to $n$
  terms

  $= \frac{\sin x}{2\sin x\cos x} + \frac{\sin x}{2^2\sin x\cos x\cos2x} + \frac{\sin x}{2^3\sin x\cos
    x\cos2x\cos2^2x} + \cdots$ to $n$

  $= \sin x\left[\frac{1}{\sin 2x} + \frac{1}{\sin 4x} + \frac{1}{\sin 8x} + \cdots\right]$ to $n$ terms

  Now $\csc2x = \frac{1}{\sin2x} = \frac{\sin x}{\sin2x\sin x} = \frac{\sin(2x - x)}{\sin 2x\sin x} = \cot x
  - \cot 2x$

  Thus, we can write

  \startformula\startalign
    \NC \csc2x \NC = \cot x - \cot2x\NR
    \NC \csc4x \NC = \cot2x - \cot4x\NR
    \NC \csc8x \NC = \cot4x - \cot8x\NR
    \NC \NC \cdots\NR
    \NC csc2^nx \NC = \cot2^{n - 1}x - \cot2^nx\NR
  \stopalign\stopformula

  Adding we get $2S_n = \sin x[\cot x - \cot2^nx]$.
\item Let $t_n = \tan^{-1}\frac{nx}{1 + n(n + 1)x^2} = \tan^{-1}\frac{(n + 1)x - nx}{1 + [(n + 1)x]nx} =
  \tan^{-1}(n + 1)x - \tan^{-1}nx$

  Thus, we can write

  \startformula\startalign
    \NC t_1 \NC = \tan^{-1}2x - \tan^{-1}x\NR
    \NC t_2 \NC = \tan^{-1}3x - \tan^{-1}2x\NR
    \NC t_3 \NC = \tan^{-1}4x - \tan^{-1}3x\NR
    \NC \NC \cdots\NR
    \NC t_n \NC = \tan^{-1}(n + 1)x - \tan^{-1}nx\NR
  \stopalign\stopformula

  Adding and representing the sum as $S$, we have $S = \tan^{-1}(n + 1)x - \tan^{-1}x$.
\item Let $t_n$ denote the $n$th term of the series. Then $t_n = \tan^{-1}\frac{1}{2n^2} =
  \tan^{-1}\frac{2}{4n^2}$

  $= \tan^{-1}\left[\frac{(2n + 1) - (2n - 1)}{1 + (2n + 1)(2n - 1)}\right] = \tan^{-1}(2n + 1) -
  \tan^{-1}(2n - 1)$

  Thus, we can write

  \startformula\startalign
    \NC t_1 \NC = \tan^{-1}3 - \tan^{-1}1\NR
    \NC t_2 \NC = \tan^{-1}5 - \tan^{-1}3\NR
    \NC t_3 \NC = \tan^{-1}7 - \tan^{-1}5\NR
    \NC \NC \cdots\NR
    \NC t_n \NC = \tan^{-1}(2n + 1) - \tan^{-1}(2n - 1)\NR
  \stopalign\stopformula

  Adding and representing the sum as $S_n$, we have $S_n = \tan^{-1}(2n + 1) - \tan^{-1}1$

  Now $\displaystyle S_\infty = \lim_{n\to\infty}\left(\tan^{-1}(2n + 1) - \tan^{-1}1\right) = \frac{\pi}{2}
  - \frac{\pi}{4} = \frac{\pi}{4}$.
\item Let $t_n$ denote the $n$th term. $t_n = \tan^{-1}\frac{1}{3 + 3n + n^2} = \tan^{-1}\frac{1}{1 + n^2 +
  3n + 2}$

  $=\tan^{-1}\frac{1}{1 + (n + 2)(n + 1)} = \tan^{-1}(n + 2) - \tan^{-1}(n + 1)$

  Thus, we can write

  \startformula\startalign
    \NC t_1 \NC = \tan^{-1}3 - \tan^{-1}2\NR
    \NC t_2 \NC = \tan^{-1}4 - \tan^{-1}3\NR
    \NC t_3 \NC = \tan^{-1}5 - \tan^{-1}4\NR
    \NC \NC \cdots\NR
    \NC t_n \NC = \tan^{-1}(n + 2) - \tan^{-1}(n + 1)\NR
  \stopalign\stopformula

  Adding and representing the sum as $S$, we have $S = \tan^{-1}(n + 2) - \tan^{-1}2$.
\item Let $t_n$ represent the $n$th term. Then $t_n = \sin^{-1}\left[\frac{\sqrt{n} - \sqrt{n -
      1}}{\sqrt{n}\sqrt{n + 1}}\right]$

  $= \sin^{-1}\left[\frac{\sqrt{n}}{\sqrt{n}\sqrt{n + 1}} - \frac{\sqrt{n - 1}}{\sqrt{n}\sqrt{n + 1}}\right]
  = \sin^{-1}\left[\frac{1}{\sqrt{n}}\sqrt{1 - \frac{1}{n + 1}} - \frac{1}{\sqrt{n + 1}}\sqrt{1 -
      \frac{1}{n}}\right]$

  Let $\frac{1}{\sqrt{n}} = \sin A$ and $\frac{1}{\sqrt{n + 1}} = \sin B$ then $\cos A = \sqrt{1 -
    \frac{1}{n}}$ and $\cos B = \sqrt{1 - \frac{1}{n + 1}}$

  $\Rightarrow t_n = \sin^{-1}[\sin A\cos B - \sin B\cos A] = \sin^{-1}\sin(A - B) = A - B =
  \sin^{-1}\frac{1}{\sqrt{n}} - \sin^{-1}\frac{1}{\sqrt{n + 1}}$

  Thus, we can write

  \startformula\startalign
    \NC t_1 \NC = \sin^{-1}1 - \sin^{-1}\frac{1}{\sqrt{2}}\NR
    \NC t_2 \NC = \sin^{-1}\frac{1}{\sqrt{2}} - \sin^{-1}\frac{1}{\sqrt{3}}\NR
    \NC t_3 \NC = \sin^{-1}\frac{1}{\sqrt{3}} - \sin^{-1}\frac{1}{\sqrt{4}}\NR
    \NC \NC \cdots\NR
    \NC t_n \NC = \sin^{-1}\frac{1}{\sqrt{n}} - \sin^{-1}\frac{1}{\sqrt{n + 1}}\NR
  \stopalign\stopformula

  Adding and representing the sum as $S$, we have $S = \sin^{-1}1 - \sin^{-1}\frac{1}{\sqrt{n + 1}}$

  $\displaystyle\lim_{n\to\infty}S = \frac{\pi}{2}$.
\item Let $C = \cos\alpha + x\cos(\alpha + \beta) + \frac{x^2}{2!}\cos(\alpha + 2\beta) +
  \cdots$ to $\infty$

  and $S = \sin\alpha + x\sin(\alpha + \beta) + \frac{x^2}{2!}\sin(\alpha + 2\beta) +
  \cdots$ to $\infty$

  $C + iS = e^{i\alpha} + xe^{i(\alpha + \beta)} + \frac{x^2}{2!}e^{i(\alpha + 2\beta)} + \cdots$ to
  $\infty$

  $= e^{i\alpha}\left[1 + xe^{i\beta} + \frac{x^2}{2!}e^{i2\beta} + \cdots \text{ to } \infty\right]$

  $= e^{i\alpha}\left[1 + z + \frac{z^2}{2!} + \cdots \text{ to } \infty\right]$ where $z = x.e^{i\beta}$

  $= e^{i\alpha}e^{xe^{i\beta}} = e^{i\alpha}e^{x\cos\beta}e^{i\sin\beta} = e^{x\cos\beta}[\cos(\alpha +
    x\sin\beta) + i\sin(\alpha + i\sin\beta)]$

  Comparing real parts we have $C = e^{x\cos\beta}\cos(\alpha + x\sin\beta)$.
\item Let $C = \cos\theta - \frac{\cos2\theta}{2!} + \frac{\cos3\theta}{3!} - \cdots$ to $\infty$

  and $S = \sin\theta - \frac{\sin2\theta}{2!} + \frac{\sin3\theta}{3!} - \cdots$ to
  $\infty$

  Then $C + iS = (\cos\theta + i\sin\theta) - \frac{\cos2\theta + i\sin2\theta}{2!} + \frac{\cos3\theta +
    i\sin3\theta}{3!} - \cdots$ to $\infty$

  $= e^{i\theta} - \frac{e^{i2\theta}}{2!} + \frac{e^{i3\theta}}{3!} + \cdots$ to $\infty = 1 -
  e^{-i\theta}$

  $= 1 - e^{-(\cos\theta + i\sin\theta)} = 1 - e^{-\cos\theta}e^{-i\sin\theta} = 1 -
  e^{-\cos\theta}[\cos(\sin\theta) - i\sin(\sin\theta)]$

  Comparing imaginary parts $C = e^{-\cos\theta}\sin(\sin\theta)$.
\item Let $S = \sin\theta\cos\theta + \frac{\sin2\theta\cos^2\theta}{2!} +
  \frac{\sin3\theta\cos^3\theta}{3!} + \cdots $ to $\infty$

  and $C = 1 + \cos\theta\cos\theta + \frac{\cos2\theta\cos^2\theta}{2!} +
  \frac{\cos3\theta\cos^2\theta}{3!} + \cdots$ to $\infty$

  Now $C + iS = 1 + \cos\theta e^{i\theta} + \frac{\cos^2\theta}{2!}e^{i2\theta} +
  \frac{\cos^3\theta}{3!}e^{i3\theta} + \cdots$ to $\infty$

  $= 1 + z + \frac{z^2}{2!} + \frac{z^3}{3!} + \cdots$ to $\infty$[where $z = \cos\theta e^{i\theta}$]

  $ = e^z = e^{\cos\theta e^{i\theta}} = e^{\cos^2\theta}e^{i\sin\theta\cos\theta} =
  e^{\cos^2\theta}[\cos(\sin\theta\cos\theta) + i\sin(\sin\theta\cos\theta)]$

  Comparing imaginary parts $S = e^{\cos^2\theta}\sin(\sin\theta\cos\theta)$.
\item Let $C = \cos\theta + \frac{\sin\theta}{1!}\cos2\theta + \frac{\sin^2\theta}{2!}\cos3\theta +
  \frac{\sin^3\theta}{3!}\cos4\theta + \cdots$ to $\infty$

  and $S = \sin\theta + \frac{\sin\theta}{1!}\sin2\theta + \frac{\sin^2\theta}{2!}\sin3\theta +
  \frac{\sin^3\theta}{3!}\sin4\theta + \cdots$ to $\infty$

  $C + iS = e^{i\theta} + \frac{\sin\theta}{1!}e^{i2\theta} + \frac{\sin^2\theta}{2!}e^{i3\theta} + \cdots$
  to $\infty$

  $= e^{i\theta}\left[1 + \frac{\sin\theta}{1!}e^{i\theta} + \frac{\sin^2\theta}{2!}e^{i2\theta} + \cdots
    \infty\right]$

  $= e^{i\theta}\left[1 + z + \frac{z^2}{2!} + \frac{z^3}{3!} + \cdots \infty\right]$ where $z = \sin\theta
  e^{i\theta}$

  $= e^{i\theta}e^z = e^{i\theta}e^{\sin\theta e^{i\theta}} = e^{i\theta}e^{\sin\theta(\cos\theta +
    i\sin\theta)}$

  $= e^{\sin\theta\cos\theta}e^{i(\theta + \sin^2\theta)} = e^{\sin\theta\cos\theta}[cos(\theta +
    \sin^2\theta) + i\sin(\theta + \cos^2\theta)]$

  Comparing real parts, we have sum as $C = e^{\sin\theta\cos\theta}\cos(\theta + \sin^2\theta)$.
\item Let $C = x\cos\theta + \frac{x^2}{2}\cos2\theta + \frac{x^3}{3}\cos3\theta +
  \cdots$ to $\infty$

  and $S = x\sin\theta + \frac{x^2}{2}\sin2\theta + \frac{x^3}{3}\sin3\theta +
  \cdots$ to $\infty$

  $C + iS = xe^{i\theta} + \frac{x^2}{2}e^{i2\theta} + \frac{x^3}{3}e^{i3\theta} + \cdots$ to $\infty$

  $= z + \frac{z^2}{2} + \frac{z^3}{3} + \cdots$ to $\infty$ where $z = xe^{i\theta}$

  $= -\log_e(1 - z) = -\log_e\left(1 - xe^{i\theta}\right) = -\log_e[(1 - x\cos\theta) - ix\sin\theta]$

  We know that $\log_e(\alpha + \beta) = \frac{1}{2}\log_e\left(\alpha^2 + \beta^2\right) +
  i\tan^{-1}\frac{\beta}{\alpha}$

  Using the formula above we can write

  $C + iS = -\left[\frac{1}{2}\log_e\left\{(1 - x\cos\theta)^2 + x^2\sin^2\theta\right\} -
    i\tan^{-1}\frac{x\sin\theta}{1 - x\cos\theta}\right]$

  Comparing real parts we get $C = -\frac{1}{2}\log_e(1 - 2x\cos\theta + x^2)$.
\item We proceed like previous problem and compare imaginary parts to obtain the sum as $S =
  \tan^{-1}\frac{x\sin\theta}{1 - x\cos\theta}$.
\item Let $C = \cos\theta + \frac{\cos2\theta}{2} + \frac{\cos3\theta}{3} +
  \frac{\cos4\theta}{4} + \cdots$ to $\infty$

  and $S = \sin\theta - \frac{1}{2}\sin2\theta + \frac{1}{3}\sin3\theta - \cdots \infty$

  $\Rightarrow C + iS = e^{i\theta} - \frac{1}{2}e^{i2\theta} + \frac{1}{3}e^{i3\theta} - \cdots \infty$

  $= \log_e\left(1 + e^{i\theta}\right) = \log_e(1 + \cos\theta + i\sin\theta)$

  $= \frac{1}{2}\log_e\left[(1 + \cos\theta)^2 + \sin^2\theta\right] + i\tan^{-1}\frac{\sin\theta}{1 +
    \cos\theta}$

  $ = \frac{1}{2}\log_e\left(4\cos^2\frac{\theta}{2}\right) + i\tan^{-1}\tan\frac{\theta}{2} =
  \log_e\left(2\cos\frac{\theta}{2}\right) + i\frac{\theta}{2}$

  Comparing real parts, we get $C = \log_e\left(2\cos\frac{\theta}{2}\right)$.
\item Let $C = \cos\frac{\pi}{3} + \frac{1}{2}\cos\frac{2\pi}{3} + \frac{1}{3}\cos\frac{3\pi}{3} +
  \cdots$ to $\infty = 0$

  and $S = \sin\frac{\pi}{3} + \frac{1}{2}\sin\frac{2\pi}{3} + \frac{1}{3}\sin\frac{3\pi}{3} + \cdots$ to
  $\infty$

  $\Rightarrow C + iS = e^{i\frac{\pi}{3}} + \frac{1}{2}e^{i\frac{2\pi}{3}} + \frac{1}{3}e^{i\frac{3\pi}{3}}
  + \cdots$ to $\infty$

  $= -\log_e\left(1 - e^{i\frac{\pi}{3}}\right) = -\log_e\left[1 - \left(\cos\frac{\pi}{3} +
    i\sin\frac{\pi}{3}\right)\right] = -\log_e\left[\frac{1}{2} - i\frac{\sqrt{3}}{2}\right]$

  $= -\frac{1}{2}\log_e\left(\frac{1}{3} + \frac{3}{4}\right) -i\tan^{-1}\sqrt{3} = 0 + i\frac{\pi}{3}$

  Comparing real and imaginary parts $C = 0, S = \frac{\pi}{3}$.
\item Let $S = \sin\alpha + n\sin(\alpha + \beta) + \frac{n(n - 1)}{2!}\sin(\alpha + 2\beta) + \cdots$ upto
  $(n + 1)$ terms

  and $C = \cos\alpha + n\cos(\alpha + \beta) + \frac{n(n - 1)}{2!}\cos(\alpha + 2\beta) + \cdots$ upto $(n
  + 1)$ terms

  $C + iS = e^{i\alpha} + ne^{i(\alpha + \beta)} + \frac{n(n - 1)}{2}e^{i(\alpha + 2\beta)} + \cdots$

  $= e^{i\alpha}\left[1 + ne^{i\beta} + \frac{n(n - 1)}{2}e^{i2\beta} + \cdots\right] = e^{i\alpha}\left(1 +
  e^{ii\beta}\right)^n$

  $= e^{i\alpha}\left[\left(1 + \cos\beta + i\sin\beta\right)^n\right] =
  e^{i\alpha}\left[2\cos^2\frac{\beta}{2} + i.2\sin\frac{\beta}{2}\cos\frac{\beta}{2}\right]^n$

  $= e^{i\alpha}\left(2\cos\frac{\beta}{2}\right)^ne^{in\frac{\beta}{2}} =
  2^n\cos^n\frac{\beta}{2}\left[\cos\left(\alpha + \frac{n\beta}{2}\right) + i\sin\left(\alpha +
    \frac{n\beta}{2}\right)\right]$

  Equating imaginary parts gives us

  $S = 2^n\cos^n\frac{\beta}{2}\sin\left(\alpha + \frac{n\beta}{2}\right)$.
\item Let $S = \sin\theta + \frac{1}{2}\sin3\theta + \frac{1.3}{2.4}\sin5\theta + \cdots$
  to $\infty$

  and $C = \cos\theta + \frac{1}{2}\cos3\theta + \frac{1.3}{2.4}\cos5\theta + \cdots$
  to $\infty$

  $\Rightarrow C + iS = e^{i\theta} + \frac{1}{2}e^{i3\theta} + \frac{\frac{1}{2}\left(\frac{1}{2} +
    1\right)}{1.2}e^{i5\theta} + \cdots$ to $\infty$

  $= e^{i\theta}\left[1 + \frac{1}{2}e^{i2\theta} + \frac{\frac{1}{2}\left(\frac{1}{2} +
      1\right)}{1.2}e^{i4\theta} + \cdots \text{ to } \infty\right]$

  $= e^{i\theta}\left(1 - e^{i2\theta}\right)^{-\frac{1}{2}} = e^{i\theta}[1 - (\cos2\theta +
    i\sin\theta)]^{-\frac{1}{2}}$

  $= e^{\theta}(2\sin\theta)^{-\frac{1}{2}}[\sin\theta - i\cos\theta]^{-\frac{1}{2}} =
  (2\sin\theta)^{-\frac{1}{2}}e^{i\theta}\left[\cos\left(\frac{\pi}{2} - \theta\right) +
    i\sin\left(\frac{\pi}{2} - \theta\right)\right]^{-\frac{1}{2}}$

  $= \frac{1}{\sqrt{2\sin\theta}}e^{\left(\theta + \frac{pi}{4} - \frac{\theta}{2}\right)} =
  \frac{1}{\sqrt{2\sin\theta}}\left[\cos\left(\frac{\pi}{4} + \frac{\theta}{2}\right) +
    i\sin\left(\frac{\pi}{4} + \frac{\theta}{2}\right)\right]$

  Equating imaginary parts gives us $S = \frac{1}{\sqrt{2\sin\theta}}\sin\left(\frac{\pi}{4} +
  \frac{\theta}{2}\right)$.
\item Let $C = 1 - \frac{1}{2}\cos\theta + \frac{1.3}{2.4}\cos2\theta - \frac{1.3.5}{2.4.6}\cos3\theta +
  \cdots$ to $\infty$

  and $S = - \frac{1}{2}\sin\theta + \frac{1.3}{2.4}\sin2\theta - \frac{1.3.5}{2.4.6}\sin3\theta + \cdots$
  to $\infty$

  $\Rightarrow C + iS = 1 - \frac{1}{2}e^{i\theta} + \frac{1.3}{2.3}e^{i2\theta} -
  \frac{1.3.5}{2.4.6}e^{i3\theta} + \cdots$ to $\infty$

  $= \left(1 + e^{i\theta0}\right)^{-1/2} = (1 + \cos\theta + i\sin\theta)^{-1/2} =
  \left[2\cos^2\frac{\theta}{2} + 2i\sin\frac{\theta}{2}\cos\frac{\theta}{2}\right]^{-1/2}$

  $= \frac{1}{\sqrt{2\cos\frac{\theta}{2}}}\left[\cos\frac{\theta}{2} + i\sin\frac{\theta}{2}\right]^{-1/2}
  = \frac{1}{\sqrt{2\cos\frac{theta}{2}}}\left(e^{i\theta/2}\right)^{-1/2}$

  $= \frac{1}{\sqrt{2\cos\frac{\theta}{2}}}\left(\cos\frac{\theta}{4} - i\sin\frac{\theta}{4}\right)$

  Comparing real parts of both sides gives us

  $C = \frac{1}{\sqrt{2\cos\frac{\theta}{2}}}\cos\frac{\theta}{4}$.
\item Let $S = \frac{1}{2}\sin\theta + \frac{1.3}{2.4}\sin2\theta + \frac{1.3.5}{2.4.6}\sin3\theta + \cdots$
  to $\infty$

  and $C = 1 = \frac{1}{2}\cos\theta + \frac{1.3}{2.4}\cos2\theta + \frac{1.3.5}{2.4.6}\cos3\theta + \cdots$
  to $\infty$

  $\Rightarrow C + iS = 1 + \frac{1}{2}e^{i\theta} + \frac{1.3}{2.4}e^{2i\theta} +
  \frac{1.3.5}{2.4.6}e^{3i\theta} + \cdots$ to $\infty$

  $= \left(1 - e^{i\theta}\right)^{-1/2} = \left(1 - \cos\theta + i\sin\theta\right)^{-1/2} =
  \left(2\sin^2\frac{\theta}{2} - 2i\sin\frac{\theta}{2}\cos\frac{\theta}{2}\right)^{-1/2}$

  $= \frac{1}{\sqrt{2\sin\frac{\theta}{2}}}\left(\sin\frac{\theta}{2} - i\cos\frac{\theta}{2}\right)^{-1/2}
  = \frac{1}{\sqrt{2\sin\frac{\theta}{2}}}\left[e^-i\left(\frac{\pi}{2} - \theta\right)\right]^{-1/2}$

  $= \frac{1}{\sqrt{2\sin\frac{\theta}{2}}}\left[\cos\left(\frac{\pi}{4} - \frac{\theta}{2}\right) +
    i\sin\left(\frac{\pi}{4} - \frac{\theta}{2}\right)\right]$

  Comparing imaginary parts gives us $S = \frac{1}{\sqrt{2\sin\frac{\theta}{2}}}\sin\left(\frac{\pi}{4} -
  \frac{\theta}{2}\right)$.
\item Let $S = \sin x + \frac{1}{2}\sin2x + \frac{1}{2^2}\sin3x + \cdots$ to $\infty$

  and $C = \cos x + \frac{1}{2}\cos2x + \frac{1}{2^2}\cos3x + \cdots$ to $\infty$

  $\Rightarrow C + iS = e^{ix} + \frac{1}{2}e^{i2x} + \frac{1}{2^2}e^{i3x} + \cdots$ to $\infty$

  $= \frac{2e^{ix}}{2 - e^{ix}} = 2\left[\frac{\cos x + i\sin x}{2 - \cos x - i\sin x}\right] =
  2\left[\frac{\cos x + i\sin x}{2 - \cos x - i\sin x}.\frac{2 - \cos x + i\sin x}{2 - \cos x + i\sin
      x}\right]$

  $= 2\left[\frac{2\cos x - 1 + i2\sin x}{5 - 4\cos x}\right]$

  $\Rightarrow S = \frac{4\sin x}{5 - 4\cos x}$.
\item Let $C = 1 + x\cos\theta + x^2\cos2\theta + \cdots + x^{n - 1}\cos(n - 1)\theta$

  and $S = x\sin\theta + x^2\sin2\theta + \cdots + x^{n - 1}\sin(n - 1)\theta$

  $\Rightarrow C + iS = 1 + xe^{i\theta} + x^2e^{i2\theta} + \cdots + x^{n - 1}e^{i(n - 1)\theta}$

  $= \frac{1 - \left(x^{i\theta}\right)^n}{1 - xe^{i\theta}} = \frac{1 - \left(x^{i\theta}\right)^n}{1 -
    xe^{i\theta}}.\frac{1 - \left(x^{-i\theta}\right)}{1 - xe^{-i\theta}}$

  $= \frac{1 - x^ne^{in\theta} - xe^{-i\theta0} + x^{n + 1}e^{i(n - 1)\theta}}{1 - 2x\cos\theta + x^2}$

  $= \frac{1 - x^n(\cos n\theta + i\sin n\theta) - x(\cos\theta - i\sin\theta) + x^{n + 1}[\cos(n - 1)\theta
      + i\sin(n - 1)\theta]}{1 - 2x\cos\theta + x^2}$

  $\Rightarrow C = \frac{1 - x^nn\cos n\theta - x\cos\theta + x^{n + 1}\cos(n - 1)\theta}{1 - 2x\cos\theta +
    x^2}$.
\item Let $C = \sin\alpha + x\sin(\alpha + \beta) + x^2\sin(\alpha + 2\beta) + \cdots$

  and $S = \cos\alpha + x\cos(\alpha + \beta) + x^2\cos(\alpha + 2\beta) + \cdots$

  $\Rightarrow C + iS = e6{i\alpha} + xe^{i(\alpha + \beta)} + x^2e^{i(\alpha + 2\beta)} + \cdots$

  $= e^{i\alpha}\left[1 + xe^{i\beta} + x^2e^{i2\beta} + \cdots\right] = e^{i\alpha}\left[\frac{1 -
      x^ne^{in\beta}}{1 - xe^{i\beta}}\right]$

  $= e^{i\alpha}\frac{1 - x^ne^{in\beta}}{1 - xe^{i\beta}}\frac{1 - xe^{-i\beta}}{1 - xe^{-i\beta}} =
  \frac{e^{i\alpha}}{1 - x\left(e^{i\beta} + e^{-i\beta} + x^2\right)}\left[1 - xe^{-i\beta} -
    x^ne^{in\beta} + x^{n + 1}e^{i(n - 1)\beta}\right]$

  Proceeding like previous problems and comparing imaginary parts gives us

  $S = \frac{\sin\alpha - x\sin(\alpha - \beta) - x^n\cos(\alpha + n\beta) + x^{n + 1}\sin[\alpha + (n -
      1)\beta]}{1 - 2x\cos\beta + x^2}$.
\item Let $C = \cos\theta + 2\cos2\theta + 3\cos3\theta + \cdots + n\cos n\theta$

  and $S = \sin\theta + 2\sin2\theta + 3\sin3\theta + \cdots + n\sin n\theta$

  $\Rightarrow C + iS = e^{i\theta} + 2e^{i2\theta} + 3e^{i3\theta} + \cdots + ne^{in\theta}$

  $\Rightarrow \left(1 - e^{i\theta}\right) = \frac{e^{i\theta}\left(1 - e^{in\theta}\right)}{1 -
    e^{i\theta}} - 2.e^{i(n + 1)\theta}$

  $\Rightarrow C + iS = \frac{e^{i\theta} + (n + 1)e^{i(n + 1)\theta} + ne^{i(n + 2)\theta}}{\left(1 -
    e^{i\theta}\right)^2}$

  Now, $\left(1 - e^{i\theta}\right)^2 = (1 - \cos\theta - i\sin\theta)^2 = \left(2\sin^2\frac{\theta}{2} -
  2i\sin\frac{\theta}{2}\cos\frac{\theta}{2}\right)^2$

  $= 4\sin^2\frac{\theta}{2}\left[-i\left(\cos\frac{\theta}{2} + i\sin\frac{\theta}{2}\right)\right]^2 =
  -4\sin^2\frac{\theta}{2}e^{i\theta}$

  $\Rightarrow C + iS = -\frac{1}{4\sin^2\frac{\theta}{2}}\left[1 - (n + 1)e^{in\theta} + n.e^{i(n +
      1)\theta}\right]$

  Comparing real parts gives us

  $C = -\frac{1}{4\sin^2\frac{\theta}{2}}\left[1 - (n + 1)\cos n\theta + n\cos(n + 1)\theta\right]$.
\item Let $C = \cos\theta - \frac{1}{3}\cos3\theta + \frac{1}{5}\cos5\theta - \cdots$ to
  $\infty$

  and $S = \sin\theta - \frac{1}{3}\sin3\theta + \frac{1}{5}\sin5\theta - \cdots$ to
  $\infty$

  $\Rightarrow C + iS = e^{i\theta} - \frac{e^{i3\theta}}{3} + \frac{e^{i5\theta}}{5} + \cdots$ to $\infty$

  $= \tan^{-1}e^{i\theta}$

  Taking conjugate gives us $C - iS = \tan^{-1}e^{-i\theta}$

  Adding the two gives us

  $2C = \tan^{-1}(\cos\theta + i\sin\theta) + \tan^{-1}(\cos\theta - i\sin\theta) =
  \tan^{-1}\left[\frac{2\cos\theta}{1 - \left(\cos^2\theta + \sin^2\theta\right)}\right]$

  $2C = \tan^{-1}\infty C = \frac{\pi}{4}$.
\item Using the formula $\cos^2\theta = \frac{1 + \cos2\theta}{2}$ we can rewite the given series(let its
  sum be $S$) as

  $S = \frac{n}{2} + \frac{\left[\cos\frac{2\pi}{n} + \cos\frac{6\pi}{n} + \cos\frac{10\pi}{n}\right]}{2}$

  We know that $\cos\alpha + \cos(\alpha + \beta) + \cos(\alpha + 2\beta) + \cdots + \cos(\alpha + (n -
  1)\beta) = \frac{\sin\frac{n\beta}{2}}{\sin\frac{\beta}{2}}\cos\left[\alpha + (n -
    1)\frac{\beta}{2}\right]$

  Here $\alpha = \frac{2\pi}{n}, \beta = \frac{4\pi}{n}$, which makes

  $\sin\frac{n\beta}{2} = \sin2\pi$, which makes right hand side equal to zero.

  Thus, $S = \frac{n}{2}$.
\item Let the sum of the series be $S$. We substitute $\sin^2\theta = \frac{1 - \cos2\theta}{2}$ to obtain

  $S = \frac{n}{2} - \frac{\left[\cos2\alpha + \cos\left(2\alpha + \frac{4\pi}{n}\right) + \cdots\right]}{2}$

  Following like previous problem we obtain the same sum and we have same value of $\beta$ which will make
  the sum zero.

  Thus, $S = \frac{n}{2}$.
\item Let the sum of the series be $S$. We again substitute $\sin^2\theta = \frac{1 - \cos2\theta}{2}$

  $S = \frac{1 - \cos2\theta}{2} + \frac{1 - \cos6\theta}{2} + \frac{1 - \cos10\theta}{2} + \cdots$ to $2n$
  terms

  $= n - \frac{\left[\cos2\theta + \cos6\theta + \cos10\theta + \cdots \text{ to } 2n \text{
        terms}\right]}{2}$

  Now in this case $\alpha = 2\theta$ and $\beta = 4\theta$

  Thus, $\sin
  n\beta = \sin 4n\theta = \sin\pi$ which is zero.

  Thus, $S = 0$.
\item We use the result of first two problems to obtain numerator as $\frac{\sin n\alpha}{\sin\alpha}\sin
  n\alpha$ and denominator as $\frac{\sin n\alpha}{\sin\alpha}\cos n\alpha$.

  Thus, given fraction is $\tan n\alpha$.
\item We use the formula $\sin^3\theta = \frac{3\sin\theta - \sin3\theta}{4}$, the series becomes

  $S = \frac{3\sin\theta - \sin3\theta}{4} + \frac{3\sin3\theta - \sin9\theta}{4} + \frac{3\sin5\theta -
  \sin15\theta}{4} + \cdots$

  Again using the fact that $\sin\alpha + \sin(\alpha + \beta) + \sin(\alpha + 2\beta) + \cdots =
  \frac{\sin\frac{n\beta}{2}}{\sin\frac{\beta}{2}}\sin\left[\alpha + (n - 1)\frac{\beta}{2}\right]$

  $S = \frac{3}{4}\frac{\sin^2n\theta}{\sin\theta} - \frac{1}{4}.\frac{\sin^23n\theta}{\sin3\theta}$.
\item We can write given series as $\cos\theta + \cos\left[\theta + \left(\theta +
  \frac{\pi}{2}\right)\right] + \cos\left[\theta + 2\left(\theta + \frac{\pi}{2}\right)\right] + \cdots$

  $= \frac{\sin \frac{n}{2}\left(\theta + \frac{\pi}{2}\right)}{\sin\frac{\left(\theta +
    \frac{\pi}{2}\right)}{2}}\sin\left(\theta + (n - 1)\frac{\left(\theta + \frac{\pi}{2}\right)}{2}\right)$.
\item We can use teh formula $2\sin A\sin B = \cos(A - B) - \cos(A + B)$, and the given series can be
  written as

  $\frac{1}{2}\left[\cos\beta - \cos\left(\alpha + \frac{\beta}{2}\right) - \cos\beta + \cos\left(\alpha +
  \frac{3\beta}{2}\right) + \cdots\right]$

  $= -\frac{1}{2}\left[\cos\left(\alpha + \frac{\beta}{2}\right) + \cos\left(\alpha + \pi + \frac{3\beta}{2}\right)
    + \cos\left(\alpha + 2\pi + \frac{5\beta}{2}\right) + \cdots\right]$

  Now from  the formula $\alpha = \alpha + \frac{\beta}{2}$ and $\beta = \pi + \beta$, and thus, the sum
  becomes

  $S = -\frac{1}{2}\frac{\sin \frac{n(\pi + \beta)}{2}}{\sin\frac{\pi + \beta}{2}}\cos\left[\alpha +
    \frac{\beta}{2} + (n - 1)\frac{\pi + \beta}{2}\right]$.
\item Using the formula $\cos^2\theta - \frac{1 + \cos2\theta}{2}$, we can write the given series as

  $S = \frac{n}{2} + \cos2\alpha + \cos2(\alpha + \beta) + \cos2(\alpha + 2\beta) + \cdots$

  Now usiing the formula we have $\alpha = 2\alpha$ and $\beta = 2\beta$, which gives us

  $S = \frac{n}{2} + \frac{\sin n\beta}{\sin\beta}\cos[2\alpha + (n - 1)\beta]$.
\item We can use the formula $\cos A\cos B = \frac{\cos(A + B) + \cos(A - B)}{2}$, and thus the sum becomes

  $S = \frac{1}{2}\left[\cos\theta + \cos3\theta + \cos\theta + \cos7\theta + \cdots + \cos\theta + \cos(4n
  - 1)\theta\right]$

  $= \frac{n\cos\theta}{2} + \frac{1}{2}[\cos3\theta  + \cos7\theta + \cos11\theta + \cdots + \cos(4k -
  1)\theta]$

  Comparing from the foumula $\alpha = 3\theta, \beta = 4\theta$, which gives the sum as

  $S = \frac{n\cos\theta}{2} + \frac{\sin2n\theta}{\sin2\theta}\cos[3\theta + (n - 1)2\theta]$.
\item We can write the given series as

  $S = \frac{\sin\theta + \sin(\pi + 2\theta) + \sin(2\pi + 3\theta) + \sin(3\pi + 4\theta) +
  \cdots}{\cos\theta + \cos(\pi + 2\theta) + \cos(2\pi + 3\theta) + \cos(3\pi + 4\theta) + \cdots}$

  Using the formulas we have $\alpha = \theta, \beta = \pi + \theta$, which gives us

  $S = \tan\left[\theta + \frac{(n - 1)(\pi + \theta)}{2}\right] = \tan\frac{(n + 1)}{2}(\pi + \theta)$.
\item We can write $\sqrt{1 - \sin2\theta} = \sqrt{\sin^2\theta + \cos^2\theta - 2\sin\theta\cos\theta} =
  \pm(\sin\theta -\cos\theta)$ and so on.

  Thus, the series becomes $\pm[\sin\theta - \cos\theta + \sin2\theta - \cos2\theta + \sin3\theta -
    \cos3\theta + \cdots]$

  Now using the formula we have the sum as

  $\pm\left[\frac{\sin\frac{n\theta}{2}}{\sin\frac{\theta}{2}}\right]\left[\sin\left(\theta + (n -
    1)\frac{\theta}{2} - \cos\left(\theta + (n - 1)\frac{\theta}{2}\right)\right)\right]$.
\item Sum of the series can be obtained from the formula as

  $S = \frac{\sin\frac{n\phi}{2}}{\sin\frac{\phi}{2}}\left[\cos\left(\theta + (n -
  1)\frac{\phi}{2}\right)\right]$

  Given that $\phi$ is the exterior angle of a regular polygon of $n$ sides, which makes $\phi =
  \frac{2\pi}{n}$, which make the denominator term $\sin\frac{n\phi}{2} = \sin\pi = 0$ making the entire sum
  zero.
\item We write $\cos^2x = \frac{1 + \cos2x}{2}$ to transform the given series to

  $S = \frac{1}{2}[1 + \cos2x + 1 + \cos6x + 1 + \cos10x + \cdots] = \frac{1}{2}[n + \cos2x + \cos6x +
  \cos10x + \cdots]$

  Clearly, $\alpha = 2x$ and $\beta = 4x$, which gives the sum of the series as

  $S = \frac{1}{2}\left[n + \frac{\sin 2nx}{\sin 2x}\cos[2x + (n - 1)2x]\right] = \frac{1}{2}\left[n +
    \frac{\sin4nx}{2\sin2x}\right]$.
\item We make use of the formula $\cos^3\theta = \frac{\cos3\theta + \cos\theta}{4}$ to transform the given
  series to

  $S = \frac{1}{4}\left[\cos3\theta + \cos3\left(\theta - \frac{2\pi}{n}\right) + \cos3\left(\theta -
    \frac{4\pi}{n}\right) + \cdots + 3\cos\theta + \3cos\left(\theta - \frac{2\pi}{n}\right) +
    3\cos\left(\theta - \frac{4\pi}{n}\right) + \cdots\right]$

  In both the series $\beta$ is a multiple of $\frac{2\pi}{n}$, which makes the term in denominator
  $\sin\frac{n\beta}{2} = \sin\pi = 0$, which makes the entire sum zero.
\item $\sin^4x = \left(\frac{1 - \cos2x}{2}\right)^2 = \frac{1 - 2\cos2x + \cos^22x}{4} = \frac{1}{4}\left(1
  - 2\cos2x + \frac{1 + \cos4x}{2}\right) = \frac{3}{8} - \frac{1}{2}\cos2x + \frac{1}{8}\cos4x$

  We observe the series and the transformation we obtain to deduce that $\beta$ will be $\frac{2\pi}{n}$ so
  the sum of the terms of $\cos x$ would be zero and $\frac{3}{8}$ would be multiplied with $n$. Hence, sum
  would be $\frac{3n}{8}$.
\item Given series can be written as $\cos x + \sin3x + \cos5x + \sin7x + \cdots + \cos(4n - 3)x + \sin(4n -
  1)x$

  For two parts $\alpha = x, 3x$ and $\beta = 4x$. Thus, sum would be

  $S = \frac{\sin2nx}{\sin2x}[\cos\{x + (n - 1)2x\} - \sin\{3x + (n - 1)2x\}]$.
\item Using the formula $\sin^3\theta = \frac{3\sin\theta - \sin3\theta}{4}$ we can write

  \startformula\startalign
    \NC \sin^3\frac{x}{3}\NC = \frac{1}{4}\left[3\sin\frac{x}{3} - \sin x\right]\NR
    \NC 3\sin^3\frac{x}{3^2}\NC = \frac{1}{4}\left[3^2\sin\frac{x}{3^2} - 3\sin\frac{x}{3}\right]\NR
    \NC \NC \cdots\NR
    \NC 3^{n - 1}\sin\frac{x}{3^n}\NC = \frac{1}{4}\left[3^n\sin\frac{x}{3^n} - 3^{n - 1}\sin\frac{x}{3^{n - 1}}\right]
  \stopalign\stopformula

  Adding these we get sum of terms $S_n$ as $S_n = \frac{1}{4}\left[3^n\sin\frac{x}{3^n} - \sin x\right]$,
  and thus, $S_\infty = -\frac{3}{4}\sin x$.
\item We write $\sec\theta\sec2\theta = \frac{1}{\cos\theta\cos2\theta} = \frac{\sin(2\theta -
  \theta)}{\sin\theta\cos\theta\cos2\theta} = \frac{\sin2\theta\cos\theta -
  \cos2\theta\sin\theta}{\sin\theta\cos\theta\cos2\theta} = \frac{1}{\sin\theta}[\tan2\theta - \tan\theta]$

  Thus, we can write terms as

  \startformula\startalign
    \NC \sec\theta\sec2\theta \NC = \frac{1}{\sin\theta}[\tan2\theta - \tan\theta]\NR
    \NC \sec2\theta\sec3\theta \NC = \frac{1}{\sin\theta}[\tan3\theta - \tan2\theta]\NR
    \NC\NC \cdots \NR
    \NC \sec n\theta\sec(n + 1)\theta \NC = \frac{1}{\sin\theta}[\tan(n + 1)\theta - \tan n\theta]\NR
  \stopalign\stopformula

  Adding these we get sum $S$ as $S = \frac{1}{\sin\theta}[\tan(n + 1)\theta - \tan\theta]$.
\item We know that $\tan2A - \tan A = \tan A\sec2A$ and thus we can write given series in termwise manner

  \startformula\startalign
    \NC \tan\frac{\theta}{2}\sec\theta \NC = \tan\theta - \tan\frac{\theta}{2}\NR
    \NC \tan\frac{\theta}{2^2}\sec\frac{\theta}{2} \NC = \tan\frac{\theta}{2} - \tan\frac{\theta}{2^2}\NR
    \NC \tan\frac{\theta}{2^3}\sec\frac{\theta}{2^2} \NC = \tan\frac{\theta}{2^2} - \tan\frac{\theta}{2^3}\NR
    \NC\NC \cdots \NR
  \stopalign\stopformula

  Thus, sum of infinite terms would be $\tan\theta$.
\item We know that $\cot(A - B) = \frac{\cot A\cot B + 1}{\cot B - \cot A}$, and it follows that

  $\cot r\theta\cot(r + 1)\theta = \cot\theta\cot r\theta - \cot\theta\cot(r + 1)\theta - 1$

  Writing the series in termwise fashion gives us

  \startformula\startalign
    \NC \cot\theta\cot2\theta \NC = \cot\theta\cot\theta - \cot\theta\cot2\theta - 1\NR
    \NC \cot2\theta\cot3\theta \NC = \cot\theta\cot2\theta - \cot\theta\cot3\theta - 1\NR
    \NC\NC\cdots\NR
    \NC \cot n\theta\cot(n + 1)\theta \NC = \cot\theta\cot n\theta - \cot\theta\cot(n + 1)\theta - 1\NR
  \stopalign\stopformula

  Adding gives the sum $S$ as $S = \cot^2\theta - \cot\theta\cot(n + 1)\theta - n$.
\item We know that $\cos C - \cos D = 2\sin\frac{C + D}{2}\sin\frac{D - C}{2}$

  $\Rightarrow \frac{1}{\cos\theta - \cos3\theta} = \frac{1}{2\sin2\theta\sin\theta} = \frac{\sin(2\theta -
  \theta)}{2\sin\theta\sin\theta\sin2\theta} = \frac{1}{2\sin\theta}[\cot\theta - \cot2\theta]$

  Writing the series in termwise fashion gives us

  \startformula\startalign
    \NC \frac{1}{\cos\theta - \cos3\theta} \NC = \frac{1}{2\sin\theta}[\cot\theta - \cot2\theta]\NR
    \NC \frac{1}{\cos\theta - \cot5\theta} \NC = \frac{1}{2\sin\theta}[\cot2\theta - \cot3\theta]\NR
    \NC \frac{1}{\cos\theta - \cot7\theta} \NC = \frac{1}{2\sin\theta}[\cot3\theta - \cot4\theta]\NR
    \NC\NC \cdots\NR
    \NC \frac{1}{\cos\theta - \cot(2n + 1)\theta} \NC = \frac{1}{2\sin\theta}[\cot n\theta - \cot(n + 1)\theta]\NR
  \stopalign\stopformula

  Adding gives the sum $S$ as $S = \frac{1}{2\sin\theta}[\cot\theta - \cot(n + 1)\theta]$.
\item $\sec x\sec(x + y) = \frac{\sin(x + y - x)}{\sin y\cos x \cos(x + y)} = \frac{1}{\sin y}[\tan(x + y) -
  \tan x]$

  Writing the series in termwise fashion gives us

  \startformula\startalign
    \NC \sec x\sec(x + y) \NC = \frac{1}{\sin y}[\tan(x + y) - \tan x]\NR
    \NC \sec(x + y)\sec(x + 2y) \NC = \frac{1}{\sin y}[\tan(x + 2y) - \tan(x + y)]\NR
    \NC \sec(x + 2y)\sec(x + 3y) \NC = \frac{1}{\sin y}[\tan(x + 3y) - \tan(x + 2y)]\NR
    \NC\NC \cdots \NR
    \NC \sec(x + (n- 1)y)\sec(x + ny) \NC = \frac{1}{\sin y}[\tan(x + ny) - \tan(x + (n - 1)y)]\NR
  \stopalign\stopformula

  Adding gives the sum $S$ as $S = \frac{1}{\sin y}[\tan(x + ny) - \tan x]$.
\item $\frac{\sin2\theta}{\cos\theta\cos3\theta} =
  \frac{2\sin\theta\sin2\theta}{2\sin\theta\cos\theta\cos2\theta} = \frac{\cos\theta -
    \cos3\theta}{2\sin\theta\cos\theta\cos2\theta} = \frac{1}{2\sin\theta}[\sec3\theta - \sec\theta]$

  Writing the series in termwise fashion gives us

  \startformula\startalign
    \NC \frac{\sin2\theta}{\cos\theta\cos3\theta} \NC = \frac{1}{2\sin\theta}[\sec3\theta - \sec\theta]\NR
    \NC \frac{\sin4\theta}{\cos3\theta\cos5\theta} \NC = \frac{1}{2\sin\theta}[\sec5\theta - \sec3\theta]\NR
    \NC \frac{\sin6\theta}{\cos5\theta\cos7\theta} \NC = \frac{1}{2\sin\theta}[\sec7\theta - \sec5\theta]\NR
    \NC\NC \cdots \NR
    \NC \frac{\sin2n\theta}{\cos(2n - 1)\theta\cos(2n + 1)\theta} \NC = \frac{1}{2\sin\theta}[\sec(2n +
      1)\theta - \sec(2n - 1)\theta]\NR
  \stopalign\stopformula

  Adding gives the sum $S$ as $S = \frac{1}{2\sin\theta}[\sec(2n + 1)\theta - \sec\theta]$.
\item $\frac{\sin2\theta}{\sin\theta\sin3\theta} =
  \frac{2\cos\theta\sin2\theta}{2\cos\theta\sin\theta\sin3\theta} = \frac{\sin3\theta -
    \sin\theta}{2\cos\theta\sin\theta\sin3\theta} = \frac{1}{2\cos\theta}[\csc\theta - \csc3\theta]$

  Writing the series in termwise fashion gives us

  \startformula\startalign
    \NC \frac{\sin2\theta}{\sin\theta\sin3\theta}\NC = \frac{1}{2\cos\theta}[\csc\theta - \csc3\theta]\NR
    \NC \frac{\sin4\theta}{\sin3\theta\sin5\theta}\NC = \frac{1}{2\cos\theta}[\csc3\theta - \csc5\theta]\NR
    \NC \frac{\sin6\theta}{\sin5\theta\sin7\theta}\NC = \frac{1}{2\cos\theta}[\csc5\theta - \csc7\theta]\NR
    \NC\NC \cdots\NR
    \NC \frac{\sin2n\theta}{\sin(2n - 1)\theta\sin(2n + 1)\theta}\NC = \frac{1}{2\cos\theta}[\csc(2n -
      1)\theta - \csc(2n + 1)\theta]\NR
  \stopalign\stopformula

  Adding gives the sum $S$ as $S = \frac{1}{2\cos\theta[\csc\theta - csc(2n + 1)\theta]}$.
  %65
\item We know that $\tan x\sec2x = \tan2x - \tan x$, and thus, we can write given series in termwise manner
  to obtain

  \startformula\startalign
    \NC \tan x\sec2x \NC = \tan2x - \tan x\NR
    \NC \tan2x\sec2^2\NC = \tan2^2x - \tan2x\NR
    \NC \tan2^2x\sec2^3x \NC = \tan2^3x - \tan2^2x\NR
    \NC\NC \cdots\NR
    \NC \tan2^{n - 1}x\sec2^nx \NC = \tan2^nx - \tan2^{n - 1}x\NR
  \stopalign\stopformula

  Adding gives us $S = \tan2^nx - \tan x$.
  %66
\item It can be easily proven than $2\csc2\theta\cot2\theta = \csc^2\theta - 2\csc^22\theta$

  Now we can write the given series in termwise fashion as given below:

  \startformula\startalign
    \NC 2\csc2\theta\cot2\theta \NC = \csc^2\theta - 2\csc^22\theta\NR
    \NC 4\csc4\theta\cot4\theta \NC = 2\csc^22\theta - 2^2\csc^22^2\theta\NR
    \NC 8\csc8\theta\cot8\theta \NC = 2^2\csc^22^2\theta - 2^3\csc^22^3\theta\NR
    \NC\NC \cdots\NR
    \NC 2^n\csc2^n\theta\cot2^n\theta \NC = 2^{n - 1}\csc^22^{n - 1}\theta - 2^n\csc^22^n\theta\NR
  \stopalign\stopformula

  Adding we get the sum of the series as $\csc^2\theta - 2^n\csc^22^n{\theta}$.
  %67
\item We find that $\sin\theta\sin3\theta = \frac{1}{2}[\cos2\theta - \cos4\theta]$. Writing rest of the
  terms similarly gives us

  \startformula\startalign
    \NC\sin\theta\sin3\theta\NC = \frac{1}{2}[\cos2\theta - \cos4\theta]\NR
    \NC\sin\frac{\theta}{2}\sin\frac{3\theta}{2} \NC = \frac{1}{2}\left[\cos\theta - \cos2\theta\right]\NR
    \NC\sin\frac{\theta}{2^2}\sin\frac{3\theta}{2^2} \NC = \frac{1}{2}\left[\cos\frac{\theta}{2} -
      \cos\theta\right]\NR
    \NC\NC\cdots\NR
    \NC\sin\frac{\theta}{2^{n - 1}}\sin\frac{3\theta}{2^{n - 1}} \NC =
    \frac{1}{2}\left[\cos\frac{\theta}{2^{n - 2}} - \cos\frac{\theta}{2^{n - 1}}\right]\NR
  \stopalign\stopformula

  Adding gives us sum as $\frac{1}{2}\left[\cos\frac{\theta}{2^{n - 2}} - \cos4\theta\right]$.
  %68
\item $\cos\frac{\theta}{2} = \frac{2\sin\theta}{2\sin\theta}.\cos\frac{\theta}{2} =
  \frac{\sin\theta}{2}.\frac{\cos\frac{\theta}{2}}{2\sin\frac{\theta}{2}\cos\frac{\theta}{2}}$

  $= \frac{\sin\theta}{2}.\csc\frac{\theta}{2} =
  \frac{2\sin\theta}{2}.\frac{\sin\frac{\theta}{4}}{\sin\frac{\theta}{2}\sin\frac{\theta}{4}} =
  \frac{\sin\theta}{2}.\frac{\sin\left(\frac{\theta}{2} -
    \frac{\theta}{4}\right)}{.\sin\frac{\theta}{2}.\sin\frac{\theta}{4}} =
  \frac{\sin\theta}{2}\left[\cot\frac{\theta}{4} - \cot\frac{\theta}{2}\right]$

  Writing rest of the terms similarly gives us

  \startformula\startalign
    \NC\cos\frac{\theta}{2}\NC = \frac{\sin\theta}{2}\left[\cot\frac{\theta}{4} - \cot\frac{\theta}{2}\right]\NR
    \NC2\cos\frac{\theta}{2}\cos\frac{\theta}{2^2}\NC = \frac{\sin\theta}{2}\left[\cot\frac{\theta}{8} -
      \cot\frac{\theta}{4}\right]\NR
    \NC\NC\cdots\NR
    \NC2^n\cos\frac{\theta}{2}\cos\frac{\theta}{2^2}\cdots\cos\frac{\theta}{2^n}\NC =
    \frac{\sin\theta}{2}\left[\cot\frac{\theta}{2^{n + 1}} -
      \cot\frac{\theta}{2^n}\right]\NR
  \stopalign\stopformula

  Adding gives the sum as $\frac{\sin\theta}{2}\left[\cot\frac{\theta}{2^{n + 2}} -
    \cot\frac{\theta}{2}\right]$.
  %69
\item $\tan^2x\tan2x = \frac{2\tan x.\tan^2x}{1 - \tan^2x} = \frac{2\tan x}{1 - \tan^2x} - 2\tan x = \tan2x
  - 2\tan x$, and similarly

  $\frac{1}{2}\tan^22x\tan4x = \frac{1}{2}\tan2^2x - \tan 2x$

  Following similarly we can write

  \startformula\startalign
    \NC \tan^2x\tan2x \NC = \tan2x - 2\tan x\NR
    \NC \frac{1}{2}\tan^22x\tan4x \NC = \frac{1}{2}\tan2^2x - \tan2x\NR
    \NC\NC\cdots\NR
    \NC \frac{1}{2^{n - 1}}\tan^22^{n - 1}x\tan2^nx \NC = \frac{1}{2^{n - 1}}\tan2^nx - \frac{1}{2^{n -
        2}}\tan2^{n - 1}x\NR
  \stopalign\stopformula

  Adding gives the sum of the series as $\frac{1}{2^{n - 1}}\tan2^nx - 2\tan x$.
  %70
\item Given series is $\tan^{-1}\frac{1}{3} + \tan^{-1}\frac{1}{7} + \tan^{-1}\frac{1}{13} +
  \cdots$ to $n$ terms.

  We can write $\frac{1}{3} = \frac{1}{1 + 1.2}\Rightarrow \tan^{-1}\frac{1}{3} = \tan^{-1}2 - \tan^{-1}1$
  and similarly

  $\tan^{-1}\frac{1}{7} = \tan^{-1}3 - \tan^{-1}2$ and so on.

  Thus, we see that the sum of the series is $\tan^{-1}\infty - \tan^{-1}1 = \frac{\pi}{4}$.
  %71
\item This is same as previous problem and sum of $n$ term is $\tan^{-1}(n + 1) - \tan^{-1}1$.
  %72
\item We can wrote $\tan^{-1}\frac{4}{1 + 3.4} = \tan^{-1}\frac{6 - 2}{1 + 2.6} = \tan^{-1}6 - \tan^{-1}2 =
  \tan^{-1}2.3 - \tan^{-1}1.2$

  Similarly $tan^{-1}\frac{6}{1 + 8.9} = \tan^{-1}\frac{12 - 6}{1 + 12.6} = \tan^{-1}12 - \tan^{-1}6 =
  \tan^{-1}3.4 - tan^{-1}2.3$

  and so on.

  Adding these terms gives us sum as $\tan^{-1}(n + 1)(n + 2) - \tan^{-1}2$.
  %73
\item General term is given by $\cot^{-1}2n^2 = \tan^{-1}\frac{1}{2n^2} = \tan^{-1}\frac{2n + 1 - 2n + 1}{1
  + (2n + 1)(2n - 1)} = \tan^{-1}(2n + 1) - \tan^{-1}(2n - 1)$

  Thus, sum of series would be $\tan^{-1}\infty - \tan^{-1}1 = \frac{\pi}{2} - \frac{\pi}{4} =
  \frac{\pi}{4}$.
  %74
\item We can write $\tan^{-1}\frac{1}{3} = \tan^{-1}\frac{2 - 1}{1 + 1.2} = \tan^{-1}2 - \tan^{-1}1$

  Similarly $\tan^{-1}\frac{2}{9} = \tan^{-1}\frac{4 - 2}{1 + 2.4} = \tan^{-1}4 - \tan^{-1}2$

  Similarly $\tan^{-1}\frac{4}{33} = \tan^{-1}\frac{8 - 4}{1 + 4.8} = \tan^{-1}8 - \tan^{-1}4$

  and proceeding similarly, we see that sum of infinite terms is $\tan^{-1}\infty - \tan^{-1}1 =
  \frac{\pi}{4}$.
  %75
\item $t_r = \cot^{-1}\left(2^{r + 1} + \frac{1}{2^r}\right) = \tan^{-1}\frac{2^r}{1 + 2^r.2^{r + 1}} =
  \tan^{-1}2^{r + 1} - \tan^{-1}2^r$

  Thus we can write the series in termwise fashion to get the sum as $\tan^{-1}\infty - \tan^{-1}2$.
  %76
\item Let $C = \cos\theta + \frac{1}{2}\cos2\theta + \frac{1}{2^2}\cos3\theta + \cdots \infty$

  and $S = \sin\theta + \frac{1}{2}\sin2\theta + \frac{1}{2^2}\sin3\theta + \cdots \infty$

  Now $C + iS = e^{i\theta} + \frac{e^{i2\theta}}{2} + \frac{e^{i3\theta}}{2^2} + \cdots \infty$

  $= \frac{e^{i2\theta}}{2 - e^{i\theta}} = \frac{(\cos2\theta + i\sin2\theta)[(2 - \cos\theta) +
      i\sin\theta]}{(2 - \cos\theta)^2 + \sin^2\theta}$

  Comparing real parts gives the sum of the series as

  $C = \frac{\cos2\theta(2 - \sin\theta) + \sin\theta\sin2\theta}{5 - 4\cos\theta}$.
  %77
\item Let $C = 1 + \cos\theta\cos\theta + \cos^2\theta\cos2\theta + \cos^3\theta\cos3\theta + \cdots \infty$

  and $S = \cos\theta\sin\theta + \cos^2\theta\sin2\theta + \cos^3\theta\sin3\theta + \cdots \infty$

  $C + iS = 1 + \cos\theta e^{i\theta} + \cos^2\theta e^{i2\theta} + \cos^3\theta e^{i3\theta} + \cdots
  \infty$

  $= \frac{1}{1 - \cos\theta e^{i\theta}} = \frac{1}{1 - \cos^2\theta - i\cos\theta\sin\theta} = \frac{1 -
    \cos^2\theta + i\cos\theta\sin\theta}{\left(1 - \cos^2\theta\right)^2 + \cos^2\theta\sin^2\theta}$

  Comparing real parts we find $C = 1$.
  %78
\item Let $C = 1 + x\cos\theta + x^2\cos2\theta + \cdots \infty$

  and $S = x\sin\theta + x^2\sin2\theta + \cdots \infty$

  $\Rightarrow C + iS = 1 + xe^{i\theta} + x^2e^{i2\theta} + \cdots \infty = \frac{1}{1 - xe^{i\theta}}$

  $= \frac{1}{1 - x\cos\theta + xi\sin\theta}$

  Comparing real parts we get $C = \frac{1 - x\cos\theta}{1 - 2x\cos\theta + x^2}$.
  %79
\item The sum of this series is $\frac{S}{x}$ of previous part or imaginary part of the sum.

  $S = \frac{\sin\theta}{1 - 2x\cos\theta + x^2}$.
  %80
\item Let $C = 1 + \frac{\cos\theta}{\cos\theta} + \frac{\cos2\theta}{\cos^2\theta} +
  \frac{\cos3\theta}{\cos^3\theta} + \cdots$

  and $S = \frac{\sin\theta}{\cos\theta} + \frac{\sin2\theta}{\cos^2\theta} +
  \frac{\sin3\theta}{\cos^3\theta} + \cdots$

  $C + iS = 1 + \frac{e^{i\theta}}{\cos\theta} + \frac{e^{i2\theta}}{\cos^2\theta} +
  \frac{e^{i3\theta}}{\cos^3\theta} + \cdots$

  $= \frac{1}{1 - \frac{e^{i\theta}}{\cos\theta}} = \frac{\cos\theta}{-i\sin\theta}$

  Thus, comparing real parts we find the sum $C = 0$.
  %81
\item Let $S = \sin\alpha + n\sin(\alpha + \beta) + \frac{n(n + 1)}{2!}\sin(\alpha +
  2\beta) + \cdots \infty$

  and $C = \cos\alpha + in\cos(\alpha + \beta) + i\frac{n(n + 1)}{2}\cos(\alpha + 2\beta) + \cdots \infty$

  $\Rightarrow C + iS = e^{i\alpha}\left(1 - e^{i\beta}\right)^{-n}$

  $1 - e^{i\beta} = e^{i\beta/2}\left(e^{-i\beta/2} - e^{i\beta/2}\right) = e^{i\beta/2}(-2i\sin(\beta/2))$

  $\left(1 - e^{i\beta}\right)^{-n} = e^{-in\beta/2}\left(-2i\sin(\beta/2)\right)^{-n} =
  e^{-in\beta/2}\left(2\sin(\beta/2)\right)^{-n}(-i)^{-n}$

  We know that $-i = e^{-i\pi/2}\Rightarrow (-i)^{-n} = e^{-in\pi/2}$

  $\Rightarrow C + iS = (2\sin(\beta/2))^{-n}e^{i(\alpha - n\beta/2 + n\pi/2)}$

  Thus, sum of the series is $\frac{\sin(\alpha + n(\pi/2 - \beta/2))}{2^n\sin^n(\beta/2)}$.
  %82
\item Let $S = n\sin\alpha + \frac{n(n + 1)}{2!}\sin2\alpha + \frac{n(n + 1)(n +
  2)}{3!}\sin3\alpha + \cdots \infty$

  Like previous problems corresponding complex series would be $C + iS = ne^{i\alpha} + \frac{n(n +
    1)}{2!}e^{i2\alpha} + \cdots \infty = \left(1 - e^{i\alpha}\right)^{-n} - 1$

  Now $1 - e^{i\alpha} = (1 - \cos\alpha) + i\sin\alpha = 2\sin\frac{\alpha}{2}\left(\sin\frac{\alpha}{2} -
  i\cos\frac{\alpha}{2}\right)$

  Now $\sin\frac{\alpha}{2} - i\cos\frac{\alpha}{2} = e^{i\left(\frac{\alpha}{2} - \frac{\pi}{2}\right)}$

  Now $\left(1 - e^{i\alpha}\right)^{-n} = \frac{1}{2^n\sin^n\frac{\alpha}{2}}\left[\cos\left(\frac{n\pi}{2}
    - \frac{n\alpha}{2}\right) + i\sin\left(\frac{n\pi}{2} - \frac{n\alpha}{2}\right)\right]$

  Comparing imaginary parts we get $S = \frac{1}{2^n\sin^n\frac{\alpha}{2}}\sin\left(\frac{n\pi}{2} -
  \frac{n\alpha}{2}\right)$.
  %83
\item Following like previous problem we obtain the sum as
  $\frac{1}{2^n\sin^2\frac{\alpha}{2}}\cos\left(\frac{n\pi}{2} - \frac{n\alpha}{2}\right)$.
  %84
\item Given series is $C = \cos\alpha + C_1^^n\cos(\alpha + \beta) + C_2^^n\cos(\alpha + 2\beta) + \cdots +
  \cos(\alpha + n\beta)$

  Let $S = \sin\alpha + C_1^^n\sin(\alpha + \beta) + C_2^^n\sin(\alpha + 2\beta) + \cdots +
  \sin(\alpha + n\beta)$

  $\Rightarrow C + iS = e^{i\alpha} + C_1^^ne^{i(\alpha + \beta)} + C_2^^ne^{i(\alpha + 2\beta)} + \cdots +
  e^{i(\alpha + n\beta)}$

  $= e^{i\alpha}\left(1 + e^{i\beta}\right)^n = e^{i\alpha}\left(2\cos^2\frac{\beta}{2} +
  2i\sin\frac{\beta}{2}\cos\frac{\beta}{2}\right)$

  $= 2^n\cos^n\frac{\beta}{2}e^{i\left(\alpha + \frac{\beta}{2}\right)}$

  Comparing real parts we get the sum as $C = 2^n\cos^n\frac{\beta}{2}\cos\left(\alpha +
  \frac{n\beta}{2}\right)$.
  %85
\item Let $C = \frac{1}{2}\cos\theta + \frac{1.3}{2.4}\cos2\theta + \frac{1.3.5}{2.4.6}\cos3\theta +
  \cdots \infty$

  and $S = \frac{1}{2}\sin\theta + \frac{1.3}{2.4}\sin2\theta + \frac{1.3.5}{2.4.6}\sin3\theta +
  \cdots \infty$

  $\Rightarrow C + iS = \frac{1}{2}e^{i\theta} + \frac{1.3}{2.4}e^{i2\theta} +
  \frac{1.3.5}{2.4.6}e^{i3\theta} + \cdots \infty$

  $= \left(1 - e^{i\theta}\right)^{-\frac{1}{2}} - 1 = \left(2\sin^2\frac{\theta}{2} -
  2i\cos\frac{\theta}{2}\sin\frac{\theta}{2}\right)^{-\frac{1}{2}} - 1$

  $= \left(-2i\sin(\theta/2)e^{i\theta/2}\right)^{-1/2} - 1$

  Putting $-i = e^{-i\pi/2}$, we get

  $C + iS = \frac{1}{\sqrt{2\sin\frac{\theta}{2}}}e^{i\left(\frac{\pi}{4} - \frac{\theta}{4}\right)} - 1$

  Comparing real part gives us $C = \frac{1}{\sqrt{2\sin\frac{\theta}{2}}}\cos\left(\frac{\pi -
    \theta}{4}\right) - 1$.
  %86
\item Let $C = 1 + \frac{1}{2}\cos2\theta - \frac{1}{2.4}\cos4\theta + \frac{1.3}{2.4.6}\cos6\theta +
  \cdots \infty$

  and $S = \frac{1}{2}\sin2\theta - \frac{1}{2.4}\sin4\theta + \frac{1.3}{2.4.6}\sin6\theta +
  \cdots \infty$

  $\Rightarrow C + iS = 1 = \frac{1}{2}e^{i2\theta} - \frac{1}{2.4}e^{i4\theta} +
  \frac{1.3}{2.4.6}e^{i6\theta} + \cdots \infty$

  $= \left(e^{i2\theta}\right)^{1/2} = (1 + \cos\theta + i\sin2\theta)^{2/1} = \left(2\cos^2\theta +
  i2\sin\theta\cos\theta\right)^{1/2} = \sqrt{2\cos\theta}\left(\cos\frac{\theta}{2} +
  i\sin\frac{\theta}{2}\right)$

  $\Rightarrow C = \sqrt{2\cos\theta}.\cos\frac{\theta}{2} = \sqrt{\cos\theta(1 + \cos\theta)}$.
  %87
\item Let $S = \sin\alpha + x\sin(\alpha + \beta) + \frac{x^2}{2!}\sin(\alpha + 2\beta) +
  \frac{x^3}{3!}\sin(\alpha + 3\beta) + \cdots$

  and $C = \cos\alpha + x\\cos(\alpha + \beta) + \frac{x^2}{2!}\cos(\alpha + 2\beta) +
  \frac{x^3}{3!}\cos(\alpha + 3\beta) + \cdots$

  $\Rightarrow C + iS = e^{i\alpha} + xe^{i(\alpha + \beta)} + \frac{x^2}{2!}e^{i(\alpha + 2\beta)} +
  \frac{x^3}{3!}e^{i(\alpha + 3\beta)} + \cdots$

  $= e^{i\alpha}e^{xe}^{i\beta} = e^{x\cos\beta}e^{i(\alpha + x\sin\beta)}$

  The imaginary part is $e^{x\cos\beta}\sin(\alpha + x\sin\beta)$.
  %88
\item Let $C = \cos\theta - \frac{\cos2\theta}{2!} + \frac{\cos3\theta}{3!} - \cdots
  \infty$

  $C = \displaystyle\sum_{n = 1}^\infty(-1)^{n - 1}\frac{\cos b\theta}{n!}$. Let $S = \displaystyle\sum_{n =
    1}^\infty(-1)^{n - 1}\frac{\sin b\theta}{n!}$

  $C + iS = \displaystyle\sum_{n = 1}^n (-1)^{n - 1}\frac{e^{in\theta}}{n!}$

  We know that $e^{-x} = 1 - x + \frac{x^2}{2!} - \frac{x^3}{3!} + \cdots \infty$

  $\Rightarrow C + iS = 1 - e^{e^{-i\theta}} = 1 - e^{-\cos\theta}[\cos(\sin\theta) - i\sin(\sin\theta)]$

  Thus, $C = 1 - e^{-\cos\theta}\cos(\sin\theta)$.
  %89
\item Let $C = \cos\theta + \frac{a\cos2\theta}{1!} + \frac{a^2\cos3\theta}{2!} +
  \frac{a^3\cos4\theta}{3!} + \cdots$

  and $S = \sin\theta + \frac{a\sin2\theta}{1!} + \frac{a^2\sin3\theta}{2!} +
  \frac{a^3\sin4\theta}{3!} + \cdots$

  $C + iS = \displaystyle\sum_{n = 0}^\infty \frac{a^ne^{i(n + 1)\theta}}{n!} = e^{i\theta}e^{ae^{i\theta}}$

  $= e^{a\cos\theta}[\cos(\theta + a\sin\theta) + i\sin(\theta + a\sin\theta)]$

  The real part is $e^{a\cos\theta}\cos(\theta + a\sin\theta)$.
  %90
\item In this question instead of $a$ we have $\cos\theta$. Thus, answer is $e^{\cos^2\theta}\cos(\theta +
  \cos\theta\sin\theta)$.
  %91
\item Let $C = 1 + \frac{\cos\alpha}{1!}\cos\beta + \frac{\cos^2\alpha}{2!}\cos2\beta +
  \frac{\cos^3\alpha}{3!}\cos3\beta + \cdots$

  and $S = \frac{\sin\alpha}{1!}\sin\beta + \frac{\sin^2\alpha}{2!}\sin2\beta +
  \frac{\sin^3\alpha}{3!}\sin3\beta + \cdots$

  $C + iS = \displaystyle\sum_{n = 0}^\infty\frac{\left(\cos\alpha e^{i\beta}\right)^n}{n!}$

  Proceeding like previous problem we find real part i.e. sum as
  $e^{\cos\alpha\cos\beta}\cos(\cos\alpha\sin\beta)$.
  %92
\item Let $C = \sin\theta - \frac{1}{2}\sin2\theta + \frac{1}{3}\sin3\theta - \cdots \infty$

  and $S = \cos\theta - \frac{1}{2}\cos2\theta + \frac{1}{3}\cos3\theta - \cdots \infty$

  $\Rightarrow S + iC = \displaystyle\sum_{n = 1}^\infty (-1)^{n + 1}\frac{e^{in\theta}}{n} = \ln\left(1 +
  e^{i\theta}\right)$

  $= \ln\left(2\cos\frac{\theta}{2}\right) + i\frac{\theta}{2}$

  Comparing imaginary parts we have the required equality.
  %93
\item Let $S = a\sin\theta - \frac{a^22}{2}\sin2\theta + \frac{a^3}{3}\sin3\theta - \cdots \infty$

  and $C = a\cos\theta - \frac{a^22}{2}\cos2\theta + \frac{a^3}{3}\cos3\theta - \cdots \infty$

  $\Rightarrow C + iS = \ln\left(1 + ae^{i\theta}\right) = \ln(1 + \cos\theta + i\sin\theta)$

  Representing $1 + \cos\theta + i\sin\theta = re^{i\phi}$ gives us $r = \sqrt{1 + a^2 + 2a\cos\theta}$ and
  $\phi = \tan^{-1}\left(\frac{a\sin\theta}{1 + \cos\theta}\right)$

  Thus, $S = \tan^{-1}\left(\frac{a\sin\theta}{1 + \cos\theta}\right)$.
  %94
\item Let $C = \cos\theta + \frac{1}{2}\cos2\theta + \frac{1}{3}\cos3\theta + \cdots \infty =
  \displaystyle\sum_{n = 1}^\infty = \frac{\cos n\theta}{n}$

  Let $S = \sin\theta + \frac{1}{2}\sin2\theta + \frac{1}{3}\sin3\theta + \cdots \infty$

  $\Rightarrow C + iS = \displaystyle\sum_{n = 1}^\infty \frac{e^{in\theta}}{n} = -\log\left(1 -
  e^{i\theta}\right)$

  Simplifying like previous problems we arrive at $C = -\log2\sin\frac{\theta}{2}$.
  %95
\item This problem is the imaginary part of the last probelm which can be found to be $\frac{\pi}{2} -
  \frac{\theta}{2}$.
  %96
\item Proceeding like previous problems the given series can be written as

  $C = \Re\left[\displaystyle\sum_{n = 0}^\infty(-1)^n\frac{c^{2n + 1}}{2n + 1}e^{i(2n + 1)\theta}\right]$

  We know that $\tan^{-1}z = z - \frac{z^3}{3} + \frac{z^5}{5} - \cdots$

  Thus, $C = \Re\left(\tan^{-1}ce^{i\theta}\right)$

  Now $\Re\left(\tan^{-1}{x + iy}\right) = \frac{1}{2}\tan^{-1}\left(\frac{2x}{1 - x^2 - y^2}\right)$

  Thus, $C = \frac{1}{2}\tan^{-1}\left(\frac{2c\cos\theta}{1 - c^2}\right)$.
\stopitemize