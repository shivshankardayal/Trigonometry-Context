% -*- mode: context; -*-
\chapter{Miscellaneous Problems}
\startitemize[n, 1*broad]
% 1
\item L.H.S.\ $= \frac{\sqrt{1 + \cos x} + \sqrt{1 - \cos x}}{\sqrt{1 + \cos x} - \sqrt{1 - \cos x}}$

  $= \frac{\sqrt{2\cos^2\frac{x}{2}} + \sqrt{2\sin^2\frac{x}{2}}}{\sqrt{2\cos^2\frac{x}{2}}
  - \sqrt{2\sin^2\frac{x}{2}}} = \frac{\left|\cos\frac{x}{x}\right|
  + \left|\sin\frac{x}{x}\right|}{\left|\cos\frac{x}{x}\right| - \left|\sin\frac{x}{x}\right|}$

  $= \frac{-\cos\frac{x}{2} + \sin\frac{x}{2}}{-\cos\frac{x}{2} - \sin\frac{x}{2}}[\because \pi < x <
  2\pi\Rightarrow \frac{\pi}{2} < \frac{x}{2} < \pi]$

  $= \frac{\cot\frac{x}{2} - 1}{\cot\frac{x}{2} + 1} = \cot\left(\frac{x}{2} + \frac{\pi}{4}\right)$.
  %2
\item $2A = A + B + A - B \Rightarrow \tan2A = \frac{\tan(A + B) + \tan(A - B)}{1 - \tan(A + B)\tan(A - B)}$

  From question $0 < A < \frac{\pi}{4}$ anda $0 < B < \frac{\pi}{4} \Rightarrow 0 < A + B < \frac{\pi}{2}$

  Also, $-\frac{\pi}{4} < A - B < \frac{\pi}{4}$, however, $\sin(A - B) = \frac{1}{\sqrt{10}}\Rightarrow 0 <
  A - B < \frac{\pi}{4}$

  $\Rightarrow \tan(A - B) = \frac{1}{3}, \cos(A + B) = \frac{2}{\sqrt{29}}\Rightarrow \tan(A + B)
  = \frac{5}{2}$

  $\Rightarrow \tan2A = \frac{\frac{5}{2} + \frac{1}{3}}{1 - \frac{5}{2}.\frac{1}{3}} = 17$.
  %3
\item Given that $\sin^3x\sin3x = \displaystyle\sum_{m = 0}^nc_m\cos mx \Rightarrow \frac{3\sin x
  - \sin3x}{4}\sin3x = \displaystyle\sum_{m = 0}^nc_m\cos mx$

  $\Rightarrow \frac{3}{8}(2\sin3x.\sin x) - \frac{1}{8}2\sin^23x = \displaystyle\sum_{m = 0}^nc_m\cos
  mx \Rightarrow \frac{3}{8}[\cos2x - \cos4x] - \frac{1}{8}[1 - \cos6x] = \displaystyle\sum_{m = 0}^nc_m\cos
  mx$

  $\Rightarrow -\frac{1}{8} + \frac{3}{8}\cos2x - \frac{3}{8}\cos4x + \frac{1}{8}\cos6x
  = \displaystyle\sum_{m = 0}^nc_m\cos mx$

  Comparing we get $n = 6$.
  %4
\item L.H.S. $= \left[\sin^4\frac{\pi}{8} + \sin^4\frac{7\pi}{8}\right] + \left[\sin^4\frac{3\pi}{8}
  + \sin^4\frac{5\pi}{8}\right]$

  $= \left[\sin^4\frac{\pi}{8} + \sin^4\left(\pi - \frac{\pi}{8}\right)\right] + \left[\sin^4\frac{3\pi}{8}
  + \sin^4\left(\pi - \frac{3\pi}{8}\right)\right]$

  $= 2\sin^4\frac{\pi}{8} + 2\sin^4\frac{3\pi}{8} = 2\left[\left(\sin^2\frac{\pi}{8}\right)^2
  + \left(\sin^2\frac{3\pi}{8}\right)^2\right]$

  $= 2\left[\left(\frac{1 - \cos\frac{\pi}{4}}{2}\right)^2 + \left(\frac{1
    - \cos\frac{3\pi}{4}}{2}\right)^2\right] = \frac{1}{2}\left[\left(1 - \frac{1}{\sqrt{2}}\right)^2
  + \left(1 + \frac{1}{\sqrt{2}}\right)^2\right] = \frac{3}{2} =$ R.H.S.
  %5
\item $\displaystyle\sum_{q = 1}^{10}\left(\sin\frac{2q\pi}{11} - i\cos\frac{2q\pi}{11}\right)$

  $= \sin\frac{2\pi}{11} + \sin\frac{4\pi}{11} + \sin\frac{6\pi}{11} + \cdots$ to $10$ terms
  $- i\cos\frac{2\pi}{11} + \cos\frac{4\pi}{11} + \cos\frac{6\pi}{11} + \cdots$ to $10$ terms

  $= \frac{\sin10\frac{\pi}{11}}{\sin\frac{\pi}{11}}\sin\left[\frac{2\pi}{11} + (10 -
  1)\frac{\pi}{11}\right] - i\frac{\sin10\frac{\pi}{11}}{\sin\frac{\pi}{11}}\cos\left[\frac{2\pi}{11} + (10
  - 1)\frac{\pi}{11}\right]$

  $= \frac{\sin\left(\pi - \frac{\pi}{11}\right)}{\sin\frac{\pi}{11}}.\sin\pi - i\frac{\sin\left(\pi
  - \frac{\pi}{11}\right)}{\sin\frac{\pi}{11}}.\cos\pi = i$

  $\displaystyle\sum_{p = 1}^{32}(3p + 2)\left[\displaystyle\sum_{q = 1}^{10}\sin\frac{2q\pi}{11} -
  i\cos\frac{2q\pi}{11}\right]^p = \sum_{p=1}^{32}(3p + 2)i^p = 5i + 8i^2 + 11i^3 + \cdots$ upto $32$ terms

  $= 5i - 8 - 11i + 14 + 17i - 20 + \cdots$ upto $32$ terms

  $= (5i - 11i + 17i - 23i + \cdots)$ upto $16$ temrs $+ (-8 + 14 - 20 + 26 - \cdots)$ upto $16$ terms

  $= 48(1 - i)$.
  %6
\item Let $x = 2^n\cos\theta\cos2\theta\cos2^2\theta\ldots\cos2^{n - 1}\theta$

  $x\sin\theta = 2^n(\sin\theta\cos\theta)\cos2\theta\cos2^2\theta\ldots\cos2^{n - 1}\theta = 2^{n -
  1}\sin2\theta\cos2\theta\cos2^2\theta\ldots\cos2^{n - 1}\theta$

  Proceeding similarly, $x\sin\theta = \sin2^n\theta$. Given than $\theta = \frac{\pi}{2^n - 1} \Rightarrow
  2^n\theta = \pi + \theta$

  $\Rightarrow x\sin\theta = \sin(\pi + \theta) \Rightarrow x = -1$.
  %7
\item Let $x = \cos\frac{2\pi}{7}\cos\frac{4\pi}{7}\cos\frac{6\pi}{7} = \frac{1}{8}
  = \cos\frac{2\pi}{7}\cos\frac{4\pi}{7}\cos\left(\pi - \frac{\pi}{7}\right)$

  $= -\cos\frac{\pi}{7}\cos\frac{2\pi}{7}\cos\frac{4\pi}{7}$

  Multiplying both sides with $\sin\frac{\pi}{7}$ gives us

  $x\sin\frac{\pi}{7} = -\frac{1}{2}.2\sin\frac{\pi}{7}\cos\frac{\pi}{7}\cos\frac{2\pi}{7}\cos\frac{4\pi}{7}
  = -\frac{1}{2}\sin\frac{2\pi}{7}\cos\frac{2\pi}{7}\cos\frac{4\pi}{7}$

  $= -\frac{1}{4}\sin\frac{4\pi}{7}\cos\frac{4\pi}{7} = -\frac{1}{8}\sin\frac{8\pi}{7} =
  -\frac{1}{8}\sin\left(\pi + \frac{\pi}{7}\right)$

  $x\sin\frac{\pi}{7} = \frac{1}{8}.\sin\frac{\pi}{7}\Rightarrow x = \frac{1}{8}$.
  %8
\item Let $y = \sin\frac{2\pi}{7} + \sin\frac{4\pi}{7} + \sin\frac{8\pi}{7} = \frac{\sqrt{7}}{2}$

  Squaring both sides $y^2 = \sin^2\frac{2\pi}{7} + \sin^2\frac{4\pi}{7} + \sin^2\frac{8\pi}{7} +
  2\sin\frac{2\pi}{7}\sin\frac{4\pi}{7} + 2\sin\frac{4\pi}{7}\sin\frac{8\pi}{7} +
  2\sin\frac{2\pi}{7}\sin\frac{8\pi}{7}$

  Let $y_1 = \sin^2\frac{2\pi}{7} + \sin^2\frac{4\pi}{7} + \sin^2\frac{8\pi}{7}$ and $y_2 =
  2\sin\frac{2\pi}{7}\sin\frac{4\pi}{7} + 2\sin\frac{4\pi}{7}\sin\frac{8\pi}{7} +
  2\sin\frac{2\pi}{7}\sin\frac{8\pi}{7}$

  $\Rightarrow y^2 = y_1 + y_2$. Now

  $y_1 = \frac{1 - \cos\frac{4\pi}{7}}{2} + \frac{1 - \cos\frac{8\pi}{7}}{2} + \frac{1
    - \cos\frac{16\pi}{7}}{2}$

  $= \frac{3}{2} - \frac{1}{2}\left[\cos\frac{2\pi}{7} + \cos\frac{4\pi}{7} + \cos\frac{6\pi}{7}\right]$

  Let $z = \cos\frac{2\pi}{7} + \cos\frac{4\pi}{7} + \cos\frac{6\pi}{7}$

  Multiplying both sides by $2\sin\frac{\pi}{7}$ gives is

  $2z\sin\frac{\pi}{7} = 2\sin\frac{\pi}{7}\left(\cos\frac{2\pi}{7} + \cos\frac{4\pi}{7}
  + \cos\frac{6\pi}{7}\right)$

  $= \sin\frac{3\pi}{7} - \sin\frac{\pi}{7} + \sin\frac{5\pi}{7} - \sin\frac{3\pi}{7} + \sin\frac{7\pi}{7} -
  \sin\frac{5\pi}{7}$

  $2z\sin\frac{\pi}{7} = \sin\pi - \sin\frac{\pi}{7} \Rightarrow z = -\frac{1}{2}\Rightarrow y_1
  = \frac{7}{4}$

  Now $y_2 = 2\sin\frac{2\pi}{7}\sin\frac{4\pi}{7} + 2\sin\frac{4\pi}{7}\sin\frac{8\pi}{7} +
  2\sin\frac{2\pi}{7}\sin\frac{8\pi}{7}$

  $= \cos\frac{2\pi}{7} - \cos\frac{6\pi}{7} + \cos\frac{4\pi}{7} - \cos\frac{12\pi}{7} + \cos\frac{6\pi}{7}
  - \cos\frac{10\pi}{7} = 0$.

  Hence proven.
  %9
\item We can rewrite given equation as L.H.S.\ $= \tan^2\frac{\pi}{16} + \tan^2\frac{7\pi}{16}
  + \tan^2\frac{2\pi}{16} + \tan^2\frac{6\pi}{16} + \tan^2\frac{3\pi}{16} + \tan^2\frac{5\pi}{16}
  + \tan^2\frac{4\pi}{16}$

  $= \left(\tan^2\frac{\pi}{16} + \cot^2\frac{\pi}{16}\right) + \left(\tan^2\frac{2\pi}{16}
  + \cot^2\frac{2\pi}{16}\right) + \left(\tan^2\frac{3\pi}{16} + \cot^2\frac{3\pi}{16}\right) +
  1[\because \tan\frac{7\pi}{16} = \tan\left(\frac{\pi}{2} - \frac{\pi}{16}\right) = \cot\frac{\pi}{16}]$

  $= \left(\tan\frac{\pi}{16} + \cot\frac{\pi}{16}\right)^2 + \left(\tan\frac{2\pi}{16}
  + \cot\frac{2\pi}{16}\right)^2 + \left(\tan\frac{3\pi}{16} + \cot\frac{3\pi}{16}\right)^2 - 2 - 2 - 2 +
  1[\because (\tan\theta + \cot\theta)^2 = \tan^2\theta + \cot^2\theta - 2\tan\theta\cot\theta]$

  $= \left(\frac{1}{\sin\frac{\pi}{16}\cos\frac{\pi}{16}}\right)^2
  + \left(\frac{1}{\sin\frac{2\pi}{16}\cos\frac{2\pi}{16}}\right)^2
  + \left(\frac{1}{\sin\frac{3\pi}{16}\cos\frac{3\pi}{16}}\right)^2 - 5$

  $= \frac{4}{\sin^2\frac{\pi}{8}} + \frac{4}{\sin^2\frac{\pi}{4}} + \frac{4}{\sin^2\frac{3\pi}{8}} - 5$

  $= \frac{4}{\sin^2\frac{\pi}{8}} + \frac{4}{\sin^2\frac{3\pi}{8}} + 4.2 - 5$

  $= \frac{4}{\sin^2\frac{\pi}{8}} + \frac{4}{\cos^2\frac{\pi}{8}} + 3
  = \frac{4}{\sin^2\frac{\pi}{8}\cos^2\frac{\pi}{8}} + 3 = \frac{16}{\sin^2\frac{\pi}{4}} + 3 = 35$.
  %10
\item Let $\theta = \frac{\pi}{7} \Rightarrow 7\theta = \pi \Rightarrow 4\theta + 3\theta = \pi$

  $\Rightarrow \tan4\theta = \tan(\pi - 3\theta) \Rightarrow \frac{4\tan\theta - 4\tan^3\theta}{1 -
  6\tan^2\theta + \tan^4\theta} = -\frac{3\tan\theta - \tan^3\theta}{1 - 3\tan^2\theta}$

  Let $\tan\theta = z\Rightarrow \frac{4z - 4z^3}{1 - 6z^2 + z^4} = -\frac{\3z - z^3}{1 - 3z^2}$

  $\Rightarrow z^6 - 21z^4 + 35z^2 - 7 = 0$

  This is a cubic equation in $z^2$ i.e. $\tan^2\theta$, therefore, the roots of the equation are
  $\tan^2\frac{\pi}{7}, \tan^2\frac{2\pi}{7}$ and $\tan^2\frac{3\pi}{7}$.

  From Vieta's relattionships the product of roots gives us
  $\tan\frac{\pi}{7}\tan\frac{2\pi}{7}\tan\frac{3\pi}{7} = \sqrt{7}$.
  %11
\item We have obtained the equation $z^6 - 21z^4 + 35z^2 - 7 = 0$ whose roots are
  $\tan^2\frac{\pi}{7}, \tan^2\frac{2\pi}{7}$ and $\tan^2\frac{3\pi}{7}$.

  Thus, sum of roots is $35$. Putting $z = \frac{1}{y}$ gives us $-7y^6 + 35y^4 - 21y^2 + 1 = 0$ whose roots
  would be $\cot^2\frac{\pi}{7}, \cot^2\frac{2\pi}{7}$ and $\cot^2\frac{3\pi}{7}$.

  Sum of these is $\frac{-35}{-7} = 3$.

  Thus, $\left(\tan^2\frac{\pi}{7} + \tan^2\frac{2\pi}{7} + \tan^2\frac{3\pi}{7}\right)
  + \left(\cot^2\frac{\pi}{7} + \cot^2\frac{2\pi}{7} + \cot^2\frac{3\pi}{7}\right) = 105$.
  %12
\item L.H.S.\ $= \sqrt{\tan x + \sin x} + \sqrt{\tan x - \sin x} = \sqrt{\tan x}(\sqrt{1 + \cos x} + \sqrt{1
  - \cos x})$

  $= \sqrt{2\tan x}\left(\cos\frac{x}{2} + \sin\frac{x}{2}\right) = 2\sqrt{\tan
  x}\left(\frac{1}{\sqrt{2}}\cos x + \frac{1}{\sqrt{2}}\sin\frac{x}{2}\right)$

  $= 2\sqrt{\tan x}\cos\left(\frac{\pi}{4} - \frac{x}{2}\right)[\because 0 < x < \frac{\pi}{2}]$.
  %13
\item We know that $\sin3x = 3\sin x - 4\sin^3x$, which makes L.H.S.\ $= \sin^3x\left(3\sin x -
  4\sin^3x\right)$

  $= 3\sin^4x - 4\sin^6x = 3\left(1 - \cos^2x\right)^2 - \left(1 - \sin^2x\right)^3$

  Thus, coefficient of $\cos^4x$  or $c_4$ is $-9$.
  %14
\item We have to find the value of $\sin7^\circ + \sin77^\circ + \sin293^\circ + \sin149^\circ
  + \sin221^\circ$

  $= \sin5^\circ + 2\sin185^\circ\cos108^\circ + 2\sin185^\circ\cos36^\circ$

  $= \sin5^\circ - 2\sin5^\circ\left(\cos108^\circ + \cos36^\circ\right) = \sin5^\circ -
  4\sin5^\circ\left(\cos72^\circ\cos36^\circ\right)$

  $= 2\sin5^\circ - 4\sin5^\circ\left[\frac{\sqrt{5} - 1}{4}.\frac{\sqrt{5 + 1}}{4}\right] = 0$.
  %15
\item Consider the complex sum $Z = \displaystyle\sum_{k = 1}^{n - 1}e^{ik\pi/n}$, then real part of this
  sum is given series, which we let as $S$.

  $Z = \frac{e^{i\pi/n} - e^{i\pi}}{1 - e^{i\pi/n}} = \frac{e^{i\pi/n} + 1}{1 - e^{i\pi/n}} =
  i\cot\frac{\pi}{2n}$, which is purely imaginary number. Thus, $S = 0$.
  %16
\item Let $\omega = e^{i2\pi/7} = \cos\frac{2\pi}{7} + i\sin\frac{2\pi}{7} \Rightarrow \omega^7 = 1$

  $\Rightarrow (\omega - 1)\left(\omega^6 + \omega^5 + \omega^4 + \omega^3 + \omega^2 + \omega + 1\right) =
  0$

  $\because \omega\neq 1 \Rightarrow \omega^6 + \omega^5 + \omega^4 + \omega^3 + \omega^2 + \omega + 1 = 0$

  $S = \cos\frac{2\pi}{7} + \cos\frac{4\pi}{7} + \cos\frac{6\pi}{7} = \frac{\omega + \omega^{-1}}{2}
  + \frac{\omega^2 + \omega^{-2}}{2} + \frac{\omega^3 + \omega^{-3}}{2}$

  $= \frac{1}{2}\left(\omega^6 + \omega^5 + \omega^4 + \omega^3 + \omega^2 + \omega\right) = -\frac{1}{2}$.
  %17
\item We see that $\sin\left(\frac{3\pi}{2} - \alpha\right) = -\cos\alpha, \sin(3\pi + \alpha) = \sin(\pi
  + \alpha) = -\sin\alpha,$

  $\sin\left(\frac{\pi}{2} + \alpha\right) = \cos\alpha,$ and $\sin(5\pi - \alpha) = \sin(\pi - \alpha)
  = \sin\alpha$

  Thus, L.H.S. $= 3\left[\cos^4\alpha + \sin^4\alpha\right] - 2\left[\cos^6\alpha + \sin^6\alpha\right]$

  $ 3\left(1 - \sin^2\alpha\cos^2\alpha\right) - 2\left(1 - 3\cos^2\alpha\sin^2\alpha\right) = 1$.
  %18
\item $\sin36^\circ\sin72^\circ\sin108^\circ\sin144^\circ = \sin36^\circ\sin72^\circ\sin72^\circ\sin36^\circ
  = \left(\sin36^\circ\sin72^\circ\right)^2$

  $= \frac{1}{4}\left(\cos(36^\circ - 72^\circ) - \cos(36^\circ + 72^\circ)\right)^2
  = \frac{1}{4}\left(\cos36^\circ + \sin18^\circ\right)^2$

  $= \frac{1}{4}\left(\frac{\sqrt{5 + 1}}{4} + \frac{\sqrt{5} - 1}{4}\right)^2 = \frac{5}{16}$.
  %19
\item We have to prove that $\sin^212^\circ + \sin^221^\circ + \sin^239^\circ + \sin^248^\circ = 1
  + \sin^29^\circ + \sin^218^\circ$

  We know that $\sin^2\theta = \frac{1 - \cos2\theta}{2}$, which transforma above to

  $2 - \cos24^\circ - \cos42^\circ - \cos78^\circ - \cos96^\circ = 2 - \cos18^\circ - \cos36^\circ$

  Now, $\cos24^\circ + \cos42^\circ + \cos78^\circ + \cos96^\circ = 2\cos60^\circ\cos36^\circ +
  2\cos60^\circ\cos18^\circ$

  $= \cos36^\circ + \cos18^\circ$. Hence proven.
  %20
\item In the given range, the cosine function is non-negative. For $135^\circ\leq \alpha/2\leq 180^\circ$
  sine function is positive and negative in $180^\circ\leq \alpha/2 \leq 225^\circ$.

  Thus, $\sin\frac{\alpha}{2} = \frac{1}{2}\left[\sqrt{1 - \sin\alpha} - \sqrt{1 + \sin\alpha}\right]$

  $\cos\frac{\alpha}{2} = -\frac{1}{2}\left[\sqrt{1 + \sin\alpha} + \sqrt{1 - \sin\alpha}\right]$.
  %21
\item We know that $\tan142^\circ30' = \frac{1 - \cos285^\circ}{\sin285^\circ}$

  $\sin285^\circ = -\sin75^\circ = \sin\left(45^\circ + 30^\circ\right) = -\frac{\sqrt{6} + \sqrt{2}}{4}$

  $\cos285^\circ = \cos75^\circ = \cos\left(45^\circ + 30^\circ\right) = \frac{\sqrt{6} - \sqrt{2}}{4}$

  $\Rightarrow \tan142^\circ30' = \frac{1 - \frac{\sqrt{6} - \sqrt{2}}{4}}{-\frac{\sqrt{6} + \sqrt{2}}{4}}$

  $= \frac{\sqrt{6} - \sqrt{2} - \sqrt{4}}{\sqrt{6} + \sqrt{2}}$

  Rationalizing gives us $\frac{\sqrt{6} - \sqrt{2} - \sqrt{4}}{\sqrt{6} + \sqrt{2}}.\frac{\sqrt{6}
    - \sqrt{2}}{\sqrt{6} - \sqrt{2}}$

  On simplification we get the desired result.
  %22
\item Let $\tan\frac{\alpha}{2} = t$, then $\frac{2t}{1 + t^2} + \frac{1 - t^2}{1 + t^2}
  = \frac{\sqrt{7}}{2}$

  $\Rightarrow \left(\sqrt{7} + 2\right)t^2 - 4t + \left(\sqrt{7} - 2\right) = 0$

  $\Rightarrow D = \sqrt{16 - 4.3} = \pm2$

  $t = \frac{2\pm 1}{\sqrt{7} + 2}$.
  %23
\item Given that $\frac{\sin(x - \alpha)}{\sin(x - \beta)} = \frac{a}{b} \Rightarrow b\sin(x - \alpha) =
  a\sin(x - \beta)$

  $\Rightarrow b(\sin x\cos\alpha - \cos x\sin\alpha) = a(\sin x\cos\alpha - \cos x\sin\alpha)$

  $\Rightarrow \tan x = \frac{b\sin\alpha - a\sin\beta}{b\cos\alpha - a\cos\beta}$

  Similarly from $\frac{\cos(x - \alpha)}{\cos(x - \beta)} = \frac{A}{B}$ we get

  $\tan x = \frac{A\cos\beta - B\cos\alpha}{B\sin\alpha - A\sin\beta}$

  $\Rightarrow \frac{b\sin\alpha - a\sin\beta}{b\cos\alpha - a\cos\beta} = \frac{A\cos\beta -
    B\cos\alpha}{B\sin\alpha - A\sin\beta}$

  Cross-multiplying and simplifying gives us desired relation.
  %24
\item Since $\cot\alpha, \cot\beta, \cot\gamma$ are in A.P. $\Rightarrow 2\cot\beta = \cot\alpha
  + \cot\beta$

  $\cot(\alpha + \beta + \gamma) = \frac{\pi}{2}\Rightarrow \frac{\cot\alpha\cot\beta\cot\gamma - \cot\alpha
    - \cot\beta - \cot\gamma}{\cot\alpha\cot\beta + \cot\beta\cot\gamma + \cot\gamma\cot\alpha - 1}  = 0$

  $\Rightarrow \cot\alpha\cot\beta\cot\gamma = \cot\alpha + \cot\beta + \cot\gamma = 3\cot\beta$

  $\Rightarrow \cot\beta(\cot\alpha\cot\gamma - 3) = 0$

  Since $\alpha + \beta + \gamma = \frac{\pi}{2}$ all three cannot be $\frac{\pi}{2}$, thus,
  $\cot\alpha\cot\gamma = 3$.
  %25
\item We can write given expression as
  $8.\frac{2\sin\frac{2\pi}{15}}{\sin\frac{2\pi}{15}}\cos\frac{2\pi}{15}\cos\frac{4\pi}{15}\cos\frac{8\pi}{15}\cos\frac{16\pi}{15}$

  $= \frac{4}{\sin\frac{2\pi}{15}}.2\sin\frac{4\pi}{15}\cos\frac{4\pi}{15}\cos\frac{8\pi}{15}\cos\frac{16\pi}{15}$

  and proceeding similarly we arrive at $\frac{\sin\frac{32\pi}{15}}{\sin\frac{2\pi}{15}}
  = \frac{\sin\left(2\pi + \frac{2\pi}{15}\right)}{\sin\frac{2\pi}{15}} = 1$.
  %26
\item Since $\alpha$ and $\beta$ lie between $0$ and $\frac{\pi}{4}$, therefore, $\alpha + \beta$ lie
  betwewen $0$ and $\frac{\pi}{2}$.

  $\Rightarrow \tan(\alpha + \beta) = \frac{3}{4}$ and $\tan(\alpha - \beta) = \frac{5}{12}$

  $\tan2\alpha = \tan(\alpha + \beta + \alpha - \beta) = \frac{\tan(\alpha + \beta) + \tan(\alpha
    - \beta)}{1 - \tan(\alpha + \beta)\tan(\alpha - \beta)}$

  $= \frac{56}{33}$.
  %27
\item We can write $\tan(\alpha + \beta - \alpha) = \frac{\tan(\alpha + \beta) - \tan\alpha}{1
  + \tan\alpha\tan(\alpha + \beta)}$

  $\Rightarrow \tan\alpha\tan(\alpha + \beta) = \frac{\tan(\alpha + \beta) - \tan\alpha
  - \tan\beta}{\tan\beta}$

  Similarly, $\tan(\alpha+ \beta)\tan(\alpha+ + 2\beta) = \frac{\tan(\alpha + 2\beta) - \tan(\alpha + \beta)
    - \tan\beta}{\tan\beta}$ and so on.

  Adding we get $\tan\alpha\tan(\alpha + \beta) + \tan(\alpha + \beta)\tan(\alpha + 2\beta)
  + \tan(\alpha + 2\beta)\tan(\alpha + 3\beta) + \cdots = \frac{\tan(\alpha + n\beta) - \tan\alpha -
    n\tan\beta}{\tan\beta}$.
\stopitemize
