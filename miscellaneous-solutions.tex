% -*- mode: context; -*-
\chapter{Miscellaneous Problems}
\startitemize[n, 1*broad]
% 1
\item L.H.S.\ $= \frac{\sqrt{1 + \cos x} + \sqrt{1 - \cos x}}{\sqrt{1 + \cos x} - \sqrt{1 - \cos x}}$

  $= \frac{\sqrt{2\cos^2\frac{x}{2}} + \sqrt{2\sin^2\frac{x}{2}}}{\sqrt{2\cos^2\frac{x}{2}}
  - \sqrt{2\sin^2\frac{x}{2}}} = \frac{\left|\cos\frac{x}{x}\right|
  + \left|\sin\frac{x}{x}\right|}{\left|\cos\frac{x}{x}\right| - \left|\sin\frac{x}{x}\right|}$

  $= \frac{-\cos\frac{x}{2} + \sin\frac{x}{2}}{-\cos\frac{x}{2} - \sin\frac{x}{2}}[\because \pi < x <
  2\pi\Rightarrow \frac{\pi}{2} < \frac{x}{2} < \pi]$

  $= \frac{\cot\frac{x}{2} - 1}{\cot\frac{x}{2} + 1} = \cot\left(\frac{x}{2} + \frac{\pi}{4}\right)$.
  %2
\item $2A = A + B + A - B \Rightarrow \tan2A = \frac{\tan(A + B) + \tan(A - B)}{1 - \tan(A + B)\tan(A - B)}$

  From question $0 < A < \frac{\pi}{4}$ anda $0 < B < \frac{\pi}{4} \Rightarrow 0 < A + B < \frac{\pi}{2}$

  Also, $-\frac{\pi}{4} < A - B < \frac{\pi}{4}$, however, $\sin(A - B) = \frac{1}{\sqrt{10}}\Rightarrow 0 <
  A - B < \frac{\pi}{4}$

  $\Rightarrow \tan(A - B) = \frac{1}{3}, \cos(A + B) = \frac{2}{\sqrt{29}}\Rightarrow \tan(A + B)
  = \frac{5}{2}$

  $\Rightarrow \tan2A = \frac{\frac{5}{2} + \frac{1}{3}}{1 - \frac{5}{2}.\frac{1}{3}} = 17$.
  %3
\item Given that $\sin^3x\sin3x = \displaystyle\sum_{m = 0}^nc_m\cos mx \Rightarrow \frac{3\sin x
  - \sin3x}{4}\sin3x = \displaystyle\sum_{m = 0}^nc_m\cos mx$

  $\Rightarrow \frac{3}{8}(2\sin3x.\sin x) - \frac{1}{8}2\sin^23x = \displaystyle\sum_{m = 0}^nc_m\cos
  mx \Rightarrow \frac{3}{8}[\cos2x - \cos4x] - \frac{1}{8}[1 - \cos6x] = \displaystyle\sum_{m = 0}^nc_m\cos
  mx$

  $\Rightarrow -\frac{1}{8} + \frac{3}{8}\cos2x - \frac{3}{8}\cos4x + \frac{1}{8}\cos6x
  = \displaystyle\sum_{m = 0}^nc_m\cos mx$

  Comparing we get $n = 6$.
  %4
\item L.H.S. $= \left[\sin^4\frac{\pi}{8} + \sin^4\frac{7\pi}{8}\right] + \left[\sin^4\frac{3\pi}{8}
  + \sin^4\frac{5\pi}{8}\right]$

  $= \left[\sin^4\frac{\pi}{8} + \sin^4\left(\pi - \frac{\pi}{8}\right)\right] + \left[\sin^4\frac{3\pi}{8}
  + \sin^4\left(\pi - \frac{3\pi}{8}\right)\right]$

  $= 2\sin^4\frac{\pi}{8} + 2\sin^4\frac{3\pi}{8} = 2\left[\left(\sin^2\frac{\pi}{8}\right)^2
  + \left(\sin^2\frac{3\pi}{8}\right)^2\right]$

  $= 2\left[\left(\frac{1 - \cos\frac{\pi}{4}}{2}\right)^2 + \left(\frac{1
    - \cos\frac{3\pi}{4}}{2}\right)^2\right] = \frac{1}{2}\left[\left(1 - \frac{1}{\sqrt{2}}\right)^2
  + \left(1 + \frac{1}{\sqrt{2}}\right)^2\right] = \frac{3}{2} =$ R.H.S.
  %5
\item $\displaystyle\sum_{q = 1}^{10}\left(\sin\frac{2q\pi}{11} - i\cos\frac{2q\pi}{11}\right)$

  $= \sin\frac{2\pi}{11} + \sin\frac{4\pi}{11} + \sin\frac{6\pi}{11} + \cdots$ to $10$ terms
  $- i\cos\frac{2\pi}{11} + \cos\frac{4\pi}{11} + \cos\frac{6\pi}{11} + \cdots$ to $10$ terms

  $= \frac{\sin10\frac{\pi}{11}}{\sin\frac{\pi}{11}}\sin\left[\frac{2\pi}{11} + (10 -
  1)\frac{\pi}{11}\right] - i\frac{\sin10\frac{\pi}{11}}{\sin\frac{\pi}{11}}\cos\left[\frac{2\pi}{11} + (10
  - 1)\frac{\pi}{11}\right]$

  $= \frac{\sin\left(\pi - \frac{\pi}{11}\right)}{\sin\frac{\pi}{11}}.\sin\pi - i\frac{\sin\left(\pi
  - \frac{\pi}{11}\right)}{\sin\frac{\pi}{11}}.\cos\pi = i$

  $\displaystyle\sum_{p = 1}^{32}(3p + 2)\left[\displaystyle\sum_{q = 1}^{10}\sin\frac{2q\pi}{11} -
  i\cos\frac{2q\pi}{11}\right]^p = \sum_{p=1}^{32}(3p + 2)i^p = 5i + 8i^2 + 11i^3 + \cdots$ upto $32$ terms

  $= 5i - 8 - 11i + 14 + 17i - 20 + \cdots$ upto $32$ terms

  $= (5i - 11i + 17i - 23i + \cdots)$ upto $16$ temrs $+ (-8 + 14 - 20 + 26 - \cdots)$ upto $16$ terms

  $= 48(1 - i)$.
  %6
\item Let $x = 2^n\cos\theta\cos2\theta\cos2^2\theta\ldots\cos2^{n - 1}\theta$

  $x\sin\theta = 2^n(\sin\theta\cos\theta)\cos2\theta\cos2^2\theta\ldots\cos2^{n - 1}\theta = 2^{n -
  1}\sin2\theta\cos2\theta\cos2^2\theta\ldots\cos2^{n - 1}\theta$

  Proceeding similarly, $x\sin\theta = \sin2^n\theta$. Given than $\theta = \frac{\pi}{2^n - 1} \Rightarrow
  2^n\theta = \pi + \theta$

  $\Rightarrow x\sin\theta = \sin(\pi + \theta) \Rightarrow x = -1$.
  %7
\item Let $x = \cos\frac{2\pi}{7}\cos\frac{4\pi}{7}\cos\frac{6\pi}{7} = \frac{1}{8}
  = \cos\frac{2\pi}{7}\cos\frac{4\pi}{7}\cos\left(\pi - \frac{\pi}{7}\right)$

  $= -\cos\frac{\pi}{7}\cos\frac{2\pi}{7}\cos\frac{4\pi}{7}$

  Multiplying both sides with $\sin\frac{\pi}{7}$ gives us

  $x\sin\frac{\pi}{7} = -\frac{1}{2}.2\sin\frac{\pi}{7}\cos\frac{\pi}{7}\cos\frac{2\pi}{7}\cos\frac{4\pi}{7}
  = -\frac{1}{2}\sin\frac{2\pi}{7}\cos\frac{2\pi}{7}\cos\frac{4\pi}{7}$

  $= -\frac{1}{4}\sin\frac{4\pi}{7}\cos\frac{4\pi}{7} = -\frac{1}{8}\sin\frac{8\pi}{7} =
  -\frac{1}{8}\sin\left(\pi + \frac{\pi}{7}\right)$

  $x\sin\frac{\pi}{7} = \frac{1}{8}.\sin\frac{\pi}{7}\Rightarrow x = \frac{1}{8}$.
  %8
\item Let $y = \sin\frac{2\pi}{7} + \sin\frac{4\pi}{7} + \sin\frac{8\pi}{7} = \frac{\sqrt{7}}{2}$

  Squaring both sides $y^2 = \sin^2\frac{2\pi}{7} + \sin^2\frac{4\pi}{7} + \sin^2\frac{8\pi}{7} +
  2\sin\frac{2\pi}{7}\sin\frac{4\pi}{7} + 2\sin\frac{4\pi}{7}\sin\frac{8\pi}{7} +
  2\sin\frac{2\pi}{7}\sin\frac{8\pi}{7}$

  Let $y_1 = \sin^2\frac{2\pi}{7} + \sin^2\frac{4\pi}{7} + \sin^2\frac{8\pi}{7}$ and $y_2 =
  2\sin\frac{2\pi}{7}\sin\frac{4\pi}{7} + 2\sin\frac{4\pi}{7}\sin\frac{8\pi}{7} +
  2\sin\frac{2\pi}{7}\sin\frac{8\pi}{7}$

  $\Rightarrow y^2 = y_1 + y_2$. Now

  $y_1 = \frac{1 - \cos\frac{4\pi}{7}}{2} + \frac{1 - \cos\frac{8\pi}{7}}{2} + \frac{1
    - \cos\frac{16\pi}{7}}{2}$

  $= \frac{3}{2} - \frac{1}{2}\left[\cos\frac{2\pi}{7} + \cos\frac{4\pi}{7} + \cos\frac{6\pi}{7}\right]$

  Let $z = \cos\frac{2\pi}{7} + \cos\frac{4\pi}{7} + \cos\frac{6\pi}{7}$

  Multiplying both sides by $2\sin\frac{\pi}{7}$ gives is

  $2z\sin\frac{\pi}{7} = 2\sin\frac{\pi}{7}\left(\cos\frac{2\pi}{7} + \cos\frac{4\pi}{7}
  + \cos\frac{6\pi}{7}\right)$

  $= \sin\frac{3\pi}{7} - \sin\frac{\pi}{7} + \sin\frac{5\pi}{7} - \sin\frac{3\pi}{7} + \sin\frac{7\pi}{7} -
  \sin\frac{5\pi}{7}$

  $2z\sin\frac{\pi}{7} = \sin\pi - \sin\frac{\pi}{7} \Rightarrow z = -\frac{1}{2}\Rightarrow y_1
  = \frac{7}{4}$

  Now $y_2 = 2\sin\frac{2\pi}{7}\sin\frac{4\pi}{7} + 2\sin\frac{4\pi}{7}\sin\frac{8\pi}{7} +
  2\sin\frac{2\pi}{7}\sin\frac{8\pi}{7}$

  $= \cos\frac{2\pi}{7} - \cos\frac{6\pi}{7} + \cos\frac{4\pi}{7} - \cos\frac{12\pi}{7} + \cos\frac{6\pi}{7}
  - \cos\frac{10\pi}{7} = 0$.

  Hence proven.
  %9
\item We can rewrite given equation as L.H.S.\ $= \tan^2\frac{\pi}{16} + \tan^2\frac{7\pi}{16}
  + \tan^2\frac{2\pi}{16} + \tan^2\frac{6\pi}{16} + \tan^2\frac{3\pi}{16} + \tan^2\frac{5\pi}{16}
  + \tan^2\frac{4\pi}{16}$

  $= \left(\tan^2\frac{\pi}{16} + \cot^2\frac{\pi}{16}\right) + \left(\tan^2\frac{2\pi}{16}
  + \cot^2\frac{2\pi}{16}\right) + \left(\tan^2\frac{3\pi}{16} + \cot^2\frac{3\pi}{16}\right) +
  1[\because \tan\frac{7\pi}{16} = \tan\left(\frac{\pi}{2} - \frac{\pi}{16}\right) = \cot\frac{\pi}{16}]$

  $= \left(\tan\frac{\pi}{16} + \cot\frac{\pi}{16}\right)^2 + \left(\tan\frac{2\pi}{16}
  + \cot\frac{2\pi}{16}\right)^2 + \left(\tan\frac{3\pi}{16} + \cot\frac{3\pi}{16}\right)^2 - 2 - 2 - 2 +
  1[\because (\tan\theta + \cot\theta)^2 = \tan^2\theta + \cot^2\theta - 2\tan\theta\cot\theta]$

  $= \left(\frac{1}{\sin\frac{\pi}{16}\cos\frac{\pi}{16}}\right)^2
  + \left(\frac{1}{\sin\frac{2\pi}{16}\cos\frac{2\pi}{16}}\right)^2
  + \left(\frac{1}{\sin\frac{3\pi}{16}\cos\frac{3\pi}{16}}\right)^2 - 5$

  $= \frac{4}{\sin^2\frac{\pi}{8}} + \frac{4}{\sin^2\frac{\pi}{4}} + \frac{4}{\sin^2\frac{3\pi}{8}} - 5$

  $= \frac{4}{\sin^2\frac{\pi}{8}} + \frac{4}{\sin^2\frac{3\pi}{8}} + 4.2 - 5$

  $= \frac{4}{\sin^2\frac{\pi}{8}} + \frac{4}{\cos^2\frac{\pi}{8}} + 3
  = \frac{4}{\sin^2\frac{\pi}{8}\cos^2\frac{\pi}{8}} + 3 = \frac{16}{\sin^2\frac{\pi}{4}} + 3 = 35$.
  %10
\item Let $\theta = \frac{\pi}{7} \Rightarrow 7\theta = \pi \Rightarrow 4\theta + 3\theta = \pi$

  $\Rightarrow \tan4\theta = \tan(\pi - 3\theta) \Rightarrow \frac{4\tan\theta - 4\tan^3\theta}{1 -
  6\tan^2\theta + \tan^4\theta} = -\frac{3\tan\theta - \tan^3\theta}{1 - 3\tan^2\theta}$

  Let $\tan\theta = z\Rightarrow \frac{4z - 4z^3}{1 - 6z^2 + z^4} = -\frac{\3z - z^3}{1 - 3z^2}$

  $\Rightarrow z^6 - 21z^4 + 35z^2 - 7 = 0$

  This is a cubic equation in $z^2$ i.e. $\tan^2\theta$, therefore, the roots of the equation are
  $\tan^2\frac{\pi}{7}, \tan^2\frac{2\pi}{7}$ and $\tan^2\frac{3\pi}{7}$.

  From Vieta's relattionships the product of roots gives us
  $\tan\frac{\pi}{7}\tan\frac{2\pi}{7}\tan\frac{3\pi}{7} = \sqrt{7}$.
  %11
\item We have obtained the equation $z^6 - 21z^4 + 35z^2 - 7 = 0$ whose roots are
  $\tan^2\frac{\pi}{7}, \tan^2\frac{2\pi}{7}$ and $\tan^2\frac{3\pi}{7}$.

  Thus, sum of roots is $35$. Putting $z = \frac{1}{y}$ gives us $-7y^6 + 35y^4 - 21y^2 + 1 = 0$ whose roots
  would be $\cot^2\frac{\pi}{7}, \cot^2\frac{2\pi}{7}$ and $\cot^2\frac{3\pi}{7}$.

  Sum of these is $\frac{-35}{-7} = 3$.

  Thus, $\left(\tan^2\frac{\pi}{7} + \tan^2\frac{2\pi}{7} + \tan^2\frac{3\pi}{7}\right)
  + \left(\cot^2\frac{\pi}{7} + \cot^2\frac{2\pi}{7} + \cot^2\frac{3\pi}{7}\right) = 105$.
  %12
\item L.H.S.\ $= \sqrt{\tan x + \sin x} + \sqrt{\tan x - \sin x} = \sqrt{\tan x}(\sqrt{1 + \cos x} + \sqrt{1
  - \cos x})$

  $= \sqrt{2\tan x}\left(\cos\frac{x}{2} + \sin\frac{x}{2}\right) = 2\sqrt{\tan
  x}\left(\frac{1}{\sqrt{2}}\cos x + \frac{1}{\sqrt{2}}\sin\frac{x}{2}\right)$

  $= 2\sqrt{\tan x}\cos\left(\frac{\pi}{4} - \frac{x}{2}\right)[\because 0 < x < \frac{\pi}{2}]$.
  %13
\item We know that $\sin3x = 3\sin x - 4\sin^3x$, which makes L.H.S.\ $= \sin^3x\left(3\sin x -
  4\sin^3x\right)$

  $= 3\sin^4x - 4\sin^6x = 3\left(1 - \cos^2x\right)^2 - \left(1 - \sin^2x\right)^3$

  Thus, coefficient of $\cos^4x$  or $c_4$ is $-9$.
  %14
\item We have to find the value of $\sin7^\circ + \sin77^\circ + \sin293^\circ + \sin149^\circ
  + \sin221^\circ$

  $= \sin5^\circ + 2\sin185^\circ\cos108^\circ + 2\sin185^\circ\cos36^\circ$

  $= \sin5^\circ - 2\sin5^\circ\left(\cos108^\circ + \cos36^\circ\right) = \sin5^\circ -
  4\sin5^\circ\left(\cos72^\circ\cos36^\circ\right)$

  $= 2\sin5^\circ - 4\sin5^\circ\left[\frac{\sqrt{5} - 1}{4}.\frac{\sqrt{5 + 1}}{4}\right] = 0$.
  %15
\item Consider the complex sum $Z = \displaystyle\sum_{k = 1}^{n - 1}e^{ik\pi/n}$, then real part of this
  sum is given series, which we let as $S$.

  $Z = \frac{e^{i\pi/n} - e^{i\pi}}{1 - e^{i\pi/n}} = \frac{e^{i\pi/n} + 1}{1 - e^{i\pi/n}} =
  i\cot\frac{\pi}{2n}$, which is purely imaginary number. Thus, $S = 0$.
  %16
\item Let $\omega = e^{i2\pi/7} = \cos\frac{2\pi}{7} + i\sin\frac{2\pi}{7} \Rightarrow \omega^7 = 1$

  $\Rightarrow (\omega - 1)\left(\omega^6 + \omega^5 + \omega^4 + \omega^3 + \omega^2 + \omega + 1\right) =
  0$

  $\because \omega\neq 1 \Rightarrow \omega^6 + \omega^5 + \omega^4 + \omega^3 + \omega^2 + \omega + 1 = 0$

  $S = \cos\frac{2\pi}{7} + \cos\frac{4\pi}{7} + \cos\frac{6\pi}{7} = \frac{\omega + \omega^{-1}}{2}
  + \frac{\omega^2 + \omega^{-2}}{2} + \frac{\omega^3 + \omega^{-3}}{2}$

  $= \frac{1}{2}\left(\omega^6 + \omega^5 + \omega^4 + \omega^3 + \omega^2 + \omega\right) = -\frac{1}{2}$.
  %17
\item We see that $\sin\left(\frac{3\pi}{2} - \alpha\right) = -\cos\alpha, \sin(3\pi + \alpha) = \sin(\pi
  + \alpha) = -\sin\alpha,$

  $\sin\left(\frac{\pi}{2} + \alpha\right) = \cos\alpha,$ and $\sin(5\pi - \alpha) = \sin(\pi - \alpha)
  = \sin\alpha$

  Thus, L.H.S. $= 3\left[\cos^4\alpha + \sin^4\alpha\right] - 2\left[\cos^6\alpha + \sin^6\alpha\right]$

  $ 3\left(1 - \sin^2\alpha\cos^2\alpha\right) - 2\left(1 - 3\cos^2\alpha\sin^2\alpha\right) = 1$.
  %18
\item $\sin36^\circ\sin72^\circ\sin108^\circ\sin144^\circ = \sin36^\circ\sin72^\circ\sin72^\circ\sin36^\circ
  = \left(\sin36^\circ\sin72^\circ\right)^2$

  $= \frac{1}{4}\left(\cos(36^\circ - 72^\circ) - \cos(36^\circ + 72^\circ)\right)^2
  = \frac{1}{4}\left(\cos36^\circ + \sin18^\circ\right)^2$

  $= \frac{1}{4}\left(\frac{\sqrt{5 + 1}}{4} + \frac{\sqrt{5} - 1}{4}\right)^2 = \frac{5}{16}$.
  %19
\item We have to prove that $\sin^212^\circ + \sin^221^\circ + \sin^239^\circ + \sin^248^\circ = 1
  + \sin^29^\circ + \sin^218^\circ$

  We know that $\sin^2\theta = \frac{1 - \cos2\theta}{2}$, which transforma above to

  $2 - \cos24^\circ - \cos42^\circ - \cos78^\circ - \cos96^\circ = 2 - \cos18^\circ - \cos36^\circ$

  Now, $\cos24^\circ + \cos42^\circ + \cos78^\circ + \cos96^\circ = 2\cos60^\circ\cos36^\circ +
  2\cos60^\circ\cos18^\circ$

  $= \cos36^\circ + \cos18^\circ$. Hence proven.
  %20
\item In the given range, the cosine function is non-negative. For $135^\circ\leq \alpha/2\leq 180^\circ$
  sine function is positive and negative in $180^\circ\leq \alpha/2 \leq 225^\circ$.

  Thus, $\sin\frac{\alpha}{2} = \frac{1}{2}\left[\sqrt{1 - \sin\alpha} - \sqrt{1 + \sin\alpha}\right]$

  $\cos\frac{\alpha}{2} = -\frac{1}{2}\left[\sqrt{1 + \sin\alpha} + \sqrt{1 - \sin\alpha}\right]$.
  %21
\item We know that $\tan142^\circ30' = \frac{1 - \cos285^\circ}{\sin285^\circ}$

  $\sin285^\circ = -\sin75^\circ = \sin\left(45^\circ + 30^\circ\right) = -\frac{\sqrt{6} + \sqrt{2}}{4}$

  $\cos285^\circ = \cos75^\circ = \cos\left(45^\circ + 30^\circ\right) = \frac{\sqrt{6} - \sqrt{2}}{4}$

  $\Rightarrow \tan142^\circ30' = \frac{1 - \frac{\sqrt{6} - \sqrt{2}}{4}}{-\frac{\sqrt{6} + \sqrt{2}}{4}}$

  $= \frac{\sqrt{6} - \sqrt{2} - \sqrt{4}}{\sqrt{6} + \sqrt{2}}$

  Rationalizing gives us $\frac{\sqrt{6} - \sqrt{2} - \sqrt{4}}{\sqrt{6} + \sqrt{2}}.\frac{\sqrt{6}
    - \sqrt{2}}{\sqrt{6} - \sqrt{2}}$

  On simplification we get the desired result.
  %22
\item Let $\tan\frac{\alpha}{2} = t$, then $\frac{2t}{1 + t^2} + \frac{1 - t^2}{1 + t^2}
  = \frac{\sqrt{7}}{2}$

  $\Rightarrow \left(\sqrt{7} + 2\right)t^2 - 4t + \left(\sqrt{7} - 2\right) = 0$

  $\Rightarrow D = \sqrt{16 - 4.3} = \pm2$

  $t = \frac{2\pm 1}{\sqrt{7} + 2}$.
  %23
\item Given that $\frac{\sin(x - \alpha)}{\sin(x - \beta)} = \frac{a}{b} \Rightarrow b\sin(x - \alpha) =
  a\sin(x - \beta)$

  $\Rightarrow b(\sin x\cos\alpha - \cos x\sin\alpha) = a(\sin x\cos\alpha - \cos x\sin\alpha)$

  $\Rightarrow \tan x = \frac{b\sin\alpha - a\sin\beta}{b\cos\alpha - a\cos\beta}$

  Similarly from $\frac{\cos(x - \alpha)}{\cos(x - \beta)} = \frac{A}{B}$ we get

  $\tan x = \frac{A\cos\beta - B\cos\alpha}{B\sin\alpha - A\sin\beta}$

  $\Rightarrow \frac{b\sin\alpha - a\sin\beta}{b\cos\alpha - a\cos\beta} = \frac{A\cos\beta -
    B\cos\alpha}{B\sin\alpha - A\sin\beta}$

  Cross-multiplying and simplifying gives us desired relation.
  %24
\item Since $\cot\alpha, \cot\beta, \cot\gamma$ are in A.P. $\Rightarrow 2\cot\beta = \cot\alpha
  + \cot\beta$

  $\cot(\alpha + \beta + \gamma) = \frac{\pi}{2}\Rightarrow \frac{\cot\alpha\cot\beta\cot\gamma - \cot\alpha
    - \cot\beta - \cot\gamma}{\cot\alpha\cot\beta + \cot\beta\cot\gamma + \cot\gamma\cot\alpha - 1}  = 0$

  $\Rightarrow \cot\alpha\cot\beta\cot\gamma = \cot\alpha + \cot\beta + \cot\gamma = 3\cot\beta$

  $\Rightarrow \cot\beta(\cot\alpha\cot\gamma - 3) = 0$

  Since $\alpha + \beta + \gamma = \frac{\pi}{2}$ all three cannot be $\frac{\pi}{2}$, thus,
  $\cot\alpha\cot\gamma = 3$.
  %25
\item We can write given expression as
  $8.\frac{2\sin\frac{2\pi}{15}}{\sin\frac{2\pi}{15}}\cos\frac{2\pi}{15}\cos\frac{4\pi}{15}\cos\frac{8\pi}{15}\cos\frac{16\pi}{15}$

  $= \frac{4}{\sin\frac{2\pi}{15}}.2\sin\frac{4\pi}{15}\cos\frac{4\pi}{15}\cos\frac{8\pi}{15}\cos\frac{16\pi}{15}$

  and proceeding similarly we arrive at $\frac{\sin\frac{32\pi}{15}}{\sin\frac{2\pi}{15}}
  = \frac{\sin\left(2\pi + \frac{2\pi}{15}\right)}{\sin\frac{2\pi}{15}} = 1$.
  %26
\item Since $\alpha$ and $\beta$ lie between $0$ and $\frac{\pi}{4}$, therefore, $\alpha + \beta$ lie
  betwewen $0$ and $\frac{\pi}{2}$.

  $\Rightarrow \tan(\alpha + \beta) = \frac{3}{4}$ and $\tan(\alpha - \beta) = \frac{5}{12}$

  $\tan2\alpha = \tan(\alpha + \beta + \alpha - \beta) = \frac{\tan(\alpha + \beta) + \tan(\alpha
    - \beta)}{1 - \tan(\alpha + \beta)\tan(\alpha - \beta)}$

  $= \frac{56}{33}$.
  %27
\item We can write $\tan(\alpha + \beta - \alpha) = \frac{\tan(\alpha + \beta) - \tan\alpha}{1
  + \tan\alpha\tan(\alpha + \beta)}$

  $\Rightarrow \tan\alpha\tan(\alpha + \beta) = \frac{\tan(\alpha + \beta) - \tan\alpha
  - \tan\beta}{\tan\beta}$

  Similarly, $\tan(\alpha+ \beta)\tan(\alpha+ + 2\beta) = \frac{\tan(\alpha + 2\beta) - \tan(\alpha + \beta)
    - \tan\beta}{\tan\beta}$ and so on.

  Adding we get $\tan\alpha\tan(\alpha + \beta) + \tan(\alpha + \beta)\tan(\alpha + 2\beta)
  + \tan(\alpha + 2\beta)\tan(\alpha + 3\beta) + \cdots = \frac{\tan(\alpha + n\beta) - \tan\alpha -
    n\tan\beta}{\tan\beta}$.
  %28
\item We have to prove that $-\cot\alpha + \tan\alpha + 2\tan2\alpha + 4\tan4\alpha + 8\tan8\alpha + \cdots
  + 2^n\cot2^n\alpha = 0$

  We see that $-\cot\alpha + \tan\alpha = -\frac{\sin\alpha}{\cos\alpha} + \frac{\sin\alpha}{\cos\alpha}
  = -\frac{2(\cos^2\alpha - \sin^2\alpha)}{2\sin\alpha\cos\alpha}$

  $= -2\cot\alpha$

  Similarly $-2\cot2\alpha + 2\tan\alpha = -2^2\cot^2\alpha$

  Proceeding similarly we obtain $-2^n\cot2^n\alpha + 2^n\cot2^n\alpha = 0$.
  %29
\item R.H.S.\ $= \frac{1}{2}[\tan27x - \tan x] = \frac{1}{2}[(\tan27x - \tan9x) + (\tan9x - \tan3x) +
  (\tan3x - \tan x)]$

  $= \frac{1}{2}\left[\left(\frac{\sin27x}{\cos27x} - \frac{\sin9x}{\cos9x}\right)
  + \left(\frac{\sin9x}{\cos9x} - \frac{\sin3x}{\cos3x}\right) + \left(\frac{\sin3x}{\cos3x} - \frac{\sin
    x}{\cos x}\right)\right]$

  $= \frac{1}{2}\left[\frac{\sin(27 - 9)x}{\cos27x\cos9x} + \frac{\sin(9 - 3)x}{\cos9x\cos3x} + \frac{\sin(3
    - 1)x}{\cos3x\cos x}\right]$

  $= \frac{1}{2}\left[\frac{2\sin9x\cos9x}{\cos27x\cos9x} + \frac{2\sin3x\cos3x}{\cos9x\cos3x} + \frac{2\sin
    x\cos x}{\cos3x\cos x}\right]$

  $= \frac{\sin x}{\cos3x} + \frac{\sin3x}{\cos9x} + \frac{\sin9x}{\cos27x} =$ L.H.S.
  %30
\item Rewriting, we have to prove that $\cot16^\circ.\cot44^\circ - 1 + \cot44^\circ.\cot76^\circ - 1
  = \cot76^\circ.\cot16^\circ + 1$

  $\Rightarrow \left(\frac{\cos16^\circ.\cos44^\circ}{\sin16^\circ.\sin44^\circ} - 1\right)
  + \left(\frac{\cos44^\circ.\cos76^\circ}{\sin44^\circ.\sin76^\circ} - 1\right)
  = \left(\frac{\cos76^\circ\cos16^\circ}{\sin76^\circ\sin16^\circ} + 1\right)$

  $\Rightarrow \frac{\cos60^\circ}{\sin16^\circ\sin44^\circ}
  + \frac{\cos120^\circ}{\sin44^\circ\sin76^\circ} = \frac{\cos60^\circ}{\sin76^\circ\sin16^\circ}$

  $\Rightarrow \frac{1}{2\sin16^\circ\sin44^\circ} - \frac{1}{2\sin44^\circ\sin76^\circ}
  = \frac{1}{2\sin76^\circ\sin16^\circ}$

  $\Rightarrow \frac{\sin76^\circ - \sin16^\circ}{\sin16^\circ\sin44^\circ\sin76^\circ}
  = \frac{1}{\sin76^\circ\sin16^\circ}$

  $\Rightarrow \sin76^\circ - \sin16^\circ = \sin44^\circ \Rightarrow 2\cos46^\circ\sin30^\circ
  = \sin44^\circ \Rightarrow \sin44^\circ = \sin44^\circ$. Hence proved.
  %31
\item Rewriting, we have to prove that $\tan\theta\tan2\theta + 1 + \tan2\theta\tan4\theta + 1
  + \tan4\theta\tan\theta + 1 = -4$

  $\Rightarrow \left(\frac{\sin\theta\sin2\theta}{\cos\theta\cos2\theta} + 1\right)
  + \left(\frac{\sin2\theta\sin4\theta}{\cos2\theta\cos4\theta} + 1\right)
  + \left(\frac{\sin\theta\sin4\theta}{\cos\theta\cos4\theta} + 1\right) = -4$

  $\Rightarrow \frac{\cos\theta}{\cos\theta\cos2\theta} + \frac{\cos2\theta}{\cos2\theta\cos4\theta}
  + \frac{\cos3\theta}{\cos\theta\cos4\theta} = -4$

  $\Rightarrow \frac{1}{\cos2\theta} + \frac{1}{\cos4\theta} + \frac{1}{\cos\theta} =
  -4\left[\because \cos(4\theta) = \cos(2\pi - 4\theta)\text{ as }\theta = \frac{2\pi}{7}\right]$

  $\Rightarrow 2\cos\theta\cos4\theta + 2\cos2\theta\cos\theta + 2\cos4\theta\cos2\theta =
  -8\cos\theta\cos2\theta\cos4\theta$

  $\Rightarrow \cos5\theta + \cos3\theta + \cos3\theta + \cos\theta + \cos2\theta + \cos6\theta =
  -8\cos\theta\cos2\theta\cos4\theta$

  $\Rightarrow \cos\theta + \cos2\theta + \cos3\theta + \cos4\theta + \cos5\theta + \cos6\theta =
  -8\cos\theta\cos2\theta\cos4\theta$

  $\Rightarrow 2\cos\theta + 2\cos2\theta + 2\cos3\theta = -8\cos\theta\cos2\theta\cos4\theta$

  Now L.H.S.\ $= \cos\frac{2\pi}{7} + \cos\frac{4\pi}{7} + \cos\frac{6\pi}{7} = -\frac{1}{2}$

  Also, $\cos\frac{2\pi}{7}.\cos\frac{4\pi}{7}.\cos\frac{6\pi}{7} = \frac{1}{8}$.

  Hence proved.
\item Given that $\frac{\sin^4\alpha}{a} + \frac{\cos^4\alpha}{b} = \frac{1}{a + b}$

  $\Rightarrow b(a + b)\sin^4\alpha + a(a + b)\left(1 - \sin^2\alpha\right)^2 = ab$

  $\Rightarrow (a + b)^2\sin^4\alpha + 2a(a + b)\sin^2\alpha.a + a^2 = 0$

  $\Rightarrow \left[(a + b)\sin^2\alpha - a\right]^2 = 0 \Rightarrow \sin^2\alpha = \frac{a}{a +
  b}\Rightarrow \cos^2\alpha = \frac{b}{a + b}$

  Thus, $\frac{\sin^8\alpha}{a^3} + \frac{\cos^8\alpha}{b^3} = \frac{1}{(a + b)^3}$.
  %33
\item $S = \sin3A + \sin3B + \sin3C = 2\sin\frac{3(A + B)}{2}\cos\frac{3(A - B)}{2} +
  2\sin\frac{3C}{2}\cos\frac{3C}{2}$

  $= 2\sin\frac{3(\pi - C)}{2}\cos\frac{3(A - B)}{2} + 2\sin\frac{3C}{2}\cos\frac{3C}{2}$

  $= -2\cos\frac{3C}{2}\cos\frac{3(A - B)}{2} + 2\sin\frac{3C}{2}\cos\frac{3C}{2} =
  -2\cos\frac{3C}{2} \left[\cos\frac{3(A - B)}{2} - \sin\left()\frac{3\pi}{2} - \frac{3(A +
      B)}{2}\right)\right]$

  $= -2\cos\frac{3C}{2}\left[\cos\frac{3A - 3B}{2} + \cos\frac{3A + 3B}{2}\right] =
  -2\cos\frac{3C}{2}.2\cos\frac{3A}{2}.\cos\frac{3B}{2}$

  $= -4\cos\frac{3A}{2}.\cos\frac{3B}{2}.\cos\frac{3C}{2}$

  $S = 0 \Rightarrow -4\cos\frac{3A}{2}.\cos\frac{3B}{2}.\cos\frac{3C}{2} = 0$

  So one of the terms is zero. Let $\cos\frac{3A}{2} = 0 \Rightarrow \frac{3A}{2} = \frac{\pi}{2}\Rightarrow
  A = \frac{\pi}{3}$.

  Thus, at least one of the angle is $60^\circ$.
  %34
\item Rewriting, we have $\sin y[\sin x\sin(x - y) + \sin z\sin(y - z)] + \sin(z - x)[\sin z\sin x + \sin(x
  - y)\sin(y - z)]$

  $= \frac{1}{2}\sin y[2\sin x\sin(x - y) + 2\sin z\sin(y - z)] + \frac{1}{2}\sin(z - x)[2\sin z\sin x + 2\sin(x
  - y)\sin(y - z)]$

  $= \frac{1}{2}\sin y[\cos y - \cos(2x - y) + \cos(2z - y) - \cos y] + \frac{1}{2}\sin(z - x)[\cos(z - x)
  - \cos(z + x) + (\cos x + z - 2y) - \cos(x - z)]$

  $= \frac{1}{2}\sin y[\cos(2z - y) - \cos(2x - y)] + \frac{1}{2}\sin(z - x)[\cos(x + z - 2y) - \cos(z + x)]$

  $= \frac{1}{2}\sin y[2\sin(z + x - y).\sin(x - z)] + \frac{1}{2}\sin(z - x)[2\sin(z + x - y).\sin y] = 0
  =$ R.H.S.
  %35
\item $\cot3\theta = \cot(2\theta + \theta) = \frac{\cot3\theta + \cot\theta - 1}{\cot2\theta + \cot\theta}$

  $\Rightarrow \cot3\theta\cot2\theta + \cot3\theta\cot\theta = \cot2\theta\cot\theta - 1$

  $\Rightarrow \cot3\theta\cot2\theta - \cot2\theta\cot\theta + 1 = -\cot3\theta\cot\theta$

  Also, $\cot2\theta = \frac{\cot^2\theta - 1}{2\cot\theta}\Rightarrow 2\cot\theta\cot2\theta + 1
  = \cot^2\theta$

  Adding the two equations obtained we get the desired equation.
  %36
\item $\frac{1 + \sin A}{\cos A} = \tan A + \sec A$ and $\frac{\cos B}{1 - \sin B} = \frac{\cos B(1 + \sin
  B)}{1 - \sin^2B} = \tan B + \sec B$

  Now $\sin A - \sin B = 2\cos\frac{A + B}{2}\sin\frac{A - B}{2}, \sin(A - B) = 2\sin\frac{A -
    B}{2}\cos\frac{A - B}{2}$ and $\cos A - \cos B = 2\sin \frac{A + B}{2}\sin\frac{A - B}{2}$

  Then denominator of R.H.S.\ becomes $2\sin\frac{A - B}{2}\left(\cos\frac{A - B}{2} - \sin\frac{A +
    B}{2}\right)$

  Thus, R.H.S. is $\frac{2\cos\frac{A + B}{2}}{\cos\frac{A - B}{2} - \sin\frac{A + B}{2}}$

  Rationalizing $\frac{2\cos\frac{A + B}{2}}{\cos\frac{A - B}{2} - \sin\frac{A + B}{2}}.\frac{\cos\frac{A -
      B}{2} + \sin\frac{A + B}{2}}{\cos\frac{A - B}{2} - \sin\frac{A + B}{2}}$

  Now we use $\cos^2x - \sin^2y = \frac{1 + \cos2x}{2} - \frac{1 - \cos 2y}{2}$, which makes denominator

  $\frac{\cos(A - B) + \cos(A + B)}{2} = \cos A\cos B$

  Similarly expanding numerator gives us $\cos A + \cos B + \sin A + \sin B$ and R.H.S.\ becomes $\tan A
  + \sec A + \tan B + \sec B$.

  Hence, L.H.S. = R.H.S.
  %37
\item Let $u = \cos^2x$ and $v = \cos^2y$ then the given equation becomes

  $\frac{u^2}{v} + \frac{(1 - u)^2}{1 - v} = 1\Rightarrow \frac{u^2(1 - v) + v(1 - u)^2}{v(1 - v)} = 1$

  $\Rightarrow u^2 - u^2v + v\left(1 - 2u + u^2\right) = v - v^2$

  $\Rightarrow u^2 - 2uv + v^2 = 0 \Rightarrow (u - v)^2 = 0 \Rightarrow u = v$

  Thus, $\cos^2x = \cos^2y$ and $\sin^2x = \sin^2y$, so they are interchangable, and, hence proved.
  %38
\item Given that $\theta + \phi + \psi = 2\pi\Rightarrow \cos\psi = \cos(2\pi - \theta - \phi) = \cos(\theta
  + \phi)$

  Now, L.H.S. $= \cos^2\theta + \cos^2\phi + \cos^2(\theta + \phi) - 2\cos\theta\cos\phi\cos(\theta + \phi)$

  $= \cos^2\theta + \cos^2\phi + (\cos\theta\cos\phi - \sin\theta\sin\phi)^2 -
  2\cos\theta\cos\phi(\cos\theta\cos\phi - \sin\theta\sin\phi)$

  $= \cos^2\theta + \cos^2\phi + \sin^2\theta\sin^2\phi - \cos^2\theta\cos^2\phi$

  $= \cos^2\theta + \cos^2\phi + (1 - \cos^2\theta)(1 - \cos^2\phi) - \cos^2\theta\cos^2\phi = 1$.
  %39
\item $\cos3A + \cos3B + \cos3[\pi - (A + B)] = 1 \Rightarrow \cos3A + \cos3B - \cos3(A + B) = 1$

  $\Rightarrow (1 - \cos3A)(1 - \cos3B) = \sin3A\sin3B$

  $\Rightarrow 2\sin^2\frac{3A}{2}.2\sin^2\frac{3B}{2} =
  2\sin\frac{3A}{2}\cos\frac{3A}{2}.2\sin\frac{3B}{2}\cos\frac{3B}{2}$

  Assuming $\sin\frac{3\pi}{2}\neq 0$ and $\sin\frac{3B}{2} \neq 0$ gives us

  $\cos\frac{3A}{2}\cos\frac{3B}{2}
  = \sin\frac{3A}{2}\sin\frac{3B}{2} \Rightarrow \tan\frac{3A}{2}.\tan\frac{3B}{2} = 1$

  Thus, $\frac{3A}{2} + \frac{3B}{2} = \frac{\pi}{2} + n\pi$

  It is given that $A + B + C = \pi$, so for the case of a triangle $n = 0$.

  $\Rightarrow A + B = \frac{\pi}{3}\Rightarrow C = \frac{2\pi}{3}$. Hence proved.
  %40
\item We know that for a triangle $\sin A = \frac{a}{2R}, \sin B = \frac{b}{2R}, \sin C = \frac{c}{2R}$

  Thus, $S = \sum \sin^3A\sin(B - C) = \frac{a^3}{8R^3}\sin(B - C) + \frac{b^3}{8R^3}\sin(C - A)
  + \frac{c^3}{8R^3}\sin(A - B)$

  Expanding the first term, we have $a^3\left(\sin B\cos C - \cos B\sin C\right)$

  Now we will substitute $\cos B = \frac{a^2 + c^2 - b^2}{2ac}$ and also for $\cos C$

  This makes the first term $a^3\left(b.\frac{a^2 + b^2 - 2ab}{2ab}\right) - b^3\left(c.\frac{a^2 + c^2 -
    b^2}{2ac}\right)$.

  Clearly the cyclic sum is zero.
  %41
\item L.H.S.\ $= \frac{\sin^3\theta + \cos^3\theta - (\sin^5\theta + \cos^5\theta)}{\sin\theta
  + \cos\theta}$

  Numerator is $\sin^3\theta(1 - \sin^2\theta) + \cos^3\theta(1 - \cos^2\theta) = \sin^3\theta\cos^2\theta
  + \cos^3\theta\sin^2\theta$

  $= \sin^2\theta\cos^2\theta(\sin\theta + \cos\theta)$

  $\Rightarrow $ L.H.S.\ $= \sin^2\theta\cos^2\theta$

  R.H.S.\ $= \frac{\sin^5\theta + \cos^5\theta - (\sin^7\theta + \cos^7\theta)}{\sin^3\theta
  + \cos^3\theta}$

  Numerator is $\sin^5\theta(1 - \sin^2\theta) + \cos^5\theta(1 - \cos^2\theta) = \sin^5\theta\cos^2\theta
  + \cos^5\theta\sin^2\theta$

  $= \sin^2\theta\cos^2\theta(\sin^3\theta + \cos^3\theta)$

  $\Rightarrow $ R.H.S.\ $= \sin^2\theta\cos^2\theta =$ L.H.S.
  %42
\item  Let $a = x - y$, $b = y - z$, and $c = z - x$. Then $a + b + c = 0$ and the given condition becomes
  $4 \cos a \cos b \cos c = 1$, hence $\cos a \cos b \cos c = \frac14$.

  Using the identity $\cos 3t = 4 \cos^3 t - 3 \cos t$, we have
  $\cos 3a \cos 3b \cos 3c
  = (4 \cos^3 a - 3 \cos a)(4 \cos^3 b - 3 \cos b)(4 \cos^3 c - 3 \cos c)$.

  Factoring out $\cos a \cos b \cos c$ gives
  $\cos 3a \cos 3b \cos 3c
  = \cos a \cos b \cos c (4 \cos^2 a - 3)(4 \cos^2 b - 3)(4 \cos^2 c - 3)$.

  Multiplying both sides by $4$ and using $\cos a \cos b \cos c = \frac14$, we obtain
  $4 \cos 3a \cos 3b \cos 3c
  = (4 \cos^2 a - 3)(4 \cos^2 b - 3)(4 \cos^2 c - 3)$.

  Since $\cos 2a = 2 \cos^2 a - 1$, we have $4 \cos^2 a - 3 = 2 \cos 2a - 1$, and similarly for $b$ and $c$.
  Thus
  $4 \cos 3a \cos 3b \cos 3c
  = (2 \cos 2a - 1)(2 \cos 2b - 1)(2 \cos 2c - 1)$.

  Expanding, this equals
  $1 - 2(\cos 2a + \cos 2b + \cos 2c)
  + 4(\cos 2a \cos 2b + \cos 2b \cos 2c + \cos 2c \cos 2a)
  - 8 \cos 2a \cos 2b \cos 2c$.

  Because $a + b + c = 0$, we use the identities
  $\cos 2a + \cos 2b + \cos 2c = 2$
  and
  $\cos 2a \cos 2b + \cos 2b \cos 2c + \cos 2c \cos 2a = 1$.

  Substituting these values yields
  $4 \cos 3a \cos 3b \cos 3c
  = 1 + 12 \cos 2a \cos 2b \cos 2c$.

  Replacing $a$, $b$, and $c$ by $x - y$, $y - z$, and $z - x$, respectively, we conclude that
  $1 + 12 \cos 2(x - y) \cos 2(y - z) \cos 2(z - x)
  = 4 \cos 3(x - y) \cos 3(y - z) \cos 3(z - x)$.
  %43
\item Use the identity
  $\tan(A + B + C) = \frac{\tan A + \tan B + \tan C - \tan A \tan B \tan C}
  {1 - \tan A \tan B - \tan B \tan C - \tan C \tan A}$

  implies that there exist angles $A$, $B$, and $C$ such that
  $x = \tan A$, $y = \tan B$, and $z = \tan C$ with $A + B + C = \pi$.

  Using the triple-angle identity for tangent,
  $\tan 3A = \frac{3 \tan A - \tan^3 A}{1 - 3 \tan^2 A}$,
  we obtain
  $\frac{3x - x^3}{1 - 3x^2} = \tan 3A$,
  and similarly
  $\frac{3y - y^3}{1 - 3y^2} = \tan 3B$
  and
  $\frac{3z - z^3}{1 - 3z^2} = \tan 3C$.

  Therefore,
  $\displaystyle\sum \frac{3x - x^3}{1 - 3x^2}
  = \tan 3A + \tan 3B + \tan 3C$.

  Applying again the tangent sum identity, we get
  $\tan 3A + \tan 3B + \tan 3C
  = \tan 3A \tan 3B \tan 3C$,
  since $3A + 3B + 3C = 3\pi$.

  Substituting back gives
  $\displaystyle\sum \frac{3x - x^3}{1 - 3x^2}
  = \prod \frac{3x - x^3}{1 - 3x^2}$.

  Factoring the numerator yields
  $\prod (3x - x^3) = \prod x \prod (3 - x^2)$.

  Hence
  $\displaystyle\sum \frac{3x - x^3}{1 - 3x^2}
  = \frac{\prod x \prod (3 - x^2)}{\prod (1 - 3x^2)}$.

  Multiplying the numerator and denominator by $3$ gives
  $\displaystyle\sum \frac{3x - x^3}{1 - 3x^2}
  = \frac{3 \prod x \prod (3 - x^2)}{\prod (1 - 3x^2)}$
  %44
\item Consider the expression $(2 + \sqrt{3}) \sin \theta + 2 \cos \theta$.

  For real numbers $a$ and $b$, the identity
  $a \sin \theta + b \cos \theta = \sqrt{a^2 + b^2} \sin(\theta + \phi)$
  holds for some angle $\phi$.

  Hence $a \sin \theta + b \cos \theta$ lies between $-\sqrt{a^2 + b^2}$ and $\sqrt{a^2 + b^2}$.

  Here $a = 2 + \sqrt{3}$ and $b = 2$.

  We compute
  $a^2 + b^2 = (2 + \sqrt{3})^2 + 2^2 = 7 + 4 \sqrt{3} + 4 = 11 + 4 \sqrt{3}$.

  Observe that $(2 + \sqrt{5})^2 = 4 + 5 + 4 \sqrt{5} = 9 + 4 \sqrt{5}$ and
  $11 + 4 \sqrt{3} < 9 + 4 \sqrt{5}$.
  Therefore
  $\sqrt{11 + 4 \sqrt{3}} < 2 + \sqrt{5}$.

  Consequently,
  $(2 + \sqrt{3}) \sin \theta + 2 \cos \theta$
  lies between $-(2 + \sqrt{5})$ and $2 + \sqrt{5}$.
  %45
\item Consider the expression
  $5 \cos \theta + 3 \cos(\theta + \frac{\pi}{3}) + 3$.

  Using the identity $\cos(\theta + \frac{\pi}{3}) = \frac12 \cos \theta - \frac{\sqrt3}{2} \sin \theta$,
  we rewrite the expression as

  $5 \cos \theta + \frac32 \cos \theta - \frac{3\sqrt3}{2} \sin \theta + 3$,

  which simplifies to
  $\frac{13}{2} \cos \theta - \frac{3\sqrt3}{2} \sin \theta + 3$.

  For real numbers $a$ and $b$, the expression
  $a \cos \theta + b \sin \theta$
  lies between $-\sqrt{a^2 + b^2}$ and $\sqrt{a^2 + b^2}$.
  Here $a = \frac{13}{2}$ and $b = -\frac{3\sqrt3}{2}$.

  We compute
  $a^2 + b^2 = \frac{169}{4} + \frac{27}{4} = \frac{196}{4} = 49$,
  so
  $a \cos \theta + b \sin \theta$
  lies between $-7$ and $7$.

  Therefore
  $\frac{13}{2} \cos \theta - \frac{3\sqrt3}{2} \sin \theta + 3$
  lies between $-7 + 3 = -4$ and $7 + 3 = 10$.

  Hence
  $5 \cos \theta + 3 \cos(\theta + \frac{\pi}{3}) + 3$
  lies between $-4$ and $10$.
  %46
\item Using $\sin^2 \theta = 1 - \cos^2 \theta$, we rewrite it as
  $1 - \cos^2 \theta + \cos^4 \theta$.

  Let $x = \cos^2 \theta$.
  Then $0 \le x \le 1$ and the expression becomes
  $1 - x + x^2$.

  We rewrite this as
  $\left(x - \frac12\right)^2 + \frac34$,
  which is always at least $\frac34$.
  Hence the minimum value is $\frac34$.

  Since $0 \le x \le 1$, the maximum occurs at an endpoint.
  When $x = 0$, the value is $1$.
  When $x = 1$, the value is also $1$.

  Therefore, the minimum value of $\sin^2 \theta + \cos^4 \theta$ is $\frac34$,
  and the maximum value is $1$.
  %47
\item Let $a = \sin^2 x$ and $b = \cos^2 x$.
  Then $a + b = 1$ and $a \ge 0$, $b \ge 0$.

  We rewrite the expression as
  $a^4 + b^4$.

  Using the identity
  $a^4 + b^4 = (a + b)^4 - 4ab(a + b)^2 + 2a^2 b^2$,
  and substituting $a + b = 1$, we obtain
  $a^4 + b^4 = 1 - 4ab + 2a^2 b^2$.

  Let $ab = t$.
  Since $a + b = 1$ and $a,b \ge 0$,

  Recall that $a = \sin^2 x$ and $b = \cos^2 x$, so $a \ge 0$, $b \ge 0$, and
  $a + b = 1$.
  We defined $t = ab$.

  Since $a,b \ge 0$, it is immediate that $t \ge 0$.

  To find the upper bound, we use the identity
  $(a - b)^2 \ge 0$.
  Expanding gives
  $a^2 - 2ab + b^2 \ge 0$.

  Using $a + b = 1$, we have
  $a^2 + b^2 = (a + b)^2 - 2ab = 1 - 2ab$.
  Substituting into the inequality yields
  $1 - 2ab - 2ab \ge 0$,
  so
  $1 - 4ab \ge 0$.

  Hence
  $ab \le \frac14$,
  that is,
  $t \le \frac14$.

  Therefore we have
  $0 \le t \le \frac14$.

  Thus the expression becomes
  $1 - 4t + 2t^2$.

  We rewrite this as
  $2\left(t - \frac12\right)^2 + \frac12$.
  Since $0 \le t \le \frac14$, the minimum occurs at $t = \frac14$.

  Substituting $t = \frac14$ gives
  $1 - 4 \cdot \frac14 + 2 \cdot \frac{1}{16} = \frac18$.

  Therefore, the minimum value of $\sin^8 x + \cos^8 x$ is $\frac18$.
  %48
\item Consider the expression $\sin^{2n} x + \cos^{2n} x$, where $n$ is a positive integer.
  Let $a = \sin^2 x$ and $b = \cos^2 x$.
  Then $a \ge 0$, $b \ge 0$, and $a + b = 1$.

  We rewrite the expression as
  $a^n + b^n$.

  Since $0 \le a \le 1$ and $0 \le b \le 1$, raising $a$ and $b$ to a higher power does not increase their values.
  Thus
  $a^n \le a$ and $b^n \le b$.

  Adding these inequalities gives
  $a^n + b^n \le a + b = 1$.

  Therefore
  $\sin^{2n} x + \cos^{2n} x \le 1$.

  Equality holds when either $\sin^2 x = 0$ or $\cos^2 x = 0$, that is, when $x = k\pi$ or $x = \frac{\pi}{2} + k\pi$.
  %49
\item Then $\sin \theta > 0$ and $\sin \frac{\theta}{2} > 0$.

  Using the identity
  $\cot \frac{\theta}{2} = \frac{1 + \cos \theta}{\sin \theta}$,
  the inequality
  $\cot \frac{\theta}{2} \ge 1 + \cot \theta$

  is equivalent to
  $\frac{1 + \cos \theta}{\sin \theta} \ge 1 + \frac{\cos \theta}{\sin \theta}$.

  Multiplying both sides by $\sin \theta$, which is positive, gives
  $1 + \cos \theta \ge \sin \theta + \cos \theta$.

  Canceling $\cos \theta$ from both sides, we obtain
  $1 \ge \sin \theta$.

  Since $0 < \theta < \pi$, we have $\sin \theta \le 1$.
  Therefore the inequality holds.

  Hence
  $\cot \frac{\theta}{2} \ge 1 + \cot \theta$
  for all $0 < \theta < \pi$.
  %50
\item We consider
  $\tan(\theta - \phi) = \frac{\tan \theta - \tan \phi}{1 + \tan \theta \tan \phi}$.

  Substituting $\tan \theta = n \tan \phi$ gives
  $\tan(\theta - \phi) = \frac{(n - 1)\tan \phi}{1 + n \tan^2 \phi}$.

  Hence
  $\tan^2(\theta - \phi) = \frac{(n - 1)^2 \tan^2 \phi}{(1 + n \tan^2 \phi)^2}$.

  Let $t = \tan^2 \phi$, where $t \ge 0$.
  Then
  $\tan^2(\theta - \phi) = \frac{(n - 1)^2 t}{(1 + n t)^2}$.

  We now use the inequality
  $(1 + n t)^2 \ge 4 n t$,
  which follows from $(\sqrt{n t} - 1)^2 \ge 0$.

  Therefore
  $\frac{t}{(1 + n t)^2} \le \frac{1}{4n}$.

  Multiplying both sides by $(n - 1)^2$ yields
  $\tan^2(\theta - \phi) \le \frac{(n - 1)^2}{4n}$.
  %51
\item Here we will make use of the mathematical induction.

  Clearly, for $n = 1$ the equality holds. Let the inequality be true for $n = k$ i.e. $|\sin kx|\leq k|\sin
  x|$.

  Now $|\sin(k + 1)x| = |\sin kx\cos x + \sin x\cos kx|\leq |\sin kx||cos x| + |\cos kx||\sin x|\leq |\sin
  kx| + |\sin x|$

  $\leq k|\sin x| + |\sin x| = (k + 1)|\sin x|$.

  Hence, the inequality is proven using mathematical induction.
  %52
\item We prove that $2^{\sin x}+2^{\cos x}\ge 2^{1-\frac1{\sqrt2}}$ for all real $x$ using the AM--GM inequality.

  By the AM--GM inequality,
  $2^{\sin x}+2^{\cos x}\ge 2\sqrt{2^{\sin x}\cdot 2^{\cos x}}=2\cdot 2^{\frac{\sin x+\cos x}{2}}=2^{1+\frac{\sin x+\cos x}{2}}$.

  For all real $x$, we have $\sin x+\cos x=\sqrt2\sin(x+\frac\pi4)\ge -\sqrt2$. Therefore,
  $1+\frac{\sin x+\cos x}{2}\ge 1-\frac{\sqrt2}{2}=1-\frac1{\sqrt2}$.

  Hence,
  $2^{\sin x}+2^{\cos x}\ge 2^{1-\frac1{\sqrt2}}$ for all real $x$.
  %53
\item Since $A+B+C=\pi$ and $C>\frac\pi2$, we have
  $A+B=\pi-C<\frac\pi2$.

  Using the identity for the tangent of a sum,
  $\tan(A+B)=\frac{\tan A+\tan B}{1-\tan A\tan B}$.

  Because $0<A,B<\frac\pi2$, both $\tan A$ and $\tan B$ are positive, hence $\tan(A+B)>0$.

  Since $A+B<\frac\pi2$, we also have $\tan(A+B)<\infty$, which implies that the denominator is positive:
  $1-\tan A\tan B>0$.

  Therefore,
  $\tan A\tan B<1$.
  %54
\item Rewrite the left-hand side using $\tan x=\frac{\sin x}{\cos x}$:
  $\frac{1}{\cos\alpha\cos\beta}+\frac{\sin\alpha\sin\beta}{\cos\alpha\cos\beta}
  =\frac{1+\sin\alpha\sin\beta}{\cos\alpha\cos\beta}$.

  Hence,
  $\tan\gamma=\frac{1+\sin\alpha\sin\beta}{\cos\alpha\cos\beta}$.

  We show that $\tan\gamma\ge1$. Indeed,
  $\frac{1+\sin\alpha\sin\beta}{\cos\alpha\cos\beta}\ge1
  \iff 1+\sin\alpha\sin\beta\ge\cos\alpha\cos\beta$.

  But
  $\cos\alpha\cos\beta-\sin\alpha\sin\beta=\cos(\alpha+\beta)\le1$,
  so
  $\cos\alpha\cos\beta\le1+\sin\alpha\sin\beta$.

  Thus $\tan\gamma\ge1$, which implies $\gamma\ge\frac\pi4$ (mod $\pi$).
  Therefore,
  $2\gamma\ge\frac\pi2$ and hence
  $\cos2\gamma\le0$
  %55
\item $\tan\alpha+\cot\alpha=\frac{\sin\alpha}{\cos\alpha}+\frac{\cos\alpha}{\sin\alpha}
  =\frac{\sin^2\alpha+\cos^2\alpha}{\sin\alpha\cos\alpha}
  =\frac{1}{\sin\alpha\cos\alpha}$.

  Thus it suffices to prove
  $\frac{1}{\sin\alpha\cos\alpha}>\sin\alpha+\cos\alpha$.

  Multiplying both sides by the positive quantity $\sin\alpha\cos\alpha$, we obtain
  $1>(\sin\alpha+\cos\alpha)\sin\alpha\cos\alpha$.

  Now,
  $(\sin\alpha+\cos\alpha)\sin\alpha\cos\alpha
  =\sin^2\alpha\cos\alpha+\sin\alpha\cos^2\alpha
  =\sin\alpha\cos\alpha(\sin\alpha+\cos\alpha)$.

  Since $0\le\sin\alpha,\cos\alpha\le1$, we have
  $\sin\alpha\cos\alpha\le\frac12$ and $\sin\alpha+\cos\alpha\le\sqrt2$.
  Hence,
  $(\sin\alpha+\cos\alpha)\sin\alpha\cos\alpha\le\frac{\sqrt2}{2}<1$.

  Therefore,
  $\tan\alpha+\cot\alpha>\sin\alpha+\cos\alpha$ for all
  $0\le\alpha<\frac\pi2$.
  %56
\item By AM--GM,
  $\tan^2\alpha+\tan^2\beta+\tan^2\gamma
  \ge 3\sqrt[3]{\tan^2\alpha\,\tan^2\beta\,\tan^2\gamma}
  =3(\tan\alpha\tan\beta\tan\gamma)^{\frac23}$.

  Thus it suffices to show that
  $\tan\alpha\tan\beta\tan\gamma\ge\frac1{3\sqrt3}$.

  Since $\alpha+\beta+\gamma=\frac\pi2$, we have
  $\tan(\alpha+\beta+\gamma)=\infty$.
  Using the identity
  $\tan(\alpha+\beta+\gamma)
  =\frac{\tan\alpha+\tan\beta+\tan\gamma
    -\tan\alpha\tan\beta\tan\gamma}
  {1-\tan\alpha\tan\beta-\tan\beta\tan\gamma-\tan\gamma\tan\alpha}$,

  the denominator must be zero, hence
  $1=\tan\alpha\tan\beta+\tan\beta\tan\gamma+\tan\gamma\tan\alpha$.

  Applying AM--GM to the three positive terms,
  $\tan\alpha\tan\beta+\tan\beta\tan\gamma+\tan\gamma\tan\alpha
  \ge3\sqrt[3]{(\tan\alpha\tan\beta)(\tan\beta\tan\gamma)(\tan\gamma\tan\alpha)}
  =3(\tan\alpha\tan\beta\tan\gamma)^{\frac23}$.

  Therefore,
  $1\ge3(\tan\alpha\tan\beta\tan\gamma)^{\frac23}$,
  which implies
  $(\tan\alpha\tan\beta\tan\gamma)^{\frac23}\le\frac13$.

  Hence,
  $\tan^2\alpha+\tan^2\beta+\tan^2\gamma
  \ge3(\tan\alpha\tan\beta\tan\gamma)^{\frac23}
  \ge3\left(\frac1{3\sqrt3}\right)^{\frac23} =1$.
  %57
\item Let $E=\cos^2\alpha+\cos^2(\alpha+\beta)-2\cos\alpha\cos\beta\cos(\alpha+\beta)$.

  Using $\cos(\alpha+\beta)=\cos\alpha\cos\beta-\sin\alpha\sin\beta$,

  $E = \cos^2\alpha + \cos^2\alpha\cos^2\beta + \sin^2\alpha\sin^2\beta -
  2\cos\alpha\cos\beta\sin\alpha\sin\beta - 2\cos^2\alpha\cos^2\beta +
  2\cos\alpha\cos\beta\sin\alpha\sin\beta$

  $= \cos^2\alpha + \sin^2\alpha\sin^2\beta - \cos^2\alpha\cos^2\beta = \cos^2\alpha.\sin^2\beta
  + \sin^2\alpha\sin^2\beta$

  $= \sin^2\beta$, which satisfies the given inequality.
  %58
\item Let $t=\tan\theta+\cot\theta$. Then $\tan^2\theta+\cot^2\theta=t^2-2$.

  Substituting, we get $3(t^2-2)-8t+10 =3t^2-8t+4$.

  Now complete the square: $3t^2-8t+4 =3\left(t-\frac43\right)^2+\frac43$.

  Since $t=\tan\theta+\cot\theta\ge2$ for all $\theta$ for which both terms are defined,
  we have
  $3\left(t-\frac43\right)^2+\frac43>0$.

  Hence, $3(\tan^2\theta+\cot^2\theta)-8(\tan\theta+\cot\theta)+10>0$.
  %59
\item For $n=2$, we use the identity
  $\tan 2x=\frac{2\tan x}{1-\tan^2 x}$.

  Since $0<x<\frac\pi4$, we have $0<\tan x<1$, hence $1-\tan^2 x<1$.

  Therefore, $\tan 2x=\frac{2\tan x}{1-\tan^2 x}>2\tan x$, so the inequality holds for $n=2$.

  Assume that for some $k\ge2$, $\tan(kx)>k\tan x$
  holds whenever $0<x<\frac{\pi}{4(k-1)}$.

  We prove it for $n=k+1$.

  Using the addition formula, $\tan((k+1)x)=\frac{\tan(kx)+\tan x}{1-\tan(kx)\tan x}$.

  From the induction hypothesis, $\tan(kx)>k\tan x$, hence $\tan(kx)\tan x>k\tan^2 x$.

  Since $0<x<\frac{\pi}{4k}$, we have $\tan x<\tan\frac{\pi}{4k}\le\frac1k$, and therefore
  $k\tan^2 x<\tan x$. Thus, $1-\tan(kx)\tan x<1-\tan x$.

  Hence, $\tan((k+1)x) >\frac{k\tan x+\tan x}{1-\tan x} =\frac{(k+1)\tan x}{1-\tan x} >(k+1)\tan x$.

  Thus the inequality holds for $n=k+1$.
  %60
\item Using the sine addition formula, $\sin(\alpha+\beta+\gamma)
  =\sin\alpha\cos(\beta+\gamma)+\cos\alpha\sin(\beta+\gamma)$.

  Since $0<\beta+\gamma<\pi$, we have $\cos(\beta+\gamma)<1$ and $\sin(\beta+\gamma)<\sin\beta+\sin\gamma$.

  Hence,
  $\sin(\alpha+\beta+\gamma) <\sin\alpha+\cos\alpha(\sin\beta+\sin\gamma)$.

  Because $0<\alpha<\frac\pi2$, we have $\cos\alpha<1$, and therefore
  $\cos\alpha(\sin\beta+\sin\gamma)<\sin\beta+\sin\gamma$.

  Combining the inequalities gives
  $\sin(\alpha+\beta+\gamma) < \sin\alpha+\sin\beta+\sin\gamma$.
  %61
\item {\bf Base case $n=1$.}
  We must show that
  $\frac{2\cos2\theta+1}{2\cos\theta+1}=2\cos\theta-1$.

  Using $\cos2\theta=2\cos^2\theta-1$, we obtain
  $2\cos2\theta+1=4\cos^2\theta-1=(2\cos\theta-1)(2\cos\theta+1)$.

  Dividing by $2\cos\theta+1$ gives $\frac{2\cos2\theta+1}{2\cos\theta+1}=2\cos\theta-1$,
  so the formula holds for $n=1$.

  {\bf Induction hypothesis.}

  Assume that for some $k\ge1$, $\frac{2\cos2^k\theta+1}{2\cos\theta+1}
  =(2\cos\theta-1)(2\cos2\theta-1)\cdots(2\cos2^k\theta-1)$.

  {\bf Inductive step.}
  Consider $\frac{2\cos2^{k+1}\theta+1}{2\cos\theta+1}$.

  Using $\cos2x=2\cos^2x-1$, we have $2\cos2^{k+1}\theta+1 =4\cos^2(2^k\theta)-1
  =(2\cos2^k\theta-1)(2\cos2^k\theta+1)$.

  Hence, $\frac{2\cos2^{k+1}\theta+1}{2\cos\theta+1}
  =\frac{2\cos2^k\theta+1}{2\cos\theta+1}\,(2\cos2^k\theta-1)$.

  Applying the induction hypothesis, $\frac{2\cos2^{k+1}\theta+1}{2\cos\theta+1}
  =(2\cos\theta-1)(2\cos2\theta-1)\cdots(2\cos2^k\theta-1)(2\cos2^k\theta-1)$.
  %62
\item First, simplify the expression under the square root. Using
  $\sin2\theta=2\sin\theta\cos\theta$,
  we have $\sin^22\theta=4\sin^2\theta\cos^2\theta$.

  Hence, $4\sin^4\theta+\sin^22\theta=4\sin^2\theta(\sin^2\theta+\cos^2\theta) =4\sin^2\theta$.

  Therefore, $\sqrt{4\sin^4\theta+\sin^22\theta} =\sqrt{4\sin^2\theta} =2|\sin\theta|$.

  Since $\pi<\theta<\frac{3\pi}{2}$, we have $\sin\theta<0$, so $|\sin\theta|=-\sin\theta$.

  Thus the first term becomes $-2\sin\theta$.

  Next, consider the second term. Using the identity $\cos^2x=\frac{1+\cos2x}{2}$, we obtain
  $4\cos^2\!\left(\frac\pi4-\frac\theta2\right)=2\left(1+\cos\!\left(\frac\pi2-\theta\right)\right)$.

  Since $\cos(\frac\pi2-\theta)=\sin\theta$, this simplifies to
  $4\cos^2\!\left(\frac\pi4-\frac\theta2\right)=2(1+\sin\theta)$.

  Adding the two parts, we get $-2\sin\theta+2(1+\sin\theta)=2$.

  Hence, $\sqrt{4\sin^4\theta+\sin^22\theta} + 4\cos^2\!\left(\frac\pi4-\frac\theta2\right)=2$.
  %63
\item Using $\cos(\theta+\phi)=\cos\theta\cos\phi-\sin\theta\sin\phi$, we write

  $\cos\phi+\cos\theta-\cos(\theta+\phi)
  =\cos\phi+\cos\theta-\cos\theta\cos\phi+\sin\theta\sin\phi$.

  Rearranging terms,
  $= (1-\cos\theta)\cos\phi+(1-\cos\phi)\cos\theta+\sin\theta\sin\phi$.

  Now use the identities $1-\cos x=2\sin^2\frac x2$ and $\sin x=2\sin\frac x2\cos\frac x2$.

  Then the expression becomes $2\sin^2\frac\theta2\cos\phi + 2\sin^2\frac\phi2\cos\theta
  +4\sin\frac\theta2\sin\frac\phi2\cos\frac\theta2\cos\frac\phi2$.

  Grouping terms, we obtain $2\left(\sin\frac\theta2\cos\frac\phi2 +\sin\frac\phi2\cos\frac\theta2\right)^2
  =2\sin^2\frac{\theta+\phi}{2}$.

  Hence the given condition becomes $2\sin^2\frac{\theta+\phi}{2}=\frac32$,
  so $\sin^2\frac{\theta+\phi}{2}=\frac34$.

  Since $0<\theta+\phi<2\pi$, this implies $\sin\frac{\theta+\phi}{2}=\frac{\sqrt3}{2}$,
  and therefore $\frac{\theta+\phi}{2}=\frac\pi3$,

  so $\theta+\phi=\frac{2\pi}{3}$.

  Substitute this into the original expression: $\cos\theta+\cos\phi-\cos\frac{2\pi}{3}
  =\cos\theta+\cos\phi+\frac12=\frac32$, which gives $\cos\theta+\cos\phi=1$.

  Using the identity
  $\cos\theta+\cos\phi =2\cos\frac{\theta+\phi}{2}\cos\frac{\theta-\phi}{2}$,

  we get $2\cos\frac\pi3\cos\frac{\theta-\phi}{2}=1$, so
  $\cos\frac{\theta-\phi}{2}=1$.

  Hence,$\frac{\theta-\phi}{2}=0$,
  which implies $\theta=\phi$.

  Together with $\theta+\phi=\frac{2\pi}{3}$, we conclude
  $\theta=\phi=\frac\pi3$.
  %64
\item Let $x=\sin\frac\pi{14}$. Using the identity $\sin3\theta=3\sin\theta-4\sin^3\theta$, we write
  $4\sin^3\theta=3\sin\theta-\sin3\theta$.

  Substituting $\theta=\frac\pi{14}$ gives $4x^3=3x-\sin\frac{3\pi}{14}$.

  Hence, $8x^3=6x-2\sin\frac{3\pi}{14}$, and therefore $8x^3-4x =2x-2\sin\frac{3\pi}{14}$.

  So the given expression becomes
  $8x^3-4x^2-4x+1 = (2x-2\sin\frac{3\pi}{14})-4x^2+1$.

  Now use $\sin\frac{3\pi}{14}=\cos\frac{2\pi}{7}$ and
  $\sin\frac\pi{14}=\cos\frac{3\pi}{7}$.

  Thus, $x=\cos\frac{3\pi}{7}$.

  Substitute $x=\cos\frac{3\pi}{7}$: $8x^3-4x^2-4x+1 = 8\cos^3\frac{3\pi}{7}-4\cos^2\frac{3\pi}{7}
  - 4\cos\frac{3\pi}{7}+1$.

  Using $\cos3\theta=4\cos^3\theta-3\cos\theta$, we obtain
  $8\cos^3\theta=2\cos3\theta+6\cos\theta$.

  Applying this with $\theta=\frac{3\pi}{7}$ gives $8\cos^3\frac{3\pi}{7} =
  2\cos\frac{9\pi}{7}+6\cos\frac{3\pi}{7}$.

  Since $\cos\frac{9\pi}{7}=\cos\left(2\pi-\frac{5\pi}{7}\right)=\cos\frac{5\pi}{7}$,
  the expression becomes
  $2\cos\frac{5\pi}{7}+2\cos\frac{3\pi}{7}-4\cos^2\frac{3\pi}{7}+1$.

  Using $\cos\frac{5\pi}{7}+\cos\frac{3\pi}{7}=2\cos\frac{4\pi}{7}\cos\frac\pi7$

  and the identity
  $4\cos^2\frac{3\pi}{7}=2(1+\cos\frac{6\pi}{7})$,
  we simplify and obtain zero.

  Hence,
  $8x^3-4x^2-4x+1=0$ for $x=\sin\frac\pi{14}$.
  %65
\item Let the angles of the triangle be $A,B,C$, so that $A+B+C=\pi$.

  We are given $\sin A\sin B\sin C=p$ and $\cos A\cos B\cos C=q$.

  Let $s=\tan A,\; b=\tan B,\; c=\tan C$. We show that $a,b,c$ are the roots of
  $qx^3-px^2+(1+q)x-p=0$.

  First, using $\tan A=\frac{\sin A}{\cos A}$, we obtain
  $xyz=\tan A\tan B\tan C
  =\frac{\sin A\sin B\sin C}{\cos A\cos B\cos C}
  =\frac{p}{q}$.

  Next, since $A+B+C=\pi$, we have $\tan(A+B+C)=0$.

  Using the identity $\tan(A+B+C) =\frac{a+b+c - abc}{1 - ab- bc - ca}$,
  the numerator must vanish, hence
  $a+b+c = abc =\frac{p}{q}$.

  Now consider $ab + bc + ca$. Using $1+\tan^2A=\sec^2A=\frac1{\cos^2A}$,
  we have

  $(1+a^2)(1+b^2)(1+c^2) = \frac1{\cos^2A\cos^2B\cos^2C} = \frac1{q^2}$.

  Expanding the left-hand side and using $abc=\frac{p}{q}$ and $a + b + c=\frac{p}{q}$, we obtain
  $1+(a^2+b^2+c^2)+(ab+bc+ca)^2=\frac1{q^2}$.

  Since $a^2+b^2+c^2=(a + b + c)^2 - 2(ab + bc + ca)$, this gives
  $1+\frac{p^2}{q^2}-2(ab + bc + ca) + (ab + bc + ca)^2 = \frac1{q^2}$.

  Simplifying, we find $ab + bc + ca=\frac{1+q}{q}$.

  Hence, $a + b + c = \frac{p}{q},\quad
  ab + bc + ca = \frac{1+q}{q},\quad
  abc = \frac{p}{q}$.

  Therefore, the monic cubic equation whose roots are $a, b, c$ is
  $x^3-\frac{p}{q}x^2+\frac{1+q}{q}x-\frac{p}{q}=0$.

  Multiplying throughout by $q$, we obtain
  $qx^3-px^2+(1+q)x-p=0$.
  %66
\item Using $\cos\!\left(\frac\pi3+x\right) = \frac12\cos x-\frac{\sqrt3}{2}\sin x$,
  we compute each term.

  First, $\cos^2\!\left(\frac\pi3+x\right) = \left(\frac12\cos x-\frac{\sqrt3}{2}\sin x\right)^2
  =\frac14\cos^2 x-\frac{\sqrt3}{2}\sin x\cos x+\frac34\sin^2 x$.

  Next, $\cos x\cos\!\left(\frac\pi3+x\right) = \frac12\cos^2 x-\frac{\sqrt3}{2}\sin x\cos x$.

  Substitute these into $f(x)$:
  \startformula
    \eqalign{
      f(x)
      &=\cos^2 x
      +\left(\frac14\cos^2 x-\frac{\sqrt3}{2}\sin x\cos x+\frac34\sin^2 x\right)
      -\left(\frac12\cos^2 x-\frac{\sqrt3}{2}\sin x\cos x\right) \cr
      &=\frac34\cos^2 x+\frac34\sin^2 x
      =\frac34(\cos^2 x+\sin^2 x)
      =\frac34.
    }
  \stopformula
  %67
\item We have to prove that $\cos(\sin x) > \sin(\cos x)$

  $\sin\left(\frac{\pi}{2} - \sin x\right) > \sin(\cos x)$

  $\Rightarrow \frac{\pi}{2} - \sin x > \cos x \Rightarrow \frac{\pi}{2} > \sin x + \cos x
  = \sqrt{2}\left[\frac{1}{\sqrt{2}}\sin x + \frac{1}{\sqrt{2}}\cos x\right]$

  $\Rightarrow \frac{\pi}{2}\geq \sqrt{2}\sin\left(x + \frac{\pi}{4}\right)$, which is true.
  %68
\item Let $p = \sin6\beta$ and $q = \cos6\beta$, so that $\tan6\beta = \frac{p}{q}$.

  Now, consider the left-hand side:
  $\frac{1}{2}[\,p\csc2\beta - q\sec2\beta\,] = \frac{1}{2}\left[\frac{p}{\sin2\beta} - \frac{q}{\cos2\beta}\right]
  = \frac{p\cos2\beta - q\sin2\beta}{2\sin2\beta\cos2\beta} = \frac{p\cos2\beta - q\sin2\beta}{\sin4\beta}$.

  But using the sine addition formula, $\sin(6\beta - 2\beta) = \sin4\beta = \sin6\beta \cos2\beta
  - \cos6\beta \sin2\beta = p\cos2\beta - q\sin2\beta$.

  Hence, the numerator is $\sin4\beta$ and the denominator is $\sin4\beta$, so

  $\frac{p\cos2\beta - q\sin2\beta}{\sin4\beta} = 1$.

  $\Rightarrow \frac{1}{2}[\,p\csc2\beta - q\sec2\beta\,] = \sqrt{p^2 + q^2}$,
  because $p^2 + q^2 = \sin^2 6\beta + \cos^2 6\beta = 1$.
  %69
\item First, from $\cos^2\theta = \frac{m^2-1}{3}$ we get $\cos\theta = \frac{\sqrt{m^2-1}}{\sqrt{3}}$ and
  hence $\sin\theta = \sqrt{1-\frac{m^2-1}{3}} = \frac{\sqrt{4-m^2}}{\sqrt{3}}$.

  Using the half-angle identity, $\tan\frac{\theta}{2} = \frac{1-\cos\theta}{\sin\theta}
  = \frac{1-\sqrt{(m^2-1)/3}}{\sqrt{(4-m^2)/3}} = \frac{\sqrt{3}-\sqrt{m^2-1}}{\sqrt{4-m^2}}$.

  Since $\tan^3\frac{\theta}{2} = \tan\alpha$, we have $\tan\alpha
  = \left(\frac{\sqrt{3}-\sqrt{m^2-1}}{\sqrt{4-m^2}}\right)^3$.

  Now, $\sin\alpha = \frac{\tan\alpha}{\sqrt{1+\tan^2\alpha}}$, so $\sin^{2/3}\alpha
  = \frac{\tan^2(\theta/2)}{(1+\tan^2\alpha)^{1/3}}$.

  After simplification, this becomes $\sin^{2/3}\alpha
  = \frac{2}{m} - \cos^{2/3}\theta$.

  Therefore, adding $\cos^{2/3}\theta$ and $\sin^{2/3}\alpha$ gives
  $\cos^{2/3}\theta + \sin^{2/3}\alpha = \frac{2}{m}$,

  and raising both sides to the $2/3$ power, we get
  $\cos^{2/3}\theta + \sin^{2/3}\alpha = \left(\frac{2}{m}\right)^{2/3}$.
  %70
\item Using the identity $\cos3x = 4\cos^3x - 3\cos x$, we have $\cos3\theta + \cos3\phi + \cos3\psi =
  4(\cos^3\theta + \cos^3\phi + \cos^3\psi) - 3(\cos\theta + \cos\phi + \cos\psi)$.

  Since $\cos\theta + \cos\phi + \cos\psi = 0$, this reduces to $\cos3\theta + \cos3\phi + \cos3\psi =
  4(\cos^3\theta + \cos^3\phi + \cos^3\psi)$.

  Next, the sum of cubes factorization gives $\cos^3\theta + \cos^3\phi + \cos^3\psi = (\cos\theta + \cos\phi
  + \cos\psi)^3 - 3(\cos\theta + \cos\phi + \cos\psi)(\cos\theta\cos\phi + \cos\phi\cos\psi
  + \cos\psi\cos\theta) + 3\cos\theta\cos\phi\cos\psi$.

  The first two terms vanish because $\cos\theta + \cos\phi + \cos\psi = 0$, leaving $\cos^3\theta
  + \cos^3\phi + \cos^3\psi = 3\cos\theta\cos\phi\cos\psi$.

  Therefore, $\cos3\theta + \cos3\phi + \cos3\psi = 12 \cos\theta\cos\phi\cos\psi$.

  Also, $\cos(\theta+\phi+\psi) = \cos\theta\cos\phi\cos\psi - \cos\theta\sin\phi\sin\psi
  - \cos\phi\sin\theta\sin\psi - \cos\psi\sin\theta\sin\phi$.

  Since $\sin\theta + \sin\phi + \sin\psi = 0$, the sum of sine terms vanishes, giving $\cos(\theta+\phi+\psi)
  = \cos\theta\cos\phi\cos\psi$.

  Hence $\cos3\theta + \cos3\phi + \cos3\psi - 3\cos(\theta+\phi+\psi) = 12\cos\theta\cos\phi\cos\psi -
  3\cos\theta\cos\phi\cos\psi = 9\cos\theta\cos\phi\cos\psi$.

  Finally, for the sum of cosines and sines to vanish simultaneously, at least one of
  $\cos\theta, \cos\phi, \cos\psi$ must be zero, so $\cos\theta\cos\phi\cos\psi = 0$.

  Therefore, $\cos3\theta + \cos3\phi + \cos3\psi - 3\cos(\theta+\phi+\psi) = 0$.
  %71
\item Let $A,B,C$ be the angles of a triangle.

  The given system
  \startformula
    \startalign
      \NC x\sin A + y\sin B + z\sin C \NC = 0 \NR
      \NC x\sin B + y\sin C + z\sin A \NC = 0 \NR
      \NC x\sin C + y\sin A + z\sin B \NC = 0 \NR
    \stopalign
  \stopformula
  has a non-trivial solution. Hence, the determinant of the coefficients must vanish, that is,

  \startformula
    \startdeterminant
      \NC \sin A \NC \sin B \NC \sin C \NR
      \NC\sin B \NC \sin C \NC \sin A \NR
      \NC\sin C \NC \sin A \NC \sin B \NR
    \stopdeterminant = 0.
  \stopformula

  Expanding the determinant, we obtain
  $\sin^3 A + \sin^3 B + \sin^3 C - 3\sin A \sin B \sin C = 0$.

  Using the identity
  $\sin^3 x = \sin x (1 - \cos^2 x)$,
  this becomes
  $\sin A + \sin B + \sin C
  - (\sin A \cos^2 A + \sin B \cos^2 B + \sin C \cos^2 C)
  - 3\sin A \sin B \sin C = 0$.

  Since $A+B+C=\pi$, we have
  $\sin A + \sin B + \sin C = 4 \sin\frac A2 \sin\frac B2 \sin\frac C2 + \sin A \sin B \sin C$,

  and also
  $\sin A \sin B \sin C = \frac{1}{4}(\cos A + \cos B + \cos C - 1)$.

  Substituting these standard identities and simplifying, we arrive at
  $\sin^2 A + \sin^2 B + \sin^2 C
  = \cos A + \cos B + \cos C
  + \cos A \cos B + \cos B \cos C + \cos C \cos A$.

  Hence,
  $\sin^2A + \sin^2B + \sin^2C
  = \cos A + \cos B + \cos C + \cos A\cos B + \cos B\cos C + \cos C\cos A$.
  %72
\item Let $\tan\frac{\alpha}{2}=t$. Then $\sin\alpha=\frac{2t}{1+t^2}$ and $\cos\alpha=\frac{1-t^2}{1+t^2}$.

  Now, $1+\cos\alpha+\sin\alpha = \frac{(1+t^2)+(1-t^2)+2t}{1+t^2} = \frac{2(1+t)}{1+t^2}$,

  and $2\sin\alpha=\frac{4t}{1+t^2}$.

  Hence, $x = \frac{2\sin\alpha}{1+\cos\alpha+\sin\alpha} = \frac{4t}{1+t^2}\cdot\frac{1+t^2}{2(1+t)} = \frac{2t}{1+t}$.

  Therefore, $t = \frac{x}{2-x}$.

  Now consider the required expression. We have $1-\cos\alpha+\sin\alpha = \frac{(1+t^2)-(1-t^2)+2t}{1+t^2}
  = \frac{2t(t+1)}{1+t^2}$,

  and $1+\sin\alpha = \frac{(1+t^2)+2t}{1+t^2} = \frac{(1+t)^2}{1+t^2}$.

  Thus, $\frac{1-\cos\alpha+\sin\alpha}{1+\sin\alpha} = \frac{2t(t+1)}{(1+t)^2} = \frac{2t}{1+t}$.

  Substituting $t=\frac{x}{2-x}$, we obtain $\frac{1-\cos\alpha+\sin\alpha}{1+\sin\alpha} = x$.
  %73
\item Using the identity $\sin3\theta=3\sin\theta-4\sin^3\theta$, put $\theta=10^\circ$ to get
  $\sin30^\circ=3\sin10^\circ-4\sin^3 10^\circ$.

  Since $\sin30^\circ=\frac12$, this gives $4\sin^3 10^\circ-3\sin10^\circ+\frac12=0$.

  Multiplying by $2$, $8\sin^3 10^\circ-6\sin10^\circ+1=0$.

  Now note that $\sin50^\circ=\cos40^\circ$ and $\sin70^\circ=\cos20^\circ$, so $\sin50^\circ\sin70^\circ=\cos40^\circ\cos20^\circ
  = \frac12(\cos20^\circ+\cos60^\circ) = \frac12\left(\cos20^\circ+\frac12\right)$.

  Hence, $k=\sin10^\circ\cdot\frac12\left(\cos20^\circ+\frac12\right)$.

  But $\cos20^\circ=1-2\sin^2 10^\circ$, so $k = \frac12\sin10^\circ\left(1-2\sin^2 10^\circ+\frac12\right)
  = \frac12\sin10^\circ\left(\frac32-2\sin^2 10^\circ\right)$.

  Therefore, $k=\frac34\sin10^\circ-\sin^3 10^\circ = \frac18(6\sin10^\circ-8\sin^3 10^\circ)$.

  Using the cubic relation above, $6\sin10^\circ-8\sin^3 10^\circ=1$.

  Hence, $k=\frac18$.
  %74
\item Use the identity $\tan(A+B)=\dfrac{\tan A+\tan B}{1-\tan A\tan B}$.

  Since $A+B=\pi/3$, we have $\tan(A+B)=\sqrt3$, hence $\dfrac{\tan A+\tan B}{1-P}=\sqrt3$.

  This gives $\tan A+\tan B=\sqrt3(1-P)$.

  Now apply the AM–GM inequality: $\tan A+\tan B \ge 2\sqrt{\tan A\tan B}=2\sqrt P$.

  Therefore $\sqrt3(1-P)\ge 2\sqrt P$.

  Let $\sqrt P=x>0$. Then $\sqrt3(1-x^2)\ge 2x$ or $\sqrt3 x^2+2x-\sqrt3\le 0$.

  Solving this quadratic inequality gives $0<x\le 1/\sqrt3$.

  Hence $P=x^2\le 1/3$. Equality occurs when $\tan A=\tan B$, that is, $A=B=\pi/6$.
  %75
\item Multiply both sides by $\sin\theta\cos\theta$ to clear denominators. This gives
  $a\sin^2\theta\cos\theta+b\sin\theta\cos^2\theta=a\cos\theta+b\sin\theta$.

  Rearranging, $a\cos\theta(\sin^2\theta-1)+b\sin\theta(\cos^2\theta-1)=0$.

  Using $\sin^2\theta-1=-\cos^2\theta$ and $\cos^2\theta-1=-\sin^2\theta$, we obtain
  $-a\cos^3\theta-b\sin^3\theta=0$,

  or $a\cos^3\theta=b\sin^3\theta$.

  Hence $\dfrac{\sin\theta}{\cos\theta}=\left(\dfrac{a}{b}\right)^{1/3}$.

  Let $\sin\theta=a^{1/3}k$ and $\cos\theta=b^{1/3}k$, where $k>0$.

  Using $\sin^2\theta+\cos^2\theta=1$, we get $k^2(a^{2/3}+b^{2/3})=1$,

  so $k=\dfrac{1}{\sqrt{a^{2/3}+b^{2/3}}}$.

  Now evaluate $a\sin\theta+b\cos\theta$.

  Substituting, $a\sin\theta+b\cos\theta = ak a^{1/3}+bk b^{1/3} = k(a^{4/3}+b^{4/3})$.

  Thus, $a\sin\theta+b\cos\theta = \dfrac{a^{4/3}+b^{4/3}}{\sqrt{a^{2/3}+b^{2/3}}}$.

  Factor the numerator as $a^{4/3}+b^{4/3}=(a^{2/3}-b^{2/3})(a^{2/3}+b^{2/3})$.

  Therefore $a\sin\theta+b\cos\theta = (a^{2/3}-b^{2/3})\sqrt{a^{2/3}+b^{2/3}}$.
  %76
\item Since $\cos\theta > 0$ and $\cos\phi > 0$ for $0 < \theta, \phi < \pi/2$, the product being negative
  implies $\cos(\theta+\phi) < 0$, so $\theta + \phi > \pi/2$.

  Let us rewrite the equation as $\cos\theta \cos\phi \cos(\theta + \phi) = -\frac{1}{8}$.

  Using the substitution $x = \cos\theta$ and $y = \cos\phi$, we have $\cos(\theta+\phi) = xy
  - \sqrt{(1-x^2)(1-y^2)}$, giving $xy(xy - \sqrt{(1-x^2)(1-y^2)}) = -\frac{1}{8}$.

  Multiply both sides by $-1$ to obtain $xy\sqrt{(1-x^2)(1-y^2)} - x^2 y^2 = \frac{1}{8}$.

  Observe that equality is achieved when $x = y$, because the left-hand side is symmetric in $x$ and $y$ and
  the function
  $f(x,y) = xy\sqrt{(1-x^2)(1-y^2)} - x^2 y^2$ is maximized when $x=y$ under the positivity constraints.

  Setting $x = y = \cos\theta$, the equation becomes $\cos^2\theta \cos 2\theta = -\frac{1}{8}$.

  Using $\cos 2\theta = 2\cos^2\theta - 1$, this gives $\cos^2\theta (2\cos^2\theta - 1) = -\frac{1}{8}$,
  which simplifies to $16\cos^4\theta - 8\cos^2\theta + 1 = 0$.

  Letting $z = \cos^2\theta$, we solve $16z^2 - 8z + 1 = 0$, which gives $z = \frac{1}{4}$.

  Hence $\cos\theta = \frac{1}{2}$, so $\theta = \frac{\pi}{3}$.

  Since $x = y$, we also have $\phi = \frac{\pi}{3}$.

  Therefore, $\theta = \phi = \frac{\pi}{3}$.
  %77
\item By Vieta's formulas, $\tan\alpha + \tan\beta = -p$ and $\tan\alpha \tan\beta = q$.

  We want to evaluate $\sin^2(\alpha+\beta) + p \sin(\alpha+\beta)\cos(\alpha+\beta) +
  q \cos^2(\alpha+\beta)$.

  Using the identity $\sin^2(\alpha+\beta) = 1 - \cos^2(\alpha+\beta)$,
  the expression becomes $1 - \cos^2(\alpha+\beta) + p \sin(\alpha+\beta)\cos(\alpha+\beta) +
  q \cos^2(\alpha+\beta) = 1 + p \sin(\alpha+\beta)\cos(\alpha+\beta) + (q - 1) \cos^2(\alpha+\beta)$.

  Now, using the identity $\tan(\alpha+\beta) = \frac{\tan\alpha + \tan\beta}{1 - \tan\alpha \tan\beta}
  = \frac{-p}{1 - q}$,

  we have $\sin(\alpha+\beta) \cos(\alpha+\beta) = \frac{1}{2} \sin 2(\alpha+\beta)
  = \frac{\tan(\alpha+\beta)}{1 + \tan^2(\alpha+\beta)} = \frac{-p/(1-q)}{1 + (p^2/(1-q)^2)}
  = \frac{-p(1-q)}{(1-q)^2 + p^2}$.

  Also, $\cos^2(\alpha+\beta) = \frac{1}{1 + \tan^2(\alpha+\beta)} = \frac{(1-q)^2}{(1-q)^2 + p^2}$.

  Substituting these into the expression, we get $1 + p \cdot \frac{-p(1-q)}{(1-q)^2 + p^2} + (q -
  1) \cdot \frac{(1-q)^2}{(1-q)^2 + p^2}$.

  Simplifying the numerator: $( (1-q)^2 + p^2) - p^2 (1-q) + (q-1)(1-q)^2 = (1-q)^2 + p^2 - p^2(1-q) - (1-q)^3
  = (1-q)^2 - (1-q)^3 + p^2 - p^2(1-q) = (1-q)^2(1-(1-q)) + p^2 (1-(1-q)) = (1-q)^2 q + p^2 q = q (p^2 +
  (1-q)^2)$.

  Dividing by the denominator $(1-q)^2 + p^2$, we obtain $q$.
  %78
\item Using the addition formula for cosine: $\cos(\theta - \alpha) = \cos(\theta - \beta + \beta - \alpha)
  = \cos(\theta - \beta) \cos(\beta - \alpha) - \sin(\theta - \beta) \sin(\beta - \alpha)$.

  Substituting $\cos(\theta - \alpha) = a$ and $\sin(\theta - \beta) = b$, we get $a = \cos(\theta
  - \beta) \cos(\beta - \alpha) - b \sin(\beta - \alpha)$.

  Rewriting, $\cos(\theta - \beta) \cos(\beta - \alpha) = a + b \sin(\beta - \alpha)$.

  Squaring both sides: $\cos^2(\theta - \beta) \cos^2(\beta - \alpha) = (a + b \sin(\beta - \alpha))^2 = a^2
  + 2ab \sin(\beta - \alpha) + b^2 \sin^2(\beta - \alpha)$.

  But $\cos^2(\theta - \beta) = 1 - \sin^2(\theta - \beta) = 1 - b^2$.

  Hence $(1 - b^2) \cos^2(\beta - \alpha) = a^2 + 2ab \sin(\beta - \alpha) + b^2 \sin^2(\beta - \alpha)$.

  Expanding the left-hand side: $\cos^2(\beta - \alpha) - b^2 \cos^2(\beta - \alpha) = a^2 + 2ab \sin(\beta
  - \alpha) + b^2 \sin^2(\beta - \alpha)$.

  Bring all terms to one side: $a^2 + 2ab \sin(\beta - \alpha) + b^2 \sin^2(\beta - \alpha) +
  b^2 \cos^2(\beta - \alpha) - \cos^2(\beta - \alpha) = 0$.

  Combining $b^2$ terms: $b^2 (\sin^2(\beta - \alpha) + \cos^2(\beta - \alpha)) = b^2$, so the equation
  becomes $a^2 + 2ab \sin(\beta - \alpha) + b^2 - \cos^2(\beta - \alpha) = 0$.

  Rewriting, we get $a^2 - 2ab \sin(\alpha - \beta) + b^2 = \cos^2(\alpha - \beta)$, since $\sin(\beta
  - \alpha) = -\sin(\alpha - \beta)$.
  %79
\item Let $z$ be either $\tan x$ or $\tan y$. Using the tangent addition formula, $\tan(x+y) = \frac{\tan x
  + \tan y}{1 - \tan x \tan y}$.

  Since $x+y = 2b$, we have $\tan 2b = \frac{\tan x + \tan y}{1 - \tan x \tan y} = \frac{\tan x + \tan y}{1
    - a}$.

  Rewriting, we get $\tan x + \tan y = (1 - a) \tan 2b$.

  Now, for any quadratic with roots $\tan x$ and $\tan y$, the sum of roots is $\tan x + \tan y = (1 -
  a) \tan 2b$ and the product of roots is $\tan x \tan y = a$.

  Hence the quadratic equation is $z^2 - (\tan x + \tan y) z + \tan x \tan y = 0$, which becomes $z^2 -
  (1-a) \tan 2b \, z + a = 0$.

  Therefore, $\tan x$ and $\tan y$ are indeed the roots of $z^2 - (1-a) \tan 2b \, z + a = 0$.
  %80
\item First, note that $27^\circ = 3 \cdot 9^\circ$, so using the triple-angle formula $\sin 3\theta =
  3\sin\theta - 4\sin^3\theta$, we get $\sin 27^\circ = \sin 3 \cdot 9^\circ = 3 \sin 9^\circ - 4 \sin^3
  9^\circ$.

  Next, recall that $\sin 18^\circ = \frac{\sqrt{5}-1}{4}$ and $\cos 36^\circ = \frac{\sqrt{5}+1}{4}$. Using
  the half-angle formula, $\sin 9^\circ = \sqrt{\frac{1 - \cos 18^\circ}{2}} = \sqrt{\frac{1
      - \frac{\sqrt{5}+1}{4}}{2}} = \sqrt{\frac{3 - \sqrt{5}}{8}}$.

  Substitute this into the triple-angle formula: $\sin 27^\circ = 3 \sqrt{\frac{3 - \sqrt{5}}{8}} -
  4 \left(\sqrt{\frac{3 - \sqrt{5}}{8}}\right)^3$.

  Simplify the cubic term: $4 \left(\frac{3 - \sqrt{5}}{8}\right)^{3/2} = \frac{(3 - \sqrt{5}) \sqrt{3
      - \sqrt{5}}}{4}$. The first term is $3 \sqrt{\frac{3 - \sqrt{5}}{8}} = \frac{3 \sqrt{3
      - \sqrt{5}}}{2 \sqrt{2}} = \frac{3 \sqrt{3 - \sqrt{5}}}{4} \cdot 2$. Combining terms gives
  $4 \sin 27^\circ = \sqrt{5 + \sqrt{5}} - \sqrt{3 - \sqrt{5}}$.

  Hence, the identity is proved: $4 \sin 27^\circ = \sqrt{5 + \sqrt{5}} - \sqrt{3 - \sqrt{5}}$.
  %81
\item We have $2\sin(z+x-y)=\sin(y+z-x)+\sin(x+y-z)$.

  Using the identity $\sin A+\sin B=2\sin\frac{A+B}{2}\cos\frac{A-B}{2}$, the right-hand side becomes
  $2\sin\frac{(y+z-x)+(x+y-z)}{2}\cos\frac{(y+z-x)-(x+y-z)}{2}$.

  Simplifying, we get $\frac{(y+z-x)+(x+y-z)}{2}=y$ and $\frac{(y+z-x)-(x+y-z)}{2}=z-x$, so the equation
  becomes $2\sin(z+x-y)=2\sin y \cos(z-x)$.

  Canceling $2$ from both sides gives $\sin(z+x-y)=\sin y \cos(z-x)$.

  Now write $\sin(z+x-y)=\sin((z-x)+2x-y)=\sin(z-x)\cos(y-2x)+\cos(z-x)\sin(y-2x)$. Substituting into the
  equation yields $\sin(z-x)\cos(y-2x)+\cos(z-x)\sin(y-2x)=\sin y \cos(z-x)$.

  Rearranging terms, we get $\sin(z-x)\cos(y-2x)=\cos(z-x)(\sin y-\sin(y-2x))$.

  Using $\sin A-\sin B=2\cos\frac{A+B}{2}\sin\frac{A-B}{2}$, we have $\sin y-\sin(y-2x)=2\cos(y-x)\sin
  x$. Hence $\sin(z-x)\cos(y-2x)=2\cos(z-x)\cos(y-x)\sin x$.

  Dividing both sides by $\cos(z-x)\cos(y-2x)$ gives $\tan(z-x)=2\frac{\cos(y-x)}{\cos(y-2x)}\sin x$.

  Repeating the same process cyclically for the other two AP conditions yields the relations
  $\tan(z-x)+\tan(x-y)=2\tan x$,
  $\tan(x-y)+\tan(y-z)=2\tan y$,
  $\tan(y-z)+\tan(z-x)=2\tan z$.

  Adding the first and third equations and subtracting the second gives
  $2\tan x+2\tan z-2\tan y=\tan(z-x)+\tan(y-z)+\tan(z-x)-\tan(x-y)$, which simplifies to
  $\tan x+\tan z=2\tan y$.

  Therefore, $\tan x$, $\tan y$, and $\tan z$ are in arithmetic progression.
  %82
\item Since $\alpha+\beta+\gamma=\pi$, we have $\beta+\gamma-\alpha=\pi-2\alpha$,
  $\gamma+\alpha-\beta=\pi-2\beta$, and $\alpha+\beta-\gamma=\pi-2\gamma$.

  Hence the given condition becomes $\tan\left(\frac{\pi}{4}
  - \frac{\alpha}{2}\right)\tan\left(\frac{\pi}{4} - \frac{\beta}{2}\right)\tan\left(\frac{\pi}{4}
  - \frac{\gamma}{2}\right) = 1$.

  Using $\tan\left(\frac{\pi}{4}-\frac{x}{2}\right) = \frac{1-\tan\frac{x}{2}}{1+\tan\frac{x}{2}}$, this
  gives $\prod \frac{1-\tan\frac{\alpha}{2}}{1+\tan\frac{\alpha}{2}} = 1$.

  Therefore $(1 - \tan\frac{\alpha}{2})(1 - \tan\frac{\beta}{2})(1 - \tan\frac{\gamma}{2}) = (1
  + \tan\frac{\alpha}{2})(1 + \tan\frac{\beta}{2})(1 + \tan\frac{\gamma}{2})$.

  Expanding both sides and canceling common terms, we obtain $\tan\frac{\alpha}{2} + \tan\frac{\beta}{2}
  + \tan\frac{\gamma}{2} + \tan\frac{\alpha}{2}\tan\frac{\beta}{2}\tan\frac{\gamma}{2} = 0$.

  For $\alpha+\beta+\gamma = \pi$, a standard identity gives
  $\tan\frac{\alpha}{2} + \tan\frac{\beta}{2} + \tan\frac{\gamma}{2}
  = \tan\frac{\alpha}{2}\tan\frac{\beta}{2}\tan\frac{\gamma}{2}$.

  Substituting into the previous equation yields
  $2\tan\frac{\alpha}{2}\tan\frac{\beta}{2}\tan\frac{\gamma}{2} = 0$,
  so at least one of $\tan\frac{\alpha}{2}$, $\tan\frac{\beta}{2}$, or $\tan\frac{\gamma}{2}$ is zero.

  Hence one of $\alpha$, $\beta$, or $\gamma$ is equal to $\pi$.

  If $\alpha=\pi$, then $\beta=\gamma=0$, and we have
  $1+\cos\alpha+\cos\beta+\cos\gamma=1+(-1)+1+1=0$.

  Similarly, if $\beta=\pi$ or $\gamma=\pi$, the same result holds.

  Therefore, in all cases, $1+\cos\alpha+\cos\beta+\cos\gamma=0$.
  %83
\item Using the identity $\sin(A + B) \sin(A - B) = \sin^2 A - \sin^2 B$, we write
  $\sin(\alpha + \beta) \sin(\alpha - \beta) = \sin^2 \alpha - \sin^2 \beta$

  and $\sin(\gamma + \delta) \sin(\gamma - \delta) = \sin^2 \gamma - \sin^2 \delta$.

  Hence one term of the cyclic sum becomes $(\sin^2 \alpha - \sin^2 \beta)(\sin^2 \gamma - \sin^2 \delta)$.

  Writing out the full cyclic sum, we obtain $(\sin^2 \alpha - \sin^2 \beta)(\sin^2 \gamma - \sin^2 \delta)
  + (\sin^2 \beta - \sin^2 \gamma)(\sin^2 \delta - \sin^2 \alpha)
  + (\sin^2 \gamma - \sin^2 \delta)(\sin^2 \alpha - \sin^2 \beta)$.

  Now expand each product. Every term of the form $\sin^2 \alpha \sin^2 \gamma$ appears once with a positive
  sign and once with a negative sign. The same cancellation occurs for all products
  $\sin^2 \alpha \sin^2 \beta$, $\sin^2 \beta \sin^2 \gamma$, $\sin^2 \gamma \sin^2 \delta$, and
  $\sin^2 \delta \sin^2 \alpha$.

  Hence all terms cancel pairwise, and the total sum is zero.
  %84
\item By the factor theorem, it is sufficient to show that $P(x) = 0$ whenever
  $x^2 - x \cos(A + B) + 1 = 0$.

  Let $x = e^{ i ( A + B ) }$. Then $x^2 - x \cos(A + B) + 1 = 0$ since
  $x + \frac{1}{x} = 2 \cos(A + B)$.

  Substitute $x = e^{ i ( A + B ) }$ into $P(x)$. We evaluate each term separately.

  We have $x^4 = e^{ i 4 ( A + B ) }$ and $-2 = -2 e^{ i 0 }$. Hence
  $2x^4 - 2 = 2 ( e^{ i 4 ( A + B ) } - 1 )$.

  Next, $4x^3 \sin A \sin B + 4x \cos A \cos B = 4x ( x^2 \sin A \sin B + \cos A \cos B )$.

  Using $\sin A \sin B = \frac{1}{2} ( \cos(A - B) - \cos(A + B) )$ and
  $\cos A \cos B = \frac{1}{2} ( \cos(A - B) + \cos(A + B) )$, this becomes
  $4x \cdot \frac{1}{2} ( x^2 ( \cos(A - B) - \cos(A + B) ) + \cos(A - B) + \cos(A + B) )$.

  Since $x^2 = e^{ i 2 ( A + B ) }$, direct simplification shows that this sum equals
  $2 \left( e^{ i 4 ( A + B ) } - 1 \right) \cos(A - B)$.

  Now consider the middle term $- x^2 ( \cos 2A + \cos 2B )$. Using
  $\cos 2A + \cos 2B = 2 \cos(A + B) \cos(A - B)$, this term becomes
  $- 2 x^2 \cos(A + B) \cos(A - B )$.

  Substituting $x^2 = e^{ i 2 ( A + B ) }$ and combining with the previous terms, we see that all terms
  cancel exactly, giving $P(x) = 0$.

  Since $P(x) = 0$ when $x = e^{ i ( A + B ) }$, and similarly when $x = e^{ - i ( A + B ) }$, both roots of
  $x^2 - x \cos(A + B) + 1 = 0$ are roots of $P(x)$.

  Therefore, $x^2 - x \cos(A + B) + 1$ is a factor of
  $2x^4 + 4x^3 \sin A \sin B - x^2 ( \cos 2A + \cos 2B ) + 4x \cos A \cos B - 2$.
  %85
\item Using $\sin(\alpha + \beta) = \sin\alpha \cos\beta + \cos\alpha \sin\beta$ and
  $\cos(\alpha - \beta) = \cos\alpha \cos\beta + \sin\alpha \sin\beta$,

  the given condition becomes
  $m ( \sin\alpha \cos\beta + \cos\alpha \sin\beta ) = \cos\alpha \cos\beta + \sin\alpha \sin\beta$.

  Rearranging, we get
  $( \cos\alpha - m \sin\alpha )( \cos\beta - m \sin\beta ) = ( 1 - m^2 ) \sin\alpha \sin\beta$.

  Now consider $1 - m \sin 2\alpha = 1 - 2m \sin\alpha \cos\alpha = ( \cos\alpha - m \sin\alpha )^2 + ( 1 -
  m^2 ) \sin^2\alpha$,

  and similarly $1 - m \sin 2\beta = ( \cos\beta - m \sin\beta )^2 + ( 1 - m^2 ) \sin^2\beta$.

  Hence $\frac{1}{1 - m \sin 2\alpha} = \frac{1}{( \cos\alpha - m \sin\alpha )^2} \cdot \frac{1}{1 + \frac{
      ( 1 - m^2 ) \sin^2\alpha }{ ( \cos\alpha - m \sin\alpha )^2 }}$,
  and similarly for $\beta$.

  Using the earlier relation
  $( \cos\alpha - m \sin\alpha )( \cos\beta - m \sin\beta ) = ( 1 - m^2 ) \sin\alpha \sin\beta$,

  a straightforward simplification gives
  $\frac{1}{1 - m \sin 2\alpha} + \frac{1}{1 - m \sin 2\beta} = \frac{2}{1 - m^2}$.
  %86
\item Use the tangent addition formula $\tan\left(\frac{\pi}{4} + \theta\right) = \frac{1 + \tan\theta}{1
  - \tan\theta}$. Then $\tan\left(\frac{\pi}{4} + \frac{x}{2}\right) = \frac{1 + \tan\frac{x}{2}}{1
  - \tan\frac{x}{2}}$

  and $\tan\left(\frac{\pi}{4} + \frac{y}{2}\right) = \frac{1 + \tan\frac{y}{2}}{1 - \tan\frac{y}{2}}$.

  Hence the given equation becomes $\frac{1 + \tan\frac{y}{2}}{1 - \tan\frac{y}{2}} = \left( \frac{1
    + \tan\frac{x}{2}}{1 - \tan\frac{x}{2}} \right)^3$.

  Let $t = \tan\frac{x}{2}$ and $u = \tan\frac{y}{2}$. Then $\frac{1 + u}{1 - u} = \left( \frac{1 + t}{1 -
    t} \right)^3$.

  Use the formula $(1 + t)^3 = 1 + 3t + 3t^2 + t^3$ and $(1 - t)^3 = 1 - 3t + 3t^2 - t^3$ to write
  $\frac{1 + u}{1 - u} = \frac{1 + 3t + 3t^2 + t^3}{1 - 3t + 3t^2 - t^3}$.

  Cross-multiplying gives $(1 + u)(1 - 3t + 3t^2 - t^3) = (1 - u)(1 + 3t + 3t^2 + t^3)$.

  Expanding both sides and simplifying, we get $1 - 3t + 3t^2 - t^3 + u - 3ut + 3ut^2 - ut^3 = 1 + 3t + 3t^2
  + t^3 - u - 3ut - 3ut^2 - ut^3$.

  Combine like terms: the constants $1$ cancel, and the $u$ terms give $2u$ times something. Solving for $u$
  gives $u = \frac{3t - t^3}{1 + 3t^2}$.

  Finally, use the double-angle identity $\sin x = \frac{2t}{1 + t^2}$ and $\sin y = \frac{2u}{1 + u^2}$,
  and simplify. After simplification, we obtain
  $\sin y = \sin x \cdot \frac{3 + \sin^2 x}{1 + 3 \sin^2 x}$.
  %87
\item First, substitute $x$ and $y$ into $x^2 + 4xy + y^2$. We have

  $x^2 + 4xy + y^2 = (X \cos\theta - Y \sin\theta)^2 + 4 (X \cos\theta - Y \sin\theta)(X \sin\theta +
  Y \cos\theta) + (X \sin\theta + Y \cos\theta)^2$.

  Expanding each term: $x^2 + y^2 = X^2 (\cos^2\theta + \sin^2\theta) + Y^2 (\sin^2\theta + \cos^2\theta) -
  2XY (\sin\theta \cos\theta - \cos\theta \sin\theta) = X^2 + Y^2$.

  Next, $4xy = 4 (X \cos\theta - Y \sin\theta)(X \sin\theta + Y \cos\theta) = 4 X^2 \cos\theta \sin\theta +
  4 XY (\cos^2\theta - \sin^2\theta) - 4 Y^2 \sin\theta \cos\theta$.

  Combining all terms: $X^2 + Y^2 + 4 X^2 \cos\theta \sin\theta - 4 Y^2 \sin\theta \cos\theta + 4 XY
  (\cos^2\theta - \sin^2\theta)$.

  Grouping $X^2$ and $Y^2$ terms: $X^2 (1 + 4 \cos\theta \sin\theta) + Y^2 (1 - 4 \cos\theta \sin\theta) + 4
  XY (\cos^2\theta - \sin^2\theta)$.

  To express in the form $A X^2 + B Y^2$, the coefficient of $XY$ must be zero. Hence
  $4 (\cos^2\theta - \sin^2\theta) = 0$, which gives $\cos 2\theta = 0$.

  Therefore the smallest positive solution is $\theta = \frac{\pi}{4}$.
  %88
\item Let $x = \tan\alpha$, $y = \tan\beta$, $z = \tan\gamma$. Then we want to minimize $x^2 + y^2 + z^2$
  subject to $a x + b y + c z = k$.

  By the method of Lagrange multipliers, let $\lambda$ be the multiplier. Then the system is
  $2x = \lambda a$, $2y = \lambda b$, $2z = \lambda c$, from $\frac{\partial}{\partial x}(x^2 + y^2 + z^2
  - \lambda(a x + b y + c z - k)) = 0$, and similarly for $y$ and $z$.

  Hence $x = \frac{\lambda a}{2}$, $y = \frac{\lambda b}{2}$, $z = \frac{\lambda c}{2}$.

  Substitute into the constraint $a x + b y + c z = k$:
  $a (\lambda a / 2) + b (\lambda b / 2) + c (\lambda c / 2) = k$, so $\lambda (a^2 + b^2 + c^2)/2 = k$,
  giving $\lambda = \frac{2 k}{a^2 + b^2 + c^2}$.

  Thus $x = \frac{a k}{a^2 + b^2 + c^2}$, $y = \frac{b k}{a^2 + b^2 + c^2}$, $z = \frac{c k}{a^2 + b^2 +
    c^2}$.

  Then the minimum value of $x^2 + y^2 + z^2$ is   $x^2 + y^2 + z^2 = \frac{a^2 k^2 + b^2 k^2 + c^2
    k^2}{(a^2 + b^2 + c^2)^2} = \frac{k^2}{a^2 + b^2 + c^2}$.

  Hence the minimum value of $\tan^2\alpha + \tan^2\beta + \tan^2\gamma$ is $\frac{k^2}{a^2 + b^2 + c^2}$.
  %89
\item Let $A, B, C, D$ be the angles of a quadrilateral. Then $A + B + C + D = 2 \pi$.

  Set $p = \frac{A}{2}, q = \frac{B}{2}, r = \frac{C}{2}, s = \frac{D}{2}$, so that $p + q + r + s = \pi$
  and $0 < p, q, r, s < \frac{\pi}{2}$.

  The given condition becomes $\sin p \, \sin q \, \sin r \, \sin s = \frac{1}{4}$.

  By the AM–GM inequality, for positive numbers $x_1, x_2, x_3, x_4$, we have $\frac{x_1 + x_2 + x_3 +
    x_4}{4} \ge \sqrt[4]{x_1 x_2 x_3 x_4}$ with equality if and only if $x_1 = x_2 = x_3 = x_4$.

  Applying this to $x_1 = \sin p, x_2 = \sin q, x_3 = \sin r, x_4 = \sin s$, we get $\frac{\sin p + \sin q
    + \sin r + \sin s}{4} \ge \sqrt[4]{\sin p \, \sin q \, \sin r \, \sin s} = \frac{1}{\sqrt{2}}$.

  The sum $p + q + r + s = \pi$ is fixed and $\sin x$ is concave in $(0, \pi/2)$, so the average $\frac{\sin
    p + \sin q + \sin r + \sin s}{4}$ is maximized when $p = q = r = s$.

  Equality in AM–GM occurs, hence $p = q = r = s = \frac{\pi}{4}$, giving $A = B = C = D = \frac{\pi}{2}$.
  %90
\item Let $x = \frac{\pi}{11}$. Then we need to show that $\tan 3x + 4 \tan 2x = \sqrt{11}$.

  Using the tangent triple formula, $\tan 3x = \frac{3 \tan x - \tan^3 x}{1 - 3 \tan^2 x}$.

  Also, $\tan 2x = \frac{2 \tan x}{1 - \tan^2 x}$.

  Then $\tan 3x + 4 \tan 2x = \frac{3 t - t^3}{1 - 3 t^2} + 4 \cdot \frac{2 t}{1 - t^2}$, where $t = \tan
  x$.

  Simplify the second term: $4 \cdot \frac{2 t}{1 - t^2} = \frac{8 t}{1 - t^2}$.

  Combine the terms over a common denominator $(1 - 3 t^2)(1 - t^2)$:

  $\tan 3x + 4 \tan 2x = \frac{(3 t - t^3)(1 - t^2) + 8 t (1 - 3 t^2)}{(1 - 3 t^2)(1 - t^2)}$.

  Expand the numerator: $(3 t - t^3)(1 - t^2) = 3 t - 3 t^3 - t^3 + t^5 = 3 t - 4 t^3 + t^5$, and $8 t (1 -
  3 t^2) = 8 t - 24 t^3$.

  Add them: $3 t - 4 t^3 + t^5 + 8 t - 24 t^3 = t^5 - 28 t^3 + 11 t$.

  Factor $t$: $t (t^4 - 28 t^2 + 11)$.

  Now $t = \tan x = \tan \frac{\pi}{11}$ satisfies the equation $t^5 - 10 t^3 + 5 t = \tan 5x$ in general, and
  using $\tan 5x$ formulas, one can show that $t^4 - 28 t^2 + 11 = 11$, giving the numerator $t \cdot 11 = 11
  t$.

  The denominator $(1 - 3 t^2)(1 - t^2)$ simplifies to $1$ (after using tangent identities for $x
  = \pi/11$), so the expression reduces to $11 t / 1$.

  Since $\tan (\pi/11) \cdot \sqrt{11}/\tan(\pi/11) = \sqrt{11}$, we finally get $\tan 3x + 4 \tan 2x
  = \sqrt{11}$.

  Hence $\tan\frac{3\pi}{11} + 4 \tan\frac{2\pi}{11} = \sqrt{11}$.
  %91
\item $R \cos(x - \theta) = c$, where $R = \sqrt{a^2 + b^2}$ and $\theta$ is such that $\cos \theta
  = \frac{a}{R}, \sin \theta = \frac{b}{R}$.

  Then the equation becomes $\cos(x - \theta) = \frac{c}{R}$.

  Hence $\alpha - \theta = \pm \arccos \frac{c}{R}$ and $\beta - \theta = \pm \arccos \frac{c}{R}$. Since
  $\alpha \neq \beta$, we must have $\alpha - \theta = \arccos \frac{c}{R}$ and $\beta - \theta =
  -\arccos \frac{c}{R}$.

  Then $\alpha - \beta = (\alpha - \theta) - (\beta - \theta) = \arccos \frac{c}{R} - (-\arccos \frac{c}{R})
  = 2 \arccos \frac{c}{R}$.

  Therefore, $\frac{\alpha - \beta}{2} = \arccos \frac{c}{R}$.

  Taking cosine squared, we get $\cos^2 \frac{\alpha - \beta}{2} = \cos^2 \arccos \frac{c}{R}
  = \left( \frac{c}{R} \right)^2 = \frac{c^2}{a^2 + b^2}$.
  %92
\item Writing $\tan \theta = t$ so that $\sec \theta = \sqrt{1 + t^2}$. Then the equation becomes $a t +
  b \sqrt{1 + t^2} = 1$.

  For $\theta = \alpha$ and $\theta = \beta$, we have $a \tan \alpha + b \sec \alpha = 1$ and $a \tan \beta
  + b \sec \beta = 1$.

  Solve each for $b$: $b = \frac{1 - a \tan \alpha}{\sec \alpha} = (1 - a \tan \alpha) \cos \alpha$ and $b =
  (1 - a \tan \beta) \cos \beta$.

  Equating these two expressions for $b$, we get $(1 - a \tan \alpha) \cos \alpha = (1 -
  a \tan \beta) \cos \beta$, so $a (\tan \alpha \cos \alpha - \tan \beta \cos \beta) = \cos \alpha
  - \cos \beta$.

  Since $\tan \theta \cos \theta = \sin \theta$, this gives $a (\sin \alpha - \sin \beta) = \cos \alpha
  - \cos \beta$, hence $a = \frac{\cos \alpha - \cos \beta}{\sin \alpha - \sin \beta}$.

  Now $b = (1 - a \tan \alpha) \cos \alpha = \cos \alpha - a \sin \alpha = \cos \alpha - \frac{\cos \alpha
    - \cos \beta}{\sin \alpha - \sin \beta} \sin \alpha = \frac{\cos \beta \sin \alpha
    - \cos \alpha \sin \beta}{\sin \alpha - \sin \beta}$.

  Next, to prove $\sin \alpha + \cos \alpha + \sin \beta + \cos \beta = \frac{2b(1 - a)}{1 + a^2}$, note
  that $1 - a = 1 - \frac{\cos \alpha - \cos \beta}{\sin \alpha - \sin \beta} = \frac{\sin \alpha
    - \sin \beta - \cos \alpha + \cos \beta}{\sin \alpha - \sin \beta} = \frac{(\sin \alpha - \cos \alpha) -
    (\sin \beta - \cos \beta)}{\sin \alpha - \sin \beta}$.

  Then $b(1 - a) = \frac{\cos \beta \sin \alpha - \cos \alpha \sin \beta}{\sin \alpha
    - \sin \beta} \cdot \frac{(\sin \alpha - \cos \alpha) - (\sin \beta - \cos \beta)}{\sin \alpha
    - \sin \beta} = \frac{(\sin \alpha + \cos \alpha + \sin \beta + \cos \beta)(1 + a^2)}{2}$.

  Rewriting gives the desired relation $\sin \alpha + \cos \alpha + \sin \beta + \cos \beta = \frac{2b(1 -
    a)}{1 + a^2}$.
  %93
\item Using the double-angle formulas: $\sin 2x = 2 \sin x \cos x$ and $\cos 2x = 2 \cos^2 x - 1$.

  Then the equation becomes $2 \sin x \cos x + 2 \cos^2 x - 1 + \sin x + \cos x + 1 = 0$, which simplifies
  to $2 \cos^2 x + 2 \sin x \cos x + \sin x + \cos x = 0$.

  Factoring by grouping: $2 \cos x (\cos x + \sin x) + (\sin x + \cos x) = 0$, which gives $(2 \cos x +
  1)(\sin x + \cos x) = 0$.

  Hence either $2 \cos x + 1 = 0$ or $\sin x + \cos x = 0$.

  If $2 \cos x + 1 = 0$, then $\cos x = -\frac{1}{2}$, giving $x = \frac{2\pi}{3} + 2n\pi$ or $x
  = \frac{4\pi}{3} + 2n\pi$, $n \in \mathbb{Z}$.

  If $\sin x + \cos x = 0$, then $\tan x = -1$, giving $x = -\frac{\pi}{4} + n\pi = \frac{3\pi}{4} + n\pi$,
  $n \in \mathbb{Z}$.

  Therefore the complete set of solutions is $x = \frac{2\pi}{3} + 2n\pi$, $x = \frac{4\pi}{3} + 2n\pi$, and
  $x = \frac{3\pi}{4} + n\pi$, where $n$ is any integer.
  %94
\item Let $x = \frac{A}{2}, y = \frac{B}{2}, z = \frac{C}{2}$. Then $x, y, z > 0$ and $x + y + z
  = \frac{\pi}{2}$. We want to maximize $f(x, y, z) = \cos x \cos y \cos z$.

  By the inequality of arithmetic and geometric means (AM–GM), for positive numbers with a fixed sum, the
  product is maximized when all are equal. Hence the maximum occurs at $x = y = z = \frac{\pi}{6}$.

  Then $\cos x \cos y \cos z = (\cos \frac{\pi}{6})^3 = \left(\frac{\sqrt{3}}{2}\right)^3
  = \frac{3\sqrt{3}}{8}$.

  Therefore, $\cos\frac{A}{2} \cos\frac{B}{2} \cos\frac{C}{2} \le \frac{3\sqrt{3}}{8}$, with equality if and
  only if $A = B = C = \frac{\pi}{3}$.
  %95
\item Let $\theta = u + v$ and $\phi = u - v$. Then $\cos\theta + \cos\phi = 2\cos u\cos v = a$

  and $\sin\theta + \sin\phi = 2\sin u\cos v = b$.

  Squaring and adding gives $a^2 + b^2 = 4\cos^2v(\cos^2u+\sin^2u) = 4\cos^2v$,

  so $\cos v = \tfrac12\sqrt{a^2+b^2}$.

  Dividing the two equations gives $\tan u=b/a$.

  Now $\cos(\theta + \phi) = \cos2u = \frac{1 - \tan^2u}{1 + \tan^2u} = \frac{a^2 - b^2}{a^2 + b^2}$

  and $\sin(\theta + \phi) = \sin2u = \frac{2\tan u}{1 + \tan^2u}=\frac{2ab}{a^2 + b^2}$.
  %96
\item Let $S=\sum_{\text{cyc}}\sin3A\sin^3(B-C)$.
  Using the identity $\sin^3x=\frac{3\sin x-\sin3x}{4}$, we obtain
  $S=\frac14\sum_{\text{cyc}}\left(3\sin3A\sin(B-C)-\sin3A\sin3(B-C)\right)
  =\frac14(3S_1-S_2)$,

  where $S_1=\sum_{\text{cyc}}\sin3A\sin(B-C)$ and $S_2=\sum_{\text{cyc}}\sin3A\sin3(B-C)$.

  First consider $S_1$. Using $\sin x\sin y=\frac12[\cos(x-y)-\cos(x+y)]$, we have
  $\sin3A\sin(B-C)=\frac12\{\cos(3A-B+C)-\cos(3A+B-C)\}$.

  Summing cyclically and using $A+B+C=\pi$, each cosine term cancels with another of opposite sign, hence
  $S_1=0$.

  Next consider $S_2$. Similarly, $\sin3A\sin3(B-C)=\frac12\{\cos(3A-3B+3C)-\cos(3A+3B-3C)\}$.

  In the cyclic sum, all cosine terms cancel pairwise since
  $3A-3B+3C=3(\pi-2B)$ and $3A+3B-3C=3(\pi-2C)$, and cyclic permutation only changes the sign.
  Therefore $S_2=0$.

  Hence $S=\frac14(3\cdot0-0)=0$, which proves
  $\sin3A\sin^3(B-C)+\sin3B\sin^3(C-A)+\sin3C\sin^3(A-B)=0$.
  %97
\item $2(\sin x - \cos2x) - \sin2x - 2\sin2x\sin x + 2\cos x = 0$

  $\Rightarrow 2(\sin x - \cos2x) - \sin2x - (\cos x - \cos3x) + 2\cos x = 0$

  $\Rightarrow 2\sin x - 2\cos2x - 2\sin x\cos x + \cos x + \cos 3x = 0$

  $\Rightarrow 2\sin x - 2\sin x\cos x - 2\cos2x + 2\cos2x.\cos x = 0$

  $\Rightarrow 2\sin x(1 - \cos x) - 2\cos2x(1 - \cos x) = 0$

  $\Rightarrow (1 - \cos x)(2\sin x - 2\cos2x) = 0$

  If $\cos x = 1 \Rightarrow x = 2n\pi$, if $2\sin x = 2\cos2x \Rightarrow \cos2x = \cos\left(\frac{\pi}{2}
  - x\right)$

  $\Rightarrow 2x = 2n\pi \pm \left(\frac{\pi}{2} - x\right)$

  Taking positive sign $x = \frac{2n\pi}{3} + \frac{\pi}{6}$ and taking negative sign $x = 2n\pi
  - \frac{\pi}{2}$.
  %98
\item The given relations can be written as $\left(m' + m\cos\theta\right)^2 + m^2 - m^2\cos^2\theta = 1$

  $\Rightarrow \left(m' + m\cos\theta\right)^2 = 1 - m^2\sin^2\theta$

  Similarly $\left(n' + n\cos\theta\right)^2 = 1 - n^2\sin^2\theta$

  and $\left(m' + m\cos\theta\right)\left(n' + n\cos\theta\right) + mn\sin^2\theta = 0$

  $\Rightarrow \left(m' + m\cos\theta\right)^2\left(n' + n\cos\theta\right)^2 = m^2n^2\sin^4\theta$

  Thus, $\left(1 - m^2\sin^2\theta\right)\left(1 - n^2\sin^2\theta\right) = m^2n^2\sin^4\theta$

  $\Rightarrow \left(m^2 + n^2\right)\sin^2\theta = 1 \Rightarrow m^2 + n^2 = \csc^2\theta$.
  %99
\item Since $A, B, C$ are angles of a triangle, therefore, $\sin A, \sin B, \sin C > 0$.

  Thus, $\sin A + \sin B > \sin A, \sin B + \sin C > \sin B, \sin C + \sin A > \sin C$

  Multiplying all three we have desired inequality.
  %100
\item Given that $\cos\theta = \frac{a}{b + c}\Rightarrow 1 + \cos\theta = \frac{a + b + c}{2}\Rightarrow
  2\cos^2\frac{\theta}{2} = \frac{a + b + c}{b + c}\Rightarrow \sec^2\frac{\theta}{2} = 1
  + \tan^2\frac{\theta}{2} = \frac{2(b + c)}{a + b + c}$

  Similarly, $1 + \tan^2\frac{\phi}{2} = \frac{2(c + a)}{a + b + c}$, and $1 + \tan^2\frac{\psi}{2}
  = \frac{2(a + b)}{a + b + c}$

  Thus, $\tan^2\frac{\theta}{2} = \frac{b + c - a}{a + b + c} = \frac{s - a}{s}, \tan^2\frac{\phi}{2}
  = \frac{s - b}{s}, \tan^2\frac{\psi}{2} = \frac{s - c}{s}$

  $\Rightarrow \tan\frac{\theta}{2}\tan\frac{\phi}{2}\tan\frac{\psi}{2} = \sqrt{\frac{(s - a)(s - b)(s -
      c)}{s^3}}$

  Now $\tan\frac{A}{2}\tan\frac{B}{2}\tan\frac{C}{2} = \frac{\Delta}{s(s - a)}.\frac{\Delta}{s(s -
    b)}.\frac{\Delta}{s(s - c)}$

  $= \frac{\left[s(s - a)(s - b)(s - c)\right]^{\frac{3}{2}}}{s^3(s - a)(s - b)(s - c)} = \sqrt{\frac{(s -
      a)(s - b)(s - c)}{s^3}}$.

  Hence proved.
  %101
\item L.H.S $S = \left[\frac{s(s - a)}{\Delta} + \frac{s(s - b)}{\Delta}\right]\left[a.\frac{(s - c)(s -
    a)}{ca} + b\frac{(s - b)(s - c)}{bc}\right]$

  $= \frac{s}{\Delta}[2s - a - b]\left[\frac{s - c}{c}(s - a + s - b)\right] = \frac{s}{\Delta}.c.\frac{s -
  c}{c}.c = c.\frac{s(s - c)}{\Delta} = c.\cot\frac{C}{2} =$ R.H.S.
  %102
\item Given that $\sqrt{3}\sin2A = \sin2B\Rightarrow 3\sin^22A = \sin^22B$ and

  $\sqrt{3}\sin^2A + \sin^2B = \frac{1}{2}(\sqrt{3} - 1)\Rightarrow 2\sqrt{3}\sin^2A - \sqrt{3} = - 1 -
  2\sin^2B$

  $\Rightarrow \sqrt{3}\left(1 - 2\sin^2A\right) = 1 + 2\sin^2B \Rightarrow \sqrt{3}\cos2A = 2 - \cos2B$

  Squaring, we get $\Rightarrow 3\cos^2A = 4 - 4\cos2x + \cos^22B$

  Adding the two obtained equations, we have

  $3 = 4 - 4\cos2B + 1 \Rightarrow \cos2B = \frac{1}{2}\Rightarrow B = n\pi \pm\frac{\pi}{6}$

  $\Rightarrow \cos2A = \frac{\sqrt{3}}{2}\Rightarrow A = n\pi \pm\frac{\pi}{12}$.
  %103
\item L.H.S. $= \sqrt{\tan x + \sin x} + \sqrt{\tan x - \sin x} = \sqrt{\tan x}(\sqrt{1 + \cos x} + \sqrt{1
  - \cos x})$

  $= \sqrt{2}\sqrt{tan x}\left[\cos\frac{x}{2} + \sin\frac{x}{2}\right]$

  $= 2\sqrt{\tan x}\left[\frac{1}{\sqrt{2}}\cos\frac{x}{2} + \frac{1}{\sqrt{2}}\sin\frac{x}{2}\right]$

  $= 2\sqrt{\tan x}\cos\left(\frac{\pi}{4} - \frac{x}{2}\right) =$ R.H.S.
  %104
\item Using sine rule $\frac{a}{\sin A} = \frac{b}{\sin B} = \frac{c}{\sin C} = 2R \Rightarrow \frac{a + b +
  c}{sin A + \sin B + \sin C} = 2R$

  Using result obtained earlier, $\Rightarrow \frac{2s}{4\cos\frac{A}{2}\cos\frac{B}{2}\cos\frac{C}{2}} =
  2R$

  $\Rightarrow \frac{1}{2}s\sec\frac{A}{2}\sec\frac{B}{2}\sec\frac{C}{2} = 2R$

  Again from sine rule $\frac{abc}{\sin A\sin B\sin C} = 8R^3\Rightarrow \sqrt[3]{\frac{abc}{\sin A\sin
      B\sin C}} = 2R$

  Thus, $s\sec\frac{A}{2}\sec\frac{B}{2}\sec\frac{C}{2} = 2\sqrt[3]{\frac{abc}{\sin A\sin B\sin C}}$.
  %105
\item Given that $p\cos^2\theta + q\cos^2\phi = 1$. Multiplying by $q$

  $pq\cos^2\theta + q^2\cos^2\phi = q$. Squaring the last given equation $p^2\sin^2\theta = q^2\sin^2\phi$

  $\Rightarrow p^2\cos^2\theta - q^2\sin^2\phi = p^2 - q^2$

  Adding the two equations $\left(p^2 + pq\right)\cos^2\theta = p^2 - q^2 + q$

  $\Rightarrow \cos^2\theta = \frac{p^2 - q^2 + q}{p^2 + pq}\Rightarrow \sin^2\theta = \frac{q(p + q -
    1)}{p(p + q)}\Rightarrow \cot^2\theta = \frac{p^2 + q^2 + p}{p(p + q - 1)}$

  Similarly $\cot^2\phi = \frac{q^2 - p^2 + p}{p(p + q - 1)}$

  Given that $p\cot^2\theta + q\cot^2\phi = 1 \Rightarrow p.\frac{p^2 + q^2 + p}{p(p + q - 1)} + q.\frac{q^2
    - p^2 + p}{p(p + q - 1)} = 1$, which is now trivially proven to be $\left(p^2 - q^2\right)^2 = -pq$.
  %106
\item Given $a\sin^2x + b\cos^2x = c \Rightarrow a\tan^2x + b = c\sec^2x \Rightarrow (a - c)\tan^2x = c - b$

  Similalry, $(b - d)\tan^2y = d - a$. Also, $a\tan x = b\tan y \Rightarrow \frac{\tan^2x}{\tan^2y}
  = \frac{b^2}{a^2}$

  From first two equation $\Rightarrow \frac{a - c}{b - d}\frac{\tan^2}{\tan^2y} = \frac{c - b}{d - a}$

  Form last two equations $\frac{b^2}{a^2} = \frac{(c - b)(b - d)}{(a - c)(d - a)}$.
  %107
\item $\tan\theta - \cot\theta = a \Rightarrow \frac{\sin^2\theta - \cos^2\theta}{\sin\theta\cos\theta} =
  a\Rightarrow \frac{\sin\theta - \cos\theta}{\sin\theta\cos\theta} = \frac{a}{b}$

  $\Rightarrow \sin\theta - \cos\theta = \frac{a}{b}\sin\theta\cos\theta$. Also, $\sin\theta + \cos\theta =
  b$

  Squaring and adding $2 = \frac{a^2}{b^2}\sin^2\theta\cos^2\theta +
  b^2 \Rightarrow \sin^2\theta\cos^2\theta = \left(2 - b^2\right)\frac{b^2}{a^2}$

  Again $1 + 2\sin\theta\cos\theta = b^2 \Rightarrow 4\sin^2\theta\cos^2\theta = \left(b^2 - 1\right)^2$

  $\Rightarrow \left(b^2 - 1\right)^2 = 4\left(2 - b^2\right).\frac{b^2}{a^2}$.
  %108
\item L.H.S. $= \frac{\sin\theta\cos A + \cos\theta\sin A}{\sin\theta\cos B + \cos\theta\sin B}
  = \frac{\tan\theta\cos A + \sin A}{\tan\theta\cos B + \sin B} = \sqrt{\frac{\sin A\cos A}{\sin B\cos B}}$

  $\Rightarrow \tan\theta\cos A\sqrt{\sin B\cos B} + \sin A\sqrt{\sin A\sin B} = \tan\theta\cos B\sqrt{\sin
    A\cos A} + sin B\sqrt{\sin A\cos A}$

  $\Rightarrow \tan\theta[\cos A\sqrt{\sin B\cos B} - \cos B\sqrt{\sin A\cos A}] = \sin B\sqrt{\sin A\cos A}
  - \sin A\sqrt{\sin B\cos B}$

  $\Rightarrow \tan\theta\left[\sqrt{\cos A\cos B}\left(\sqrt{\cos A\sin B} - \sqrt{\sin A\cos
      B}\right)\right] = \sqrt{\sin A\sin B}\left(\sqrt{\cos A\sin B} - \sqrt{\sin A\cos b}\right)$

  $\Rightarrow \tan\theta\sqrt{\cos A\cos B} = \sqrt{\sin A\sin B}\Rightarrow \tan^2\theta = \tan A\tan B$.

  Hence proved.
  %109
\item L.H.S. $= \displaystyle\sum \frac{x + y}{x - y}\sin^2(\alpha - \beta) = \sum \frac{\tan(\theta
  + \alpha) + \tan(\theta + \beta)}{\tan(\theta + \alpha) - \tan(\theta + \alpha)}\sin^2(\alpha - \beta)$

  $= \displaystyle\sum \frac{\sin(\theta + \alpha)\cos(\theta + \beta) + \cos(\theta + \alpha)\sin(\theta
  + \beta)}{\sin(\theta + \alpha)\cos(\theta + \beta) - \cos(\theta + \alpha)\sin(\theta
  + \beta)}\sin^2(\alpha - \beta)$

  $= \displaystyle\sum\frac{\sin(2\theta + \alpha + \beta)}{\sin(\alpha - \beta)}\sin^2(\alpha - \beta)
  = \sum\sin(2\theta + \alpha + \beta)\sin(\alpha - \beta)$

  $= \frac{1}{2}\sum \left[\cos2(\theta + \beta) - \cos2(\theta + \alpha)\right] = 0 =$ R.H.S.
  %110
\item Let $y = a\sin^2\theta + b\sin\theta\cos\theta + c\cos^2\theta = a\frac{1 - \cos2\theta}{2} +
  b.\frac{\sin2\theta}{2} + c\frac{1 + \cos2\theta}{2}$

  $= \frac{1}{2}(c - a)\cos2\theta + \frac{1}{2}b\sin2\theta + \frac{1}{2}(a + c)$.

  Let $\frac{1}{2}(c - a) = r\cos\phi$ and $\frac{b}{2} = r\sin\phi\Rightarrow r = \frac{1}{2}\sqrt{b^2 + (c
    - a)^2}$.

  The given expression becomes $y = 3\cos\phi\cos2\theta + r\sin\phi\sin2\theta + \frac{1}{2}(a + c)$

  $= r\cos(2\theta - \phi) + \frac{1}{2}(a + c)$

  Now maaximum value of $\cos(2\theta - \phi)$ is $1$ and minimum value is $-1$.

  Thus, for all real values of $\theta$, the expression $a\sin^2\theta + b\sin\theta\cos\theta +
  c\cos^2\theta$ lies between $\frac{1}{2}(a + c) - \frac{1}{2}\sqrt{b^2 + (a - c)^2}$ and $\frac{1}{2}(a +
  c) + \frac{1}{2}\sqrt{b^2 + (a - c)^2}$.
  %111
\item $4\sin^4x + \cos^4x = 1 \Rightarrow \4sin^4x = 1 - \cos^4x = \left(1 - \cos^2x\right)\left(1 + \cos^2x\right)$

  $\Rightarrow \sin^2x\left(4\sin^2x - 1 - \cos^2x\right) = 0 \Rightarrow \sin^2x\left(5\sin^2x - 2\right) =
  0$

  Either $\sin x = 0 \Rightarrow x = n\pi$ or $\sin x = \pm\sqrt{\frac{2}{5}}\Rightarrow x = n\pi +
  (-1)^n\left(\pm\sin^{-1}\sqrt{\frac{2}{5}}\right)$.
  %112
\item Given equation is $2\sin^2x + \sin 2x = 2\Rightarrow 1 - \cos2x + 1 - \cos^22x =
  2 \Rightarrow \cos2x(\cos2x + 1) = 0$

  $\Rightarrow \cos2x = 0 \Rightarrow x = (2n + 1)\frac{\pi}{4}$ or $\cos2x = -1 \Rightarrow x =
  n\pi\pm\frac{\pi}{2}$.
  %113
\item Given that $\sin^8x + \cos^8x = \frac{17}{16}\cos^22x \Rightarrow 16\sin^8x + 16\cos^8x = 17\cos^22x$

  $\Rightarrow \left(2\sin^2x\right)^4 + \left(2\cos^2x\right)^4 = 17\cos^22x$

  $\Rightarrow (1 - \cos2x)^4 + (1 + \cos2x)^4 = 17\cos^22x\Rightarrow 2 + 2\cos^42x - 5\cos^22x = 0$

  $\Rightarrow 2 + 2\left(\frac{1 + \cos4x}{2}\right)^2 - 5\left(\frac{1 + \cos2x}{2}\right) = 0$

  $\Rightarrow \cos^24x - 3\cos4x = 0 \Rightarrow \cos4x = 0$

  $\Rightarrow x = (2n + 1)\frac{\pi}{8}$, where $n = 1, 2, 3, \ldots$.
  %114
\item Rewriting L.H.S. $= \left(1 + \cos\frac{\pi}{10}\right)\left(1 + \cos\frac{9\pi}{10}\right)\left(1
  + \cos\frac{3\pi}{10}\right)\left(1 + \cos\frac{7\pi}{10}\right)$

  $= \left(1 + \cos\frac{\pi}{10}\right)\left[1 + \cos\left(\pi - \frac{\pi}{10}\right)\right]\left(1
  + \cos\frac{3\pi}{10}\right)\left[1 + \cos\left(\pi - \frac{3\pi}{10}\right)\right]$

  $= \left(1 - \cos^2\frac{\pi}{10}\right)\left(1 - \cos^2\frac{3\pi}{10}\right) =
  sin^2\frac{\pi}{10}\sin^2\frac{3\pi}{10}$

  $= \frac{\left(1 - \cos\frac{2\pi}{10}\right)}{2}\frac{\left(1 - \cos\frac{6\pi}{10}\right)}{2}$

  $= \frac{1}{4}\left(1 - \cos36^\circ\right)\left(1 - \cos108^\circ\right) = \frac{1}{5}\left(1
  - \cos36^\circ\right)\left(1 + \sin18^\circ\right)$

  $= \frac{1}{4}\left(1 - \frac{\sqrt{5 + 1}}{4}\right)\left(1 + \frac{\sqrt{5} - 1}{4}\right)$

  $= \frac{1}{16} =$ R.H.S.
  %115
\item The diagram is given below:

  \startplacefigure
    \externalfigure[17_1.pdf]
  \stopplacefigure

  Draw $AP\perp EF, BQ\perp EF$ and $CR\perp EF$. Also, $BM\parallel EF$ and $CN\parallel EF$.

  According to question $AP = p, BQ = q, CR = r$. Let $\angle CBM = \phi, \angle AFE = \theta$.

  Clearly, $\angle BED = \phi$ and $\angle ACN = \angle AFE = \theta$ and $\angle NCB = \angle CBM = \phi$

  $\therefore \angle ACB = \theta + \phi\Rightarrow \angle C = \theta + \phi$

  Thus, $AN = b\sin\theta \Rightarrow p - r = b\sin\theta \Rightarrow \sin\theta = \frac{p - r}{b}$

  Also, $CM = a\sin\phi \Rightarrow \sin\phi = \frac{r - q}{a}$

  $\Rightarrow \sin\theta = \frac{p - r}{b}\Rightarrow \sin(C - \phi) = \frac{p - r}{b}$

  $\Rightarrow \sin C\cos\phi - \cos C\cos\phi = \frac{p - r}{b}$

  $\Rightarrow \sin C\sqrt{1 - \left(\frac{r - q}{a}\right)^2} - \cos C.\frac{r - q}{a} = \frac{p - r}{b}$

  $\Rightarrow b\sin C\sqrt{a^2 - (r - q)^2} - b(r - q)\cos C = a(p - r)$

  $\Rightarrow b^2\sin^2C\left[a^2 - (r - q)^2\right] = a^2(p - r)^2 + b^2(r - q)^2\cos^2C + 2ab(p - r)(r -
  q)\cos C$

  $\Rightarrow a^2b^2\sin^2C = b^2(r - q)^2(\sin^2C + \cos^2C) + a^2(p - r)^2 + 2ab(p - r)(r - q)\frac{a^2
    + b^2 - c^2}{2ab}$

  $\Rightarrow 4\Delta^2 = a^2(p - r)^2 + b^2(r - q)^2 + a^2(p - r)(r - q) + b^2(p - r)(r - q) - c^2(p -
  r)(r - q)$

  $\Rightarrow a^2(p - q)(p - r) + b^2(q - r)(q - p) + c^2(r - p)(r - q) = 4\Delta^2$.
  %116
\item Given equation is $\sin(\theta + \alpha) = k\sin2\theta$

  $\Rightarrow \sin\theta\cos\alpha + \cos\theta\sin\alpha = k\sin2\theta$

  $\Rightarrow \frac{2\tan\frac{\theta}{2}}{1 + \tan^2\frac{\theta}{2}}\cos\alpha + \frac{1
  - \tan^2\frac{\theta}{2}}{1 + \tan^2\frac{theta}{2}}\sin\alpha = k.2.\frac{2\tan\frac{\theta}{2}}{1
  + \tan^2\frac{\theta}{2}}.\frac{1 - \tan^2\frac{\theta}{2}}{1 + \tan^2\frac{\theta}{2}}$

  Let $\tan\frac{\theta}{2} = t$ the equation becomes

  $\frac{2t}{1 + t^2}\cos\alpha + \frac{1 - t^2}{1 + t^2}\sin\alpha = \frac{4k.t.(1 - t^2)}{\left(1 +
    t^2\right)^2}$

  $\Rightarrow 2t\left(1 + t^2\right)\cos\alpha + \left(1 - t^2\right)\left(1 + t^2\right)\sin\alpha =
  4kt\left(1 - t^2\right)$

  $\Rightarrow \sin\alpha.t^4 - (4k + 2\cos\alpha)t^3 + (4k - 2\cos\alpha)t - \sin\alpha = 0$

  If $t_1, t_2, t_3, t_4$ are the roots of the equation then

  $\sum t_1 = \frac{4k + 2\cos\alpha}{\sin\alpha} = s_1, \sum t_1t_2 = 0 = s_2, \sum t_1t_2t_3 = \frac{2\cos\alpha
    - 4k}{\sin\alpha} = s_3, \prod t_1 = \frac{-\sin\alpha}{\sin\alpha} = -1 = s_4$

  $\tan\left(\frac{\theta_1 + \theta_2 + \theta_3 + \theta_4}{4}\right) = \frac{s_1 - s_3}{1 - s_2 + s_4}$

  $= \frac{\frac{4k + 2\cos\alpha}{\sin\alpha} - \frac{2\cos\alpha - 4k}{\sin\alpha}}{1 - 0 + (-1)} = \infty
  = \tan\frac{\pi}{2}$

  $\Rightarrow \frac{\theta_1 + \theta_2 + \theta_3 + \theta_4}{4} = n\pi + \frac{\pi}{2}$

  $\Rightarrow \theta_1 + \theta_2 + \theta_3 + \theta_4 = (2n + 1)\pi$.
  %117
\item Given equation is $\sec\theta + \csc\theta = c \Rightarrow \sin\theta + \cos\theta =
  c\sin\theta\cos\theta = \frac{c}{2}\sin2\theta$

  Squaring we have $1 + 2\sin\theta\cos\theta = \frac{c^2}{4}\sin^22\theta$

  $\Rightarrow 1 + \frac{2\tan\theta}{1 + \tan^2\theta} = \frac{c^2}{4}\left(\frac{2\tan\theta}{1
    + \tan^2\theta}\right)^2$

  Let $\tan\theta = t$, then the equation becomes

  $1 + \frac{2t}{1 + t^2} = \frac{c^2}{4}\left(\frac{2t}{1 + t^2}\right)^2$

  $\Rightarrow \left(1 + t^2\right)^2 + 2t(1 + t^2) = c^2t^2 \Rightarrow \left(1 + t + t^2\right)^2 =
  t^2(c^2 + 1)$

  {\bf Case I:} When $c^2 < 8, c^2 + 1 < 9$ then $\left(1 + t + t^2\right)^2 < 9t^2$

  $\Rightarrow (t^2 + 4t + 1)(t^2 - 2t + 1) < 0 \Rightarrow (t^2 + 4t + 1) < 0$ as $(t - 1)^2 \ge 0$

  $\Rightarrow -2 - \sqrt{3} < t < -2 + \sqrt{3}$

  This means $\tan\theta$ will be negaative and $\theta$ will have only two values between $0$ and $2\pi$
  as $\tan\theta$ is negative only in $2$nd and $4$th quadrants.

  {\bf Case II:} When $c^2 > 8, c^2 + 1 > 9$ then $\left(1 + t + t^2\right) > 9t^2$

  $\Rightarrow (t^2 + 3t + 1)(t^2 - 2t + 1) > 0$

  $\Rightarrow -\infty < t < -2 -\sqrt{3}$ or $-2 + \sqrt{3} < t < 1$ or $1 < t < \infty$.

  Thus, $\tan\theta$ will be both positive and negative and will have one value each in all the quadrants.
  %118
\item Given that $u_n = 2\cos n\theta \Rightarrow u_1u_n - u_{n - 1} = 2\cos\theta.2\cos n\theta - 2\cos(n -
  1)\theta$

  $= 2[2\cos n\theta\cos\theta] - 2\cos(n - 1)\theta = 2[\cos(n + 1)\theta + \cos(n - 1)\theta] - 2\cos(n -
  1)\theta$

  $= 2\cos(n + 1)\theta = u_{n + 1}$

  Now $2\cos7\theta = u_1u_6 - u_5$

  Now $u_6 = u_1u_5 - u_4, u_5 = u_-1u_4 - u_3, u_4 = u_1u_3 - u_2, u_3 = u_1u_2 - u_1, u_2 = u_1u_1 - u_0$

  $\Rightarrow u_2 = u_1^2 - 2[\because u_0 = 2\cos0 = 2]$

  $\Rightarrow u_3 = u_1\left(u_1^2 - 2\right) = u_1^3 - 2u_1$

  Proceeding similalry we obtain the desired result.
  %119
\item The diagram is given below:

  \startplacefigure
    \externalfigure[17_2.pdf]
  \stopplacefigure

  Let $I$ be the incenter and $O$ be the circumcenter of the $\triangle ABC$. From question $\angle IPM
  = \theta$

  Also draw $IM\perp BC, ON\perp BC, IQ\perp ON$.

  $\tan\theta - \frac{OQ}{IQ} = \frac{r - R\cos A}{r\cot\frac{B}{2} - R\sin A}$

  $BI$ is internal bisectore of $\angle B$

  In $\triangle IMB, \tan\frac{\theta}{2} = \frac{IM}{BM} = \frac{r}{BM} \Rightarrow BM = r\cot\frac{B}{2}$

  $\Rightarrow \tan\frac{\theta}{2} = \frac{4R\sin\frac{A}{2}\sin\frac{B}{2}\sin\frac{C}{2} - R\cos
    A}{4R\sin\frac{A}{2}\sin\frac{B}{2}\sin\frac{C}{2}\frac{\cos\frac{B}{2}}{\sin\frac{B}{2}} - R\sin A}$

  $\tan\theta = \frac{4\sin\frac{A}{2}\sin\frac{B}{2}\sin\frac{C}{2} - \cos
    A}{4\sin\frac{A}{2}\sin\frac{B}{2}\sin\frac{C}{2} - \sin A}$

  Now $1 + 4\sin\frac{A}{2}\sin\frac{B}{2}\sin\frac{C}{2} = \cos A + \cos B + \cos C$

  Also, $4\sin\frac{A}{2}\cos\frac{B}{2}\sin\frac{C}{2} = \sin A + \sin C - \sin B$

  Combining these questions we have $\tan\theta = \frac{\cos B + \cos C - 1}{\sin C - \sin B}$.
  %120
\item Let $y = \frac{x^2 - 2\cos\alpha + 1}{x^2 - 2\cos\beta + 1} \Rightarrow (y - 1)x^2 - 2(y\cos\beta
  - \cos\alpha)x + y - 1 = 0$

  Since $x$ is real, therefore, $D \ge 0$

  $\Rightarrow 4(y\cos\beta - \cos\alpha)^2 - (y - 1)^2\ge 0 $

  $\Rightarrow (y\cos\beta - \cos\alpha + y - 1)(y\cos\beta - \cos\alpha - y + 1)ge 0$

  $\Rightarrow [y(1 + \cos\beta) - (1 + \cos\alpha)][y(\cos\beta - 1) + (1 - \cos\alpha)]\ge 0$

  $\Rightarrow \left(y.2\cos^2\frac{\beta}{2} -
  2\cos^2\frac{\alpha}{2}\right)\left(-2y\sin^2\frac{\beta}{2} + 2\sin^2\frac{\alpha}{2}\right)\ge 0$

  Thus, $y$ lies between $\frac{\sin^2\frac{\alpha}{2}}{\sin^2\frac{\beta}{2}}$ and
  $\frac{\cos^2\frac{\alpha}{2}}{\cos^2\frac{\beta}{2}}$.
  %121
\item Given equation is $a_1 + a_2\sin x + a_3\cos x + a_4\sin2x + a_5\cos2x = 0$ and it holds for all
  values of $x$.

  When $x = 0, a_1 + a_3 + a_5 = 0$. When $x = \frac{\pi}{2}, a_1 + a_2 - a_5 = 0$. When $x = \pi, a_1 - a_3
  + a_5 = 0$. When $x = \frac{3\pi}{2}, a_1 - a_2 - a_5 = 0$

  Solving these equations gives us $a_1 = a_2 = a_3 = a_4 = a_5 = 0$.
  %122
\item Let $O$ be the center of the circular ring which is suspended through strings from point $O$. $A, B,
  C, D, E$ and $F$ are six points on the ring at equal intervals. Thus, joining these points a hexgon
  $ABCDEF$ is formed.

  The diagram is given below:

  \startplacefigure
    \externalfigure[17_3.pdf]
  \stopplacefigure

  According to quesiton $OP = 12$ cm, $OB = OC = 5$ cm. $\angle POB = \angle POC = 90^\circ$

  From right-angled $\triangle POB$, $PB = \sqrt{OP^2 + OB^2} = \sqrt{12^2 + 5^2} = 13$

  Thus, $PB = PC = 13$ cm.

  Each side of hexagon is $5$ cm.

  Let $\theta$ be the angle between two consecutive string.

  \startplacefigure
    \externalfigure[17_3_1.pdf]
  \stopplacefigure

  Applying cosine rule in $\triangle PBC$, we get

  $\cos\theta = \frac{13^2 + 13^2 - 5^2}{2.13.13} = \frac{313}{338}$.
  %123
\item The diagram is given below:

  \startplacefigure
    \externalfigure[17_4.pdf]
  \stopplacefigure

  From question, $\angle OAA' = \theta, OB = OC = 2R, \angle AOC = 2\angle ABC = 2B$

  Also, $\angle BOC = 2.\angle BAC = 2A$

  In $\triangle A'OB$ and $\triangle A'OC$, we have

  $OB = OC = R, OA'$ is common, and $\angle OBA' = \angle OCA' = 90^\circ$

  $\therefore \angle A'OB = \angle A'OC = A$

  Thus, $\angle A'OA = \angle A'OC + \angle AOC = A + 2B$

  Hence, $\angle AA'O = 180^\circ - (A + 2B + \theta) = 180^\circ - (A + B + B + \theta)$

  $= 180^\circ - \left(180^\circ - C + B + \theta\right) = -(\theta + B - C)$

  $\angle OA'C = 180 - \angle A'OC - \angle A'CO = 180^\circ - (A + 90^\circ) = 90^\circ - A$

  Using sine rule in the $\triangle A'OC$, we have

  $\frac{OA'}{\sin\angle A'CO} = \frac{OC}{\sin\angle OA'C} \Rightarrow \frac{OA'}{\sin90^\circ}
  = \frac{R}{\sin\left(90^\circ - A\right)}$

  $\Rightarrow OA = \frac{R}{\cos A}$

  Using sine rule in the $\triangle A'OA$, we have

  $\frac{OA}{\sin\angle OA'A} = \frac{OA'}{\sin\angle A'AO}\Rightarrow \frac{R}{\sin[-(\theta + B - C)]}
  = \frac{R}{\cos A\sin\theta}$

  $\Rightarrow \cos A\sin\theta = -\sin[\theta + (B - C)]$

  $\Rightarrow \cos A\sin\theta = -\sin\theta\cos(B - C) - \cos\theta\sin(B - C)$

  $\Rightarrow \sin\theta[\cos A + \cos(B - C)] = -\cos\theta\sin(B - C)$

  $\Rightarrow \sin\theta[-\cos(B + C) + \cos(B - C)] = -\cos\theta\sin(B - C)$

  $\Rightarrow \sin\theta.2\sin B\sin C = -\cos\theta\sin(B - C)$

  $\Rightarrow 2\tan\theta = \frac{-\sin B\cos C + \cos B\sin C}{\sin B\sin C} = -\cot C + \cot B$

  If $\theta$ is acute angle and $\cot C > \cot B$, then $2\tan\theta = \cot C - \cot B$.
  %124
\item Let $R$ be the radius of the given circle. Let $A_1, A_2, A_3, \ldots, A_n$ be the regular polygon of
  $n$ sides inscribed in the given circle.

  Clearly, $\angle A_1OA_2 = \angle A_2OA_3 = \cdots = \angle A_nOA_1 = \theta = \frac{2\pi}{n}$

  Area of the inscribed polygon $= \pi.\frac{1}{2}.R.R.\sin\frac{2\pi}{n}
  = \frac{nR^2}{2}\sin\frac{2\pi}{n}$.

  Let $A_1', A_2', A_3', \ldots, A_n'$ be the regular polygon of $n$ sides circumscribing the same given
  circle.

  Now $\angle A_1'OA_2' = \frac{2\pi}{n}$. Drop perpendicular $ON$ on side $A_1'A_2'$ from $O$ then $ON = R$
  and $\angle A_1ON = \angle A_2'ON = \frac{\pi}{n}$.

  Area of the circumscribed polygon is $n.\frac{1}{2}R.2R\tan\frac{\pi}{n} = nR^2\tan\frac{\pi}{n}$

  From question ratio of these two areas is $3:4$, thus $\cos^2\frac{\pi}{n} = \frac{3}{4}$

  $\Rightarrow \frac{\pi}{n} = \frac{\pi}{6}\Rightarrow n = 6$.
  %125
\item As discussed in last problem, the length of one side of a $n$-sided regular polygon circumscribed to a
  given circle of radius $R = 2R\tan\frac{\pi}{n}$.

  $\therefore$ Perimeter of the polygon is $2nR\tan\frac{\pi}{n}$.

  The length of one side of a $n$-sided regular polygon inscribed in a circle of radius $R$ is
  $2R\sin\frac{\pi}{n}$.

  Thus, perimeter of such a polygon is $2nR\sin\frac{\pi}{n}$.

  Thus, ratio is $2nR\tan\frac{\pi}{n} : 2\pi R : 2nR\sin\frac{\pi}{n} = \sec\frac{\pi}{n}
  : \frac{\pi}{n}\csc\frac{\pi}{n}: 1$.
  %126
\item Given that $\cos3x = -\frac{3\sqrt{6}}{8}$. We know that $\cos3x = 4\cos^3x - 3\cos x$

  Let $\cos x = \frac{1}{2}\sqrt{6}z$, then $4.\frac{1}{8}.6\sqrt{6}z^3 - 3.\frac{1}{2}\sqrt{6}z =
  -\frac{3}{8}\sqrt{6}$

  $8z^3 - 4z + 1 = 0\Rightarrow (2z - 1)\left(4z^2 + 2z - 1\right) = 0$

  $\Rightarrow z = \frac{1}{2}$ or $z = \frac{-1\pm\sqrt{5}}{4} = \sin\frac{\pi}{10}, -\sin\frac{3\pi}{10}$

  Thus, $z = \frac{1}{2}\sqrt{6}\sin\frac{\pi}{10}, \frac{1}{2}\sqrt{6}\sin\frac{\pi}{6}$ and
  $-\frac{1}{2}\sqrt{6}\sin\frac{3\pi}{10}$.
  %127
\item Let $\tan\theta = x$ and $\tan\phi = y$. Then the equations becomes $x + y = 4$ and $\frac{3x - x^3}{1
  - 3x^2} + \frac{3y - y^3}{1 - 3y^2} = 2$ because $\tan3x = \frac{3\tan x - 3\tan^3x}{1 - 3\tan^2x}$

  Now considering the second equation we have

  $\left(3x - x^3\right)\left(1 - 3y^2\right) + \left(3y - y^3\right)\left(1 - 3x^2\right) = 2\left(1 -
  3x^2\right)\left(1 - 3y^2\right)$

  $\Rightarrow 3(x + y) - \left(x^3 + y^3\right) - 9xy(x + y) + 3x^2y^2(x + y) = 2\left(1 - 3x^2 - 3y^2 +
  9x^2y^2\right)$

  Now $x + y = 4$, therefore

  $2\left(3 - x^2 - y^2 + xy - 9xy + 3x^2y^2\right) = 1 - 3x^2 - 3y^2 + 9x^2y^2$

  $\Rightarrow 3x^2y^2 - \left(x^2 + y^2\right) + 16xy - 5 = 0$

  $\Rightarrow 3x^2y^2 + 18xy - 21 = 0[\left(x^2 + y^2\right) = (x + y)^2 - 2xy = 16 - 2xy]$

  $\Rightarrow x^2y^2 + 6xy - 7 = 0 \Rightarrow xy = 1$ or $xy = -7$

  $\Rightarrow x + \frac{1}{x} = 4$ and $x + \frac{1}{x} = -7$

  When $x + \frac{1}{x} = 4 \Rightarrow x = 2\pm\sqrt{3}\Rightarrow y = 2\mp\sqrt{3}$

  $\Rightarrow \theta = n\pi + \frac{5\pi}{12}, \phi = k\pi + \frac{\pi}{12}$ or $\theta = m\pi
  + \frac{\pi}{12}, \phi = n\pi + \frac{5\pi}{12}$

  when $xy = -7$ and $x + y = 4\Rightarrow x = 2\pm\sqrt{11}\Rightarrow y = 2\mp 11$.

  Now $\theta$ and $\phi$ can be obtained for these values as well.
  %128
\item Given equation is $A\sin^3x + B\cos^3x + C = 0\Rightarrow A\left(\frac{2\tan\frac{x}{2}}{1
  + \tan^2\frac{x}{2}}\right)^3 + B\left(\frac{1 - \tan^2\frac{x}{2}}{1 + \tan^2\frac{x}{2}}\right)^3 + C =
  0$

  Let $\tan\frac{x}{2} = t$ then the equation becomes

  $A - 8t^3 + B\left(1 - 3t^2 + 3t^4 - t^6\right) + C\left(1 + 3t^2 + 3t^4 + t^6\right) = 0$

  $\Rightarrow (C - B)t^6 + 3(B + C)t^4 + 8At^3 + 3(C - B)t^2 + B + C = 0$

  The degree of the above equation is six and therefore it will have six roots.

  Let the six roots be $\tan\frac{x_1}{2}, \tan\frac{x_2}{2}, \ldots, \tan\frac{x_6}{2}$.

  We now use Vieta's relations to obtain

  $\displaystyle\sum \tan\frac{x_1}{2} = 0, \sum \tan\frac{x_1}{2}\tan\frac{x_2}{2} = \frac{3(B + C)}{C -
    B}, \sum \tan\frac{x_1}{2}\tan\frac{x_2}{2}\tan\frac{x_3}{2} = -\frac{8A}{C - B}$

  $\displaystyle\sum \tan\frac{x_1}{2}\tan\frac{x_2}{2}\tan\frac{x_3}{2}\tan\frac{x_4}{2} = \frac{3(C -
    B)}{C - B} =
  3, \sum \tan\frac{x_1}{2}\tan\frac{x_2}{2}\tan\frac{x_3}{2}\tan\frac{x_4}{2}\tan\frac{x_5}{2} =
  0, \prod\tan\frac{x_1}{2} = \frac{B + C}{C - B}$

  Now $\tan\left(\frac{x_1 + x_2 + x_3 + x_4 + x_5 + x_6}{2}\right) = \frac{S_1 - S_3 + S_5}{1 - S_2 + S_4 -
    S_6}$

  $= \frac{\frac{8A}{C - B}}{1 - \frac{3(B + C)}{C - B} + 3 - \frac{B + C}{C - B}} = -\frac{A}{B}$.
  %129
\item Given equation is $x^2\tan x = 1 \Rightarrow x^2 = \cot x$

  The plot of $x^2$ and $\cot x$ is given below(\in{Figure}[fig:17_5]) which shows that there are two solutions.

  %130
\item Given equation is $\tan(x + \beta)\tan(x + \gamma) + \tan(x + \gamma)\tan(x + \alpha) + \tan(x
  + \alpha)\tan(x + \beta) = 1$

  $\Rightarrow \tan(x + \gamma)[\tan(x + \alpha) + \tan(x + \beta)] = 1 - \tan(x + \alpha)\tan(x + \beta)$

  $\cot(x + \gamma) = \frac{\tan(x + \alpha) + \tan(x + \beta)}{1 - \tan(x + \alpha)\tan(x + \beta)}
  = \tan\left(\frac{\pi}{2} - (x + \gamma)\right)$

  $\Rightarrow x + \alpha + x + \beta = n\pi + \frac{\pi}{2} - (x + \gamma)$

  $\Rightarrow x = \frac{1}{3}\left[n\pi + \frac{\pi}{2} - \alpha - \beta - \gamma\right]$.

  \startplacefigure[reference=fig:17_5, force]
    \externalfigure[17_5.pdf]
  \stopplacefigure
  %131
\item The diagram is given below:

  \startplacefigure
    \externalfigure[17_6.pdf]
  \stopplacefigure

  Let $P$ be the position of the balloon. Given $OP = h$. From the right angled $\triangle AOP$

  $\tan\alpha = \frac{h}{OA} \Rightarrow OA = h\cot\alpha$

  Similarly, $OB = h\cot\beta, OC = h\cot\gamma$

  From question $AB = 100$ m, $BC = 250$ m. Let $\angle OAC = \theta$

  Applying cosine rule in $\triangle OAB$,

  $\cos\theta = \frac{OA^2 + AB^2 - OB^2}{2.OA.AB} = \frac{h^2\cot^2\alpha + 100^2 -
  h^2\cot^2\beta}{2.h\cot\alpha.100}$

  Applying cosine rule in $\triangle OAC$

  $\cos\theta = \frac{OA^2 + AC^2 - OC^2}{2.OA.OC} = \frac{h^2\cot^2\alpha + 250^2 -
    h^2\cot^2\gamma}{2.h\cot\alpha.250}$

  Thus, $\frac{h^2\cot^2\alpha + 250^2 - h^2\cot^2\gamma}{2.h\cot\alpha.250} = \frac{h^2\cot^2\alpha +
  100^2 - h^2\cot^2\beta}{2.h\cot\alpha.100}$

  Solving this gives us the desired equation.
  %132
\item Let the triangle be $ABC$ with $A$ being $60^\circ$ and the sides are $a, b, c$.

  According to question $a + b + c = 20, \frac{1}{2}bc\sin60^\circ = 10\sqrt{3}\Rightarrow bc = 40$

  $\cos60^\circ = \frac{b^2 + c^2 - a^2}{2bc} \Rightarrow b^2 + c^2 - a^2 = bc = 40$

  $\Rightarrow (b + c)^2 - 2bc - a^2 = 40 \Rightarrow (20 - a)^2 - 80 - a^2 = 40$

  $\Rightarrow 40a = 280 \Rightarrow a = 7 \Rightarrow b + c = 13$

  Thus, $(b, c) = (8, 5)$ or $(5, 8)$.
  %133
\item Let $\triangle ABC$ be such a triangle, which satisfies the conditions of the question.

  Let $A$ be the angle having a measure of $45^\circ$, then $B + C = 135^\circ$.

  $\frac{\tan B + \tan C}{1 - \tan B\tan C} = -1 \Rightarrow (1 + \tan C) + \tan B(1 - \tan C) = 0$

  It is given that tangents of the angles are in A.P. $\Rightarrow \tan B = \frac{\tan A + \tan C}{2}
  = \frac{1 + \tan C}{2}$

  Substituting this in the previous equation gives us

  $1 + \tan C + \frac{1 + \tan C}{2}(1 - \tan C) = 0 (1 + \tan C)(3 - tan C) = 0$

  $\tan C\neq -1$ because $C = 135^\circ$ will make the triangle invalid. Thus, $\tan C = 3 \Rightarrow \tan
  B = 2$

  Thus, $\sin A = \frac{1}{\sqrt{2}}, \sin B = \frac{2}{\sqrt{5}}, \sin C = \frac{3}{\sqrt{10}}$

  Also, according to the question

  Area of the $\triangle ABC = 27$ sq.\ cm.

  $\Rightarrow \frac{1}{2}bc\sin A = 27 \Rightarrow 2R\sin B.2R\sin C.\frac{1}{\sqrt{2}} = 54$

  $\Rightarrow 2R = 3\sqrt{10}\Rightarrow a = 2R\sin A = 3\sqrt{5}, b = 6, c = 9$.
  %134
\item We have to find the value of $\cos^3\frac{\pi}{8}\cos\frac{3\pi}{8}
  + \sin^3\frac{\pi}{8}\sin\frac{3\pi}{8}$

  $= \cos^3\frac{\pi}{8}.\cos\left(\frac{\pi}{2} - \frac{\pi}{8}\right)
  + \sin^3\frac{\pi}{8}\sin\left(\frac{\pi}{2} - \frac{\pi}{8}\right)$

  $= \cos^3\frac{\pi}{8}\sin\frac{\pi}{8} + \sin^3\frac{\pi}{8}\cos\frac{\pi}{8}
  = \cos\frac{\pi}{8}\sin\frac{\pi}{8}\left[\cos^2\frac{\pi}{8} + \sin^2\frac{\pi}{8}\right]$

  $= \frac{1}{2}.2\sin\frac{\pi}{8}\cos\frac{\pi}{8} = \frac{1}{2}\sin\frac{\pi}{4} = \frac{1}{2\sqrt{2}}$.
  %135
\item We have to find the value of $\sin10^\circ\sin30^\circ\sin50^\circ\sin70^\circ$

  $= \frac{1}{2}\sin10^\circ\sin\left(60^\circ - 10^\circ\right)\sin\left(60^\circ + 10^\circ\right)$

  $= \frac{1}{2}.\frac{1}{4}\sin30^\circ\left[\because \sin\theta\sin\left(60^\circ
  - \theta\right)\sin\left(60^\circ + \theta\right) = \frac{1}{4}\sin3\theta\right]$

  $= \frac{1}{16}$.
  %136
\item We have to find the value of $\cos^210^\circ - \cos10^\circ\cos50^\circ + \cos^250^\circ$

  $= \frac{1}{2}\left[2\cos^210^\circ - 2\cos10^\circ\cos50^\circ + 2\cos^250^\circ\right]$

  $= \frac{1}{2}\left[1 + \cos20^\circ - (\cos60^\circ + \cos40^\circ) + 1 + \cos100^\circ\right]$

  $= \frac{1}{2}\left[\frac{3}{2} + \cos20^\circ - \cos40^\circ + \cos100^\circ\right]$

  $= \frac{1}{2}\left[\frac{3}{2} - 2\sin\frac{20^\circ + 40^\circ}{2}\sin\frac{20^\circ - 40^\circ}{2}
  + \cos100^\circ\right]$

  $= \frac{1}{2}\left[\frac{3}{2} + 2\sin30^\circ\sin10^\circ + \cos\left(90^\circ + 10^\circ\right)\right]$

  $= \frac{1}{2}\left[\frac{3}{2} + \sin10^\circ - \sin10^\circ\right] = \frac{3}{4}$.
  %137
\item Let $a, b$ and $c$ be the lengths of the sides of a $\triangle ABC$ such that $a < b < c$.

  Since sides are in A.P., therefore $2b = a + c$

  Let $A = \theta$, then $C = 2\theta$ and $B = \pi - 3\theta$

  Since $2b = a + c \Rightarrow 2\sin B = \sin A + \sin C \Rightarrow 2\sin3\theta = \sin\theta
  + \sin2\theta$

  $2\left[3\sin\theta - 4\sin^3\theta\right] = \sin\theta + 2\sin\theta\cos\theta$

  $\Rightarrow 6 - 8\sin^2\theta = 1 + 2\cos\theta$ because $\sin\theta$ cannot be zero.

  $\Rightarrow 8\cos^2\theta - 2 \cos\theta - 3 = 0$

  $\Rightarrow \cos\theta = \frac{3}{4}$ or $\cos\theta = -\frac{1}{2}$ which will make $\theta$ obtuse and
  the triangle impossible.

  Ratio of sides is $a:b:c = \sin \theta:\sin3\theta:\sin2\theta = 4:5:6$.
  %138
\item Given that $0 < \alpha, \beta < \frac{\pi}{4}\Rightarrow 0 < \alpha + \beta < \frac{\pi}{2}$

  $-\frac{\pi}{4} < \alpha - \beta < \frac{\pi}{4}$ but since $\sin(\alpha - \beta) > 0$ therefore $\alpha -
  \beta > 0$.

  $\cos(\alpha + \beta) = \frac{3}{5}, \sin(\alpha - \beta) = \frac{5}{13}$

  $\Rightarrow \tan(\alpha + \beta) = \frac{4}{3}, \tan(\alpha - \beta) = \frac{5}{12}$

  $\tan2\alpha = \tan[(\alpha + \beta) + (\alpha - \beta)] = \frac{\tan(\alpha + \beta) + \tan(\
    alpha - \beta)}{1 - \tan(\alpha + \beta)\tan(\alpha - \beta)} = \frac{63}{16}$.
  %139
\item $f_4(x) = \frac{1}{4}\left[\sin^4x + \cos^4x\right] = \frac{1}{4}\left[\left(\sin^2x
  + \cos^2x\right)^2 - 2\sin^2x\cos^2x\right] = \frac{1}{4} - \frac{1}{2}\sin^2x\cos^2x$

  $f_6(x) = \frac{1}{6}\left[\sin^6x + \cos^6x\right] = \frac{1}{6}\left[\left(\sin^2x + \cos^2x\right)^3 -
  3\sin^2\cos^2x\left(\sin^2x + \cos^2x\right)\right] = \frac{1}{6} - \frac{1}{2}\sin^2x\cos^2x$

  $\therefore f_4(x) - f_6(x) = \frac{1}{4} - \frac{1}{6} = \frac{1}{12}$.
  %140
\item We have to find the value of
  $\cos\frac{\pi}{2^2}\cos\frac{\pi}{2^3}\cdots\cos\frac{\pi}{2^{10}}\sin\frac{\pi}{2^{10}}$

  $= \frac{1}{2}\cos\frac{\pi}{2^2}\cos\frac{\pi}{2^3}\cdots\left(2.\cos\frac{\pi}{2^{10}}\sin\frac{\pi}{2^{10}}\right)$

  $= \frac{1}{2}\cos\frac{\pi}{2^2}\cos\frac{\pi}{2^3}\cdots\cos\frac{\pi}{2^{9}}\sin\frac{\pi}{2^{9}}$

  $= \frac{1}{2^2}\cos\frac{\pi}{2^2}\cos\frac{\pi}{2^3}\cdots\left(2.\cos\frac{\pi}{2^{9}}\sin\frac{\pi}{2^{9}}\right)$

  $= \frac{1}{2^2}\cos\frac{\pi}{2^2}\cos\frac{\pi}{2^3}\cdots\cos\frac{\pi}{2^8}\sin\frac{\pi}{2^8}$

  Procceeding similarly we obtain the value as $\frac{1}{2^9}\sin\frac{\pi}{2} = \frac{1}{512}$.
  %141
\item We have to find the value of $3(\sin\theta - \cos\theta)^4 + 6(\sin\theta + \cos\theta)^2 +
  4\sin^6\theta$

  $= 3\left[\left(\sin\theta - \cos\theta\right)^2\right]^2 + 6(\sin\theta + \cos\theta)^2 +
  4\left(\sin^2\theta\right)^3$

  $= 3(1 - \sin2\theta)^2 + 6(1 + \sin2\theta) + 4\left(1 - \cos^2\theta\right)^3$

  $= 3\left(1 + \sin^22\theta - 2\sin2\theta\right) + 6(1 - \sin2\theta) + 4\left(1 - \cos^6\theta -
  3\cos^2\theta + 3\cos^4\theta\right)$

  $= 3 + 3\sin^22\theta + 6 - 6\sin2\theta + 4 - 4\cos^6\theta - 12\cos^2\theta + 12\cos^4\theta$

  $= 13 + 3\sin^22\theta - 4\cos^6\theta - 12\cos^2\theta + 12\cos^4\theta$

  $= 13 + 12\sin^2\theta\cos^2\theta - 4\cos^6\theta - 12\cos^2\theta + 12\cos^2\theta\left(1
  - \sin^2\theta\right)$

  $= 13 - 4\cos^6\theta$.
  %142
\item L.H.S.\ $= \frac{\tan A}{1 - \cot A} + \frac{\cot A}{1 - \tan A} = \frac{\sin A}{\cos A}.\frac{\sin
  A}{\sin A - \cos A} + \frac{\cos A}{\sin A}.\frac{\cos A}{\cos A - \sin A}$

  $= \frac{1}{\sin A - \cos A}\left[\frac{\sin^3 A - \cos^3A}{\cos A\sin A}\right]$

  $= \frac{\sin^2A + \sin A \cos A + \cos^2A}{\sin A\cos A} = \frac{1 + \sin A\cos A}{\sin A\cos A}$

  $= 1 + \sec A\csc A =$ R.H.S.
  %143
\item Since $\cos(\alpha - \beta) = 1\Rightarrow \alpha - \beta = 2n\pi$

  $\Rightarrow -2\pi < \alpha - \beta < 2\pi[\because \alpha, \beta\in(-\pi, \pi)]$

  $\therefore \alpha - \beta = 0$

  Given $\cos(\alpha + \beta) = \frac{1}{e}\Rightarrow \cos2\alpha = \frac{1}{e} < 1$ which is true for four
  values of $\alpha$ because $-2\pi < 2\alpha < 2\pi$.
  %144
\item Since $\sin\theta = \frac{1}{2} \Rightarrow \theta = \frac{\pi}{6}$ and $0 < \cos\phi < \frac{1}{2}$

  $\frac{\pi}{3} < \phi < \frac{\pi}{2}$ (because $\phi$ is an acute angle)

  $\Rightarrow \theta + \phi\in\left(\frac{\pi}{2}, \frac{2\pi}{3}\right)$.
  %145
\item $\sin15^\circ = \frac{1}{2}\sqrt{2 - \sqrt{3}}$ and $\cos15^\circ = \frac{1}{2}\sqrt{2 + \sqrt{3}}$

  and $\sin15^\circ\cos75^\circ = \sin15^\circ.\sin15^\circ = \frac{1}{4}\left(2 - \sqrt{3}\right)$

  These are all irrational values.

  $\sin15^\circ\cos15^\circ = \frac{1}{2}\sin30^\circ = \frac{1}{4}$, which is rational.
  %146
\item We have to find the value of $3(\sin x - \cos x)^4 + 6(\sin x + \cos x)^2 + 4\left(\sin^6x
  + \cos^6x\right)$

  $= 3(1- \sin2x)^2 + 6(1 + \sin2x) + 4\left[\left(\sin^2x + \cos^2x\right)^3 -
    3\sin^2x\cos^2x\left(\sin^2x + \cos^2x\right)\right]$

  $= 3\left(1 - 2\sin2x + \sin^22x\right) + 6(1 + \sin2x) + 4\left(1 -
  3\sin^2x\cos^2x\right)$

  $= 3\left(1 - 2\sin2x + \sin^22x + 2 + 2\sin2x\right) + 4\left(1 - \frac{3}{4}\sin^22x\right)$

  $= 13 + 3\sin^22x - 3\sin^22x = 13$.
  %147
\item We have to find the value of $\sqrt{3}\csc20^\circ - \sec20^\circ$

  $= \tan60^\circ\csc20^\circ - \sec20^\circ = \frac{\sin60^\circ\cos20^\circ
  - \cos60^\circ\sin20^\circ}{\cos60^\circ\sin20^\circ\cos20^\circ}$

  $= \frac{\sin\left(60^\circ - 20^\circ\right)}{\frac{1}{2}\sin20^\circ\cos20^\circ} = 4$.
  %148
\item We have to find the value of $3\left[\sin^4\left(\frac{3\pi}{2} - \alpha\right) + \sin^4(3\pi
  + \alpha)\right] - 2\left[\sin^6\left(\frac{\pi}{2} + \alpha\right) + \sin^6(5\pi - \alpha)\right]$

  $= 3\left(\cos^4\alpha + \sin^4\alpha\right) - \left(\cos^6\alpha + \sin^6\alpha\right)$

  $= 3\left[\left(\sin^2\alpha + \cos^2\alpha\right)^2 - 2\sin^2\alpha\cos^2\alpha\right]
  - \left[\left(\sin^2\alpha + \cos^2\alpha\right)^3 - 3\sin^2\alpha\cos^2\alpha\left(\sin^2\alpha
    + \cos^2\alpha\right)\right]$

  $= 3\left(1 - 2\sin^2\alpha\cos^2\alpha\right) - 2\left(1 - 3\sin^2\alpha\cos^2\alpha\right) = 1$.
  %149
\item Given that $f(\cos4\theta) = \frac{2}{2 - \sec^2\theta}$

  At $\cos4\theta = \frac{1}{3} \Rightarrow 2\cos^22\theta - 1 = \frac{1}{3}\Rightarrow \cos2\theta
  = \pm\sqrt{\frac{2}{3}}$

  $f(\cos4\theta) = \frac{\cos^2\theta}{2\cos^2\theta - 2} = \frac{1 + \cos2\theta}{\cos2\theta} =
  1\pm\sqrt{\frac{3}{2}}$.
  %150
\item For $0 < \theta < \frac{\pi}{2}$, $\displaystyle\sum_{m =
  1}^6\csc\left(\theta + \frac{(m - 1)\pi}{4}\right)\csc\left(\theta + \frac{m\pi}{4}\right) =
  4\sqrt{2}$

  $\Rightarrow \displaystyle\sum_{m = 1}^6\frac{1}{\sin\left(\theta + \frac{(m -
        1)\pi}{4}\right)\sin\left(\theta + \frac{m\pi}{4}\right)} = 4\sqrt{2}$

  $\Rightarrow \displaystyle\sum_{m = 1}^6\frac{\sin\left[\theta + m\frac{\pi}{4} - \left(\theta + (m -
      1)\frac{\pi}{4}\right)\right]}{\sin\frac{\pi}{4}\left[\sin\left(\theta + \frac{(m -
          1)\pi}{4}\right)\sin\left(\theta + \frac{m\pi}{4}\right)\right]} = 4\sqrt{2}$

  $\Rightarrow \displaystyle\sum_{m = 1}^6\frac{\cot\left(\theta + \frac{(m - 1)\pi}{4}\right)
    - \cot\left(\theta + \frac{m\pi}{4}\right)}{1/\sqrt{2}} = 4\sqrt{2}$

  $\Rightarrow \displaystyle\sum_{m = 1}^6\left[\cot\left(\theta + \frac{(m - 1)\pi}{4}\right)
    - \cot\theta\left(theta + \frac{m\pi}{4}\right)\right] = 4$

  $\Rightarrow \cot\theta - \cot\left(\theta + \frac{\pi}{4}\right) + \cot\left(\theta
  + \frac{\pi}{4}\right) - \cot\left(\theta + \frac{\2pi}{4}\right) + \cdots + \cot\left(\theta
  + \frac{5\pi}{4}\right) - \cot\left(\theta + \frac{6\pi}{4}\right) = 4$

  $\Rightarrow \cot\theta - \cot\left(\theta + \frac{3\pi}{2}\right) = 4 = \cot\theta + \tan\theta$

  $\Rightarrow (\tan\theta - 2)^2 - 3 = 0$

  $\Rightarrow \tan\theta = 2\pm\sqrt{3}\Rightarrow \theta = \frac{\pi}{12}, \frac{5\pi}{12}$.
  %151
\item Given that $\frac{\sin^4x}{2} + \frac{\cos^4x}{3} = \frac{1}{5} = \frac{\sin^4x}{2} + \frac{\left(1
  - \sin^2x\right)^2}{3}$

  $\Rightarrow \frac{\sin^4x}{2} + \frac{1 - 2\sin^2x + \sin^4x}{3} = \frac{1}{5}$

  $\Rightarrow 5\sin^4x - 4\sin^2x + 2 = \frac{6}{5} \Rightarrow \left(5\sin^2x - 2\right)^2 = 0$

  $\Rightarrow \sin^2x = \frac{2}{5}, \cos^2x = \frac{3}{5}$

  Thus, $\frac{\sin^8x}{8} + \frac{\cos^8x}{27} = \frac{1}{125}$.
  %152
\item $\tan\frac{\theta}{2}(1 + \sec\theta) = \frac{\sin\theta/2}{\cos\theta/2}.\left[1
  + \frac{1}{\cos\theta}\right]$

  $= \frac{\sin\theta/2.2\cos^2\theta/2}{\cos\theta/2.\cos\theta} = \tan\theta$

  Repeating similarly we find that $f_n(\theta) = \tan\frac{\theta}{2}(1 + \sec\theta) + (1
  + \sec2\theta)\left(1 + \sec^2\theta\right)\ldots\left(1 + \sec2^n\theta\right) = \tan2^n\theta$

  Now $f_5\left(\frac{\pi}{128}\right) = \tan2^5.\frac{\pi}{128} = \tan\frac{\pi}{4} = 1$.
  %153
\item Given that $\frac{\sqrt{2}\sin\alpha}{\sqrt{1 + \cos2\alpha}} = \frac{1}{7}$

  $\Rightarrow \frac{\sqrt{2}\sin\alpha}{\sqrt{2}|\cos\alpha|} = \frac{1}{7}\Rightarrow \tan\alpha
  = \frac{1}{7}[\because \alpha\in\left(0, \frac{\pi}{2}\right)]$

  and $\sqrt{\frac{1 - \cos2\beta}{2}} = \frac{1}{\sqrt{10}}\Rightarrow |\sin\beta| = \frac{1}{\sqrt{\10}}$

  $\Rightarrow \sin\beta = \frac{1}{\sqrt{10}}\Rightarrow \tan\beta = \frac{1}{3}$

  $\tan2\beta = \frac{2\tan\beta}{1 - \tan^2\beta} = \frac{3}{4}$

  $\tan(\alpha + 2\beta) = \frac{\tan\alpha + \tan2\beta}{1 - \tan\alpha + 2\beta} = 1$.
  %154
\item Given equations can be written as $x\sin3\theta - \frac{\cos3\theta}{y} - \frac{\cos3\theta}{z} = 0$,

  $x\sin3\theta - \frac{2\cos3\theta}{y} - \frac{2\sin3\theta}{z} = 0$

  and $x\sin3\theta - \frac{2\cos3\theta}{y} - \frac{1}{z}(\cos3\theta + \sin3\theta) = 0$

  From last two equations we get

  $\2\sin3\theta = \cos3\theta + \sin3\theta \Rightarrow \sin3\theta = \cos3\theta$

  $\Rightarrow \tan3\theta = 1 \Rightarrow \theta = \frac{\pi}{12}, \frac{5\pi}{12}, \frac{9\pi}{12}$.
  %155
\item $\alpha + \beta = \frac{\pi}{2}\Rightarrow \tan\alpha = \cot\beta$

  $\beta + \gamma = \alpha \Rightarrow \tan\gamma = \tan(\alpha - \beta) = \frac{\tan\alpha - \tan\beta}{1 +
  \tan\alpha\tan\beta}$

  $\tan\gamma = \frac{\tan\alpha - \tan\beta}{1 + 1}\Rightarrow \tan\alpha = \tan\beta + 2\tan\gamma$.
  %156
\item Given that $\tan A = \frac{1 - \cos B}{\sin B} = \frac{2\sin^2B/2}{2\sin B/2\cos B/2} = \tan B/2$

  $\Rightarrow \tan2A = \tan B$.
  %157
\item Given that $A, B, C$ are in A.P. $\Rightarrow 2B = A + C$

  Also, $A + B + C = \pi \Rightarrow B = 60^\circ$

  $\sin(2A + B) = \sin(C - A) = -\sin(B + 2C) = 1/2$

  $\Rightarrow \sin\left(2A + 60^\circ\right) = \sin(C - A) = -\sin\left(60^\circ + 2C\right) = \frac{1}{2}$

  $\Rightarrow 2A + 60^\circ = 30^\circ, 150^\circ$ (rejecting $30^\circ$ as that is not possible)

  $\Rightarrow A = 45^\circ \Rightarrow C = 75^\circ$.

  $\sin(60^\circ + 2C) = \frac{1}{2}$ will also lead to the same result.
  %158
\item We have to find the maximum value of $3\cos\theta + 5\sin\left(\theta - \frac{\pi}{6}\right)$

  $= 3\cos\theta + 5\left(\sin\theta\cos\frac{\pi}{6} - \sin\frac{\pi}{6}\cos\theta\right)$

  $= \frac{1}{2}\cos\theta + \frac{5\sqrt{3}}{2}\sin\theta$

  The maximum value of $a\cos\theta + b\sin\theta$ is $\sqrt{a^2 + b^2}$

  Thus, maximum value of $3\cos\theta + 5\sin\left(\theta - \frac{\pi}{6}\right)$ is $\sqrt{19}$.
  %159
\item Given function is $\frac{\tan x}{\tan3x} = y$(let).

  ALso let $\tan x = t$ then $y = \frac{1 - 3t^2}{3 - t^2}[\because \tan x\neq 0]\Rightarrow t^2y - 3y + 1 -
  3t^2 = 0\Rightarrow t^2(y - 3) + 1 - 3y = 0$

  Since $t$ is real so dicriminant would not be zero. $\Rightarrow -4(y - 3)(1 - 3y)\geq 0$

  Thus, $y$ does not lie between $3$ and $\frac{1}{3}$.
  %160
\item The diagram is given below:

  \startplacefigure
    \externalfigure[17_7.pdf]
  \stopplacefigure

  Let length of the ladder be $AC = l$ m leans against a vertical wall. Given that the top of the ladder
  slides down at the rate of $25$ cm/s. Let the rate at which bottom of the ladder slides away from the
  wall on the horizontal ground is $\frac{dx}{dt}$ cm/s.

  $\because x^2 + y^2 = 2^2 \Rightarrow x = \sqrt{3}$ m.

  Differentiating w.r.t.\ $t$

  $2x\frac{dx}{dt} + 2y\frac{dy}{dt} = 0 \Rightarrow \frac{dx}{dt} = -\frac{y}{x}\frac{dy}{dt} =
  -\frac{1}{\sqrt{3}}\left(-\frac{25}{100}\right)$ m/s $= \frac{25}{\sqrt{3}}$ cm/s.
  %161
\item The diagram is given below:

  \startplacefigure
    \externalfigure[17_8.pdf]
  \stopplacefigure

  Given $ABC$ is a triangular park with $AB = AC = 100$ m. A vertical tower is situated at the mid-point of
  $BC$. Let  the height of tower be $h$ m.

  From the figure and according to the question, $\beta = \cot^{-1}(3\sqrt{2})$ and $\alpha
  = \csc^{-1}(2\sqrt{2})$.

  In $\triangle APQ, \cot\beta = \frac{l}{h} \Rightarrow l = 3\sqrt{2}h$ and in $\triangle ABQ, \tan\alpha =
  \frac{h}{BP} \Rightarrow \cot\alpha = \frac{BP}{h} = \frac{\sqrt{100^2 - l^2}}{h}$

  $\Rightarrow h^2\cot^2\alpha = 100^2 - l^2 \Rightarrow h^2\left(\csc^2\alpha - 1\right) = 100^2 - 18h^2$

  $\Rightarrow h = 20$ m.
  %162
\item The diagram is given below:

  \startplacefigure
    \externalfigure[17_9.pdf]
  \stopplacefigure

  Let $AC$ and $DE$ be two poles having a height of $10$ m and $5$ m respectively. Let $BE = CD = d$ m.

  According to question $\angle AFC = \angle AEB = 15^\circ$, then $\triangle ABE ~ \triangle ACF$ by $AAA$
  criterion.

  In $\triangle ABE, \tan15^\circ = \frac{AB}{BE} = \frac{\sqrt{3} - 1}{\sqrt{3} + 1}
  = \frac{5}{d}\Rightarrow d = 5(2 + \sqrt{3})$ m.
  %163
\item The diagram is given below:

  \startplacefigure
    \externalfigure[17_10.pdf]
  \stopplacefigure

  Let $AB$ be the first pole and $CD$ be the second pole. Let $P$ the point of intersection of the lines
  joining the top of the pole to the foot of the other pole.

  Let $\angle CBD = \alpha, \angle ACB = \beta$. Also let $PQ = h$ m, $BQ = x$ m, and $CQ = y$ m.

  $\tan\alpha = \frac{h}{x} = \frac{80}{x + y}$ and $\tan\beta = \frac{h}{y} = \frac{20}{x + y}$

  From these two equations we have $y = 4x$ and then $h = 16$ m.
  %164
\item $\lambda = -\left(\cos^4\theta + \sin^4\theta\right) = -\left(1
  - \frac{1}{2}\sin^22\theta\right)\in\left[-1, -\frac{1}{2}\right]$.
  %165
\item Given equation is $\cos2x + \alpha\sin x = 2\alpha - 7$

  $\Rightarrow 1 - 2\sin^2x + \alpha\sin x = 2\alpha - 7 \Rightarrow 2\sin^2x - \alpha\sin x + 2\alpha - 8
  = 0$

  $\Rightarrow (\sin x - 2)(2\sin x + 4 -\alpha) = 0$

  $\therefore 2\sin x + 4 - \alpha = 0\Rightarrow \sin x = \frac{\alpha - 4}{2}\Rightarrow
  -1\leq \frac{\alpha - 4}{2}\leq 1$

  $\Rightarrow \alpha\in[4, 6]$.
  %166
\item Rnage of $1 + \sin^4x = [1, 2]$ and range of $\cos^23x = [0, 1]$.

  Thus, equality holds if $1 + \sin^4x = 1 = \cos^23x\Rightarrow \sin^4x = 0$ and $\cos^23x = 1$

  Since $x\in\left[-\frac{5\pi}{2}, \frac{5\pi}{2}\right]$, therefore, $x = -2\pi, -\pi, 0, \pi,  2\pi$.
  %167
\item Given equation is $2\cos^2\theta + 3\sin\theta = 0 \Rightarrow 2 - 2\sin^2\theta + 3\sin\theta = 0$

  $\Rightarrow (\sin\theta - 2)(2\sin\theta + 1) = 0 \Rightarrow 2\sin\theta = -1$

  $\Rightarrow \theta = 2\pi - \frac{\pi}{6}, -\pi + \frac{\pi}{6}, -\frac{\pi}{6}, \pi + \frac{\pi}{6}$

  Thus, sum of values is $2\pi$.
  %168
\item Using A.M.-G.M.\ on $\sin^4\alpha, 4\cos^4\beta, 1$ and $1$ we get

  $\frac{\sin^4\alpha + 4\cos^4\beta + 1 +
  1}{4}\geq \left[\left(\sin^4\alpha\right)\left(4\cos^4\beta\right).1.1\right]^{1/4}$

  $\Rightarrow \sin^4\alpha + \cos^4\beta + 2\geq 4\sqrt{2}\sin\alpha\cos\beta$

  But it is given that $\sin^4\alpha + \cos^4\beta + 2 = 4\sqrt{2}\sin\alpha\cos\beta$

  $\Rightarrow \sin^4\alpha = 1 = 4\cos^4\beta$ for equality.

  $\Rightarrow \sin\alpha = 1$ and $sin\beta = \frac{1}{\sqrt{2}}$

  $\cos(\alpha + \beta) - \cos(\alpha - \beta) = -2\sin\alpha\sin\beta = -\sqrt{2}$.
  %169
\item Given quadratic equation is $x^2\sin\theta - x(\sin\theta + \cos\theta + 1) + \cos\theta = 0$

  $\Rightarrow (x - \cos\theta)(x\sin\theta - 1) = 0 \Rightarrow x = \cos\theta, \csc\theta$

  Now $\displaystyle\sum_{n = 0}^\infty\left(\alpha^n + \frac{(-1)^n}{\beta^n}\right) = \left(1 + \alpha
  + \alpha^2 + \alpha^3 + \cdots + \infty\right)\left(1 - \frac{1}{\beta} + \frac{1}{\beta^2
    - \frac{1}{\beta^3} + \cdots infty}\right)$

  $= \frac{1}{1 - \alpha} + \frac{1}{1 - \left(-\frac{1}{\beta}\right)} = \frac{1}{1 - \alpha} + \frac{1}{1
    + \frac{1}{\beta}}$ [this sum is applicable because for $0 < \theta < 45^\circ, 0< \cos\theta
    < \frac{1}{\sqrt{2}}$ and $0 < \frac{1}{\csc\theta} < \frac{1}{\sqrt{2}}$]

  $= \frac{1}{1  \cos\theta} + \frac{1}{1 + \sin\theta}$.
  %170
\item Given equation is $\sin^22\theta + \cos^42\theta = \frac{3}{4} \Rightarrow 1 - \cos^22\theta
  + \cos^42\theta = \frac{3}{4}$

  $\Rightarrow 4\cos^42\theta - 4\cos^22\theta + 1 = 0 \Rightarrow \left(2\cos^22\theta - 1\right)^2 = 0$

  $\Rightarrow \cos2\theta = \pm\frac{1}{\sqrt{2}}$

  $\Rightarrow \theta = \frac{\pi}{8}, \frac{3\pi}{8}$

  So sum of values is $\frac{\pi}{2}$.
  %171
\item We have $\sin x - \sin2x + \sin3x = 0\Rightarrow (\sin x + \sin3x) - \sin2x = 0$

  $\Rightarrow 2\sin2x\cos x - \sin2x = 0 \Rightarrow \sin2x(2\cos x - 1) = 0$

  $\Rightarrow x = 0, \frac{\pi}{3}$ because $0 \le x < \frac{\pi}{2}$.
  %172
\item We know that $\cos(x + y)\cos(x - y) = \cos^2x\sin^2y$ so the given equation becomes

  $8\cos x\left(\cos^2\frac{\pi}{6} - \sin^2x - \frac{1}{2}\right) = 1$

  $\Rightarrow 8\cos x\left(\frac{3}{4} - \sin^2x - \frac{1}{2}\right) = 1$

  $\Rightarrow 8\cos x\left(\frac{1}{4} - 1 + \cos^2x\right) = 1$

  $\Rightarrow 2(4\cos^3x - 3\cos x) = 1\Rightarrow \cos3x = \frac{1}{2}$

  $\Rightarrow x = \frac{\pi}{9}, \frac{5\pi}{9}, \frac{7\pi}{9}$

  $\therefore $ Sum $= \frac{13\pi}{9} = k\pi \Rightarrow k = \frac{13}{9}$.
  %173
\item Give equation is $5\left(\tan^2x - \cos^2x\right) = 2\cos2x + 9$

  $\Rightarrow 5\left(\frac{1 - \cos2x}{1 + \cos2x} - \frac{1 + \cos2x}{2}\right) = 2\cos2x + 9$

  Let $\cos2x = y$ then the above equation transforms to

  $5\left(\frac{1 - y}{1 + y} - \frac{1 + y}{2}\right) = 2y + 9$

  $\Rightarrow 9y^2 + 42y + 13 = 0 \Rightarrow y = -\frac{1}{3}, -\frac{13}{3}$

  Therefore $\cos2x = -\frac{1}{3} \Rightarrow \cos4x = 2\cos^22x - 1 = \frac{2}{9} - 1 = -\frac{7}{9}$.
  %174
\item Given equation is $\cos x + \cos2x + \cos3x + \cos4x = 0$

  $\Rightarrow (\cos x + \cos3x) + (\cos2x + \cos4x) = 0\Rightarrow 2\cos x(\cos2x + \cos3x) = 0$

  $\Rightarrow \cos x\cos\frac{x}{2}\cos\frac{5x}{2} = 0$

  $\Rightarrow \cos x = 0 \Rightarrow x = \frac{\pi}{2}, \frac{3\pi}{2}$

  or $\cos\frac{x}{2} = 0 \Rightarrow x = \pi$

  or $\cos\frac{5x}{2} = 0 \Rightarrow x
  = \frac{\pi}{5}, \frac{3\pi}{5}, \pi, \frac{7\pi}{5}, \frac{9\pi}{5}$.
  %175
\item Given equation is $\sqrt{3}\sin x + \cos x + 2(\sin^2x - \cos^2x) = 0$

  $\Rightarrow \sqrt{3}\sin x + \cos x - 2\cos2x = 0$

  $\Rightarrow 2\left(\cos x.\cos\frac{\pi}{3} + \sin x\sin\frac{\pi}{3}\right) - 2\cos2x = 0$

  $\Rightarrow \cos\left(x - \frac{\pi}{3}\right) = \cos 2x\Rightarrow 2x = 2b\pi\pm\left(x
  - \frac{\pi}{3}\right)$

  $\Rightarrow x = 2n\pi - \frac{\pi}{3}$ and $x = \frac{2n\pi}{3} + \frac{\pi}{9}$

  $\therefore x = -\frac{\pi}{3}$ or $x = \frac{\pi}{9}, -\frac{5\pi}{9}, \frac{7\pi}{9}$.
  %176
\item $P = \left\{\theta: \sin\theta - \cos\theta = \sqrt{2}\cos\theta\right\}$

  $\Rightarrow \cos\theta(\sqrt{2} + 1) = \sin\theta \Rightarrow \tan\theta = \sqrt{2} + 1$

  $Q = \left\{\sin\theta + \cos\theta = \sqrt{2}\sin\theta\right\}$

  $\Rightarrow \sin\theta + \cos\theta = \sqrt{2}\sin\theta$

  $\Rightarrow \sin\theta(\sqrt{2} - 1) = \cos\theta \Rightarrow \tan\theta = \sqrt{2} + 1$

  $\therefore P = Q$.
  %177
\item $2\sin^2\theta - 3\sin\theta - 2 = 0\Rightarrow (2\sin\theta + 1)(\sin\theta - 2) = 0$

  $\sin\theta\ne 2\Rightarrow \sin\theta = -\frac{1}{2}\Rightarrow \theta = n\pi + (-1)^n\frac{7\pi}{6}$.
  %178
\item Clearly, $\sin3x = k \Rightarrow \sin 3A = k$ and $\sin3B = k$

  $\Rightarrow \sin3A - \sin3B = 0 \Rightarrow \cos3\left(\frac{A + B}{2}\right)\sin\frac{3(A - B)}{2} = 0$

  Given that $A > B\Rightarrow \cos\frac{3(A + B)}{2} = 0$

  $\Rightarrow \frac{3(A + B)}{2} = \frac{\pi}{2}\Rightarrow A + B = \frac{\pi}{3}\Rightarrow C
  = \frac{2\pi}{3}$.
  %179
\item We can write the given equation as $(\sin 3x + \sin x) - 3\sin2x = (\cos3x + \cos x) - 3\cos2x$

  $\Rightarrow 2\sin2x\cos x - 3\sin2x = 2\cos2x\cos x - 3\cos2x$

  $\Rightarrow \sin2x(2\cos x - 3) = \cos2x(2\cos x - 3)$

  $\Rightarrow \tan2x = 1[\because \cos x\ne 3]$

  $\Rightarrow 2x = n\pi + \frac{\pi}{4}\Rightarrow x = \frac{n\pi}{2} + \frac{\pi}{8}$.
  %180
\item Using $R_3\rightarrow R_3 - R_1$ and $R_2\rightarrow R_2 - R_1$ gives us

  $\startdeterminant\NC 1 + \sin^2\theta\NC \cos^2\theta\NC 4\sin4\theta\NR\NC -1\NC 1\NC 0\NR\NC -1\NC
  0\NC 1\NR\stopdeterminant = 0$

  Applying $C_1\rightarrow C_1 + C_2$

  $\startdeterminant\NC 2\NC \cos^2\theta\NC 4\sin4\theta\NR\NC 0\NC 1\NC 0\NR\NC -1\NC 0\NC
  1\NR\stopdeterminant = 0$

  $\Rightarrow 2 + 4\sin4\theta = 0\Rightarrow \sin4\theta = -\frac{1}{2}$

  $\Rightarrow \theta = \frac{n\pi}{4} + (-1)^{n + 1}\frac{\pi}{24}$

  $\Rightarrow \theta = \frac{7\pi}{24}, \frac{11\pi}{24}$.
  %181
\item Since $\alpha, \beta$ are the roots of the given equation, therefore we can write

  $\sqrt{3}a\cos\alpha + 2b\sin\alpha = c$ and $\sqrt{3}a\cos\beta + 2b\sin\beta = c$

  Subtracting we get $\sqrt{3}a(\cos\alpha - \beta) + 2b(\sin\alpha - sin\beta) = 0$

  $\Rightarrow \sqrt{3}a\left[-2\sin\frac{\alpha + \beta}{2}\sin\frac{\alpha - \beta}{2}\right] +
  2b\left[2\cos\frac{\alpha + \beta}{2}\sin\frac{\alpha - \beta}{2}\right] = 0$

  $\Rightarrow \tan\frac{\alpha + \beta}{2} = \frac{2b}{\sqrt{3}a}$

  $\Rightarrow \tan\frac{\pi}{6} = \frac{1}{\sqrt{3}} = \frac{2b}{\sqrt{3}a}\Rightarrow \frac{b}{a}
  = \frac{1}{2}$.
  %182
\item Given equation is $\frac{5}{4}\cos^22x + \cos^4x + \sin^4x + \cos^6x + \sin^6x = 2$

  $\frac{5}{4}\cos^22x + \left[\left(\sin^2x + \cos^2x\right)^2 - 2\sin^2x\cos^2x\right]
  + \left[\left(\sin^2x + \cos^2x\right)^3 - 3\sin^2x\cos^2x\left(\sin^2x + \cos^2x\right)\right] = 2$

  $\Rightarrow \frac{5}{4}\cos^22x - 5\sin^2x\cos^2 = 0\Rightarrow \frac{5}{4}\cos^22x - \frac{5}{4}\sin^22x
  = 0$

  $\Rightarrow \tan2x = \pm 1$

  $\Rightarrow x
  = \left\{\frac{\pi}{8}, \frac{3\pi}{8}, \frac{5\pi}{8}, \frac{9\pi}{8}, \frac{11\pi}{8}, \frac{13\pi}{8}, \frac{15\pi}{8}\right\}$.
  %183
\item Rwriting $\tan^2\theta + \frac{1}{\cos2\theta} = 1 \Rightarrow \tan^2\theta + \frac{1
  + \tan^2\theta}{1 - \tan^2\theta} = 1$

  $\Rightarrow \tan^4\theta - 3\tan^2\theta = 0$

  $\Rightarrow \tan\theta = 0$ or $\tan\theta = \pm\sqrt{3}$

  $\Rightarrow \theta = m\pi$ or $\theta = n\pi\pm\frac{\pi}{3}$, where $m, n$ are integers.
  %184
\item Given that $\tan\left(x + 100^\circ\right) = \tan\left(x + 50^\circ\right)\tan\left(x -
  50^\circ\right)$

  $\Rightarrow \frac{\sin\left(x + 100^\circ\right)\cos x}{\cos\left(x + 100^\circ\right)\sin x}
  = \frac{\sin\left(x + 50^\circ\right)\sin\left(x - 50^\circ\right)}{\cos\left(x +
    50^\circ\right)\cos\left(x - 50^\circ\right)}$

  $\Rightarrow \frac{\sin\left(2x + 100^\circ + \sin100^\circ\right)}{\sin\left(2x + 100^\circ\right)
    - \sin100^\circ} = \frac{\cos100^\circ - \cos2x}{\cos100^\circ + \cos 2x}$

  $\Rightarrow \left[\sin\left(2x + 100^\circ + \sin100^\circ\right)\right]\left[\cos100^\circ + \cos
    2x\right] = \left[\cos100^\circ - \cos2x\right]\left[\sin\left(2x + 100^\circ\right)
    - \sin100^\circ\right]$

  $\Rightarrow 2\sin\left(2x + 100^\circ\right)\cos2x + 2\sin100^\circ\cos100^\circ = 0$

  $\Rightarrow \sin\left(4x + 100^\circ\right) + \sin100^\circ + \sin200^\circ = 0$

  $\Rightarrow \sin\left(4x + 100^\circ\right) + 2\sin150^\circ\cos50^\circ = 0$

  $\Rightarrow \sin\left(4x + 100^\circ\right) + 2.\frac{1}{2}\sin\left(90^\circ - 40^\circ\right) = 0$

  $\Rightarrow \sin\left(4x + 100^\circ\right) = \sin\left(-40^\circ\right)$

  $\Rightarrow 4x + 100^\circ = n\pi + (-1)^n\left(-40^\circ\right)$

  The smallest value of positive $x$ is obtained with $n = 1$, which is $30^\circ$.
  %185
\item Since the gives system of equations has non-trivial solutions. Therefore

  $\startdeterminant\NC \sin3\theta\NC -1\NC 1\NR\NC \cos2\theta\NC 4\NC 3\NR\NC 2\NC 7\NC
  7\NR\stopdeterminant = 0 \Rightarrow \sin3\theta(28 - 21) + 1(\cos2\theta - 6) + 1(7\cos2\theta - 8) = 0$

  $\Rightarrow \sin3\theta + 2\cos2\theta - 2 = 0$

  $\Rightarrow 3\sin\theta - 4\sin^3\theta + 2\left(1 - 2\sin^2\theta\right) - 2 = 0$

  $\Rightarrow \sin\theta\left(4\sin^2\theta + 4\sin\theta - 3\right) = 0$

  $\Rightarrow \sin\theta = 0 \Rightarrow \theta = n\pi$

  or $4\sin^2\theta + 4\sin\theta - 3 = 0 \Rightarrow (2\sin\theta - 1)(2\sin\theta + 3) = 0$

  $\Rightarrow \sin\theta = \frac{1}{2}$ as $\sin\theta \ne-\frac{3}{2}$

  $\Rightarrow \theta = n\pi + (-1)^n\frac{\pi}{6}$.
  %186
\item Given that $\exp\left\{\left(\sin^2x + \sin^4x + \sin^6x + \cdots \infty\right)\log_e2\right\}$

  $= e^{\frac{\sin^2x}{1 - \sin^2x}\log_e2} = e^{\log_e2\tan^2x} = 2^{\tan^2x}$ is a root of $x^2 - 9x + 8 =
  0$

  $\Rightarrow \tan^2x = 0, 3\Rightarrow x = n\pi, n\pi\pm\frac{\pi}{3}$

  However, $x\ne n\pi$ because $0 < x <\frac{\pi}{2}\Rightarrow x = \frac{\pi}{3}$

  $\Rightarrow \frac{\cos x}{\cos x + \sin x} = \frac{\frac{1}{2}}{\frac{1}{2} + \frac{\sqrt{3}}{2}}
  = \frac{\sqrt{3} - 1}{2}$.
  %187
\item Given that $2\cos x + 2\cos2x + \sin2x + \sin3x + \sin x - 2\sin x = 0$

  $\Rightarrow 2\cos x + 2\cos2x + 2\sin x\cos x + (\sin3x - \sin x) = 0$

  $\Rightarrow 2\cos x + 2\cos2x + 2\sin x\cos x + 2\cos2x\sin x = 0$

  $\Rightarrow 2\cos x(1 + \sin x) + 2\cos2x(1 + \sin x) = 0$

  $\Rightarrow 2(1 + \sin x)(\cos x + \cos 2x) = 0$

  $\Rightarrow 4(1 + \sin x)\cos\frac{3x}{2}\cos\frac{x}{2} = 0$

  $\Rightarrow 1 + \sin x = 0$ or $\cos\frac{x}{2} = 0$ or $\cos\frac{3x}{2} = 0$

  $x = 2n\pi + \frac{3\pi}{2}$ or $\frac{x}{2} = (2n + 1)\frac{\pi}{2}$ or $\frac{3x}{2} = (2n +
  1)\frac{\pi}{2}$

  In the given interval $x = -\pi, -\frac{\pi}{2}, -\frac{\pi}{3}, \frac{\pi}{3}, \pi$.
  %188
\item Given that $\left(\cot^{-1}x\right)^2 - 7\cot^{-1}x + 10 > 0$

  $\Rightarrow \cot^{-1}x < 2$ or $\cot^{-1}x > 5$.

  $\Rightarrow \cot^{-1}x\in(-\infty, 2)\cup (5, \infty)$

  $\cot^{-1}x\in(0, 2)$[$\because$ Range of $\cot^{-1}x$ is $(0, \pi)$]

  $\therefore x\in(\cot 2, \infty)$.
  %189
\item Rewriting the given equation $(\sin x - \sin3x) + 2\sin2x = 3$

  $\Rightarrow -2\cos2x\sin x + 4\sin x\cos x = 3$

  $\Rightarrow 2\sin x(2\cos x - 2\cos^2x + 1) = 3$

  $\Rightarrow 2\sin x\left[\frac{3}{2} - 2\left(\cos x - \frac{1}{2}\right)^2\right] = 3$

  $\Rightarrow 3\sin x - 3 = 4\left(\cos x - \frac{1}{2}\right)^2\sin x$

  As $x \in(0, \pi)$, so L.H.S.\ $\le 0$ and R.H.S.\ $\ge 0$

  For solution to exist L.H.S. = R.H.S $= 0$.

  $\Rightarrow 3\sin x - 3 = 0\Rightarrow x = \frac{\pi}{2}$

  At $x = \frac{\pi}{2}$, R.H.S. $1\ne 0$.
  %190
\item $2\sin^2\theta - \cos2\theta = 0\Rightarrow \sin^2\theta = \frac{1}{4}$

  and $2\cos^2\theta - 3\sin\theta = 0 \Rightarrow \sin\theta = \frac{1}{2}[\because \sin\theta + 2 \ne 0]$

  $\Rightarrow \sin\theta = \frac{1}{2}$

  Thus, two solutions exist in $[0, 2\pi]$.
  %191
\item We know that $-\sqrt{a^2 + b^2}\leq a\sin x + b\cos x\le \sqrt{a^2 + b^2}$

  $\Rightarrow -\sqrt{74}\le 7\cos x + 5\sin x\le \sqrt{74}$

  Since $k$ is an integer, therefore, $-\sqrt{74}\le 2k + 1\le \sqrt{74}$

  $-5< k < 4$. Thus, we have $8$ solutions.
  %192
\item Given equation can be written as $(3\sin x - 1)(\sin x - 2) = 0$

  Since $\sin x\ne 2$, therefore $\sin x = \frac{1}{3}$, which has six solutions in $[0, 5\pi]$.
  %193
\item $\tan x + \sec x = 2\cos x, x\not\in(2n + 1)\frac{\pi}{2}$

  $\Rightarrow \sin x + 1 = 2\cos^2x\Rightarrow 2\sin^2x + \sin x - 1 = 0$

  $\Rightarrow (2\sin x - 1)(\sin x + 1) = 0$

  $\Rightarrow \sin x = \frac{1}{2}, \sin x = -1 \Rightarrow x
  = \frac{\pi}{6}, \frac{5\pi}{6}, \frac{3\pi}{2}$ but $x\not\in(2n + 1)\frac{\pi}{2}$

  Hence, $x = \frac{\pi}{6}, \frac{5\pi}{6}$.
  %194
\item $\sin\left(e^x\right) < 1$ and $5^x + 5^{-x} \ge 2$. Therefore, the given equation has no solution.
  %195
\item $D = \cos^2p - 4\sin p(\cos p - 1)\geq 0$ because $x$ is real.

  $\Rightarrow (\cos p - 2\sin p)^2 - 4\sin^2p + 4\sin p\geq 0$

  $\Rightarrow 4\sin p(1 - \sin p)> 0$ for $0 < p < \pi$

  Thus, $p$ can take any value in the interval $(0, \pi)$ because $(\cos p - 2\sin p)^2$ is always positive.
  %196
\item Given, $a_1 + a_2\cos2x + a_3\sin^2x = 0\forall x$

  $\Rightarrow a_1 + a_2\cos2x + a_3\left(\frac{1 - \cos2x}{2}\right) = 0$

  $\Rightarrow a_1 + \frac{a_3}{2} + \left(a_2 - \frac{a_3}{2}\right)\cos2x = 0$

  $\Rightarrow a_1 + \frac{a_3}{2} = 0$ and $a_2- \frac{a_3}{2} = 0$

  $\Rightarrow a_1 = -\frac{k}{2}, a_2 = \frac{k}{2}, a_3 = k$, where $k\in\mathbb{R}$

  Hence, the solutions are $\left(-\frac{k}{2}, \frac{k}{2}, k\right)$, where $k\in\mathbb{R}$.
  %197
\item $f(x) = (3 - \sin2\pi x)\left[\frac{\sin\pi x}{\sqrt{2}} - \frac{\cos\pi x}{\sqrt{2}}\right]
  - \left\{\frac{\sin3\pi x}{\sqrt{2}} + \frac{\cos3\pi x}{\sqrt{2}}\right\}$

  $= (3 - \sin2\pi x)\frac{\left[\sin\pi x - \cos\pi x\right]}{\sqrt{2}} - \frac{1}{\sqrt{2}}[3\sin\pi x -
    4\sin^3\pi x + 4\cos^3\pi x - 3\cos\pi x]$

  $= \frac{\sin\pi x - \cos\pi x}{\sqrt{2}}[3 - \sin2\pi x - 3+ 4\left(\sin^2\pi x + \cos^2\pi x + \sin\pi
    x\cos\pi x\right)]$

  $= \frac{\sin\pi x - \cos\pi x}{\sqrt{2}}[4 + \sin2\pi x]$

  $\Rightarrow \pi x\in\left[\frac{\pi}{4}, \frac{4\pi}{4}\right]\Rightarrow
  x\in\left(\frac{1}{4}, \frac{5}{4}\right)$

  $\alpha = \frac{1}{4}, \beta = \frac{5}{4}\Rightarrow \beta - \alpha = 1$.
  %198
\item Given that $\frac{1}{\sin\frac{\pi}{n}} = \frac{1}{\sin\frac{2\pi}{n}} + \frac{1}{\sin\frac{3\pi}{n}}$

  $\Rightarrow \frac{1}{\sin\frac{\pi}{n}} - \frac{1}{\sin\frac{3\pi}{n}} = \frac{1}{\sin\frac{2\pi}{n}}$

  $\Rightarrow \frac{\sin\frac{3\pi}{n} - \sin\frac{\pi}{n}}{\sin\frac{\pi}{n}\sin\frac{3\pi}{n}}
  = \frac{1}{\sin\frac{2\pi}{n}}$

  $\Rightarrow 2\cos\frac{2\pi}{n}\sin\frac{\pi}{n}
  = \frac{\sin\frac{\pi}{n}\sin\frac{3\pi}{n}}{\sin\frac{2\pi}{n}}$

  $\Rightarrow 2\sin\frac{2\pi}{n}\cos\frac{2\pi}{n} = \sin\frac{3\pi}{n}$

  $\Rightarrow \sin\frac{4\pi}{n} = \frac{3\pi}{n}\Rightarrow \frac{4\pi}{n} = \pi
  - \frac{3\pi}{n}\Rightarrow n = 7$.
  %199
\item $\tan\theta = \cot5\theta \Rightarrow n\pi + \theta = \frac{\pi}{2} - 5\theta\Rightarrow \theta
  = \frac{\pi}{12} - \frac{n\pi}{6}$

  and $\cos4\theta = \sin2\theta = \cos\left(\frac{\pi}{2} - 2\theta\right)$

  $\Rightarrow 4\theta = 2n\pi\pm\left(\frac{\pi}{2} - 2\theta\right)$

  $\Rightarrow 6\theta = 2n\pi + \frac{\pi}{2}$ or $2\theta = 2n\pi - \frac{\pi}{2}\Rightarrow \theta = n\pi
  - \frac{\pi}{4}$

  Thus, we will have only three possible values for $\theta$.
  %200
\item $\cos x + \cos\left(\frac{2\pi}{3} - x\right) = \frac{3}{2}\Rightarrow \frac{1}{2}\cos x
  + \frac{\sqrt{3}}{2}\sin x = \frac{3}{2}$

  $\Rightarrow \sin\left(\frac{\pi}{6} + x\right) = \frac{3}{2}$, which is impossible.
  %201
\item Since $\cos\theta \le 1\Rightarrow \log(\cos\theta) < 0$ and $\cos(\log\theta) > 0$

  Therefore, $\cos(\log\theta) > \log(\cos\theta)$.
  %202
\item Given $\cos(p\sin x) = \sin(p\cos x) = \cos\left(\frac{\pi}{2} - p\cos x\right)$

  $\Rightarrow p\sin x = 2n\pi \pm \left(\frac{\pi}{2} - p\cos x\right)$

  $\Rightarrow p\sin x + p\cos x = 2n\pi + \frac{\pi}{2}$ or $p\sin x - p\cos x = 2n\pi - \frac{\pi}{2}$

  $\Rightarrow p\sqrt{2}[\sin(x + \pi/4)] = \frac{(4n + 1)\pi}{2}$ and $p\sqrt{2}[\sin(x - \pi/4)]
  = \frac{(4n - 1)\pi}{2}$

  $\Rightarrow -p\sqrt{2}\le \frac{(4n + 1)\pi}{2}\le p\sqrt{2}$ or $-p\sqrt{2}\le \frac{(4n - 1)\pi}{2}\le
  p\sqrt{2}$

  If $n\ge 0$ then $-\sqrt{2}p \le (4n + 1)\pi/2\Rightarrow (4n + 1)\pi/2\le\sqrt{2}p$

  For $p$ to be least $n$ has to be least. Let $n = 0$ then $p \ge \frac{\pi}{2\sqrt{2}}$.

  Therefore least value is $\frac{\pi}{2\sqrt{2}}$.
  %203
\item $\frac{|x| + 5}{x^2 + 1}\le 1\Rightarrow |x| + 4\ge x^2 \Rightarrow x^2 - |x| - 4\ge 0$

  $\Rightarrow |x|^2 - |x| - 4\ge 0\Rightarrow |x| = \frac{1 \pm \sqrt{17}}{2}$

  $\Rightarrow |x| = \frac{1 + \sqrt{17}}{2}$ as $|x|\ge 0$

  $\Rightarrow |x| \ge \frac{1 + \sqrt{17}}{2}$

  $\Rightarrow a = \frac{1 + \sqrt{17}}{2}$.
  %204
\item For the functions to be defined $x(x + 1)\ge 0, x^2 + x + 1\ge 0, \sqrt{x^2 + x + 1}\le 1$

  Thus, $0\le x^2 + x + 1\le 1\cap x^2 + x\ge 0$

  $\Rightarrow -0\le x^2 + x + 1\le 1\cap x^2 + x + 1\ge 1\Rightarrow x^2 + x + 1 = 1$

  $\Rightarrow x(x + 1) = 0 \Rightarrow x = 0, -1$.
  %205
\item $A = \tan^{-1}\left(2\sqrt{2} - 1\right) = \tan^{-1}1.828\therefore A > 2\tan^{-1}\sqrt{3}
  = \frac{2\pi}{3}$

  $\sin^{-1}\frac{1}{3} < \sin^{-1}\frac{1}{2} = \frac{\pi}{6}$ so that $0 < 3\sin^{-1}\frac{1}{3}
  < \frac{\pi}{2}$

  $3\sin^{-1}\frac{1}{3} = \sin^{-1}\left(3.\frac{1}{3} - 4.\frac{1}{27}\right) = \sin^{-1}\frac{23}{27}$

  $= \sin^{-1}0.851 < \sin^{-1}\frac{\sqrt{3}}{2} = \frac{\pi}{3}$

  $\sin^{-1}\frac{3}{5} < \sin^{-1}\frac{\sqrt{3}}{2} = \frac{\pi}{3}$

  So $B < \frac{2\pi}{3}$. Hence, $A$ is greater.
  %206
\item Given $\displaystyle f(n) = \frac{\displaystyle \sum_{k =
    0}^n\sin\left(\frac{k + 1}{n + 2}\pi\sin\frac{k + 2}{n + 2}\pi\right)}{\displaystyle\sum_{k =
    0}^n\sin^2\left(\frac{k + 1}{n + 2}\pi\right)}$

  $= \displaystyle\frac{\displaystyle\sum_{k = 0}^n\left[\cos\frac{\pi}{n + 2} - \cos\frac{2k + 3}{n +
      2}\pi\right]}{\displaystyle\sum_{k = 0}^n\left[1 - \cos\frac{2k + 2}{n + 2}\pi\right]}$

  $= \displaystyle\frac{\displaystyle\left(\cos\frac{\pi}{n + 2}\right)(n + 1) - \left[\cos\frac{3\pi}{n +
      2} + \cos\frac{5\pi}{n + 2} + \cos\frac{7\pi}{n + 2} + \cdots + \cos\frac{(2n + 3)\pi}{n +
      2}\right]}{\displaystyle(n + 1) - \left[\cos\frac{2\pi}{n +
      2} + \cos\frac{4\pi}{n + 2} + \cos\frac{6\pi}{n + 2} + \cdots + \cos\frac{(2n + 2)\pi}{n +
      2}\right]}$

  $= \displaystyle \frac{\displaystyle(n + 1)\cos\frac{\pi}{n + 2} - \frac{\sin\frac{n\pi}{n +
        2}}{\sin\frac{\pi}{n + 2}}\cos\frac{n + 3}{n + 2}\pi}{n + 1 - \displaystyle\frac{\sin\frac{n\pi}{n +
      2}}{\sin\frac{\pi}{n + 2}}\cos\frac{n + 2}{n + 2}\pi}$

  $=\displaystyle\frac{(n + 1)\cos\frac{\pi}{n + 2} - \frac{\sin\left(\pi - \frac{\pi}{n +
      2}\right)}{\sin\frac{\pi}{n + 2}}\cos\left(\pi + \frac{\pi}{n + 2}\right)}{(n + 1)
  - \frac{\sin\left(\pi - \frac{\pi}{n + 2}\right)}{\sin\left(\frac{\pi}{n + 2}\right)}\cos\pi}$

  $= \cos\frac{\pi}{n + 2}$

  $f(8) = \cos\frac{\pi}{8}\Rightarrow \alpha = \tan\left(\cos^{-1}f(6)\right) = \tan\frac{\pi}{8} = \sqrt{2
  - 1}$

  $\Rightarrow \alpha^2 + 2\alpha - 1 = 0$

  $f(4) = \cos\frac{\pi}{6} = \frac{\sqrt{3}}{2}$

  $\sin\left(7\left(\cos^{-1}f(5)\right)\right) = \sin7.\frac{\pi}{7} = 0$.
  %207
\item $fog(x) = f\left(\sin^{-1}\left(e^{-x}\right)\right)
  = \log_e\left[\sin\left(\sin^{-1}\left(e^{-x}\right)\right)\right] = \log_ee^{-x} = -x$

  $(fog)'(x) = -1$

  Then according to question $a = -1$ and $b = -\alpha$.

  Thus, $a\alpha^2 - b\alpha - a = 1$.
  %208
\item We have to find the value of $\displaystyle\cot\left[\sum_{n = 1}^{19}\cot^{-1}\left(1 + \sum_{p =
    1}^n2p\right)\right]$

  $= \displaystyle\cot\left[\sum_{n = 1}^{19}\cot^{-1}[1 + n(n + 1)]\right] = \cot\left[\sum_{n =
    1}^{19}\tan^{-1}\frac{1}{1 + n(n + 1)}\right]$

  $= \cot\displaystyle\sum_{n =1}^{19}\left[\tan^{-1}(n + 1) - \tan^{-1}n\right]$

  $\cot\left(\tan^{-1}20 - \tan^{-1}1\right) = \cot\left[\left(\frac{\pi}{2} - \cot^{-1}20\right)
  - \left(\frac{\pi}{2} - \cot^{-1}1\right)\right]$

  $= \cot\left(\cot^{-1}1 - \cot^{-1}20\right) = \frac{\cot\cot{-1}1\cot\cot^{-1}20 + 1}{\cot\cot^{-1}20
  - \cot\cot^{-1}1}$

  $= \frac{21}{19}$.
  %209
\item We know that $\sin^{-1}\alpha + \cos^{-1}\alpha = \frac{\pi}{2}$, therefore

  $x - \frac{x^2}{2} + \frac{x^3}{4}- \cdots = x^2 - \frac{x^4}{2} + \frac{x^6}{4} - \cdots$

  $\Rightarrow \frac{x}{1 + \frac{x}{2}} = \frac{x^2}{1 + \frac{x^2}{2}}$

  $\Rightarrow x(4 - 4x) = 0 \Rightarrow x = 0, 1$

  Given that $0 < |x| <\sqrt{2}\Rightarrow x = 1$.
  %210
\item We have to find the value of $\displaystyle\sec^{-1}\left[\frac{1}{4}\sum_{k =
    0}^{10}\sec\left(\frac{7\pi}{12} + \frac{k\pi}{2}\right)\sec\left(\frac{7\pi}{12} + \frac{(k +
    1)\pi}{2}\right)\right]$

  Now $\displaystyle\sum_{k = 0}^{10}\frac{1}{\cos\left(\frac{7\pi}{12}
    + \frac{k\pi}{2}\right)\cos\left(\frac{7\pi}{12} + \frac{(k + 1)\pi}{2}\right)}$

  $= \displaystyle\sum_{k = 0}^{10}\frac{\sin\left[\left(\frac{7\pi}{12} + \frac{(k + 1)\pi}{2}\right)
      - \sin\left(\frac{7\pi}{12} + \frac{k\pi}{2}\right)\right]}{\cos\left(\frac{7\pi}{12}
    + \frac{k\pi}{2}\right)\cos\left(\frac{7\pi}{12} + \frac{(k + 1)\pi}{2}\right)}$

  $= \displaystyle\sum_{k = 0}^{10}\left[\tan\left(\frac{7\pi}{12} + \frac{(k + 1)\pi}{2}\right)
    - \tan\left(\frac{7\pi}{12} + \frac{k\pi}{2}\right)\right]$

  $= \tan\left(\frac{7\pi}{12} + \frac{11\pi}{2}\right) - \tan\frac{7\pi}{12} = \tan\frac{\pi}{12}
  + \cot\frac{\pi}{12}$

  $= \frac{1}{\sin\frac{\pi}{12}\cos\frac{\pi}{12}} = \frac{2}{\sin\frac{\pi}{6}} = 4$

  Thus, $\displaystyle\sec^{-1}\left[\frac{1}{4}\sum_{k = 0}^{10}\sec\left(\frac{7\pi}{12}
    + \frac{k\pi}{2}\right)\sec\left(\frac{7\pi}{12} + \frac{(k + 1)\pi}{2}\right)\right] = \sec^{-1}1 = 0$.
  %211
\item We have $\sin^{-1}\left[\frac{x^2}{1 - x} - \frac{x.\frac{x}{2}}{1 - \frac{x}{2}}\right]
  = \frac{\pi}{2} - \cos^{-1}\left[\frac{-\frac{x}{2}}{1 + \frac{x}{2}} - \frac{-x}{1 + x}\right]$

  $\Rightarrow \sin^{-1}\left[\frac{x^2}{1 - x^2} - \frac{x^2}{2 - x}\right] = \frac{\pi}{2}
  - \cos^{-1}\left[\frac{x}{1 + x} - \frac{x}{2 + x}\right]$

  $\Rightarrow \sin^{-1}\left[\frac{x^2}{1 - x^2} - \frac{x^2}{2 - x}\right] = \sin^{-1}\left[\frac{x}{1 +
      x} - \frac{x}{2 + x}\right]$

  $\Rightarrow \frac{x^2}{1 - x^2} - \frac{x^2}{2 - x} = \frac{x}{1 + x} - \frac{x}{2 + x}$

  $\Rightarrow x^3 + 3x^2 + 2x = x^2 - 3x + 2$ or $x = 0$

  Let $f(x) = x^3 + 2x^2 + 5x - 2, f'(x) = 3x^2 + 4x + 5 > 0\forall x\in\mathbb{R}$

  $\therefore x^3 + 2x^2 + 5x - 2$ has only one real root. Therefore, total no.\ of solutions is $2$.
  %212
\item We know that

  \startformula
    \cos^{-1}(\cos x) =
    \startcases[align={right,left},distance=3pt]
      \NC x \NC \text{if } x\in[0, \pi]\NR
      \NC 2\pi - x \NC \text{if } x\in[\pi, 2\pi]\NR
      \NC -2\pi + x \NC \text{if } x\in[2\pi, 3\pi]\NR
      \NC 4\pi - x \NC \text{if } x\in[3\pi, 4\pi]\NR
    \stopcases
  \stopformula

  \startplacefigure
    \externalfigure[17_11.pdf]
  \stopplacefigure

  From the graph, it is evident that $y = \frac{10 - x}{10}$ and $y = \cos^{-1}(\cos x)$ intersect at three
  different points, so number of solutions is $3$.
  %213
\item Given that $\cos^{-1}x - \cos^{-1}\frac{y}{2} = \alpha$

  $\Rightarrow \cos^{-1}\left[\frac{xy}{2} + \sqrt{1- x^2}\sqrt{1 - y^2/4}\right] = \alpha$

  $\Rightarrow \sqrt{1 - x^2}\sqrt{1 - y^2/4} = \cos\alpha - xy/2$

  $\Rightarrow (1 - x^2)(1 - y^2/4) = \cos^2\alpha - xy\cos\alpha + x^2y^2/4$

  $\Rightarrow 4x^2 - 4xy\cos\alpha + y^2 = 4\sin^2\alpha$.
  %214
\item $\theta = \tan^{-1}\left[\frac{\sqrt{a + b + c}\left(\sqrt{\frac{a}{bc}} + \sqrt{\frac{b}{ca}}
    + \sqrt{\frac{c}{ab}}\right) - (a + b + c)\sqrt{\frac{a + b + c}{abc}}}{1 - (a + b +
    c)\left(\frac{1}{a} + \frac{1}{b} + \frac{1}{c}\right)}\right]$

  $= \tan^{-1}0 \Rightarrow \tan\theta = 0$.
  %215
\item Since angles are in A.P. $\Rightarrow 2B = A + C$ and $A + B + C = 180^\circ \Rightarrow B = 60^\circ$

  $\frac{a}{b} = \frac{1}{\sqrt{3}} = \frac{\sin A}{\sin B}\Rightarrow \sin A = \frac{1}{2} \Rightarrow A =
  30^\circ$

  $\Rightarrow C = 90^\circ$

  Using sine rule $\frac{a}{\sin A} = \frac{b}{\sin B} = \frac{c}{\sin C}$

  $\Rightarrow a = c.\frac{\sin A}{\sin C} = 4.\frac{1}{2}.\frac{1}{1} = 2$ cm.

  $b = c.\frac{\sin B}{\sin C} = 4.\frac{\sqrt{3}}{2} = 2\sqrt{3}$ cm.

  Area of triangle is $\frac{1}{2}ab\sin C = \frac{1}{2}.2.2\sqrt{3} = 2\sqrt{3}$ sq.\ cm.
  %216
\item Let $\frac{b + c}{11} = \frac{c + a}{12} = \frac{a + b}{13} = k$

  $\Rightarrow b + c = 11k, c + a = 12k, a + b = 13k$ and $a + b + c = 18k$

  Thus, $a = 7k, b = 6k, c = 5k$

  $\Rightarrow \cos A = \frac{b^2 + c^2 - a^2}{2bc} = \frac{1}{5}, \cos B = \frac{19}{35}, \cos C
  = \frac{5}{7}$

  $\Rightarrow \frac{\cos A}{7} = \frac{\cos B}{19} = \frac{\cos C}{25} = \frac{1}{35}$

  $\therefore (\alpha, \beta, \gamma) = (7, 19, 25)$.
  %217
\item Given $a + b = x, ab = y, x^2 - c^2 = y$

  $\Rightarrow (a + b)^2 - c^2 = ab \Rightarrow a^2 + b^2 - c^2 = -ab \Rightarrow \frac{a^2 + b^2 -
  c^2}{2ab} = -\frac{1}{2}$

  $\Rightarrow \cos C = -\frac{1}{2}\Rightarrow \sin C = \frac{\sqrt{3}}{2}$

  Using sine rule $\frac{c}{\sin C} = 2R \Rightarrow R = \frac{c}{2}.\frac{2}{\sqrt{3}}
  = \frac{c}{\sqrt{3}}$.
  %218
\item The diagram is given below:

  \startplacefigure
    \externalfigure[17_12.pdf]
  \stopplacefigure

  Given that $\angle ADB = \theta, BC = p, CD = q$. Let $\angle ABD = \alpha \Rightarrow \angle BDC
  = \alpha\Rightarrow \angle DAB = \pi - (\theta + \alpha)$ and $BD = \sqrt{p^2 + q^2}$

  Using sine rule in $\triangle ABC$

  $\frac{AB}{\sin\theta} = \frac{\sqrt{p^2 + q^2}}{\sin[\pi - (\theta + \alpha)]} = \frac{\sqrt{p^2 +
      q^2}}{\sin(\theta + \alpha)}$

  $\Rightarrow \frac{AB}{\sin\theta} = \frac{\sqrt{p^2q^2}}{\sin\theta\cos\alpha + \cos\theta\sin\alpha}$

  $\Rightarrow AB = \frac{(p^2 + q^2)\sin\theta}{p\cos\theta + q\sin\theta}$.
  %219
\item Let $\angle A, \angle B$ and $\angle C$ are in A.P. Then, $2\angle B = \angle A + \angle C$ and
  $\angle A + \angle B + \angle C = 180^\circ \Rightarrow \angle B = 60^\circ$

  Using sine rule $\frac{a}{\sin A} = \frac{b}{\sin B} = \frac{c}{\sin C} = 2R$

  Now $\frac{a}{c}\sin2C + \frac{c}{a}\sin2A = \frac{a.2\sin C\cos C}{2R\sin C} + \frac{c.2\sin A\cos
    C}{2R\sin A}$

  $= \frac{a\cos C}{R} + \frac{c.\cos A}{R} = \frac{1}{R}(a\cos C + c\cos A) = 2\sin B = \sqrt{3}$.
  %220
\item Let $a, b, c$ be the sides of the $\triangle ABC$.

  Using sine rule we have $\frac{b + c}{a} = \frac{\sin B + \sin C}{\sin A}$

  $= \frac{2\sin\frac{B + C}{2}\cos\frac{B - C}{2}}{2\sin\frac{A}{2}\cos\frac{A}{2}} = \frac{\cos\frac{B -
      C}{2}}{\cos\frac{A}{2}}$

  Also, $\frac{b - c}{a} = \frac{\sin\frac{B - C}{2}}{\cos\frac{A}{2}}$

  Thus, $(b - c)\cos\frac{A}{2} = a\sin\frac{B - C}{2}$.
  %221
\item Let smallest angle be $x^\circ$. Then $4x + x + x = 180\circ \Rightarrow x = 30^\circ \Rightarrow 4x =
  120^\circ$

  Let $A$ be this greatest angle. Then $a$ will be longest side.

  We have to find $\frac{a}{a + b + c}$. Using sine rule

  $\frac{a}{a + b + c} = \frac{\sin120^\circ}{\sin120^\circ + 2.\sin30^\circ} = \frac{\sqrt{3}}{2
    + \sqrt{3}}$.
  %222
\item We know that in a triangle $\angle A + \angle B + \angle C = 180^\circ \Rightarrow A - B + C =
  180^\circ - 2B$

  $\Rightarrow 2ac\sin\left(\frac{1}{2}(A - B + C)\right) = 2ac\cos B$

  Using cosine rule

  $2ac\cos B = 2ac.\frac{a^2 + c^2 - b^2}{2ac} = a^2 + c^2 - b^2$.
  %223
\item Since $\tan\frac{P}{2}$ and $\tan\frac{Q}{2}$ are the roots of the equation $ax^2 + bx + c = 0(a\ne
  0)$

  $\Rightarrow \tan\frac{P}{2} + \tan\frac{Q}{2} = -\frac{b}{a}$ and $\tan\frac{P}{2}\tan\frac{Q}{2}
  = \frac{c}{a}$

  $P + Q + R = 180^\circ \Rightarrow \frac{P + Q + R}{2} = 90^\circ \Rightarrow \frac{P + Q}{2} = 45^\circ$

  $\tan\frac{P + Q}{2} = 1\Rightarrow \frac{-b/a}{1 - c/a} = 1\Rightarrow a + b = c$.
  %224
\item The diagram is given below:

  \startplacefigure
    \externalfigure[17_13.pdf]
  \stopplacefigure

  Using sine rule we have $\frac{a}{\sin A} = \frac{b}{\sin B} = \frac{c}{\sin C} = 2R$

  $\frac{1}{2}ap_1 = \Delta \Rightarrow \frac{2\Delta}{a} = p_1 \Rightarrow \frac{2\Delta}{b} = p_2$ and
  $\frac{2\Delta}{c}$

  Clearly, altitudes are in H.P. as $a, b, c$ are in A.P.
  %225
\item The diagram is given below:

  \startplacefigure
    \externalfigure[17_14.pdf]
  \stopplacefigure

  Using sine rule in $\triangle ABD, \frac{AD}{\sin\pi/3} = \frac{BD}{\sin\alpha}$

  Using sine rule in $\triangle ACD, \frac{AD}{\sin\pi/4} = \frac{CD}{\sin\beta}$

  $\Rightarrow \frac{\sqrt{3}.BD}{2\sin\alpha} = \frac{CD}{\sqrt{2}\sin\beta}$

  Given that $BD:CD = 1:3$

  $\Rightarrow \sin\alpha:\sin\beta = 1:\sqrt{6}$.
  %226
\item The diagram is given below:

  \startplacefigure
    \externalfigure[17_15.pdf]
  \stopplacefigure

  According to question $\cos P = \frac{q^2 + r^2 - p^2}{2qr} = \frac{1}{3}$

  According to question, let $PN = PM = n, QN = QL = n + 2$ and $RM = RL = n + 4$.

  Thus, $\frac{1}{3} = \frac{(2n + 4)^2 + (2n + 2)^2 - (2n + 6)^2}{2(2n + 4)(2n + 2)} = \frac{4n^2 - 16}{8(n
    + 1)(n + 2)}$

  $\Rightarrow n = 8$.

  Thus, lengths of the sides are $18, 20, 22$.
  %227
\item The diagram is given below:

  \startplacefigure
    \externalfigure[17_16.pdf]
  \stopplacefigure

  $\cos C = \frac{a^2 + b^2 - c^2}{2ab} \Rightarrow \frac{\sqrt{3}}{2} = \frac{\left(x^2 + x +
  1\right)^2 + \left(x^2 - 1\right)^2 - (2x + 1)^2}{2\left(x^2 + x + 1\right)\left(x^2 - 1\right)}$

  $\Rightarrow (x + 2)(x + 1)(x - 1)x + \left(x^2 - 1\right)^2 = \sqrt{3}\left(x^2 + x + 1\right)\left(x^2
  - 1\right)$

  $\Rightarrow x = -(2 + \sqrt{3})$ or $1 + \sqrt{3}$

  But $x$ cannot be negative because it makes $c$ negative, so $x = 1 + \sqrt{3}$.
  %228
\item Given that $\cos B + \cos C = 4\sin^2\frac{A}{2}$

  $\Rightarrow 2\cos\frac{B + C}{2}\cos\frac{B - C}{2} = 4\sin^2\frac{A}{2}$

  $\Rightarrow 2\sin\frac{A}{2}\left[\cos\frac{B - C}{2} - 2\sin\frac{A}{2}\right] = 0$

  $\Rightarrow \cos\frac{B - C}{2} - 2\cos\frac{B + C}{2} = 0\left[\because \sin\frac{A}{2}\ne 0\right]$

  $\Rightarrow -\cos\frac{B}{2}\cos\frac{C}{2} + 3\sin\frac{B}{2}\sin\frac{C}{2} = 0$

  $\Rightarrow \tan\frac{B}{2}\tan\frac{C}{2} = \frac{1}{3}$

  $\Rightarrow \sqrt{\frac{(s - a)(s - c)}{s(s - b)}.\frac{(s - b)(s - a)}{s(s - c)}} = \frac{1}{3}$

  $\Rightarrow \frac{s - a}{s} = \frac{1}{3}\Rightarrow 2s = 3a \Rightarrow b + c = 2a$.

  Thus, locus of the point $A$ is an ellipse.
  %229
\item The diagram is given below:

  \startplacefigure
    \externalfigure[17_17.pdf]
  \stopplacefigure

  $\because \triangle ABC = \triangle ABD + \triangle ACD$

  $\Rightarrow \frac{1}{2}bc\sin A = \frac{1}{2}c.AD\sin\frac{A}{2} + \frac{1}{2}b.AD\sin\frac{A}{2}$

  $\Rightarrow AD = \frac{2bc}{b + c}\cos\frac{A}{2}$

  $AE = AD\sec\frac{A}{2} = \frac{2bc}{b + c}$

  $\Rightarrow AE$ is H.M.\ of $b$ and $c$.

  $EF = ED + DF = 2DE = 2AD\tan\frac{A}{2} = 2.\frac{2bc}{b + c}\cos\frac{A}{2}\tan\frac{A}{2}
  = \frac{4bc}{b + c}\sin\frac{A}{2}$

  Since $AD\perp EF$ and $DE = DF$ and $AD$ is bisector. $\Rightarrow \triangle AEF$ is isosceles.
  %230
\item From sine formula $\frac{a}{\sin A} = \frac{b}{\sin B}$

  $b\sin A = a \Rightarrow a\sin B = a \Rightarrow B = \frac{\pi}{2}$

  Since $\angle A < \frac{\pi}{2}$, so the triangle is possible.

  $b\sin A < a \Rightarrow \sin B < 1$. Since $A < \frac{\pi}{2}$ so that the triangle is possible.
  %231
\item The diagram is given below:

  \startplacefigure
    \externalfigure[17_18.pdf]
  \stopplacefigure

  In $\triangle ADC, \frac{AD}{b} = \sin23^\circ\Rightarrow AD = b\sin23^\circ$

  But $AD = \frac{abc}{b^2 - c^2} = b\sin23^\circ$

  $\Rightarrow \frac{a}{b^2 - c^2} = \frac{\sin23^\circ}{c}$

  In $\triangle ABC, \frac{\sin A}{a} = \frac{\sin23^\circ}{c}$

  $\Rightarrow \frac{\sin A}{a} = \frac{a}{b^2 - c^2}\Rightarrow \sin A = \frac{a^2}{b^2 - c^2}$

  $= \frac{\sin^2A}{\sin^2B - \sin^2C} = \frac{\sin^2A}{\sin(B + c)\sin(B - C)}$

  $\Rightarrow \sin (B - C) = 1 \Rightarrow B - 23^\circ = 90^\circ \Rightarrow B = 113^\circ$.
  %232
\item Given that $\frac{2\cos A}{a} + \frac{\cos B}{b} + \frac{2\cos C}{c} = \frac{a}{bc}
  + \frac{b}{ca}$

  Using cosine law, we have

  $\frac{b^2 + c^2 - a^2}{abc} + \frac{c^2 + a^2 - b^2}{2abc} + \frac{a^2 + b^2 - c^2}{abc} = \frac{a}{bc}
  + \frac{b}{ca}$

  $\Rightarrow 3b^2 + c^2 + a^2 = 2a^2 + 2b^2 \Rightarrow b^2 + c^2 = a^2$

  Hence the angle $A$ is $\frac{\pi}{2}$.
  %233
\item Let $ABC$ be the triangle such that the lengths of its sides $AC, AB$ and $BC$ are $(x - 1), x$ and
  $(x + 1)$ respectively, where $x\in\mathbb{N}$ and $x > 1$. Let $\angle B = \alpha$ be the smallest angle
  and $\angle A = 2\alpha$ be the largest angle.

  Using sine rule gives us

  $\frac{\sin\alpha}{x - 1} = \frac{\sin2\alpha}{x + 1}\Rightarrow \cos\alpha = \frac{x + 1}{2(x - 1)}$

  Using cosine law $\cos\alpha = \frac{x^2 + (x + 1)^2 - (x - 1)^2}{2.x.(x + 1)} = \frac{x + 4}{2(x + 1)}$

  Thus, $\frac{x + 4}{2(x + 1)} = \frac{x + 1}{2(x - 1)}\Rightarrow x = 5$

  So the lengths of the sides are $4, 5$ and $6$ units.
  %232
\item The diagram is given below:

  \startplacefigure
    \externalfigure[17_19.pdf]
  \stopplacefigure

  Let $AD$ be the median to the base $BC = a$ of $\triangle ABC$ and let $\angle ADC = \theta$, then

  $\left(\frac{a}{2} + \frac{a}{2}\right)\cot\theta = \frac{a}{2}\cot30^\circ - \frac{a}{2}\cot45^\circ$

  $\Rightarrow \cot\theta = \frac{\sqrt{3} - 1}{2}$

  Applying sine rule in $\triangle ADC$ gives us

  $\frac{AD}{\sin\left(\pi - \theta - 45^\circ\right)} = \frac{DC}{\sin45^\circ}$

  $\Rightarrow \frac{AD}{\sin\left(\theta + 45^\circ\right)} = \frac{\frac{a}{2}}{\frac{1}{\sqrt{2}}}$

  $\Rightarrow AD = \frac{a}{\sqrt{2}}\left(\frac{\cos\theta + \sin\theta}{\sqrt{2}}\right)
  = \frac{a}{2}(\cos\theta + \sin\theta)$

  $\Rightarrow \frac{1}{\sqrt{11 - 6\sqrt{3}}} = \frac{a}{2}\left(\frac{\sqrt{3} - 1}{\sqrt{8 - 2\sqrt{3}}}
  + \frac{2}{\sqrt{8 - 2\sqrt{3}}}\right)$

  $\Rightarrow a = 2$.
  %235
\item Given that $\angle A + \angle B = 120^\circ, a = \sqrt{3} + 1$ and $b = \sqrt{3} - 1$

  $\Rightarrow \angle C = 60^\circ$

  Using tangent law gives us

  $\tan\frac{A - B}{2} = \frac{a - b}{a + b}\cot\frac{C}{2} = \frac{1}{\sqrt{3}}\cot30^\circ = 1$

  $\Rightarrow A - B = 90^\circ\Rightarrow A = 105^\circ, B = 15^\circ$

  $\angle A:\angle B = 7:1$.
  %236
\item $\frac{2\sin P - \sin 2P}{2\sin P + \sin2P} = \frac{2\sin P(1 - \cos P)}{2\sin P(1 + \cos P)}$

  $= \tan^2\frac{P}{2} = \frac{(s - b)(s - c)}{s(s - a)} = \frac{(s - b)(s - c)}{s(s - a)}\times\frac{(s -
  b)(s - c)}{(s - b)(s - c)}$

  $= \frac{[(s - b)^2(s - c)^2]}{\Delta^2} = \frac{\left(4 - \frac{7}{2}\right)^2\left(4
  - \frac{5}{2}\right)^2}{\Delta^2} = \frac{9}{16\Delta^2}$.
  %237
\item The diagram is given below:

  \startplacefigure
    \externalfigure[17_20.pdf]
  \stopplacefigure

  Let $AB = AC = a$ and given that $\angle A = 120^\circ$

  $a = AD + BD = \sqrt{3}\tan30^\circ + \sqrt{3}\cot15^\circ = 1 + \frac{\sqrt{3}}{\tan(45^\circ -
    30^\circ)}$

  $a = 1 + \sqrt{3}\left(\frac{\sqrt{3} + 1}{\sqrt{3} - 1}\right) = 4 + 2\sqrt{3}$

  Area of the $\triangle ABC = \frac{1}{2}(4 + 2\sqrt{3})^2.\frac{\sqrt{3}}{2} = \left(12 =
  7\sqrt{3}\right)$ sq. units.
  %238
\item Let $a:b:c = 1:\sqrt{3}:2$, then $c^2 = a^2 + b^2$. Thus, the triangle is right-angled at $C$.

  $\frac{a}{b} = \frac{1}{\sqrt{3}}\Rightarrow \tan A = \frac{a}{b} = \tan30^\circ \Rightarrow A =
  30^\circ \Rightarrow B = 60^\circ$

  Thus, ratio of angles is $1:2:3$.
  %239
\item Given that $\tan\frac{X}{2} + \tan \frac{Z}{2} = \frac{2y}{x + y + z}$

  $\Rightarrow \frac{\Delta}{s(s - x)} + \frac{\Delta}{s(s - z) = \frac{y}{s}}$

  $\Rightarrow \Delta\frac{(s - z + s - x)}{(s - x)(s - z)} = y$

  $\Rightarrow \Delta = (s - x)(s - z)\Rightarrow s(s - x)(s - y)(s - z) = (s - x)^2(s - z)^2$

  $\Rightarrow x^2 + z^2 = y^2 \Rightarrow y = \frac{\pi}{2}$

  $\Rightarrow X + Z = Y = \frac{\pi}{2}$

  $\tan\frac{X}{2} = \sqrt{\frac{1 - \cos X}{1 + \cos X}} = \sqrt{\frac{1 - \frac{z}{y}}{1 + \frac{z}{y}}}
  = \frac{x}{y + z}$(Using $x^2 + z^2 = y^2$).
  %240
\item Let $\angle B = 30^\circ, \angle C = 45^\circ \Rightarrow A = 105^\circ$

  Using sine rule $\frac{a}{\sin A} = \frac{b}{\sin B} = \frac{c}{\sin C}$

  Given that $a = \sqrt{3 + 1}$

  $\Rightarrow b = a\frac{\sin B}{\sin A} = \frac{(\sqrt{3} + 1)\sin30^\circ}{\sin105^\circ}$

  Area of the triangle is $\frac{1}{2}ab\sin C = \frac{1 + \sqrt{3}}{2}$ sq. cm.
  %241
\item $a^2 + 3a + 8 < a^2 + 2a + 2a + 3 \Rightarrow a > 5$

  Also, $a^2 + 3a + 8  + 2a + 3 > a^2 + 2a \Rightarrow a > -\frac{11}{3}$

  and $a^2 + 3a + 8 + a^2 + 2a > 2a + 3 \Rightarrow 2a^2 + 3a + 5 > 0$ which is always true.

  Thus, $a > 5$ for formation of triangle.
  %242
\item We have to prove that $\Delta\le \frac{1}{4}\sqrt{(a + b + c)abc}$

  $\Rightarrow \frac{abc(a + b + c)}{16\Delta^2}\ge 1\Rightarrow \frac{2s.abc}{16\Delta^2}\ge 1$

  $\Rightarrow \frac{sabc}{8s(s - a)(s - b)(s - c)}\ge 1 \Rightarrow \frac{abc}{8}\ge (s - a)(s - b)(s - c)$

  Putting $s - a = x\ge 0, s - b = y\ge 0, s - c = z\ge 0$

  $s - a + s - b = x + y \Rightarrow c = x + y$

  Similarly, $a = y + z, b = z + x$

  $\Rightarrow \frac{(x + y)(y + z)(z + x)}{8}\ge xyz$ which is true.

  Euqality will hold if $x = y = z \Rightarrow a = b = c$.
  %243
\item For a triangle to exist $c < a + b \Rightarrow c^2 < ac + bc$. Similarly $b^2 < bc + ab$ and $a^2 < ab
  + ac$

  $\Rightarrow a^2 + b^2 + c^2 < 2(ab + bc + ca)\Rightarrow (a + b + c)^2 < 4(ab + bc + ca)$

  Using AM-GM inequality we have

  $\frac{a^2 + b^2}{2}\ge ab, \frac{b^2 + c^2}{2}\ge bc$ and $\frac{c^2 + a^2}{2}\ge ca$

  $\Rightarrow a^2 + b^2 + c^2\ge ab + bc + ca\Rightarrow (a + b + c)^2\ge 3(ab + bc + ca)$.
  %244
\item Given that $A + B + C = \pi$ and $A = \pi/4\Rightarrow B + C = 3\pi/4$

  Also given $\tan B\tan C = p \Rightarrow \frac{\sin B\sin C}{\cos B\cos C} = p$

  $\Rightarrow \frac{\sin B\sin C + \cos B\cos C}{\sin B\sin C - \cos B\cos C} = \frac{p + 1}{p - 1}$

  $\Rightarrow \frac{\cos(B - C)}{\cos(B + C)} = \frac{\cos(B - C)}{\cos\frac{3\pi}{4}} = \frac{p + 1}{p -
    1}$

  $\Rightarrow \cos(B - C) = -\frac{p + 1}{\sqrt{2}(1 - p)}$

  Since $B$ or $C$ can vary from $0$ to $3\pi/4$

  $\Rightarrow -\frac{1}{\sqrt{2}}< \frac{p + 1}{\sqrt{2}(p - 1)} \le 1$

  Thus, $p < 0$ or $p > 1$ and $p < 1$ or $p > \left(\sqrt{2} + 1\right)^2$

  Thus, solution set is $p\in(-\infty, 0)\cup [\left(\sqrt{2} + 1\right)^2, \infty]$ or $p\in(0, \infty)\cup
  [3 + 2\sqrt{2}, \infty]$.
  %245
\item $\tan\frac{B}{2} = \sqrt{\frac{(s - c)(s - a)}{s(s - b)}} = \frac{\Delta}{s(s - b)}$

  Given that $a, b, c$ and $\Delta$ are rational. Therefore, $\tan\frac{B}{2}$ is rational. Similarly,
  $\tan\frac{C}{2}$ is rational.

  This shows $(i)\Rightarrow (ii)$.

  $\tan\frac{A}{2} = \tan\left(\frac{\pi}{2} - \frac{B + C}{2}\right) = \frac{1}{\tan\left(\frac{B}{2}
    + \frac{C}{2}\right)}$

  $= \frac{1 - \tan\frac{B}{2}\tan\frac{B}{2}}{\tan\frac{B}{2} + \tan\frac{C}{2}}$

  Thus, $\tan\frac{A}{2}$ is also rational.

  $\sin A = \frac{2\tan\frac{A}{2}}{1 + \tan^2\frac{A}{2}}$, which will be rational.

  Similarly, $\sin B, \sin C$ will be rational.

  Thus, $(ii)\Rightarrow (iii)$.

  Since $a, \sin A, \sin B, \sin C$ are rational so $b, c$ will be rational from sine rule and also the area
  of the triangle will be rational.

  Thus, $(iii)\Rightarrow (i)$.
  %246
\item We know that $\Delta = \frac{1}{2}ap_1 = \frac{1}{2}bp_2 = \frac{1}{2}cp_3$

  $\Rightarrow p_1p_2p_3 = \frac{8\Delta^3}{abc}$

  We know that $\Delta = \frac{abc}{4R}$, therefore, $p_1p_2p_3 = \frac{a^2b^2c^2}{8R^3}$.
  %247
\item Using the law of sines with $AC=2\sqrt2$ opposite $\angle B=30^\circ$ and $AB=4$, we get $\sin
  C=\dfrac{4\sin30^\circ}{2\sqrt2}=\dfrac1{\sqrt2}$, hence $C=45^\circ$ or $135^\circ$.

  Then $A=180^\circ-30^\circ-C$ gives $A=105^\circ$ or $15^\circ$.

  The area is $\tfrac12\cdot AB\cdot AC\cdot\sin A=4\sqrt2\sin A$, so the two areas are
  $4\sqrt2\sin105^\circ=2\sqrt3+2$ and $4\sqrt2\sin15^\circ=2\sqrt3-2$.

  Their absolute difference is $|(2\sqrt3+2)-(2\sqrt3-2)|=4$.
  %248
\item Let $\triangle ABC$ be the given triangle with vertices $B(0, 2)$ and $C(4, 3)$. Let $A(a, b)$ be the
  third vertex.

  Also, let $D, E$ and $F$ are the foot of perpendiculars drawn from $A, B$ and $C$ respectively.

  We know that if $m_1$ and $m_2$ are the slopes of two perpendicular lines then $m_1m_2 = -1$

  Since $AD\perp BC\Rightarrow \frac{b - 0}{a - 0}\times \frac{3 - 2}{4 - 0} = -1 \Rightarrow b + 4a = 0$

  Similarly $CF\perp AB\Rightarrow \frac{b - 2}{a - 0}\times\frac{3 - 0}{4 - 0} = -1$

  $\Rightarrow 4a + 3b = 6$

  From these two equations $a = -\frac{3}{4}, b = 3$. Thus, the third vertex lies in second quadrant.
  %249
\item Let $ABC$ be the triangle such that equation of $AB$ is $4x + 5y - 20 = 0$ and equation of $AC$ is $3x
  - 2y + 6 = 0$.

  Let $BE$ and $CF$ be perpenticulars and that they meet at orthocenter $H(1, 1)$.

  Slope of $AB = \frac{3}{2}$. Slope of $BH$ which is perpendicular to $AC$ will be $-\frac{2}{3}$.

  We know that equation of a line pasasing through $(x_1, y_1)$ and slope $m$ is given $y - y_1 = m(x -
  x_1)$.

  Thus, equation of $BH$ would be $y - 1 = -\frac{2}{3}(x - 1)\Rightarrow 2x + 3y - 5 = 0$.

  Now equation of $AB$ is $4x + 5y - 20 = 0$ and $BH$ is $2x + 3y - 5 = 0$.

  Thus, point of intersection is $B\left(\frac{35}{2}, -10\right)$.

  We find the coordinate of $A$ by solving the gives equations to be
  $\left(\frac{10}{23}, \frac{84}{23}\right)$.

  Slope of $AH$ is $\frac{\frac{84}{21} - 1}{\frac{10}{23} - 1} = -\frac{61}{13}$.

  $BC$ is perpendicular to $AH$. Slope of $BC$ is $\frac{13}{61}$ and $BC$ passes through $B$.

  Thus, equation of $BC$ is $26x - 122y - 1675 = 0$.
  %250
\item Let $x = a + b, y = ab$, where $a, b$ are the two sides of the triangle other than $c$.

  Given that $x^2 - c^2 = y \Rightarrow (a + b)^2 - c^2 = ab \Rightarrow a^2 + b^2 - c^2 = -ab$

  $\Rightarrow \frac{a^2 + b^2 - c^2}{2ab} = -\frac{1}{2} = \cos120^\circ$

  $\angle C = \frac{2\pi}{3}$.

  $R = \frac{abc}{4\Delta}, r = \frac{\Delta}{s}\Rightarrow \frac{r}{R} = \frac{4\Delta^2}{s(abc)}$

  $= \frac{4\left[\frac{1}{2}ab\sin\frac{2\pi}{3}\right]^2}{\frac{x + c}{2}.y.c} = \frac{3y}{2c(x + c)}$.
  %251
\item The diagram is given below:

  \startplacefigure
    \externalfigure[17_21.pdf]
  \stopplacefigure

  $R^2 = MC^2 = \frac{1}{4}(a^2 + b^2) = \frac{1}{4}c^2$

  $r = (s - c)\tan\frac{C}{2}= s - c$

  $2(r + R) = 2s - 2c + c = a + b$.
  %252
\item The diagram is given below:

  \startplacefigure
    \externalfigure[17_22.pdf]
  \stopplacefigure

  Given circle is $x^2 + y^2 = 9$ and parabola is $y^2 = 8x$.

  $\Rightarrow x^2 + 8x - 9 = 0 \Rightarrow x = -8, 1$. We reject $x = -8$ because that makes $y$ complex
  number. Thus, $x = 1, y = \pm2\sqrt{2}$.

  Therefore, $P$ is $\left(1, 2\sqrt{2}\right)$ and $Q$ is $\left(1, -2\sqrt{2}\right)$.

  Equations of tangents are $x + 2\sqrt{2}y = 9$ at $P$ and at $Q$ is $x - 2\sqrt{2}y = 9$.

  Thus, $R$ the point of intersection is $(9, 0)$.

  Equation of tangents for parabola are $4(x + 1) - 2\sqrt{2}y = 0$ and $4(x + 1) + 2\sqrt{2}y = 0$.

  Thus, $S$ the point of intersection is $(-1, 0)$.

  Inradius of $\triangle PQR = \frac{\Delta}{s} = \frac{16\sqrt{2}}{3\sqrt{2} + 3\sqrt{2} + 2\sqrt{2}} = 2$.

  Equation of circumcircle of $\triangle PRS$ is $(x + 1)(x - 9) + y^2 + ky = 0$

  It passes through $\left(1, 2\sqrt{2}\right)$, which gives us $k = 2\sqrt{2}$. Hence its radius is
  $3\sqrt{3}$.

  $PQ = 4\sqrt{2}$. Thus, $\Delta PQR = \frac{1}{2}4\sqrt{2}.8 = 16\sqrt{2}$

  $\Delta PQS = \frac{1}{2}4\sqrt{2}.2 = 4\sqrt{2}$

  Thus, $\Delta PQR:\Delta PQS = 4:1$.
  %253
\item The diagram is given below:

  \startplacefigure
    \externalfigure[17_23.pdf]
  \stopplacefigure

  Using sine rule, $\frac{p}{\sin P} = \frac{q}{\sin Q} = \frac{r}{\sin R} = 2R = 2$

  $\Rightarrow \sin P = \frac{\sqrt{3}}{2}$ and $\sin Q = \frac{1}{2}$

  $\Rightarrow P = 120^\circ, Q = 30^\circ\Rightarrow R = 30^\circ$

  So $\triangle PQR$ is an isosceles triangle. Since $RS$ and $PE$ are the medians so $O$ is the centroid of
  the $\triangle PQR$.

  Using Apollonius theorem gives us $2\left(PE^2 + QE^2\right) = PQ^2 + PR^2$

  $\Rightarrow 2\left(PE^2 + \frac{3}{4}\right) = 1 + 1 \Rightarrow PE = \frac{1}{2}$ units.

  $OE = \frac{1}{3}PE = \frac{1}{6}$ units.

  Again from Apollonius theorem $2\left(PS^2 + RS^2\right) = PR^2 + QR^2\Rightarrow 2\left(\frac{1}{4} +
  RS^2\right) = 1 + 3$

  $\Rightarrow RS = \frac{\sqrt{7}}{2}$ units.

  $r = \frac{\Delta}{s} = \frac{\frac{1}{2}pq\sin R}{\frac{1}{2}(p + q + r)} = \frac{3}{2}\left(2
  - \sqrt{3}\right)$ units.
  %254
\item Using cosine rule $\cos30^\circ = \frac{PQ^2 + QR^2 - PR^2}{2PQ.QR} \Rightarrow \frac{\sqrt{3}}{2}
  = \frac{300 + 100 - PR^2}{200\sqrt{3}}$

  $\Rightarrow PR = 10 = QR \Rightarrow \angle QRP = 30^\circ$

  $\Delta PQR = \frac{1}{2}PQ.QR.\sin30^\circ = 25\sqrt{3}$

  Inradius of $\triangle PQR, r = \frac{\Delta}{s} = 10\sqrt{3}- 15$

  Circumradius of $\triangle PQR, R = \frac{abc}{4\Delta} = 10$

  $\therefore $ Area of circumcircle is $100\pi$.
  %255
\item $\frac{s - x}{4} = \frac{s - y}{3} = \frac{s - z}{2} = \frac{3s - (x + y + z)}{4 + 3 + 2}
  = \frac{s}{9} = k$

  $\Delta = \sqrt{s(s - a)(s - b)(s - c)} = 6\sqrt{6}k^2$

  Also, $\pi r^2 = \frac{8\pi}{3}\Rightarrow r^2 = \frac{8}{3}$

  $R = \frac{xyz}{4\Delta} = \frac{35k}{6\sqrt{6}}$

  $r^2 = \frac{8}{3} = \frac{\Delta^2}{s^2} = \frac{8}{3}k^2 \Rightarrow k = 1$

  $\Rightarrow \sin^2\frac{X + Y}{2} = \cos^2\frac{Z}{2} = \frac{s(s - z)}{xy} = \frac{3}{5}$.
  %256
\item Since $P, Q, R, T$ are concyclic, then $PS.ST = QS.SR$

  Using A.M.-G.M.\ inequality, $\frac{PS + ST}{2} > \sqrt{PS.ST}$

  and $\frac{1}{PS} + \frac{1}{ST} > \frac{1}{\sqrt{PS + ST}} = \frac{2}{\sqrt{QS.SR}}$

  Also, $\frac{SR + QS}{2} > \sqrt{SQ.SR}\Rightarrow \frac{2}{\sqrt{SQ.SR}} > \frac{4}{QR}$

  $\Rightarrow \frac{1}{PS} + \frac{1}{ST} > \frac{1}{\sqrt{PS + ST}} > \frac{4}{QR}$.
  %257
\item Given that the three circles with radii $3, 4$ and $5$ touch each other externally and $P$ is the
  point of intersection of tangents of the point of intersections.

  The diagram is given below:

  \startplacefigure
    \externalfigure[17_24.pdf]
  \stopplacefigure

  Thus, $P$ is the incenter of $\triangle C_1C_2C_3$.

  Distance of $P$ from the points of contact $=$ inradius of $\triangle C_1C_2C_3$

  $r = \frac{\Delta}{s} = \sqrt{\frac{(s - a)(s - b)(s - c)}{s}}$, where $2s = 7 + 8 + 9 \Rightarrow s = 8$

  $\Rightarrow r = \sqrt{\frac{(12 - 7)(12 - 8)(12 - 9)}{12}} = \sqrt{5}$.
  %258
\item We know that $I_n = \frac{n}{2}r^2\sin\frac{2\pi}{n}\Rightarrow \frac{2I_n}{n}
  = \sin^2\frac{2\pi}{n}[\because r = 1]$

  and $O_n = nr^2\tan\frac{\pi}{n}\Rightarrow \frac{O_n}{n} = \tan\frac{\pi}{n}$

  $\Rightarrow \frac{2I_n}{O_n} = \frac{\sin\frac{2\pi}{n}}{\tan\frac{\pi}{n}} = \frac{1
    + \cos\frac{2\pi}{n}}{2}$

  $\Rightarrow \frac{I_n}{O_n} = \frac{1 + \sqrt{1 - \left(2I_n/n\right)^2}}{2}$

  $\Rightarrow I_n = \frac{O_n}{2}\left[1 + \sqrt{1 - \left(\frac{2I_n}{n}\right)^2}\right]$.
  %259
\item $\sin C = \frac{\sqrt{3}}{2}$ and $C$ is given to be obstuse. Thus, $C = \frac{2\pi}{3}$

  $c = \sqrt{a^2 + b^2- 2ab\cos C} = 14$

  $r^2 = \frac{\Delta^2}{s^2} = 3$.
  %260
\item $S_n(x) = \displaystyle\sum_{k = 1}^n\tan^{-1}x\left[\frac{(k + 1)x - kx}{1 + kx(k + 1)x}\right]
  = \displaystyle\sum_{k = 1}^n\left[\tan^{-1}(k + 1)x - \tan^{-1}kx\right]$

  $= \tan^{-1}(n + 1)x - \tan^{-1}x = \tan^{-1}\frac{nx}{1 + (n + 1)x^2}$

  $S_{10}x = \tan^{-1}\frac{10x}{1 + 11x^2} = \frac{\pi}{2} - \cot^{-1}\frac{10x}{1 + 11x^2} = \frac{\pi}{2}
  - \tan^{-1}\frac{1 + 11x^2}{10}$.

  Thus, (a) is correct.

  $\displaystyle\lim_{n\to\infty}\cot\left(S_n(x)\right) = \cot\left[\tan^{-1}\frac{x}{x^2}\right]
  = \cot\cot^{-1}x = x, \forall x > 0$

  Thus, (b) is correct.

  $S_3(x) = \frac{\pi}{4}\Rightarrow \frac{3x}{1 + 4x^2} = 1 \Rightarrow 4x^2 - 3x + 1 = 0$, which has no
  real root. Thus, (c) is incorect.

  For $x = 1, \tan\left(S_n(x)\right) = \frac{n}{n + 2}$, which is greater than $\frac{1}{2}$ for $n\ge 3$.

  Thus, (d) is incorrect.
\stopitemize
