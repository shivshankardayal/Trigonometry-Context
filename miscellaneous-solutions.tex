% -*- mode: context; -*-
\chapter{Miscellaneous Problems}
\startitemize[n, 1*broad]
% 1
\item L.H.S.\ $= \frac{\sqrt{1 + \cos x} + \sqrt{1 - \cos x}}{\sqrt{1 + \cos x} - \sqrt{1 - \cos x}}$

  $= \frac{\sqrt{2\cos^2\frac{x}{2}} + \sqrt{2\sin^2\frac{x}{2}}}{\sqrt{2\cos^2\frac{x}{2}}
  - \sqrt{2\sin^2\frac{x}{2}}} = \frac{\left|\cos\frac{x}{x}\right|
  + \left|\sin\frac{x}{x}\right|}{\left|\cos\frac{x}{x}\right| - \left|\sin\frac{x}{x}\right|}$

  $= \frac{-\cos\frac{x}{2} + \sin\frac{x}{2}}{-\cos\frac{x}{2} - \sin\frac{x}{2}}[\because \pi < x <
  2\pi\Rightarrow \frac{\pi}{2} < \frac{x}{2} < \pi]$

  $= \frac{\cot\frac{x}{2} - 1}{\cot\frac{x}{2} + 1} = \cot\left(\frac{x}{2} + \frac{\pi}{4}\right)$.
  %2
\item $2A = A + B + A - B \Rightarrow \tan2A = \frac{\tan(A + B) + \tan(A - B)}{1 - \tan(A + B)\tan(A - B)}$

  From question $0 < A < \frac{\pi}{4}$ anda $0 < B < \frac{\pi}{4} \Rightarrow 0 < A + B < \frac{\pi}{2}$

  Also, $-\frac{\pi}{4} < A - B < \frac{\pi}{4}$, however, $\sin(A - B) = \frac{1}{\sqrt{10}}\Rightarrow 0 <
  A - B < \frac{\pi}{4}$

  $\Rightarrow \tan(A - B) = \frac{1}{3}, \cos(A + B) = \frac{2}{\sqrt{29}}\Rightarrow \tan(A + B)
  = \frac{5}{2}$

  $\Rightarrow \tan2A = \frac{\frac{5}{2} + \frac{1}{3}}{1 - \frac{5}{2}.\frac{1}{3}} = 17$.
  %3
\item Given that $\sin^3x\sin3x = \displaystyle\sum_{m = 0}^nc_m\cos mx \Rightarrow \frac{3\sin x
  - \sin3x}{4}\sin3x = \displaystyle\sum_{m = 0}^nc_m\cos mx$

  $\Rightarrow \frac{3}{8}(2\sin3x.\sin x) - \frac{1}{8}2\sin^23x = \displaystyle\sum_{m = 0}^nc_m\cos
  mx \Rightarrow \frac{3}{8}[\cos2x - \cos4x] - \frac{1}{8}[1 - \cos6x] = \displaystyle\sum_{m = 0}^nc_m\cos
  mx$

  $\Rightarrow -\frac{1}{8} + \frac{3}{8}\cos2x - \frac{3}{8}\cos4x + \frac{1}{8}\cos6x
  = \displaystyle\sum_{m = 0}^nc_m\cos mx$

  Comparing we get $n = 6$.
  %4
\item L.H.S. $= \left[\sin^4\frac{\pi}{8} + \sin^4\frac{7\pi}{8}\right] + \left[\sin^4\frac{3\pi}{8}
  + \sin^4\frac{5\pi}{8}\right]$

  $= \left[\sin^4\frac{\pi}{8} + \sin^4\left(\pi - \frac{\pi}{8}\right)\right] + \left[\sin^4\frac{3\pi}{8}
  + \sin^4\left(\pi - \frac{3\pi}{8}\right)\right]$

  $= 2\sin^4\frac{\pi}{8} + 2\sin^4\frac{3\pi}{8} = 2\left[\left(\sin^2\frac{\pi}{8}\right)^2
  + \left(\sin^2\frac{3\pi}{8}\right)^2\right]$

  $= 2\left[\left(\frac{1 - \cos\frac{\pi}{4}}{2}\right)^2 + \left(\frac{1
    - \cos\frac{3\pi}{4}}{2}\right)^2\right] = \frac{1}{2}\left[\left(1 - \frac{1}{\sqrt{2}}\right)^2
  + \left(1 + \frac{1}{\sqrt{2}}\right)^2\right] = \frac{3}{2} =$ R.H.S.
  %5
\item $\displaystyle\sum_{q = 1}^{10}\left(\sin\frac{2q\pi}{11} - i\cos\frac{2q\pi}{11}\right)$

  $= \sin\frac{2\pi}{11} + \sin\frac{4\pi}{11} + \sin\frac{6\pi}{11} + \cdots$ to $10$ terms
  $- i\cos\frac{2\pi}{11} + \cos\frac{4\pi}{11} + \cos\frac{6\pi}{11} + \cdots$ to $10$ terms

  $= \frac{\sin10\frac{\pi}{11}}{\sin\frac{\pi}{11}}\sin\left[\frac{2\pi}{11} + (10 -
  1)\frac{\pi}{11}\right] - i\frac{\sin10\frac{\pi}{11}}{\sin\frac{\pi}{11}}\cos\left[\frac{2\pi}{11} + (10
  - 1)\frac{\pi}{11}\right]$

  $= \frac{\sin\left(\pi - \frac{\pi}{11}\right)}{\sin\frac{\pi}{11}}.\sin\pi - i\frac{\sin\left(\pi
  - \frac{\pi}{11}\right)}{\sin\frac{\pi}{11}}.\cos\pi = i$

  $\displaystyle\sum_{p = 1}^{32}(3p + 2)\left[\displaystyle\sum_{q = 1}^{10}\sin\frac{2q\pi}{11} -
  i\cos\frac{2q\pi}{11}\right]^p = \sum_{p=1}^{32}(3p + 2)i^p = 5i + 8i^2 + 11i^3 + \cdots$ upto $32$ terms

  $= 5i - 8 - 11i + 14 + 17i - 20 + \cdots$ upto $32$ terms

  $= (5i - 11i + 17i - 23i + \cdots)$ upto $16$ temrs $+ (-8 + 14 - 20 + 26 - \cdots)$ upto $16$ terms

  $= 48(1 - i)$.
  %6
\item Let $x = 2^n\cos\theta\cos2\theta\cos2^2\theta\ldots\cos2^{n - 1}\theta$

  $x\sin\theta = 2^n(\sin\theta\cos\theta)\cos2\theta\cos2^2\theta\ldots\cos2^{n - 1}\theta = 2^{n -
  1}\sin2\theta\cos2\theta\cos2^2\theta\ldots\cos2^{n - 1}\theta$

  Proceeding similarly, $x\sin\theta = \sin2^n\theta$. Given than $\theta = \frac{\pi}{2^n - 1} \Rightarrow
  2^n\theta = \pi + \theta$

  $\Rightarrow x\sin\theta = \sin(\pi + \theta) \Rightarrow x = -1$.
  %7
\item Let $x = \cos\frac{2\pi}{7}\cos\frac{4\pi}{7}\cos\frac{6\pi}{7} = \frac{1}{8}
  = \cos\frac{2\pi}{7}\cos\frac{4\pi}{7}\cos\left(\pi - \frac{\pi}{7}\right)$

  $= -\cos\frac{\pi}{7}\cos\frac{2\pi}{7}\cos\frac{4\pi}{7}$

  Multiplying both sides with $\sin\frac{\pi}{7}$ gives us

  $x\sin\frac{\pi}{7} = -\frac{1}{2}.2\sin\frac{\pi}{7}\cos\frac{\pi}{7}\cos\frac{2\pi}{7}\cos\frac{4\pi}{7}
  = -\frac{1}{2}\sin\frac{2\pi}{7}\cos\frac{2\pi}{7}\cos\frac{4\pi}{7}$

  $= -\frac{1}{4}\sin\frac{4\pi}{7}\cos\frac{4\pi}{7} = -\frac{1}{8}\sin\frac{8\pi}{7} =
  -\frac{1}{8}\sin\left(\pi + \frac{\pi}{7}\right)$

  $x\sin\frac{\pi}{7} = \frac{1}{8}.\sin\frac{\pi}{7}\Rightarrow x = \frac{1}{8}$.
  %8
\item Let $y = \sin\frac{2\pi}{7} + \sin\frac{4\pi}{7} + \sin\frac{8\pi}{7} = \frac{\sqrt{7}}{2}$

  Squaring both sides $y^2 = \sin^2\frac{2\pi}{7} + \sin^2\frac{4\pi}{7} + \sin^2\frac{8\pi}{7} +
  2\sin\frac{2\pi}{7}\sin\frac{4\pi}{7} + 2\sin\frac{4\pi}{7}\sin\frac{8\pi}{7} +
  2\sin\frac{2\pi}{7}\sin\frac{8\pi}{7}$

  Let $y_1 = \sin^2\frac{2\pi}{7} + \sin^2\frac{4\pi}{7} + \sin^2\frac{8\pi}{7}$ and $y_2 =
  2\sin\frac{2\pi}{7}\sin\frac{4\pi}{7} + 2\sin\frac{4\pi}{7}\sin\frac{8\pi}{7} +
  2\sin\frac{2\pi}{7}\sin\frac{8\pi}{7}$

  $\Rightarrow y^2 = y_1 + y_2$. Now

  $y_1 = \frac{1 - \cos\frac{4\pi}{7}}{2} + \frac{1 - \cos\frac{8\pi}{7}}{2} + \frac{1
    - \cos\frac{16\pi}{7}}{2}$

  $= \frac{3}{2} - \frac{1}{2}\left[\cos\frac{2\pi}{7} + \cos\frac{4\pi}{7} + \cos\frac{6\pi}{7}\right]$

  Let $z = \cos\frac{2\pi}{7} + \cos\frac{4\pi}{7} + \cos\frac{6\pi}{7}$

  Multiplying both sides by $2\sin\frac{\pi}{7}$ gives is

  $2z\sin\frac{\pi}{7} = 2\sin\frac{\pi}{7}\left(\cos\frac{2\pi}{7} + \cos\frac{4\pi}{7}
  + \cos\frac{6\pi}{7}\right)$

  $= \sin\frac{3\pi}{7} - \sin\frac{\pi}{7} + \sin\frac{5\pi}{7} - \sin\frac{3\pi}{7} + \sin\frac{7\pi}{7} -
  \sin\frac{5\pi}{7}$

  $2z\sin\frac{\pi}{7} = \sin\pi - \sin\frac{\pi}{7} \Rightarrow z = -\frac{1}{2}\Rightarrow y_1
  = \frac{7}{4}$

  Now $y_2 = 2\sin\frac{2\pi}{7}\sin\frac{4\pi}{7} + 2\sin\frac{4\pi}{7}\sin\frac{8\pi}{7} +
  2\sin\frac{2\pi}{7}\sin\frac{8\pi}{7}$

  $= \cos\frac{2\pi}{7} - \cos\frac{6\pi}{7} + \cos\frac{4\pi}{7} - \cos\frac{12\pi}{7} + \cos\frac{6\pi}{7}
  - \cos\frac{10\pi}{7} = 0$.

  Hence proven.
  %9
\item We can rewrite given equation as L.H.S.\ $= \tan^2\frac{\pi}{16} + \tan^2\frac{7\pi}{16}
  + \tan^2\frac{2\pi}{16} + \tan^2\frac{6\pi}{16} + \tan^2\frac{3\pi}{16} + \tan^2\frac{5\pi}{16}
  + \tan^2\frac{4\pi}{16}$

  $= \left(\tan^2\frac{\pi}{16} + \cot^2\frac{\pi}{16}\right) + \left(\tan^2\frac{2\pi}{16}
  + \cot^2\frac{2\pi}{16}\right) + \left(\tan^2\frac{3\pi}{16} + \cot^2\frac{3\pi}{16}\right) +
  1[\because \tan\frac{7\pi}{16} = \tan\left(\frac{\pi}{2} - \frac{\pi}{16}\right) = \cot\frac{\pi}{16}]$

  $= \left(\tan\frac{\pi}{16} + \cot\frac{\pi}{16}\right)^2 + \left(\tan\frac{2\pi}{16}
  + \cot\frac{2\pi}{16}\right)^2 + \left(\tan\frac{3\pi}{16} + \cot\frac{3\pi}{16}\right)^2 - 2 - 2 - 2 +
  1[\because (\tan\theta + \cot\theta)^2 = \tan^2\theta + \cot^2\theta - 2\tan\theta\cot\theta]$

  $= \left(\frac{1}{\sin\frac{\pi}{16}\cos\frac{\pi}{16}}\right)^2
  + \left(\frac{1}{\sin\frac{2\pi}{16}\cos\frac{2\pi}{16}}\right)^2
  + \left(\frac{1}{\sin\frac{3\pi}{16}\cos\frac{3\pi}{16}}\right)^2 - 5$

  $= \frac{4}{\sin^2\frac{\pi}{8}} + \frac{4}{\sin^2\frac{\pi}{4}} + \frac{4}{\sin^2\frac{3\pi}{8}} - 5$

  $= \frac{4}{\sin^2\frac{\pi}{8}} + \frac{4}{\sin^2\frac{3\pi}{8}} + 4.2 - 5$

  $= \frac{4}{\sin^2\frac{\pi}{8}} + \frac{4}{\cos^2\frac{\pi}{8}} + 3
  = \frac{4}{\sin^2\frac{\pi}{8}\cos^2\frac{\pi}{8}} + 3 = \frac{16}{\sin^2\frac{\pi}{4}} + 3 = 35$.
  %10
\item Let $\theta = \frac{\pi}{7} \Rightarrow 7\theta = \pi \Rightarrow 4\theta + 3\theta = \pi$

  $\Rightarrow \tan4\theta = \tan(\pi - 3\theta) \Rightarrow \frac{4\tan\theta - 4\tan^3\theta}{1 -
  6\tan^2\theta + \tan^4\theta} = -\frac{3\tan\theta - \tan^3\theta}{1 - 3\tan^2\theta}$

  Let $\tan\theta = z\Rightarrow \frac{4z - 4z^3}{1 - 6z^2 + z^4} = -\frac{\3z - z^3}{1 - 3z^2}$

  $\Rightarrow z^6 - 21z^4 + 35z^2 - 7 = 0$

  This is a cubic equation in $z^2$ i.e. $\tan^2\theta$, therefore, the roots of the equation are
  $\tan^2\frac{\pi}{7}, \tan^2\frac{2\pi}{7}$ and $\tan^2\frac{3\pi}{7}$.

  From Vieta's relattionships the product of roots gives us
  $\tan\frac{\pi}{7}\tan\frac{2\pi}{7}\tan\frac{3\pi}{7} = \sqrt{7}$.
  %11
\item We have obtained the equation $z^6 - 21z^4 + 35z^2 - 7 = 0$ whose roots are
  $\tan^2\frac{\pi}{7}, \tan^2\frac{2\pi}{7}$ and $\tan^2\frac{3\pi}{7}$.

  Thus, sum of roots is $35$. Putting $z = \frac{1}{y}$ gives us $-7y^6 + 35y^4 - 21y^2 + 1 = 0$ whose roots
  would be $\cot^2\frac{\pi}{7}, \cot^2\frac{2\pi}{7}$ and $\cot^2\frac{3\pi}{7}$.

  Sum of these is $\frac{-35}{-7} = 3$.

  Thus, $\left(\tan^2\frac{\pi}{7} + \tan^2\frac{2\pi}{7} + \tan^2\frac{3\pi}{7}\right)
  + \left(\cot^2\frac{\pi}{7} + \cot^2\frac{2\pi}{7} + \cot^2\frac{3\pi}{7}\right) = 105$.
  %12
\item L.H.S.\ $= \sqrt{\tan x + \sin x} + \sqrt{\tan x - \sin x} = \sqrt{\tan x}(\sqrt{1 + \cos x} + \sqrt{1
  - \cos x})$

  $= \sqrt{2\tan x}\left(\cos\frac{x}{2} + \sin\frac{x}{2}\right) = 2\sqrt{\tan
  x}\left(\frac{1}{\sqrt{2}}\cos x + \frac{1}{\sqrt{2}}\sin\frac{x}{2}\right)$

  $= 2\sqrt{\tan x}\cos\left(\frac{\pi}{4} - \frac{x}{2}\right)[\because 0 < x < \frac{\pi}{2}]$.
  %13
\item We know that $\sin3x = 3\sin x - 4\sin^3x$, which makes L.H.S.\ $= \sin^3x\left(3\sin x -
  4\sin^3x\right)$

  $= 3\sin^4x - 4\sin^6x = 3\left(1 - \cos^2x\right)^2 - \left(1 - \sin^2x\right)^3$

  Thus, coefficient of $\cos^4x$  or $c_4$ is $-9$.
  %14
\item We have to find the value of $\sin7^\circ + \sin77^\circ + \sin293^\circ + \sin149^\circ
  + \sin221^\circ$

  $= \sin5^\circ + 2\sin185^\circ\cos108^\circ + 2\sin185^\circ\cos36^\circ$

  $= \sin5^\circ - 2\sin5^\circ\left(\cos108^\circ + \cos36^\circ\right) = \sin5^\circ -
  4\sin5^\circ\left(\cos72^\circ\cos36^\circ\right)$

  $= 2\sin5^\circ - 4\sin5^\circ\left[\frac{\sqrt{5} - 1}{4}.\frac{\sqrt{5 + 1}}{4}\right] = 0$.
  %15
\item Consider the complex sum $Z = \displaystyle\sum_{k = 1}^{n - 1}e^{ik\pi/n}$, then real part of this
  sum is given series, which we let as $S$.

  $Z = \frac{e^{i\pi/n} - e^{i\pi}}{1 - e^{i\pi/n}} = \frac{e^{i\pi/n} + 1}{1 - e^{i\pi/n}} =
  i\cot\frac{\pi}{2n}$, which is purely imaginary number. Thus, $S = 0$.
  %16
\item Let $\omega = e^{i2\pi/7} = \cos\frac{2\pi}{7} + i\sin\frac{2\pi}{7} \Rightarrow \omega^7 = 1$

  $\Rightarrow (\omega - 1)\left(\omega^6 + \omega^5 + \omega^4 + \omega^3 + \omega^2 + \omega + 1\right) =
  0$

  $\because \omega\neq 1 \Rightarrow \omega^6 + \omega^5 + \omega^4 + \omega^3 + \omega^2 + \omega + 1 = 0$

  $S = \cos\frac{2\pi}{7} + \cos\frac{4\pi}{7} + \cos\frac{6\pi}{7} = \frac{\omega + \omega^{-1}}{2}
  + \frac{\omega^2 + \omega^{-2}}{2} + \frac{\omega^3 + \omega^{-3}}{2}$

  $= \frac{1}{2}\left(\omega^6 + \omega^5 + \omega^4 + \omega^3 + \omega^2 + \omega\right) = -\frac{1}{2}$.
  %17
\item We see that $\sin\left(\frac{3\pi}{2} - \alpha\right) = -\cos\alpha, \sin(3\pi + \alpha) = \sin(\pi
  + \alpha) = -\sin\alpha,$

  $\sin\left(\frac{\pi}{2} + \alpha\right) = \cos\alpha,$ and $\sin(5\pi - \alpha) = \sin(\pi - \alpha)
  = \sin\alpha$

  Thus, L.H.S. $= 3\left[\cos^4\alpha + \sin^4\alpha\right] - 2\left[\cos^6\alpha + \sin^6\alpha\right]$

  $ 3\left(1 - \sin^2\alpha\cos^2\alpha\right) - 2\left(1 - 3\cos^2\alpha\sin^2\alpha\right) = 1$.
  %18
\item $\sin36^\circ\sin72^\circ\sin108^\circ\sin144^\circ = \sin36^\circ\sin72^\circ\sin72^\circ\sin36^\circ
  = \left(\sin36^\circ\sin72^\circ\right)^2$

  $= \frac{1}{4}\left(\cos(36^\circ - 72^\circ) - \cos(36^\circ + 72^\circ)\right)^2
  = \frac{1}{4}\left(\cos36^\circ + \sin18^\circ\right)^2$

  $= \frac{1}{4}\left(\frac{\sqrt{5 + 1}}{4} + \frac{\sqrt{5} - 1}{4}\right)^2 = \frac{5}{16}$.
  %19
\item We have to prove that $\sin^212^\circ + \sin^221^\circ + \sin^239^\circ + \sin^248^\circ = 1
  + \sin^29^\circ + \sin^218^\circ$

  We know that $\sin^2\theta = \frac{1 - \cos2\theta}{2}$, which transforma above to

  $2 - \cos24^\circ - \cos42^\circ - \cos78^\circ - \cos96^\circ = 2 - \cos18^\circ - \cos36^\circ$

  Now, $\cos24^\circ + \cos42^\circ + \cos78^\circ + \cos96^\circ = 2\cos60^\circ\cos36^\circ +
  2\cos60^\circ\cos18^\circ$

  $= \cos36^\circ + \cos18^\circ$. Hence proven.
  %20
\item In the given range, the cosine function is non-negative. For $135^\circ\leq \alpha/2\leq 180^\circ$
  sine function is positive and negative in $180^\circ\leq \alpha/2 \leq 225^\circ$.

  Thus, $\sin\frac{\alpha}{2} = \frac{1}{2}\left[\sqrt{1 - \sin\alpha} - \sqrt{1 + \sin\alpha}\right]$

  $\cos\frac{\alpha}{2} = -\frac{1}{2}\left[\sqrt{1 + \sin\alpha} + \sqrt{1 - \sin\alpha}\right]$.
  %21
\item We know that $\tan142^\circ30' = \frac{1 - \cos285^\circ}{\sin285^\circ}$

  $\sin285^\circ = -\sin75^\circ = \sin\left(45^\circ + 30^\circ\right) = -\frac{\sqrt{6} + \sqrt{2}}{4}$

  $\cos285^\circ = \cos75^\circ = \cos\left(45^\circ + 30^\circ\right) = \frac{\sqrt{6} - \sqrt{2}}{4}$

  $\Rightarrow \tan142^\circ30' = \frac{1 - \frac{\sqrt{6} - \sqrt{2}}{4}}{-\frac{\sqrt{6} + \sqrt{2}}{4}}$

  $= \frac{\sqrt{6} - \sqrt{2} - \sqrt{4}}{\sqrt{6} + \sqrt{2}}$

  Rationalizing gives us $\frac{\sqrt{6} - \sqrt{2} - \sqrt{4}}{\sqrt{6} + \sqrt{2}}.\frac{\sqrt{6}
    - \sqrt{2}}{\sqrt{6} - \sqrt{2}}$

  On simplification we get the desired result.
  %22
\item Let $\tan\frac{\alpha}{2} = t$, then $\frac{2t}{1 + t^2} + \frac{1 - t^2}{1 + t^2}
  = \frac{\sqrt{7}}{2}$

  $\Rightarrow \left(\sqrt{7} + 2\right)t^2 - 4t + \left(\sqrt{7} - 2\right) = 0$

  $\Rightarrow D = \sqrt{16 - 4.3} = \pm2$

  $t = \frac{2\pm 1}{\sqrt{7} + 2}$.
  %23
\item Given that $\frac{\sin(x - \alpha)}{\sin(x - \beta)} = \frac{a}{b} \Rightarrow b\sin(x - \alpha) =
  a\sin(x - \beta)$

  $\Rightarrow b(\sin x\cos\alpha - \cos x\sin\alpha) = a(\sin x\cos\alpha - \cos x\sin\alpha)$

  $\Rightarrow \tan x = \frac{b\sin\alpha - a\sin\beta}{b\cos\alpha - a\cos\beta}$

  Similarly from $\frac{\cos(x - \alpha)}{\cos(x - \beta)} = \frac{A}{B}$ we get

  $\tan x = \frac{A\cos\beta - B\cos\alpha}{B\sin\alpha - A\sin\beta}$

  $\Rightarrow \frac{b\sin\alpha - a\sin\beta}{b\cos\alpha - a\cos\beta} = \frac{A\cos\beta -
    B\cos\alpha}{B\sin\alpha - A\sin\beta}$

  Cross-multiplying and simplifying gives us desired relation.
  %24
\item Since $\cot\alpha, \cot\beta, \cot\gamma$ are in A.P. $\Rightarrow 2\cot\beta = \cot\alpha
  + \cot\beta$

  $\cot(\alpha + \beta + \gamma) = \frac{\pi}{2}\Rightarrow \frac{\cot\alpha\cot\beta\cot\gamma - \cot\alpha
    - \cot\beta - \cot\gamma}{\cot\alpha\cot\beta + \cot\beta\cot\gamma + \cot\gamma\cot\alpha - 1}  = 0$

  $\Rightarrow \cot\alpha\cot\beta\cot\gamma = \cot\alpha + \cot\beta + \cot\gamma = 3\cot\beta$

  $\Rightarrow \cot\beta(\cot\alpha\cot\gamma - 3) = 0$

  Since $\alpha + \beta + \gamma = \frac{\pi}{2}$ all three cannot be $\frac{\pi}{2}$, thus,
  $\cot\alpha\cot\gamma = 3$.
  %25
\item We can write given expression as
  $8.\frac{2\sin\frac{2\pi}{15}}{\sin\frac{2\pi}{15}}\cos\frac{2\pi}{15}\cos\frac{4\pi}{15}\cos\frac{8\pi}{15}\cos\frac{16\pi}{15}$

  $= \frac{4}{\sin\frac{2\pi}{15}}.2\sin\frac{4\pi}{15}\cos\frac{4\pi}{15}\cos\frac{8\pi}{15}\cos\frac{16\pi}{15}$

  and proceeding similarly we arrive at $\frac{\sin\frac{32\pi}{15}}{\sin\frac{2\pi}{15}}
  = \frac{\sin\left(2\pi + \frac{2\pi}{15}\right)}{\sin\frac{2\pi}{15}} = 1$.
  %26
\item Since $\alpha$ and $\beta$ lie between $0$ and $\frac{\pi}{4}$, therefore, $\alpha + \beta$ lie
  betwewen $0$ and $\frac{\pi}{2}$.

  $\Rightarrow \tan(\alpha + \beta) = \frac{3}{4}$ and $\tan(\alpha - \beta) = \frac{5}{12}$

  $\tan2\alpha = \tan(\alpha + \beta + \alpha - \beta) = \frac{\tan(\alpha + \beta) + \tan(\alpha
    - \beta)}{1 - \tan(\alpha + \beta)\tan(\alpha - \beta)}$

  $= \frac{56}{33}$.
  %27
\item We can write $\tan(\alpha + \beta - \alpha) = \frac{\tan(\alpha + \beta) - \tan\alpha}{1
  + \tan\alpha\tan(\alpha + \beta)}$

  $\Rightarrow \tan\alpha\tan(\alpha + \beta) = \frac{\tan(\alpha + \beta) - \tan\alpha
  - \tan\beta}{\tan\beta}$

  Similarly, $\tan(\alpha+ \beta)\tan(\alpha+ + 2\beta) = \frac{\tan(\alpha + 2\beta) - \tan(\alpha + \beta)
    - \tan\beta}{\tan\beta}$ and so on.

  Adding we get $\tan\alpha\tan(\alpha + \beta) + \tan(\alpha + \beta)\tan(\alpha + 2\beta)
  + \tan(\alpha + 2\beta)\tan(\alpha + 3\beta) + \cdots = \frac{\tan(\alpha + n\beta) - \tan\alpha -
    n\tan\beta}{\tan\beta}$.
  %28
\item We have to prove that $-\cot\alpha + \tan\alpha + 2\tan2\alpha + 4\tan4\alpha + 8\tan8\alpha + \cdots
  + 2^n\cot2^n\alpha = 0$

  We see that $-\cot\alpha + \tan\alpha = -\frac{\sin\alpha}{\cos\alpha} + \frac{\sin\alpha}{\cos\alpha}
  = -\frac{2(\cos^2\alpha - \sin^2\alpha)}{2\sin\alpha\cos\alpha}$

  $= -2\cot\alpha$

  Similarly $-2\cot2\alpha + 2\tan\alpha = -2^2\cot^2\alpha$

  Proceeding similarly we obtain $-2^n\cot2^n\alpha + 2^n\cot2^n\alpha = 0$.
  %29
\item R.H.S.\ $= \frac{1}{2}[\tan27x - \tan x] = \frac{1}{2}[(\tan27x - \tan9x) + (\tan9x - \tan3x) +
  (\tan3x - \tan x)]$

  $= \frac{1}{2}\left[\left(\frac{\sin27x}{\cos27x} - \frac{\sin9x}{\cos9x}\right)
  + \left(\frac{\sin9x}{\cos9x} - \frac{\sin3x}{\cos3x}\right) + \left(\frac{\sin3x}{\cos3x} - \frac{\sin
    x}{\cos x}\right)\right]$

  $= \frac{1}{2}\left[\frac{\sin(27 - 9)x}{\cos27x\cos9x} + \frac{\sin(9 - 3)x}{\cos9x\cos3x} + \frac{\sin(3
    - 1)x}{\cos3x\cos x}\right]$

  $= \frac{1}{2}\left[\frac{2\sin9x\cos9x}{\cos27x\cos9x} + \frac{2\sin3x\cos3x}{\cos9x\cos3x} + \frac{2\sin
    x\cos x}{\cos3x\cos x}\right]$

  $= \frac{\sin x}{\cos3x} + \frac{\sin3x}{\cos9x} + \frac{\sin9x}{\cos27x} =$ L.H.S.
  %30
\item Rewriting, we have to prove that $\cot16^\circ.\cot44^\circ - 1 + \cot44^\circ.\cot76^\circ - 1
  = \cot76^\circ.\cot16^\circ + 1$

  $\Rightarrow \left(\frac{\cos16^\circ.\cos44^\circ}{\sin16^\circ.\sin44^\circ} - 1\right)
  + \left(\frac{\cos44^\circ.\cos76^\circ}{\sin44^\circ.\sin76^\circ} - 1\right)
  = \left(\frac{\cos76^\circ\cos16^\circ}{\sin76^\circ\sin16^\circ} + 1\right)$

  $\Rightarrow \frac{\cos60^\circ}{\sin16^\circ\sin44^\circ}
  + \frac{\cos120^\circ}{\sin44^\circ\sin76^\circ} = \frac{\cos60^\circ}{\sin76^\circ\sin16^\circ}$

  $\Rightarrow \frac{1}{2\sin16^\circ\sin44^\circ} - \frac{1}{2\sin44^\circ\sin76^\circ}
  = \frac{1}{2\sin76^\circ\sin16^\circ}$

  $\Rightarrow \frac{\sin76^\circ - \sin16^\circ}{\sin16^\circ\sin44^\circ\sin76^\circ}
  = \frac{1}{\sin76^\circ\sin16^\circ}$

  $\Rightarrow \sin76^\circ - \sin16^\circ = \sin44^\circ \Rightarrow 2\cos46^\circ\sin30^\circ
  = \sin44^\circ \Rightarrow \sin44^\circ = \sin44^\circ$. Hence proved.
  %31
\item Rewriting, we have to prove that $\tan\theta\tan2\theta + 1 + \tan2\theta\tan4\theta + 1
  + \tan4\theta\tan\theta + 1 = -4$

  $\Rightarrow \left(\frac{\sin\theta\sin2\theta}{\cos\theta\cos2\theta} + 1\right)
  + \left(\frac{\sin2\theta\sin4\theta}{\cos2\theta\cos4\theta} + 1\right)
  + \left(\frac{\sin\theta\sin4\theta}{\cos\theta\cos4\theta} + 1\right) = -4$

  $\Rightarrow \frac{\cos\theta}{\cos\theta\cos2\theta} + \frac{\cos2\theta}{\cos2\theta\cos4\theta}
  + \frac{\cos3\theta}{\cos\theta\cos4\theta} = -4$

  $\Rightarrow \frac{1}{\cos2\theta} + \frac{1}{\cos4\theta} + \frac{1}{\cos\theta} =
  -4\left[\because \cos(4\theta) = \cos(2\pi - 4\theta)\text{ as }\theta = \frac{2\pi}{7}\right]$

  $\Rightarrow 2\cos\theta\cos4\theta + 2\cos2\theta\cos\theta + 2\cos4\theta\cos2\theta =
  -8\cos\theta\cos2\theta\cos4\theta$

  $\Rightarrow \cos5\theta + \cos3\theta + \cos3\theta + \cos\theta + \cos2\theta + \cos6\theta =
  -8\cos\theta\cos2\theta\cos4\theta$

  $\Rightarrow \cos\theta + \cos2\theta + \cos3\theta + \cos4\theta + \cos5\theta + \cos6\theta =
  -8\cos\theta\cos2\theta\cos4\theta$

  $\Rightarrow 2\cos\theta + 2\cos2\theta + 2\cos3\theta = -8\cos\theta\cos2\theta\cos4\theta$

  Now L.H.S.\ $= \cos\frac{2\pi}{7} + \cos\frac{4\pi}{7} + \cos\frac{6\pi}{7} = -\frac{1}{2}$

  Also, $\cos\frac{2\pi}{7}.\cos\frac{4\pi}{7}.\cos\frac{6\pi}{7} = \frac{1}{8}$.

  Hence proved.
\item Given that $\frac{\sin^4\alpha}{a} + \frac{\cos^4\alpha}{b} = \frac{1}{a + b}$

  $\Rightarrow b(a + b)\sin^4\alpha + a(a + b)\left(1 - \sin^2\alpha\right)^2 = ab$

  $\Rightarrow (a + b)^2\sin^4\alpha + 2a(a + b)\sin^2\alpha.a + a^2 = 0$

  $\Rightarrow \left[(a + b)\sin^2\alpha - a\right]^2 = 0 \Rightarrow \sin^2\alpha = \frac{a}{a +
  b}\Rightarrow \cos^2\alpha = \frac{b}{a + b}$

  Thus, $\frac{\sin^8\alpha}{a^3} + \frac{\cos^8\alpha}{b^3} = \frac{1}{(a + b)^3}$.
  %33
\item $S = \sin3A + \sin3B + \sin3C = 2\sin\frac{3(A + B)}{2}\cos\frac{3(A - B)}{2} +
  2\sin\frac{3C}{2}\cos\frac{3C}{2}$

  $= 2\sin\frac{3(\pi - C)}{2}\cos\frac{3(A - B)}{2} + 2\sin\frac{3C}{2}\cos\frac{3C}{2}$

  $= -2\cos\frac{3C}{2}\cos\frac{3(A - B)}{2} + 2\sin\frac{3C}{2}\cos\frac{3C}{2} =
  -2\cos\frac{3C}{2} \left[\cos\frac{3(A - B)}{2} - \sin\left()\frac{3\pi}{2} - \frac{3(A +
      B)}{2}\right)\right]$

  $= -2\cos\frac{3C}{2}\left[\cos\frac{3A - 3B}{2} + \cos\frac{3A + 3B}{2}\right] =
  -2\cos\frac{3C}{2}.2\cos\frac{3A}{2}.\cos\frac{3B}{2}$

  $= -4\cos\frac{3A}{2}.\cos\frac{3B}{2}.\cos\frac{3C}{2}$

  $S = 0 \Rightarrow -4\cos\frac{3A}{2}.\cos\frac{3B}{2}.\cos\frac{3C}{2} = 0$

  So one of the terms is zero. Let $\cos\frac{3A}{2} = 0 \Rightarrow \frac{3A}{2} = \frac{\pi}{2}\Rightarrow
  A = \frac{\pi}{3}$.

  Thus, at least one of the angle is $60^\circ$.
  %34
\item Rewriting, we have $\sin y[\sin x\sin(x - y) + \sin z\sin(y - z)] + \sin(z - x)[\sin z\sin x + \sin(x
  - y)\sin(y - z)]$

  $= \frac{1}{2}\sin y[2\sin x\sin(x - y) + 2\sin z\sin(y - z)] + \frac{1}{2}\sin(z - x)[2\sin z\sin x + 2\sin(x
  - y)\sin(y - z)]$

  $= \frac{1}{2}\sin y[\cos y - \cos(2x - y) + \cos(2z - y) - \cos y] + \frac{1}{2}\sin(z - x)[\cos(z - x)
  - \cos(z + x) + (\cos x + z - 2y) - \cos(x - z)]$

  $= \frac{1}{2}\sin y[\cos(2z - y) - \cos(2x - y)] + \frac{1}{2}\sin(z - x)[\cos(x + z - 2y) - \cos(z + x)]$

  $= \frac{1}{2}\sin y[2\sin(z + x - y).\sin(x - z)] + \frac{1}{2}\sin(z - x)[2\sin(z + x - y).\sin y] = 0
  =$ R.H.S.
  %35
\item $\cot3\theta = \cot(2\theta + \theta) = \frac{\cot3\theta + \cot\theta - 1}{\cot2\theta + \cot\theta}$

  $\Rightarrow \cot3\theta\cot2\theta + \cot3\theta\cot\theta = \cot2\theta\cot\theta - 1$

  $\Rightarrow \cot3\theta\cot2\theta - \cot2\theta\cot\theta + 1 = -\cot3\theta\cot\theta$

  Also, $\cot2\theta = \frac{\cot^2\theta - 1}{2\cot\theta}\Rightarrow 2\cot\theta\cot2\theta + 1
  = \cot^2\theta$

  Adding the two equations obtained we get the desired equation.
  %36
\item $\frac{1 + \sin A}{\cos A} = \tan A + \sec A$ and $\frac{\cos B}{1 - \sin B} = \frac{\cos B(1 + \sin
  B)}{1 - \sin^2B} = \tan B + \sec B$

  Now $\sin A - \sin B = 2\cos\frac{A + B}{2}\sin\frac{A - B}{2}, \sin(A - B) = 2\sin\frac{A -
    B}{2}\cos\frac{A - B}{2}$ and $\cos A - \cos B = 2\sin \frac{A + B}{2}\sin\frac{A - B}{2}$

  Then denominator of R.H.S.\ becomes $2\sin\frac{A - B}{2}\left(\cos\frac{A - B}{2} - \sin\frac{A +
    B}{2}\right)$

  Thus, R.H.S. is $\frac{2\cos\frac{A + B}{2}}{\cos\frac{A - B}{2} - \sin\frac{A + B}{2}}$

  Rationalizing $\frac{2\cos\frac{A + B}{2}}{\cos\frac{A - B}{2} - \sin\frac{A + B}{2}}.\frac{\cos\frac{A -
      B}{2} + \sin\frac{A + B}{2}}{\cos\frac{A - B}{2} - \sin\frac{A + B}{2}}$

  Now we use $\cos^2x - \sin^2y = \frac{1 + \cos2x}{2} - \frac{1 - \cos 2y}{2}$, which makes denominator

  $\frac{\cos(A - B) + \cos(A + B)}{2} = \cos A\cos B$

  Similarly expanding numerator gives us $\cos A + \cos B + \sin A + \sin B$ and R.H.S.\ becomes $\tan A
  + \sec A + \tan B + \sec B$.

  Hence, L.H.S. = R.H.S.
  %37
\item Let $u = \cos^2x$ and $v = \cos^2y$ then the given equation becomes

  $\frac{u^2}{v} + \frac{(1 - u)^2}{1 - v} = 1\Rightarrow \frac{u^2(1 - v) + v(1 - u)^2}{v(1 - v)} = 1$

  $\Rightarrow u^2 - u^2v + v\left(1 - 2u + u^2\right) = v - v^2$

  $\Rightarrow u^2 - 2uv + v^2 = 0 \Rightarrow (u - v)^2 = 0 \Rightarrow u = v$

  Thus, $\cos^2x = \cos^2y$ and $\sin^2x = \sin^2y$, so they are interchangable, and, hence proved.
  %38
\item Given that $\theta + \phi + \psi = 2\pi\Rightarrow \cos\psi = \cos(2\pi - \theta - \phi) = \cos(\theta
  + \phi)$

  Now, L.H.S. $= \cos^2\theta + \cos^2\phi + \cos^2(\theta + \phi) - 2\cos\theta\cos\phi\cos(\theta + \phi)$

  $= \cos^2\theta + \cos^2\phi + (\cos\theta\cos\phi - \sin\theta\sin\phi)^2 -
  2\cos\theta\cos\phi(\cos\theta\cos\phi - \sin\theta\sin\phi)$

  $= \cos^2\theta + \cos^2\phi + \sin^2\theta\sin^2\phi - \cos^2\theta\cos^2\phi$

  $= \cos^2\theta + \cos^2\phi + (1 - \cos^2\theta)(1 - \cos^2\phi) - \cos^2\theta\cos^2\phi = 1$.
  %39
\item $\cos3A + \cos3B + \cos3[\pi - (A + B)] = 1 \Rightarrow \cos3A + \cos3B - \cos3(A + B) = 1$

  $\Rightarrow (1 - \cos3A)(1 - \cos3B) = \sin3A\sin3B$

  $\Rightarrow 2\sin^2\frac{3A}{2}.2\sin^2\frac{3B}{2} =
  2\sin\frac{3A}{2}\cos\frac{3A}{2}.2\sin\frac{3B}{2}\cos\frac{3B}{2}$

  Assuming $\sin\frac{3\pi}{2}\neq 0$ and $\sin\frac{3B}{2} \neq 0$ gives us

  $\cos\frac{3A}{2}\cos\frac{3B}{2}
  = \sin\frac{3A}{2}\sin\frac{3B}{2} \Rightarrow \tan\frac{3A}{2}.\tan\frac{3B}{2} = 1$

  Thus, $\frac{3A}{2} + \frac{3B}{2} = \frac{\pi}{2} + n\pi$

  It is given that $A + B + C = \pi$, so for the case of a triangle $n = 0$.

  $\Rightarrow A + B = \frac{\pi}{3}\Rightarrow C = \frac{2\pi}{3}$. Hence proved.
  %40
\item We know that for a triangle $\sin A = \frac{a}{2R}, \sin B = \frac{b}{2R}, \sin C = \frac{c}{2R}$

  Thus, $S = \sum \sin^3A\sin(B - C) = \frac{a^3}{8R^3}\sin(B - C) + \frac{b^3}{8R^3}\sin(C - A)
  + \frac{c^3}{8R^3}\sin(A - B)$

  Expanding the first term, we have $a^3\left(\sin B\cos C - \cos B\sin C\right)$

  Now we will substitute $\cos B = \frac{a^2 + c^2 - b^2}{2ac}$ and also for $\cos C$

  This makes the first term $a^3\left(b.\frac{a^2 + b^2 - 2ab}{2ab}\right) - b^3\left(c.\frac{a^2 + c^2 -
    b^2}{2ac}\right)$.

  Clearly the cyclic sum is zero.
  %41
\item L.H.S.\ $= \frac{\sin^3\theta + \cos^3\theta - (\sin^5\theta + \cos^5\theta)}{\sin\theta
  + \cos\theta}$

  Numerator is $\sin^3\theta(1 - \sin^2\theta) + \cos^3\theta(1 - \cos^2\theta) = \sin^3\theta\cos^2\theta
  + \cos^3\theta\sin^2\theta$

  $= \sin^2\theta\cos^2\theta(\sin\theta + \cos\theta)$

  $\Rightarrow $ L.H.S.\ $= \sin^2\theta\cos^2\theta$

  R.H.S.\ $= \frac{\sin^5\theta + \cos^5\theta - (\sin^7\theta + \cos^7\theta)}{\sin^3\theta
  + \cos^3\theta}$

  Numerator is $\sin^5\theta(1 - \sin^2\theta) + \cos^5\theta(1 - \cos^2\theta) = \sin^5\theta\cos^2\theta
  + \cos^5\theta\sin^2\theta$

  $= \sin^2\theta\cos^2\theta(\sin^3\theta + \cos^3\theta)$

  $\Rightarrow $ R.H.S.\ $= \sin^2\theta\cos^2\theta =$ L.H.S.
  %42
\item  Let $a = x - y$, $b = y - z$, and $c = z - x$. Then $a + b + c = 0$ and the given condition becomes
  $4 \cos a \cos b \cos c = 1$, hence $\cos a \cos b \cos c = \frac14$.

  Using the identity $\cos 3t = 4 \cos^3 t - 3 \cos t$, we have
  $\cos 3a \cos 3b \cos 3c
  = (4 \cos^3 a - 3 \cos a)(4 \cos^3 b - 3 \cos b)(4 \cos^3 c - 3 \cos c)$.

  Factoring out $\cos a \cos b \cos c$ gives
  $\cos 3a \cos 3b \cos 3c
  = \cos a \cos b \cos c (4 \cos^2 a - 3)(4 \cos^2 b - 3)(4 \cos^2 c - 3)$.

  Multiplying both sides by $4$ and using $\cos a \cos b \cos c = \frac14$, we obtain
  $4 \cos 3a \cos 3b \cos 3c
  = (4 \cos^2 a - 3)(4 \cos^2 b - 3)(4 \cos^2 c - 3)$.

  Since $\cos 2a = 2 \cos^2 a - 1$, we have $4 \cos^2 a - 3 = 2 \cos 2a - 1$, and similarly for $b$ and $c$.
  Thus
  $4 \cos 3a \cos 3b \cos 3c
  = (2 \cos 2a - 1)(2 \cos 2b - 1)(2 \cos 2c - 1)$.

  Expanding, this equals
  $1 - 2(\cos 2a + \cos 2b + \cos 2c)
  + 4(\cos 2a \cos 2b + \cos 2b \cos 2c + \cos 2c \cos 2a)
  - 8 \cos 2a \cos 2b \cos 2c$.

  Because $a + b + c = 0$, we use the identities
  $\cos 2a + \cos 2b + \cos 2c = 2$
  and
  $\cos 2a \cos 2b + \cos 2b \cos 2c + \cos 2c \cos 2a = 1$.

  Substituting these values yields
  $4 \cos 3a \cos 3b \cos 3c
  = 1 + 12 \cos 2a \cos 2b \cos 2c$.

  Replacing $a$, $b$, and $c$ by $x - y$, $y - z$, and $z - x$, respectively, we conclude that
  $1 + 12 \cos 2(x - y) \cos 2(y - z) \cos 2(z - x)
  = 4 \cos 3(x - y) \cos 3(y - z) \cos 3(z - x)$.
  %43
\item Use the identity
  $\tan(A + B + C) = \frac{\tan A + \tan B + \tan C - \tan A \tan B \tan C}
  {1 - \tan A \tan B - \tan B \tan C - \tan C \tan A}$

  implies that there exist angles $A$, $B$, and $C$ such that
  $x = \tan A$, $y = \tan B$, and $z = \tan C$ with $A + B + C = \pi$.

  Using the triple-angle identity for tangent,
  $\tan 3A = \frac{3 \tan A - \tan^3 A}{1 - 3 \tan^2 A}$,
  we obtain
  $\frac{3x - x^3}{1 - 3x^2} = \tan 3A$,
  and similarly
  $\frac{3y - y^3}{1 - 3y^2} = \tan 3B$
  and
  $\frac{3z - z^3}{1 - 3z^2} = \tan 3C$.

  Therefore,
  $\displaystyle\sum \frac{3x - x^3}{1 - 3x^2}
  = \tan 3A + \tan 3B + \tan 3C$.

  Applying again the tangent sum identity, we get
  $\tan 3A + \tan 3B + \tan 3C
  = \tan 3A \tan 3B \tan 3C$,
  since $3A + 3B + 3C = 3\pi$.

  Substituting back gives
  $\displaystyle\sum \frac{3x - x^3}{1 - 3x^2}
  = \prod \frac{3x - x^3}{1 - 3x^2}$.

  Factoring the numerator yields
  $\prod (3x - x^3) = \prod x \prod (3 - x^2)$.

  Hence
  $\displaystyle\sum \frac{3x - x^3}{1 - 3x^2}
  = \frac{\prod x \prod (3 - x^2)}{\prod (1 - 3x^2)}$.

  Multiplying the numerator and denominator by $3$ gives
  $\displaystyle\sum \frac{3x - x^3}{1 - 3x^2}
  = \frac{3 \prod x \prod (3 - x^2)}{\prod (1 - 3x^2)}$
  %44
\item Consider the expression $(2 + \sqrt{3}) \sin \theta + 2 \cos \theta$.

  For real numbers $a$ and $b$, the identity
  $a \sin \theta + b \cos \theta = \sqrt{a^2 + b^2} \sin(\theta + \phi)$
  holds for some angle $\phi$.

  Hence $a \sin \theta + b \cos \theta$ lies between $-\sqrt{a^2 + b^2}$ and $\sqrt{a^2 + b^2}$.

  Here $a = 2 + \sqrt{3}$ and $b = 2$.

  We compute
  $a^2 + b^2 = (2 + \sqrt{3})^2 + 2^2 = 7 + 4 \sqrt{3} + 4 = 11 + 4 \sqrt{3}$.

  Observe that $(2 + \sqrt{5})^2 = 4 + 5 + 4 \sqrt{5} = 9 + 4 \sqrt{5}$ and
  $11 + 4 \sqrt{3} < 9 + 4 \sqrt{5}$.
  Therefore
  $\sqrt{11 + 4 \sqrt{3}} < 2 + \sqrt{5}$.

  Consequently,
  $(2 + \sqrt{3}) \sin \theta + 2 \cos \theta$
  lies between $-(2 + \sqrt{5})$ and $2 + \sqrt{5}$.
  %45
\item Consider the expression
  $5 \cos \theta + 3 \cos(\theta + \frac{\pi}{3}) + 3$.

  Using the identity $\cos(\theta + \frac{\pi}{3}) = \frac12 \cos \theta - \frac{\sqrt3}{2} \sin \theta$,
  we rewrite the expression as

  $5 \cos \theta + \frac32 \cos \theta - \frac{3\sqrt3}{2} \sin \theta + 3$,

  which simplifies to
  $\frac{13}{2} \cos \theta - \frac{3\sqrt3}{2} \sin \theta + 3$.

  For real numbers $a$ and $b$, the expression
  $a \cos \theta + b \sin \theta$
  lies between $-\sqrt{a^2 + b^2}$ and $\sqrt{a^2 + b^2}$.
  Here $a = \frac{13}{2}$ and $b = -\frac{3\sqrt3}{2}$.

  We compute
  $a^2 + b^2 = \frac{169}{4} + \frac{27}{4} = \frac{196}{4} = 49$,
  so
  $a \cos \theta + b \sin \theta$
  lies between $-7$ and $7$.

  Therefore
  $\frac{13}{2} \cos \theta - \frac{3\sqrt3}{2} \sin \theta + 3$
  lies between $-7 + 3 = -4$ and $7 + 3 = 10$.

  Hence
  $5 \cos \theta + 3 \cos(\theta + \frac{\pi}{3}) + 3$
  lies between $-4$ and $10$.
  %46
\item Using $\sin^2 \theta = 1 - \cos^2 \theta$, we rewrite it as
  $1 - \cos^2 \theta + \cos^4 \theta$.

  Let $x = \cos^2 \theta$.
  Then $0 \le x \le 1$ and the expression becomes
  $1 - x + x^2$.

  We rewrite this as
  $\left(x - \frac12\right)^2 + \frac34$,
  which is always at least $\frac34$.
  Hence the minimum value is $\frac34$.

  Since $0 \le x \le 1$, the maximum occurs at an endpoint.
  When $x = 0$, the value is $1$.
  When $x = 1$, the value is also $1$.

  Therefore, the minimum value of $\sin^2 \theta + \cos^4 \theta$ is $\frac34$,
  and the maximum value is $1$.
  %47
\item Let $a = \sin^2 x$ and $b = \cos^2 x$.
  Then $a + b = 1$ and $a \ge 0$, $b \ge 0$.

  We rewrite the expression as
  $a^4 + b^4$.

  Using the identity
  $a^4 + b^4 = (a + b)^4 - 4ab(a + b)^2 + 2a^2 b^2$,
  and substituting $a + b = 1$, we obtain
  $a^4 + b^4 = 1 - 4ab + 2a^2 b^2$.

  Let $ab = t$.
  Since $a + b = 1$ and $a,b \ge 0$,

  Recall that $a = \sin^2 x$ and $b = \cos^2 x$, so $a \ge 0$, $b \ge 0$, and
  $a + b = 1$.
  We defined $t = ab$.

  Since $a,b \ge 0$, it is immediate that $t \ge 0$.

  To find the upper bound, we use the identity
  $(a - b)^2 \ge 0$.
  Expanding gives
  $a^2 - 2ab + b^2 \ge 0$.

  Using $a + b = 1$, we have
  $a^2 + b^2 = (a + b)^2 - 2ab = 1 - 2ab$.
  Substituting into the inequality yields
  $1 - 2ab - 2ab \ge 0$,
  so
  $1 - 4ab \ge 0$.

  Hence
  $ab \le \frac14$,
  that is,
  $t \le \frac14$.

  Therefore we have
  $0 \le t \le \frac14$.

  Thus the expression becomes
  $1 - 4t + 2t^2$.

  We rewrite this as
  $2\left(t - \frac12\right)^2 + \frac12$.
  Since $0 \le t \le \frac14$, the minimum occurs at $t = \frac14$.

  Substituting $t = \frac14$ gives
  $1 - 4 \cdot \frac14 + 2 \cdot \frac{1}{16} = \frac18$.

  Therefore, the minimum value of $\sin^8 x + \cos^8 x$ is $\frac18$.
  %48
\item Consider the expression $\sin^{2n} x + \cos^{2n} x$, where $n$ is a positive integer.
  Let $a = \sin^2 x$ and $b = \cos^2 x$.
  Then $a \ge 0$, $b \ge 0$, and $a + b = 1$.

  We rewrite the expression as
  $a^n + b^n$.

  Since $0 \le a \le 1$ and $0 \le b \le 1$, raising $a$ and $b$ to a higher power does not increase their values.
  Thus
  $a^n \le a$ and $b^n \le b$.

  Adding these inequalities gives
  $a^n + b^n \le a + b = 1$.

  Therefore
  $\sin^{2n} x + \cos^{2n} x \le 1$.

  Equality holds when either $\sin^2 x = 0$ or $\cos^2 x = 0$, that is, when $x = k\pi$ or $x = \frac{\pi}{2} + k\pi$.

\stopitemize
