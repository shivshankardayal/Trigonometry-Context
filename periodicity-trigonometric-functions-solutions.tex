% -*- mode: context; -*-
\chapter{Periodicity of Trigonometrical Functions}
\startitemize[n, 1*broad]
\item The solutions are given below:
  \startitemize[i]
   \item $f(x) = 10\sin3x$. Let $f(x + T) = f(x) \Rightarrow 10\sin3(x + T) = 10\sin3x$

     $\Rightarrow \sin3(x + T) = \sin3x$

     $\Rightarrow 3x + 3T = n\pi + (-1)^n3x$, where $n = 0, \pm1, \pm2, \pm3, \ldots$

     The positive value of $T$ independent of $x$ are given by $3T = n\pi$, where
     $n = 2, 4, 6, \ldots$

     Least positive value of $T = \frac{2\pi}{3}$.

     Hence, $f(x)$ is a periodic function with a period of $\frac{2\pi}{3}$.

   \item $f(x) = a\sin\lambda x + b\cos\lambda x = \sqrt{a^2 + b^2}\left(\frac{a}{\sqrt{a^2 +
           b^2}}\sin\lambda x + \frac{b}{\sqrt{a^2 + b^2}}\cos\lambda x\right)$

     $= \sqrt{a^2+b^2}(\cos\alpha\sin\lambda x + \sin\alpha\cos\lambda x)$, where $\cos\alpha
     = \frac{a}{\sqrt{a^2 + b^2}}$

     $= \sqrt{a^2 + b^2}\sin(\lambda x + \alpha)$

     which is a periodic function with a period of $\frac{2\pi}{|\lambda|}$.

   \item $f(x) = \sin^3x = \frac{3\sin x - \sin3x}{4} = \frac{3}{4}\sin x - \frac{1}{4}\sin3x$

     $\sin x$ is a periodic function with period $2\pi$ and $\sin3x$ is a periodic
     function with period $2\pi$ so the period of the required function will be L.C.M. of these two
     periods which will be $2\pi$.

   \item $f(x) = \cos x^2$. Let $f(x + T) = f(x) \Rightarrow \cos(x + T)^2 = \cos x^2$

     $\Rightarrow (x + T)^2 = 2n\pi \pm x^2$

     In the above expression $x$ cannot be elimminated until $T = 0$ so the given function is
     non-periodic.

   \item $f(x) = \sin\sqrt{x}$. Let $f(x + T) = f(x) \Rightarrow \sin\sqrt{x + T} = \sin\sqrt{x}$

     $\Rightarrow \sqrt{x + T} = n\pi + (-1)^n\sqrt{x}$

     which will give no positive value of $T$ independent of $x$ because $\sqrt{x}$ can
     be cancelled out only if $T = 0$. Hence, $f(x)$ is a non-periodic function.

   \item $f(x) = \sqrt{\tan x}$. Let $f(x + T) = f(x) \Rightarrow \sqrt{\tan(x + T)} = \sqrt{\tan
       x} \Rightarrow \tan(x + T) = \tan x$

     $\Rightarrow x + T = n\pi + x, n = 0, \pm1, \pm2, \ldots$

     From this positive values of $T$ independent of $x$ are given by $T = n\pi, n = 1,
     2, 3, \ldots$

     $\therefore$ Least positive value of $T$ independent of $x$ is $\pi$. Hence,
     $f(x)$ is a periodic function of period $\pi$.

   \item $f(x) = x - [x]$, where $[x]$ denotes the integral part of $x$. Let
     $f(x + T) = f(x) \Rightarrow (x + T) - [x + T] = x - [x] \Rightarrow T = [x + T] - [x] =$ an
     integer

     Hence least positive value of $T$ independent of $x$ is $1$. Hence, $f(x)$
     is a periodic function having a period of $1$.

   \item $f(x) = x\cos x$. Let $f(x + T) = f(x) \Rightarrow (x + T)\cos(x + T) = x\cos x$

     $\Rightarrow T\cos(x + T) = x[\cos x - \cos(x + T)]$

     From this no value of $T$ independent of $x$ can be found because on R.H.S. one factor
     is $x$ which is an algebraid function and on L.H.S. there is no algebraic function and hance
     $x$ cannot be eliminated.

     Hence $f(x)$ is a non-periodic function.
     \stopitemize
\item The solutions are given below:
  \startitemize[i]
\item $f(x) = 4\sin\left(3x + \frac{\pi}{4}\right)$. From the sixth result of section Some Results we
  know that this is a periodic function with period $\frac{2\pi}{3}$ because $\sin3x$ is a
  periodic function with period $2\pi$.

\item $f(x) = 3\cos\frac{x}{2} + 4\sin\frac{x}{2}$. We know that both $\sin x$ and $\cos
  x$ are periodic functions with period $2\pi$. Therefore $\sin\frac{x}{2}$ and
  $\cos\frac{x}{2}$ will have a period of $4\pi$. Now the function $f(x)$ will have
  period equal to L.C.M. of periods of these two funcitons which is equal to $4\pi$.

\item $f(x) = \cot\frac{x}{2}$. We know that $\cot x$ has a period of $\pi$ therefore
  $f(x)$ will have period equal to $2\pi$.

\item $f(x) = \sin^2x = \frac{1 - \cos 2x}{2}$. We know that $\cos x$ is a periodic function
  with a period of $2\pi$ therefore $f(x)$ will be a periodic function with period of
  $\pi$.

\item $f(x) = \sin x^2$. Let $f(x + T) = f(x) \Rightarrow \sin(x + T)^2 = \sin x^2 \Rightarrow
  (x + T)^2 = n\pi + (-1)^nx^2$ which will yield no value of $T$ independent of $x$ unless
  $T = 0$. Thus, the given function is non-periodic.

\item $f(x) = \sin\frac{1}{x}$. Let $f(x + T) = f(x) \Rightarrow \sin\frac{1}{x + T} =
  \sin\frac{1}{x} \Rightarrow \frac{1}{x + T} = n\pi + (-1)^n\frac{1}{x}$ which will give no value of
  $T$ independent of $x$ unless $T = 0$. Thus, the given function is non-periodic.

\item $f(x) = 1 + \tan x$. We know that $\tan x$ is a periodic function with a period
  $\pi$. Hence, $f(x)$ will also be a periodic function with a period of $\pi$.

\item $f(x) = [x]$, where $[x]$ is integral value of $x$. Let $f(x + T) = f(x)
  \Rightarrow [x + T] = [x] \Rightarrow [x + T] - [x] = 0$ which is not true for any value of
  $T$ as for any value of $T$ it is possiblel that $[x + T] - [x] = 1$. Thus,
  $f(x)$ is non-periodic.

\item $f(x) = 5$. Let $f(x + T) = f(x) \Rightarrow 5 = 5$ which is true but gives us no value
  for $T$. Thus, the given function is periodic but has no fundamental period.

\item $f(x) = |\cos x| \Rightarrow f(x) = -\cos x$ if $\cos x < 0$ and $f(x) = \cos x$ if
  $\cos x > 0$. We know that $\cos x$ has a period of $2\pi$ therefore $f(x)$
  will have period equal to half the period of that of $\cos x$ i.e. $\pi$.

\item $f(x) = \sin^4x + \cos^4x = (\sin^2x + \cos^2x)^2 - 2\sin^2x\cos^2x = 1 - \frac{\sin^22x}{2} =
  1 - \frac{1 - \cos4x}{4} = \frac{3}{4} + \frac{1}{4}\cos4x$. We know that $\cos x$ is a
  function having period $2\pi$ therefore $f(x)$ will be a periodic function with a period
  $2\pi/4$ i.e. $\frac{\pi}{2}$.

\item $f(x) = x + \sin x$. Let $f(x + T) = f(x) \Rightarrow x + T + \sin(x + T) = x + \sin x$
  $\Rightarrow T = \sin x - \sin(x + T)$ which will give no value of $T$ independent of
  $x$ as R.H.S. is a trigonometric function in $x$ but L.H.S. is not. So the function
  $f(x)$ is non-periodic.

\item $f(x) = \cos\sqrt{x}$. Following the fifth problem of previous problem we can deduce that
  given function is non-periodic.

\item $f(x) = \tan^{-1}(\tan x)$. Let $f(x + T) = f(x) \Rightarrow \tan^{-1}\tan(x + T) =
  \tan^{-1}(\tan x) \Rightarrow \tan(x + T) = \tan x$ which gives $T = \pi$ as the period.

\item $f(x) = |\sin x| + |\cos x|$ which will yield four different equations depending on whether
  $\sin x$ and $\cos x$ are positive or negative. Also, the period of $\sin x$ and
  $\cos x$ is $2\pi$ for both of the functions. Thus, the given function will have a period
  of $2\pi/4 = \frac{\pi}{2}$.

\item $f(x) = \sin\frac{\pi x}{3} + \sin\frac{\pi x}{4}$. We know that $\sin x$ has a period
  of $2\pi$ therefore $\sin\frac{\pi x}{3}$ will have a period of $6$ and
  $\sin \frac{\pi x}{4}$ will have a period of $8$. The given function will have period
  equal to L.C.M. of $6$ and $8$ i.e. 24.

\item $f(x) = \sin\left(2\pi x + \frac{\pi}{3}\right) + 2\sin\left(3\pi x + \frac{\pi}{4}\right) +
  3\sin\pi x$. We know that the period of $\sin x$ has a period of $2\pi$ so the three
  terms will have period of $1, 2/3$ and $2$ respectively. Thus, given function will have
  period equal to L.C.M. of these three periods i.e. $2$.

\item $f(x) = \sin x + \cos\sqrt{x}$. Now we have proven that $\cos\sqrt{x}$ is a
  non-periodic function therefore $f(x)$ will also be non-periodic.
  \stopitemize
\item $f(x) = 2\sin x + 3\cos 2x$. We know that both $\sin x$ and $\cos x$ have a period of
  $2\pi$ therefore period of first term would be $2\pi$ and of the second term will be
  $\pi$. $f(x)$ will have period equal to L.C.M. of these two terms i.e. $2\pi$.

\item a. Given, $f(x) = 2x - [2x]$. Let $f(x + T) = f(x) \Rightarrow 2(x + T) - [2(x + T)] = 2x -
  [2x]$

  $\Rightarrow T = \frac{[2x + 2T] - [2x]}{2} = \frac{\mathrm{an integer}}{2}$.

  Therefore, positive value of $T$ independentt of $x$ can be found and least such value is
  $\frac{1}{2}$.

  b. Given, $g(x) = 1 + \frac{3}{2 - \sin^2x}$. Let $g(x + T) = g(x)$

  $\Rightarrow \frac{3}{2 - \sin^2(x + T)} = \frac{3}{2 - \sin^2x}$

  $\Rightarrow \sin^2(x + T) = \sin^2x \Rightarrow x + T = n\pi +(-1)^n(\pm x) = n\pi\pm x$

  which gives us a periodic function with $T = \pi$.

\item $1 - \frac{1}{4}\sin^2\left(\frac{\pi}{3} - \frac{3x}{2}\right) = \frac{7}{8} +
  \frac{1}{8}\cos\left(3x - \frac{2\pi}{3}\right)$ which is a periodic function with period $2\pi/3$.
\stopitemize
