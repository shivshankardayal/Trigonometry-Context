% -*- mode: context; -*-
\chapter{Trigonometric Inequations and Inequalities}
\startitemize[n, 1*broad]
\item Consider the following diagram:
  \startplacefigure
    \externalfigure[35_1.pdf]
  \stopplacefigure
  As you can see the value of $\sin$ is greater than $\frac{1}{2}$ between $30^\circ$ and $150^\circ$ as
  shown in the shaded area..

  So the general solution will be $2n\pi + \frac{\pi}{6} < x < 2n\pi + \frac{5\pi}{6}$ where $n = 0, \pm1,
  \pm 2., \ldots$.
\item When $\cos x = -\frac{1}{2}$ it is satisfied for $x = \frac{2\pi}{3}, \frac{4\pi}{3}$.

  $\Rightarrow 2n\pi - \frac{2\pi}{3}\leq x\leq 2n\pi + \frac{2\pi}{3}$, where $n = 0, \pm 1, \pm2, \ldots$.
\item $\tan x$ is positive in first and third quadrant so it will be greater than $-\sqrt{3}$. In the second
  quadrant between $\frac{\pi}{2}$ and $\frac{5\pi}{6}$ its value will be less than $-\sqrt{3}$. In fourth
  quadrant between $\frac{3\pi}{2}$ and $\frac{5\pi}{3}$ its value will be less than $-\sqrt{3}$.
\item Given $2\sin^2x - 3\sin x + 1\geq 0\Rightarrow \sin x\geq 1$ or $\sin x \leq \frac{1}{2}$

  $\Rightarrow \sin x = 1$ or $\sin x\leq \frac{1}{2}$

  If $\sin x = 1$ then $x = \frac{\pi}{2}$ in the interval $[0, \pi]$.

  If $\sin x\leq \frac{1}{2}$ then $\frac{5\pi}{6}\geq x\leq \frac{\pi}{6}$.
\item $2\cos^2\theta + \sin\theta \leq 2\Rightarrow 2 - 2\sin^2\theta + \sin\theta\leq 2$

  $\Rightarrow 2\sin^2\theta - \sin\theta \geq 0 \Rightarrow \sin\theta\leq 0$ or $\sin\theta\geq
  \frac{1}{2}$

  For $\sin\theta \leq 0 \Rightarrow \pi\leq\theta\leq 2\pi$

  $\sin\theta\geq \frac{1}{2}\Rightarrow \frac{\pi}{6}\theta\leq \frac{5\pi}{6}$

  $\Rightarrow A\cap B = \left[\frac{\pi}{2, \frac{5\pi}{6}}\right]\cup \left[\pi, \frac{3\pi}{2}\right]$.
\item Since $A, B, C$ are acute, therefore,  the tangents of these angles will be positive. Using A.M.-G.M.\
  inequality, we have

  $\frac{\tan A + \tan B + \tan C}{3}\geq \sqrt[3]{\tan A\tan B\tan C}$

  In $\triangle ABC, \tan A + \tan B + \tan C = \tan A\tan B\tan C$

  $\Rightarrow \left(\frac{\tan A\tan B\tan C}{3}\right)^3\geq \tan A\tan B\tan C \Rightarrow \tan
  A\tan B\tan C\geq 3\sqrt{3}$.
\item We know that in a $\triangle ABC, \cos A + \cos B + \cos C = 1 +
  4\sin\frac{A}{2}\sin\frac{B}{2}\sin\frac{C}{2}$

  $\Rightarrow \sin\frac{A}{2}\sin\frac{B}{2}\sin\frac{C}{2} = \frac{1}{8}$

  Again $\cos A + \cos B + \cos C = 1 + 4\sin\frac{A}{2}\sin\frac{B}{2}\sin\frac{C}{2}$

  $\Rightarrow 1 - 2\sin^2\frac{A}{2} + 1 - 2\sin^2\frac{B}{2} + 1 - \sin^2\frac{C}{2} = \frac{3}{2}$

  Let $\sin\frac{A}{2} = x, \sin\frac{B}{2} = y, \sin\frac{C}{2} = z$, then

  $1 - 2x^2 + 1 - 2y^2 + 1 - 2z^2 = \frac{3}{2} \Rightarrow x^2 + y^2 + z^2 = \frac{3}{4}$

  A.M. is $\frac{x^2 + y^2 + z^2}{3} = \frac{1}{4}$ and G.M. is $\sqrt[3]{x^2y^2z^2} = \frac{1}{4}$

  Thus, A.M = G.M.\ which will be true only if $x = y = z$

  $\Rightarrow x = \frac{1}{2}$ which gives us $A = B = C = \frac{\pi}{3}$, and hence, the triangle is
  equilateral.
\item Let $f(x) = x - \sin x\Rightarrow f'(x) = 1 - \cos x > 0$ in the interbval $\left]0,
  \frac{\pi}{2}\right[, f'(x) > 0$

  Thus, $f(x)$ is an increasing function in the given interval.

  $\Rightarrow \alpha < \beta \Rightarrow \alpha - \sin\alpha < \beta - \sin\beta$.
\item Let $f(x) = \cos x - 1 + \frac{x^2}{2} \Rightarrow f'(x) = -\sin x + x$

  Let $g(x) = x - \sin x \Rightarrow g'(x) = 1 - \cos x$, which is an increasing function in the given
  interval.

  $\Rightarrow x - \sin x > 0$ so $f(x)$ is also an increasing function in the given interval.

  $\Rightarrow \cos x - 1 + \frac{x^2}{2} > 0 \Rightarrow \cos x > 1 - \frac{x^2}{2}$.
\item Let $y = 3\sin \sqrt{\frac{\pi^2}{16} - x^2}$, then domain of $y = \left[-\frac{\pi}{4},
  \frac{\pi}{4}\right]$

  $\frac{dy}{dx} = \frac{-3x\cos\sqrt{\frac{\pi^2}{16} - x^2}}{\sqrt{\frac{\pi^2}{16} - x^2}}$

  If $0 < \sqrt{\frac{\pi^2}{16} - x^2}\leq \frac{\pi}{4}$ then $\frac{dy}{dx} < 0$

  So $y$ has maximum value at $x = 0$, which is $y = s3\sin\frac{\pi}{4} = \frac{3}{\sqrt{2}}$

  When $x = -\frac{\pi}{4}, y = 0$. Hence $0\leq y\leq\frac{3}{\sqrt{2}}$.
\item Given that $\alpha + \beta + \gamma < \frac{\pi}{2}$, therefore

  $\tan(\alpha + \beta + \gamma) = \frac{\tan\alpha + \tan\beta + \tan\gamma -
  \tan\alpha\tan\beta\tan\gamma}{1 - \tan\alpha\tan\beta - \tan\beta\tan\gamma - \tan\gamma\tan\alpha}$

  The L.H.S.\ is positive for given range of angles and thus we have to have a positive denominator.

  $\tan\alpha\tan\beta + \tan\beta\tan\gamma + \tan\gamma\tan\alpha < 1$.

  {\bf Hints:} If $\alpha + \beta + \gamma = \frac{\pi}{2}$ then tangent of the sum of angle is infinity,
  which is possible only if denominator is zero. One could argue that both the denominator and numerator are
  negative. However, we know that $\tan\alpha\tan\beta + \tan\beta\tan\gamma + \tan\gamma\tan\alpha = 1$,
  when sum of angles is is $\frac{\pi}{2}$.

  Assume $\gamma_1 < \gamma$ then $\tan\alpha\tan\beta + \tan\beta\tan\gamma_1 + \tan\gamma_1\tan\alpha < 1$.
\item $0 < \sin\alpha _1 < \sin\alpha_2 < \cdots < \sin\alpha_n$ and $\cos\alpha_1 > \cos\alpha_2 >
  \cdots > \cos\alpha_n > 0$

  $\Rightarrow n\sin\alpha_1 < \sin\alpha_1 + \sin\alpha_2 + \cdots + \sin\alpha_n < n\sin\alpha_n$ and
  $n\cos\alpha_1 > \cos\alpha_1 + \cos\alpha_2 + \cdots + \cos\alpha_n > n\cos\alpha_m$

  Dividing we get the desired result.
\item Let $S = \cos^2A + \cos^2B + \cos^2C + 2\cos A\cos B\cos C$

  $= \frac{a^2(b^2 + c^2 - a^2) + b^2(c^2 + a^2 - b^2)^2 + c^2(a^2 + b^2 - c^2)^2 + (b^2 + c^2 - a^2)(c^2 +
  a^2 - b^2)(a^2 + b^2 - c^2)}{4a^2b^2c^2}$

  $= 1$

  $\Rightarrow \cos^2A + \cos^2 B + \cos^2C = 1 - 2\cos A\cos B\cos C$

  Since $A, B, C$ are acute, R.H.S.\ $> 0\Rightarrow \cos^2A + \cos^2B + \cos^2C < 1$.
\item Following previous example if one of the roots is not acute then $1 - 2\cos A\cos B\cos C \geq 1$

  Thus, $\cos^2A + \cos^2B + \cos^2C \geq 1$.
\item L.H.S.\ $= \frac{3 - \cos2\alpha - \cos2\beta - \cos2\gamma}{2}$

  $2\gamma = \pi - 2(\alpha + \beta)\Rightarrow \cos2\gamma = - \cos2(\alpha + \beta)$

  $\cos2\alpha + \cos2\beta = 2\cos(\alpha + \beta)\cos(\alpha - \beta)$

  Now $\cos2\alpha + \cos2\beta - \cos(2\alpha + 2\beta) = 2\cos(\alpha + \beta)\cos(\alpha - \beta) -
  (2\cos^2(\alpha + \beta) - 1)$

  $= 1 + 2\cos(\alpha + \beta)(\cos(\alpha - \beta) - \cos(\alpha + \beta)) \leq 1 + 2\cos(\alpha + \beta)(1
  - \cos(\alpha + \beta))$

  Let $\cos(\alpha - \beta) = t$, then $f(t) = 1 = 2t - 2t^2$, which is maximum at $t = \frac{1}{2}$

  $f\left(\frac{1}{2}\right) = \frac{3}{2}$

  $\Rightarrow \cos2\alpha + \cos2\beta + \cos2\gamma\leq \frac{3}{2}$

  $\Rightarrow \sin^2\alpha + \sin^2\beta + \sin^2\gamma \geq \frac{3}{4}$
\item We have to prove that $\cos(\alpha + \beta)\cos(\alpha - \beta)\leq \cos^2\alpha$

  $\Rightarrow \frac{\cos2\alpha + \cos2\beta}{2}\leq \cos^2\alpha = \frac{1 + \cos2\alpha}{2}$

  $\Rightarrow \cos2\alpha + \cos2\beta\leq 1 + \cos2\alpha \Rightarrow \cos2\beta\leq 1$, which is always
  true for all real $\beta$.
\item Let $E = \cos^2\alpha + \cos^2(\alpha + \beta) - 2\cos\alpha\cos\beta\cos(\alpha + \beta)$

  $E = \cos^2\alpha + (\cos\alpha\cos\beta - \sin\alpha\sin\beta)^2 - 2\cos\alpha\cos\beta\cos(\alpha +
  \beta)$

  $= \cos^2\alpha + \cos^2\alpha\cos^2\alpha - 2\cos\alpha\cos\beta\sin\alpha\sin\beta -
  2\cos^2\alpha\cos^2\beta + 2\cos\alpha\cos\beta\sin\alpha\sin\beta$

  $= \cos^2\alpha - \cos^2\alpha\cos^2\beta + \sin^2\alpha\sin^2\beta = \cos^2\alpha(1 - \cos^2\beta) +
  \sin^2\alpha\sin^2\beta$

  $= \cos^2\alpha\sin^2\beta + \sin^2\alpha\sin^2\beta = \sin^2\beta$

  Thus, $0 \leq E \leq 1$.
\item $\left(\sin^2\alpha + \cos^2\alpha\right)^3 = 1 \Rightarrow \sin^6\alpha + \cos^6\alpha +
  3\sin^2\alpha\cos^2\alpha(\sin^2\alpha + \cos^2\alpha) = 1$

  $\Rightarrow \sin^6\alpha + \cos^6\alpha + \frac{3}{4}(2\sin\alpha\cos\alpha)^2 = 1 \Rightarrow
  \sin^6\alpha + \cos^6\alpha = 1 - \frac{3}{4}\sin^22\alpha$

  Since $0 \leq \sin^22\alpha \leq 1\Rightarrow \sin^6\alpha + \cos^6\alpha \geq \frac{1}{4}$.
\item $\cos A + \cos B + \cos C = 2\sin\frac{C}{2}\cos\frac{A - B}{2} + \cos C = 2\sin\frac{C}{2}\cos\frac{A
  - B}{2} + 1 - 2\sin^2\frac{C}{2}\leq 2\sin\frac{C}{2} + 1 - 2\sin^2\frac{C}{2}$

  Let $\sin\frac{C}{2} = x$, and $f(x) = 2x + 1 - 2x^2\leq \frac{3}{2}\Rightarrow -4x^2 + 4x - 1 \leq 0
  \Rightarrow 4x^2 - 4x + 1\geq 0$

  $(2x - 1)^2\geq 0$, which is always true.
\item From previous problem we have proven that $\cos A + \cos B + \cos C\leq \frac{3}{2}$.

  Now, $\cos A + \cos B + \cos C = 1 + 4\sin\frac{A}{2}\sin\frac{B}{2}\sin\frac{C}{2}\leq \frac{3}{2}$

  $\Rightarrow \sin\frac{A}{2}\sin\frac{B}{2}\sin\frac{C}{2}\leq \frac{1}{8}$.
\item L.H.S. $= \cos\alpha + 3\cos3\alpha + 6\cos6\alpha = \cos\alpha + 3\cos3\alpha + 6\left(2\cos^23\alpha
  - 1\right)$

  $= \cos\alpha + 3\left(4\cos^23\alpha + \cos3\alpha\right) - 6 = \cos\alpha + 3\left(2\cos3\alpha +
  \frac{1}{4}\right)^2 - \frac{3}{16} - 6 > -\frac{115}{16}$

  $\Rightarrow \cos\alpha + 3\left(2\cos3\alpha + \frac{1}{4}\right)^2\geq -1$, which is always true.
\item We have to prove that $\cos36^\circ > \tan36^\circ \Rightarrow \cos^236^\circ > \sin36^\circ$

  $\Rightarrow 1 + \cos72^\circ > 2\sin36^\circ \Rightarrow 1 + \sin18^\circ > 2\sin(30^\circ + 6^\circ)$

  $\Rightarrow 1 + 2\sin9^\circ\cos9^\circ > \cos6^\circ + 2\sin6^\circ\cos30^\circ$

  which is true because $1 > \cos6^\circ, \sin9^\circ > \sin6^\circ$ and $\cos9^\circ > \cos30^\circ$.
\item We have to prove that $\sin\alpha\sin2\alpha\sin3\alpha < \frac{3}{4}$

  $\Rightarrow \frac{1}{2}(2\sin\alpha\sin3\alpha)\sin2\alpha < \frac{3}{4}$

  $\Rightarrow \frac{1}{2}(\cos2\alpha - \sin4\alpha)\sin2\alpha = \frac{1}{4}(\sin4\alpha -
  2\sin2\alpha\sin4\alpha)$

  $= \frac{1}{4}\sin4\alpha(1 - 2\sin2\alpha)$. Now maximum value of $\sin4\alpha = 1$ and that of $1 -
  2\sin2\alpha = 3$

  Thus, $\frac{1}{4}\sin4\alpha(1 - 2\sin2\alpha) < \frac{3}{4}$.
\item Using A.M.-G.M.\ inequality $\frac{\sin\alpha + \sin\beta + \sin\gamma}{3} \geq
  \sqrt[2]{\sin\alpha\sin\beta\sin\gamma}$

  Squaring $(\sin\alpha + \sin\beta + \sin\gamma)^2\geq 9\sqrt[3]{(\sin\alpha\sin\beta\sin\gamma)^2}$.

  Now $\sqrt[3]{(\sin\alpha\sin\beta\sin\gamma)^2} \geq \sin\alpha\sin\beta\sin\gamma$ because
  $\sin\alpha\sin\beta\sin\gamma\leq 1$

  Hence, $(\sin\alpha + \sin\beta + \sin\gamma)^2\geq 9\sin\alpha\sin\beta\sin\gamma$.
\item $\left(\sin^2\alpha + \cos^2\alpha\right)^2 = 1 \Rightarrow \sin^4\alpha + \cos^4\alpha +
  2\sin^2\alpha\cos^2\alpha = 1$

  $\Rightarrow \sin^4\alpha + \cos^4\alpha = 1 - \frac{1}{2}\sin^2\alpha$. Now maximum value of
  $\sin^22\alpha = 1$

  $\Rightarrow \sin^4\alpha + \cos^4\alpha \geq \frac{1}{2}$.
\item Let $f(x) = \frac{\sin x}{x}$ so it would be sufficient to show that $f(x)$ is strictly decreasing in
  the interval of $\left(0, \frac{\pi}{2}\right)$.

  $f'(x) = \frac{x\cos x - \sin x}{x^2}$. Let $g(x) = x\cos x - \sin x$. In the given interval sign of
  $f'(x)$ is governed by sign of $g(x)$.

  At $x = 0^+, g(x)\rightarrow 0$ and $g'(x) = \cos x -x\sin x - \cos x = -x\sin x < 0, \forall x\in\left(0,
  \frac{\pi}{2}\right)$. Thus, $g(x)$ is strictly decreasing function making $f(x)$ also strictly
  decreasing.

  $\Rightarrow f(\alpha) > f(\beta)\Rightarrow\frac{\sin\alpha}{\alpha} > \frac{\sin\beta}{\beta}$.
\item We have to show that $\frac{\tan\alpha}{\alpha} > \frac{\alpha}{\sin\alpha} \Rightarrow
  \frac{\sin\alpha}{\alpha} > \sqrt{\cos\alpha}$

  Llet $f(x) = \frac{\sin\alpha}{\alpha\sqrt{\cos\alpha}}$, taking derivative of log

  $\frac{d}{dx}\ln f(x) = \cot x - \frac{1}{x} + \frac{\tan x}{2}$, which can be proven to be strictly
  increasing. Hence, $f(x)$ is strictly increasing in the given range, which proves the inequality.
\item Let $f(x) = \alpha - \tan\frac{\alpha}{2}\Rightarrow f'(x) = 1 - \frac{1}{2\cos^2\frac{\alpha}{2}}$,
  which is positive. Hence, $f(x)$ is a strictly increasing function.

  Thus, $\alpha > \tan\frac{\alpha}{2}$.
\item Let $f(x) = \sin x - x + \frac{x^3}{3}, f'(x) = \cos x - 1 + x^2 = x^2 - (1 - \cos x)$

  We know that $1 - \cos x < \frac{x^2}{2} \Rightarrow f'(x) > \frac{x^2}{2} > 0$, which makes $f(x)$
  strictly increasing.

  Hence, $\sin \alpha> \alpha - \frac{\alpha^3}{3}$.
\item Squaring $\cos\theta < 2\cos\frac{\theta}{2}\Rightarrow 2\cos^2\frac{\theta}{2} - -1 -
  2\cos\frac{\theta}{2} < 0$

  $\Rightarrow x = \frac{1\pm\sqrt{3}}{2}$, where $x = 2\cos\frac{\theta}{2}$, hence, the inequality holds.
\item Since $\cos x \neq 0$, we can divide both numerator and denominator by $\cos^3 x$ gives us

  $\frac{\cos x}{\sin^2x(\cos x - \sin x)} = \frac{1 + \tan^2x}{\tan^2x(1 - \tan x)} = \frac{1 - \tan
  x}{\tan^2x} + \frac{2}{\tan x(1 - \tan x)}$

  In the given range both $\tan x$ and $1 - \tan x$ are positive and $\tan x + 1 - \tan x = 1$.

  From A.M.-G.M.\ inequality the least value of $\frac{2}{\tan x(1 - \tan x)} = 8$

  Also, $\frac{1 - \tan x}{\tan^2x} > 0$ for the given range. Hence, the inequality is proven.
\item We have proven that $\sin^4x + \cos^4x\geq \frac{1}{2}$. Squaring $(\sin^4x + \cos^4x)\geq
  \frac{1}{4}$

  L.H.S. $= \sin^8x + \cos^8x + 2\sin^4x\cos^4x = \sin^8x + \cos^8x + \frac{1}{8}\sin^42x$

  Maximum value of $\sin^42x = 1$ and thus, $\sin^8x + \cos^8x \geq \frac{1}{8}$.
\item L.H.S. $= (x + y)^2 + 2(x + y)\cos x + \cos^2 x + \cos^2x\geq 0$

  $\Rightarrow (x + y + \cos x)^2 + \cos^2x \geq 0$, which is a strict inequality. The equality holds only
  if $x + y + \cos x = 0$ and $\cos x = 0\Rightarrow x + y = 0$

  The quality holds if $x = \frac{\pi}{2}(2k + 1)$ and $y = -\frac{\pi}{2}(2k + 1), k\in\mathbb{Z}$.
\item Let $f(x) = \cos2x + 3\sin x = -2\sin62x + 3\sin x + 1$. Let $\sin x = z$, where $|z|\leq 1$, then
  $f(x) = -2z^2 + 3z + 1$.

  For $z = \frac{-3}{2(-2)} = \frac{3}{4}$, the function will take the greatest value of $\frac{17}{8}$.

  To find the least value of $f(x)$ on the closed interval $\left[-1, 1\right]$, we can prove by virtue of
  properties of a quadratic equation that it is enough to compare its values at the end of intervals.

  The least value is $-4$ at $z = -1$. Thus, the inequality is proven.
\item For $\sin x$ we have $\sin^8x\leq \sin^2x$ and for $\cos x, \cos^{14}x\leq \cos^2 x$.

  Hence, $0 < \sin^28x + \cos^{14}x\leq 1$. It will be equal to zero if both $\sin x$ and $\cos x$ are zero
  at the same time, which is impossible.
\stopitemize
