% -*- mode: context; -*-
\chapter{Trigonometrical Equations}
\startitemize[n, 1*broad]
\item Given equation is $\sin\theta = -1$

  $\Rightarrow \sin\theta = \sin\left(-\frac{\pi}{2}\right)$

  $\Rightarrow \theta = n\pi + (-1)^n\left(-\frac{\pi}{2}\right)$

  $\theta = n\pi + (-1)^{n +1}\frac{\pi}{2}$ where $n\in I.$
\item Given equation is $\cos\theta = -\frac{1}{2}$

  $\cos\theta = \cos\frac{2\pi}{3} \Rightarrow \theta = 2n\pi \pm \frac{2\pi}{3}$ where $n\in I.$
\item Given equation is $\tan\theta = -\sqrt{3}$

  $\Rightarrow \tan\theta = \tan\left(-\frac{\pi}{3}\right) \Rightarrow \theta = n\pi + \left(-\frac{\pi}{3}\right)$

  $= n\pi - \frac{\pi}{3}$ where $n\in I.$
\item Given equation is $\sec\theta = -\sqrt{2}$

  $\Rightarrow \sec\theta = \sec\frac{3\pi}{4}\Rightarrow \theta = 2n\pi\pm \frac{3\pi}{4}$ where $x\in I$
\item Given equation is $\sin8\theta = \sin\theta \Rightarrow \sin9\theta - \sin\theta = 0$

  $\Rightarrow 2\cos5\theta.\sin4\theta = 0$

  Either $\cos5\theta = 0$ or $\sin4\theta = 0$

  $\Rightarrow 5\theta = (2n + 1)\frac{\pi}{2}$ or $4\theta = n\pi$

  $\theta = \frac{n\pi}{4}, (2n + 1)\frac{\pi}{10}$ where $n\in I.$
\item Given equation is $\sin5x = \cos2x$

  $\Rightarrow \cos2x = \cos\left(\frac{\pi}{2} - 5x\right)$

  $2x = 2n\pi \pm \left(\frac{\pi}{2} - 5x\right)$

  $x = (4n + 1)\frac{\pi}{14}, -(4n - 1)\frac{\pi}{6}$ where $n\in I$
\item Given equation is $\sin3x = \sin x \Rightarrow \sin3x - \sin x= 0$

  $\Rightarrow \cos2x.\sin x = 0$

  Either $\cos2x = 0$ or $\sin x= 0$

  $\Rightarrow 2x = (2n + 1)\frac{\pi}{2}$ or $x = n\pi$

  $x = n\pi, (2n + 1)\frac{\pi}{4}$ where $n\in I$
\item Given equation is $\sin3x = \cos2x \Rightarrow \cos2x = \cos\left(\frac{\pi}{2} - 3x\right)$

  $\Rightarrow 2x = 2n\pi \pm \left(\frac{\pi}{2} - 3x\right)$

  $x = \frac{2n\pi}{5} + \frac{\pi}{10}, -2n\pi + \frac{\pi}{2}$
\item Given equation is $\sin ax + \cos bx = 0$

  $\Rightarrow \sin ax + \sin\left(\frac{\pi}{2} - bx\right) = 0$

  $\Rightarrow 2\sin\left(\frac{\pi}{4} + \frac{(a - b)x}{2}\right)\cos\left(\frac{(a + b)x}{2} - \frac{\pi}{4}\right) = 0$

  Either $\Rightarrow \sin\left(\frac{\pi}{4} + \frac{(a - b)x}{2}\right) = 0$ or $\cos\left(\frac{(a + b)x}{2} -
  \frac{\pi}{4}\right) = 0$

  $\Rightarrow \frac{\pi}{4} + \frac{(a - b)x}{2} = n\pi$ or $\frac{(a + b)x}{2} - \frac{\pi}{4} = (2n +
  1)\frac{\pi}{2}$

  $x = \frac{2n\pi - \frac{\pi}{2}}{a - b}, \frac{(2n + 1)\pi + \frac{\pi}{2}}{a + b}$
\item Given $\tan x\tan 4x =1 \Rightarrow \sin x\sin4x = \cos x\cos4x$

  $\Rightarrow \cos x\cos4x - \sin x\sin 4x = 0$

  $\cos 5x = 0 \Rightarrow 5x = (2n + 1)\frac{\pi}{2} \Rightarrow x = \frac{(2n + 1)\pi}{10}$
\item Given equation is $\cos\theta = \sin105^\circ + \cos 105^\circ$

  $\sin105^\circ = \sin(60^\circ + 45^\circ) = \frac{\sqrt{3} + 1}{2\sqrt{2}}$

  $\cos105^\circ = \cos(60^\circ + 45^\circ) = \frac{1 - \sqrt{3}}{2\sqrt{2}}$

  $\Rightarrow \cos\theta = \frac{1}{\sqrt{2}} \Rightarrow \theta = 2n\pi \pm \frac{\pi}{4}$
\item Given equation is $7\cos^2\theta + 3\sin^2\theta = 4$

  $\Rightarrow 4\cos^2\theta + 3 = 4 \Rightarrow \cos\theta = \pm \frac{1}{2}$

  If $\cos\theta = \frac{1}{2}\Rightarrow \theta = 2n\pi\pm\frac{\pi}{3}$

  If $\cos\theta = -\frac{1}{2}\Rightarrow \theta = 2n\pi\pm\frac{2\pi}{3}$
\item Given equation is $3\tan(\theta - 15^\circ) = \tan(\theta + 15^\circ)$

  $\Rightarrow \frac{\tan(\theta + 15^\circ)}{\tan(\theta - 15^\circ)} = \frac{3}{1}$

  Applying componendo and dividendo

  $\Rightarrow \frac{\tan(\theta + 15^\circ) + \tan(\theta - 15^\circ)}{\tan(\theta + 15^\circ) - \tan(\theta - 15^\circ)}
  = \frac{4}{2}$

  $\Rightarrow \frac{\sin(\theta + 15^\circ + \theta - 15^\circ)}{\sin(\theta + 15^\circ - \theta + 15^\circ)} = 2$

  $\Rightarrow \sin2\theta = 2 \Rightarrow 2\theta = n\pi + (-1)^n\frac{\pi}{2} \Rightarrow \theta = \frac{n\pi}{2} +
  (-1)^n\frac{\pi}{4}$
\item Given equation is $\tan x + \cot x = 2 \Rightarrow \tan^2x - 2\tan x + 1 = 0$

  $\Rightarrow (\tan x - 1)^2 = 0 \Rightarrow \tan x = 1 \Rightarrow x = n\pi + \frac{\pi}{4}$
\item Given equation is $\sin^2\theta = \sin^2\alpha \Rightarrow \sin\theta = \pm \sin\alpha$

  $\theta = n\pi \pm\alpha$
\item Given equation is $\tan^2x + \cot^2x = 2$

  $\Rightarrow \tan^4x - 2\tan^2x + 1 = 0 \Rightarrow (\tan^2x - 1)^2 = 0$

  $\tan x = \pm \Rightarrow x = n\pi \pm \frac{\pi}{4}$
\item Given equation is $\tan^2x = 3\csc^2x - 1$

  $\Rightarrow \tan^2x = 2 + 3\cot^2x \Rightarrow \tan^4x -2\tan^2x - 3 = 0$

  $\Rightarrow (\tan^2x + 1)(\tan^2x - 3) = 0$

  If $\tan^2x + 1 = 0$ then $x$ will become imaginary.

  $\therefore \tan x = \pm\sqrt{3} \Rightarrow x = n\pi \pm \frac{\pi}{3}$
\item Given equation is $2\sin^2x + \sin^22x = 2$

  $\Rightarrow 2\sin^2x + 4\sin^2x\cos^2x = 2 \Rightarrow \sin^2x + 2\sin^2x(1 - \sin^2x) = 1$

  $\Rightarrow (2\sin^2x - 1)(\sin^2x - 1) = 0$

  $\Rightarrow \sin x = \pm\frac{1}{\sqrt{2}}$ or $\sin x = \pm 1$

  $\Rightarrow x = n\pi \pm \frac{\pi}{4}, (2n + 1)\frac{\pi}{2}$
\item Given equation is $7\cos^2 x + 3\sin^2 x = 4$

  $\Rightarrow 4\cos^2 x + 3 = 4 \Rightarrow \cos x = \pm \frac{1}{2}$

  If $\cos x = \frac{1}{2}\Rightarrow x = 2n\pi\pm\frac{\pi}{3}$

  If $\cos x = -\frac{1}{2}\Rightarrow x = 2n\pi\pm\frac{2\pi}{3}$
\item Given equation is $2\cos2x + \sqrt{2\sin x} = 2$

  $\Rightarrow \sqrt{2\sin x} = 2(1 - \cos2x) = 4\sin^2x$

  $\Rightarrow \sqrt{2\sin x}\left(1 - 2\sqrt{2}\sin^{\frac{3}{2}}x\right) = 0$

  Either$\sin x = 0 \Rightarrow x = n\pi$ where $n\in I$

  or $\sin^{\frac{3}{2}}x = \frac{1}{2\sqrt{2}} \Rightarrow \sin x = \frac{1}{2}$

  $\Rightarrow x = n\pi + (-1)^n\frac{\pi}{6}$
\item We know that $\tan^2\frac{x}{2} = \frac{1 - \cos x}{1 + \cos x}$

  $\therefore 8\left(\frac{1 - \cos x}{1 + \cos x}\right) = 1 + sec x = \frac{1 + \cos x}{\cos x}$

  $\Rightarrow 8\cos x - 8\cos^2x = (1 + \cos x)^2$

  $\Rightarrow 9\cos^2x - 6\cos x + 1 = 0 \Rightarrow (3\cos x - 1)^2 = 0$

  $\cos x = \frac{1}{3} \Rightarrow x = 2n\pi \pm \cos^{-1}\frac{1}{3}$ where $n in I.$

  Check $\frac{x}{2}\neq (2n + 1)\frac{\pi}{2}$ and $\cos x \neq = 0$ else equation will be meaningless.

  $\Rightarrow x\neq (2n + 1)\pi$ and $x\neq (2n + 1)\frac{\pi}{2}$
\item Given equation is $\cos x\cos2x\cos3x = \frac{1}{4}$

  $\Rightarrow (2\cos x\cos3x)2\cos2x = 1 \Rightarrow 2\cos4x\cos2x + 2\cos^22x - 1 = 0$

  $\Rightarrow \cos4x[2\cos2x + 1] = 0$

  If $\cos4x = 0 \Rightarrow x = (2n + 1)\frac{\pi}{8}$

  If $2\cos2x + 1 = 0 \Rightarrow 2x = 2n\pi \pm \frac{2\pi}{3}$

  $x = n\pi \pm \frac{\pi}{3}$
\item Given equation is $\tan x + \tan2x + \tan3x = 0$

  $\Rightarrow \tan x +\tan2x + \frac{\tan x + \tan 2x}{1 - \tan x\tan 2x} = 0$

  $\Rightarrow (\tan x + \tan 2x)\left(1 + \frac{1}{1 - \tan x\tan 2x}\right) = 0$

  If $\tan x + \tan 2x = 0 \Rightarrow \tan x = -\tan 2x \Rightarrow x = n\pi -2x \Rightarrow x = \frac{n\pi}{3}$

  If $1 + \frac{1}{1 - \tan x\tan 2x} = 0$ then $\tan x\tan 2x = 2$

  $\frac{\tan^2x}{1 -\tan^2x} = 1 \Rightarrow \tan x = \pm\frac{1}{\sqrt{2}}$

  $x = n\pi \pm\tan^{-1}\frac{1}{\sqrt{2}}$
\item Given equation is $\cot x - \tan x - \cos x + \sin x = 0$

  $\Rightarrow \frac{\cos^2x - \sin^2x}{\cos x\sin x} - (\cos x - \sin x) = 0$

  $\Rightarrow (\cos x - \sin x)\left(\frac{\cos x + \sin x}{\cos x\sin x} - 1\right) = 0$

  If $\cos x - \sin x = 0 \Rightarrow \tan x = 1\Rightarrow x = n\pi + \frac{\pi}{4}$

  If $\frac{\cos x + \sin x}{\cos x\sin x} - 1= 0$

  $\Rightarrow \cos x + \sin x = \cos x\sin x$

  Squaring, we get $1 + \sin2x = \frac{1}{4}\sin^2x$

  $\Rightarrow \sin2x = 2\pm 2\sqrt{2}$

  However, $2 + 2\sqrt{2} > 1$ which is not possible.

  $\Rightarrow \sin 2x = 2 - 2\sqrt{2} = \sin\alpha$ (let)

  $x = \frac{n\pi}{2} + \frac{(-1)^n\alpha}{2}$
\item Given equation is $2\sin^2x - 5\sin x\cos x - 8\cos^2x = -2$

  Clearly, $\cos x \neq 0$ else $\sin^2x = -1$ which is not possible.

  Therefore, we can divide both sides by $\cos^2x$ which yields

  $2\tan^2x - 5\tan x -8 = -2\sec^2x$

  $\Rightarrow 4\tan^2x - 5\tan x - 6 = 0$

  $\Rightarrow (\tan x - 2)(4\tan x + 3) = 0$

  Thus, $x = n\pi + \tan^{-1}2, b\pi + \tan^{-1}\left(\frac{-3}{4}\right)$

\item Given equation is $(1 - \tan x)(1 + \sin2x) = 1 + \tan x$

  $\Rightarrow (1 - \tan x)\left(1 + \frac{2\tan x}{1 + \tan^2x}\right) = 1 + \tan x$

  $\Rightarrow (1 - \tan x)(1 + \tan x)^2 = (1 + \tan x)(1 + \tan^2x)$

  $\Rightarrow (1 + \tan x)[(1 - \tan x)(1 + \tan x) - (1 + \tan^2x)] = 0$

  $\Rightarrow (1 + \tan x)(-2\tan^2x) = 0$

  If $\tan^2x = 0 \Rightarrow \tan x = 0\Rightarrow x = n\pi$

  If $1 + \tan x = 0 \Rightarrow x = n\pi - \frac{\pi}{4}$

  where $n \in I$

\item Given equation is $2(\cos x + \cos2x) + \sin2x(1 + 2\cos x) = 2\sin x$

  $\Rightarrow 4\cos\frac{3x}{2}\cos\frac{x}{2} + 2\sin\frac{5x}{2}\cos\frac{x}{2} - 2\sin\frac{x}{2}\cos\frac{x}{2} = 0$

  $\Rightarrow 2\cos\frac{x}{2}\left[2\cos\frac{3x}{2} + \sin\frac{5x}{2} - \sin\frac{x}{2}\right] = 0$

  $\Rightarrow 2\cos\frac{x}{2}\left[2\cos\frac{3x}{2} + 2\cos\frac{3x}{2}\sin x\right] = 0$

  $\Rightarrow 4\cos\frac{x}{2}\cos\frac{3x}{2}[1 + \sin x] = 0$

  If $\cos\frac{x}{2} = 0 \Rightarrow x = (2n + 1)\pi$

  If $\cos\frac{3x}{2} = 0 \Rightarrow x = (2n + 1)\frac{\pi}{3}$

  If $1 + \sin x = 0 \Rightarrow x = n\pi + (-1)^{n + 1}\frac{\pi}{2}$

  So the values in the given range are $x = -\pi, -\frac{\pi}{2}, -\frac{\pi}{3}, \frac{\pi}{3}, \pi$

\item Given equation is $4\cos^2x\sin x - 2\sin^2x = 3\sin x$

  $\Rightarrow \sin x[4\cos^2x - 2\sin x - 3] = 0$

  $\Rightarrow \sin x[4 - 4\sin^2x - 2\sin x - 3] = 0$

  $\Rightarrow \sin x[4\sin^2x + 2\sin x - 1] = 0$

  If $\sin x = 0 \Rightarrow x = n\pi$

  If $4\sin^2x + 2\sin x - 1 = 0$

  $\sin x = \frac{-1\pm\sqrt{5}}{4}$

  If $\sin x = \frac{-1 + \sqrt{5}}{4} \Rightarrow \sin x = \sin\frac{\pi}{10} \Rightarrow x = n\pi + (-1)^n\frac{\pi}{10}$

  If $\sin x = \frac{-1 - \sqrt{5}}{4}\Rightarrow \sin x = -\sin54^\circ = \sin\left(\frac{-3\pi}{10}\right)$

  $\Rightarrow x = n\pi + (-1)^{n + 1}\frac{3\pi}{10}$

\item Given equation is $2 + 7\tan^2x = 3.25\sec^2x$

  $\Rightarrow 8 + 28\tan^2x = 13\sec^2x = 13 + 13\tan^2x$

  $\Rightarrow 15\tan^2x = 5 \Rightarrow \tan x = \pm \frac{1}{\sqrt{3}}$

  $\Rightarrow x = n\pi \pm \frac{\pi}{6}$

\item Given equation is $\cos 2x + \cos 4x = 2\cos x$

  $\Rightarrow \cos4x + \cos2x - 2\cos x = 0$

  $\Rightarrow 2\cos3x\cos x - \cos x = 0$

  $2\cos x[\cos 3x - 1] = 0$

  If $\cos x = 0 \Rightarrow x = (2n + 1)\frac{\pi}{2}$

  If $\cos 3x - 1 = 0\Rightarrow 3x = 2n\pi \Rightarrow x = \frac{2n\pi}{3}$

\item Given equation is $3\tan x + \cot x = 5\csc x$

  $\Rightarrow \frac{3\sin x}{\cos x} + \frac{\cos x}{\sin x} = \frac{5}{\sin x}$

  $\Rightarrow \sin x(3\sin^2x + \cos^2x) = 5\sin x\cos x$

  $\Rightarrow \sin x(2\sin^2x - 5\cos x + 1) = 0$

  $\Rightarrow \sin x(2\cos^2x + 5\cos x - 3) = 0$

  $\Rightarrow \sin x(2\cos x + 3)(2\cos x - 1) = 0$

  $\sin x\neq 0$ because that will make $\csc x$ and $\cot x \infty.$

  $2\cos x + 3 \neq 0$ because $-1\leq \cos x\leq 1$

  $\therefore 2\cos x - 1 = 0\Rightarrow \cos x = \frac{1}{2} \Rightarrow x = 2n\pi \pm \frac{\pi}{3}$

\item Given equation is $2\sin^2x = 3\cos x$

  $\Rightarrow 2\cos^2x + 3\cos x - 2 = 0$

  $\Rightarrow (2\cos x - 1)(\cos x + 2) = 0$

  $\cos x \neq 2 \because -1\leq \cos x\leq 1$

  $\therefore 2\cos x - 1 = 0\Rightarrow x = 2n\pi \pm \frac{\pi}{3}~\forall~n\in I$

  $0\leq x \leq 2\pi \therefore x = \frac{\pi}{3}, \frac{5\pi}{3}$

\item Given equation is $\sin^2x - \cos x = \frac{1}{4}$

  $\Rightarrow 4\sin^2x - 4\cos x = 1 \Rightarrow 4 - 4\cos^2x - 4\cos x = 1$

  $\Rightarrow 4\cos^2x + 4\cos x -3 = 0$

  $\Rightarrow (2\cos x + 3)(2\cos x - 1) = 0$

  $\cos x \neq 2 \because -1\leq \cos x\leq 1$

  $\therefore 2\cos x - 1 = 0\Rightarrow x = 2n\pi \pm \frac{\pi}{3}~\forall~n\in I$

  $0\leq x \leq 2\pi \therefore x = \frac{\pi}{3}, \frac{5\pi}{3}$

\item Given equation is $3\tan^2x - 2\sin x = 0$

  $\Rightarrow 3\sin^2x - 2\sin x\cos^2x = 0$

  $\Rightarrow 3\sin^2x - 2\sin x + 2\sin^3x = 0$

  $\Rightarrow \sin x(2\sin^2x + 3\sin x - 2) = 0$

  $\Rightarrow \sin x(\sin x + 2)(2\sin x - 1) = 0$

  $\sin x\neq -2 \because -1\leq \sin x\leq 1$

  If $\sin x = 0 \Rightarrow x = n\pi$

  If $2\sin x - 1 = 0 \Rightarrow x = n\pi + (-1)^n\frac{\pi}{6}$

\item Given equation is $\sin x + \sin5x = \sin 3x$

  $\Rightarrow \sin5x - \sin3x + \sin x = 0$

  $\Rightarrow 2\cos4x\sin x + \sin x = 0$

  $\sin x(2\cos 4x + 1) = 0$

  If $\sin x = 0\Rightarrow x = n\pi~\forall~x\in I$

  If $2\cos4x + 1 = 0 \Rightarrow 4x = 2n\pi \pm\frac{2\pi}{3}$

  $x = \frac{n\pi}{2}\pm \frac{\pi}{6}~\forall~x\in I$

  Thus, $x = 0, \pi$ and $x = \frac{\pi}{6}, \frac{\pi}{3}, \frac{2\pi}{3}, \frac{5\pi}{6}$

\item Given equation is $\sin6x = \sin4x - \sin2x$

  $\Rightarrow \sin6x + \sin2x - \sin4x = 0$

  $\Rightarrow 2\sin4x\cos2x - \sin4x = 0$

  $\Rightarrow \sin4x(2\cos2x - 1) = 0$

  If $\sin4x = 0 \Rightarrow x = \frac{n\pi}{4}$

  If $2\cos2x - 1 = 0 \Rightarrow \cos2x = \frac{1}{2}\Rightarrow 2x = 2n\pi \pm \frac{\pi}{3}$

  $\Rightarrow x = n\pi \pm \frac{\pi}{6}$

\item Given equation is $\cos6x + \cos 4x + \cos 2x + 1 = 0$

  $\Rightarrow 2\cos5x\cos x + 2\cos^2x = 0$

  $\Rightarrow 2\cos x(\cos5x + \cos x) = 0$

  $\Rightarrow 4\cos x\cos2x\cos3x = 0$

  If $\cos x = 0 \Rightarrow x = 2n\pi \pm\frac{\pi}{2}$

  If $\cos2x = 0 \Rightarrow x = n\pi \pm \frac{\pi}{4}$

  If $\cos3x = 0 \Rightarrow x = \frac{2n\pi}{3}\pm\frac{\pi}{6}$

\item Given equation is $\cos x + \cos 2x + \cos 3x = 0$

  $\Rightarrow (\cos x + \cos3x) + \cos 2x = 0$

  $\Rightarrow 2\cos2x\cos x + \cos 2x = 0$

  $\Rightarrow \cos2x(2\cos x + 1) = 0$

  If $\cos 2x = 0 \Rightarrow x = (2n + 1)\frac{\pi}{4}$

  If $2\cos x + 1 = 0 \Rightarrow x = 2n\pi\pm\frac{2\pi}{3}$

\item Given equation is $\cos3x + \cos2x = \sin\frac{3x}{2} +  \sin\frac{x}{2}$

  $\Rightarrow 2\cos\frac{5x}{2}\cos\frac{x}{2} - 2\sin x\cos\frac{x}{2} = 0$

  $\Rightarrow 2\cos\frac{x}{2}\left[\cos\frac{5x}{2} - \sin x\right] = 0$

  If $\cos\frac{x}{2} = 0 \Rightarrow \frac{x}{2} = \left(n + \frac{1}{2}\right)\pi$

  $x = (2n + 1)\pi$

  If $\cos\frac{5x}{2} = \sin x = \cos\left(\frac{\pi}{2} - x\right)$

  $\Rightarrow \frac{5x}{2} = 2n\pi \pm \left(\frac{\pi}{2} - x\right)$

  $\Rightarrow x = (4n + 1)\pi/7, (4n - 1)\pi/3$

  Thus, between $0$ and $2\pi$ the values of $x$ are $\frac{\pi}{7}, \frac{5\pi}{7}, \pi, \frac{9\pi}{7},
  \frac{13\pi}{7}.$

\item Given equation is $\tan x+ \tan2x + \tan3x = \tan x.\tan2x.\tan3x$

  $\Rightarrow \tan x + \tan2x = \tan3x(\tan x\tan2x - 1)$

  $\Rightarrow \frac{\tan x + \tan 2x}{1 - \tan x\tan2x} = -\tan3x$

  $\Rightarrow \tan(x + 2x) = -\tan3x \Rightarrow 2\tan 3x = 0$

  $\Rightarrow 3x = n\pi \Rightarrow x = \frac{n\pi}{3}$

\item Given equation is $\tan x + \tan 2x + \tan x\tan 2x = 1$

  $\Rightarrow \tan x + \tan2x = 1 - \tan x\tan 2x$

  $\Rightarrow \frac{\tan x + \tan2x}{1 - \tan x\tan2x} = 1$

  $\Rightarrow \tan 3x = \tan\frac{\pi}{4}$

  $3x = n\pi + \frac{\pi}{4} \Rightarrow x = (4n + 1)\frac{\pi}{12}$

\item Given equation is $\sin2x + \cos2x + \sin x + \cos x + 1 = 0$

  $\Rightarrow 2\sin x\cos x + 2\cos^2x - 1 + \sin x + \cos x + 1 = 0$

  $\Rightarrow \sin x(2\cos x + 1) + \cos x(2\cos x + 1) = 0$

  $\Rightarrow (2\cos x + 1)(\sin x + \cos x) = 0$

  If $\cos x = -\frac{1}{2} \Rightarrow x = 2n\pi\pm\frac{2\pi}{3}$

  If $\sin x + \cos x = 0 \Rightarrow \tan x = -1 \Rightarrow x = n\pi - \frac{\pi}{4}$

\item We have to prove that $\sin x + \sin 2x + \sin 3x = \cos x + \cos 2x + \cos 3x$

  $\Rightarrow (\sin x + \sin 3x) + \sin 2x = (\cos x + \cos3x) + \cos 2x$

  $\Rightarrow 2\sin2x\cos x + \sin 2x = 2\cos2x\cos x + \cos 2x$

  $\Rightarrow \sin2x(2\cos x + 1) = \cos2x(2\cos x + 1)$

  $\Rightarrow (2\cos x + 1)(\sin2x - \cos2x) = 0$

  If $2\cos x + 1 = 0 \Rightarrow x = 2n\pi \pm \frac{2\pi}{3}$

  If $\sin 2x - \cos2x = 0 \Rightarrow \tan2x = 1 = \tan\frac{\pi}{4} \Rightarrow x = \frac{n\pi}{2}+\frac{\pi}{8}$

\item Given equation is $\cos6x + \cos4x = \sin3x + \sin x$

  $\Rightarrow 2\cos5x\cos x = 2\sin2x\cos x$

  $\Rightarrow \cos x(\cos5x - \sin2x) = 0$

  If $x = 0 \Rightarrow x = 2n\pi\pm\frac{\pi}{2}$

  If $\cos5x = \sin2x \Rightarrow \cos5x = \cos\left(\frac{\pi}{2} - 2x\right)$

  $\Rightarrow 5x = 2n\pi\pm \left(\frac{\pi}{2} - 2x\right)$

  Taking +ve sign $7x = 2n\pi + \frac{\pi}{2} \Rightarrow x = (4n + 1)\frac{\pi}{14}$

  Taling -ve sign $3x = 2n\pi - \frac{\pi}{2} \Rightarrow x = (4n - 1)\frac{\pi}{6}$

\item Given equation is $\sec4x - \sec2x = 2$

  $\Rightarrow \cos2x - \cos4x = 2\cos2x\cos4x$ where $\cos2x, \cos4x\neq 0$

  $\Rightarrow \cos2x - \cos4x = \cos6x + \cos 2x$

  $\Rightarrow \cos6x + \cos4x = 0 \Rightarrow 2\cos5x\cos x = 0$

  If $\cos 5x = 0 \Rightarrow 5x = 2n\pi\pm\frac{\pi}{2}\Rightarrow x = \frac{2n\pi}{5}\pm\frac{\pi}{10}$

  If $\cos x = 0 \Rightarrow x = 2n\pi\pm\frac{\pi}{2}$

\item Given equation is $\cos2x = (\sqrt{2} + 1)\left(\cos x - \frac{1}{\sqrt{2}}\right)$

  $\Rightarrow (2\cos^2x - 1) = \frac{\sqrt{2} + 1}{\sqrt{2}}\left(\sqrt{2}\cos x - 1\right)$

  $\Rightarrow (\sqrt{2}\cos x - 1)\left(\sqrt{2}\cos x + 1 - 1 - \frac{1}{\sqrt{2}}\right) = 0$

  $\Rightarrow (\sqrt{2}\cos x - 1)(2\cos x - 1) = 0$

  If $\sqrt{2}\cos x - 1 = 0 \Rightarrow x = 2n\pi \pm \frac{\pi}{4}$

  If $2\cos x - 1 = 0 \Rightarrow x = 2n\pi\pm\frac{\pi}{3}$

\item Given equation is $5\cos2x + 2\cos^2\frac{x}{2} + 1 = 0$

  $\Rightarrow 10\cos^2x - 5 + \cos x + 2 = 0[\because \cos2x = 2\cos^2x - 1]$

  $\Rightarrow 10\cos^2x + \cos x - 3 = 0$

  $\Rightarrow (2\cos x -1)(5\cos x + 3) = 0$

  If $\cos x = 1/2 \Rightarrow x = \frac{\pi}{3} [\because -\pi \leq x\leq \pi]$

  If $5\cos x + 3 = 0 \Rightarrow x = \pi - \cos^{-1}\frac{3}{5}$

\item Given equation is $\cot x - \tan x = sec x$

  $\Rightarrow \cos x(\cos^2x - \sin^2x) = \sin x\cos x$

  $\Rightarrow \cos x(2\sin^2x + \sin x - 1) = 0$

  $\Rightarrow \cos x(2\sin x - 1)(\sin x + 1) = 0$

  $\cos x\neq 0$ and $\sin \neq -1$ because that will render original equation meaningless.

  $\therefore 2\sin x - 1= 0 \Rightarrow x = n\pi + (-1)^n\frac{\pi}{6}$

\item Given equation is $1 + \sec x = \cot^2\frac{x}{2}$

  $\Rightarrow \frac{1 + \cos x}{\cos x} = \frac{\cos^2\frac{x}{2}}{\sin^2\frac{x}{2}}$

  $\Rightarrow 2\sin^2\frac{x}{2}\cos^2\frac{x}{2} = \cos x\cos^2\frac{x}{2}$

  $\Rightarrow \cos^2\frac{x}{2}\left(2\sin^2\frac{x}{2} - \cos x\right) = 0$

  $\Rightarrow \cos^2\frac{x}{2}\left(1 - 2\cos x\right) = 0$

  If $\cos\frac{x}{2} = 0 \Rightarrow \frac{x}{2} = n\pi + \frac{pi}{2} \Rightarrow x = (2n + 1)\pi$

  If $1 - 2\cos x =  0\Rightarrow x = 2n\pi \pm \frac{\pi}{3}$

\item Given equation is $\cos3x\cos^3x + \sin3x\sin^3x = 0$

  $\Rightarrow (4\cos^3x - 3\cos x)\cos^3x + (3\sin x - 4\sin^3x)\sin^3x = 0$

  $\Rightarrow 3(\sin^4x - \cos^4x) - 4(\sin^6x - \cos^6x) = 0$

  $\Rightarrow 3(\sin^2x - \cos^2x) - 4(\sin^2x - \cos^2x)(\sin^4x + \cos^4x + \sin^2x\cos^2x) = 0$

  $\Rightarrow \cos2x[-3 + 4\{\sin^2x(\sin^2x + \cos^2x) + \cos^4x\}] = 0$

  $\Rightarrow \cos 2x[4\cos^4x - 4\cos^2x + 1] = 0$

  $\Rightarrow \cos2x(2\cos^2x - 1)^2 = 0$

  $\Rightarrow \cos^32x = 0$

  $\Rightarrow \cos 2x = 0$

  $2x = n\pi + \frac{\pi}{2} \Rightarrow x = (2n + 1)\frac{\pi}{4}$

\item Given equation is $\sin^3x + \sin x\cos x + \cos^3x = 1$

  $\Rightarrow \sin^3x + \cos^3x + \sin x\cos x - 1 = 0$

  $\Rightarrow (\sin x + \cos x)(\sin^2x - \sin x\cos x + \cos^2x) + (\sin x\cos x - 1) = 0$

  $\Rightarrow (1 - \sin x\cos x)(\sin x + \cos x - 1) = 0$

  If $1 - \sin x\cos x = 0 \Rightarrow \sin 2x = 2$ which is not possible.

  $\therefore \sin x + \cos x - 1 = 0 \Rightarrow \frac{1}{\sqrt{2}}\sin x + \frac{1}{\sqrt{2}}\cos x = \frac{1}{\sqrt{2}}$

  $\Rightarrow \cos\left(x - \frac{\pi}{4}\right) = \cos\frac{\pi}{4}$

  $\Rightarrow x - \pi/4 = 2n\pi \pm \pi/4 = 2n\pi, 2n\pi + \pi/2$

\item Given equation is $\sin 7x + \sin4x + \sin x = 0$

  $\Rightarrow 2\sin4x\cos3x + \sin4x = 0$

  $\Rightarrow \sin4x(2\cos3x + 1) = 0$

  If $\sin4x = 0 \Rightarrow x = n\pi/4 \Rightarrow x = \pi/4~\forall 0\leq x\leq\pi/2$

  If $\cos3x = -1/2 \Rightarrow x = \frac{2\pi}{9}, \frac{4\pi}{9}~\forall 0\leq x\leq\pi/2$

\item Given equation is $\sin x + \sqrt{3}\cos x = \sqrt{2}$

  Dividing both sides by $2$ [we arrive at this no. by squaring and adding coefficients of $\sin x$ and $\cos x$
    and then taking square root]

  $\Rightarrow \frac{1}{2}\sin x + \frac{\sqrt{3}}{2}\cos x = \frac{1}{\sqrt{2}}$

  $\Rightarrow \sin\frac{\pi}{6}\sin x + \cos\frac{\pi}{6}\cos x = \cos\frac{\pi}{4}$

  $\Rightarrow \cos\left(x - \frac{\pi}{6}\right) = \cos\frac{\pi}{4}$

  $\Rightarrow x - \frac{\pi}{6} = 2n\pi\pm\frac{\pi}{4}$

  $\Rightarrow x = 2n\pi\pm\frac{5\pi}{12}, 2n\pi-\frac{\pi}{12}$

\item We have to find minimum value of $27^{\cos2x}.81^{\sin2x}$

  $27^{\cos2x}.81^{\sin2x} = 3^{3\cos2x + 4\sin2x}$

  It will be minimum when $3\cos2x + 4\sin2x$ will be minimum.

  Dividing and multiplying with $5,$ we get

  $5\left(\frac{3}{5}\cos2x + \frac{4}{5}\sin2x\right)$

  $\Rightarrow 5\cos(2x - y)$ where $\tan y = \frac{4}{3}$

  For minimum value $\cos(2x - y) = -1 = \cos\pi \Rightarrow 2x - y = 2n\pi\pm\pi$

  $x = \frac{2n\pi\pm\pi + \tan^{-1}\frac{4}{3}}{2}, n \in I$

  Minimum value will be $3^{-5} = \frac{1}{243}$

\item Given $3\cos 2x = 1 \Rightarrow \cos2x = \frac{1}{3}$

  $\Rightarrow \tan^2x = \frac{1 - \cos2x}{1 + \cos2x} = \frac{1}{2}$

  Given $32\tan^8x = 2\cos^2y - 3\cos y$

  $\Rightarrow 32.\frac{1}{2^4} = 2\cos^2y - 3\cos y$

  $\Rightarrow 2\cos^2y - 3\cos y - 2 = 0$

  $(2\cos y + 1)(\cos y - 2) = 0$

  $\because \cos y \neq 2 \Rightarrow 2\cos y + 1 = 0$

  $\Rightarrow y = 2n\pi\pm\frac{2\pi}{3}$

\item Given equation is $(1 - \tan x)(1 + \tan x)sec^2x + 2^{\tan^2x} = 0$

  $\Rightarrow (1 - \tan^2x)(1 + \tan^2x) + 2^{\tan^2x} = 0$

  $\Rightarrow 1 + 2^{\tan^2x} = \tan^4x$

  Clearly, $\tan^2x = 3$ is the solution of the above equation.

  $\Rightarrow \tan x = \pm\sqrt{3} \Rightarrow x = n\pi\pm\frac{\pi}{3}$

  Values of $x$ in the given interval are $\pm\frac{\pi}{3}.$

\item Given equation is $e^{\cos x} = e^{-\cos x} + 4.$

  $\Rightarrow e^{2\cos x} - 4e^{\cos x} - 1= 0$

  $\Rightarrow e^{\cos x} = 2\pm\sqrt{5}$

  If $e^{\cos x} = 2 + \sqrt{5}$ then $\cos x > 1$ which is not possible.

  If $e^{\cos x} = 2 - \sqrt{5}$ then $\cos x$ is an imaginary number. Thus, no solutions for given equation are
  possible.

\item Given equation is $(1 + \tan x)(1 + \tan y) = 2$

  $\Rightarrow 1 + \tan x + \tan y + \tan x\tan y = 2$

  $\Rightarrow \tan x + \tan y = 1 - \tan x\tan y$

  $\Rightarrow \frac{\tan x + \tan y}{1 - \tan x\tan y} = 1$

  $\Rightarrow \tan(x + y) = \tan\frac{\pi}{4}$

  $\Rightarrow x + y = n\pi\pm\frac{\pi}{4}$

\item Given equation is $\tan(\cot x) = \cot(\tan x)$

  $\Rightarrow \tan(\cot x) = \tan\left(\frac{\pi}{2} - \tan x\right)$

  $\Rightarrow \cot x = n\pi + \left(\frac{\pi}{2} - \tan x\right)$

  $\Rightarrow \tan x + \cot x = n\pi + \frac{\pi}{2}$

  $\Rightarrow \frac{\sin^2x + \cos^2x}{\sin x + \cos x} = (2n + 1)\pi/2$

  $\Rightarrow \frac{1}{\sin2x} = (2n + 1)\pi/4 \Rightarrow \sin2x = \frac{4}{(2n + 1)\pi}$

\item We have $a\tan z+ b\sec z = c$

  $\Rightarrow c - a\tan z = b\sec z \Rightarrow (c - a\tan z)^2 = b^2sec^2z$

  $\Rightarrow (a^2 - b^2)\tan^2x -2ac\tan z + (c^2 - b^2) = 0$

  Given that $x$ and $y$ are roots of original equation so $\tan x$ and $\tan y$ will be roots of last
  equation.

  $\Rightarrow \tan x + \tan y = \frac{2ac}{a^2 - b^2}$ and $\tan x\tan y = \frac{c^2 - b^2}{a^2 - b^2}$

  $\Rightarrow \tan(x + y) = \frac{\tan x + \tan y}{1 - \tan x\tan y} = \frac{2ac}{a^2 - c^2}$

\item Given $\sin(\pi\cos x) = \cos(\pi\sin x)$

  $\Rightarrow \pi\cos x = \frac{\pi}{2} - \pi\sin x$

  $\Rightarrow \sin x + \cos x = \frac{1}{2}$

  1. Fividing both sides by $\sqrt{2}$

  $\Rightarrow \frac{\sin x}{\sqrt{2}} + \frac{\cos x}{\sqrt{2}} = \frac{1}{2\sqrt{2}}$

  $\cos\left(x \pm \frac{\pi}{4}\right) = \frac{1}{2\sqrt{2}}$

  2. Squaring both sides

  $\Rightarrow \sin^2x + \cos^2x + 2\sin x\cos x = \frac{1}{4}$

  $\Rightarrow \sin 2x = -\frac{3}{4}$

\item Given equation is $\tan(x + 100^\circ) = \tan(x + 50^\circ).\tan x.\tan(x - 50^\circ)$

  $\Rightarrow \frac{\tan(x+100^\circ)}{\tan x} = \tan(x+50^\circ)\tan(x-50^\circ)$

  $\Rightarrow \frac{\sin(x+100^\circ)\cos x}{\cos(x + 100^\circ)\sin x} =
  \frac{\sin(x+50^\circ)\sin(x-50^\circ)}{\cos(x+50^\circ)\cos(x-50^\circ)}$

  Appplying componendo and dividendo

  $\Rightarrow \frac{\sin(x + 100^\circ)\cos x + \cos(x + 100^\circ)\sin x}{\sin(x + 100^\circ)\cos x - \cos(x +
    100^\circ)\sin x} = \frac{\sin(x+50^\circ)\sin(x-50^\circ) + \cos(x+50^\circ)\cos(x-50^\circ)}{\sin(x+50^\circ)\sin(x-50^\circ)
    -\cos(x+50^\circ)\cos(x-50^\circ)}$

  $\Rightarrow \frac{\sin 2x + 100^\circ}{\sin100^\circ} = \frac{\cos100^\circ}{-\cos 2x}$

  $\Rightarrow -2\sin(2x + 100^\circ)\cos2x = 2\sin100^\circ\cos100^\circ$

  $\Rightarrow -\sin(4x + 100^\circ) - \sin100^\circ = \sin200^\circ$

  $\Rightarrow \sin(4x + 100^\circ) = -2\sin1506\circ.\cos50^\circ = -\cos50^\circ = \sin220^\circ$

  Thus, minimum value of $x$ is $30^\circ.$

\item We have to find $x$ for which $\tan^2x + \sec 2x = 1$

  $\Rightarrow \tan^2x + \frac{1 + \tan2x}{1 - \tan^2x} = 1$

  $\Rightarrow \tan^2x - \tan^4x + 1 + \tan^2x = 1 - \tan^2x$

  $\Rightarrow \tan^4x - 3\tan^2x = 0$

  $\Rightarrow \tan^2x(\tan^2x - 3) = 0$

  If $\tan^2x = 0 \Rightarrow x = n\pi$

  If $\tan^2x = 3\Rightarrow x = n\pi\pm\frac{\pi}{3}$

  Clearly, for all these values $\tan x$ and $\sec 2x$ are defined.

\item Given equation is $\sec x - \csc x = \frac{4}{3}$

  $\Rightarrow \frac{1}{\cos x}- \frac{1}{\sin x} = \frac{4}{3}$

  $\Rightarrow 3(\sin x - \cos x) = 4\sin x\cos x$

  Squaring both sides

  $\Rightarrow 9(\sin^2x + \cos^2x - 2\sin x\cos x) = 16\sin^2x\cos^2x$

  $\Rightarrow 9(1 - \sin2x) = 4\sin^22x$

  $\Rightarrow 4\sin^22x + 9\sin2x - 9 = 0$

  $\Rightarrow (\sin2x + 3)(4\sin2x - 3) = 0$

  $\sin2x \neq 3\therefore 4\sin2x = 3 \Rightarrow x = \frac{n\pi}{2} + (-1)^n/2.\sin^{-1}\frac{3}{4}$

\item Given equation is $\sin2x - 12(\sin x - \cos x) + 12 = 0.$

  $\Rightarrow 1 - \sin2x + 12(\sin x - \cos x) - 13 = 0$

  $\Rightarrow (\sin x - \cos x)^2 + 12(\sin x - \cos x) - 13 = 0$

  $\Rightarrow (\sin x - \cos x - 1)(\sin  x - \cos x + 13) = 0$

  Clearly, $\sin x - \cos x + 13 \neq 0$

  $\therefore \sin x - \cos x = 1$

  $\Rightarrow \frac{1}{\sqrt{2}}\sin x - \frac{1}{\sqrt{2}}\cos x = \frac{1}{\sqrt{2}}$

  $\Rightarrow \sin\left(x - \frac{\pi}{4}\right) = \sin\frac{\pi}{4}$

  $\Rightarrow x = n\pi + (-1)^n\frac{\pi}{4} + \frac{\pi}{4}$

\item Given equation is $\cos(p\sin x) = \sin(p\cos x)$

  $\Rightarrow p\sin x = 2n\pi \pm \left(\frac{\pi}{2} - p\cos x\right)$

  Taking positive sign

  $p\sin x = 2n\pi + \frac{\pi}{2} - p\cos x$

  $\Rightarrow p(\sin x + \cos x) = (4n + 1)\frac{\pi}{2}$

  $\Rightarrow \sin x + \cos x = \frac{(4n + 1)\pi}{2p}$

  $\Rightarrow \frac{1}{\sqrt{2}}\sin x + \frac{1}{\sqrt{2}}\cos x = \frac{(4n + 1)\pi}{2\sqrt{2}p}$

  $\Rightarrow \sin(x + \pi/4) = \frac{(4n + 1)p}{2\sqrt{2}p}$

  Clealry, $\left|\frac{(4n + 1)\pi}{2\sqrt{2}p}\right|\leq 1$

  $\Rightarrow p \geq \frac{(4n + 1)\pi}{2\sqrt{2}}$

  For smallest positive value $n = 0 \therefore p = \frac{\pi}{2\sqrt{2}}$

  Taking negative sing

  $p\sin x = 2n\pi - \frac{\pi}{2} + p\cos x$

  $\Rightarrow p(\cos x - \sin x) = -2n\pi + \frac{\pi}{2}$

  Proceeding similarly, $p \geq \frac{(-4n + 1)\pi}{2\sqrt{2}}$

  Smallest positive value of $p = \frac{\pi}{2\sqrt{2}}$

\item Given equation is $\cos x + \sqrt{3}\sin x = 2\cos2x$

  $\Rightarrow \frac{\cos x}{2} + \frac{\sqrt{3}\sin x}{2} = \cos 2x$

  $\Rightarrow \cos \left(x - \frac{\pi}{3}\right) = \cos 2x$

  $\Rightarrow x - \frac{\pi}{3} = 2n\pi \pm 2x$

  Taking positive sign, $x = -2n\pi - \frac{\pi}{3}$

  Taking negative sing $x = \frac{2n\pi}{3} + \frac{\pi}{9}$

\item Given equation is $\tan x+ \sec x = \sqrt{3}$

  $\Rightarrow \frac{\sin x}{\cos x} + \frac{1}{\cos x} = \sqrt{3}$

  $\Rightarrow \sqrt{3}\cos x - \sin x = 1$

  Dividing both sides by $2,$ we get

  $\frac{\sqrt{3}}{2}\cos x - \frac{1}{2}\sin x = \frac{1}{2}$

  $\Rightarrow \cos\left(x + \frac{\pi}{6}\right) = \cos \frac{\pi}{3}$

  $\Rightarrow x + \frac{\pi}{6} = 2n\pi \pm \frac{\pi}{3}$

  Taking positive sign

  $x = 2n\pi + \frac{\pi}{6}$

  Taking negative sign

  $x = (4n - 1)\frac{\pi}{2}$

  Values of $x$ between $0$ and $2\pi$ are $\frac{\pi}{6}, \frac{3\pi}{2}$

  However, when $x = \frac{3\pi}{2}, \cos x = 0$ which will be rejected.

  $\therefore x = \frac{\pi}{6}$

\item Given equation is $1 + \sin^3x + \cos^3x = \frac{3}{2}\sin2x$

  $\Rightarrow 1 + \sin^3x + \cos^3x = 3\sin x\cos x$

  $\Rightarrow 1 + \sin^3x + \cos^3x - 3\sin x\cos x = 0$

  $\Rightarrow (1 + \sin x+\cos x[(1 - \sin x)^2 + (\sin x - \cos x)^2 + (\cos x - 1)^2] = 0[\because a^3 + b^3 + c^3 -
  3abc = (a + b + c).\frac{1}{2}[(a - b)^2 + (b - c)^2 + (c - a)^3]]$

  If $1 + \sin x + \cos x = 0 \Rightarrow \cos x + \sin x = -1$

  $\Rightarrow \frac{1}{\sqrt{2}}\cos x + \frac{1}{\sqrt{2}}\sin x = \frac{-1}{\sqrt{2}}$

  $\Rightarrow \cos\left(x - \frac{\pi}{4}\right) = \cos\frac{3\pi}{4}$

  $\Rightarrow x - \frac{\pi}{4} = 2n\pi \pm\frac{3\pi}{4} \Rightarrow x = 2n\pi + \frac{\pi}{4}\pm \frac{3\pi}{4}$

  else $\sin x = 1, \sin x = \cos x, \cos x = 1$ which is not possible.

\item Given equation is $(2 + \sqrt{3})\cos x = 1 - \sin x$

  $\Rightarrow \frac{1 - \sin x}{\cos x} = 2 + \sqrt{3}$

  $\Rightarrow \frac{1 - \sin x}{\cos x}.\frac{1 + \sin x}{1 + \sin x} = (2 + \sqrt{3}).\frac{2 - \sqrt{3}}{2 - \sqrt{3}}$

  $\Rightarrow \frac{\cos x}{1 + \sin x} = \frac{1}{2 - \sqrt{3}}$ [note that we have cancelled $\cos x$ here]

  $\Rightarrow \frac{1 + \sin x}{\cos x} = 2 - \sqrt{3}$

  $\Rightarrow \frac{1 + \sin x + 1 - \sin x}{\cos x} = 2 + \sqrt{3} + 2 - \sqrt{3}$

  $\Rightarrow \frac{2}{\cos x} = 4 \Rightarrow \cos x = \frac{1}{2} \Rightarrow x = 2n\pi \pm \frac{\pi}{3}$

  Since we have cancelled $\cos x$ one of the possible solutions is $\cos x = 0 \Rightarrow x = 2n\pi + \frac{\pi}{2}$

\item Given equation is $\tan\left(\frac{\pi}{2}\sin x\right) = \cot\left(\frac{\pi}{2}\cos x\right)$

  $\Rightarrow \frac{\pi}{2}\sin x = \frac{\pi}{2} - \frac{\pi}{2}\cos x$

  $\Rightarrow \sin x = 1 - \cos x$

  $\Rightarrow \sin x + \cos x = 1$

  $\Rightarrow \frac{1}{\sqrt{2}}\sin x + \frac{1}{\sqrt{2}}\cos x = \frac{1}{\sqrt{2}}$

  $\Rightarrow \cos(x - \frac{\pi}{4}) = \cos \frac{\pi}{4}$

  $\Rightarrow x - \frac{\pi}{4} = 2n\pi \pm \frac{\pi}{4}$

  Taking positive sign $x = 2n\pi + \pi/2$

  Taking negative sign $x = 2n\pi$

\item Given equation is $8\cos x\cos2x\cos4x = \frac{\sin6x}{\sin x}$

  $\Rightarrow 8\sin x\cos x\cos2x\cos4x = \sin6x$

  $\Rightarrow 4\sin2x\cos2x\cos4x = \sin6x$

  $\Rightarrow 2\sin4x\cos4x = \sin6x \Rightarrow \sin8x = \sin6x$

  $\Rightarrow 2\cos7x\sin x = 0$

  If $\cos7x = 0 \Rightarrow 7x = (2n + 1)\frac{\pi}{2}\Rightarrow x = (2n + 1)\frac{\pi}{14}$

  $\sin x$ cannot be zeor as it is in denominator.

\item Given equation is $3 - 2\cos x - 4\sin x -\cos 2x + \sin 2x = 0$

  $\Rightarrow 3 - 2\cos x - 4\sin x -(1 - 2\sin^2) + 2\sin x\cos x = 0$

  $\Rightarrow 2(\sin^2x - 2\sin x + 1) + 2\cos x(\sin x -1) = 0$

  $\Rightarrow (\sin x- 1)(2\cos x + 2\sin x - 2) = 0$

  If $\sin x - 1 = 0 \Rightarrow x = n\pi + (-1)^n\frac{\pi}{2}$

  If $\sin x + \cos x = 1$

  Like previous examples $x = 2n\pi, 2n\pi + \frac{\pi}{2}$

\item Given equation is $\sin x - 3\sin 2x + \sin 3x = \cos x - 3\cos 2x + \cos 3x$

  $\Rightarrow 2\sin2x\cos x - 3\sin 2x = 2\cos2x\cos x - 3\cos 2x$

  $\Rightarrow \sin2x(2\cos x - 3) = \cos2x(2\cos x - 3)$

  $\because 2\cos x \neq 3 \therefore \sin 2x = \cos 2x$

  $\Rightarrow \frac{1}{\sqrt{2}}\cos2x - \frac{1}{\sqrt{2}}\sin2x = 0$

  $\Rightarrow 2x + \pi/4 = n\pi \Rightarrow x = n\pi/2 + \pi/8$

\item Given equation is $\sin^2x\tan x + \cos^2x\cot x - \sin 2x = 1 + \tan x + \cot x$

  $\Rightarrow \frac{\sin^3x}{\cos x} + \frac{\cos^3x}{\sin x} - \sin 2x = 1 + \frac{\sin x}{\cos x} + \frac{\cos x}{\sin
  x}$

  $\Rightarrow \frac{\sin^4x + \cos^4x}{\sin x\cos x} - \sin2x = \frac{\sin x\cos x + \sin^2x + \cos^2x}{\sin x\cos x}$

  $\Rightarrow \frac{1 - 2\sin^2x\cos^2x}{\sin x\cos x} - \sin2x = 1 + \frac{1}{\sin x\cos x}$

  $\Rightarrow -2\sin x\cos x - \sin 2x = 1$

  $\Rightarrow \sin 2x = -1/2 \Rightarrow 2x = n\pi + (-1)^{n + 1}\frac{\pi}{6}$

  $\Rightarrow x = \frac{n\pi}{2} + (-1)^{n + 1}\frac{\pi}{12}$

\item $\sin x = -\frac{1}{2} \Rightarrow x = \frac{7\pi}{6}, \frac{11\pi}{6}$

  $\tan x = \frac{1}{\sqrt{3}}\Rightarrow x = \frac{\pi}{6}, \frac{7\pi}{6}$

  So common value is $\frac{7\pi}{6}$

  Period of $\sin x$ is $2\pi$ and period of $\tan x$ is $n\pi$ so common period is $2n\pi$

  Thus, most general value of $x$ is $2n\pi +\frac{7\pi}{6}$

\item $\tan(x - y) = 1 \Rightarrow x - y = \frac{\pi}{4}, \frac{5\pi}{4}$

  $\sec(x + y) = \frac{2}{\sqrt{3}}\Rightarrow x + y = \frac{\pi}{6}, \frac{11\pi}{6}$

  Since $x$ and $y$ are positve so $x + y > x - y$

  When $x - y = \frac{\pi}{4}$ and $x + y = \frac{11\pi}{6}$

  $x = \frac{25\pi}{24} y = \frac{19\pi}{24}$

  When $x - y = \frac{5\pi}{5}$ and $x + y = \frac{11\pi}{6}$

  $x = \frac{37\pi}{24}, y = \frac{7\pi}{24}$

  General Solution:

  $\tan(x - y) = 1 \Rightarrow x - y = m\pi + \frac{\pi}{4}$

  $\sec(x + y) = \frac{2}{\sqrt{3}} \Rightarrow x + y = 2n\pi \pm \frac{\pi}{6}$

  $x = (2m + n)\frac{\pi}{2} + \frac{\pi}{8}\pm \frac{\pi}{12}$

  $y = (2m - n)\frac{\pi}{2} - \frac{pi}{8}\pm \frac{\pi}{12}$

\item Given curves are $y = \cos x$ and $y = \sin 3x$

  For intersection point both the equations must be satisfied, thus

  $\cos x = \sin 3x = \cos\left(\frac{\pi}{2} - 3x\right)$

  $\Rightarrow x = 2n\pi \pm \left(\frac{\pi}{2} - 3x\right)$

  $\Rightarrow x = \frac{n\pi}{2} + \frac{\pi}{8}, n\pi + \frac{\pi}{4}$

  So in the given interval values of $x$ are $\frac{\pi}{8}, -\frac{3\pi}{8}, \frac{\pi}{4}.$

\item From first equation $r\sin x = \sqrt{3} \Rightarrow r = \frac{\sqrt{3}}{\sin x}$

  Substituting this value in the second equagtion, we get

  $\frac{\sqrt{3}}{\sin x} + 4\sin x = 2(\sqrt{3} + 1)$

  $\Rightarrow 4\sin^2x - 2\sqrt{3}\sin x - 2\sin x + \sqrt{3} = 0$

  $\Rightarrow (2\sin x - \sqrt{3})(2\sin x - 1) = 0$

  If $2\sin x - \sqrt{3} = 0 \Rightarrow x = n\pi + (-1)^n\frac{\pi}{3}$

  If $2\sin x - 1 = 0 \Rightarrow x = n\pi + (-1)^n\frac{\pi}{6}$

  Thus, for $o\leq x\leq 2\pi, x = \frac{\pi}{6}, \frac{\pi}{3}, \frac{2\pi}{3}, \frac{5\pi}{6}$

\item Given $x - y = \frac{\pi}{4}$ and $\cot x + \cot y = 2$

  From second equation $\frac{\cos x}{\sin x} + \frac{\cos y}{\sin y} = 2$

  $\Rightarrow \sin(x + y) = 2\sin x\sin y = \cos(x - y) - \cos(x + y) = \cos \frac{\pi}{4} - \cos(x + y)$

  $\Rightarrow \sin(x + y) + \cos (x + y) = \cos\frac{\pi}{4}$

  $\Rightarrow \frac{1}{\sqrt{2}}\sin(x + y) + \frac{1}{\sqrt{2}}\cos(x + y) = \frac{1}{2} = \cos\frac{\pi}{3}$

  $\Rightarrow \cos(x + y - \pi/4) = \cos\frac{\pi}{3}$

  $\Rightarrow x + y - \frac{\pi}{4} = 2n\pi\pm \frac{\pi}{3}$

  $x + y = 2n\pi \pm \frac{\pi}{3} + \frac{\pi}{4}$

  For $n = 0, x + y = \frac{7\pi}{12}[\because x, y > 0]$

  $\Rightarrow x = \frac{5\pi}{12}, y = \frac{\pi}{6}$

\item Given equations are $5\sin x\cos y = 1$ and $4\tan x = \tan y$

  $\Rightarrow 4\frac{\sin x}{\cos x} = \frac{\sin y}{\cos y}$

  $\Rightarrow 4\sin x\cos y = \sin y \cos x \Rightarrow \frac{4}{5} = \sin y\cos x$

  Thus, $\sin x\cos y + \cos x\sin y = 1$

  $\Rightarrow \sin(x + y) = \sin\frac{\pi}{2}$

  $\Rightarrow x + y = n\pi + (-1)^n\frac{\pi}{2}$

  and $\sin x\cos y - \cos x\sin y = -\frac{3}{5}$

  $\Rightarrow \sin(x - y) = -\frac{3}{5}$

  $\Rightarrow x - y = n\pi + (-1)^k\sin^{-1}\frac{-3}{5}$

  $\therefore x = \frac{1}{2}\left[(n - k)\pi +(-1)^n\frac{\pi}{2} + (-1)^k\sin^{-1}\frac{-3}{5}\right]$

  and $y = \frac{1}{2}\left[(n - k)\pi +(-1)^n\frac{\pi}{2} - (-1)^k\sin^{-1}\frac{-3}{5}\right]$

\item Given equations are $r\sin x = 3$ and $r = 4(1 + \sin x)$

  $\Rightarrow r = \frac{3}{\sin x}$

  Substituting this in second equation

  $3 = 4\sin x + 4\sin^2x \Rightarrow 4\sin^2x + 4\sin x - 3 = 0$

  $\Rightarrow (2\sin x + 3)(2\sin x - 1) = 0$

  $\because 2\sin \neq -3 \therefore 2\sin x = 1$

  $x = n\pi + (-1)^n\frac{\pi}{6}$

  Thus, values of $x$ between $0$ and $2\pi$ are $\pi/6$ and $5\pi/6.$

\item Given $\sin x = \sin y$ and $\cos x = \cos y$

  Clearly, one of the solutions is $x = y$

  $x = n\pi + (-1)^ny$ and $x = 2n\pi\pm y$

  $\therefore x - y = 2n\pi$ is the only other solution.

\item Given equations are $\cos(x - y) = \frac{1}{2}$ and $\sin(x + y) = \frac{1}{2}$

  $x - y = \frac{\pi}{3}$

  $x + y = \frac{\pi}{6}, \frac{5\pi}{6}$

  For positive values of $x$ and $y, x + y > x - y$

  $\therefore x + y = \frac{5\pi}{6}$

  $2x = \frac{7\pi}{6}\Rightarrow x = \frac{7\pi}{12}$

  $\Rightarrow y = \frac{\pi}{4}$

  General values:

  $x - y = 2n\pi \pm \frac{\pi}{3}$

  $x + y = m\pi + (-1)^m\frac{pi}{6}$

  $\therefore x = (2n + m)\frac{\pi}{2}\pm\frac{\pi}{6} + (-1)^m\frac{\pi}{12}$

  and $y = (m - 2n)\frac{\pi}{2}\mp\frac{\pi}{6} + (-1)^m\frac{\pi}{12}$

\item Given curves are $y = \cos 2x$ and $y = \sin x$

  For them to intersect both equations must be satisfied together. Thus,

  $\cos2x = \sin x$

  $\Rightarrow 2\sin^2x + \sin x - 1 = 0$

  $\Rightarrow (2\sin x - 1)(\sin x + 1) = 0$

  If $2\sin x - 1 = 0 \Rightarrow x = \frac{\pi}{6} [\because -\frac{\pi}{2}\leq x\leq
    \frac{\pi}{2}]$

  $y = \frac{1}{2}.$ So the point is $\left(\frac{\pi}{6}, \frac{1}{2}\right)$

  If $\sin x = -1 \Rightarrow x = -\frac{\pi}{2}$

  So the point is $\left(-\frac{\pi}{2}, -1\right)$

\item Given equations are $\cos x = \frac{1}{\sqrt{2}}$ and $\tan x = -1$

  $\Rightarrow x = 2n\pi \pm \frac{\pi}{4}$ and $x = m\pi - \frac{\pi}{4}$

  $2n\pi \pm \frac{\pi}{4}$ lies in first and fourth quadrant while $m\pi - \frac{\pi}{4}$ lies in second and fouth
  quadrant.

  Thus, most general value will be $2k\pi + \frac{7\pi}{4}.$

\item Given equations are $\tan x = \sqrt{3}$ and $\csc x = -\frac{2}{\sqrt{3}}$

  $\Rightarrow x = n\pi + \frac{\pi}{3}$ and $x = m\pi + (-1)^{m + 1}\frac{\pi}{3}$

  $n\pi + \frac{\pi}{3}$ lies in first and third quadrant while $m\pi + (-1)^{m + 1}\frac{\pi}{3}$ lies in third and
  fourth quadrant.

  Therefore common general value is $2n\pi + \frac{4\pi}{3}.$

\item Since $x, y$ satisfies $3\cos z + 4\sin z = 2,$ therefore

  $3\cos x + 4\sin x = 2$ and $3\cos y + 4\sin y = 2$

  Subtracting, we get

  $3(\cos x - \cos y) + 4(\sin x - \sin y) = 0$

  $\Rightarrow 6\sin\frac{x + y}{2}\sin\frac{y - x}{2} + 8\cos\frac{x + y}{2}\sin\frac{x - y}{2} = 0$

  $\Rightarrow 2\sin\frac{x - y}{2}\left[4\cos\frac{x + y}{2} - 3\sin\frac{x + y}{2}\right] = 0$

  $\sin\frac{x - y}{2}\neq 0 \because x\neq y$

  $\therefore \tan\frac{x + y}{2}  = \frac{4}{3} \Rightarrow \sin(x + y) = \frac{24}{25}$

\item Let $y = 2\cos^2\frac{x}{2}\sin^2x = x^2+ x^{-2}$

  $y = (1 + \cos x)\sin^2x = [< 2].[\leq 1] [\because 0<x\leq \frac{\pi}{2}]$

  $y < 2$

  $y = x^2 + x^{-2} = \left(x - \frac{1}{x}\right)^2 + 2\geq 2$

  Thus no solution is possible.

\item Given equation is $y = \frac{3 + 2i\sin x}{1 - 2i\sin x}$

  $= \frac{3 + 2i\sin x}{1 - 2i\sin x}.\frac{1 + 2i\sin x}{1 + 2i\sin x}$

  $= \frac{3 - 4\sin^2x}{1 + 4\sin^2x} + i\frac{8\sin x}{1 + 4\sin^2x}$

  For $y$ to be purely real, imaginary part has to be zero.

  $\Rightarrow \sin x = 0 \Rightarrow x = n\pi$

  For $y$ to be purely imaginary, real part has to be zero.

  $\Rightarrow \sin x = \pm\frac{\sqrt{3}}{2}$

  $x = n\pi + (-1)^n\left(\pm\frac{\pi}{3}\right)$

\item Given equation is $a^2 - 2a + \sec^2\pi(a + x) = 0$

  $\Rightarrow a^2 - 2a + 1 + \tan^2\pi(a + x) = 0$

  $\Rightarrow (a - 1)^2 + \tan^2\pi(a + x) = 0$

  For L.H.S. to be zero both terms must be zero. Thus, $(a - 1)^2 = 0$

  $\Rightarrow a = 1$ and $\tan^2\pi(1 + x) = 0$

  $\Rightarrow \pi(1 + x) = n\pi$

  $x = n - 1 = m$ where $m\in I$

\item Given equation is $8^{1 + |\cos x| + \cos^2x + |\cos^3 x| + \ldots \text{~to~}\infty} = 4^3$

  $\Rightarrow 1 + |\cos x| + \cos^2x + |\cos^3 x| + \ldots \text{~to~}\infty = 2$

  This is a geomtric progression with common ratio $|\cos x|.$ We know that $|\cos x|\leq 1$ but $|\cos x| =1$
  will render the previous equation meaningless($\infty=2$)

  $\Rightarrow \frac{1}{1 - |\cos x|} = 2 \Rightarrow |\cos x| = \frac{1}{2}$

  $\cos x = \pm\frac{1}{2} \Rightarrow x = 2n\pi\pm\frac{\pi}{3}, 2n\pi\pm\frac{2\pi}{3}$

  The values of $x$ in the given interval are $\pm\frac{\pi}{3}, \pm\frac{2\pi}{3}$

\item Given equation is $|\cos x|^{\sin^2x - \frac{3}{2}\sin x + \frac{1}{2}} = 1$

  Taking log of both sides,

  $\left(\sin^2x - \frac{3}{2}\sin x + \frac{1}{2}\right)\log |\cos x|$

  If $\sin^2x - \frac{3}{2}\sin x + \frac{1}{2} = 0$

  $\Rightarrow (\sin x - 1)(2\sin x - 1) = 0$

  When $\sin x = 1 \Rightarrow |\cos x| = 0$ which is not a solution because it means $0^0$ for original equation.

  If $2\sin x - 1= 0 \Rightarrow x = n\pi + (-1)^n\frac{\pi}{6}$

  If $\log|\cos x| = 0 \Rightarrow \cos x = \pm1$

  $x = 2n\pi, 2n\pi\pm\pi$

\item Given equation is $3^{\sin2x + 2\cos^2x} + 3^{1 -\sin2x + 2\sin^2x} = 28.$

  $\Rightarrow 3^{\sin2x + 2\cos^2x} + 3^{3 - \sin2x + 2\cos^2x} = 28$

  $\Rightarrow 3^{\sin2x + 2\cos^2x} + \frac{3^3}{3^{\sin2x + 2\cos^2x}} = 28$

  Let $3^{\sin2x + 2\cos^2x} = y$

  $\Rightarrow y + \frac{27}{y} = 28$

  $\Rightarrow (y - 1)(y - 27) = 0$

  If $y = 27 \Rightarrow \sin2x + 2\cos^2x = 3$ which is not possible for any value of $x.$

  If $y = 1 \Rightarrow \sin 2x + 2\cos^2x = 0 \Rightarrow 2\cos x(\sin x + \cos x) = 0$

  $\cos x = 0 \Rightarrow x = 2n\pi + \frac{\pi}{2}$

  If $\sin x + \cos x = 0 \Rightarrow \tan x = -1 = \tan\left(-\frac{\pi}{4}\right)$

  $x = n\pi - \frac{\pi}{4}$

\item Given $2\cos^2x + \sin x\leq 2 \Rightarrow 2(1 - \sin^2x) + \sin x\leq 2$

  $\Rightarrow -2\sin^2x + \sin x\leq 0$

  $\Rightarrow \sin x(2\sin x - 1)\geq 0$

  $\Rightarrow \sin x \leq 0$ or $\sin x\geq \frac{1}{2}$

  $\Rightarrow \pi \leq x \leq 2\pi$ or $\frac{\pi}{6}\geq x \leq \frac{5\pi}{6}$

  Also, $\frac{\pi}{2}\leq x\leq \frac{3\pi}{2}$ from second condition.

  Thus intersection of these two will be the solution.

  $\Rightarrow A\cap B = \left\{x/\pi\leq x\leq \frac{3\pi}{2}, \frac{\pi}{2}\leq x\leq\frac{5\pi}{6}\right\}$

\item Given equation is $\sin x + \cos x = 1 + \sin x\cos x$

  Squaring $\sin^2x + \cos^2x + 2\sin x\cos x = 1 + \sin^2x\cos^2x + 2\sin x\cos x$

  $\Rightarrow 1 + \sin 2x = 1 + \sin 2x + \sin^2x\cos^2x$

  $\Rightarrow \sin x = 0$ or $\cos x = 0$

  $x = n\pi$ or $x = 2n\pi \pm \frac{\pi}{2}$

\item Given $\sin 6x + \cos4x + 2 = 0$

  $\Rightarrow \sin6x = -1$ and $\cos4x=-1$ and both must be satisfied simultaneously.

  $\Rightarrow 6x = 2n\pi + \frac{3\pi}{2} \Rightarrow x = n\pi/3 + \pi/4$

  $\Rightarrow 4x = 2n\pi + \pi \Rightarrow x = n\pi/2 + \pi/4$

  Thus, general solution is $m\pi + \pi/4$

\item Let $n = 3$ then

  $\sin 2x + \sin 3x = 2$

  This will be true if $\sin 2x = 1$ and $\sin3x =1$ simultaneously.

  $\Rightarrow x = \frac{\pi}{4}, \frac{5\pi}{4}$ and $x = \frac{\pi}{6}, \frac{5\pi}{6}, \frac{9\pi}{6}$

  Clearly there is no solution for $n =3$ and thus there will be no solution for higher vallues of $n.$

\item Given equation is $\cos^7x + \sin^4x = 1$

  $\cos^7x \leq \cos^2x$ and $\sin^4x\leq \sin^2x$

  $\therefore \cos^7x + \sin^4x \leq 1$

  The equality is satisfied only when $\cos^7x = \cos^2x$ and $\sin^4x = \sin^2x$

  $\Rightarrow x = (2n + 1)\frac{\pi}{2}$ or $x = 2n\pi$

\item Given equation is $\sin3x -\sin x -2\sin2x + 3 = 0$

  $\Rightarrow 2\sin x\cos2x - 4\sin x\cos x + 3 = 0$

  $\Rightarrow \sin x(2\cos2x - 4\cos x) + 3 = 0$

  $\Rightarrow \sin x(4\cos^2x - 4\cos x -2) + 3 = 0$

  $\Rightarrow \sin x(2\cos x - 1)^2 + 3(1 - \sin x) = 0$

  In the interval $0\leq x\leq \pi, 1 - \sin x\geq 0$

  Also,, $(2\cos x - 1)^2\geq 0$

  Thus above equation holds true only if $\sin x(2\cos x - 1)^2 = 0$ and $1 - \sin x = 0$

  $\sin x = 1 \Rightarrow \cos x = 0 \Rightarrow \sin x(2\cos x - 1)^2 = 1\neq 0$

  Thus all the equations are not satisfied simultaneously. Hence, no solution is possible.

\item Given equation is $\sin x + \cos(k + x) + \cos(k - x) = 2$

  $\Rightarrow \sin x + 2\cos k\cos x = 2$

  Dividing both sides by $\sqrt{1 + 4\cos^2k}$

  $\frac{\sin x}{\sqrt{1 + 4\cos^2k}}\ + \frac{2\cos k\cos x}{\sqrt{1 + 4\cos^2k}} = \frac{2}{\sqrt{1 + 4\cos^2k}}$

  L.H.S. if of the form $\cos(x + y)$ and thus for solutions to exist

  $2\leq \sqrt{1 + 4\cos^2k} \Rightarrow \cos^2k \geq \frac{3}{4}$

  $\Rightarrow 1 - \cos^2k\leq \frac{1}{4} \Rightarrow \sin^2k\leq \frac{1}{4}$

  $\Rightarrow \left(\sin k + \frac{1}{2}\right)\left(\sin k - \frac{1}{2}\right)\leq 0$

  $\Rightarrow -\frac{1}{2}\leq \sin k\leq \frac{1}{2}$

  $\Rightarrow n\pi - \frac{\pi}{6}leq k \leq n\pi + \frac{\pi}{6}$

\item Given equations are $x\cos^3y + 3x\cos y.\sin^2y = 14$ and $x\sin^3y + 3x\cos^2y\sin y = 13$

  Clearly $x\neq 0,$ dividing both the equations

  $\frac{\cos^3y + 3\cos y\sin^2y}{\sin^3y + 3\cos^2y\sin y} = \frac{14}{13}$

  By componendo and dividendo, we get

  $\left(\frac{\cos y + \sin y}{cos y - \sin y}\right)^3 = 27$

  $\Rightarrow \frac{\cos y + \sin y}{cos y - \sin y} = 3$

  Dividing numerator and denominator by $\cos y,$ we get

  $\frac{1 + \tan y}{1 - \tan y} = 3$

  $\Rightarrow \tan y = \frac{1}{2}$

  When $y$ is in first quadrant $\sin y = \frac{1}{\sqrt{5}}, \cos y = \frac{2}{\sqrt{5}}$

  When $y$ is in third quadrant $\sin y = -\frac{1}{\sqrt{5}}, \cos y = -\frac{2}{\sqrt{5}}$

  Thus, when $y$ is in first quadrant $x = 5\sqrt{5}$

  and when $y$ is in third quadrant $x = -5\sqrt{5}.$

\item Given equation is $\sin^4x + \cos^4x + \sin2x + \alpha = 0$

  $\Rightarrow (\sin^2x + \cos^2x)^2 - 2\sin^2x\cos^2x + \sin2x + \alpha = 0$

  $\Rightarrow \sin^22x - 2\sin2x - 2(\alpha + 1) = 0$

  The above equation is a quadratic equation in $\sin2x,$

  $\therefore \sin2x = \frac{2\pm\sqrt{4 + 8(\alpha + 1)}}{2} = 1\pm \sqrt{2\alpha + 3}$

  $\sin2x = 1 + \sqrt{2\alpha + 3}$ is rejected because it is greater than $1$ and if $2\alpha + 3 = 0$ then
  it will be included in follwing.

  $\therefore \sin2x = 1 - \sqrt{2\alpha + 3}$

  For $\sin2x$ to be real $\sqrt{2\alpha + 3}\geq 0$

  $\Rightarrow \alpha \geq \frac{-3}{2}$

  Also, $-1\leq \sin2x\leq 1 \Rightarrow \alpha \leq \frac{1}{2}$

  Thus possible solutions are $-\frac{3}{2}\leq \alpha \leq \frac{1}{2}$

  The general solution is $x = \frac{n\pi}{2} + (-1)^n\frac{\sin^{-1}()1 - \sqrt{2\alpha + 3}}{2}$

\item Given equation is $\tan\left(x + \frac{\pi}{4}\right) = 2\cot x - 1$

  $\Rightarrow \frac{\tan x + \tan\frac{\pi}{4}}{1 - \tan x\tan\frac{\pi}{4}} = \frac{2}{\tan x} - 1$

  $\Rightarrow (1 + \tan x)\tan x = (1 - \tan x)(2 - \tan x)$

  $\Rightarrow 4\tan x = 2\Rightarrow \tan x = \frac{1}{2}$

  $x = n\pi + \tan^{-1}\frac{1}{2}$

  For $\tan\left(x + \frac{\pi}{4}\right)$ to be defined.

  $x + \frac{\pi}{4}\neq$ odd multiple of $\frac{\pi}{2}$

  $x + \frac{\pi}{4}\neq (2n + 1)\frac{\pi}{2}$

  Also for $\cot x$ to be defined. $x\neq$ a multiple of $\pi$

  $\Rightarrow x \neq n\pi$

  We have restrocted the domain above but there many be a root loss. So we need to check if $x = (2n + 1)\frac{\pi}{2}$
  satisfies the original equation.

  $\tan\left(n\pi + \frac{\pi}{2} + \frac{\pi}{4}\right) = -1$

  $2\cot x - 1 = -1$

  Thus, $(2n + 1)\frac{\pi}{2}$ is a solution of the equation.

\item Given equation is $a\cos 2z + b\sin2z = c$

  $\Rightarrow b\sin 2z = c - a\cos2z \Rightarrow b^2\sin^22z = (c - a\cos2z)^2$

  $\Rightarrow b^2(1 - \cos^22z) = c^2 + a^2\cos^22z -2ac\cos2z$

  $\Rightarrow (a^2 + b^2)\cos^22z -2ac\cos2z + c^2 - b^2= 0$

  $\Rightarrow (a^2 + b^2)(2\cos^2z - 1)^2 - 2ac(2\cos^2z - 1) + c^2 - b^2 = 0$

  $\Rightarrow 4(a^2 + b^2)\cos^4z - 4(a^2 + b^2 + ac)\cos^2z + (a + c)^2 = 0$

  This is a quadratic equation in $\cos^2z,$ and sum of roots $= \frac{4(a^2 + b^2 + ac)}{4(a^2 + b^2)}$

  Now of $x$ and $y$ satisfy the equation then

  $\cos^2x + \cos^2y = \frac{4(a^2 + b^2 + ac)}{4(a^2 + b^2)}$

\item Given equation is $\sin(x + y) = k\sin 2x$

  $\Rightarrow \sin x\cos y + \cos x\sin y = k\sin2x$

  $\Rightarrow \frac{s\tan\frac{x}{2}}{1 + \tan^2\frac{x}{y}}\cos y + \frac{1 -\tan^2\frac{x}{2}}{1 +
  \tan^2\frac{x}{2}}\sin y = k.2.\frac{2\tan\frac{x}{2}}{1 + \tan^2\frac{x}{2}}\frac{1 - \tan^2\frac{x}{2}}{1 +
  \tan^2\frac{x}{2}}$

  Let $\tan\frac{x}{2} = t,$ then

  $2t(1 + t^2)\cos y + (1 - t^2)(1 + t^2)\sin y = 4kt(1 - t^2)$

  $\Rightarrow \sin y.t^4 - (4k + 2\cos y)t^3 + (4k - 2\cos y)t - \sin y = 0$

  If $x_1, x_2, x_3, x_y$ are roots of this equation then

  $\sum x_1 = x_1 + x_2 + x_3 + x_4 = \frac{4k + 2\cos y}{\sin y} = s_1$

  $\sum x_1x_2 = x_1x_2 + x_2x_3 + \ldots = 0 = s_2$

  $\sum x_1x_2x_3 = \frac{2\cos y - 4k}{\sin y} = s_3$

  $x_1x_2x_3x_4 = \frac{-\sin y}{\sin y} = -1 = s_4$

  Now, $\tan\left(\frac{x_1 + x_2 + x_3 + x_4}{2}\right) = \frac{s_1 - s_3}{1 - s_2 + s_4}$

  $= \frac{8k}{\sin y.0} = \tan\frac{\pi}{2}$

  $x_1 + x_2 + x_3 + x_4 = 2n\pi + \pi$

\item Given equation is $\sec x + \csc x = c$

  $\Rightarrow \sin x + \cos x = c\sin x\cos x$

  Squaring, we get

  $1 + \sin2x = \frac{c^2}{4}\sin2x$

  $1 + \frac{2\tan x}{1 + \tan^2x} = \frac{c^2}{4}\left(\frac{2\tan x}{1 + \tan^2x}\right)^2$

  Let $\tan x = t,$ then

  $1 + \frac{2t}{1 + t^2} = \frac{c^2}{4}.\frac{4t^2}{(1 + t^2)^2}$

  $\Rightarrow (1 + t + t^2)^2 = t^2(c^2 + 1)$

  {\bf Case I:} When $c^2< 8$

  $\Rightarrow (1 + t + t^2) < 9t^2$

  $\Rightarrow (t^2 + 4t + 1)(t^2 - 2t + 1) < 0$

  $t^2 - 2t + 1 > 0 \because (t - 1)^2 > 0$

  $\therefore t^2 + 4t + 1 < 0$

  $\Rightarrow 2 - \sqrt{3}< t < -2 + \sqrt{3}$

  $\Rightarrow t$ is negative i.e. $\tan x$ will be negative.

  Thus, it will have two values between $0$ and $2\pi.$

  {\bf Case II:} When $c^2 > 8$

  $\Rightarrow (t^2 + 4t + 1)(t - 1)^2 > 0$

  $\Rightarrow -\infty < t < -2 -\sqrt{3}$ or $-2 + \sqrt{3}< t< 1$ or $1< t< \infty$

  Thus, $t$ will be negative and positve and hence $\tan x$ will be positive and negative.

  $\Rightarrow x$ will have four roots between $0$ and $2\pi.$

\item For non-trivial solutions

  $\startbmatrix\NC\lambda NC \sin\alpha \NC \cos\alpha \NR\NC 1 \NC \cos\alpha \NC \sin\alpha \NR\NC -1 \NC \sin\alpha \NC
  -\cos\alpha\NR\stopbmatrix = 0$

  $\Rightarrow \lambda(-\cos^2\alpha - \sin^2\alpha) + \sin\alpha(-\sin\alpha + \cos\alpha) + \cos\alpha(\sin\alpha +
  \cos\alpha) = 0$

  $\Rightarrow \lambda = \cos2\alpha + \sin2\alpha$

  $\Rightarrow \frac{\lambda}{\sqrt{2}} = \frac{\cos2\alpha}{\sqrt{2}} + \frac{\sin2\alpha}{\sqrt{2}}$

  Clearly. $|\lambda| \leq \sqrt{2}$

  When $\lambda = 1\Rightarrow \cos\left(2\alpha - \frac{\pi}{4}\right) = \cos\frac{\pi}{4}$

  $\Rightarrow 2\alpha - \frac{\pi}{4} = 2n\pi\pm\frac{\pi}{4}$

  $\Rightarrow \alpha = n\pi, n\pi + \frac{\pi}{4}$

\item Given equation is $\cos x \cos y\cos(x + y) = -\frac{1}{8}$

  $\Rightarrow 8\cos x\cos y\cos(x + y) + 1 = 0$

  $\Rightarrow 4[\cos(x + y) + \cos(x - y)]\cos(x + y) + 1 = 0$

  $\Rightarrow 4\cos^2(x + y) + 4\cos(x - y)\cos(x + y) + 1 = 0$

  This is a quadratic equation in $\cos(x + y)$

  For real value of $\cos(x + y), D\geq 0$

  $\Rightarrow 16\cos^2(x - y) - 16\geq0$

  $\Rightarrow \sin^2(x - y)\leq 0$

  $\Rightarrow \sin^2(x - y) = 0 \Rightarrow x = y$

  $\Rightarrow 4\cos^22x + 3\cos2x + 1 = 0$

  $\Rightarrow (2\cos2x + 1)^2 = 0$

  $\Rightarrow \cos2x = -\frac{1}{2} \Rightarrow 2\alpha = \frac{2\pi}{3}\Rightarrow x = \frac{\pi}{3}$

  $\therefore x = y = \frac{\pi}{3}$

\item Given $\startbmatrix\NC\sin x\NC \cos x \NC \cos x \NR\NC\cos x \NC \sin x \NC \cos x\NR\NC\cos x \NC \cos x \NC \sin x\NR\stopbmatrix = 0$

  $C_1\Rightarrow C_1 + C_2 + C_3$ and taking that out

  $\Rightarrow (\sin x + 2\cos x)\startbmatrix\NC1 \NC \cos x \NC \cos x\NR\NC1\NC\sin x\NC \cos x\NR\NC1\NC\cos x\NC\sin x\NR\stopbmatrix = 0$

  $\Rightarrow (\sin x + 2\cos x)\startbmatrix\NC1 \NC \cos x \NC \cos x\NR\NC0 \NC \sin x - \cos x \NC 0\NR\NC0 \NC 0 \NC \sin x - \cos
  x\NR\stopbmatrix = 0$

  $\Rightarrow (\sin x + 2\cos x)(\sin x - \cos x)^2 = 0$

  $\Rightarrow$ either $\tan x = 1$ or $\tan x = -2$

  For $-\frac{\pi}{4}\leq x\leq \frac{\pi}{4} \tan \neq -2$

  $\therefore \tan x = 1 \Rightarrow x = \frac{\pi}{4}$

\item Given $3\sin^2x - 7\sin x + 2 = 0.$

  $\Rightarrow (\sin x - 2)(3\sin x - 1) = 0$

  $\because \sin \neq 2 \therefore \sin x = \frac{1}{3}$

  $\Rightarrow x$ will have six values between $[0, 5\pi]$

\item Given equation is $y + \cos x = \sin x$

  $\Rightarrow \frac{y}{\sqrt{2}} = \frac{\sin x}{\sqrt{2}} - \frac{\cos x}{\sqrt{2}}$

  $\Rightarrow \frac{y}{\sqrt{2}} = -\cos\left(x - \frac{\pi}{4}\right)$

  $-1 \leq \cos\left(x - \frac{\pi}{4}\right)\leq 1$

  $\Rightarrow -\sqrt{2}\leq y \leq \sqrt{2}$

  If $y = 1,$ then

  $\cos\left(x + \frac{\pi}{4}\right) = -1/\sqrt{2} = -\cos\frac{\pi}{4} = \frac{3\pi}{4}, \frac{5\pi}{4}[because 0\leq
    x\leq 2\pi]$

  $x = \frac{\pi}{2}, \pi$

\item Given equation is $\sum_{r = 1}^n\sin(rx)\sin(r^2x) = 1$

  $\Rightarrow t_r = \frac{1}{2}2\sin(rx)\sin(r^2x) = \frac{1}{2}[\cos(r^2 - r)x - \cos(r^2 + r)x]$

  $\Rightarrow t_1 = \frac{1}{2}[\cos0 - \cos2x]$

  $t_2 = \frac{1}{2}[\cos2x - \cos6x]$

  $t_3 = \frac{1}{2}[\cos6x - \cos12x]$

  $\ldots$

  $t_n = \frac{1}{2}[\cos(n^2 - n)x - \cos(n^2 + n)x]$

  Adding all these

  $\sum_{r = 1}^n\sin(rx)\sin(r^2x) = \frac{1}{2}[\cos0 - \cos(n^2 + n)x] = 1$

  $\Rightarrow \cos(n^2+ n)x = -1 = \cos\pi$

  $x = \frac{(2m + 1)\pi}{n(n + 1)}$

\item Given equation is $\sin x(\sin x + \cos x) = a$

  $\Rightarrow \sin^2x + \sin x\cos x = a$

  $\Rightarrow \sin^2x\cos^2x = a^2 + \sin^4x -2a\sin^2x$

  $\Rightarrow 2\sin^4x - \sin^2x(2a + 1) + a^2 = 0$

  This is a quadratic equation in $\sin^2x$ which is real so $D\geq 0$

  $\Rightarrow (2a + 1)^2 - 8a^2 \geq 0$

  $\Rightarrow 4a^2 - 4a - 1\leq 0$

  $\Rightarrow \frac{1}{2}(1 - \sqrt{2})\leq a\leq \frac{1}{2}(1 + \sqrt{2})$

\item Given equation is $2\cos^2\frac{x^2 + x}{6} = 2^x + 2^{-x}$

  $-1 \leq \cos\frac{x^2 + x}{6} \leq 1$

  $\Rightarrow 0\leq \cos^2\frac{x^2 + x}{6}\leq 1$

  Also, becasue A.M. $\geq$ G.M.

  $\frac{2^x + 2^{-x}}{2}\geq \sqrt{2^x.2^{-x}} = 1$

  $\Rightarrow \cos^2\frac{x^2 + x}{6} = 1 = \cos 0$

  $\Rightarrow x = 0$

\item Given inequality is $\sin x\geq \cos2x$

  $\Rightarrow \sin x\geq 1 - 2\sin^2x$

  $\Rightarrow 2\sin^2x + \sin x -1 \geq 0$

  $\Rightarrow (2\sin x - 1)(\sin x + 1)\geq 0$

  Limiting value of $\sin x + 1 = 0 \Rightarrow \sin x = -\sin\frac{\pi}{2}$

  $x = 2n\pi -\frac{\pi}{2}$

  Also, $2\sin x - 1\geq 0 \Rightarrow \sin x>\frac{1}{2}\Rightarrow 2n\pi + \frac{\pi}{6}\leq x \leq 2n\pi +
  \frac{5\pi}{6}$

\item Given equation is $\left(\cos\frac{x}{4} - 2\sin x\right)\sin x + \left(1 + \sin \frac{x}{4} -2\cos x\right)\cos x = 0$

  $\Rightarrow \sin x\cos\frac{x}{4} -2\sin^2x + \cos x + \cos x\sin\frac{x}{4} - 2\cos^2x = 0$

  $\sin\left(x +\frac{x}{4}\right) + \cos x = 2$

  This is possible only if $\sin\frac{5x}{4} = 1$ and $\cos x = 1$ simultaneously.

  $\frac{5x}{4} = 2n\pi + \frac{\pi}{2}$ and $x = 2m\pi$

  $x = \frac{8n\pi + 2\pi}{5}$ and $x = 2m\pi$

  Thus, general solution is $(8n + 2)\pi$

\item Given equation is $2(\sin x -\cos2x) - \sin2x(1 + 2\sin x) + 2\cos x = 0$

  $\Rightarrow 2\sin x -\sin2x - 2\cos2x -2\sin x\sin2x + 2\cos x = 0$

  $\Rightarrow 2\sin x -\sin2x - 2\cos2x - (\cos x - \cos3x) + 2\cos x = 0$

  $\Rightarrow 2\sin x(1 - \cos x) + 4\cos^3x - 3\cos x + \cos x - 2(2\cos^2x - 1) = 0$

  $\Rightarrow 2\sin x(1 - \cos x) -4\cos^2x(1 - \cos x) + 2(1 - \cos x) = 0$

  $\Rightarrow (1 - \cos x)[2\sin x - 4(1 - \sin^2x) + 2] =0$

  $\Rightarrow \cos x = 1$ or $\sin x - 2(1 -\sin^2x) + 1 = 0$

  $x = 2n\pi$ or $(2\sin x - 1)(\sin x+ 1) = 0$

  $x = 2n\pi$ or $\sin x = \frac{1}{2}$ or $\sin x = -1$

  $x = 2n\pi, n\pi + (-1)^n\frac{\pi}{6}, n\pi - (-1)^n\frac{\pi}{2}$

\item Given equation is $\frac{\sin2x}{\sin\frac{2x + \pi}{3}} = 0$

  This is true if $\sin2x = 0$ and $\sin\frac{2x + \pi}{3}\neq 0$

  $2x = n\pi$ and $\frac{2x + \pi}{3} = m\pi$

  $x = n\pi/2$ and $x = (3m - 1)\pi/2 \neq n\pi/2 \Rightarrow n \neq 3m - 1$

\item Given equation is $3\tan2x - 4\tan3x = \tan^23x\tan2x$

  $\Rightarrow 3\tan2x - 3\tan3x = \tan3x + \tan^23x\tan2x$

  $\Rightarrow 3(\tan2x - \tan3x) = \tan3x(1 + \tan3x\tan2x)$

  $\Rightarrow -3\tan(2x - 3x) = \tan 3x \Rightarrow \tan3x + 3\tan x = 0$

  Let $\tan x = t,$ then $3t + \frac{3t - t^3}{1 - 3t^2} = 0$

  $\Rightarrow 6t - 10t^3 = 0$

  $t = 0$ or $t = \pm\sqrt{3/5}$

  $x = n\pi$ or $x = n\pi \pm \tan^{-1}\sqrt{3/5}$

\item Given equation is $\sqrt{1 + \sin2x} = \sqrt{2}\cos2x$

  $\Rightarrow \sqrt{\sin^2x + \cos^2x + 2\sin x\cos x} = \sqrt{2}\cos2x$

  $\Rightarrow \sin x + \cos x = \sqrt{2}\cos2x$

  $\Rightarrow \frac{\sin x}{\sqrt{2}} + \frac{\cos x}{\sqrt{2}} = \cos2x$

  $\Rightarrow \cos\left(x - \frac{\pi}{4}\right) = \cos 2x$

  $\Rightarrow x - \frac{\pi}{4} = 2n\pi \pm 2x$

  Now it is trivial to find $x$

\item Given equation is $1 + \sin^2ax = \cos x$

  This is only possible if $\sin^2ax = 0$ and $\cos x = 1$ simultaneously.

  Thus, $x = 0$ is the only solution because $a$ is irrational.

\item For non-trivial solutions

  $\startbmatrix\NC\sin3\theta \NC -1 \NC 1\NR\NC\cos2\theta \NC 4 \NC 3\NR\NC2 \NC 7 \NC 7\NR\stopbmatrix = 0$

  Applying $C_2\Rightarrow C_2 + C_3$

  $\Rightarrow \startbmatrix\NC\sin3\theta \NC 0 \NC 1\NR\NC\cos2\theta \NC 7 \NC 3\NR\NC2 \NC 14 \NC 7\NR\stopbmatrix = 0$

  $\Rightarrow 7\sin3\theta + 14\cos2\theta - 14 = 0$

  $\Rightarrow 3\sin\theta - 4\sin^3\theta + 2 - 4\sin^2\theta + 2 = 0$

  $\Rightarrow \sin\theta(4\sin^2\theta + 4\sin\theta - 3) = 0$

  $\Rightarrow \sin\theta(2\sin\theta + 3)(2\sin\theta - 1) = 0$

  $2\sin\theta + 3 \neq 0$

  $\therefore \sin\theta = 0, 2\sin\theta - 1 = 0$

  $\Rightarrow \theta = n\pi, n\pi + (-1)^n\frac{\pi}{6}$

\item Given equation is $\sin x + \sin\frac{\pi}{8}\sqrt{(1 - \cos x)^2 + \sin^2x} = 0$

  $\Rightarrow \sin x + \sin\frac{\pi}{8}\sqrt{1 - 2\cos x + \cos^2x + \sin^2x} = 0$

  $\Rightarrow \sin x + \sin\frac{\pi}{8}\sqrt{2 - 2\cos x} = 0$

  $\Rightarrow 2\sin\frac{x}{2}\cos\frac{x}{2} + \sin\frac{\pi}{8}2\sin\frac{x}{2} = 0$

  $\Rightarrow 2\sin\frac{x}{2}\left(\cos\frac{x}{2} + \sin\frac{\pi}{8}\right) = 0$

  If $\sin\frac{x}{2} = 0 \Rightarrow x = 2n\pi$ and $\cos\frac{x}{2} = -\sin\frac{\pi}{8} = \sin\frac{9\pi}{8}$

  $x = 2n\pi$ is not valid for given range. $\therefore \frac{x}{2} = 2n\pi\pm\frac{5\pi}{8}$

  In the given range $x = \frac{11\pi}{4}$ is the only solution.

\item $\tan x - \tan^2x > 0 \Rightarrow \tan x > 0, \tan x < 1$ $x \in \left(0, \frac{\pi}{4}\right)$

  $|\sin x|< \frac{1}{2} \Rightarrow -\frac{1}{2}< \sin x < \frac{1}{2}$

  $x\in \left(0, \frac{\pi}{6}\right)$ and $x\in \left(\pi, \frac{7\pi}{6}\right)$

  Clearly, $A\cap B = x\in \left(0, \frac{\pi}{6}\right) \cup x\in \left(\pi, \frac{7\pi}{6}\right)$

\item Given equation is $2^{\frac{1}{\sin^2x}}\sqrt{y^2 - 2y + 2}\leq 2$

  $\sqrt{y^2 - 2y + 2} = \sqrt{(y - 1)^2 + 1} \geq 1$

  Thus, $2^{\frac{1}{\sin^2x}} leq 2$ for equality to be satisfied.

  If $|\sin x| < 1$ then equality will not hold true.

  $\therefore |\sin x| = 1 \Rightarrow x = \frac{\pi}{2}, \frac{3\pi}{2}$

  And thus $y - 1 = 0 \Rightarrow y = 1$

\item Given $|\tan x| = \tan x + \frac{1}{\cos x}$

  If $\tan x = \tan x + \frac{1}{\cos x} \Rightarrow \sec x = 0$ which is not possible.

  $\therefore -\tan x = \tan x + \frac{1}{\cos x} \Rightarrow \sin x = -\frac{1}{2}$

  $\therefore x = \frac{7\pi}{6}, \frac{11\pi}{6}$ in the given interval.

\item Given equation is $\log_{\cos x}\sin x + \log_{\sin x}\cos x = 2$

  Let $y = \log_{\cos x}\sin x,$ then

  $y + \frac{1}{y} = 2 \Rightarrow (y - 1)^2 = 0 \Rightarrow y = 1$

  $\Rightarrow \sin x = \cos x \Rightarrow x = \frac{\pi}{4}$

\item Given equation is $\sin x\cos x + \frac{1}{2}\tan x\geq 1$

  $\Rightarrow \frac{\sin 2x}{2} + \frac{\tan x}{2}\geq 1$

  $\Rightarrow \frac{2\tan x}{1 + \tan^2x} + \tan x \geq 2$

  $\Rightarrow (\tan x - 1)(\tan^2x - \tan x + 2)\geq 0$

  $\tan^2x - \tan x + 2\geq 0~\forall~x$

  $\Rightarrow \tan x \geq 1$

  $x\in\left(n\pi + \pi/4, n\pi + \pi2\right)$

\item Given equation is $\tan x^{\cos^2 x} = \cot x^{\sin x}$

  $\Rightarrow (\sin x)^{\cos^2x + \sin x} = (\cos x)^{\cos^2x + \sin x}$

  **Case I:** $\sin x = \cos x\Rightarrow x = n\pi +\frac{\pi}{4}$

  **Case II:** $\cos^2x + \sin x = 0 \Rightarrow \sin^2x - \sin x - 1 = 0$

  $\sin x = \frac{1\pm\sqrt{5}}{2}$ but $\frac{1 + \sqrt{5}}{2} > 1$ so it is rejected.

  $\Rightarrow \sin x = \frac{1 - \sqrt{5}}{2} = \sin y$ (say)

  $x = n\pi + (-1)^ny$

\item Given equation is $x^2 + 4 + 3\cos(\alpha x + \beta) = 2x$

  $\Rightarrow 3\cos(\alpha x + \beta) = -3 - 3(x - 1)^2$

  For this to have a real solution $x = 1$

  $\Rightarrow \cos(\alpha + \beta) = \pi, 3\pi$

\item Slope of $y = |x| + a = 1, -1$

  $y = 2\sin x \Rightarrow \frac{dy}{dx} = 2\cos x = 1$

  $x = \frac{\pi}{3}$

  So if $a + \frac{\pi}{3} > 2\sin\frac{\pi}{3}$ then it will have no solution.

  $a > \frac{3\sqrt{3} - \pi}{3}$
\stopitemize
