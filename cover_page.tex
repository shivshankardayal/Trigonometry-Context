\environment ../algebra_env.lmtx

\startcomponent cover_page.lmtx

  \setvariables
    [document]
    [titleA=An Angle in Trigonometry,
    titleB=A problem-oriented approach,
    author=Shiv Shankar Dayal]

  \startpagemakeup

    \startMPcode

      StartPage ;

        %fill Page enlarged 10mm withcolor white;
      draw anchored.lrt(image(draw textext("\getvariable{document}{titleA}")  xsized(.70PaperWidth)      withcolor \MPcolor{purple}),(lrcorner Page) shifted (-PaperWidth/18, PaperWidth/3.5));
      draw (0, PaperWidth/3.6)--(PaperHeight, PaperWidth/3.6) withcolor \MPcolor{purple} withpen pencircle scaled 2pt;
      draw anchored.lrt(image(draw textext("\getvariable{document}{titleB}")  xsized(.60PaperWidth)      withcolor \MPcolor{purple}),(lrcorner Page) shifted (-PaperWidth/18, PaperWidth/4.5)) ;
      %       draw anchored.lrt(image(draw textext("\getvariable{document}{titleC}")  xsized(.750PaperWidth)      withcolor white),(lrcorner Page) shifted (-PaperWidth/20, PaperWidth/5)) ;
      draw anchored.urt(image(draw textext("\getvariable{document}{author}")   xsized(.27PaperWidth) rotated 270 withcolor \MPcolor{purple}),(urcorner Page) shifted (-PaperWidth/20,-PaperWidth/20)) ;

        setbounds currentpicture to Page ;
        fill fullcircle scaled 12cm shifted (.5PaperWidth, .5PaperHeight) withcolor \MPcolor{purple};
        pair z, x, xn, y, yn;
        z := (.5PaperWidth, .5PaperHeight);
        x := z + (6cm, 0);
        xn := z - (6cm, 0);
        y := z + (0, 6cm);
        yn := z - (0, 6cm);
        drawdblarrow x--xn withcolor white;
        drawdblarrow y--yn withcolor white;
        draw function(2,"x","sin(x)",-6,6,.1) xyscaled(1cm, 1cm) shifted z withcolor white;
        draw function(2,"x","cos(x)",-6,6,.1) xyscaled(1cm, 1cm) shifted z withcolor white;
        draw function(2,"x","tan(x)",-1.5,-0.001,.1) xyscaled(1cm, 1cm) shifted z withcolor white;
        draw function(2,"x","tan(x)",.001,1.5,.1) xyscaled(1cm, 1cm) shifted z withcolor white;
        label("$y = sin x$", z + (2.8cm, 1.1cm)) withcolor white;
        label("$y = cos x$", z + (3.8cm, -1.2cm)) withcolor white;
        label("$y = tan x$", z + (2.2cm, 3cm)) withcolor white;

      StopPage ;

    \stopMPcode

  \stoppagemakeup

  {
    \setupbodyfont[9pt,rm, mf] % The document's main (body) font
    \page[yes]
    \page[blank]\parindent0pt
    \ \vfill

    {\bf An Angle in Trigonometry}\\
    {\color[red]{Early Draft}} [\RevisionDate]\\
    \blank

    Copyright, \copyright~ Shiv Shankar Dayashru, 2025. All rights reserved.\\\\
    Permission is granted to copy, distribute and/or modify this document under the
    terms of the GNU Free Documentation License, Version 1.3 or any later version
    published by the Free Software Foundation; with no Invariant Sections, no
    Front-Cover Texts, and no Back-Cover Texts. A copy of the license is included
    in the section entitled \quotation{GNU Free Documentation License}.

  }
  {
      \setupbodyfont[9pt,rm, mf] % The document's main (body) font
      \page[yes]
    \page[blank]\parindent0pt
    ~~~
    \blank[4*big]
    \vskip 3cm
    \startalignment [middle]
    {\bigskip\it Dedicated to my wife, Binita}
    \stopalignment
  }

%   % Table of contents
  \setupbodyfont[9pt,rm, mf] % The document's main (body) font

  \starttitle
    [title={Table of Contents}]

    \setcounter[userpage][1]

    % Format of entries in the ToC
    \setuplist
      [part]
      [
        before={\blank[2*big]},
        after={\blank[big]},
        style=\bfa,
        aligntitle=yes,
      ]

    \setuplist
      [chapter]
      [
        before=\blank,
        style=\bf,
        margin=.5cm,
        aligntitle=yes
      ]

    \setuplist
      [section]
      [
        aligntitle=yes,
        %style=\tfx,
        margin=1.5cm
      ]

    % Let us define our list (called a Toc)
    \definecombinedlist
      [Toc]
      [part, chapter, section]
      [level=subsection, alternative=c]

    % and insert it in the context of smaller interline space
    \start
      \setupwhitespace[none]
      \switchtobodyfont[10pt]
      \setupinterlinespace[small]
      \placeToc
    \stop

  \stoptitle

\stopcomponent
