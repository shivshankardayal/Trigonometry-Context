% -*- mode: context; -*-
\chapter{Properties of Triangles}
\startitemize[n, 1*broad]
  %1
\item Let $a = 8$ cm, $b = 10$ cm and $c = 12$ cm. So the smallest angle will be $A$ and greatest angle will
  be $C.$ So from cosine rule,

  $\cos A = \frac{b^2 + c^2 - a^2}{2bc} = \frac{100 + 144 - 64}{2.10.12} = \frac{3}{4}$

  $\cos C = \frac{a^2 + b^2 - c^2}{2ab} = \frac{64 + 100 - 144}{2.8.10} = \frac{1}{8}$

  $\cos 2A = 2\cos^2A - 1 = 2.\frac{9}{16} - 1 = \frac{1}{8} = \cos C$

  Thus, we see that greatest angle is double to that of smallest angle.
  %2
\item Let $\frac{b + c}{11} = \frac{c + a}{12} = \frac{a + b}{13}, = k$

  $\therefore b + c = 11k, c + a = 12k, a + b = 13k \Rightarrow a + b + c = 18k$

  $\therefore a = 7k, b = 6k, c = 5k$

  $\frac{\cos A}{7} = \frac{b^2 + c^2 - a^2}{2bc} = \frac{36k^2 + 25k^2 - 49k^2}{2.6.7.7k^2} = \frac{1}{35}$

  $\frac{\cos B}{19} = \frac{c^2 + a^2 - b^2}{2ca} = \frac{25k^2 + 49k^2 - 36k^2}{2.5.7.19.k^2} = \frac{1}{35}$

  $\frac{\cos C}{25} = \frac{a^2 + b^2 - c^2}{2ab} = \frac{49k^2 + 36k^2 - 25k^2}{2.7.6.25.k^2} = \frac{1}{35}$
  %3
\item Given, $\Delta = a^2 - (b - c)^2 \Rightarrow \Delta = a^2 - b^2 - c^2 + 2bc$

  $\Rightarrow b^2 + c ^2 - a^2 = 2bc - \Delta \Rightarrow 2bc\cos A = 2bc - \frac{1}{2}bc\sin A$

  $\Rightarrow 4\cos A + \sin A = 4 \Rightarrow 4\left(1 - 2\sin^2\frac{A}{2}\right) + 2\sin\frac{A}{2}\cos\frac{A}{2} = 4$

  $2\sin\frac{A}{2}\left(\cos\frac{A}{2} - 4\sin\frac{A}{2}\right) = 0$

  $\Rightarrow$ either $\sin\frac{A}{2} = 0$ or $\cos\frac{A}{2} - 4\sin\frac{A}{2} = 0$

  $A = 0$ is not possible. $\therefore \tan\frac{A}{2} = \frac{1}{4}$

  $\tan A = \frac{2\tan\frac{A}{2}}{1 - \tan^2\frac{A}{2}} = \frac{8}{15}.$
  %4
\item Since $A, B, C$ are in A.P. $\therefore 2B = A + C$

  $\because A + B + C = 180^\circ \Rightarrow B = 60^\circ$

  $\cos B = \frac{c^2 + a^2 - b^2}{2ca} \Rightarrow \frac{1}{2} = \frac{c^2 + a^2 - b^2}{2ac} \Rightarrow a^2 - ac + c^2 =
  b^2$

  $\Rightarrow \frac{a + c}{\sqrt{a^2 - ac + a^2}} = \frac{a + c}{b}$

  $= \frac{k(\sin A + \sin C)}{k\sin B} = \frac{2\sin\frac{A + C}{2}\cos\frac{A - C}{2}}{\sin B}$

  $= \frac{2\sin B\cos \frac{A - C}{2}}{\sin B} = 2\cos\frac{A - C}{2}$
  %5
\item $\Delta = \frac{1}{2}.a.p_1 \therefore \frac{1}{p_1} = \frac{a}{\Delta}$

  Similarly, $\frac{1}{p_2} = \frac{b}{\Delta}, \frac{1}{p_3} = \frac{c}{\Delta}$

  L.H.S. $= \frac{a + b - c}{2\Delta} = \frac{(a + b)^2 - c^2}{2\Delta(a + b + c)}$

  $= \frac{2ab + a^2 + b^2 - c^2}{2\Delta(a + b + c)} = \frac{2ab + 2ab\cos C}{2\Delta(a + b + c)}$

  $= \frac{ab(1 + \cos C)}{\Delta(a + b + c)} = \frac{2ab\cos^2\frac{C}{2}}{\Delta(a + b + c)}$
  %6
\item $\because :\tan\theta = \frac{2\sqrt{ab}}{a - b}\sin\frac{C}{2}$

  $\Rightarrow (a - b)^2\tan^2\theta = 4ab\sin^2\frac{C}{2} \Rightarrow (a - b)^2(\sec^2\theta - 1) = 4ab\sin^2\frac{C}{2}$

  $(a - b)^2\sec^2\theta = a^2b^2 -2ab\left(1 - 2\sin^2\frac{C}{2}\right)$

  $(a - b)^2\sec^2\theta = a^2 + b^2 - 2ab\cos C = c^2[\because \cos C = \frac{a^2 + b^2 - c^2}{2ab}]$

  $\Rightarrow c = (a - b)\sec\theta.$
  %7
\item $\Delta ABC = \frac{1}{2}bc\sin A = \frac{1}{2}6.3.\sin C = 9\sin C$

  $\tan^2\frac{A - B}{2} = \frac{1 - \cos(A - B)}{1 + \cos(A - B)} = \frac{1}{9}$

  $\tan\frac{A - B}{2} = \pm\frac{1}{3}$

  $\because 0 < \frac{A - B}{2} < 90^\circ \therefore \tan\frac{A - B}{2} = \frac{1}{3}$

  $\tan\frac{A - B}{2} = \frac{a - b}{a + b}\cot\frac{C}{2} \Rightarrow \cot\frac{C}{2} = 1$

  $\Rightarrow C = 90^\circ$

  Thus, required area $= 8\sin90^\circ = 9$
  %8
\item The diagram is given below:

  \startplacefigure
    \externalfigure[18_1.pdf]
  \stopplacefigure

  From question, $\angle A = 75^\circ, \angle C = 60^\circ \Rightarrow \angle B = 45^\circ$

  Also given, $\frac{\Delta BAD}{\Delta BCD} = \sqrt{3} = \frac{c.x.\sin\theta}{a.x.\sin(45^\circ - \theta)}$ where
  $BD = x$

  $\Rightarrow \frac{c\sin\theta}{a\sin(45^\circ - \theta)} = \sqrt{3}$

  $\Rightarrow \frac{\sin C\sin\theta}{\sin A\sin(45^\circ - \theta)} = \sqrt{3}$

  $\Rightarrow \frac{\frac{\sqrt{3}}{2}\sin\theta}{\frac{\sqrt{3} + 1}{2\sqrt{2}}\sin(45^\circ - \theta)} = \sqrt{3}$

  $\Rightarrow \sqrt{2}\sin\theta = (\sqrt{3} + 1)\sin(45^\circ - \theta)$

  $\Rightarrow \sqrt{2}\sin\theta = (\sqrt{3} + 1)\left(\frac{1}{\sqrt{2}}\cos\theta - \frac{1}{\sqrt{2}}\sin\theta\right)$

  $\Rightarrow 2\sin\theta = (\sqrt{3} + 1)(\cos\theta - \sin\theta)$

  $\tan\theta = \frac{1}{\sqrt{3}} \Rightarrow \theta = 30^\circ$
  %9
\item We find the largest angle which is opposite to side $7,$ if any angle can be obtuse angle then this one can.

  $\cos \theta = \frac{3^2 + 5^2 - 7^2}{2.3.5} = -\frac{15}{30} = -\frac{1}{2}$

  $\Rightarrow \theta = 120^\circ$
  %10
\item Given $\angle A = 45^\circ, \angle B = 75^\circ \Rightarrow \angle = 60^\circ$

  We know that $\frac{a}{\sin A} = \frac{b}{\sin B} = \frac{c}{\sin C} = k$

  $a = \frac{1}{\sqrt{2}}k, b = \frac{\sqrt{3} + 1}{2\sqrt{2}}k, c = \frac{\sqrt{3}k}{2}$

  $a + c\sqrt{2} = \frac{1}{\sqrt{\sqrt{2}}}k + \frac{\sqrt{3}}{\sqrt{2}}k = \frac{1 + \sqrt{3}}{\sqrt{2}}k$

  $2b = \frac{\sqrt{3} + 1}{\sqrt{2}}$

  $\Rightarrow a + \sqrt{2}c = 2b$
  %11
\item The diagram is given below:

  \startplacefigure
    \externalfigure[18_2.pdf]
  \stopplacefigure

  $\Delta BCD + \Delta ACD = \Delta ABC$

  $\Rightarrow \frac{1}{2}3CD\sin30^\circ + \frac{1}{2}4CD\sin60^\circ = \frac{1}{2}3.4$

  $\frac{3}{4}CD + \sqrt{3}CD = 6 \Rightarrow CD = \frac{24}{3 + 4\sqrt{3}}$
  %12
\item The smallest angle will be opposite to smallest side i.e. $4$ cm. Similarly, greatest angle will be opposite to greatest
  side i.e. $6$ cm.

  Let $a = 4$ cm, $b = 5$ cm and $c = 6$ cm. Also, let opposite angles are $A, B$ and $C.$

  $\cos A = \frac{b^2 + c^2 - a^2}{2bc} = \frac{45}{60} = \frac{3}{4}$

  $\cos C = \frac{a^2 + b^2 - c^2}{2ab} = \frac{5}{40} = \frac{1}{8}$

  $\cos 2A = 2\cos^2A - 1 = 2.\frac{9}{16} - 1 = \frac{1}{8} = \cos C$

\item The diagram is given below:

  \startplacefigure
    \externalfigure[18_3.pdf]
  \stopplacefigure

  $\angle A + \angle B + \angle C = 180^\circ = 10\angle B + \angle B + \angle B \Rightarrow \angle B = 15^\circ$

  $\Rightarrow \angle A = 150^\circ$

  Let $AB = AC = x$

  $\therefore \cos 150^\circ = \frac{x^2 + x^2 - a^2}{2.x.x}$

  $\Rightarrow -\sqrt{3}x^2 = 2x^2 - a^2 \Rightarrow x = \sqrt{\frac{1}{2 + \sqrt{3}}}a$

\item Let angles are $A = 2k, B=3k, C=7k \therefore 2k + 3k + 7k = 180^\circ \Rightarrow k = 15^\circ$

  $\frac{a}{\sin A} = \frac{b}{\sin B} = \frac{c}{\sin C} = l$

  $\sin A = \frac{1}{2}, \sin B= \frac{1}{\sqrt{2}}, \sin C = \frac{\sqrt{2 + \sqrt{3}}}{2}$

  Now the ratios of sides can be proven.

\item Clearly, the sides are in the ratio of $7:6:5$

  $\therefore \cos A = \frac{6^2 + 5^2 - 7^2}{2.6.5} = \frac{1}{5}$

  $\cos B = \frac{7^2 + 5^2 - 6^2}{2.7.5} = \frac{19}{35}$

  $\cos C = \frac{7^2 + 6^2 - 5^2}{2.7.6} = \frac{5}{7}$

  $\therefore \cos A:\cos B:\cos C = 7:19;25$

\item $\tan \frac{C}{2} = \tan\frac{\pi - (A + B)}{2} = \cot\frac{A + B}{2} = \frac{1 -
  \tan\frac{A}{2}\tan\frac{B}{2}}{\tan\frac{A}{2} + \tan\frac{B}{2}}$

  $= \frac{1 - \frac{5}{6}.\frac{20}{37}}{\frac{5}{6} + \frac{20}{37}}$

  $= \frac{\frac{122}{222}}{\frac{305}{222}} = \frac{122}{305} = \frac{2}{5}$

  $\sin A = \frac{2\tan\frac{A}{2}}{1 + \tan^2\frac{A}{2}} = \frac{2.\frac{5}{6}}{1 + \frac{25}{36}}$

  $= \frac{60}{61}$

  $\sin B = \frac{2\tan\frac{B}{2}}{1 + \tan^2\frac{B}{2}} = \frac{2.\frac{20}{37}}{1 + \frac{400}{1369}}$

  $= \frac{1480}{1769}$

  $\sin B = \frac{2\tan\frac{C}{2}}{1 + \tan^2\frac{C}{2}} = \frac{2\frac{2}{5}}{1 + \frac{4}{25}}$

  $= \frac{20}{29}$

  $a + c = k(\sin A + \sin C)[\frac{a}{\sin A} = \frac{b}{\sin B} = \frac{c}{\sin C} = k]$

  $= k\left(\frac{60}{61} + \frac{20}{29}\right) = \frac{1740 + 1220}{1769} = \frac{2960}{1769} = 2b.$

\item It is much easier to prove this problem in reverse.

  Given, $\frac{1}{a + c} + \frac{1}{b + c} = \frac{3}{a + b + c}$

  Upon solving $a^2 + b^2 - c^2 = ab$

  $\Rightarrow \frac{a^2 + b^2 - c^2}{2ab} = \frac{1}{2} = \cos60^\circ = \cos C$

\item Let the sides be $a, b, c.$ We know that $\Delta = \frac{1}{2}a.\alpha$ because area = $\frac{1}{2}\times$ base $\times$ altitude

  $\Delta = \frac{1}{2}\alpha \Rightarrow \alpha = \frac{2\Delta}{a} \Rightarrow \frac{1}{\alpha} = \frac{a}{2\Delta}$

  $\frac{1}{\alpha^2} + \frac{1}{\beta^2} + \frac{1}{\gamma^2} = \frac{a^2 + b^2 + c^2}{4\Delta^2}$

  $\frac{\cot A + \cot B + \cot C}{\Delta} = \frac{\cos A}{\Delta\sin A} + \frac{\cos B}{\Delta\sin B} + \frac{\cos
  C}{\Delta\sin C}$

  $\Delta = \frac{1}{2}bc\sin A \Rightarrow \sin A = \frac{2\Delta}{bc}$

  $\therefore \frac{\cos A}{\Delta\sin A} = \frac{bc\cos A}{2\Delta^2} = \frac{b^2 + c^2 - a^2}{4\Delta^2}$

  $\therefore \frac{\cos A}{\Delta\sin A} + \frac{\cos B}{\Delta\sin B} + \frac{\cos
  C}{\Delta\sin C} = \frac{a^2 + b^2 + c^2}{4\Delta^2}$

  Hence proven.

\item Given, $\frac{a}{b} = 2 + \sqrt{3} = \tan75^\circ = \frac{\sin75^\circ}{\cos75^\circ}$

  $\frac{\sin A}{\sin B} = \frac{\sin(90^\circ + 15^\circ)}{\sin(75^\circ - 15^\circ)}$

  $\Rightarrow A = 105^\circ, B= 15^\circ$ which satisfied $A + B + C = 180^\circ$

\item $\cos C = \frac{1}{2} = \frac{a^2 + b^2 - c^2}{2ab} \Rightarrow (1 + \sqrt{3}^2) + 4 - c^2 = 2(1 + \sqrt{3})$

  $\Rightarrow c^2 = 6 \Rightarrow c = \sqrt{6}$

  $\cos A = \frac{b^2 + c^2 - a^2}{2bc} = \frac{4 + 6 - (1 + \sqrt{3})^2}{4\sqrt{6}}$

  $= \frac{6 + 2\sqrt{3}}{\sqrt{6}} = \sqrt{6} + \sqrt{2} \Rightarrow A = 75^\circ$

  Thus, $\angle B = 45^\circ$

\item Greatest angle will be opposite to greatest side i.e. $\sqrt{x^2 + xy + y^2}$

  $\cos\theta = \frac{x^2 + y^2 - x^2 - xy - y^2}{2.x.y} = -\frac{1}{2} \Rightarrow \theta = 120^\circ$

\item Given sides are $2x + 3, x^2 + 3x + 3$ and $x^2 + 2x.$ Since lengths of sides is a positive quantity, therefore
  $x^2 + 2x > 0 \Rightarrow x > 0$

  This leads to the fact that $x^2 + 3x + 3$ will be greatest side.

  $\cos\theta = \frac{(2x + 3)^2 + (x^2 + 2x)^2 - (x^2 + 3x + 3)^2}{2(2x + 3)(x^2 + 2x)}$

  $= \frac{4x^2 + 12x + 9 + x^4 + 4x^3 + 4x^2 - x^4 - 9x^2 - 9 - 6x^3 - 6x^2 - 18x}{4x^3 + 14x^2 + 12x}$

  $= \frac{-2x^3 -7x^2 - 6x}{2(2x^3 + 7x^2 + 6x)} = -\frac{1}{2} = \cos120^\circ$

\item Given, $3a = b + c.$ We know that $s = \frac{a + b + c}{2} \Rightarrow s = 2a$

  $\cot\frac{B}{2}\cot\frac{C}{2} = \frac{s(s - b)}{\Delta}.\frac{s(s - c)}{\Delta}$

  $=\frac{s^2(s - b)(s - c)}{s(s - a)(s - b)(s - c)} = \frac{s}{s - a} = 2$

\item We have to prove that $a\sin\left(\frac{A}{2} + B\right) = (b + c)\sin\frac{A}{2}$

  $\frac{b + c}{a} = \frac{k(\sin B + \sin C)}{\sin A}$

  $= \frac{2\sin\frac{B + c}{2}\cos\frac{B - C}{2}}{2\sin\frac{A}{3}\cos\frac{A}{2}}$

  $= \frac{\cos\left(\frac{B}{2} - \frac{C}{2}\right)}{\sin\frac{A}{2}}$

  $= \frac{\cos\left[\frac{B}{2} - \left\{\frac{\pi}{2} - \left(\frac{A + B}{2}\right)\right\}\right]}{}\sin\frac{A}{2}$

  $= \frac{\sin\left(\frac{A}{2} + B\right)}{\sin\frac{A}{2}}$

\item Numerator of L.H.S. $= \frac{s(s - a)}{\Delta} + \frac{s(s - b)}{\Delta} + \frac{s(s - c)}{\Delta}$

  $= \frac{s(s - a + s - b + s - c)}{\Delta} = \frac{s^2}{\Delta} = \frac{(a + b + c)^2}{4\Delta}$

  Denominator of R.H.S. $= \frac{\cos A}{\sin A} + \frac{\cos B}{\sin B} + \frac{\cos C}{\sin C}$

  $= \frac{b^2 + c^2 - a^2}{2bc\sin A} + \frac{c^2 + a^2 - b^2}{2ca\sin B} + \frac{a^2 + b^2 - c^2}{2ab\sin C}$

  [ $\Delta = \frac{1}{2}bc\sin A = \frac{1}{2}ca\sin B = \frac{1}{2}ab\sin C$ ]

  $= \frac{a^2 + b^2 + c^2}{4\Delta}$

  L.H.S. $= \frac{(a + b + c)^2}{a^2 + b^2 + c^2}$

\item First term of L.H.S. $= \frac{b^2 - c^2}{a^2}\sin2A = \frac{b^2 - c^2}{a^2}2\sin A\cos A$

  $= \frac{b^2 - c^2}{a^2}.2\frac{a}{K}.\frac{b^2 + c^2 - a^2}{2bc}$

  $= \frac{1}{Kabc}[(b^2 - c^2)(b^2 + c^2 - a^2)] = \frac{1}{Kabc}[b^4 - c^4 -a^2(b^2 - c^2)]$

  Similarly, second term $= \frac{1}{Kabc}[c^4 - a^4 - b^2(c^2 - a^2)]$

\item First term of L.H.S $= a^3\cos(B - C) = a^2[a\cos(B - C)]$

  $= Ra^2[2\sin A\cos(B - C)] = Ra^2[2\sin(B + C)\cos(B - C)] = Ra^2[\sin2B + \sin2C]$

  $=Ra^2[2\sin B\cos B + 2\sin C\cos C] = a^2[b\cos B + c\cos C]$

  Similarly, second term $= b^2[a\cos A + c\cos C]$

  and third term $= c^2[a\cos A + b\cos B]$

  Adding, $ab[a\cos B + b\cos A] + ca[c\cos A + a\cos C] + bc[b\cos C + c\cos B]$

  $= 3abc =$ R.H.S.

\item L.H.S. $= \frac{\cos^2\frac{B - C}{2}}{K^2[\sin B + \sin C]^2} + \frac{\sin^2\frac{B - C}{2}}{K^2[\sin B - \sin C]^2}$

  $= \frac{1}{K^2}\left(\frac{\cos^2\frac{B - C}{2}}{4\sin^2\frac{B + C}{2}\cos^2\frac{B - C}{2}} + \frac{\sin^2\frac{B -
    C}{2}}{4\cos^2\frac{B + C}{2}\sin^2\frac{B - C}{2}}\right)$

  $= \frac{1}{4k^2}\left(\frac{1}{\sin^2\frac{B + C}{2}} + \frac{1}{\cos^2\frac{B + C}{2}}\right)$

  $= \frac{1}{4K^2}\left(\frac{1}{\cos^2\frac{A}{2}} + \frac{1}{\sin^2\frac{A}{2}}\right)$

  $= \frac{1}{k^2}.\frac{1}{4\sin^2\frac{A}{2}\cos^2\frac{A}{2}} = \frac{1}{a^2}$

\item First term of L.H.S. $= \frac{a}{\cos B\cos C} = \frac{2R\sin A}{\cos B\cos C}$

  $= \frac{2R\sin(B + C)}{\cos B\cos C} = 2R(\tan B + \tan C)$

  Similarly, second term $= 2R(\tan C + \tan A)$ and third term $= 2R(\tan A + \tan B)$

  L.H.S. $= 4R[\tan A + \tan B + \tan C]$

  $= 4R.\tan A\tan B\tan C[\because A + B + C = \pi \therefore \tan A + \tan B + \tan C = \tan A\tan B\tan C]$

  $= 2.a\tan B\tan C\sec A =$ R.H.S.

\item We have to prove that $(b - c)\cos\frac{A}{2} = a\sin\frac{B - C}{2}$

  $\frac{b - c}{a} = \frac{\sin B - \sin C}{\sin A} = \frac{2\cos\frac{B + C}{2}\sin \frac{B -
    C}{2}}{2\sin\frac{A}{2}\cos\frac{A}{2}}$

  $= \frac{2\sin\frac{A}{2}\sin\frac{B - C}{2}}{2\sin\frac{A}{2}\cos\frac{A}{2}} = \frac{\sin\frac{B - C}{2}}{\cos\frac{A}{3}}$

\item We have to prove that $\tan\left(\frac{A}{2} + B\right) = \frac{c + b}{c - b}\tan \frac{A}{2}$

  $\frac{c + b}{c - b} = \frac{\sin C + \sin B}{\sin C - \sin B}$

  $= \frac{2\sin\frac{B + C}{2}\cos\frac{C - B}{2}}{2\cos\frac{B + C}{2}\sin\frac{C - B}{2}}$

  $= \frac{\tan\frac{B + C}{2}}{\tan\frac{C - B}{2}} = \frac{\cot\frac{A}{2}}{\tan\frac{\pi - B - A - B}{2}}$

  $= \frac{\tan\left(\frac{A}{2} + B\right)}{\tan\frac{A}{3}}$

\item We have to prove that $\tan\frac{A - B}{2} = \frac{a - b}{a + b}\cot\frac{C}{2}$

  $\frac{a - b}{a + b} = \frac{\sin A - \sin B}{\sin A + \sin B} = \frac{2\cos\frac{A + B}{2}\sin\frac{A -
    B}{2}}{2\sin\frac{A + B}{2}\cos\frac{A - B}{2}}$

  $= \frac{\tan\frac{A - B}{2}}{\tan \frac{A + B}{2}} = \frac{\tan\frac{A - B}{2}}{\cot\frac{C}{2}}$

\item L.H.S. $= (b + c)\cos A + (c + a)\cos B + (a + b)\cos C$

  $= (a\cos B + b\cos A) + (b\cos C + c\cos B) + (a\cos C + c\cos A)$

  $= c + a + b =$ R.H.S.

\item First term of L.H.S. $= \frac{\cos^2B - \cos^2C}{b + c} = \frac{1}{k}\left[\frac{(\cos B + \cos C)(\cos B - \cos C)}{\sin
    B + \sin C}\right]$

  $= \frac{1}{k}\left[\frac{2\cos\frac{B+C}{2}\cos\frac{B - C}{2}.2\sin\frac{B + C}{2}\sin\frac{C - B}{2}}{2\sin\frac{B +
      C}{2}\cos\frac{B - C}{3}}\right]$

  $= \frac{1}{k}\left[2\cos\frac{B + C}{2}\sin\frac{C - B}{2}\right] = \frac{1}{k}[\sin C - \sin B]$

  Similarly, second term $= \frac{1}{k}[\sin A - \sin C]$ and third term $= \frac{1}{k}[\sin B - \sin A]$

  Thus, L.H.S. = R.H.S. = 0

\item First term of L.H.S. $= a^3\sin(B - C) = Ra^2.2\sin A\sin(B - C) = Ra^2.2\sin(B + C)\sin(B - C)$

  $= Ra^2[\cos 2C - \cos 2B] = Ra^2(1 - \sin^2C - 1 + \sin^2B) = R[(2R\sin B)^2 - (2R\sin C)^2]$

  $= R[b^2 - c^2]$

  Similarly, second term $= R[c^2 - a^2]$ and third term $= R[a^2 - b^2]$

  Thus, L.H.S. $= 0 =$ R.H.S.

\item Consider first term i.e. $(b + c - a)\tan \frac{A}{2}$

  $b + c - a = 2s - 2a = 2(s - a)$

  $\tan\frac{A}{2} = \sqrt{\frac{(s - b)(s - c)}{s(s - a)}}$

  $\therefore (b + c - a)\tan \frac{A}{2} = 2\sqrt{\frac{(s - a)(s - b)(s - a)}{s}}$

  Similarly, $(c + a - b)\tan \frac{B}{2} = 2\sqrt{\frac{(s - a)(s - b)(s - a)}{s}} = (a + b - c)\tan\frac{C}{2}$

\item $1 - \tan\frac{A}{2}\tan\frac{B}{2} = 1 =- \sqrt{\frac{(s - b)(s - c)}{s(s - a)}.\frac{(s - a)(s - c)}{s(s - b)}}$

  $= 1 - \frac{s - c}{s} = \frac{c}{s} = \frac{2c}{a + b + c} =$ R.H.S.

\item L.H.S. $= \frac{\cos2A}{a^2} - \frac{\cos2B}{b^2}$

  $= \frac{1 - 2\sin^2A}{a^2} - \frac{1 - 2\sin^2B}{b^2}$

  $= \frac{1 - 2.\frac{a^2}{4r^2}}{a^2} - \frac{1 - 2.\frac{b^2}{4r^2}}{b^2}$

  $= \frac{1}{a^2} - \frac{1}{b^2} =$ R.H.S.

\item We have to prove that $a^2(\cos^2B - \cos^2C) + b^2(\cos^2C - \cos^2A) + c^2(\cos^2A - \cos^2B) = 0$

  L.H.S. $= a^2(\sin^2C - \sin^2B) + b^2(\sin^2A - \sin^2C) + c^2(\sin^2B - \sin^2A)$

  $= 4R^2\sin^A(\sin^2C - \sin^2B) + 4R^2\sin^2B(\sin^2A - \sin^2C) + 4R^2\sin^2C(\sin^2B - \sin^2A)$

  $= 0 =$ R.H.S.

\item First term of L.H.S. $= \frac{a^2\sin(B - C)}{\sin B + \sin C}$

  $= \frac{2Ra\sin A\sin(B - C)}{\sin B + \sin C} = \frac{2Ra\sin(B + C)\sin(B - C)}{\sin B + \sin C}$

  $= \frac{Ra(\cos 2C - \cos 2B)}{\sin B + \sin C} = \frac{Ra(2\sin^2B - 2\sin^2C)}{\sin B + \sin C}$

  $= 2Ra(\sin B - \sin C) = a(b - c)$

  Similarly, second term $= b(c - a)$ and third term $= c(a - b)$

  Thus, L.H.S. $= 0 =$ R.H.S.

\item L.H.S. $= \frac{\cos A}{a} + \frac{\cos B}{b} + \frac{\cos C}{c}$

  $= \frac{b^2 + c^2 - a^2}{2abc} + \frac{c^2 + a^2 - b^2}{2abc} + \frac{a^2 + b^2 - c^2}{2abc}$

  $= \frac{a^2 + b^2 + c^2}{2abc} =$ R.H.S.

\item First term of L.H.S. $= \frac{\cos A}{a} + \frac{a}{bc}$

  $= \frac{b^2 + c^2 - a^2}{2abc} + \frac{a}{bc} = \frac{a^2 + b^2 + c^2}{2abc}$

  Similarly, second term = third term = $\frac{a^2 + b^2 + c^2}{2abc}$

\item First term of L.H.S. $= (b^2 - c^2)\frac{\cos A}{\sin A}$

  $= \frac{(b^2 - c^2)(b^2 + c^2 - a^2)}{2abc} = \frac{b^4 - c^4 - a^2(b^2 - c^2)}{2abc}$

  Similarly, second term $= \frac{c^4 - a^4 - b^2(c^2 - a^2)}{2abc}$

  and third term $= \frac{a^4 - b^4 - c^2(a^2 - b^2)}{2abc}$

  Thus, L.H.S. $= 0 =$ R.H.S.

\item L.H.S. $= (b - c)\frac{s(s - a)}{\Delta} + (c - a)\frac{s(s - b)}{\Delta} + (a - b)\frac{s(s - c)}{\Delta}$

  $= \frac{s}{\Delta}(b^2 - c^2 + c^2 - a^2 + a^2 - b^2) = 0 =$ R.H.S.

\item L.H.S. $= (a - b)^2 + \sin^2\frac{C}{2}\left[(a + b)^2 - (a - b)^2\right]$

  $= (a - b)^2 + 2ab.2\sin^2\frac{C}{2} = (a - b)^2 + 2ab[1 - \cos C]$

  $= a^2 - 2ab + b^2 +2ab - a^2 - b^2 + c^2 = c^2 =$ R.H.S.

\item L.H.S. $= \frac{a - b}{a + b} = \frac{\sin A - \sin B}{\sin A + \sin B}$

  $= \frac{2\cos\frac{A + B}{2}\sin\frac{A - B}{2}}{2\sin\frac{A = B}{2}\cos\frac{A - B}{2}}$

  $= \cot \frac{A + B}{2}\tan\frac{A - B}{2} =$ R.H.S.

\item The diagram is given below:

  \startplacefigure
    \externalfigure[18_4.pdf]
  \stopplacefigure

  $\cos C = \frac{b}{a/2} = \frac{2b}{a}$

  $\frac{a^2 + b^2 - c^2}{2ab} = \frac{2b}{a} \Rightarrow 3b^2 = a^2 - c^2$

  $\cos A\cos C = \frac{b^2 + c^2 - a^2}{2bc}.\frac{2b}{a}$

  $= \frac{\frac{a^2 - c^2}{3} + c^2 - a^2}{ac} = \frac{2(c^2 - a^2)}{3ac} =$ R.H.S.

\item The diagram is given below:

  \startplacefigure
    \externalfigure[18_5.pdf]
  \stopplacefigure

  Here $BD = DC.$ Let $AE\perp BC$

  Now, $AC^2 - AB^2 = (AE^2 + EC^2) - (AE^2 + BE^2)$

  $= EC^2 - BE^2 = (EC + BE)(EC - BE) = BE[(ED + DC) - (BD - ED)]$

  $= 2BE.ED[\because BD=DC]$

  Also, $4\Delta = 4.\frac{1}{2}BC.AE = 2BC.AE$

  $\frac{AC^2 - AB^2}{4\Delta} = \frac{2BE.ED}{2BC.AE} = \frac{ED}{AE} = \cot\theta$

\item The diagram is given below:

  \startplacefigure
    \externalfigure[18_6.pdf]
  \stopplacefigure

  Let $\angle DBA = \alpha$ then

  $\angle BDC = \alpha [\because AB\parallel DC]$

  $\Rightarrow \angle DAB = \pi - (\theta + \alpha)$

  Now applying sine rule in $\triangle ADB$

  $\frac{AB}{\sin \theta} = \frac{\sqrt{p^2 + q^2}}{\sin(\pi  - \theta - \alpha)}$

  $AB = \frac{\sqrt{p^2 + q^2}\sin\theta}{\sin(\theta + \alpha)}$

  $= \frac{\sqrt{p^2 + q^2}\sin\theta}{\sin\theta\cos\alpha + \sin\alpha\cos\theta}$


  $= \frac{(p^2 + q^2)\sin\theta}{p\cos\theta + q\sin\theta}$

\item The diagram is given below:

  \startplacefigure
    \externalfigure[18_7.pdf]
  \stopplacefigure

  $\angle AOB = \pi - B$ and $\angle BOC = \pi - C$
  Applying sine rule in triangle $AOB,$ we have

  $\frac{OB}{\sin\theta} = \frac{c}{\sin(\pi - B)} \therefore OB = \frac{c\sin\theta}{\sin B}$

  Similarly, in triangle $BOC,$

  $OB = \frac{a\sin(C - \theta)}{\sin C}$

  $\Rightarrow \frac{2R\sin C\sin\theta}{\sin B} = \frac{2R\sin A\sin(C - \theta)}{\sin C}$

  $\frac{\sin C}{\sin A\sin B} = \frac{\sin(C - \theta)}{\sin C\sin\theta}$

  $\frac{\sin(A + B)}{\sin A\sin B} = \frac{\sin C\cos\theta - \cos C\sin\theta}{\sin C\sin\theta}$

  $\cot B + \cot A = \cot\theta - \cot C$

  $\cot\theta = \cot A + \cot B + \cot C$
\item The diagram is given below:

  \startplacefigure
    \externalfigure[18_8.pdf]
  \stopplacefigure

  From figure, $\angle ADC = 90^\circ + B$

  By applying $m:n$ rule in triangle $ABC,$ we get

  $(1 + 1)\cot(90^\circ + B) = 1.\cot90^\circ - 1.\cot(A-90^\circ)$

  $-2\tan B = 0 - \cot[-(90^\circ - A)]$

  $-2\tan B = \tan A \Rightarrow \tan A + 2\tan B = 0$

\item Given, $\cot A + \cot B + \cot C = \sqrt{3}$

  Squaring, $\cot^2A + \cot^2B + \cot^2C + 2\cot A\cot B + 2\cot B\cot C + 2\cot C\cot A = 3$

  Since $A + B + C = \pi \Rightarrow A + B = \pi - C$

  $\cot(A + B) = \cot(\pi - C)$

  $\frac{\cot A\cot B - 1}{\cot A + \cot B} = -\cot C$

  $\cot A\cot B + \cot B\cot C + \cot C\cot A = 1$

  $\Rightarrow \cot^2A + \cot^2B + \cot^2C + 2\cot A\cot B + 2\cot B\cot C + 2\cot C\cot A = 3(\cot A\cot B + \cot B\cot C +
  \cot C\cot A)$

  $\frac{1}{2}[(\cot A - \cot B)^2 + (\cot B - \cot C)^2 + (\cot C - \cot A)^2] = 0$

  $\Rightarrow \cot A = \cot B = \cot C$

  $\Rightarrow A = B = C$ i.e. triangle is equilateral.

\item Given, $(a^2 + b^2)\sin(A - B) = (a^2 - b^2)\sin(A + B)$

  $(\sin^2A + \sin^2B)\sin(A - B) = (\sin^2A - \sin^2B)\sin(A + B)$

  $[\because \sin^2A - \sin^2B = \sin(A + B)\sin(A - B)]$

  $\Rightarrow (\sin^2A + \sin^2B)\sin(A - B) = \sin^2(A + B)\sin(A - B)$

  $\Rightarrow \sin(A - B)[\sin^2A + \sin^2B - \sin^2C] = 0$

  Either $\sin(A - B) = 0$ or $\sin^2A + \sin^2B - \sin^2C = 0$

  $A = B$ or $a^2 + b^2 - c^2 = 0$

  Thus, triangle is either isosceles or right angled.

\item R.H.S. $= c(\cos A\cos\theta + \sin A\sin\theta) + a(\cos C\cos\theta - \sin C\sin\theta)$

  $= \cos\theta(c\cos A + a\cos C) + \sin\theta(c\sin A - a\sin C)$

  $=b\cos\theta + \sin\theta\left(c.\frac{a}{2R} - a.\frac{c}{2R}\right)$

  $= b\cos\theta =$ L.H.S.

\item The diagram is given below:

  \startplacefigure
    \externalfigure[18_9.pdf]
  \stopplacefigure

  $\frac{b}{\sin(B + \theta)} = \frac{a}{2\sin(A - \theta)}$

  $\frac{c}{\sin[\pi -(B + \theta)]} = \frac{a}{2\sin\theta}$

  $\Rightarrow b\sin(A - \theta) = c\sin\theta$

  $\Rightarrow \sin B(\sin A\cos\theta - \cos A\sin\theta) = \sin C\sin\theta = (\sin A\cos B + \sin B \cos A)\sin\theta$

  $\Rightarrow \cot\theta = 2\cot A + \cot B$

\item The diagram is given below:

  \startplacefigure
    \externalfigure[18_10.pdf]
  \stopplacefigure

  $\Delta ABC = \Delta ABD + \Delta ACD$

  $\frac{1}{2}bc\sin A = \frac{1}{2}AD.c\sin\frac{A}{2} + \frac{1}{2}AD.b\sin\frac{A}{2}$

  $\Rightarrow AD = \frac{2bc}{b + c}\cos\frac{A}{2}$

\item The diagram is given below:

  \startplacefigure
    \externalfigure[18_11.pdf]
  \stopplacefigure

  $\angle ACB = \pi - (\alpha + \beta + \gamma)$

  $\sin ACB = \sin(\alpha + \beta + \gamma)$

  In $\triangle ABC$

  $\frac{AB}{\sin ACB} = \frac{AC}{\sin\gamma}$

  $AC = \frac{a\sin\gamma}{\sin(\alpha + \beta + \gamma)}$

  In $\triangle ACD$

  $\frac{CD}{\sin\alpha} = \frac{AC}{\sin\beta}$

  $CD = \frac{a\sin\alpha\sin\gamma}{\sin\beta\sin(\alpha + \beta + \gamma)}$

\item Given, $2\cos A = \frac{\sin B}{\sin C}$

  $2\cos A\sin C = \sin B = \sin[\pi - (A + C)] = \sin(A + C)$

  $2\cos A\sin C = \sin A\cos C + \cos A\sin C$

  $\cos A\sin C = \sin A\cos C\Rightarrow \tan A = \tan C$

  $\Rightarrow A = C$

  Thus, the triangle is isosceles.

\item Let such angles be $A$ and $B.$ Then,

  $\cos A = \frac{1}{a}$ and $\cos B = \frac{1}{b}$

  $\Rightarrow \sin A\cos A = \sin B\cos B$

  $\sin 2A = \sin 2B$ or $\sin 2A = \sin[\pi - 2B]$

  $A = B$ or $A + B = \frac{\pi}{2} \Rightarrow C = \frac{\pi}{2}$

  Thus, the triangle is either isosceles or right angled.

\item Given $a\tan A + b\tan B = (a + b)\tan \frac{A + B}{2}$

  $\Rightarrow a\left(\frac{\sin A}{\cos A} - \frac{\sin \frac{A + B}{2}}{\cos\frac{A + B}{2}}\right) =
  b\left(\frac{\sin\frac{A + B}{2}}{\cos \frac{A + B}{2}} - \frac{\sin B}{\cos B}\right)$

  $\Rightarrow a.\frac{\sin\frac{A - B}{2}}{\cos A\cos \frac{A + B}{2}} = b.\frac{\sin\frac{A - B}{2}}{\cos B\cos\frac{A +
      B}{2}}$

  $\Rightarrow \tan A = \tan B \Rightarrow A = B$

  Thus, the triangle is isosceles.

\item Given, $\frac{\tan A - \tan B}{\tan A + \tan B} = \frac{c - b}{c}$

  $\Rightarrow \frac{\frac{\sin A}{\cos A} - \frac{\sin B}{\cos B}}{\frac{\sin A}{\cos A} + \frac{\sin B}{\cos B}} =
  \frac{\sin C - \sin B}{\sin C}$

  $\Rightarrow \frac{\sin A\cos B - \sin B\cos A}{\sin A\cos B + \sin B\cos A} = \frac{\sin (A + B) - \sin B}{\sin (A + B)}$

  $\Rightarrow \frac{\sin A\cos B - \sin B\cos A}{\sin(A + B)} = \frac{\sin(A + B) - \sin B}{\sin(A + B)}$

  $\Rightarrow \sin A\cos B - \sin B\cos A = \sin A\cos B + \sin B\cos A - \sin B$

  $\Rightarrow \sin B = 2\sin B\cos A \Rightarrow \cos A = \frac{1}{2} \Rightarrow A = 60^\circ.$

\item We know that $\cos C = \frac{a^2 + b^2 - c^2}{2ab}$

  Given, $c^4 - 2(a^2 + b^2)c^2 + a^4 + a^2b^2 + b^4 = 0$

  $\Rightarrow a^4 + b^4 + c^4 - 2a^2c^2 - 2b^2c^2 + 2a^2b^2 = a^2b^2$

  $(a^2 + b^2 -c^2)^2 = a^2b^2 \Rightarrow a^2 + b^2 + c^2 = \pm ab$

  $\Rightarrow \cos C = \pm\frac{1}{2} \Rightarrow A = 60^\circ$ or $120^\circ$

\item Given, $\frac{\cos A + 2\cos C}{\cos A + 2\cos B} = \frac{\sin B}{\sin C}$

  $\Rightarrow \cos A(\sin C - \sin B) = 2\sin B\cos B - 2\sin C\cos C = \sin 2B - \sin 2C$

  $\Rightarrow 2\cos A.\cos\frac{B + C}{2}\sin\frac{C - B}{2} = 2\cos(B + C)\sin(B - C)$

  $\cos A.\cos\frac{B + C}{2}\sin\frac{C - B}{2} = -2\cos A.\sin\frac{B - C}{2}.\cos\frac{B - C}{2}$

  If $B = C$ then above it $0 = 0$ i.e. triangle is isosceles.

  If $A = 90^\circ$ then above is $0 =0$ i.e. triangle is right angled.

\item $\because \tan\frac{A}{2}, \tan\frac{B}{2}, \tan\frac{C}{2}$ are in A.P.

  $\tan\frac{A}{2} - \tan\frac{B}{2} = \tan\frac{B}{2} - \tan \frac{C}{2}$

  $\frac{\sin\left(\frac{A}{2} - \frac{B}{2}\right)}{\cos\frac{A}{2}\cos\frac{B}{2}} = \frac{\sin\left(\frac{B}{2} -
  \frac{C}{2}\right)}{\cos\frac{B}{2}\cos\frac{C}{2}}$

  $\sin\left(\frac{A}{2} - \frac{B}{2}\right)\cos\frac{C}{2} = \sin\left(\frac{B}{2} - \frac{C}{2}\right)\cos\frac{A}{2}$

  $\sin\left(\frac{A}{2} - \frac{B}{2}\right)\sin\left(\frac{A}{2} + \frac{B}{2}\right) = \sin\left(\frac{B}{2} -
  \frac{C}{2}\right)\sin\left(\frac{B}{2} + \frac{C}{2}\right)$

  $\Rightarrow \cos B - \cos A = \cos C - \cos B$

  Thus, $\cos A, \cos B, \cos C$ are in A.P.

\item Given, $a\cos^2\frac{C}{2} + c\cos^2\frac{A}{2} = \frac{3b}{2}$

  $a.\frac{s(s - c)}{ab} + c.\frac{s(s - a)}{bc} = \frac{3b}{2}$

  $\frac{s}{b}[2s - a - c] = \frac{3b}{2}$

  $2s = 3b \Rightarrow a + c = 2b$

  We have to prove that $\cot\frac{A}{2} + \cot\frac{C}{2} = 2\cot\frac{B}{2}$

  L.H.S. $= \frac{s(s - a)}{\Delta} + \frac{s(s - c)}{\Delta}$

  $= \frac{s}{\Delta}(2s - a - c) = \frac{2s(s - b)}{\Delta} = 2\cot\frac{B}{2} =$ R.H.S.

\item Given, $a^2, b^2, c^2$ are in A.P.

  $\Rightarrow b^2 - a^2 = c^2 - b^2$

  $\Rightarrow \sin^2B - \sin^2A = \sin^2C - \sin^2B$

  $\Rightarrow \sin(B + A)\sin(B - A) = \sin(C + B)\sin(C - B)$

  $\Rightarrow \sin C\sin(B - A) = \sin A\sin(C - B)$

  $\Rightarrow \frac{\sin A\cos B - \cos B\sin A}{\sin A\sin B} = \frac{\sin B\cos C - \sin C\cos B}{\sin B\sin C}$

  $\Rightarrow \cot B - \cot A = \cot C - \cot B$

  $\therefore \cot A, \cot B, \cot C$ are in A.P.

\item Since $A, B, C$ are in A.P. $\Rightarrow 2B = A + C \Rightarrow A + B + C = 3B = 180^\circ \Rightarrow B = 60^\circ$

  Given, $2b^2 = 3c^2$

  $2.\sin^2B = 3.\sin^2C \Rightarrow \sin C = \pm\frac{1}{\sqrt{2}}$

  $\sin C\neq -\frac{1}{\sqrt{2}}$ because $C < 120^\circ$

  $\sin C = \frac{1}{\sqrt{2}} \Rightarrow C = 45^\circ$

  $\Rightarrow A = 75^\circ$

\item Given, $\tan\frac{A}{2}, \tan\frac{B}{2}, \tan\frac{C}{2}$ are in H.P.

  $\Rightarrow \cot \frac{A}{2}, \cot\frac{B}{2}, \cot\frac{C}{2}$ are in A.P.

  $\cot\frac{B}{2} - \cot\frac{A}{2} = \cot\frac{C}{2}- \cot\frac{B}{2}$

  $\frac{s(s - b) - s(s - a)}{\Delta} = \frac{s(s - c) - s(s - b)}{\Delta}$

  $a - b = b - c$

  $a, b, c$ are in A.P.

\item Given, $\frac{\sin A}{\sin C} = \frac{\sin(A - B)}{\sin(B - C)}$

  $\frac{\sin A}{\sin C} = \frac{\sin A\cos B - \sin B \cos A}{\sin B\cos C - \sin C\cos B}$

  $\sin A\sin C\cos C + \sin B\sin C\cos A = 2\sin A\sin C\cos B$

  $\sin B\sin(A + C) = 2\sin A\sin C\cos B$

  $\sin^2B = 2\sin A\sin C\cos B$

  $\cos B = \frac{b^2}{2ac} = \frac{a^2 + c^2 - b^2}{2ac}$

  $\Rightarrow c^2 + a^2 = 2b^2$

  Thus, $a^2, b^2, c^2$ are in A.P.

\item Given, $2\sin B = \sin A + \sin C$

  $4\sin\frac{B}{2}\cos\frac{B}{2} = 2\sin\frac{A + C}{2}\cos\frac{A - C}{2} = 2\cos\frac{B}{2}\cos\frac{A - C}{2}$

  $2\sin\frac{B}{2} = 3\cos\frac{A + C}{2} = \cos\frac{A - C}{2}$

  $3\sin\frac{A}{2}\sin\frac{C}{2} = \cos\frac{A}{2}\cos\frac{C}{2}$

  $3\tan\frac{A}{2}\tan\frac{C}{2} = 1$

\item Given, $a^2, b^2, c^2$ are in A.P.

  $b^2 - a^2 = c^2 - b^2$

  $\sin^2B - \sin^2A = \sin^2C - \sin^2B$

  $\sin(A + B)\sin(A - B) = \sin(B + C)\sin(B - C)$

  $\sin C\sin(A - B) = \sin A\sin(B - C)$

  $\cos B -\cot A\sin B = \sin B\cot C - \cos B$

  $2\cos B = \sin B(\cot A + \cot C)$

  $2\cot B = \cot A + \cot C$

  $\therefore \cot A, \cot B, \cot C$ are in A.P.

  $\therefore \tan A,\tan B, \tan C$ are in H.P.

\item We have proven in previous problem that $\therefore \cot A, \cot B, \cot C$ are in A.P.

\item Since $A, B, C$ are in A.P. $\Rightarrow 2B = A + C \Rightarrow A + B + C = 3B = 180^\circ \Rightarrow B =
  60^\circ$

  $b:c = \sqrt{3}:\sqrt{2} \Rightarrow \sin C = \frac{\sqrt{3}}{2}.\sqrt{\frac{2}{3}} = \frac{1}{\sqrt{2}}$

  $\Rightarrow C = 45^\circ \Rightarrow A = 75^\circ$

\item Let the sides are $a, b, c$ then $2b = a + c.$ Also, let $a$ to be greatest and $c$ to be smallest
  side.

  Then, $A = 90 + C$ then $90 + C + B + C = 180 \Rightarrow B = 90 -2C$

  $\frac{a}{\sin(90 + C)} = \frac{b}{\sin(90 - 2C)} = \frac{c}{\sin C} = 2R$

  $4R\cos2C = 2R\cos C + 2R\sin C$

  $2\cos 2C = \cos C + \sin C$

  Squaring, we get

  $4(1 - \sin^22C) = 1 + \sin 2C \Rightarrow \sin2C = \frac{3}{4}$ when $1 + \sin 2C \neq 0$

  When $1 + \sin 2C = 0 \Rightarrow C = \frac{3\pi}{4}$ which is not possible.

  $\because \sin 2C = \frac{3}{4} \Rightarrow \cos 2C = \frac{\sqrt{7}}{4}$

  Now $\sin C$ and $\cos C$ can be found and ratio can be evaluated.

\item $\because a, b, c$ are in A.P. $2b = a + c \Rightarrow a = 2b - c$

  $\cos A = \frac{b^2 + c^2 - a^2}{2bc} = \frac{b^2 + c^2 -4b^2 -c^2 +4bc}{2bc}$

  $= \frac{4bc - 3b^2}{2bc} =\frac{4c - 3b}{2c}$

\item The diagram is given below:

  \startplacefigure
    \externalfigure[18_12.pdf]
  \stopplacefigure

  Let $AB = 2, AD = 5, BC = 3$ annd $CD=x$

  Since it is cyclic quadrilateral $\angle C = 120^\circ$

  Applying cosine rule in $\Delta ABD,$ we have

  $\cos 60^\circ = \frac{AB^2 + AD^2 - BD^2}{2.AB.AD} \Rightarrow BD^2 = 19$

  Applying cosine rule in $\Delta BCD,$ we have

  $\cos 120^\circ = \frac{BC^2 + CD^2 - BD^2}{2.BC.BD} \Rightarrow x^2 + 3x -10 = 0$

  $x = -5, 2$ but $x$ cannot be -ve. $\therefore x = 2$

\item Given $(a + b + c)(b + c - a) = 3bc$

  $b^2 + c^2 - a^2 + 2bc = 3bc$

  $\frac{b^2 + c^2 - a^2}{2bc} = \frac{1}{2}$

  $\cos A = \cos 60^\circ$

  $A = 60^\circ$

\item Since $AD$ is the median

  $\therefore AB^2 + AC^2 = 2BD^2 + 2AD^2$

  $\Rightarrow b^2 + c^2 = \frac{a^2}{4} + 2AD^2$

  $4AD^2 = b^2 + c^2 + (b^2 + c^2 - a^2)$

  $\cos A = \frac{1}{2} = \frac{b^2 + c^2 - a^2}{2bc}$

  $\Rightarrow 4AD^2 = b^2 + c^2 + bc$

\item The diagram is given below:

  \startplacefigure
    \externalfigure[18_13.pdf]
  \stopplacefigure

  Since $AD$ is the median

  $\therefore AB^2 + AC^2 = 2BD^2 + 2AD^2$

  $\Rightarrow AD^2 = \frac{2b^2 + 2c^2 - a^2}{4}$

  $AO = \frac{2}{3}AD = \frac{2}{3}.\frac{1}{2}\sqrt{2b^2 + 2c^2 - a^2}$

  Similalry $BO = \frac{1}{3}\sqrt{2c^2 + 2a^2 - b^2}$

  $\because \angle AOB = 90^\circ$

  $\therefore AO^2 + BO^2 = AB^2$

  $\Rightarrow a^2 + b^2 = 5c^2$

\item Given, $\frac{\tan A}{1} = \frac{\tan B}{2} = \frac{\tan C}{3} = k$

  Since $A, B, C$ are the angles of a triangle

  $\therefore \tan A + \tan B + \tan C = \tan A\tan B\tan C$

  $k + 2k + 3k = k.2k.3k \Rightarrow k = 1$ as if $k = -1$ sum of angles will be greater than $180^\circ.$

  $\tan A = 1 \Rightarrow \sin A = \frac{1}{\sqrt{2}}$

  $\tan A = 2 \Rightarrow \sin A = \frac{2}{\sqrt{5}}$

  $\tan A = 3 \Rightarrow \sin A = \frac{3}{\sqrt{10}}$

  $\frac{a}{\sin A} = \frac{b}{\sin B} = \frac{c}{\sin C}$

  $\sqrt{2}a = \frac{\sqrt{5}b}{2} = \frac{\sqrt{10}c}{3}$

  $6\sqrt{2}a = 3\sqrt{5}b = 2\sqrt{10}c$

\item For a triangle sides are positive i.e. $a > 0, b > 0, c >0$ where $a,b,c$ are the sides.

  $2x + 1 > 0 \Rightarrow x > -\frac{1}{2}$

  $x^2 - 1>0 \Rightarrow x > 1$ because side cannot be negative.

  $x^2 + x + 1 > 0~\forall x>1$

  $a - b = x(x - 1) > 0 \Rightarrow a > b$

  $a - c = x + 2 > 0 \Rightarrow a > c$

  Hence $a$ is the greatest side.

  $\cos A = \frac{b^2 + c^2 - a^2}{2bc} = \frac{(2x + 1)^2 + (x^2 - 1)^2 - (x^2 + x + 1)}{2(2x + 1)(x^2 - 1)}$

  $= -\frac{1}{2}$

  $\Rightarrow A = 120^\circ$

\item Let the sides be $x, x+1, x+2$ where $x > 0$ and is a natural number. Let the smallest angle be $\theta$

  $\angle C = \theta \therefore \angle A = 2\theta$

  Applying sine law

  $\frac{x}{\sin\theta} = \frac{x + 1}{\sin(\pi - 3\theta)} = \frac{x + 2}{\sin2\theta}$

  $\Rightarrow \frac{x}{\sin\theta} = \frac{x + 2}{\sin2\theta} \Rightarrow 2\cos\theta = \frac{x + 2}{x}$

  $\Rightarrow \frac{x}{\sin\theta} = \frac{x + 1}{\sin3\theta} = \frac{x + 1}{3\sin\theta - 4\sin^2\theta}$

  $\Rightarrow 3 - 4\sin^2\theta = \frac{x + 1}{x}$

  $\Rightarrow 4\cos^2\theta = \frac{2x + 1}{x} = \frac{(x + 2)^2}{x^2}$

  $\Rightarrow x^2 - 3x - 4 = 0$

  $x = 4, -1$ but $-1$ is not a natural number so $x = 4.$ Hence sides are $4,5,6.$

\item Given, $a = 6$ cm, $\Delta = 12$ sq. cm. and $B - C = 60^\circ$

  $\Delta = \frac{1}{2}ab\sin C = \frac{1}{2}a.k\sin B\sin C$

  $= \frac{1}{2}.a.\frac{a}{\sin A}\sin B\sin C$

  $\Delta = \frac{1}{2}a^2\sin B\sin C = \frac{18\sin B\sin C}{\sin A}$

  $\Rightarrow \frac{2}{3} = \frac{2\sin B\sin C}{2\sin A} = \frac{\cos(B - C) - \cos(B + C)}{2\sin A}$

  $\Rightarrow \frac{2}{3} = \frac{\cos60^\circ - \cos(\pi - A)}{2\sin A}$

  $\Rightarrow 8\sin A - 6\cos A = 3$

\item Given, $\cos\theta = \frac{a}{b + c} \Rightarrow 1 + \cos\theta = \frac{a + b + c}{b + c}$

  $\Rightarrow 2\cos^2\frac{\theta}{2} = \frac{a + b + c}{b + c}$

  $\sec^2\frac{\theta}{2} = \frac{2(b + c)}{a + b + c}$

  $1 + \tan^2\frac{\theta}{2} = \frac{2(b + c)}{a + b + c}$

  Similarly, $1 + \tan^2\frac{\phi}{2} = \frac{2(c + a)}{a + b + c}$

  and $1 + \tan^2\frac{\psi}{2} = \frac{2(a + b)}{a + b + c}$

  Adding, we get $3 + \tan^2\frac{\theta}{2} + \tan^2\frac{\phi}{2} + \tan^2\frac{\psi}{2} = \frac{4(a + b + c)}{a + b +
    c}$

  $\Rightarrow \tan^2\frac{\theta}{2} + \tan^2\frac{\phi}{2} + \tan^2\frac{\psi}{2} = 1$

\item Since $C$ is the angle of a triangle, $\sin C\leq 1$

  $\therefore \cos A\cos B + \sin A\sin B\geq \cos A\cos B + \sin A\sin B\sin C$

  $\Rightarrow \cos(A - B)\geq 1$

  But $cos(A - B)$ cannot be greater than $1. \therefore \cos(A - B) = 1\Rightarrow A = B$

  Now, $\cos A\cos B + \sin A\sin B\sin C = 1$

  $\Rightarrow \cos^2A + \sin^2A\sin C = 1$

  $\Rightarrow \sin C= 1 \Rightarrow C=90^\circ\Rightarrow A = B = 45^\circ$

  $\Rightarrow a:b:c = \sin A:\sin B: \sin C = 1:1:\sqrt{2}$

\item From the question, $\sin A\sin B\sin C = p$ and $\cos A\cos B\cos C = q$

  $\therefore \tan A\tan B\tan C = \frac{p}{q}$

  Since we are dealing with a triangle $\therefore \tan A + \tan B + \tan C = \tan A\tan B\tan C$

  $\Rightarrow \tan A + \tan B + \tan C = \frac{p}{1}$

  Now, $\tan A\tan B + \tan B\tan C + \tan C\tan A = \frac{\sin A\sin B\cos C + \sin B\sin C\cos A + \sin A\sin C\cos
    B}{\cos A\cos B\cos C}$

  [ We know that in a triangle $2\sin A\sin B\cos C = \sin^2A + \sin^2B - \sin^2C$ ]

  $\Rightarrow  \frac{1}{2q}[(\sin^2A + \sin^2B - \sin^2C) + (\sin^2B + \sin^2C - \sin^2A) + (\sin^2C + \sin^2A -
    \sin^2B)]$

  $= \frac{1}{2q}[\sin^2 + \sin^2B + \sin^2C]$

  $= \frac{1}{2q}\left[\frac{(1 - \cos2A) + (1 - \cos2B) + (1 - \cos2C)}{2}\right]$

  $= \frac{1}{4q}[4 + 4\cos A\cos B\cos C] = \frac{1 + q}{q}$

  Thus, we see that $\tan A, \tan B, \tan C$ are roots of the given equation.

\item Given, $\sin^3\theta = \sin(A - \theta)\sin(B - \theta)\sin(C - \theta)$

  $4\sin^3\theta = 2\sin(A - \theta)[2\sin(B - \theta)\sin(C - \theta)]$

  $= 2\sin(A - \theta)[\cos(B - C) - \cos(B + C - 2\theta)]$

  $= 2\sin(A - \theta)\cos(B - C) -2\sin(A - \theta)\cos(B + C - 2\theta)$

  $= \sin(A + B - C - \theta) + \sin(A + C - \theta - B) - \sin(A + B + C - 3\theta) + \sin(\pi - 2B - \theta)$

  $\sin 3\theta + 4\sin^3\theta = \sin(2A + \theta) + \sin(2B + \theta) + \sin(2C + \theta)$

  $3\sin\theta = (\sin2A + \sin2B + \sin2C)\cos\theta + (\cos2A + \cos2B + \cos2C)\sin\theta$

  [ $\because \sin2A + \sin2B + \sin 2C = 4\sin A\sin B\sin C$ when $A + B + C = \pi$ ]

  $(1 - \cos 2A) + (1 - \cos 2B) + (1 - \cos 2C)\sin\theta = 4\sin A\sin B\sin C.\cos\theta$

  $2[(\sin^2A + \sin^2B + \sin^2C)\sin\theta] = 4\sin A\sin B\sin C\cos\theta$

  $2\sin\theta[(\sin^2A + \sin^2B - \sin^2C) + (\sin^2B + \sin^2C - \sin^2A) + (\sin^2C + \sin^2A - \sin^2B)] = 4\sin A\sin
  B\sin C\cos\theta$

  [ $\because \sin^2 + \sin^2B - \sin^2C = 2\sin A\sin B\cos C$ ]

  $\Rightarrow \cot\theta = \cot A + \cot B + \cot C$

\item From question $\frac{b}{c} = r \therefore b = cr$

  Let $AD\perp BC$ and let $AD = h$

  We have to prove that $h\leq \frac{ar}{1 - r^2}$

  $\Delta ABC = \frac{1}{2}c.cr.\sin A = \frac{1}{2}ah$

  $h = \frac{c^2r\sin A}{a}$

  $\cos A = \frac{c^2 + c^2r^2 - a^2}{2.c.cr}$

  $c^2 = \frac{a^2}{1 + r^2 - 2r\cos A}$

  $\Rightarrow h = \frac{a^2r\sin C}{a(1 + r^2 - 2r\cos a)} = \frac{ar\sin A}{1 + r^2 - 2r\cos A}$

  $= \frac{ar.\frac{2\tan\frac{A}{2}}{1 + \tan^2\frac{A}{2}}}{1 + r^2 - 2r\frac{1 - \tan^2\frac{A}{2}}{1 +
      \tan^2\frac{A}{2}}}$

  $\Rightarrow (1 + r^2)\tan^2\frac{A}{2} -\frac{2ar}{h}\tan\frac{A}{2} + (1 - r)^2 = 0$

  This is a quadratic equation in $\tan\frac{A}{2}$ and since it will be read $D \geq 0$

  $\Rightarrow \frac{4a^2r^2}{h^2} - 4(1 + r)^2(1 - r)^2\geq 0$

  $h \leq \frac{ar}{1 - r^2}$

\item Given $b.c = k^2,$ now

  $\cos A = \frac{b^2 + c^2 - a^2}{2bc} \Rightarrow 2k^2\cos A = b^2 + \frac{k^4}{b^2} - a^2$

  $b^4 - (a^2 + 2k^2\cos A)b^2 + k^4 = 0$ which is a quadratic equation in $b^2.$

  The triangle will not exists if discriminant is less than zero for above equation because then $b$ will become a complex
  number.

  $\Rightarrow (a^2 + 2k^2\cos A)^2 - 4k^4 < 0$

  $\Rightarrow [a^2 + 2k^2(1 + \cos A)][a^2 - 2k^2(1 - \cos A)] < 0$

  $\Rightarrow \left(a^2 + 4k^2\cos^2\frac{A}{2}\right)\left(a^2 - 4k^2\sin^2\frac{A}{3}\right) < 0$

  $\Rightarrow a^2 - 4k^2\sin^2\frac{A}{2} , 0 \left[\because a^2 + 2k^2\cos^2\frac{A}{2} > 0\right]$

  $\Rightarrow \left(a + 2k\sin\frac{A}{2}\right)\left(a - 2k\sin\frac{A}{2}\right) < 0$

  $\Rightarrow a - 2k\sin\frac{A}{2} < 0 \left[\because a + 2k\sin\frac{A}{2} > 0\right]$

  $\Rightarrow a < 2k\sin\frac{A}{2}$

\item The diagram is given below:

  \startplacefigure
    \externalfigure[18_14.pdf]
  \stopplacefigure

  The diagram is a top view. Let $O$ be the top point and $O'$ the center of ring which is $12$ cm below
  $O$ in the diagram(not shown).

  In triangle $OO'A, AO' = 5$ cm, $OO' = 12$ cm

  $AO = \sqrt{12^2 + 5^2} = 13$ cm

  Now sides of a regular hexagon are equal to the circumscribing circle. $AB= 5$ cm.

  $\cos AOB = \frac{13^2 + 13^2 - 5^2}{2.13.13} = \frac{313}{338}$

\item Given, $2b = 3a$ and $\tan^2\frac{A}{2}= \frac{3}{5}$

  $\cos A = \frac{1}{\sqrt{1 + \tan^2\frac{A}{2}}} = \frac{1}{\sqrt{1 + \frac{3}{5}}} = \sqrt{\frac{5}{8}}$

  $\cos A = \sqrt{\frac{5}{8}} = \frac{b^2 + c^2 - a^2}{2bc}$

  $= \frac{b^2 + c^2 - \frac{4b^2}{9}}{2bc}$

  $\Rightarrow \frac{\sqrt{8}(5b^2 + 9c^2)}{9} = 2\sqrt{5}bc$

  $\Rightarrow 9\sqrt{8}c^2 - 18\sqrt{5}bc + 5\sqrt{8}b^2 = 0$

  $\Rightarrow c = \frac{8\sqrt{5}}{6\sqrt{8}}, \frac{4\sqrt{5}}{6\sqrt{8}}$

  Thus, one value is double of the other.

\item Let the angles are $k, 2k,7k$ degrees. Then $k + 2k + 7k = 180^\circ \Rightarrow k = 18^\circ$

  So greatest angle is $126^\circ$ and smallest is $18^\circ.$

  Ratio of greatest to least side is given by $\sin126^\circ;;\sin18^\circ$

  $= \cos 36^\circ:\sin18^\circ = \sqrt{5} + 1:\sqrt{5} - 1$

\item Let $AF = f, BG = g, CH = h$

  Area of $\triangle ABC =$ Area of $\triangle ABF$ + Area of $\triangle ACF$

  $\frac{1}{2}bc\sin A = \frac{1}{2}.2\sin\frac{A}{2}\cos\frac{A}{2} = \frac{1}{2cf}\sin\frac{A}{2} +
  \frac{1}{2}bf\sin\frac{A}{2}$

  $\Rightarrow 2bc\cos\frac{A}{2} = (b + c)f$

  $\frac{1}{f}\cos\frac{A}{2} = \frac{1}{2}\left(\frac{1}{b} + \frac{1}{c}\right)$

  Similarly, $\frac{1}{g}\cos\frac{B}{2} = \frac{1}{2}\left(\frac{1}{a} + \frac{1}{c}\right)$

  and, $\frac{1}{h}\cos\frac{C}{2} = \frac{1}{2}\left(\frac{1}{a} + \frac{1}{b}\right)$

  Adding these three we obtain desired result.

\item Since $BD = DE = EC$ each will be equal to $\frac{5}{3}.$ Clearly the triangle is right angled because $3^2 +
  5^2 = 5^2$

  $\cos C = \frac{4}{5}$

  In $\triangle ACE, \cos C = \frac{CE^2 + 4^2 - AE^2}{2.CE.4} = \frac{\frac{25}{9} + 16 - AE^2}{2.\frac{5}{3}.4}$

  $\Rightarrow  \frac{169 - 9AE^2}{9}.\frac{3}{40} = \frac{4}{5}$

  $\Rightarrow AE^2 = \frac{73}{9}$

  $\cos\theta = \frac{AE^2 + AC^2 - CE^2}{2.AE.AC} = \frac{8}{\sqrt{73}}$

  $\tan\theta = \sqrt{sec^2\theta - 1} = \sqrt{\frac{73}{64} - 1} = \frac{3}{8}$

\item Let $O$ be the centroid i.e. point of intersection of medians.

  From geometry, we know that area of $\triangle ABC = 3\times$ area of $\triangle AOC$

  We also know that centroid divides median in the ratio of $2:1$ i.e. $AO = \frac{10}{3}$

  Applying sine rule in $\triangle AOC$

  $\frac{OC}{\sin\frac{\pi}{8}} = \frac{AO}{\sin\frac{\pi}{4}}$

  $OC = \frac{10}{3}\frac{\sin\frac{\pi}{8}}{\sin\frac{\pi}{4}}$

  Area of $\triangle AOC = \frac{1}{2}AO.OC\sin AOC$

  $= \frac{1}{2}\frac{10}{3}.\frac{10}{3}.\frac{\sin\frac{\pi}{8}}{\sin\frac{\pi}{4}}\sin\left(\frac{\pi}{2} +
  \frac{\pi}{8}\right)$

  $= \frac{25}{9}$

  $\therefore \Delta ABC = \frac{75}{9}.$

\item Let sides are $a, b, c$ then $a = 7 = \sqrt{49}$ cm, $b = 4\sqrt{3} = \sqrt{48}$ cm and $c = \sqrt{13}$
  cm.

  Clearly, $c$ is smallest and thus $C$ will be smallest.

  $\cos C = \frac{48 + 49 - 13}{2.7.4\sqrt{3}} = \frac{3}{2}$

  $\Rightarrow C = 30^\circ$

\item Let the triangle be $ABC$ having right angle at $C.$ Let $D$ be the mid-point of $AC.$

  Given that triangle is isoceles so $AC = BC$ i.e. $DC = \frac{1}{2}AC = \frac{1}{2}BC$

  Also, $\angle CAB = \angle BDA = 45^\circ$

  Let $\angle DBC = \theta$ and $\angle DBA = \phi$

  $\tan\phi = \frac{DC}{BC}= \frac{1}{2}$

  $\tan\phi = tan(45^\circ - \theta) = \frac{1 - \tan\theta}{1 + \tan\theta}$

  $\Rightarrow \tan\phi = \frac{1}{3}$

  $\therefore \cot\theta = 2, \cot\phi = 3$

\item From the given ratios we have,

  $\frac{a + b}{(1 + m^2)(1 + n^2)}= \frac{a - b}{(1 - m^2)(1 - n^2)} = \frac{c}{(1 - m^2)(1 + n^2)}$

  $\Rightarrow \frac{a + b }{c} = \frac{1 + m^2}{1 - m^2}, \frac{a - b}{c} = \frac{1 - n^2}{1 + n^2}$

  $\Rightarrow \frac{\cos\frac{A - B}{2}}{\cos\frac{A + B}{2}} = \frac{1 + m^2}{1 - m^2}, \frac{\sin\frac{A -
    B}{2}}{\sin\frac{A + B}{2}} = \frac{1 - n^2}{1 + n^2}$

  By componendo and dvidendo, we have

  $\tan\frac{A}{2}\tan\frac{B}{2} = m^2, \cot\frac{A}{2}\tan\frac{B}{2} = n^2$

  $\Rightarrow \tan^2\frac{A}{2} = \frac{m^2}{n^2}, \tan^2\frac{B}{2} = m^2n^2$

  $\Rightarrow A = 2\tan^{-1}\frac{m}{n}, B = 2\tan^{-1}mn$

  $\Delta = \frac{1}{2}bc\sin A = \frac{1}{2}bc\frac{2\tan\frac{A}{2}}{1 + \tan^2\frac{A}{2}}$

  $= \frac{mnbc}{m^2 + n^2}$

\item Since $a,b,c$ are roots of the equation $x^3 - px^2 + qx - r = 0$ therefore we have

  $a + b + c = p = 2s$ where $s$ is perimeter.

  $ab + bc + ca = q$ and $abc = r$

  $\Delta^2 = s(s- a)(s - b)(s - c)$

  $= \frac{p}{2}\left(\frac{p}{2} -a\right)\left(\frac{p}{2} - b\right)\left(\frac{p}{2} - c\right)$

  Substituting the values of we obtain the desired result.

\item Let the third side be $a$ cm. Applying cosine rule,

  $6 = a^2 + 4^2 - 2.a.4\cos30^\circ$

  $a^2 - 4\sqrt{3}a + 10 = 0$

  $a = \frac{4\sqrt{3}\pm\sqrt{48 - 40}}{2} = 2\sqrt{3} \pm \sqrt{2}$

  Both roots are positive, so two such triangles are possible.
\item The diagram is given below:

  \startplacefigure
    \externalfigure[18_15.pdf]
  \stopplacefigure

  Let $BD = DE = EC = x.$ Also let,

  $\angle BAD = \alpha, \angle DAE = \beta, EAC = \gamma, CEA = \theta$

  Given, $\tan\alpha = t_1, \tan\beta = t_2, \tan\gamma = t_3$

  Applying $m:n$ rule in $\triangle ABC,$ we get

  $(2x + x)\cot\theta = 2x\cot(\alpha + \beta) - x\cot\gamma$

  From $\triangle ADC,$ we get

  $2x\cot\theta = x\cot\beta - x\cot\gamma$

  $\Rightarrow \frac{3}{2} = \frac{2(\alpha + \beta) - \cot\gamma}{\cot\beta - \cot\gamma}$

  $\Rightarrow 3\cot\beta - 3\cot\gamma = 4\cot(\alpha + \beta) - 2\cot\gamma$

  $3\cot\beta - \cot\gamma = \frac{4(\cot\alpha\cot\beta - 1)}{\cot\alpha + \cot\beta}$

  $3\cot^2\beta - \cot\beta\cot\gamma + 3\cot\alpha\cot\beta - \cot\alpha\cot\gamma = 4\cot\alpha\cot\beta - 4$

  $4 + 4\cot^2\beta = \cot^2\beta + \cot\alpha\cot\beta + \cot\beta\cot\gamma + \cot\alpha\cot\gamma$

  $4(1 + \cot^2\beta) = (\cot\beta + \cot\alpha)(\cot\beta + \cot\gamma)$

  $4\left(1 + \frac{1}{t_2^2}\right) = \left(\frac{1}{t_1} + \frac{1}{t_2}\right)\left(\frac{1}{t_2} + \frac{1}{t_3}\right)$

\item The diagram is given below:

  \startplacefigure
    \externalfigure[18_16.pdf]
  \stopplacefigure

  Let the medians be $AD, BE$ and $CF$ meet at $O.$ From question,

  $\angle BOC=\alpha, \angle COA = \beta, \angle AOB = \gamma$

  Let $AD = p_1, BE=p_2, CF = p_3$

  $AO:OD = 2:1 \Rightarrow AO = \frac{2}{3}p_1$

  Similalrly, $OB = \frac{2}{3}p_2, OC = \frac{2}{3}p_3$

  Applying cosine rule in $\triangle AOC,$

  $\cos\beta = \frac{OA^2 + OC^2 - AC^2}{2.OA.OC} = \frac{\frac{4}{9}p_1^2 + \frac{4}{9}p_3^2 -
    b^2}{2.\frac{2}{3}p_1\frac{2}{3}p_3}$

  $\cos\beta = \frac{4p_1^2 + 4p_3^2 - 9b^2}{8p_1p_3}$

  $\Delta AOC = \frac{1}{2}.OA.OC.\sin\beta$

  $\frac{1}{3}\Delta = \frac{1}{2}\frac{2}{3}p_1\frac{2}{3}p_3\sin\beta$ where $\Delta$ is area of triangle $ABC.$

  $\sin\beta = \frac{3\Delta}{2p_1p_3}$

  $\Rightarrow \cos\beta = \frac{4p_1^2 + 4p_3^2 - 9b^2}{12\Delta}$

  $\because AD$ is mean of $\triangle ABC$

  $\therefore AB^2 + AC^2 = 2BD^2 + 2AD^2$

  $\Rightarrow b^2 + c^2 = 2\frac{a^2}{4} + 2p_1^2$

  $p_1^2 = \frac{2b^2 + 2c^2 - a^2}{4}$

  Similarly, $p_2^2 = \frac{2c^2 + 2a^2 - b^2}{4}$

  and $p_3^2 = \frac{2a^2 + 2b^2 - c^2}{4}$

  $\Rightarrow \cos\beta = \frac{(2b^2 + 2c^2 - a^2) + (2a^2 + 2b^2 - c^2) - 9b^2}{12\Delta}$

  $= \frac{a^2 + c^2 - 5b^2}{12\Delta}$

  Similarly, $\cos\alpha = \frac{b^2 + c^2 - 5a^2}{12\Delta}$

  Similarly, $\cos\gamma = \frac{a^2 + b^2 - 5c^2}{12\Delta}$

  $\cos\alpha + \cos\beta + \cos\gamma = \frac{-3(a^2 + b^2 + c^2)}{12\Delta}$

  $= -\frac{a^2 + b^2 + c^2}{4\Delta}$

  $\cot A + \cot B + \cot C = \frac{b^2 + c^2 - a^2}{2bc\sin A} + \frac{c^2 + a^2 - b^2}{2ca\sin B} + \frac{a^2 + b^2 -
    c^2}{2ab\sin C}$

  $= \frac{a^2 + b^2 + c^2}{4\Delta}$

  $\Rightarrow \cot\alpha + \cot\beta + \cot\gamma + \cot A + \cot B + \cot C = 0$

\item The diagram is given below:

  \startplacefigure
    \externalfigure[18_17.pdf]
  \stopplacefigure

  Let $AD$ be the perpendicular from $A$ on $BC.$ When $AD$ is extended it meets the circumscrbing circle
  at $E.$ Given, $DE=\alpha.$

  Since angles in the same segment are equal, $\therefore \angle AEB = \angle ACB = \angle C$

  and $\angle AEC = \angle ABC = \angle B$

  From right angled $\triangle BDE, \tan C = \frac{BD}{DE}$

  From right angled $\triangle CDE, \tan B = \frac{CD}{DE}$

  $\tan B + \tan C = \frac{a}{\alpha}$

  Similarly, $\tan C + \tan A = \frac{b}{\beta}$

  and $\tan A + \tan B = \frac{c}{\gamma}$

  Adding, we get

  $\frac{a}{\alpha} + \frac{b}{\beta} + \frac{c}{\gamma} = 2(\tan A + \tan B + \tan C)$

\item The diagram is given below:

  \startplacefigure
    \externalfigure[18_18.pdf]
  \stopplacefigure

  Let $H$ be the orthocenter of triangle $ABC.$

  From question, $HA = p, HB = q, HC = r.$

  From figure, $\angle HBD = \angle EBC = 90^\circ - C$

  $\angle HCD = \angle FCB = 90^\circ - B$

  $\therefore \angle BHC = 180^\circ - (\angle HBD + \angle HCD)$

  $= 180^\circ - [90^\circ - C + 90^\circ - B] = B + C = \pi - A$

  Similarly, $\angle AHC = \pi - B$ and $\angle AHB = \pi - C$

  Now $\Delta BHC + \Delta CHA + \Delta AHB = \Delta ABC$

  $\Rightarrow \frac{1}{2}[qr\sin BHC + rp\sin CHA + pq \sin AHB] = \Delta$

  $\Rightarrow \frac{1}{2}[qr\sin A + rp\sin B + pq\sin C] = \Delta$

  $\Rightarrow aqr + brp + cpq = abc$

\item The diagram is given below:

  \startplacefigure
    \externalfigure[18_19.pdf]
  \stopplacefigure

  Let $O$ be the center of unit circle and $A$ be the center of circle whose arc $BPC$ divides the unit circle
  in two equal parts.

  i.e area of the curve $ABPCA = \frac{1}{2}$ area of the unit circle $= \frac{\pi}{2}$

  Let the radius of this new circle be $r.$

  Then, $AC = AB = AP = r$

  $\because OB = OC = 1 \therefore \angle OCA = \angle OAC = \theta$

  Applying sine rule in $\triangle AOC,$

  $\frac{r}{\sin(\pi -2\theta)} = \frac{1}{\sin\theta}$

  $r = 2\cos\theta$

  Now area of $ABPCA = 2[$ Are of sector $ACP +$ Area of sector $OAC -$ Are of $\triangle OAC]$

  $= 2\left[\frac{1}{2}r^2\theta + \frac{1}{2}1^2(\pi - 2\theta) - \frac{1}{2}\sin(\pi -2\theta)\right]$

  $=\theta. 4\cos^2\theta + \pi - 2\theta - \sin2\theta [\because r = 2\cos\theta]$

  $= 2\theta\cos2\theta - \sin2\theta + \pi$

  $\Rightarrow \frac{\pi}{2} = 2\theta\cos2\theta - \sin2\theta + \pi$

  $\Rightarrow \frac{\pi}{2} = \sin2\theta - 2\theta\cos2\theta$

\item The diagram is given below:

  \startplacefigure
    \externalfigure[18_20.pdf]
  \stopplacefigure

  Let $EF$ be the perpendicular bisector of $BC$ and $O$ the center of the square. From question,

  Let $BF = FC = a \Rightarrow BC = EF = 2a$ and $OE=OF = a$

  Let $OP = x \Rightarrow OQ = x$

  $\Rightarrow PF = a - x, QF = a + x$

  From right angled $\triangle BPF,$

  $\tan B = \frac{PF}{BF} = \frac{a - x}{x}$

  From right angled $\triangle QFC,$

  $\tan C = \frac{a + x}{a}$

  $\Rightarrow (\tan B - \tan C)^2 = \frac{4x^2}{a^2}$

  In triangle $ABC,$

  $\tan A = \tan[\pi - (B + C)] = -\tan(B + C) = -\frac{2a^2}{x^2}$

  $\Rightarrow \tan A(\tan B - \tan C)^2 + 8 = 0$

\item The diagram is given below:

  \startplacefigure
    \externalfigure[18_21.pdf]
  \stopplacefigure

  $\because CD$ is internal bisector of $\angle C$

  $\therefore \frac{AD}{DB} = \frac{b}{a}$

  $\Rightarrow BD = \frac{ac}{a + b}$

  Since angles of the same segment are equal.

  $\therefore \angle ABE = \angle ACE = \frac{C}{2}$

  and $\angle BEC = \angle BAC = A$

  Applying sine rule in $\triangle BEC,$

  $\frac{CE}{\sin CBE} = \frac{BC}{\sin BEC} \Rightarrow CE = \frac{a\sin\left(a + \frac{C}{2}\right)}{\sin A}$

  Applying sine rule in $\triangle BDE,$

  $\frac{DE}{\sin\frac{C}{2}} = \frac{BD}{\sin A}\Rightarrow DE = \frac{ac\sin\frac{C}{2}}{(a + b)\sin A}$

  $\Rightarrow \frac{CE}{DE} = \frac{a\sin\left(B + \frac{C}{2}\right)}{ac\sin\frac{C}{2}}(a + b)$

  $\Rightarrow \frac{CE}{DE} = \frac{(a + b)\sin\left(B + \frac{C}{2}\right)}{c\sin \frac{C}{2}}$

  Now, $\frac{\sin\left(B + \frac{C}{2}\right)}{\sin\frac{C}{2}} = \frac{\sin\left(B +
    \frac{C}{2}\right).2\cos\frac{C}{2}}{2\sin\frac{C}{2}\cos\frac{C}{2}}$

  $= \frac{\sin(B + C)+ \sin B}{\sin C} = \frac{\sin A + \sin B}{\sin C} = \frac{a + b}{c}$

  Thus, $\frac{CE}{DE} = \frac{(a + b)^2}{c^2}$

\item The diagram is given below:

  \startplacefigure
    \externalfigure[20_1.pdf]
  \stopplacefigure

  $\because AD$ is the interna; bisector of angle $A,$

  $\frac{BD}{DC} = \frac{BA}{AC} = \frac{c}{b}$

  $\Rightarrow \frac{BD}{c} = \frac{DC}{b} = \frac{BD + DC}{b + c}$

  $\Rightarrow \frac{BD}{c} = \frac{a}{b + c}$

  Similarly, $\frac{BF}{a} = \frac{c}{a + b}$

  Now $\frac{\Delta BDF}{\Delta ABC} = \frac{BD.BF.\sin B}{a.c.\sin B} = \frac{ac}{(a + b)(b + c)}$

  Similarly, $\frac{\Delta CDE}{\Delta ABC} = \frac{ab}{(a + c)(b + c)}$

  and $\frac{\Delta AEF}{\Delta ABC} = \frac{bc}{(a + b)(a + c)}$

  $\therefore \frac{\Delta DEF}{\Delta ABC} = \frac{\Delta ABC - (\Delta BDF + \Delta CDE + \Delta AEF)}{\Delta ABC}$

  $= 1 - \frac{ac}{(a + b)(b + c)} - \frac{ab}{(a + c)(b + c)} - \frac{bc}{(a + b)(a + c)}$

  $= \frac{2abc}{(a + b)(b + c)(c + a)}$

  $\Delta DEF = \frac{2.\Delta .abc}{(a + b)(b + c)(c + a)}$

\item The diagram is given below:

  \startplacefigure
    \externalfigure[20_2.pdf]
  \stopplacefigure

  $\because A + B + C = \pi \Rightarrow 3\alpha + 3\beta + 3\gamma = \pi \Rightarrow \alpha + \beta + \gamma =
  \frac{\pi}{3}$

  Clearly, $\angle ADB = 60^\circ$

  Applying sine rule in $\triangle ADB,$

  $\frac{AR}{\sin\beta} = \frac{c}{\sin[\pi - (\alpha + \beta)]}$

  $AR = \frac{c\sin \beta}{\sin(\alpha + \beta)} = \frac{2R\sin C\sin\beta}{\sin(\alpha + \beta)}$

  $= \frac{2R\sin3\gamma\sin\beta}{\sin(60^\circ - \gamma)}$

  $= \frac{2R(3\sin\gamma - 4\sin^3\gamma)\sin\beta}{\sin(60^\circ - \gamma)}.\frac{\cos(30^\circ -
    \gamma)}{cos(30^\circ - \gamma}$

  $= \frac{4R\sin\beta\sin\gamma.(3 - 4\sin^2\gamma).\cos(30^\circ - \gamma)}{\sin(09^\circ - 2\gamma) + \sin 30^\circ}$

  $= \frac{4R\sin\beta\sin\gamma\cos(30^\circ - \gamma)(3 - 4\sin^2\gamma)}{\cos2\gamma + \frac{1}{2}}$

  $= \frac{8R\sin\beta\sin\gamma\cos(30^\circ - \gamma)(3 - 4\sin^2\gamma)}{2\cos2\gamma + 1}$

  $= \frac{8R\sin\beta\sin\gamma\cos(30^\circ - \gamma)(3 - 4\sin^2\gamma)}{2(1 - 2\sin^2\gamma) + 1}$

  $= 8R\sin\beta\sin\gamma\cos(30^\circ - \gamma)$

\item The diagram is given below:

  \startplacefigure
    \externalfigure[20_3.pdf]
  \stopplacefigure

  From figure, $\angle AOX = \frac{\pi}{2} - \theta$

  Since $OX$ is tangent to the circle, $OB$ will pass through the center $P$ of the circle and hence
  $OB$ will be the diameter of the given circle.

  $\Rightarrow \angle OAB = 90^\circ \Rightarrow \angle OBA = 90^\circ - \theta$

  By property of circle, $OAQ = \angle OBA = 90^\circ - \theta$

  Also, $AOQ = 90^\circ - theta[\because OQ = OA]$

  $\therefore OQA = 2\theta \Rightarrow AQX = \pi - 2\theta$

  $\angle BOX = \frac{\pi}{1}$

  Applying sine rule in $\triangle ABT, we get$

  $\frac{AB}{\sin(\pi - 2\theta)} = \frac{AT}{\sin\theta}$

  $\frac{AB}{\sin2\theta} = \frac{t}{\sin\theta} \Rightarrow AB = 2t\cos\theta$

  From right angled $\triangle AOB,$

  $\tan\theta = \frac{AB}{OA} \Rightarrow AB = c\tan\theta$

  $\Rightarrow c\tan\theta = 2t\cos\theta$

  $\Rightarrow c\sin\theta - t(1 + \cos2\theta) = 0$

  Let $AN\perp OB$

  Now, $ON + NB = OB$

  $\Rightarrow c\cos\theta + AB\sin\theta = d$

  $\Rightarrow c\cos\theta + 2t\sin\theta\cos\theta = d$

  $\Rightarrow c\cos\theta + t\sin2\theta = d$

\item Since $AD$ is the median $\therefore BD = DC = \frac{a}{2}$

  Also, $\because \angle DAE = \angle CAE = \frac{A}{3}$

  $AE$ is common and $\angle AED = angle AEC = 90^\circ$

  $\therefore AD = AC = b$

  Applying cosine rule in $\triangle ABD,$

  $\cos\frac{A}{3} = \frac{AB^2 + AD^2 - BD^2}{2.AB.AD}$

  $= \frac{c^2 + b^2 - \frac{a^2}{4}}{2.c.b} = \frac{4b^2 + 4c^2 - a^2}{8bc}$

  Applying cosine rule in $\triangle ABC,$

  $\cos A = \frac{b^2 + c^2 - a^2}{2bc}$

  $4\cos^3\frac{A}{3} - 3\cos\frac{A}{3} = \frac{b^2 + c^2 - a^2}{2bc}$

  $\Rightarrow 4\cos^3\frac{A}{3} - 4\cos\frac{A}{3} = \frac{b^2 + c^2 - a^2}{2bc} - \frac{4b^2 + 4c^2 - a^2}{8bc}$

  $\Rightarrow 4\cos\frac{A}{3}\left(1 - \cos^2\frac{A}{3}\right) = \frac{4b^2 + 4c^2 - a^2}{8bc} - \frac{b^2 + c^2 -
    a^2}{2bc}$

  $\Rightarrow \cos\frac{A}{3}.\sin^2\frac{A}{3} = \frac{3a^2}{32bc}$

\item Given, $\cos A + \cos B + \cos C = \frac{3}{2}$

  $\Rightarrow \frac{b^2 + c^2 - a^2}{2bc} + \frac{c^2 + a^2 - b^2}{2ca} + \frac{a^2 + b^2 - c^2}{2ab} = \frac{3}{2}$

  $\Rightarrow a(b^2 + c^2 - a^2) + b(c^2 + a^2 - b^2) + c(a^2 + b^2 - c^2) = 3abc$

  $\Rightarrow a(b - c)^2 + b(c - a)^2 + c(a - b)^2 = a^3 + b^3 + c^3 - 3abc = \frac{1}{2}[(a - b)^2 + (b - c)^2 + (c -
  a)^2](a + b + c)$

  $\Rightarrow \frac{b + c - a}{2}(b - c)^2 + \frac{c + a - b}{2}(c - a)^2 + \frac{a + b - c}{2}(a - b)^2 = 0$

  $\Rightarrow (a - b)^2 = (b - c)^2 = (c - a)^2 = 0$

  $\Rightarrow a = b = c$

\item If the $\triangle ABC$ is equilateral $\Rightarrow A = B = C = 60^\circ$

  $\Rightarrow \tan A + \tan B + \tan C = 3\sqrt{3}$

  If $\tan A + \tan B + \tan C = 3\sqrt{3}$

  then $\tan A\tan B\tan C = 3\sqrt{3}$

  Thus, A.M. of $\tan A, \tan B, \tan C =$ G.M. of $\tan A, \tan B, \tan C$

  $\Rightarrow \tan A = \tan B = \tan C$

\item L.H.S. $= (a + b + c)\tan\frac{C}{2} = 2R(\sin A + \sin B + \sin C)\frac{\sin\frac{C}{2}}{\cos\frac{C}{2}}$

  $= 2R\left(2\sin\frac{A + B}{2}\cos\frac{A - B}{2} + 2\sin\frac{C}{2}\cos\frac{C}{2}\right)\frac{\sin\frac{C}{2}}{\cos\frac{C}{2}}$

  $= 2R\left(2\cos\frac{C}{2}\cos\frac{A - B}{2} +
  2\sin\frac{C}{2}\cos\frac{C}{2}\right)\frac{\sin\frac{C}{2}}{\cos\frac{C}{2}}$

  $= 2R\left(2\sin\frac{C}{2}\cos\frac{A - B}{2} + 2\sin^2\frac{C}{2}\right)$

  $= 2R\left(2\cos\frac{A + B}{2}\cos\frac{A - B}{2} + 2\sin^2\frac{C}{2}\right)$

  $= 2R\left(\cos A + \cos B + 2\sin^2\frac{C}{2}\right)$

  R.H.S $= a\cot\frac{A}{2} + b\cot\frac{B}{2} - c\cot\frac{C}{2}$

  $= 2R\left(\sin A\cot\frac{A}{2} = \sin B\cot\frac{B}{2} - \sin C\cot\frac{C}{2}\right)$

  $= 2R\left(2\cos^2\frac{A}{2} + 2\cos^2\frac{B}{2} - 2\cos^2\frac{C}{2}\right)$

  $= 2R\left(2\cos^2\frac{A}{2} + 2\cos^2\frac{B}{2} - 2 + 2\sin^2\frac{C}{2}\right)$

  $= 2R\left(\cos A + \cos B + 2\sin^2\frac{C}{2}\right)$

  Thus, L.H.S. = R.H.S.

\item $\sin^2\theta = \frac{1 - \cos2\theta}{2} \Rightarrow \sin^4\theta = \frac{(1 - \cos2\theta)^2}{4}$

  Also, for a triangle $\cos 2A + \cos 2B + \cos 2C = -1 -4\cos A\cos B\cos C$

  and $\cos^22A + \cos^2B + \cos^2C = 1 + 2\cos 2A\cos 2B\cos 2C$

  L.H.S. $= \frac{(1 - \cos2A)^2}{4} + \frac{(1 - \cos2B)^2}{4} + \frac{(1 - \cos 2C)^2}{4}$

  $= \frac{1}{4}[3 - 2(\cos2A +\cos 2B + \cos 2C) + \cos^22A + \cos^22B + \cos^22C]$

  $= \frac{1}{4}[3 - 2(-1 - 4\cos A\cos B\cos C) + 1 + 2\cos 2A\cos 2B\cos 2C]$

  $= \frac{3}{2} + 2\cos A\cos B\cos C + \frac{1}{2}\cos 2A\cos 2B\cos 2C =$ R.H.S.

\item Observe the relations in previous problem.

  L.H.S. $= \frac{(1 + \cos2A)^2}{4} + \frac{(1 + \cos2B)^2}{4} + \frac{(1 + \cos2C)^2}{4}$

  $= \frac{1}{4}[3 + 2(\cos2A +\cos 2B + \cos 2C) + \cos^22A + \cos^22B + \cos^22C]$

  $= \frac{1}{4}[3 + 2(-1 - 4\cos A\cos B\cos C) + 1 + 2\cos 2A\cos 2B\cos 2C]$

  $= \frac{1}{2} - 2\cos A\cos B\cos C + \frac{1}{2}\cos2A\cos2B\cos2C =$ R.H.S.

\item L.H.S. $= \cot B + \frac{\cos C}{\cos A\sin B} = \frac{\cos B\cos A + \cos[\pi - (A + B)]}{\cos A\sin B}$

  $= \frac{\cos B\cos A - \cos(A + B)}{\cos A\sin B} = \frac{\sin A\sin B}{\cos A\sin B}$

  $= \tan A$

  R.H.S. $= \cot C + \frac{\cos B}{\cos A\sin C} = \frac{\cos C\cos A + \cos[\pi - (A + C)]}{\cos A\sin C}$

  $= \frac{\sin A\sin C}{\cos A\sin C} = \tan A$

  Thus, L.H.S. = R.H.S.

\item $\frac{a\sin(B - C)}{b^2 - c^2} = \frac{1}{2R}.\frac{\sin A\sin(B - C)}{\sin^2B - \sin^2C}$

  $= \frac{1}{2R}.\frac{\sin[\pi - (B + C)]\sin(B - C)}{\sin(B + C)\sin(B - C)}$

  $= \frac{1}{2R}[\because \sin\{\pi - (B + C) = \sin(B + C)\}]$

  Similarly, $\frac{b\sin(C - A)}{c^2 - a^2} = \frac{c\sin(A - B)}{a^2 - b^2} = \frac{1}{2R}$

\item R.H.S. $= \frac{b - c}{a}\cos\frac{A}{2} = \frac{\sin B - \sin C}{\sin A}\cos\frac{A}{2}$

  $= \frac{2\cos\frac{B + C}{2}\sin\frac{B - C}{2}}{2\sin\frac{A}{2}\cos\frac{A}{2}}\cos\frac{A}{2}$

  $= \frac{\sin\frac{A}{2}\sin\frac{B - C}{2}}{\sin\frac{A}{2}}$

  $= \sin\frac{B - C}{2} =$ L.H.S.

\item L.H.S. $= \sin^3A\cos(B - C) + \sin^3B\cos(C - A) + \sin^3C\cos(A - B) = 3\sin A\sin B\sin C$

  $= \sin^2A\sin(B + C)\cos(B - C) + \sin^2B\sin(C + A)\cos(C - A) + \sin^2C\sin(A + B)\cos(A - B)$

  $= \frac{1}{2}[\sin^2A(\sin 2B + \sin 2C) + \sin^2B(\sin 2C + \sin 2A) + \sin^2C(\sin2A + \sin 2B)]$

  $= \sin^2A(\sin B\cos B + \sin C\cos C) + \sin^2B(\sin C\cos C + \sin A\cos A) + \sin^2C(\sin A\cos A + \sin B\cos B)$

  $= \sin A\sin B(\sin A\cos B + \cos A\sin B) + \sin B\sin C(\sin B\cos C + \cos B\sin C) + \sin A\sin C(\sin A\cos C +
  \cos A\sin C)$

  $= \sin A\sin B\sin(A + B) + \sin B\sin C\sin(B + C) + \sin A\sin C\sin(A + C)$

  $= 3\sin A\sin B\sin C =$ R.H.S.

\item L.H.S. $= \sin^3A + \sin^3B + \sin^3C = \frac{3}{4}[\sin A + \sin B + \sin C] - \frac{1}{3}[\sin 3A + \sin 3B + \sin
  3C]$

  $\sin A + \sin B + \sin C = 2\sin\frac{A + B}{2}\cos\frac{A - B}{2} + 2\sin \frac{C}{2}\cos \frac{C}{2}$

  $= 2\cos\frac{C}{2}\left[\cos\frac{A - B}{2} + \cos\frac{A - B}{2}\right]$

  $= 4\cos\frac{A}{2}\cos\frac{B}{2}\cos\frac{C}{2}$

  Similarly, $\sin3A + \sin3B + \sin3C = 4\cos\frac{3A}{2}\cos\frac{3B}{2}\cos\frac{3C}{2}$

\item $\sin3A\sin^3(B - C) = \sin3A\frac{3\sin(B - C) - \sin3(B - C)}{4}$

  Now $\sin 3A\sin3(B - C) = \sin3(B + C)\sin3(B - C) = \sin^23B - \sin^23C$

  and $\sin 3A\sin(B - C) = (3\sin A - 4\sin^3A)\sin(B - C)$

  $= 3\sin(B + C)\sin(B - C) - 4\sin^2A\sin(B + C)\sin(B - C)$

  $= 3[\sin^2B - \sin^2C] - 4\sin^2A(\sin^2B - \sin^2C)$

  Thus, $\sin3A\sin^3(B - C) + \sin3B\sin^3(C - A) + \sin3C\sin^3(A - B) = 0$

\item $\sin3A\cos^3(B - C) = \sin3A.\frac{3\cos(B - C) + \cos3(B - C}{4}$

  Now, $\frac{1}{4}\sin3A \cos3(B - C) = \frac{1}{8}2\sin3(B + C)\cos3(B - C) = \frac{1}{8}(\sin 6B + \sin 6C)$

  So $\sum \sin3A \cos3(B - C) = \frac{1}{4}(\sin 6A + \sin 6B + \sin 6C)$

  Again, $\frac{3}{4}\sin3A.\cos(B - C) = \frac{3}{4}(3\sin A - 4\sin^3A)\cos(B - C)$

  $= \frac{9}{8}[(\sin 2B + \sin 2C) -3\sin^3A]\cos(B - C)$

  We have just proved that $\sum \sin^3A\cos(B - C) = 3\sin A\sin B\sin C$

  $\therefore \frac{9}{8}\sum(\sin2B + \sin 2C) = \frac{9}{4}(\sin 2A + \sin 2B + \sin 2C)$

  and $3\sum\sin^3A\cos(B - C) = 9\sin A\sin B\sin C$

  Now, $\sin2A + \sin2B + \sin2C = 4\sin A\sin B\sin C$

  and $\sin6A + \sin 6B + \sin6C = 4\sin3A\sin3B\sin3C$

  Thus, the sum would be $\sin 3A\sin3B\sin3C$

\item L.H.S. $= \left(\frac{s(s - a) + s(s - b)}{\Delta}\right)\left(\frac{a.(s - a)(s - c)}{ac} + \frac{b(s - b)(s -
  c)}{bc}\right)$

  $= \frac{s(2s - a - b)}{\Delta}\left(\frac{(s - c)(2s - a - b)}{c}\right)$

  $= c\cot\frac{C}{2} =$ R.H.S.

\item Given $a,b,c$ are in A.P. $\therefore 2b = a + c$

  $2\sin B = \sin A + \sin C \Rightarrow 4\sin\frac{B}{2}\cos\frac{B}{2} = 2\sin\frac{A + C}{2}\cos\frac{A - C}{2}$

  $\Rightarrow 2\cos\frac{A + C}{2} = \cos\frac{A - C}{2}$

  L.H.S. $= 4(1 - \cos A)(1 - \cos C) = 4.2\sin^2\frac{A}{2}.2\sin^2\frac{C}{2}$

  $4\left(2\sin\frac{A}{2}\sin\frac{C}{2}\right)^2 = 4\left(\cos\frac{A - C}{2} - \cos\frac{A + C}{2}\right)^2$

  $= 4\left(2\cos\frac{A + C}{2} - \cos\frac{A + C}{2}\right)^2 = 4\cos^2\frac{A + C}{2}$

  R.H.S. $= \cos A + \cos C = 2\cos\frac{A + C}{2}\cos\frac{A - C}{2} = 4\cos^2\frac{A + C}{2}$

  Thus, L.H.S. = R.H.S.

\item Given, $a, b, c$ are in H.P.

  $\Rightarrow \frac{1}{a}. \frac{1}{b}, \frac{1}{c}$ are in A.P.

  $\Rightarrow \frac{s}{a}, \frac{s}{b}, \frac{s}{c}$ are in A.P.

  $\Rightarrow \frac{s}{a} -1, \frac{s}{b} - 1, \frac{s}{c} - 1$ are in A.P.

  $\Rightarrow \frac{bc}{(s - b)(s - c), \frac{ca}{(s - c)(s - a)}}, \frac{ab}{(s - a)(s - c)}$ are in A.P.

  $\Rightarrow \frac{1}{\sin^2\frac{A}{2}}, \frac{1}{\sin^2\frac{B}{2}}, \frac{1}{\sin^2\frac{C}{2}}$ are in A.P.

  $\Rightarrow \sin^2\frac{A}{2}, \sin^2\frac{B}{2}, \sin^2\frac{C}{2}$ are in H.P.

\item We have to prove that $\cos A\cot\frac{A}{2}, \cos B\cot\frac{B}{2}, \cot C\cot\frac{C}{2}$ are in A.P.

  $\Rightarrow \left(1 - 2\sin^2\frac{A}{2}\right)\cot\frac{A}{2} \left(1 - 2\sin^2\frac{B}{2}\right)\cot\frac{B}{2},
  \left(1 - 2\sin^2\frac{C}{2}\right)\cot\frac{C}{2}$ are in A.P.

  $\Rightarrow \cot\frac{A}{2} - \sin A, \cot\frac{B}{2} - \sin B, \cot\frac{C}{2} - \sin C$ are in A.P.

  Thus if we prove that $\cot \frac{A}{2}, \cot \frac{B}{2}, \cot \frac{C}{2}$ and $\sin A, \sin B, \sin C$ are in
  A.P. separately then we would have prove the above in A.P.

  Now, $\cot \frac{A}{2} + \cot \frac{C}{2} = \frac{s(s - a)}{\Delta} + \frac{s(s - c)}{\Delta} = \frac{s}{\Delta}[2s -
    a - c]$

  $= \frac{s}{\Delta}(2s - 2b)[\because 2b = a + c] = 2\cot \frac{B}{2}$

  Thus, $\cot \frac{A}{2}, \cot \frac{B}{2}, \cot \frac{C}{2}$ are in A.P.

  Since $a, b, c$ are in A.P.

  $2b = a + c \Rightarrow 2\sin B = \sin A + \sin C$

  Thus, $\sin A, \sin B, \sin C$ are in A.P.

  Hence the result.

\item Let the sides be $a - d, a, a + d$

  $2s =$ sum of the sides $= 3a \therefore s = \frac{3a}{2}$

  Now, $\Delta_1 =$ Area of the triangle whose sides are in A.P.

  $= \sqrt{\frac{3a}{2}\left(\frac{3a}{2} - a + d\right)\left(\frac{3a}{2} - a\right)\left(\frac{3a}{2}- a- d\right)}$

  $= \frac{\sqrt{3}a}{4}\sqrt{(a + 2d)(a - 2d)} = \frac{\sqrt{3}a}{4}\sqrt{a^2 - 4d^2}$

  An equilateral triangle with same perimeter will have each side $= a$ because perimeter is $3a.$

  $\Delta_2 =$ Area of the equilateral triangle $= \frac{\sqrt{3}}{4}a^2$

  Given, $\frac{\Delta_1}{\Delta_2} = \frac{3}{5}$

  $\Rightarrow \frac{\sqrt{a^2 - 4d^2}}{a} = \frac{3}{5} \Rightarrow \frac{a}{d} = \frac{4}{2}[\because d > 0]$

  Ratio of sides $= a - d: a: a + d = \frac{a}{d} - 1:\frac{a}{d}:\frac{a}{d}+1 = 3:5:7$

\item Let $ABC$ be the triangle. Given, $\tan A, \tan B, \tan C$ are in A.P.

  $\therefore \tan A - \tan B = \tan B - \tan C$

  So either both sides are positive or both sides are negative.

  If both sides are positive then $\tan A$ is the greatest angle and if both sides are negative then $\tan A$ is the
  least angle.

  According to question $x$ is the least or greatest tangent $\Rightarrow \tan A = x$

  $\Rightarrow \sin^2x = \frac{x^2}{1 + x^2}$

  Now, $2\tan B = \tan A + \tan C \Rightarrow \tan B = \frac{x + \tan C}{2}$

  $B = \pi - (A + C)$

  $\Rightarrow \tan B = -\tan(A + C) = -\frac{x + \tan C}{1 - x\tan C}$

  Thus, $\frac{x + \tan C}{2} = -\frac{x + \tan C}{1 - x\tan C}$

  $\Rightarrow 1 - x\tan C = -2 \Rightarrow \tan C = \frac{3}{x}$

  $\sin^2C = \frac{9}{9 + x^2}$

  $\Rightarrow \tan B = \frac{x^2 + 3}{2x} \Rightarrow \sin^2B = \frac{(x^2 + 3)^2}{(x^2 + 1)(x^2 + 9)}$

  Now $a^2:b^2:c^2 = \sin^2A:\sin^2B:\sin^2C$

  Hence the result.

\item Let the sides be $a - d, a, a + d.$ Let $d > 0,$ then greatest side is $a + d$ and least side is $a -
  d.$

  Hence angle $A$ is the least angle and $C$ is the greatest angle. Let $\angle A = \theta \therefore C =
  \theta + \alpha \Rightarrow B = \pi - 2\theta - \alpha$

  Applying sine rule, we get

  $\frac{a - d}{\sin\theta} = \frac{a}{\sin[\pi - (2\theta + \alpha)]} = \frac{a + d}{\sin(\theta + \alpha)} =
  \frac{2a}{\sin\theta + \sin(\theta + \alpha)}$

  $\frac{a - d}{\sin\theta} = \frac{a + d}{\sin(\theta + \alpha)}$

  $\Rightarrow \frac{a - d}{a + d} = \frac{\sin\theta}{\sin(\theta + \alpha)}$

  By componendo and dividendo, we get

  $\frac{2a}{2d} = \frac{\sin\theta + \sin(\theta + \alpha)}{\sin(\theta + \alpha) - \sin\theta}$

  $\Rightarrow \frac{d}{a} = \frac{\tan\frac{\alpha}{2}}{\tan\left(\theta + \frac{\alpha}{2}\right)}$

  Now $\frac{a}{\sin(2\theta + \alpha)} = \frac{2a}{\sin\theta + \sin(\theta + \alpha)}$

  $\Rightarrow \frac{1}{2} = \frac{\cos\left(\theta + \frac{\alpha}{2}\right)}{\cos\frac{\alpha}{2}}$

  $\cos\left(\theta + \frac{\alpha}{2}\right) = \frac{\cos\frac{\alpha}{2}}{2}$

  $\tan\left(\theta + \frac{\alpha}{2}\right) = \frac{\sqrt{4 - \cos^2\frac{\alpha}{2}}}{\cos\frac{\alpha}{2}}$

  $\frac{d}{a} = \sqrt{\frac{1 - \cos\alpha}{7 - \cos\alpha}} = x$

  Thus, required ratio $= a - d:a:a + d = 1 - x: 1: 1 + x$

\item Consider that sides of the triangle are $a, ar, ar^2$ where $ar^2$ is the greatest side.

  $\because ar^2 < a + ar \Rightarrow r^2 -r - 1 < 0$

  $\left(r - \frac{1}{2}\right) - \frac{5}{4} < 0 \Rightarrow \left(r - \frac{1}{2}\right)^2 < \frac{5}{4}$

  $r - \frac{1}{2} < \frac{\sqrt{5}}{2} \therefore r < \frac{1}{2}(\sqrt{5} + 1)$

  $r^2 < \frac{1}{2}(3 + \sqrt{5})$

  $r^4 < \frac{1}{2}(7 + 3\sqrt{5})$

  $1 + r^2 - r^4 < - 1 - \sqrt{5}$

  $\therefore 1 + r^2 - r^4 < r$

  $\therefore \cos C = \frac{a^2 + a^2r^2 - a^2r^4}{2a^2r} < \frac{1}{2}$

  $\cos C < \cos \frac{\pi}{3} \therefore C > \frac{\pi}{3}$

  $\cos B = \frac{1 + r^4 - r^2}{2r^2} = \frac{1}{2}\left[\left(r - \frac{1}{3}\right)^2 + 1\right] > \frac{1}{2}$

  $\therefore \cos B > \cos\frac{\pi}{3} \therefore B < \frac{\pi}{3}$

  $\because a < ar <ar^2 \therefore A > B > C$

  Hence $A < B < \frac{\pi}{3} < C$

\item  The diagram is given below:

  \startplacefigure
    \externalfigure[20_5.pdf]
  \stopplacefigure

  We are given $AM = p, BN = q$

  Let $\angle ACM = \theta$ and $\angle BCN = \phi$

  Then, $\sin\theta = \frac{p}{b}$ and $\sin\phi = \frac{q}{a}$

  Now $C = \pi - (\theta + \phi)$

  $\cos C = -\cos(\theta + \phi) = \sin\theta\sin\phi -\cos\theta\cos\phi$

  $\Rightarrow \sqrt{1 - \frac{p^2}{b^2}}\sqrt{1 - \frac{q^2}{a^2}} = \frac{pq}{ab} - \cos C$

  Squaring, we get

  $\left(1 - \frac{p^2}{q^2}\right)\left(1 - \frac{q^2}{a^2}\right) = \frac{p^2q^2}{a^2b^2} - 2\frac{pq}{ab}\cos C +
  \cos^2C$

  $a^2b^2 + b^2q^2 - 2abpq\cos C = a^2b^2\sin^2C$

\item $\angle OCB = \theta, \angle BOC = \pi - \theta - (C - \theta) = \pi - C$

  Similarly, $\angle AOB = \pi - B$

  From $\triangle AOB,$ we have

  $\frac{OB}{\sin\theta} = \frac{AB}{\sin(\pi - B)} = \frac{c}{\sin B} \Rightarrow OB = \frac{c\sin\theta}{\sin B}$

  Again from $\triangle OBC,$ we have

  $\frac{OB}{\sin(C - \theta)} = \frac{BC}{\sin(\pi - C)} = \frac{a}{\sin C} \Rightarrow OB = \frac{a\sin(C -
    \theta)}{\sin C}$

  $\Rightarrow \frac{c\sin\theta}{\sin B} = \frac{a\sin(C - \theta)}{\sin C}$

  $\Rightarrow \sin C\sin\theta\sin C = \sin A\sin(C - \theta)\sin B$

  $\Rightarrow \sin C\sin\theta\sin(A + B) = \sin A\sin B\sin(C - \theta)$

  $\Rightarrow \sin C\sin\theta\sin A\cos B + \sin C\sin\theta\cos A\sin B = \sin A\sin B\sin C\cos\theta - \sin A\sin
  B\cos C\sin\theta$

  Dividing by $\sin A\sin B\sin C\sin\theta,$ we get

  $\Rightarrow \cot B + \cot A = \cot \theta - \cot C$

  $\cot\theta = \cot A + \cot B + \cot C$

  In a triangle $\cot A\cot B + \cot B\cot C + \cot C\cot A = 1$

  Thus, squaaring we get

  $\csc^2\theta = \csc^2A + \csc^2B + \csc^2C$

\item The diagram is given below:

  \startplacefigure
    \externalfigure[20_6.pdf]
  \stopplacefigure

  Let $O$ be the circumcenter and $OP = x.$ We have $BP= \frac{a}{2}.$

  Angle made at center will be double that made at perimeter, thus

  $\tan A = \frac{a}{2x}$

  Similarly, $\tan B = \frac{b}{2y}, \tan C = \frac{c}{2z}$

  In a $\triangle ABC,$ we know that

  $\tan A + \tan B + \tan C = \tan A\tan B\tan C$

  $\Rightarrow \frac{a}{x} + \frac{b}{y} + \frac{c}{z} = \frac{abc}{4xyz}$

\item Given, $\frac{BD}{m} = \frac{DC}{n} = \frac{BC}{m + n}$

  $\Rightarrow BD = \frac{ma}{m + n}$

  In $\triangle ABD,$ we have

  $x^2 = AB^2 + BD^2 - 2AB.BD.\cos B = c^2 + \frac{m^2a^2}{(m + n)^2} - 2.c.\frac{ma}{m + n}.\frac{c^2 + a^2 - b^2}{2ca}$

  Hence the result.

\item Given, $\sin A + \sin B + \sin C = \frac{3\sqrt{3}}{2}$

  $\Rightarrow \cos\frac{A}{2}\cos\frac{B}{2}\cos\frac{C}{2} = \left(\frac{\sqrt{3}}{2}\right)^3$

  Under the constraint $A + B + C = \pi$ the product will be maximum if $A = B = C = \frac{\pi}{3}$

  If $A = B = C$

  $\cos\frac{A}{2}\cos\frac{B}{2}\cos\frac{C}{2} = \cos^330^\circ = \left(\frac{\sqrt{3}}{2}\right)^3$

  Thus, the triangle is equilateral.

\item This problem can be solved like previous problem.

\item Given, $\cos A + 2\cos B + \cos C = 2$

  $\cos A + \cos C = 2(1 - \cos B) \Rightarrow 2\cos\frac{A + C}{2}\cos\frac{A - C}{2} = 2.2\sin^2\frac{B}{2}$

  $\cos\frac{A - C}{2} = 2.\cos\frac{A + C}{2}$

  $2\sin\frac{A + C}{2}\cos\frac{A - C}{2} = 2.2\sin\frac{A + C}{2}\cos\frac{A + C}{2}$

  $\sin A + \sin C = 2.\sin(A + C) = 2\sin B \Rightarrow a + c = 2b$

  Thus, the sides are in $a,b,c.$

\item $\tan\frac{A}{2} + \tan\frac{C}{2} = \sqrt{\frac{(s - b)(s - c)}{s(s - a)}} + \sqrt{\frac{(s - a)(s - b)}{s(s - c)}}$

  $= \sqrt{\frac{s - b}{s}}\left(\sqrt{\frac{s - c}{s - a}} + \sqrt{\frac{s - a}{s - c}}\right)$

  $= \sqrt{\frac{s - b}{s}} \left(\frac{s - c + s - a}{\sqrt{(s - a)(s - c)}}\right)$

  $= \frac{b}{s}\sqrt{\frac{s(s - b)}{(s - a)(s - c)}} = \frac{b}{s}\cot\frac{B}{2}$

  Since sides are in A.P. $2b = a + c \Rightarrow 2s = 3b$

  $\tan\frac{A}{2} + \tan\frac{C}{2} = \frac{2}{3}\cot\frac{B}{2}$

\item Given, $\frac{a - b}{b - c} = \frac{s - a}{s - c}$

  $\Rightarrow \frac{s - a}{a - b} = \frac{s - c}{b - c}$

  $\Rightarrow \frac{s - a}{(s - b) - (s - a)} = \frac{s - c}{(s - c) - (s - b)}$

  $\Rightarrow \frac{\frac{\Delta}{r_1}}{\frac{\Delta}{r_2} - \frac{\Delta}{r_1}} =
  \frac{\frac{\Delta}{r_3}}{\frac{\Delta}{r_3} - \frac{\Delta}{r_2}}$

  $\Rightarrow 2r_2 = r_1 + r_3$

  Hence the result.

\item Let the sides be $a, ar, ar^2.$

  $x = (b^2 - c^2)\frac{\tan B + \tan C}{\tan B - \tan C} = (b^2 - c^2)\frac{\sin B\cos C + \cos B\sin C}{\sin B\cos C -
  \cos B\sin C}$

  $= (b^2 - c^2)\frac{\sin(B + C)}{\sin(B - C)} = 4R^2(\sin^2B - \sin^2C)\frac{\sin^2(B + C)}{\sin^2B - \sin^2C}$

  $= a^2$

  Similalry, $y = a^2r^2$ and $z = a^2r^4$

  Thus, $x,y,z$ are in G.P.

\item Given, $r_1,r_2,r_3$ are in H.P.

  $\Rightarrow \frac{1}{r_1}, \frac{1}{r_2}, \frac{1}{r_3}$ are in A.P.

  $\Rightarrow \frac{1}{r_2} - \frac{1}{r_1} = \frac{1}{r_3}- \frac{1}{r_2}$

  $\Rightarrow \frac{s - b}{\Delta} - \frac{s - a}{\Delta} = \frac{s - c}{\Delta} - \frac{s - b}{\Delta}$

  $\Rightarrow s - b - s + a = s - c - s + b$

  $\Rightarrow a - b = b - c$

  Hence $a,b,c$ are in A.P.

\item Given, $r_1 = r_2 + r_3 + r \Rightarrow r_1 - r = r_2 + r_3$

  $\Rightarrow \frac{\Delta}{s - a} - \frac{\Delta}{s} = \frac{\Delta}{s - b} + \frac{\Delta}{s - c}$

  $\Rightarrow \frac{a}{s(s - a)} = \frac{a}{(s - b)(s - c)}$

  $\Rightarrow s(s - a) = (s - b)(s - c)$

  $\Rightarrow s(b + c - a) = bc$

  $\Rightarrow \frac{b + c - a}{2}(b + c - a) = bc$

  $\Rightarrow (b + c)^2 - a^2 = 2bc$

  $\Rightarrow b^2 + c^2 = a^2$

  Thus, the triangle is right angled.

\item R.H.S. $= 1 + \frac{r}{R} = 1 + \frac{\frac{\Delta}{s}}{\frac{abc}{4\Delta}} = 1 + \frac{4\Delta^2}{abcs}$

  L.H.S. $= \cos A + \cos B + \cos C = 2\cos\frac{A + B}{2}\cos\frac{A - B}{2} + \cos C$

  $= 2\sin\frac{C}{2}\cos\frac{A - B}{2} + 1 - 2\sin^2\frac{C}{2}$

  $= 1 + 2\sin\frac{C}{2}\left[\cos\frac{A - B}{2} - \sin\frac{C}{2}\right]$

  $= 1 + 2\sin\frac{C}{2}\left[\cos\frac{A - B}{2} - \cos\frac{A + B}{2}\right]$

  $= 1 + 4\sin\frac{A}{2}\sin\frac{B}{2}\sin\frac{C}{2}$

  $= 1 + 4\sqrt{\frac{(s - b)(s - c)}{bc}}\sqrt{\frac{(s - a)(s - c)}{ca}}\sqrt{\frac{(s - a)(s - b)}{ab}}$

  $= 1 + 4\frac{(s - a)(s - b)(s - c)}{abc}.\frac{s}{s}$

  $= 1 + 4\frac{\Delta^2}{abcs}$

  Thus, L.H.S. = R.H.S.

\item Let $r_1, r_2, r_3$ be the radii of escribed circles of triangle $ABC,$ then $r_1, r_2, r_3$ will be the
  roots of the equation,

  $x^3 - (r_1 + r_2 + r_3)x^2 + (r_1r_2 + r_2r_3 + r_3r_1)x - r_1r_2r_3 = 0$

  Now, $r_1 + r_2 + r_3 = \frac{\Delta}{s - a} + \frac{\Delta}{s - b} + \frac{\Delta}{s - c}$

  $= \Delta\left[\frac{1}{s - a} + \frac{1}{s - b}\right] + \frac{\Delta}{s - c} - \frac{\Delta}{s} + \frac{\Delta}{s}$

  $= \Delta\left[\frac{s - b + s - a}{(s - a)(s - b)}\right] + \frac{\Delta(s - s + c)}{s(s - c)} + \frac{\Delta}{s}$

  $= \frac{\Delta.c}{(s - a)(s - b)} + \frac{\Delta.c}{s(s - c)} + \frac{\Delta}{s}$

  $= \Delta.c\left[\frac{s^2 - cs + s^2 - as - bs + ab}{s(s - a)(s - b)(s - c)}\right] + \frac{\Delta}{s}$

  $= \frac{abc}{\Delta} + \frac{\Delta}{s} = r + 4R$

  Now, $r_1r_2 + r_2r_3 + r_3r_1 = \Delta^2\left[\frac{s - c + s - a + s - b}{(s - a)(s - b)(s - c)}\right]$

  $= \frac{\Delta^2.s}{(s - a)(s - b)(s - c)} = s^2$

  $r_1r_2r_3 = \frac{\Delta^3.s}{s(s - a)(s - b)(s - c)} = \Delta.s = rs^2$

  Thus, $r_1, r_2, r_3$ are roots of the equation

  $x^3 - (r + 4R)x^2 + s^2x - rs^2 = 0$

\item Let $s$ be the semi perimeter, then $s = 12$ cm. Area is $\Delta = 24$ sq. cm.

  Let $a,b,c$ be the lengths of the sides.

  $r_1 = \frac{\Delta}{s - a} = \frac{24}{12 - a}$

  $r_2 = \frac{\Delta}{s - b} = \frac{24}{12 - b}$

  $r_3 = \frac{\Delta}{s - c} = \frac{24}{12 - c}$

  Given $r_1, r_2, r_3$ are in H.P.

  $\therefore \frac{1}{r_2} - \frac{1}{r_1} = \frac{1}{r_3} - \frac{1}{r_2}$

  $\Rightarrow \frac{12 - b}{24} - \frac{12 - a}{24} = \frac{12 - c}{24} - \frac{12 - b}{24}$

  $\Rightarrow a - b = b - c \Rightarrow 2b = a + c$

  $a + b + c = 24 \Rightarrow b = 8$ cm.

  $a + c = 16 \Rightarrow c = 16 - a$

  Now, $\Delta = \sqrt{s(s - a)(s - b)(s - c)} \Rightarrow 24.24 = 12(12 - a)(12 - b)(12 - c)$

  $\Rightarrow a^2 - 16 a + 60 = 0 \Rightarrow a = 6, 10 \Rightarrow c = 10, 6$

\item $\frac{a}{\sin A} = \frac{b}{\sin B} = \frac{c}{\sin C} = 2R$

  Given, $8R^2 = a^2 + b^2 + c^2 = 4R^2(\sin^2A + \sin^2B + \sin^2C)$

  $\Rightarrow (1 - \sin^2A) + (1 - \sin^2B) - \sin^2C = 0$

  $\Rightarrow \cos^2A + \cos^2B - \sin^2C = 0$

  $\Rightarrow \cos^2A + \cos(B + C)\cos(B - C) = 0$

  $\Rightarrow \cos A[\cos A - \cos(B - C)] = 0$

  $\Rightarrow \cos A[\cos(B + C) + \cos(B - C)] = 0$

  $\Rightarrow \cos A\cos B\cos C = 0$

  Thus, either $A = 90^\circ$ or $B = 90^\circ$ or $C = 90^\circ$ and hence the triangle is $90^\circ.$

\item The diagram is given below:

  \startplacefigure
    \externalfigure[20_7.pdf]
  \stopplacefigure

  Let $O$ be the center of the inscribed circle of triangle $ABC.$ We have drawn another circle passitng through
  $O, B$ and $C.$ Suppose that the radius of this circle is $R.$ Applying sine law in $\triangle OBC,$
  we get

  $\frac{a}{\sin BOC} = 2R \Rightarrow R = \frac{a}{2\sin BOC}$

  Now since $O$ is the center of the inscribed circle. Hence $BO$ and $OC$ are bisectors of angle $B$
  and $C$ respectively

  $\angle OBC = \frac{B}{2}$ and $\angle OCB = \frac{C}{2}$

  $\Rightarrow \angle BOC = 180^\circ - \frac{B}{2} - \frac{C}{2} = 90^\circ + \frac{A}{2}$

  $\therefore R = \frac{a}{2.\sin\left(90^\circ + \frac{A}{2}\right)} = \frac{a}{2}\sec\frac{A}{2}$

\item The diagram is given below:

  \startplacefigure
    \externalfigure[20_8.pdf]
  \stopplacefigure

  Let the centers of the circle be $C_1, C_2$ and $C_3$ and theier radii be $a, b$ and $c$
  respectively. Let the circles touch each other at $P, Q$ and $R.$ Let the tangents at their points of contact
  meet at $O.$

  Since $OP$ and $OQ$ are two tangents from $O$ to the circle $C_3,$ they are equal i.e. $OP =
  OQ$

  Similarly, $OQ = OR \Rightarrow OP=OQ=OR$

  Also, $OP\perp C_1C_3, OQ\perp C_2C_3$ and $OR\perp C_1C_2$

  Hence, $OP, OQ$ and $OR$ are the in-radii of $\triangle C_1C_2C_3.$

  Let $OP=OQ=OR = r$ which is given as $4.$

  $r = \frac{\Delta}{s}$ where $s =$ semi-perimeter of $\triangle C_1C_2C_3$ and $\Delta =$ are of
  $\triangle C_1C_2C_3$

  Now, $s = \frac{(a + b) + (b + c) + (c + a)}{2} = a + b + c$ and

  $\Delta = \sqrt{s(s - a - b)(s - b - c)(s - c - a)} = \sqrt{(a + b + c)c.a.b}$

  $r = \frac{\Delta}{s} = \sqrt{\frac{abc}{a + b + c}} = 4$

  $\Rightarrow \frac{abc}{a + b + c} = 16$

\item The diagram is given below:

  \startplacefigure
    \externalfigure[20_9.pdf]
  \stopplacefigure

  Let $R$ be the circum-radius of the $\triangle ABC.$ From geometry we know that

  $AH = 2OE = 2R\cos A$ and $OA = R$

  $\angle BOC = 2A \therefore \angle COE = A \Rightarrow \angle OCE = 90^\circ - A$

  $\therefore \angle OCA = \angle BCA - \angle OCE = C - (90^\circ - A) = A + C - 90^\circ$

  $\because OA = OC \therefore \angle OAC = \angle OCA = A + C - 90^\circ$

  From $\triangle CDA, \angle CAD = 90^\circ - C$

  $\therefore \angle HAO = \angle CAD - \angle CAO = (90^\circ - C) - (A + C - 90^\circ)$

  $= 180^\circ - A - 2C = A + B + C - A - 2C = B - C$

  Applying cosine rule in $\triangle AHO,$ we get

  $\cos(B - C) = \frac{AH^2 + AO^2 - OH^2}{2AH.AO}$

  $OH^2 = 4R^2\cos^2A + R^2 - 2.2R\cos A.R\cos(B - C)$

  $= R^2[4\cos^2A + 1 - 4\cos A\cos(B - C)] = R^2[1 - 4\cos A\{\cos(B - C) - \cos A\}]$

  $= R^2[1 - 4\cos A\{\cos(B - C) + \cos(B + C)\}]$

  $= R^2[1 - 8\cos A\cos B\cos C]$

  $OH = R\sqrt{1 - 8\cos A\cos B\cos C}$
\item The diagram is given below:

  \startplacefigure
    \externalfigure[21_1.pdf]
  \stopplacefigure

  Let $ABC$ be the triangle. Let $O$ be the circumcenter and $I,$ the incenter.

  Clearly, $OA = OB = OC = R, IE=r[IE\perp AB]$

  Let $OM\perp BC$ then $\angle BOM = \angle COM = A$

  Now, $OA = R, AI = r\csc\frac{A}{2} = \frac{4R\sin\frac{A}{2}\sin\frac{B}{2}\sin\frac{C}{2}}{\sin\frac{A}{2}}$

  $= 4R\sin\frac{B}{2}\sin\frac{C}{2}$

  $\angle OAB = \angle OBA = B - (90^\circ - A) = A + B - 90^\circ = 90^\circ - C$

  $\therefore \angle OAI = \angle BAI - \angle BAO = \frac{A}{2} - (90^\circ - C)$

  $= \frac{C - B}{2}$

  Applying cosine law in $\triangle OAI,$

  $\cos\frac{C - B}{2} = \frac{OA^2 + AI^2 - OI^2}{2OA.AI}$

  $= \frac{R^2 + 16R^2\sin^2\frac{B}{2}\sin^2\frac{C}{2} - OI^2}{2.R.4R\sin\frac{B}{2}\sin\frac{C}{2}}$

  $OI^2 = R^2\left[1 + 8\sin\frac{B}{2}\sin\frac{C}{2}\left\{2\sin\frac{B}{2}\sin\frac{C}{2} - \cos\left(\frac{B -
      C}{2}\right)\right\}\right]$

  $= R^2\left[1 + 8\sin\frac{B}{2}\sin\frac{C}{2}\left\{\cos\frac{B - C}{2} - \cos\frac{B + C}{2} - \cos\frac{B -
      C}{2}\right\}\right]$

  $= R^2\left[1 - 8\sin\frac{A}{2}\sin\frac{B}{2}\sin\frac{C}{2}\right]$

  $= R^2 - 2Rr$

  Let us prove the necessary condition for the second part.

  Let $b$ be the A.M. of $a$ and $c$ i.e. $2b = a + c$

  $\Rightarrow 2\sin B = \sin A + \sin C \Rightarrow 2.2\sin\frac{B}{2}\cos\frac{B}{2} = 2\sin\frac{A + C}{2}\cos\frac{A -
    C}{2}$

  $\Rightarrow 2\sin\frac{B}{2} = \cos\frac{A - C}{2} \Rightarrow 2\cos\frac{A + C}{2} = \cos\frac{A - C}{2}$

  $\cos BIO = \frac{BI^2 + IO^2 - BO^2}{2BI.IO}$

  Now $BI^2 + IO^2 - BO^2 = r^2\csc^2\frac{B}{2} + R^2 - 2rR - R^2$

  $= r^2\csc^2\frac{B}{2} - 2rR$

  $= \frac{r.4R\sin\frac{A}{2}\sin\frac{B}{2}\sin\frac{C}{2}}{\sin^2\frac{B}{2}} - 2rR$

  $= \frac{2rR.2\sin\frac{A}{2}\sin\frac{C}{2}}{\sin\frac{B}{2}} - 2rR$

  $= \frac{2rR\left[\cos\frac{A - C}{2} - \cos\frac{A + C}{2}\right]}{\sin\frac{B}{2}} - 2rR$

  $= 2rR\frac{2\cos\frac{A + C}{2}- \cos\frac{A + C}{2}}{\sin\frac{B}{2}} - 2rR = 0$

  Thus, $\triangle BIO$ is a right angled triangle.

  Now the sufficient condition can be proved similarly.

\item $\frac{A}{2} + \frac{B}{2} = \frac{\pi}{2} - \frac{C}{2}$

  $\Rightarrow \cot\left(\frac{A}{2} + \frac{B}{2}\right) = \cot\left(\frac{\pi}{2} - \frac{C}{2}\right)$

  $\Rightarrow \frac{\cot\frac{A}{2}\cot\frac{B}{2} - 1}{\cot\frac{A}{2} + \cot\frac{B}{2}} = \tan\frac{C}{2} =
  \frac{1}{\cot\frac{C}{2}}$

  $\Rightarrow \cot\frac{A}{2} + \cot \frac{B}{2} + \cot\frac{C}{2} = \cot\frac{A}{2}\cot\frac{B}{2}\cot\frac{C}{2}$

\item The diagram is given below:

  \startplacefigure
    \externalfigure[21_2.pdf]
  \stopplacefigure

  $HL = r_2, ID=r \therefore IL = ID - LD = ID - HK = r - r_2$

  In $\triangle IHL,$

  $\cot\frac{B}{2} = \frac{HL}{IL} = \frac{r_2}{r - r_2}$

  Similarly, $\cot\frac{A}{2} = \frac{r_1}{r - r_1}$ and

  $\cot\frac{C}{2} = \frac{r_3}{r - r_3}$

  Now following the result of previous problem

  $\frac{r_1}{r - r_1} + \frac{r_2}{r - r_2} + \frac{r_3}{r - r_3} = \frac{r_1r_2r_3}{(r - r_1)(r - r_2)(r - r_3)}$

\item The diagram is given below:

  \startplacefigure
    \externalfigure[21_3.pdf]
  \stopplacefigure

  Let $O$ be the circumcenter and $P$ the orthocenter of the $\triangle ABC.$ From geometry,

  $\angle BOF = \angle COF = A$ and $AP = 2OF = 2R\cos A, BF = R\sin A$

  From right angled $\triangle ADB$

  $BD = c\cos B, AD = c\sin B$

  Now, $PM = AD - (AP + MD) = c\sin B - (2R\cos A + R\cos A) = c\sin B - 3R\cos A$

  $OM = FD = BF - BD = R\sin A - c\cos B$

  $\therefore \tan\theta = \frac{PM}{OM} = \frac{c\sin B - 3R\cos A}{R\sin A - c\cos B}$

  $= \frac{2R\sin C\sin B - 3R\cos A}{R\sin A - 2R\sin C\cos B}$

  $= \frac{2\sin B\sin C - 2\cos[\pi - (B + C)]}{\sin[\pi - (B + C)- 2\sin C\cos B]}$

  $= \frac{2\sin B\sin C + 3\cos(B + C)}{\sin(B + C) - 2\sin C\cos B}$

  $= \frac{3\cos B\cos C - \sin B\sin C}{\cos C\sin B - \sin C\cos B}$

  $= \frac{3 - \tan B\tan C}{\tan B - \tan C}$

\item The diagram is given below:

  \startplacefigure
    \externalfigure[21_4.pdf]
  \stopplacefigure

  Let $ABC$ be the triangle. Let $O$ and $I$ be the circumcenter and in-center of the $\triangle ABC.$
  Let $P$ be the center and $x,$ the radius of the circle drawn which touches the inscribed and circumscribed
  circle, of $\triangle ABC$ and the side $BC$ externally. Let us join $OP$ and extend up to $Q.$ Let
  $ID\perp BC.$ Clearly, $P$ will lie on the extended part of $ID.$ Draw the line $ON$ parallel to the
  line $IP.$ Join $NP.$

  Clearly, $OB = OC OQ = R, OP = OQ - PQ = R - x$

  $ON = OM + MN = R\cos A + x$

  $NP = MD = OM - CD = \frac{a}{2} - r\cot \frac{C}{2}$

  $= \frac{a}{2} - \frac{\Delta}{s}.\frac{s(s - c)}{\Delta} = \frac{a}{2} - (s - c)$

  $= \frac{c - b}{2}$

  From right angled $\triangle ONP, OP^2 = ON^2 + NP^2$

  $(R - x)^2 = (R\cos A + x)^2 + \left(c - b\right)^2$

  $R^2 + x^2 - 2Rx = R^2\cos^2A + x^2 + 2Rx\cos A + \left(\frac{c - b}{2}\right)^2$

  $2Rx(1 + \cos A) = R^2(1 - \cos^2A) - \left(\frac{c - b}{2}\right)^2$

  $4Rx\cos^2\frac{A}{2} = R^2\sin^2A - \left(\frac{b - c}{2}\right)^2 = \frac{a^2}{4} - \frac{(b - c)^2}{4}$

  $4Rx\frac{s(s - a)}{bc} = \frac{a +b - c}{2}.\frac{a - b + c}{2} = (s - b)(s - c)$

  $x = \frac{\Delta}{a}\tan^2\frac{A}{2}$

\item The diagram is given below:

  \startplacefigure
    \externalfigure[21_5.pdf]
  \stopplacefigure

  Since angles n the same segment of a circle are equal. $\therefore \angle BED= \angle BAD = \frac{A}{2}$

  and $\angle BEF = \angle BCF = \frac{C}{2}$

  Now $\angle DEF = \angle BEF + \angle BED = \frac{C}{2} + \frac{A}{2} = \frac{C + A}{2} = 90^\circ - \frac{B}{2}$

  Similarly, $DFE = 90^\circ - \frac{C}{2}, \angle EDF = 90^\circ - \frac{A}{2}$

  Now area of $\triangle DEF = \frac{1}{2}DE.DF.\sin \angle EDF$

  Let $R$ be the circum-radius of $\triangle ABC$ then clearly $R$ is also the circum-radius of $\triangle DEF.$

  Applying sine rule in $\triangle DEF,$ we have

  $\frac{DE}{\sin DFE} = \frac{DF}{\sin DEF} = \frac{EF}{\sin EDF} = 2R$

  $\Rightarrow DE = 2R\sin DFE = 2R\sin\left(90^\circ - \frac{C}{2}\right) = 2R\cos\frac{C}{2}$

  Similarly, $DF = 2R\cos\frac{B}{2}$

  So, area of $\triangle DEF = \frac{1}{2}.4R^2\cos\frac{B}{2}\cos\frac{C}{2}.\sin\left(90^\circ - \frac{A}{2}\right)$

  $= 2R^2\cos\frac{A}{2}\cos\frac{B}{2}\cos\frac{C}{2}$

  $= 2R^2\sqrt{\frac{s(s - a)}{bc}}\sqrt{\frac{s(s - b)}{ca}}\sqrt{\frac{s(s - c)}{ab}}$

  $= \frac{2R^2s\sqrt{s(s - a)(s - b)(s - c)}}{abc} = \frac{2R^2s\Delta}{abc} = \frac{R^2.s.4\Delta}{2abc} = \frac{R^2s}{2\frac{abc}{4\Delta}}$

  Here, are of $\triangle ABC = \Delta$

  $= \frac{R^2s}{2R} = \frac{R}{2}s$

  $\Rightarrow \frac{\Delta DEF}{\Delta ABC} = \frac{Rs}{2\Delta} = \frac{R}{2r}$

\item The diagram is given below:

  \startplacefigure
    \externalfigure[21_6.pdf]
  \stopplacefigure

  $\Delta A'B'C' = \Delta ABC - (\Delta AC'B' + \Delta BA'C' + \Delta CA'B')$

  $\Delta AC'B' = \frac{1}{2}AC'.AB'\sin A$

  $\because CC'$ is the internal bisector of $\angle C$

  $\frac{AC'}{C'B} = \frac{AC}{CB} = \frac{b}{a}$

  $\Rightarrow AC' = bk, C'B = ak,$ where $k$ is some constant dependent on angles.

  $AC' + C'B = AB \Rightarrow c = ak + bk \Rightarrow k = \frac{c}{a + b}$

  $\therefore AC' = \frac{bc}{a + b}$

  Similarly, $AB' = \frac{bc}{a + c}$

  $\Delta AC'B' = \frac{1}{2}\frac{bc}{a + b}\frac{bc}{a + c}\sin A$

  Let $\Delta$ be the area of $\triangle ABC,$ then $\Delta = \frac{1}{2}bc\sin A$

  $\therefore \Delta AC'B' = \frac{bc}{(a + b)(a + c)}\Delta$

  Similalry, $\Delta BA'C' = \frac{ac}{(a + b)(b + c)}\Delta$

  and $\Delta CA'B' = \frac{ab}{(a + c)(b + c)}\Delta$

  $\therefore \Delta A'B'C' = \Delta.\frac{2abc}{(a + b)(b + c)(c + a)}$

  $\therefore \frac{\Delta A'B'C'}{\Delta ABC} = \frac{2\sin A\sin B\sin C}{(\sin A + \sin B)(\sin B + \sin C)(\sin C + \sin A)}$

  $= \frac{2\sin\frac{A}{2}\sin\frac{B}{2}\sin\frac{C}{2}}{\cos\frac{A - B}{2}\cos\frac{B - C}{2}\cos\frac{C - A}{2}}$

\item The diagram is given below:

  \startplacefigure
    \externalfigure[21_7.pdf]
  \stopplacefigure

  Let the given triangle be $ABC$ and the similar triangle inscribed in triangle $ABC$ be $A'B'C'$ such that

  $A=A', B=B', C=C'$

  Let $B'C'=\lambda a, A'C'=\lambda b, A'B' = \lambda c$

  According to the quuestion $\angle B'OC=\theta$

  Clearly, $\angle OC'B = B - \theta = \angle AC'B'$

  $\angle BC'A' = 180^\circ - (B - \theta + C) = A + \theta$

  $\angle AB'C' = 180^\circ - (B - \theta + A) = C + \theta$

  $\angle A'B'C' = 180^\circ - (C + \theta + B) = A - \theta$

  Applying sine rule in the triangle $BC'A',$

  $\frac{BA'}{\sin(A +\theta)} = \frac{\lambda b}{\sin B}\Rightarrow BA' = \frac{\lambda b}{\sin B}\sin(A + \theta)$

  $\Rightarrow BA' = \lambda 2R\sin(A + \theta)$

  Applying sine rule in $A'B'C,$

  $\frac{A'C}{\sin(A - \theta)} = \frac{\lambda c}{\sin C}$

  $\Rightarrow A'C = \lambda 2R\sin(A - \theta)$

  $BC = BA' + A'C$

  $\Rightarrow a = \lambda 2R\sin(A + \theta) + \lambda 2R\sin(A - \theta)$

  $\Rightarrow a = 2R\lambda[\sin(A + \theta) + \sin(A - \theta)] = 2R\lambda 2\sin A\cos\theta$

  $\Rightarrow a = 2\lambda a\cos\theta$

  $\Rightarrow 2\lambda\cos\theta = 1$

\item We have to prove that $r_1 + r_2 + r_3 - r = 4R$

  L.H.S. $= \frac{\Delta}{s - a} + \frac{\Delta}{s - b} + \frac{\Delta}{s - c} - \frac{\Delta}{s}$

  $= \Delta\left[\frac{s - a + s - b}{(s - a)(s - b)} + \frac{s - s + c}{s(s - c)}\right]$

  $=\Delta\left[\frac{c}{(s - a)(s - b)} + \frac{c}{s(s - c)}\right]$

  $=\Delta.c \left[\frac{s(s - c) + (s - a)(s - b)}{s(s - a)(s - b)(s - c)}\right]$

  $=\Delta.c.\frac{1}{\Delta^2}[s^2 - sc + s^2 - (a + b)s + ab]$

  $=\frac{c}{\Delta}[2s^2 - s(a + b + c) + ab] = \frac{c}{\Delta}[2s^2 - 2s^2 + ab]$

  $= \frac{abc}{\Delta} = 4R =$ R.H.S.

\item We have to prove that $\frac{1}{r_1} + \frac{1}{r_2} + \frac{1}{r_3} = \frac{1}{r}$

  L.H.S. $= \frac{s - a}{\Delta} + \frac{s - b}{\Delta} + \frac{s - c}{\Delta}$

  $= \frac{3s - (a + b + c)}{\Delta} = \frac{3s - 2s}{\Delta} [\because 2s = a + b + c]$

  $= \frac{s}{\Delta} = \frac{1}{r} =$ R.H.S.

\item We have to prove that $\frac{1}{r_1^2} + \frac{1}{r_2^2} + \frac{1}{r_3^2} + \frac{1}{r^2} = \frac{a^2 + b^2 +
  c^2}{\Delta^2}$

  L.H.S. $= \frac{1}{r_1^2} + \frac{1}{r_2^2} + \frac{1}{r_3^2} + \frac{1}{r^2}$

  $= \frac{(s - a)^2 + (s - b)^2 + (s - c)^2 + s^2}{\Delta^2}$

  $= \frac{4s^2 - 2s(a + b + c) + a^2 + b^2 + c^2}{\Delta^2}$

  $=\frac{4s^2 - 2s.2s + a^2 + b^2 + c^2}{\Delta^2}$

  $= \frac{a^2 + b^2 + c^2}{\Delta^2} =$ R.H.S.

\item $r = \frac{\Delta}{s} = \frac{\Delta}{s}\frac{s - a}{s - a}$

  We know that $\tan \frac{A}{2} = \sqrt{\frac{(s - b)(s - c)}{s(s - a)}}$ and $\Delta = \sqrt{s(s - a)(s - b)(s -
    c)}$

  $\Rightarrow r = (s - a)\tan\frac{A}{3}$

  Similarly, $r = (s - b)\tan\frac{B}{2}, r = (s - c)\tan\frac{C}{2}$

\item We have to prove that $\frac{1}{\sqrt{A}} = \frac{1}{\sqrt{A_1}} + \frac{1}{\sqrt{A_2}} + \frac{1}{\sqrt{A_3}}$

  R.H.S. $= \frac{1}{\sqrt{A_1}} + \frac{1}{\sqrt{A_2}} + \frac{1}{\sqrt{A_3}}$

  $= \frac{1}{\sqrt{\pi}}\left(\frac{1}{r_1} + \frac{1}{r_2} + \frac{1}{r_3}\right)$

  $=\frac{1}{\sqrt{\pi}}\left(\frac{s - a}{\Delta} + \frac{s - b}{\Delta} + \frac{s - c}{\Delta}\right)$

  $= \frac{1}{\sqrt{\pi}}\left(\frac{3s - (a + b + c)}{\Delta}\right) = \frac{1}{\sqrt{\pi}}\frac{s}{\Delta} =
  \frac{1}{\sqrt{\pi}}.\frac{1}{r}$

  $= \frac{1}{\sqrt{A}} =$ L.H.S.

\item $\frac{r_1}{bc} + \frac{r_2}{ca} + \frac{r_3}{ab} = \frac{s\tan\frac{A}{2}}{bc} + \frac{s\tan\frac{B}{2}}{ca} =
  \frac{s\tan\frac{C}{2}}{ab}$

  $= \frac{s}{abc}\left(a\tan\frac{A}{2} + b\tan\frac{B}{2} + c\tan\frac{C}{2}\right)$

  $= \frac{s}{abc}\left(2R\sin A\tan\frac{A}{2} + 2R\sin B\tan\frac{B}{2} + 2R\sin C\tan\frac{C}{2}\right)$

  $= \frac{s}{abc}.4R\left(\sin^2\frac{A}{2} + \sin^2\frac{B}{2} + \sin^2\frac{C}{2}\right)$

  $= \frac{s}{\Delta}\left(\frac{1 - \cos A}{2} + \frac{1 - \cos B}{2} + \frac{1 - \cos C}{2}\right)$

  $= \frac{1}{r}\left(\frac{3}{2} - \frac{\cos A + \cos B + \cos C}{2}\right)$

  We know that $\cos A + \cos B + \cos C = 1 + \frac{r}{R}$

  $= \frac{1}{r}\left(1 - \frac{r}{2R}\right) = \frac{1}{r} - \frac{1}{2R} =$ R.H.S.

\item Let $D$ be the point where perpendicular from $A$ meets $BC.$ Then $AD = h$

  The diagram is given below:

  \startplacefigure
    \externalfigure[21_8.pdf]
  \stopplacefigure

  Clearly, $OB = r, AD = h, OD=h - r$ (If $O$ is below
  $BD$ then $OD = r - h$)

  $BD = \sqrt{OB^2 - OB^2} = \sqrt{r^2 - (h - r)^2} = \sqrt{2rh - h^2}$

  Area of triangle $= \frac{1}{2}.2.BD.h = h\sqrt{2rh - h^2}$

\item Let the sides be $a, b, c$ then $\Delta = \frac{1}{2}ap_1 = \frac{1}{2}bp_2 = \frac{1}{2}cp_3$

  $\Rightarrow p_1 = \frac{2\Delta}{a}, p_2 = \frac{2\Delta}{b}, p_3 = \frac{2\Delta}{c}$

  L.H.S. $= \frac{\cos A}{p_1} + \frac{\cos B}{p_2} + \frac{\cos C}{p_3}$

  $= \frac{1}{2\Delta}[a\cos A + b\cos B + c\cos C]$

  $= \frac{2R}{2\Delta}[\sin A\cos A + \sin B\cos B + \sin C\cos C]$

  $= \frac{R}{2\Delta}[\sin 2A + \sin 2B + \sin 2C] = \frac{R}{2\Delta}4\sin A\sin B\sin C$

  $= \frac{abc}{4\Delta}.\frac{1}{R^2} = \frac{R}{R^2} = \frac{1}{R} =$ R.H.S.

\item This has been already proved in 149.

\item L.H.S. $= r_1r_2r_3 = \frac{\Delta}{s - a}\frac{\Delta}{s - b}.\frac{\Delta}{s - c}$

  $= \frac{\Delta^3}{(s - a)(s - b))(s - c)} = \Delta .s$

  R.H.S. $= r^3\cot^2\frac{A}{2}\cot^2\frac{B}{2}\cot^2\frac{C}{2}$

  $= \frac{\Delta^3}{s^3}.\frac{s^2(s - a)^2}{\Delta^2}.\frac{s^2(s - b)^2}{\Delta^2}.\frac{s^2(s - c)^2}{\Delta^2}$

  $= \frac{s^3(s - a)^2(s - b)^2(s - c)^2}{\Delta^3} = \Delta .s$

\item We have to prove that $a(rr_1 + r2r_3) = b(rr_2 + r_3r_1) = c(rr_3 + r_1r_2) = abc$

  $a(rr_1 + r_2r_3) = a\left(\frac{\Delta}{s}.\frac{\Delta}{s - a} + \frac{\Delta}{(s - b)}\frac{\Delta}{s - c}\right)$

  $= \Delta^2.a\left(\frac{(s - b)(s - c) + s(s - a)}{s(s - a)(s - b)(s - c)}\right)$

  $= \Delta^2.a\left(\frac{2s^2 - s(a + b + c) + bc}{\Delta^2}\right)$

  $= abc$

  Similalry other terms can be evaluated to same value of $abc.$

\item We have to prove that $(r_1 + r_2)\tan\frac{C}{2} = (r_3 - r)\cot\frac{C}{2} = c$

  $(r_1 + r_2)\tan\frac{C}{2} = \left(\frac{\Delta}{s - a} + \frac{\Delta}{s - b}\right)\frac{(s - a)(s - b)}{\Delta}$

  $= s - b + s - a = c$

  $(r_3 - r)\cot\frac{C}{2} = \left(\frac{\Delta}{s - c} - \frac{\Delta}{s}\right)\frac{\Delta}{(s - a)( s - b)}$

  $= \Delta^2\left(\frac{s - s + c}{s(s - a)(s - b)(s - c)}\right) = c$

\item We have to prove that $4R\sin A\sin B\sin C = a\cos A + b\cos B + c\cos C$

  R.H.S. $= R(2\sin A\cos A + 2\sin B\cos B + 2\sin C\cos C) = R(\sin 2A + \sin 2B + \sin 2C)$

  $= R(2\sin(A + B)\cos(A - B) + 2\sin C\cos C) = 2R(\sin C\cos(A - B) + \sin C\cos C)[\because \sin(A + B) =
    \sin(\pi - C) = \sin C]$

  $= 2R\sin C[\cos(A - B) - \cos(A + B)][\because \cos C = \cos(\pi - A - B) = -\cos(A + B)]$

  $= 2R\sin C. 2\sin A\sin B = 4R\sin A\sin B\sin C =$ L.H.S.

\item We have to prove that $(r_1 - r)(r_2 - r)(r_3 - r) = 4Rr^2$

  L.H.S. $= \left(\frac{\Delta}{s - a} - \frac{\Delta}{s}\right)\left(\frac{\Delta}{s - b} -
  \frac{\Delta}{S}\right)\left(\frac{\Delta}{s - c} - \frac{\Delta}{s}\right)$

  $= \Delta^3\left(\frac{s - s + a}{s(s - a)}\right)\left(\frac{s - s + b}{s(s - b)}\right)\left(\frac{s - s + c}{s(s -
    c)}\right)$

  $= \Delta^3.\frac{abc}{s^3(s - a)(s - b)(s - c)} = \frac{\Delta^3.abc}{s^2\Delta^2} = \frac{abc.\Delta}{s^2}$

  $= \frac{abc}{\Delta}.\frac{\Delta^2}{s^2} = 4Rr^2 =$ R.H.S.

\item We have to prove that $r^2 + r_1^2 + r_2^2 + r_3^2 = 16R^2 - a^2 - b^2 - c^2$

  $(r_1 + r_2 + r_3 - r)^2 = r_1^2 + r_2^2 + r_3^2 + r^2 - 2r(r_1 + r_2 + r_3) + 2(r_1r_2 + r_2r_3 + r_3r_1)$

  We know that $r_1 + r_2 + r_3 - r = 4R$ and $(r_1r_2 + r_2r_3 + r_3r_1) = s^2$

  $r(r_1 + r_2 + r_3) = \frac{\Delta}{s}\left(\frac{\Delta}{s -a} + \frac{\Delta}{s - b} + \frac{\Delta}{s - c}\right)$

  $= \frac{\Delta^2}{s(s - a)} + \frac{\Delta^2}{s(s - b)} + \frac{\Delta^2}{s(s - c)} = -s^2 + (ab + bc + ca)$

  $16R^2 = r_1^2 + r_2^2 + r_3^2 + r^2 -2[-s^2 + (ab + bc + ca)] + 2s^2$

  $\Rightarrow r^2 + r_1^2 + r_2^2 + r_3^2 = 16R^2 - a^2 - b^2 - c^2$

\item We know that $IA = r\csc\frac{A}{2}, IB=r\csc\frac{B}{2}, IC=r\csc\frac{C}{2}$

  L.H.S. $= \frac{r^3}{\sin\frac{A}{2}\sin\frac{B}{2}\sin\frac{C}{2}} =
  \frac{r^3.4R}{4R\sin\frac{A}{2}\sin\frac{B}{2}\sin\frac{C}{2}}$

  $= \frac{r^3.4R}{r} = 4R.r^2 = 4R.\frac{\Delta^2}{s^2} = \frac{abc\Delta}{s^2}$

  R.H.S. $= abc.\frac{(s - a)^2(s - b)^2(s - c)^2}{\Delta^3} = \frac{abc\Delta}{s^2}$

\item From the distannces of circumcenter, incenter, orthocenter and centroid from vertices we can
  say that $AI_1 = r_1\csc\frac{A}{2}$

\item $II_1 = AI_1 - AI = r_1\csc\frac{A}{2} - r\csc\frac{A}{2}$

  $= \left(\frac{\Delta}{s - a} - \frac{\Delta}{s}\right)\csc\frac{A}{2}$

  $= \Delta .\frac{a}{s(s - a)}\sqrt{\frac{bc}{(s - b)(s - c)}} = a\sec\frac{A}{3}$

\item If $E_2$ be the point of contact of the circle whose center is $I_2$ with the side $AC$ of the triangle
  $ABC,$ we have

  $AI_2 = AE_2\sec I_2AE_2 = AE_2sec\left(90^\circ - \frac{A}{2}\right) = (s - b)\csc \frac{A}{2}$

  $I_2I_3 = AI_2 + AI_3 = (s - b + s - c)\csc\frac{A}{2} = a\csc\frac{A}{2}$

\item We have deduced that $II_1 = a\sec\frac{A}{2}$ in problem 176. So $II_2 = b\sec\frac{B}{2}$ and $II_3 =
  c\sec\frac{C}{2}$

  L.H.S. $= II_1.II_2.II_3 = abc\sec\frac{A}{2}\sec\frac{B}{2}\sec\frac{C}{2}$

  $=
  8R^3\frac{2\sin\frac{A}{2}\cos\frac{A}{2}.2\sin\frac{B}{2}\cos\frac{B}{2}.2\sin\frac{C}{2}\cos\frac{C}{2}}{\cos\frac{A}{2}\cos\frac{B}{2}\cos\frac{C}{2}}$

  $= 16R^2.4R\sin\frac{A}{2}\sin\frac{B}{2}\sin\frac{C}{2} = 16R^2r =$ R.H.S.

\item $II_1 = a\sec\frac{A}{2}, I_2I_3 = a\csc\frac{A}{2}$

  $II_1^2 + I_2I_3^2 = a^2\left(\frac{1}{\sin^2\frac{A}{2}} + \frac{1}{\sin^2\frac{A}{2}}\right)$

  $= \left(\frac{a}{\sin\frac{A}{2}\cos\frac{A}{2}}\right)^2 = \left(\frac{2a}{2\sin\frac{A}{2}\cos\frac{A}{2}}\right)^2$

  $= 16R^2 [\because a = 2R\sin A]$

  Similalrly other terms can be proven to be equal to $16R^2$

\item We know that $OI^2 = R^2\left(1 - 8\sin\frac{A}{2}\sin\frac{B}{2}\sin\frac{C}{2}\right)$

  $= R^2\left[1 - 4\left(\cos \frac{A - B}{2} - \cos\frac{A + B}{2}\right)\sin\frac{C}{2}\right]$

  $= R^2\left[1 - 4\cos\frac{A - B}{2}\cos\frac{A + B}{2} + 4\sin^2\frac{C}{2}\right][\because \sin \frac{C}{2} = \cos\frac{A + B}{2}]$

  $= R^2\left[1 - 2(\cos A + \cos B) + 2(1 - \cos C)\right]$

  $=R^2(3 - 2\cos A - 2\cos B - 2\cos C)$

\item We have, $IH^2 = AH^2 + AI^2 - 2.AH.AI.\cos IAH$

  $\angle IAH = \frac{A}{2} - \angle HAC = \frac{A}{2} - (90^\circ - C) = \frac{C - B}{2}$

  $IH^2 = 4R^2\cos^2A + 16R^2\sin^2\frac{B}{2}\sin^2\frac{C}{2} - 16R^2\cos A\sin\frac{B}{2}\sin\frac{C}{2}\cos\frac{C - B}{2}$

  $= 4R^2\left[\cos^2A + 4\sin^2\frac{B}{2}\sin^2\frac{C}{2} - 4\cos A\sin\frac{B}{2}\sin\frac{C}{2}\cos\frac{C}{2}\cos\frac{B}{2} - 4\cos A\sin^2\frac{B}{2}\sin^2\frac{C}{2}\right]$

  $= 4R^2\left[\cos^2A + 4\sin^2\frac{B}{2}\sin^2\frac{C}{2}(1 - \cos A) - \cos A\sin B\sin C\right]$

  $= 4R^2\left[8\sin^2\frac{A}{2}\sin^2\frac{B}{2}\sin^2\frac{C}{2} + \cos^2A - \cos A\sin B\sin C\right]$

  $= 2r^2 + 4R^2\cos A(\cos A - \sin B\sin C)$

  $= 2r^2 - 4R^2\cos A\cos B\cos C$

\item We know that $OG = \frac{1}{3}OH \Rightarrow OG^2 = \frac{OH^2}{9}$

  $= \frac{1}{9}[R^2 - 8R^2\cos A\cos B\cos C] = \frac{R^2}{9}[1 - 4\{\cos(A + B) + \cos(A - B)\}\cos C]$

  $= \frac{R^2}{9}[1 + 4\cos^2C + 4\cos(A - B)\cos(A + B)]$

  $= \frac{R^2}{9}[1 + 2(1 + \cos 2C) +2(\cos 2A + \cos 2C)]$

  $= \frac{R^2}{9}[3 + \cos 2A + \cos 2B + \cos 2C]$

  $= \frac{R^2}{9}[9 - 2(1 - \cos 2A) - 2(1 - \cos 2B) - 2(1 - \cos 2C)]$

  $= \frac{R^2}{9}[9 - 4(\sin^2A + \sin^2B + \sin^2C)]$

  $= R^2 - \frac{1}{9}(2R\sin A)^2 - \frac{1}{9}(2R\sin B)^2 - \frac{1}{9}(2R\sin C)^2$

  $= R^2 - \frac{1}{9}(a^2 + b^2 + c^2)$

\item The diagram is given below:

  \startplacefigure
    \externalfigure[21_9.pdf]
  \stopplacefigure

  Clearly, $R = \frac{b}{2\sin\frac{\alpha}{2}} = \frac{b\csc\frac{\alpha}{2}}{2}$

\item We know that in a $\triangle ABC,$ $r = 4R\sin\frac{A}{2}\sin\frac{B}{2}\sin\frac{C}{2}$

  Let $AD$ be the perpedicular bisector to $BC.$

  $\Delta = BD.AD = b\cos\alpha.b\sin\alpha = \frac{1}{2}b^2\sin2\alpha$

  $r = \frac{\Delta}{s} = \frac{\frac{1}{2}b^2\sin2\alpha}{\frac{1}{2}(b + b + 2b\cos\alpha)} = \frac{b\sin2\alpha}{2(1 + \cos\alpha)}$

\item $OI = |OD + DI| = |OD + r|$ because $\alpha < \pi/4, A>\pi/2$ and $O$ lies on $AD$ produced.

  From right-angled $\triangle ODB,$ we get

  $OD^2 = OB^2 - BD^2 = R^2 - b^2\cos^2\alpha$

  $= \frac{1}{4}\frac{b^2}{\sin^2\alpha} - b^2\cos^2\alpha$

  $= \frac{b^(1 - 4\sin^2\alpha\cos^2\alpha)}{4\sin^2\alpha} = \frac{b^2(\cos^2\alpha - \sin^2\alpha)}{4\sin^2\alpha}$

  $= \frac{b^2\cos^22\alpha}{(2\sin\alpha)^2}$

  $\therefore OI = \left|\frac{b\sin2\alpha}{2(1 + \cos\alpha)} + \frac{b\cos2\alpha}{2\sin\alpha}\right|$

  $= \left|\frac{b\sin2\alpha}{4\cos^2\frac{\alpha}{2}} + \frac{b\cos2\alpha}{4\sin\frac{\alpha}{2}\cos\frac{\alpha}{2}}\right|$

  $= \left|\frac{b}{4\cos\alpha/2}.\frac{\sin2\alpha\sin\alpha/2 + \cos2\alpha\cos\alpha/2}{\sin\alpha/2\cos\alpha/2}\right|$

  $= \left|\frac{b\cos3\alpha/2}{2\sin\alpha\cos\alpha/2}\right|$

\item L.H.S. $= \frac{1}{ab} + \frac{1}{bc} + \frac{1}{ca} = \frac{a + b + c}{abc}$

  $= \frac{2s}{4RS} = \frac{1}{2RS/s} = \frac{1}{2Rr} =$ R.H.S.

\item We know that $r_1 = \frac{\Delta}{s - a}, r_2 = \frac{\Delta}{s - b}, r_3 = \frac{\Delta}{s - c}$ and $r = \frac{\Delta}{s}$

  L.H.S. $= \frac{3\Delta}{(s - a)(s - b)(s - c)} = \frac{3s\Delta}{\Delta^2} = \frac{3s}{\Delta} = \frac{3}{r} =$ R.H.S.

\item The diagram is given below:

  \startplacefigure
    \externalfigure[21_10.pdf]
  \stopplacefigure

  $2s = 2(\alpha + \beta + \gamma) \Rightarrow s = \alpha + \beta + \gamma$

  We know that $r = \frac{\Delta}{s} \Rightarrow s^2 = \frac{(s - a)(s - b)(s - c)}{s}$

  Clearly, $s - a = \gamma, s - b = \beta, s - c = \alpha$

  $\Rightarrow s = \frac{\alpha\beta\gamma}{\alpha + \beta + \gamma}$

\item The diagram is given below:

  \startplacefigure
    \externalfigure[21_11.pdf]
  \stopplacefigure

  Let $PQ=x, PQ\parallel BC, RS = y, RS\parallel AC, TU = z, TU\parallel AB$

  In $\triangle APQ, \frac{x}{\sin A} = \frac{AQ}{\sin B} = \frac{AP}{\sin C}$

  $\Rightarrow AQ = \frac{bx}{a}, AP = \frac{cx}{a}$

  $r = \left(\frac{x + AP + AQ}{2}\right)\tan\frac{A}{2} = \frac{a + b + c}{2}x\tan\frac{A}{2}$

  $= \frac{sx}{a}\tan\frac{A}{2} = (s - a)\tan\frac{A}{2}$

  $\Rightarrow \frac{sx}{a} = s - a$

  Similarly, $\frac{sy}{b} = s - b$ and $\frac{sz}{c} = s - c$

  $\Rightarrow s\left(\frac{x}{a} + \frac{y}{b} + \frac{z}{c}\right) = 3s - (a + b + c)$

  $\Rightarrow \frac{x}{a} + \frac{x}{b} + \frac{x}{c} = 1$

\item The diagram is given below:

  \startplacefigure
    \externalfigure[21_12.pdf]
  \stopplacefigure

  Since $I$ is the incenter, $AI$ will be angle bisector. Let $AI$ cut circumcirlce at $D.$

  $\angle DBI=\angle DBC+\angle IBC=\angle DAB+\angle ABI=\angle BID$ and then $DB=DI$

  Likewise $DC = DI$ and then $DB = BI = DC$

  $I_1C$ bisects $\angle BCT \Longrightarrow \angle ICI_1 = 90^\circ$

  Let the perpendicular bisector of $BC$ cut circumcircle at $M$ also.

  $\triangle SAI_1 ~ \triangle BMD$

  Power of $I_1$ w.r.t the circumcircle of $\triangle ABC = OI_1^2 - R^2 = I_1D.I_1A$

  $\Rightarrow \frac{MD}{BD} = \frac{I_1A}{SI_1} \Rightarrow 2Rr_1 = OI_1^2 - R^2$

  Thus, $OI_1 = R^2 + 2Rr_1$

  Thus length of tangent $t_1^2 = OI_1^2 - R^2 = 2Rr_1$

  $\frac{1}{t_12} + \frac{1}{t_2^2} + \frac{1}{t_3^2} = \frac{1}{2R}\left(\frac{1}{r_1} + \frac{1}{r_2} +
  \frac{1}{r_3}\right)$

  $= \frac{1}{2R}\frac{3s - (a + b + c)}{\Delta} = \frac{4\Delta}{2}.\frac{s}{\Delta} = \frac{2s}{abc}$

\item $r_1 = \frac{\Delta}{s - a}$ and so on. Given,

  $\left(1 - \frac{r_1}{r_2}\right)\left(1 - \frac{r_1}{r_3}\right) = 2$

  $\left(1 - \frac{s - b}{s - a}\right)\left(1 - \frac{s - c}{s - a}\right) = 2$

  $(b - a)(c - a) = 2(s - a)^2$

  $bc + a^2 - ac - ab = (b + c - a)^2/2$

  $b^2 + c^2 = a^2$

  Thus the triangle is right-angled.

\item $\frac{\text{Area of in-circle}}{\text{Area of triangle}} = \frac{\pi r^2}{\Delta} = \frac{\pi}{\Delta}.\frac{\Delta^2}{s^2}$

  $= \pi .\frac{\Delta}{s^2}$

  $\cot\frac{A}{2}\cot\frac{B}{2}\cot\frac{C}{2} = \sqrt{\frac{s(s - a)}{(s - b)(s - c)}.\frac{s(s - b)}{(s - a)(s - c)}.\frac{s(s - c)}{(s - a)(s - b)}}$

  $= \sqrt{\frac{s^4}{s(s - a)(s - b)(s - c)}} = \frac{s^2}{\Delta}$

  $\Rightarrow \frac{\pi}{\cot\frac{A}{2}\cot\frac{B}{2}\cot\frac{C}{2}} = \frac{\pi\Delta}{s^2}$

  Thus, we have the desired result by combining both the equations.

\item The diagram is given below:

  \startplacefigure
    \externalfigure[21_13.pdf]
  \stopplacefigure

  Let $O$ be the center of the regular polygon $A_1, A_2, A_3, \ldots, A_n$ which has $n$ sides.

  Since it is a regular polyogn so $\angle A_1OA_2 = \angle A_2OA_3 = \ldots = \angle A_nOA_1 = \frac{2\pi}{n}$

  Also, let $OA_1 = OA_2 = \ldots = OA_n = r$

  Applying cosine rule in $\triangle A_1OA_2,$

  $\cos\frac{2\pi}{n} = \frac{OA_1^2 + OA_2^2 - A_1A_2^2}{2OA_1.OA_2}$

  $\Rightarrow A_1A_2^2 = 2r^2\left(1 - \cos \frac{2\pi}{n}\right)$

  $\Rightarrow A_1A_2^2 = 4r^2\sin^2\frac{\pi}{n}$

  $\Rightarrow A_1A_2 = 2r\sin\frac{\pi}{n}$

  Likewise $A_1A_3 = 2r\sin\frac{2\pi}{n} A_1A_4 = 2r\sin\frac{3\pi}{n}$

  Given, $\frac{1}{A_1A_2} = \frac{1}{A_1A_3} + \frac{1}{A_1A_4}$

  $\Rightarrow \frac{1}{2r\sin\frac{\pi}{n}} = \frac{1}{2r\sin\frac{2\pi}{n}} + \frac{1}{2r\sin\frac{3\pi}{n}}$

  $\Rightarrow \frac{1}{\sin\frac{\pi}{n}} - \frac{1}{\sin\frac{3\pi}{n}} = \frac{1}{\sin\frac{2\pi}{n}}$

  $\Rightarrow \frac{2\cos\frac{2\pi}{n}\sin\frac{\pi}{n}}{\sin\frac{\pi}{n}\sin\frac{3\pi}{n}} = \frac{1}{\sin\frac{2\pi}{n}}$

  $\Rightarrow 2\cos\frac{2\pi}{n}\sin\frac{2\pi}{n} = \sin\frac{3\pi}{n}$

  $\Rightarrow \sin\frac{4\pi}{n} = \sin\frac{3\pi}{n}$

  $\Rightarrow \cos\frac{7\pi}{2n}.\sin\frac{\pi}{2n} = 0$

  $\Rightarrow \cos\frac{7\pi}{2n} = 0\left[\sin\frac{\pi}{2n} \neq 0\right]$

  $\Rightarrow \frac{7\pi}{2n} = \text{odd integer} \times \frac{\pi}{2}$

  $\Rightarrow n = \frac{7}{\text{odd integer}} = 7[n\in I, n > 1]$

\item The diagram is given below:

  \startplacefigure
    \externalfigure[21_14.pdf]
  \stopplacefigure

  Let $O$ be the center of the circumscribing circle regular polygon $A_1, A_2, A_3, \ldots, A_n$ which has $n$ sides.

  Since the polygon is regular, therefore $O$ will also be the center of inscribing circle.

  Let $OD\perp A_1A_2.$ Now $\angle A_1OA_2 = \frac{2\pi}{n}$

  $\angle A_OD = \angle A_2OD = \frac{\pi}{n}$

  Also, $A_1D = A_2D = a/2$ where $a$ is the length of a side of the polygon.

  Here, $R =$ radius of the circumscribing circle $= OA$

  and $r =$ radius of the inscribing circle $= OD$

  From right angled triangle $ODA_1$

  $\sin\frac{\pi}{n} = \frac{a/2}{R} \Rightarrow R = \frac{a}{2}\csc\frac{\pi}{n}$

  and $\tan\frac{\pi}{n} = \frac{a/2}{r} \Rightarrow r = \frac{a}{2}\cot\frac{\pi}{n}$

  $\therefore R + r = \frac{a}{2}\left[\csc\frac{\pi}{n} + \cot\frac{\pi}{n}\right]$

  $= \frac{a}{2}\left[\frac{1 + \cos\frac{\pi}{n}}{\sin\frac{\pi}{n}}\right]$

  $= \frac{a}{2}\cot\frac{\pi}{2n}$

\item Let $ABCD$ be a quadrilateral such that $AB = 3 cm, BC = 4 cm, CD= 5 cm$ and $AD = 6 cm.$

  Also, let that $\angle BAC = \theta$ and $\angle BCD = 120^\circ - \theta$ as it is given that sum of pair of
  opposite angles is $120^\circ.$

  Applying cosine law in $\triangle ABD,$

  $\cos\theta = \frac{AB^2 + AD^2 - BD^2}{2.AB.AD} \Rightarrow BD = 45 - 36\cos\theta$

  Applying cosine law in $\triangle BCD,$

  $\cos(120^\circ - \theta) = \frac{BC^2 + CD^2 - BD^2}{2.BC.CD} \Rightarrow BD^2 = 41 + 20\cos \theta - 20\sqrt{3}\sin\theta$

  Thus, $45 - 36\cos\theta = 41 + 20\cos\theta - 20\sqrt{3}\sin\theta$

  $\Rightarrow 14\cos\theta - 5\sqrt{3}\sin\theta = 1$

  Area of the quadrilateral $= \Delta ABD + \Delta BCD$

  $= \frac{1}{2}3.6.\sin\theta + \frac{1}{2}4..5\sin(120^\circ - \theta)$

  $= 14\sin\theta + 5\sqrt{3}\sin\theta = z$ (let)

  Solving the two equations thus obtaiined, we get

  $196(\sin^2\theta + \cos^2\theta) + 75(\cos^2\theta + \sin^2\theta) = z^2 + 1$

  $\Rightarrow z = 2\sqrt{30}$ sq.cm.

\item Let $ABCD$ be a cyclic quadrilateral such that $AD = 2, AB = 5, \angle DAB = 60^\circ$

  Since the quadrilateral is cyclic $\angle BCD = 120^\circ$

  Area of quadrilateral $ABD = \frac{1}{2}.2.5.\sin60^\circ = \frac{5\sqrt{3}}{2}$

  Area of $\triangle BCD =$ Area of quadrilateral $ABCD$ - Area of $\triangle ABD$

  $= 4\sqrt{3} = \frac{5\sqrt{3}}{2} = \frac{3\sqrt{3}}{2}$

  Let $CD = x, BC = y$

  Now area of $\triangle BCD = \frac{1}{2}.x.y.\sin120^\circ \Rightarrow \frac{3\sqrt{3}}{2} = \frac{1}{2}xy\frac{\sqrt{3}}{2}$

  $\Rightarrow xy = 6$

  Applying cosine rule in $\triangle ABD,$

  $\cos60^\circ = \frac{AD^2 + AB^2 - BD^2}{2.AD.AB} \Rightarrow BD^2 = 19$

  Applying cosine rule in $\triangle BCD,$

  $\cos120^\circ = \frac{x^2 + y^2 - 19}{2xy}\Rightarrow x^2 + y^2 = 13$

  $(x + y)^2 = 25 \Rightarrow x + y = \pm5$

  $(x - y)^2 = 1 \Rightarrow x - y = \pm1$

  $\Rightarrow x = 3, y = 2$ or $x = 2, y = 3$

\item From question, $AB = 1, BD = \sqrt{3}.$ Let $BAD=\theta, AD = x, BC = y$ and $CD = z.$ Since the given
  circle is also circum-circle of $\triangle ABD, \Rightarrow \frac{BD}{\sin\theta} = 2R$

  $\Rightarrow \sin\theta = \frac{\sqrt{3}}{2}\Rightarrow \theta = 60^\circ$

  Now $\angle BCD = 180^\circ - 60^\circ = 120^\circ$

  Applying cosine rule in $\triangle ABD,$

  $\cos60^\circ = \frac{AB^2 + AD^2 - BD^2}{2.AB.AD} \Rightarrow \frac{1}{2} = \frac{1 + x^2 - 3}{2x}$

  $\Rightarrow x^2 - x - 2 = 0 \Rightarrow x = 2$

  Applying cosine law in $\triangle BCD,$

  $\cos120^\circ = \frac{y^2 + z^2 - 3}{2yz} \Rightarrow y^2 + z^2 + yz = 3$

  Area of quadrilateral $ABCD =$ Area of $\triangle BCD$ + Area of $\triangle ABD$

  $\frac{3\sqrt{3}}{2} = \frac{1}{2}1.x.\sin60^\circ + \frac{1}{2}yz\sin 120^\circ$

  $\Rightarrow yz = 1$

  $\Rightarrow y^2 + z^2 = 2$

  $\Rightarrow (y + z)^2 = 4, (y - z)^2 = 0$

  $\Rightarrow y = z = 1$

\item Let $ABCD$ be the cyclic quadrilateral in which $AB = a, BC = b, CD = c, DA =d.$

  Applying cosine rule in $\triangle ABC,$

  $\cos B = \frac{a^2 + b^2 - AC^2}{2ab} \Rightarrow AC^2 = a^2 + b^2 - 2ab\cos B$

  Applying cosine rule in $\triangle ADC,$

  $\cos(\pi - B) = \frac{c^2 + d^2 - AC^2}{2cd} \Rightarrow AC^2 = c^2 + d^2 + 2cd\cos B$

  $\Rightarrow \cos B = \frac{a^2 + b^2 - c^2 - d^2}{2(ab + cd)}$

  $\tan^2\frac{B}{2}= \frac{1 - \cos B}{1 + \cos B}$

  $\Rightarrow \tan\frac{B}{2} = \sqrt{\frac{(S - a)(S - b)}{(S - c)(S - d)}}$

\item Let $ABCD$ be the quadrilateral for which $Ab = a, BC=b, CD=c, DA = d.$ Let diagonals be, $AC=x, BD = y.$

  Given, $\angle AOD = \angle BOC = \alpha \therefore \angle AOB = \angle COD = 180^\circ - \alpha$

  Area of the quadrilateral $= \frac{1}{2}xy\sin\alpha$

  Applying cosine rule in $\triangle AOB,$

  $a^2 = AO^2 + BO^2 - 2.AO.BO.\cos(180^\circ - \alpha)$

  $a^2 = AO^2 + BO^2 + 2.AO.BO.\cos\alpha$

  Likewise in $\triangle BOC,$

  $b^2 = BO^2 + CO^2 + 2.BO.CO.\cos \alpha$

  And in $\triangle COD$

  $c^2 = CO^2 + DO^2 + 2.CO.DO.\cos\alpha$

  And in $\triangle AOD,$

  $d^2 = AO^2 + DO^2 + 2AO.DO.\cos\alpha$

  $a^2 + c ^2 - b^2 - d^2 = 2\cos\alpha.x.y$

  Thus, area $= \frac{1}{2}(a^2 + c^2 - b^2 - d^2)$

\item If the quadrilateral $ABCD$ can have a circle inscribed such that it touches the quadrilateral on sides $AB, BC,
  CD, DA$ at points $P,Q,R,S$ then we will have

  $AP = AS, BP = BQ, CQ = CR, DR = DS$

  Since lengths of tnagents are equal,

  $\therefore AP + BP + CR + DR = AS + BQ + CQ + DS$

  $\Rightarrow AB + CD = AD + BC$

  $\Rightarrow a + c = b + d$

  $\Rightarrow s = a + c = b + d$

  Area of cyclic quadrilateral $= \sqrt{(s - a)(s - b)(s - c)(s - d)} = \sqrt{abcd} = \frac{1}{2}r(a + b + c + d)$ where
  $r$ is the in-radius.

  $r = \frac{2\sqrt{abcd}}{a + b + c + d}$
\item Let the sides of the cyclid quadrilateral $ABCD$ having sides $AB = 3, BC = 3, CD = 4, DA = 4.$

  Join $B$ with $D.$ Clearly, $BD$ is diameter of circum-circle so it will subtend a right-angle at $A$
  and $C.$

  Clearly $\triangle ABD\cong \triangle BCD$

  $\therefore BD = \sqrt{3^2 + 4^2} = 5$

  Thus, radius of the circumcircle $= 2.5$ cm

  Also, $AD + BC = AB + CD$

  $\therefore$ Area of the quadrilateral $= \sqrt{abcd} = 12$ sq. cm.

  We know that $12 = rs \Rightarrow r = 12/7$ cm where $r$ is radius of in-circle.

\item Let the quadrilateral be $ABCD$ and let points be $E, F, G, H, I, J, K, L$ on sides in order.

  $\therefore EF=FG=GH=HI=IJ=JK=KL=LE=x$

  By symmetry, $AE=AL=BG=BF=CH=CI=DJ=DK=a$

  Using Pythagoras theorem in $\triangle ALE,$

  $LE^2 = AL^2 + AE^2 \Rightarrow x^2 = \sqrt{2}a$

  $\therefore AB = a + x + a = 2a + x = 2$

  $\Rightarrow a = \frac{\sqrt{2}}{\sqrt{2} + 1}$

  $\Rightarrow x = \frac{2}{\sqrt{2} + 1}$

  Thus, length of each side of octagon is $\frac{2}{\sqrt{2} + 1}.$

\item Let the perimeter be $6a$ the sides of hexagon will be $a$ and sides of triangle will be $2a.$

  We know that for a regular polygon having $n$ sides with each side's length $a$ the area is
  $\frac{na^2}{4}\cot\frac{\pi}{n}$

  Area of regular hexagon $= \frac{6a^2}{4}\cot30^\circ = \frac{3\sqrt{3}a^2}{2}$

  Area of equilateral triangle $= \frac{3.4a^2}{4}\cot60^\circ = \sqrt{3}a^2$

  $\therefore$ Ratio of areas $= 2/3$

\item Following the above problem $n = 3$

\item Ratio of areas $= \frac{3}{4}$

  $\therefore$ Ratio of sides $= \frac{\sqrt{3}}{2}$

  $\therefore$ Ratio of altitudes $= \sqrt{3}/2$

  $\Rightarrow \sin\theta = \sqrt{3}/2$

  $\theta = 60^\circ, 120^\circ$ so it is an equilateral triangle and a hexogon.

\item We know that angle of a polygon having $n$ sides is $(n - 2)\pi/n$

  Let there be $n$ sides in one polygon and $2n$ in another.

  Ratio $= \frac{2n - 2}{n - 2}.\frac{n}{2n} = \frac{9}{8}$

  $(n - 1)/n - 2 = 9/8 \Rightarrow n = 10 \Rightarrow  2n = 20$

\item The diagram is given below:

  \startplacefigure
    \externalfigure[21_15.pdf]
  \stopplacefigure

  The six touching circle will form a hexgon. The sector internal to hexgon is of angle $120^\circ.$
  Side of hexagon will be double the radius of circles i.e. $2a$ as shown in figure.

  Area inside circles = Area of hexgon - 6*area of sector

  $= \frac{3\sqrt{3}}{2}4a^2 - 6.\frac{\pi a^2}{3} = 6\sqrt{3}a^2 - 2\pi a^2$

  $= 2a^2(3\sqrt{3} - \pi)$

\item Let $O$ is the radius of circumcircle of the square. Given,
  $AB = 1, BD = \sqrt{3}$ Also, $OA = OB = OD = 1$

  Thus, for $\triangle ABD, R = 1, \frac{a}{\sin A = 2R} = 2$

  $\frac{\sqrt{3}}{\sin A} = 2 \therefore A = 60^\circ$

  $\Rightarrow C = 120^circ$ [opposite angle in cyclic quadrilateral]

  By cosine law in $\triangle ABD$

  $3 = 1 + x^2 - 2x\cos60^\circ$

  $x^2 - x - 2 = 0 \Rightarrow x = 2$

  Thus, $\Delta = \frac{3\sqrt{3}}{4} = \frac{1}{2}1.2.\sin60^\circ +
  \frac{1}{2}c.d\sin60^\circ$

  $\therefore cd = 1$

  By cosine law in $\triangle BCD,$

  $3 = c^2 + d^2 - 2cd\cos 120^\circ \Rightarrow c^2 + d^2 = 2$

  $\Rightarrow c = d = 1$

  $BC = 1, CD = 1, AD = x = 2$

\item Let $AB = a, BC = b, CD = c, DA = d$

  In $\triangle ABC,$

  $AC^2 = a^2 + b^2 - 2ab\cos B$

  In $\triangle ADC,$

  $AC^2 = c^2 + d^2 - 2cd\cos D$

  $B + D = \pi$ [Angles opposite in a cyclic quadrilateral]

  $\Rightarrow AC^2(ab + cd) = (a^2 + b^2)cd + (c^2 + d^2)ab$

  Simirlarly, we can find $BD$ and then we find that

  $(AC.BD)^2 = (ac + bd)62$

  $\Rightarrow AC.BD = AB.CD + BC.AD$

\item Let $p$ be the perimeter of both the polygons. So the length of the
  sides will be $p/n$ and $p/2n.$ Let $A_1$ and
  $A_2$ denote the areas for them.

  $A_1 = \frac{1}{4}.n.\frac{p^2}{n^2}\cot\frac{\pi}{n}$

  $A_2 = \frac{1}{4}.2n.\frac{p^2}{4n^2}\cot\frac{\pi}{2n}$

  $\frac{A_1}{A_2} = \frac{2\cot\frac{\pi}{n}}{\cot\frac{\pi}{2n}}$

  $= \frac{2\cos\frac{\pi}{n}}{1 + \cos\frac{\pi}{2n}}$

\item Let $P = \sin\frac{A}{2}\sin\frac{B}{2}\sin\frac{C}{2}$ and $C$ is fixed.

  $P = \frac{1}{2}\left[\cos\frac{A - B}{2} - \cos\frac{A + B}{2}\right]\sin\frac{C}{2}$

  $= \frac{1}{2}\left[\cos \frac{A - B}{2} - \sin\frac{C}{2}\right]\sin\frac{C}{2}$

  As $C$ is fixed, the value $P$ will depend on the value of $\cos\frac{A - B}{2}$ and $P$ will be
  maxmimum when $A = B$

  Similalry, when $B$ is fixed $P$ will be maxed when $A = C,$ and when $A$ is fixed $P$ will be
  maxed when $B = C$

  Thus, $P$ will be maximum when $A = B = C = \pi/3$

  $\Rightarrow P_{max} = 1/8$

  Hence proved.

\item Let $A$ be the arithmetic mean. Then $A = \frac{\cos\left(\alpha + \frac{\pi}{2}\right) + \cos\left(\beta +
  \frac{\pi}{2}\right) + \cos\left(\gamma + \frac{\pi}{2}\right)}{3}$

  $= -\frac{\sin\alpha + \sin\beta + \sin\gamma}{3}$

  Clearly, $\sin\alpha + \sin\beta + \sin\gamma$ will be maximum when $\alpha = \beta = \gamma$ as proved in last
  problem making $A$ minimum.

  $\alpha + \beta + \gamma = 2\pi$

  $A_{min} = 3\sin\frac{2\pi}{3}.\frac{1}{3} = \frac{\sqrt{3}}{2}$

\item Let $\tan\frac{A}{2} = x, \tan\frac{B}{2} = y, \tan\frac{C}{2} = z$


  We know that $x^2 + y^2 + z^2 - xy - yz - xz \geq 0$

  $x^2 + y^2 + z^2 \geq xy + yz + xz$

  $A + B + C = \pi$

  $\Rightarrow \tan\left(\frac{A}{2} + \frac{B}{2}\right) = \cot\frac{C}{2}$

  $\Rightarrow \tan\frac{A}{2}\tan\frac{B}{2} + \tan\frac{B}{2}\tan\frac{C}{2} + \tan\frac{C}{2}\tan\frac{A}{2} = 1$

  $\Rightarrow xy + yz + xz = 1$

  $\Rightarrow x^2 + y^2 + z^2 \geq 1$

\item Given, $2b = (m + 1)a \Rightarrow m = \frac{2b}{a} - 1$

  $\Rightarrow \cos A = \frac{1}{2}\sqrt{\frac{(m - 1)(m + 3)}{m}}$

  $\Rightarrow \frac{b^2 + c^2 - a^2}{2bc} = \frac{1}{2}\sqrt{\frac{\left(\frac{2b}{a} - 2\right)\left(\frac{2b}{a} + 2\right)}{m}}$

  $\Rightarrow \frac{b^2 + c^2 - a^2}{(m + 1)ac} = \frac{1}{a}\sqrt{\frac{(b - a)(b + a)}{m}}$

  $\Rightarrow c^2\sqrt{m} - (m + 1)p.c + p^2\sqrt{m} = 0$

  This is a quadratic equation in $c$ and thus it will have two values.

  $\Rightarrow c_1, c_2 = p/m, \sqrt{m}p$

  $\Rightarrow c_2/c_1 = m$

\item Let $a, b, c$ be the sides of the triangle.

  $\Rightarrow s = (a + b + c)/2$ and $s - a, s- b, s - c$ will be all greater than zero.

  For positive quantities A.M. $>$ G.M.

  $\therefore \frac{s + s - a + s - b + s - c}{4} > [s(s - a)(s - b)(s - c)]^{1/4}$

  $\Rightarrow \frac{2s}{4}>\Delta^{1/2}$

  $\Rightarrow \Delta < \frac{s^2}{4}$

\item $A + B + C = \pi$

  $B + C = \pi - \frac{\pi}{4} \Rightarrow C = \frac{3\pi}{4} - B$

  Let $p = \tan A\tan B\tan C = \tan B\tan\left(3\pi/4 - B\right)$

  $p = \tan\frac{\tan3\pi/4 - \tan B}{1 + \tan3\pi/4\tan B}$

  $= \tan B\left(\frac{-1 - \tan B}{1 - \tan B}\right)$

  $p - p\tan B = -\tan B - \tan^2B$

  $\tan^2B + (1 - p)\tan B + p = 0$

  For $\tan B$ to be real, $D\geq 0$

  $\Rightarrow (1 - p)^2 - 4p \geq 0$

  $p = 3\pm 2\sqrt{2}$

  Clearly, both $B$ and $C$ both cannot be obtuse.

  If either of $B$ or $C$ is obtuse angle, then

  $\tan B\tan C < 0$ $\Rightarrow p < 0$

  If both are acute then

  $\pi/4 < B < \pi/2, \pi/4 < C < \pi/2$

  $\Rightarrow \tan B>1, \tan C> 1$

  $\Rightarrow \tan B\tan C > 1 \Rightarrow p > 1$

  $\Rightarrow p < 0, p\geq 3 + 2\sqrt{2}$

\item Let $ABC$ be a triangle and $AD, BE, CF$ be lines drawn from vertices to opposite sides such that $\angle
  ADC = \angle BES = \angle CFB = \alpha$

  Let the triangle formed by $AD, BE, CF$ be $A'B'C'$

  Clearly, $\angle B'A'C' = \angle FA'B = \pi - (\angle BFA' + \angle FBA') = \pi - [\alpha + \pi - (\alpha + A)] = A$

  Similarly, $\angle A'B'C'= B$ and $A'C'B' = C$

  Thus, $\triangle ABC ~ \triangle A'B'C'$

  $\frac{\text{Area of }\triangle A'B'C'}{\text{Area of }\triangle ABC} = \frac{B'C'^2}{a^2}$

  Applying sine rule in $AC'B,$

  $\frac{AC'}{\sin[\pi - (A + \alpha)]} = \frac{AB}{\sin(\pi - C)}$

  $\Rightarrow AC' = 2R\sin(A + \alpha)$

  Similarly, $BC' = 2R\sin(\alpha - A)$

  $\Rightarrow B'C' = AC' - AB' = 2a\cos\alpha$

  Thus, ratio of areas $= 4\cos^2\alpha:1$

\item Given, $a,b,c$ and $\Delta$ are rational.

  $s = (a + b + c)/2$ will be rational.

  $\tan\frac{B}{2} = \frac{\Delta}{s(s - a)}$ will be rational as all terms involved are rational.

  Similarly, $\tan\frac{C}{2}$ will be rational.

  $\sin B = \frac{2\tan\frac{B}{2}}{1 + \tan^2\frac{B}{2}}$ will be rational as $\tan\frac{B}{2}$ is rational.

  Likewise $\sin C$ will be rational.

  $\frac{A}{2} = 90^\circ - (C/2 + B/2)$

  $\tan\frac{A}{2} = \cot\left(\frac{B}{2} + \frac{C}{2}\right)$

  $= \frac{1 - \tan\frac{B}{2}\tan\frac{C}{2}}{\tan\frac{B}{2} + \tan\frac{C}{2}}$ will be rational as all the terms
  involved are rational.

  Thus, $\sin A$ will also be rational.

  $\frac{a}{\sin A} = \frac{b}{\sin B} = \frac{c}{\sin C} = 2R$ are rational as all the terms involved are rational.

  $R = \frac{abc}{4\Delta} \Rightarrow \Delta = \frac{abc}{4R}$ will be rational as well.

\item Applying sine rule, $\frac{b}{\sin B} = \frac{c}{\sin C}$

  $\Rightarrow \frac{\sqrt{6}}{\sin 30^\circ} = \frac{4}{\sin C}$

  $\sin C = \frac{2}{\sqrt{6}} < 1$

  So $C$ may be acute or obtuse.

  We observed that $b < c \Rightarrow B < C,$ so $B$ may be acute or obtuse.

  If $C$ is obtuse $B$ may be acute. Hence two triangles are possible.

  Applying cosine rule, $\cos B = \frac{c^2 + a^2 - b^2}{2ac} = \frac{16 + a^2 - 6}{2.4.a}$

  $\frac{\sqrt{3}}{2} = \frac{10 + a^2}{8a} \Rightarrow a^2 - 4\sqrt{3}a + 10 = 0$

  $a = 2\sqrt{3}\pm 2$

  $\therefore \Delta_1, \Delta_2 = \frac{1}{2}a.c.\sin30^\circ = 2\sqrt{3}\pm\sqrt{2}$

\item Let $\triangle ABC$ be the equilateral triangle such that its sides have length $a.$

  $s = \frac{3a}{2}, \Delta = \frac{\sqrt{3}}{2}a$

  $r = \frac{\Delta}{s} = \frac{a}{2\sqrt{3}}$

  Diameter of incircle will be diagonal of inscribed square i.e. $2r = \frac{a}{\sqrt{3}}$

  Thus, side of square $= \frac{a}{\sqrt{6}}$

  $\therefore$ Area of square $= \frac{a^2}{6}$

\item Given, $AD = \frac{abc}{b^2 - c^2}$

  Also, $AD = b\sin 23^\circ \Rightarrow \frac{abc}{b^2 - c^2} = b\sin23^\circ$

  $\Rightarrow \frac{ac}{b^2 - c^2} = \sin23^\circ$

  $\Rightarrow \frac{\sin A\sin C}{\sin^2B - \sin^2C} = \sin23^\circ$

  $\Rightarrow \frac{\sin C\sin 23^\circ}{\sin(B + C)\sin(B - C)} = \sin23^\circ$

  $\Rightarrow \sin(B - 23^\circ) = 1 = \sin90^\circ$

  $\Rightarrow B = 113^\circ$

\item Given $a:b:c = 4:5:6 \Rightarrow a = 4k, b = 5k, c = 6k$ (let)

  $\frac{R}{r} = \frac{abc}{4\Delta}.\frac{s}{\Delta} = \frac{abc.\frac{a + b + c}{2}}{4.s(s - a)(s - b)(s - c)}$

  $= \frac{16}{7}$

\item Let $\angle BAD = \alpha, \angle CAD = \beta$

  Applying sine law in $\triangle ADB, \frac{BD}{\sin\alpha} = \frac{AD}{\sin B}$

  $\Rightarrow AD = \frac{BD}{\sin\alpha}.\frac{\sqrt{3}}{2}$

  Applying sine law in $\triangle ADC, \frac{CD}{\sin\beta} = \frac{AD}{\sin C}$

  $\Rightarrow AD = \frac{CD}{\sin\beta}.\frac{1}{\sqrt{2}}$

  $\Rightarrow \frac{BD}{CD}.\frac{\sqrt{3}}{\sqrt{2}} = \frac{\sin\alpha}{\sin\beta}$

  $\Rightarrow \frac{\sin\alpha}{\sin\beta} = \frac{1}{\sqrt{6}}$

\item Given, $3\sin x - 4\sin^2x - k = 0$

  $\sin3x = k$

  Since $A$ and $B$ satisfy the equations $\therefore \sin3A = \sin3B = k$

  $\sin3A - \sin3B = 0$

  $\therefore -2\sin\frac{3C}{2}\sin\frac{3(A - B)}{2} = 0$

  Since $A\neq B$ and also both $A$ and $B$ are less that $\pi/3[\because 0<k<1]$

  $\Rightarrow \sin\frac{3C}{2} = 0 \Rightarrow C = \frac{2\pi}{3}$

\item Since $A,B,C$ are in A.P. $\therefore 2B = A + C$

  $A + B + C = \pi \Rightarrow B = \pi/3$

  $\sin(2A + B) = \frac{1}{2} = \sin\frac{\pi}{6}$

  $\Rightarrow 2A + B = n\pi + (-1)^n\frac{\pi}{6}$

  $A = \frac{\pi}{4},\frac{11\pi}{12}$ these are the values between $0$ and $\pi.$

  But $\frac{11\pi}{12}$ is not possible as $B = \pi/3$

  $\therefore A = \pi/4$

\item Let $ABC$ be the triangle having right angle at $B$. From question, $AC = 2\sqrt{2}BD$

  Let $BD = x \therefore AC = 2\sqrt{2}x$ and $\angle C = \theta$

  $\tan C = \frac{BD}{CD} = \frac{x}{CD}\Rightarrow CD = x\cot\theta$

  $\tan(90^\circ - C) = \frac{BD}{AD} \therefore AD = x\tan\theta$

  $AD + CD = AC \Rightarrow \tan\theta + \cot\theta = 2\sqrt{2}$

  $\Rightarrow 2\sin\theta\cos\theta = \frac{1}{\sqrt{2}}$

  $\Rightarrow \sin2\theta = \sin\frac{\pi}{4}$

  $\Rightarrow \theta = \frac{\pi}{8}, \frac{3\pi}{8}$

  $\Rightarrow A = \frac{3\pi}{8}, \frac{\pi}{8}$

\item $P + Q + R = \pi$

  $\therefore P + Q = \pi/2 [\because R = \pi/2]$

  $\Rightarrow  \tan\left(\frac{P + Q}{2}\right) = 1$

  $\Rightarrow \tan\frac{P}{2} + \tan\frac{Q}{2} = 1 - \tan\frac{P}{2}\tan\frac{Q}{2}$

  Since $\tan\frac{P}{2}$ and $\tan\frac{Q}{2}$ are roots of the equation $ax^2 + bx + c = 0$

  $\Rightarrow \tan\frac{P}{2} + \tan\frac{Q}{2} = -\frac{b}{a}, \tan\frac{P}{2}\tan\frac{Q}{2} = \frac{c}{2}$

  $\Rightarrow -\frac{b}{a} = 1 - \frac{c}{a}$

  $\Rightarrow a + b = c$

\item Let $AD$ be the median such that $\angle BAD = 30^\circ, \angle DAC = 45^\circ$ and $\angle ADC = \theta$

  Applying $mn$ theorem, we get

  $2\cot\theta = \cot30^\circ - \cot45^\circ$

  $\Rightarrow \tan\theta = \sqrt{3} + 1$

  $\sin C = \frac{\sqrt{3} + 2}{\sqrt{2}\sqrt{5 + 2\sqrt{3}}}$

  Applying sine law in $\triangle ADC,$ we get

  $\frac{AD}{\sin C} = \frac{DC}{\sin45^\circ}$

  $\Rightarrow DC = 1$

  $\Rightarrow DC = BD = 1 \Rightarrow BC = 2$

\item We know that in a triangle $\tan A + \tan B + \tan C = \tan A + \tan B + \tan C$

  Also, since A.M $\geq$ G.M.

  $\Rightarrow \frac{\tan A + \tan B + \tan C}{3}\geq \sqrt[3]{\tan A\tan B\tan C}$

  $\Rightarrow \tan A\tan B\tan C\geq 3\sqrt[3]{\tan A\tan B\tan C}$

  $\Rightarrow \tan^2A\tan^2B\tan^2C\geq 27$

  $\Rightarrow \tan A + \tan B + \tan C\geq 3\sqrt{3}$

\item Given, $\cos\frac{A}{2} = \frac{1}{2}\sqrt{\frac{b}{c} + \frac{c}{b}}$

  $\sqrt{\frac{s(s - a)}{bc}} = \frac{1}{2}\sqrt{\frac{b}{c} + \frac{c}{b}}$

  $\frac{(a + b + c)(b + c - a)}{4bc} = \frac{b^2 + c^2}{4bc}$

  $\Rightarrow a^2 = 2bc$

  Thus, square described on side $a$ is twice the rectangle contained by two other sides.

\item Given, $\cos\theta = \frac{a - b}{c} \Rightarrow \sin\theta = \frac{1}{c}\sqrt{c^2 - (a - b)^2}$

  $\frac{(a + b)\sin\theta}{2\sqrt{ab}} = \frac{(a + b)\sqrt{c^2 - (a - b)^2}}{2c\sqrt{ab}}$

  $\frac{(a + b)\sqrt{2ab(1 - \cos C)}}{2c\sqrt{ab}} = \frac{a + b}{\sqrt{2}c}.\sqrt{2}\sin\frac{C}{2}$

  $= \frac{\sin A + \sin B}{\sin C}.\sin\frac{C}{2}$

  $= \cos\frac{A - B}{2}$

  $\frac{c\sin\theta}{2\sqrt{ab}} = \frac{c\sqrt{c^2 - (a - b)^2}}{2\sqrt{ab}}$

  $= \frac{c\sqrt{2ab(1 - \cos C)}}{2\sqrt{ab}} = \sin \frac{C}{2} = \cos \frac{A + B}{2}$

\item We have proven earlier that distance between circumcenter and incenter is $\sqrt{R^2 - 2rR}$

  Clearly, $\sqrt{R^2 - 2rR}\geq 0$

  $\Rightarrow R\geq 2r$

\item Given, $\tan B = \cos A$

  $\Rightarrow \frac{\sqrt{1 - \cos^2B}}{\cos B} = \cos A$

  $\Rightarrow \cos^2B = \frac{1}{2 - \sin^2A}$

  Also given that $\cos C = \tan A$

  $\Rightarrow \tan^2 C = \frac{1 - 2\sin^2A}{\sin^2A}$

  Now given that $\cos B = \tan C$

  $\Rightarrow \frac{1}{2 - \sin^2A} = \frac{1 - 2\sin^2A}{\sin^2A}$

  $\Rightarrow \sin A = \pm\sqrt{\frac{3\pm\sqrt{5}}{2}}$

  Same value will be obtained for $\sin B$ and $\sin C$ and that is equal to $2\sin 18^\circ$

\item $A + B + C = 180^\circ$

  $\Rightarrow \cos(A + B) = -\cot C$

  $\Rightarrow \cot A\cot B + \cot B\cot C + \cot C\cot A = 1$

  Now $\cot^2A + \cot^2B + \cot^2C - 1 = \cot^2A + \cot^2B + \cot^2C - (\cot A\cot B + \cot B\cot C + \cot C\cot A)$

  $= \frac{1}{2}[(\cot A - \cot B)^2 + (\cot B - \cot C)^2 + (\cot C - \cot A)^2] \geq 0$

  $\Rightarrow \cot^2A + \cot^2B + \cot^2C \geq 1$

\item We have proven in 229 that $\tan A + \tan B + \tan C \geq 3\sqrt{3}$

  We apply A.M. $\geq$ G.M. again on $\tan^2A, \tan^2B, \tan^2C$

  $\frac{\tan^2A + \tan^2B + \tan^2C}{3}\geq (\tan^2A\tan^B\tan^2C)^{1/3}$

  $\Rightarrow \tan^2A + \tan^2B + \tan^C \geq 9$

\item We know that in a $\triangle ABC, \sin\frac{A}{2} + \sin\frac{B}{2} + \sin\frac{C}{2} \leq \frac{3}{2}$

  Now we know that A.M. $\geq$ H.M.

  $\frac{\csc\frac{A}{2} + \csc\frac{B}{2} + \csc\frac{C}{2}}{3}\geq \frac{3}{\sin\frac{A}{2} + \sin\frac{B}{2} + \sin\frac{C}{2}}$

  $\Rightarrow \csc\frac{A}{2} + \csc\frac{B}{2} + \csc\frac{C}{2} \geq 6$

\item $\cos A + \cos B + \cos C -\frac{3}{2} = 2\cos\frac{A + B}{2}\cos\frac{A - B}{2} + 1 - 2\sin^2\frac{C}{2} - \frac{3}{2}$

  $\Rightarrow  -2\sin^2\frac{C}{2} + 2\cos\frac{A - B}{2}\sin\frac{C}{2} - \frac{1}{2} = 0$

  Clearly $D = 4\cos^2\frac{A - B}{2} - 4 < 0$ so sign of quadratic equation will be same as that of first term.

  $\Rightarrow \cos A + \cos B + \cos C \leq \frac{3}{2}$

  Now $\cos A + \cos B + \cos C - 1 = 2\sin\frac{C}{2}\cos\frac{A - B}{2} - 2\sin^2\frac{C}{2}$

  $= 2\sin\frac{C}{2}\left[\cos\frac{A - B}{2} - \cos\frac{A + B}{2}\right]$

  $= 4\sin\frac{A}{2}\sin\frac{B}{2}\sin\frac{C}{2} > 0$

  $\therefore \cos A + \cos B + \cos C > 1$

\item $y=2\cos A\cos B\cos C=[\cos(A-B)+\cos(A+B)]\cos C=[\cos(A-B)-\cos C]\cos C$

  $\implies\cos^2C-\cos(A-B)\cos C+y=0$

  which is quadratic equation in $\cos C$ which would be real.

  $\Rightarrow D \geq 0 \implies \cos^2(A-B)-4y\ge0\iff y\le\dfrac{\cos^2(A-B)}4\le\dfrac14$

  Hence proved.

\item Let $A$ be the center of first circle having radius $a$ and $B$ be the center of second circle having radius $b.$

  Let the two circles intersect at $C$ and $D$ so $\angle ACB = \angle ADB = \theta$

  We know that perpendicular from the center of a circle divides the chord in two equal parts. So $AB$ will divide $CD$ in two equal parts.

  Let $CD = 2x.$ Let $AB$ cut $CD$ at $O$ then $OC = OD = x$

  $AB = OA + OB = \sqrt{a^2 - x^2} + \sqrt{b^2 - x^2}$

  Clearly, $\angle ACB = \theta$

  Applying cosine law in $\triangle ABC$

  $AB^2 = AC^2 + BC^2 - 2.AC.BC\cos(\pi - \theta)$

  $\Rightarrow a^2 - x^2 + b^2 - x^2 + 2\sqrt{a^2 - x^2}\sqrt{b^2 - x^2} = a^2 + b^2 + 2ab\cos\theta$

  $\Rightarrow CD = 2x = \frac{2ab\sin\theta}{\sqrt{a^2 + b^2 + 2ab\cos\theta}}$

\item The diagram is given below:

  \startplacefigure
    \externalfigure[22_1.pdf]
  \stopplacefigure

  Let the triangle be $ABC$ which will be equilateral triangle having radius $2$ if all the circles are unit
  circle(i.e. having radius $1$ ).

  There will be two circles which will touch all three circles. One is the smaller one which will be inside the area enclosed by
  three circles and the other will be one which will enclose all the circles.

  Clearly, the center of the two circles will bisect the angles of the triangle.

  $\therefore \cos30^\circ = \frac{1}{x + 1}$ where $x$ is the radius of inner circle.

  $\implies x = \frac{2 - \sqrt{3}}{\sqrt{3}}$

  Radius of outer circle $= 2 + x = \frac{2 + \sqrt{3}}{\sqrt{3}}$

\item We have to prove that $\sum_{r=0}^n{}^nC_ra^rb^{n - r}\cos[rB - (n - r)A] = C^n$

  From De Movire's theorem(this we have not studied yet),

  L.H.S. $= \sum_{r = 0}^n(ae^{iB})^r(b.e^{-iA})^r$

  $= (ae^iB + be^{-iA})^n = (a\cos B + ia\sin B + b\cos A - ib\sin A)$

  $\left[\because \frac{a}{\sin A} = \frac{b}{\sin B}\right]$

  $= (a\cos B + b\cos A)^n = c^n$

\item Let $A = B \implies A + B + C = \pi \implies 2A + C = \pi$

  Given, $\tan A +\tan B + \tan C = k$

  $\implies 2\tan A + \tan(\pi - 2A) = k$

  $\implies 2\tan A\left(1 + \frac{1}{1 - \tan^2A}\right) = k$

  $\implies 2\tan A\frac{\tan^2A}{1 - \tan^2A} = k$

  $\implies \frac{2\tan^3A}{1 - \tan^2A} = k$

  Now, since it is an isoscles triangle $A < \pi/2$

  So there are three possibilities, $0 < A < \pi/4, A=\pi/4, \pi/4 < A < \pi/2$

  In first case $k < 0,$ in second case $k=\infty$ and in third case $k > 0$

  Whenever $k<0$ or $k>0$ multiple isosceles triangles are possible. For example, for $k < 0$
  i.e. for $0 < A < \pi/4, A$ can assume values like $\pi/6, \pi/5, \pi/7$ and so on.

  Similarly, for $k > 0, A$ can assume multiple values.

  However, if $k = \infty$ then $A, B$ can have only one value i.e. $\pi/4$ and only one isoceles triangle is possible.

\item We have to prove that $\Delta \leq \frac{s^2}{3\sqrt{3}}$

  i.e. $s(s - a)(s - b)(s - c) \leq \frac{s^4}{27}$

  $\implies \frac{(s - a)(s - b)(s - c)}{s^3}\leq 27$

  We know that A.M. $\geq$ G.M.

  $\implies \frac{s - a + s - b + s - c}{3s}\geq \sqrt[3]{\frac{(s - a)(s - b)(s - c)}{s^3}}$

  $\implies \frac{1}{3}\geq \sqrt[3]{\frac{(s - a)(s - b)(s - c)}{s^3}}$

  Cubing we get,

  $\frac{(s - a)(s - b)(s - c)}{s^3}\leq 27$

  Hence proved.

\item Let $a=3x+4y,b=4x+3y$ and $c=5x+5y$ be the largest side.

  $\cos C = \frac{a^2 + b^2 - c^2}{2ab} = \frac{-2xy}{2(3x + 4y)(4x + 3y)} < 0~\forall x, y >0$

  Thus, it is an obstuse angled triangle.

\item $\Delta = \frac{1}{2}ah_1 = \frac{1}{2}bh_2 = \frac{1}{2}ch_3$

  $\implies h_1 = \frac{2\Delta}{a}, h_2 = \frac{2\Delta}{b}, h_3 = \frac{2\Delta}{h_3}$

  We know that $r = \frac{\Delta}{s}$

  Now it is trivial to show the required condition.

\item $\Delta_0$ can be evaluated to be $\frac{1}{2}\frac{rS}{R}$

  Likewise $\Delta_1 = \frac{1}{2}\frac{r_1S}{R}$ and so on.

  Now it is trivial to prove the required equality.
\stopitemize
