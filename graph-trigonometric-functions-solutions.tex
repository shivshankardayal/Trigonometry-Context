% -*- mode: context; -*-
\chapter{Graph of Trigonometric Functions}
\startitemize[n, 1*broad]
\item  The plot of $\sin x$ \in{Figure}[fig:34_1] is shown below:

  \startplacefigure[title={Plot of $\sin x$}, reference=fig:34_1, force]
    \externalfigure[34_1.pdf]
  \stopplacefigure

\item The plot of $\cos x$ \in{Figure}[fig:34_2]  is given below:

  \startplacefigure[title={Plot of $\cos x$}, reference=fig:34_2, force]
    \externalfigure[34_2.pdf]
  \stopplacefigure

\item The plot of $\tan x$ \in{Figure}[fig:34_3] is given below:

  \startplacefigure[title={Plot of $\tan x$}, reference=fig:34_3, force]
    \externalfigure[34_3.pdf]
  \stopplacefigure

\item The plot of $\cot x$ \in{Figure}[fig:34_4] is given below:

  \startplacefigure[title={Plot of $\cot x$}, reference=fig:34_4, force]
    \externalfigure[34_4.pdf]
  \stopplacefigure

\item The plot of $\sec x$ \in{Figure}[fig:34_5] is given below:

  \startplacefigure[title={Plot of $\sec x$}, reference=fig:34_5]
    \externalfigure[34_5.pdf]
  \stopplacefigure

\item The plot of $\csc x$ \in{Figure}[fig:34_6] is given below:

  \startplacefigure[title={Plot of $\csc x$}, reference=fig:34_6]
    \externalfigure[34_6.pdf]
  \stopplacefigure

\item The plot of $\sin x + \cos x$ \in{Figure}[fig:34_7] is given below:

  \startplacefigure[title={Plot of $\sin x + \cos x$}, reference=fig:34_7]
    \externalfigure[34_7.pdf]
  \stopplacefigure

\item The plot of $x + \cos x$ \in{Figure}[fig:34_8] is given below:

  \startplacefigure[title={Plot of $x + \cos x$}, reference=fig:34_8]
    \externalfigure[34_8.pdf]
  \stopplacefigure

\item The plot of $2\sin 2x$ \in{Figure}[fig:34_9] is given below:

  \startplacefigure[title={Plot of $2\sin2x$}, reference=fig:34_9]
    \externalfigure[34_9.pdf]
  \stopplacefigure

\item $y = a^x, a > 0$ will have two different plots \in{Figure}[fig:34_10] and \in{Figure}[fig:34_10_1], first plot is for $a > 1$ and second plot is
    for $0 < a < 1$.

  \startplacefigure[title={Plot of $a^x$, $a> 1$}, reference=fig:34_10]
    \externalfigure[34_10.pdf]
  \stopplacefigure
  \startplacefigure[title={Plot of $a^x$, $0<a<1$}, reference=fig:34_10_1]
    \externalfigure[34_10_1.pdf]
  \stopplacefigure

\item The plot of $e^x$ \in{Figure}[fig:34_11] is given below:

  \startplacefigure[title={Plot of $e^x$}, reference=fig:34_11]
    \externalfigure[34_11.pdf]
  \stopplacefigure

\item The plot of $\log_ex$ \in{Figure}[fig:34_12] is given below:

  \startplacefigure[title={Plot of $\log_ex$}, reference=fig:34_12]
    \externalfigure[34_12.pdf]
  \stopplacefigure
\item The plot of $\sin2x$ \in{Figure}[fig:34_13] is given below:

  \startplacefigure[title={Plot of $\sin2x$}, reference=fig:34_13]
    \externalfigure[34_13.pdf]
  \stopplacefigure

\item The plot of $\cos x - \sin x$ \in{Figure}[fig:34_14] is given below:

  \startplacefigure[title={Plot of $\cos x - \sin x$}, reference=fig:34_14]
    \externalfigure[34_14.pdf]
  \stopplacefigure
\item The plot of $|\sin x|$ \in{Figure}[fig:34_15] is given below:

  \startplacefigure[title={Plot of $|\sin x|$}, reference=fig:34_15]
    \externalfigure[34_15.pdf]
  \stopplacefigure
\item The plot of $|\cos x|$ \in{Figure}[fig:34_16] is given below:

  \startplacefigure[title={Plot of $|\cos x|$}, reference=fig:34_16]
    \externalfigure[34_16.pdf]
  \stopplacefigure
\item The plot of $|\tan x|$ \in{Figure}[fig:34_17] is given below:

  \startplacefigure[title={Plot of $|\tan x|$}, reference=fig:34_17]
    \externalfigure[34_17.pdf]
  \stopplacefigure
\item The plot of $|\cot x|$ \in{Figure}[fig:34_18] is given below:

  \startplacefigure[title={Plot of $|\cot x|$}, reference=fig:34_18]
    \externalfigure[34_18.pdf]
  \stopplacefigure
\item The plot of $|\sec x|$ \in{Figure}[fig:34_19] is given below:

  \startplacefigure[title={Plot of $|\sec x|$}, reference=fig:34_19]
    \externalfigure[34_19.pdf]
  \stopplacefigure
\item The plot of $|\csc x|$ \in{Figure}[fig:34_20] is given below:

  \startplacefigure[title={Plot of $|\csc x|$}, reference=fig:34_20]
    \externalfigure[34_20.pdf]
  \stopplacefigure
\item We have to find number of solutions for $\tan x = x + 1$ for $-\frac{\pi}{2}\leq x\leq
  2\pi$. So we plot both $y = \tan x$ and $y = x + 1$ and no. of  intersections will be no. of
  solutions. The plot \in{Figure}[fig:34_21] is given below:

  \startplacefigure[title={Plot of $\tan x$ and $x + 1$}, reference=fig:34_21]
    \externalfigure[34_21.pdf]
  \stopplacefigure

  As we can see that there are two points of intersections so there will be two solutions of the given
  equation in the given range of $x$.

\item Given equation is $x + 2\tan x = \frac{\pi}{2} \Rightarrow \tan x = \frac{\pi}{4} -
  \frac{x}{2}$. So we plot for $y = \tan x$ and $y = \frac{\pi}{4} - \frac{x}{2}$ in the range
  of $[0, 2\pi]$. The plot \in{Figure}[fig:34_22] is given below:

  \startplacefigure[title={Plot of $\tan x$ and $\frac{\pi}{4} - \frac{x}{2}$}, reference=fig:34_22]
    \externalfigure[34_22.pdf]
  \stopplacefigure

    As we can see that there are three points of intersections so there will be three solutions of the given
    equation in the given range of $x$.

\item Given equation is $\sin x = \frac{x}{100}$. Let $y = \sin x = \frac{x}{100}$. When $x
  = 0, y = 0$ and when $x = 1, y = 0.01$.

  $\because -1\leq \sin x\leq 1 \Rightarrow -1\leq \frac{x}{100}\leq 1 \Rightarrow -100\leq x\leq
  100$

  $\Rightarrow -31.8\pi\leq x\leq 31.8x$ (approx.). Hence, the interval for $x$ will be
  between $-31.8\pi$ to $31.8\pi$. The plot \in{Figure}[fig:34_23] is given below:

  \startplacefigure[title={Plot of $\sin x$ and $\frac{x}{100}$}, reference=fig:34_23]
    \externalfigure[34_23.pdf]
  \stopplacefigure

  By looking at figure we can deduce that total no. of solutions would be $63$. $31$ of these
  will be for $x < 0, 31$ for $x > 0$ and one solution for $x = 0$.

\item We have to find no. of solutions for $e^x = x^2$ so we plot $y = e^x$ and $y = x^2$.
  The plot \in{Figure}[fig:34_24] is given below:

  \startplacefigure[title={Plot of $e^x$ and $x^2$}, reference=fig:34_24]
    \externalfigure[34_24.pdf]
  \stopplacefigure

  By looking at the graph it is clear that we will have only one solution for $x < 0$.

\item We have to find no. of solutions for $\log_{10}x = \sqrt{x}$ so we plot $y = \log_{10}x$ and
  $y = \sqrt{x}$. The plot \in{Figure}[fig:34_25] is given below:

  \startplacefigure[title={Plot of $\log x$ and $\sqrt{x}$}, reference=fig:34_25]
    \externalfigure[34_25.pdf]
  \stopplacefigure

  By looking at the graph it is clear that we will have no solution for $x > 0$.

\item Given equation is $\tan x - x = \frac{1}{2} \Rightarrow \tan x = x + \frac{1}{2}$. So we plot for
  $y = \tan x$ and $y = x + \frac{1}{2}$. The plot \in{Figure}[fig:34_26] is given below:

  \startplacefigure[title={Plot of $\tan x$ and $x + \frac{1}{2}$}, reference=fig:34_26]
    \externalfigure[34_26.pdf]
  \stopplacefigure

  By looking at the graph we can deduce that there is one solution for $x$ between $\pi/4$ and
  $\pi/2$.

\item Given below is the plot of $y = x + \cos x$ for $0\leq x\leq 2\pi$. The plot \in{Figure}[fig:34_27] is given below:

  \startplacefigure[title={Plot of $x + \cos x$}, reference=fig:34_27]
    \externalfigure[34_27.pdf]
  \stopplacefigure

\item Given below is the graph of $y = \sin \left(3x + \frac{\pi}{4}\right)$.  The plot \in{Figure}[fig:34_28] is given below:

  \startplacefigure[title={Plot of $\sin\left(3x + \frac{\pi}{4}\right)$}, reference=fig:34_28]
    \externalfigure[34_28.pdf]
  \stopplacefigure

\item Given below is the graph of $y = \tan\frac{x}{2}$. The plot \in{Figure}[fig:34_29] is given below:

  \startplacefigure[title={Plot of $\tan\frac{x}{2}$}, reference=fig:34_29]
    \externalfigure[34_29.pdf]
  \stopplacefigure

\item Given below is the graph of $y = \frac{1}{\sqrt{2}}(\sin x + \cos x)$.  The plot \in{Figure}[fig:34_31] is given below:

  \startplacefigure[title={Plot of $\frac{1}{\sqrt{2}}(\sin x + \cos x)$}, reference=fig:34_31]
    \externalfigure[34_31.pdf]
  \stopplacefigure

\item  We plot both $y = x$ and $y = \cos x$ as shown below(\in{Figure}[fig:34_32]):

  \startplacefigure[title={Plot of $\cos x$ and $x$}, reference=fig:34_32]
    \externalfigure[34_32.pdf]
  \stopplacefigure
  As we see that there is one point of intersection between $y = \cos x$ and $y = x$ so we
  conclude that there is one solution for $x = \cos x$ for $0\leq x\leq\frac{\pi}{2}$.

\item We plot both $y = \sin x$ and $y = \cos x$ as shown below(\in{Figure}[fig:34_33]):

  \startplacefigure[title={Plot of $\cos x$ and $\sin x$}, reference=fig:34_33]
    \externalfigure[34_33.pdf]
  \stopplacefigure

  As we see that there is one point of intersection between $y = \cos x$ and $y = \sin x$ so we
  conclude that there is one solution for $\sin x = \cos x$ for $0\leq x\leq\frac{\pi}{2}$.

\item  We plot both $y = \tan x$ and $y = x$ as shown below(\in{Figure}[fig:34_34]):

  \startplacefigure[title={Plot of $\tan x$ and $x$}, reference=fig:34_34]
    \externalfigure[34_34.pdf]
  \stopplacefigure

  As we see that there is one point of intersection between $y = \tan x$ and $y = x$ so we
  conclude that there is one solution for $x = \tan x$ for $0\leq x\leq\frac{\pi}{2}$.

\item We plot both $y = \tan x$ and $y = 1$ as shown below(\in{Figure}[fig:34_35]):

  \startplacefigure[title={Plot of $\tan x$ and $y = 1$}, reference=fig:34_35]
    \externalfigure[34_35.pdf]
  \stopplacefigure
  As we see that there is one point of intersection between $y = \tan x$ and $y = 1$ so we
  conclude that there is one solution for $1 = \tan x$ for $0\leq x\leq\frac{\pi}{2}$.

\item We plot both $y = \sin^2x$ and $y = \cos x$ as shown below(\in{Figure}[fig:34_36]):

  \startplacefigure[title={Plot of $\sin^2 x$ and $\cos x$}, reference=fig:34_36]
    \externalfigure[34_36.pdf]
  \stopplacefigure

  As we see that there is one point of intersection between $y = \cos x$ and $y = \sin^2x$ so we
  conclude that there is one solution for $\sin^2x = \cos x$ for $0\leq x\leq\frac{\pi}{2}$.

\item This problem has same equation as 21 just the range is different so it can be solved with a
  similar graph.

38. $y = |x - 1|$ implies $y = x - 1$ when $x \geq 1$ ad $y = 1 - x$ when $x <
    1$. So we plot the two lines and the curve $y = \sqrt{5 - x^2}$.

  \startplacefigure[reference=fig:34_38]
    \externalfigure[34_38.pdf]
  \stopplacefigure
\stopitemize
