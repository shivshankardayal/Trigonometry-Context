% -*- mode: context; -*-
\chapter{De Moivre's Theorem}

Throughout the development of mathematics, one recurring theme has been the search for elegant structures
that unify seemingly different ideas. From the earliest explorations of geometry and number theory to the
modern study of analysis, mathematicians have continually sought ways to simplify complex operations and
reveal the hidden order behind them. This desire for connection is especially evident in the evolution of
the complex number system—a framework that began as a formal solution to equations lacking real roots, yet
eventually became one of the most powerful tools in both pure and applied mathematics.

Complex numbers extend the real number line into a plane, transforming arithmetic into a geometric
language. Here, addition corresponds to translation, multiplication corresponds to a combined rotation and
dilation, and each number carries not only a magnitude but also a direction. This geometric interpretation
opened new pathways for understanding algebraic operations, enabling mathematicians to visualize
transformations that once appeared abstract. The interplay between these different viewpoints—algebraic,
geometric, and trigonometric—revealed deep structural relationships that unify broad areas of mathematics.

As complex numbers became more widely studied, it became clear that expressing them in polar form, using a
magnitude and an angle, provides exceptional clarity when dealing with multiplication and
exponentiation. Instead of expanding polynomials or dealing with unwieldy rectangular coordinates, polar
form offers a concise way to describe how complex numbers stretch and rotate the plane. This shift in
perspective set the stage for one of the most elegant connections between trigonometry and complex algebra.

It is in this context that {\bf De Moivre's Theorem} emerges as a central and beautiful result. By showing how to
raise complex numbers in trigonometric form to integer powers, the theorem creates a direct bridge between
algebraic exponentiation and angular rotation. It reveals that the behavior of powers of complex numbers
follows a surprisingly simple pattern: magnitudes grow exponentially, while angles increase linearly. This
insight not only simplifies computations that would otherwise be cumbersome, but also leads to powerful
methods for deriving trigonometric identities and finding complex roots.

Thus, {\bf De Moivre's Theorem} stands as a remarkable example of mathematical unity, where ideas from
different domains converge into a single, elegant statement that captures the essence of complex
exponentiation. The journey from algebraic equations to geometric rotations culminates in this theorem,
highlighting the profound ways in which mathematics weaves diverse concepts into coherent and beautiful
structures.

\starttheorem
If $n$ be a positive or negative number, then $(\cos\theta + i\sin\theta)^n = \cos n\theta + i\sin n\theta$
and if $n$ is a positive or negative rational(not an integer) then one of the values of $(\cos\theta +
i\sin\theta)^n$ is $\cos n\theta + i\sin n\theta$.
\stoptheorem

\startproof
{\bf Case I:} When $n$ is a positive integer.

$\left(\cos\theta_1 + i\sin\theta_1\right)\left(\cos\theta_2 + i\sin\theta_2\right) = \cos\theta_1\cos\theta_2 -
\sin\theta_1\sin\theta_2 + i\left(\cos\theta_1\sin\theta_2 + \cos\theta_2\sin\theta_1\right)$

$= \cos\left(\theta_1 + \theta_2\right) + i\sin\left(\theta_1 + \theta_2\right)$

Similarly, $\left(\cos\theta_1 + i\sin\theta_1\right)\left(\cos\theta_2 +
  i\sin\theta_2\right)\left(\cos\theta_3 + i\sin\theta_3\right) = \cos\left(\theta_1 + \theta_2 +
  \theta_3\right) + i\sin\left(\theta_1 + \theta_2 + \theta_3\right)$

Extending this we can say that $(\cos\theta + i\sin\theta)(\cos\theta + i\sin\theta) \cdots n$ times $= \cos
n\theta + i\sin n\theta$.

{\bf Case II:} When $n$ is a negative integer.

$(\cos\theta + i\sin\theta)^{-n} = \frac{1}{\cos n\theta - i\sin n\theta} = \frac{1}{\cos n\theta - i\sin
  n\theta}.\frac{\cos n\theta + i\sin n\theta}{\cos n\theta + i\sin n\theta}$

$= \cos n\theta - i\sin n\theta$

Similarly, we can prove it for a fraction.
\stopproof

\section{Applications of De Moivre's Theorem}
\startitemize[n]
\item Simplifying Powers of Complex Numbers

  One of the most direct and practical uses of De Moivre’s theorem is turning the potentially messy process
  of expanding $(\cos\theta + i\sin\theta)^n$ into a simple operation. Instead of using the binomial theorem
  and carefully separating real and imaginary parts, we can compute: $(\cos\theta + i\sin\theta)^n = \cos
  n\theta + i\sin n\theta$. This dramatically simplifies tasks in complex algebra, especially when dealing
  with high powers.
\item Deriving Trigonometric Identities

  De Moivre's theorem provides a systematic way to create multiple-angle identities such as:

  \startitemize[i]
\item $\cos2\theta.\sin2\theta$
\item $\cos3\theta.\sin3\theta$
\item General $\cos n\theta$ and $\sin n\theta$ formulas
  \stopitemize
  By expanding $(\cos⁡\theta+i\sin⁡\theta)^n$ using the binomial theorem and equating real and imaginary
  parts, one directly obtains identities like:

  $$\cos3\theta = 4\cos^3\theta - 3\cos\theta.$$

  This is a very clean pathway to advanced trigonometric formulas.
\item Computing Complex Roots

  The theorem leads naturally to a formula for the n-th roots of complex numbers. If

  $$z = r(\cos\theta + i\sin\theta),$$

  then the $n$th roots are

  $$z_k = r^{1/n}\left(\cos\frac{\theta + 2k\pi}{n} + \sin \frac{\theta + 2k\pi}{n}\right), k = 0, 1,
  \ldots, n - 1.$$

  This result implies:

  \startitemize[i]
\item The roots lie evenly spaced on a circle.
\item They form a regular $n$-gon in the complex plane.
  \stopitemize

  This is essential in fields such as algebra, number theory, and signal processing.
\item Solving Polynomial Equations

  De Moivre's method of extracting roots allows one to solve many polynomials conveniently, especially when
  roots occur in conjugate or symmetric patterns. For example, the roots of equations like:

  $$z^n = 1$$

  are directly found using the root formula derived from De Moivre’s theorem. These roots of unity play
  foundational roles in:

  \startitemize[i]
\item Fourier analysis
\item Group theory
\item Number theory
\item Discrete mathematics
  \stopitemize
\item Fourier Series and Harmonic Analysis

  The entire structure of Fourier analysis relies on the relationship between complex exponentials and
  trigonometric functions. De Moivre’s theorem is one of the first bridges showing that powers and
  combinations of trigonometric functions behave predictably and periodically.

  It underlies the idea that signals can be decomposed into sinusoidal components, which is the basis of:

  \startitemize[i]
\item Audio compression (MP3)
\item Image compression (JPEG)
\item Signal filtering
\item Vibrational analysis
  \stopitemize
\item Electrical Engineering and AC Circuit Analysis

  AC circuits use phasors, which represent sinusoidal voltages and currents as complex numbers. Multiplying
  phasors, an essential operation when analyzing networks, relies on De Moivre’s theorem:

  \startitemize[i]
\item Powers change magnitude and rotate the angle.
\item Roots help find phase relationships and resonant frequencies.
  \stopitemize

  It allows simplification of impedance calculations, modeling phase shifts, and designing filters and
  oscillators.
\item Rotations and Transformations in Geometry

  Because multiplication of complex numbers corresponds to rotation plus scaling, De Moivre’s theorem
  provides a clear way to compute repeated rotations, symmetry transformations, and spiral patterns. It
  helps generate and analyze geometric figures that arise from repeated angular transformations.
\item Fractals and Complex Dynamics

  Iterated functions like $z\mapsto z^n + c$, used in Mandelbrot and Julia sets, rely on understanding how
  powers of complex numbers behave.

  De Moivre’s theorem reveals:

  \startitemize[i]
\item How magnitudes grow under iteration
\item How angles wrap or expand
\item How periodic points behave
  \stopitemize

  This insight is used in chaos theory, visualization, and mathematical art.
\item Quantum Mechanics and Wave Functions

  Complex exponentials and their trigonometric forms are essential in quantum physics. While Euler’s formula
  is the direct tool, De Moivre’s theorem supports:

  \startitemize[i]
\item Expansion of wave functions
\item Analysis of periodic potentials
\item Manipulation of complex amplitudes
  \stopitemize

  It helps link angular frequencies (found in wave mechanics) to complex powers.
\stopitemize

\section{Problems}
\startitemize[n, 1*broad]
%1
\item Prove that $\left(\frac{\cos\theta + i\sin\theta}{\sin\theta + i\cos\theta}\right)^4 = \cos8\theta +
  i\sin8\theta$.
  %2
\item Prove that $\frac{(\cos\theta + i\sin\theta)^4}{(\sin\theta + i\cos\theta)^5} = \sin9\theta -
  i\cos9\theta$.
  %3
\item Simplify $\frac{(\cos3\theta + i\sin3\theta)^5(\cos\theta - i\sin\theta)^3}{(\cos5\theta +
  i\sin5\theta)^7(\cos2\theta - i\sin2\theta)^5}$.
  %4
\item Simplify $[(\sin15^\circ - \cos15^\circ) + i(\sin75^\circ - \cos75^\circ)]^n + [(\sin15^\circ -
  \cos15^\circ) - i(\sin75^\circ - \cos75^\circ)]^n$.\
  %5
\item Prove that $\left(\frac{1 + \cos\theta + i\sin\theta}{1 + \cos\theta - i\sin\theta}\right)^n =
  \cos n\theta + i\sin n\theta$.
  %6
\item Prove that $(1 + \cos\theta + i\sin\theta)^n(1 + \cos\theta - i\sin\theta)^n = 2^{n +
  1}\cos^n\frac{\theta}{2}\cos\frac{n\theta}{2}$.
  %7
\item If $x_r = \cos\frac{\pi}{2^r} + i\sin\frac{\pi}{2^r}$, prove that $x_1x_2x_3\ldots \infty = -1$.
  %8
\item If $x = \cos2\alpha + i\sin2\alpha, y = \cos2\beta + i\sin2\beta$ and $z = \cos2\gamma + i\sin2\gamma$
  then prove that $\sqrt{xyz} + \frac{1}{\sqrt{xyz}} = 2\cos(\alpha + \beta + \gamma)$.
  %9
\item Simplify $\frac{(\cos\theta + i\sin\theta)^5(\cos2\theta - i\sin2\theta)^2}{(\cos3\theta +
  i\sin3\theta)^{11}(\cos4\theta - i\sin4\theta)^8}$.
  %10
\item Prove that $[(\cos\theta\cos\phi) + i(\sin\theta\sin\phi)]^n + [(\cos\theta\cos\phi) -
  i(\sin\theta\sin\phi)]^n = 2^{n + 1}\cos^n\frac{\theta - \phi}{2}\cos n\frac{\theta + \phi}{2}$.
  %11
\item Prove that $\left(\frac{1 + \sin\phi + i\cos\phi}{1 + \sin\phi - i\cos\phi}\right)^n = (\sin\phi +
  i\cos\phi)^n$.
  %12
\item If $\theta = 15^\circ$, find the value of $\frac{(\cos\theta + i\sin\theta)(\cos2\theta +
  i\sin\theta)}{(\cos3\theta - i\sin3\theta)}$.
  %13
\item Express $\left(\frac{\cos\theta + i\sin\theta}{\sin\theta + i\cos\theta}\right)^5$ in the form of
  $\cos x + i\sin x$.
  %14
\item Prove that $[(\cos\theta - \cos\phi) + i(\sin\theta - \sin\phi)]^n + [(\cos\theta - \cos\phi) -
  i(\sin\theta - \sin\phi)]^n = 2^{n + 1}\sin^n\frac{\theta - \phi}{2}\cos n\left(\frac{\pi + \theta -
  \phi}{2}\right)$.
  %15
\item If $\left(a_1 + ib_1\right)\left(a_2 + ib_1\right)\cdots\left(a_n + ib_n\right) = A + iB$ then prove
  that $\left(a_1^2 + b_1^2\right)\left(a_2^2 + b_2^2\right)\cdots\left(a_n^2 + b_n^2\right) = A^2 + B^2$
  and $\tan^{-1}\frac{b_1}{a_1} + \tan^{-1}\frac{b_2}{a_2} + \cdots + \tan^{-1}\frac{b_n}{a_n} =
  \tan^{-1}\frac{B}{A}$.
  %16
\item Show that $(a + ib)^{m/n} + (a - ib)^{m/n} = 2\left(a^2 +
  b^2\right)^{m/2n}\cos\left(\frac{m}{n}\tan^{-1}\frac{b}{a}\right)$.
  %17
\item Prove that $\frac{(1 + \sin\phi + i\cos\phi)^n}{(1 = \sin\phi - i\cos\phi)^n} =
  \cos\left(\frac{n\pi}{2} - n\phi\right) + i\sin\left(\frac{n\pi}{2} - n\phi\right)$.
  %18
\item If $(1 + x)^n = p_0 + p_1x + p_2x^2 + \cdots + p_nx^n$, where $n$ is a positive integer, prove that
  $p_0 - p_2 + p_4 - \cdots = 2^{n/2}\cos\frac{n\pi}{4}$.
  %19
\item If $n$ be a positive integer, prove that $(1 + i)^n + (1 - i)^n = 2^{n/2 + 1}\cos\frac{n\pi}{4}$.
  %20
\item If $n$ be a positive integer, prove that $(\sqrt{3} + i)^n + (\sqrt{3} - i)^n = 2^{n +
  1}\cos\frac{n\pi}{6}$.
  %21
\item If $(1 + x)^n = C_0 + C_1x + C_2x^2 + \cdots + C_nx^n0,$ where $n$ is a positive integer, prove that
  $C_1 - C_3 + C_5 - C_7 + \cdots = 2^{n/2}\sin\frac{n\pi}{4}$ and $C_0 + C_4 + C_8 + \cdots = 2^{n/2 -
  1}\cos\frac{n\pi}{4} + 2^{n - 2}$.
  %22
\item If $x + \frac{1}{x} = 2\cos\theta$, show that $x^7 + \frac{1}{x^7} = 2\cos7\theta$.
  %23
\item if $x + \frac{1}{x} = 2\cos\alpha, y + \frac{1}{y} = 2\cos\beta, z + \frac{1}{z} = 2\cos\gamma$ and so
  on, prove that $xyz + \frac{1}{xyz} = 2\cos(\alpha + \beta + \gamma)$.
  %24
\item Find the equation whose roots are the $n$th power of the roots of the equation $x^2 - 2x\cos\theta +
  1= 0$.
  %25
\item If $\alpha$ and $\beta$ are the roots of the equation $x^2 - 2x + 4 = 0$, prove that $\alpha^n +
  \beta^n = 2^{n + 1}\cos\frac{n\pi}{3}$.
  %26
\item Find the general value of $\theta$ which satisfies the equation $(\cos\theta +
  i\sin\theta)(\cos2\theta + i\sin2\theta)\cdots(\cos n\theta + i\sin n\theta) = 1$.
  %27
\item If $z + \frac{1}{z} = 2\sin\theta$, find $z^{4n} + \frac{1}{z^{4n}}$, where $n$ is a positive integer.
  %28
\item If $2\cos\alpha = x + \frac{1}{x}, 2\cos\beta = y + \frac{1}{y}, 2\cos\gamma = z + \frac{1}{z}$, prove
  that $2\cos(p\alpha + q\beta + r\gamma) = x^py^qz^2 + \frac{1}{x^py^qz^r}$.
  %29
\item If $x + \frac{1}{x} = 2\cos\theta, y + \frac{1}{y} = 2\cos\phi$, prove that $xy + \frac{1}{xy} =
  2\cos(\theta + \phi)$.
  %30
\item If $x = \cos\theta + i\sin\theta$, prove that $\frac{1}{2x + \frac{1}{2x + \frac{1}{2x} + \cdots
    \infty}} = \left[\sqrt{\cos\theta + \cos^2\theta} - \cos\theta\right] + i\left[\sqrt{\cos\theta -
    \cos^2\theta} - \sin\theta\right]$.
  %31
\item If $x = \cos\theta + i\sin\theta$ and $\sqrt{1 - c^2} = nc - 1$ show that $1 + c\cos\theta =
  \frac{c}{2n}(1 + nx)\left(1 + \frac{n}{x}\right)$.
  %32
\item If $a = \cos\alpha + i\sin\alpha, b = \cos\beta + i\sin\beta, c = \cos\gamma + i\sin\gamma, d =
  \cos\delta + i\sin\delta$ prove that $\sqrt{abcd} + \frac{1}{\sqrt{abcd}} = 2\cos(\alpha + \beta + \gamma
  + \delta)$.
  %33
\item If $x = \cos\theta + i\sin\theta$ and $y = \cos\phi + i\sin\phi$ prove that $x^my^n + \frac{1}{x^my^n}
  = 2\cos(m\theta + n\phi)$.
  %34
\item If $\alpha$ be either of the roots of the equation $x^2 - 2ax\cos\theta + a^2 = 0$ prove that
  $\alpha^{2n} - 2a^nx^n\cos n\theta + a^{2n} = 0$.
  %35
\item If $x = \cos\alpha + i\sin\alpha, y = \cos\beta + i\sin\beta, z = \cos\gamma + i\sin\gamma$ and $x + y
  + z = 0$, prove that $\frac{1}{x} + \frac{1}{y} + \frac{1}{z} = 0$.
  %36
\item If $\cos\alpha + \cos\beta + \cos\gamma  = 0$ and $\sin\alpha + \sin\beta + \sin\gamma = 0$, prove
  that $\sin2\alpha + \sin2\beta + \sin2\gamma = \cos2\alpha + \cos2\beta + \cos2\gamma = 0$ and
  $\sin^2\alpha + \sin^2\beta + \sin^2\gamma = \cos^2\alpha + \cos^2\beta + \cos^2\gamma = \frac{3}{2}$.
  %37
\item If $\cos\alpha + \cos\beta + \cos\gamma  = 0$ and $\sin\alpha + \sin\beta + \sin\gamma = 0$, prove
  that $\cos3\alpha + \cos3\beta + \cos3\gamma = 3\cos(\alpha + \beta + \gamma)$ and $\sin3\alpha +
  \sin3\beta + \sin3\gamma = 3\sin(\alpha + \beta + \gamma)$.
  %38
\item If $x = \cos\alpha + i\sin\alpha, y = \cos\beta + i\sin\beta, z = \cos\gamma + i\sin\gamma$ and $x + y
  + z = xyz$ then prove that $\cos(\alpha - \beta) + \cos(\beta - \gamma) + \cos(\gamma - \alpha) = -1$.
  %39
\item If $x = \cos\theta + i\sin\theta, y = \cos\phi + i\sin\phi$, show that $\frac{x^3}{y^3} +
  \frac{y^3}{x^3} = 2\cos3(\theta - \phi)$.
  %40
\item If $\alpha, \beta, \gamma$ be three positive angles each less than $\pi$ such that $\sum\cos2\alpha =
  \sum\cos(\beta + \gamma)$ and $\sum\sin2\alpha = \sum\sin(\beta + \gamma)$, prove that $\alpha = \beta =
  \gamma$.
  %41
\item If $\cos\alpha + \cos\beta + \cos\gamma  = 0$ and $\sin\alpha + \sin\beta + \sin\gamma = 0$, prove
  that $\cos2\alpha + \cos2\beta + \cos2\gamma + 2[\cos(\alpha + \beta) + \cos(\beta + \gamma) + \cos(\gamma
    + \alpha)] = 0$.
  %42
\item Find all the values of $(-1)^{1/6}$.
  %43
\item Find all the values of $(1 + i)^{1/7}$.
  %44
\item Find all the values of $(1 + i)^{1/3}$ and prove that their product is $1 + i$.
  %45
\item Prove that the $n$th roots of unity are in G.P.
  %46
\item Apply De Moivre's theorem to solve the equation $x^7 + x^4 + x^3 + 1 = 0$.
  %47
\item Solve $x^6 + x^5 + x^4 + x^3 + x^2 + x + 1 = 0$ using De Moivre's theorem.
  %48
\item Solve $x^7 = 1$ by De Moivre's theorem and prove that the sum of the $n$th powers of the roots of the
  equation, $n$ being an integer not divisible by $7$, is zero.
  %49
\item Solve $x^{12} - 1 = 0$ and determine which of the roots also satisfy $x^4 + x^2 + 1 = 0$.
  %50
\item If $\alpha$ denotes any $n$th roots of unity, then show that $1 + \alpha + \alpha^2 + \cdots +
  \alpha^{n - 1} = 0$.
  %51
\item Show that the product of $n$, $n$th roots of unity is $(-1)^{n - 1}$ and their sum is zero.
  %52
\item Solve the equation $(x - 1)^n = x^n$, where $n$ is a positive integer.
  %53
\item Solve the equation $x^9 - 1 = 0$ by De Moivre's theorem and determine which of the roots also satisfy
  $x^3 - 1 = 0$.
  %54
\item Solve the equation $x^4 + x^3 + x^2 + x + 1 = 0$.
\stopitemize