% -*- mode: context; -*-
\chapter{Inverse Circular Functions}
Inverse functions related to trigonometric ratios are called inverse trigonometric functions. The definition of different
inverse trigonometric functions is given below:

If $\sin\theta = x$, then $\theta = \sin^{-1}x$, provided $-1\leq x\leq 1$ and $-\frac{\pi}{2}\leq\theta\leq\frac{\pi}{2}$.

If $\cos\theta = x$, then $\theta = \cos^{-1}x$, provided $-1\leq x\leq 1$ and $0\leq\theta\leq\pi$.

If $\tan\theta = x$, then $\theta = \tan^{-1}x$, provided $-\infty<x<\infty$ and $-\frac{\pi}{2}<\theta<\frac{\pi}{2}$.

If $\cot\theta = x$, then $\theta = \cot^{-1}x$, provided $-\infty<x<\infty$ and $0<\theta<\pi$.

If $\sec\theta = x$, then $\theta = \sec^{-1}x$, provided $x\leq -1$ or $x\geq 1$ and $0\leq \theta\leq
\pi,\theta\neq\frac{\pi}{2}$.

If ${\rm cosec}\theta = x$, then $\theta = {\rm cosec}^{-1}x$, provided $x\leq -1$ or $x\geq 1$ and $-\frac{\pi}{2}\leq\theta\leq
\frac{\pi}{2}, \theta\neq0$.

{\bf Note:} In the above definition, restrictions on $\theta$ are due to the consideration of principal values of inverse
terms. If these restrictions are removed, the terms will represent inverse trigonometric relations and not functions.

{\bf Notations: I.} ${\rm Arc}\sin x$ denotes the sine inverse of $x$ [General value]. $\arcsin x$ denotes the principal value
of sine inverse of $x$.

{\bf II.} $\sin^{-1}x$ denotes the principal value of sine inverse $x$. From the above notations three important results follow;
\startitemize[n]
\item $\sin^{-1}x = \theta \Rightarrow \sin\theta = x$ and $\theta$ is the principal value.
\item $\sin^{-1}x = \arcsin x, \cos^{-1}x = \arccos x$.
\item From the definition of the inverse functions, we know that if $y = f(x)$ is a function then for $f^{-1}$ to be a function,
  $f$ must be one-to-one and onto mapping.
\stopitemize

When we consider $y = {\rm Arc}\sin x$, for any $x\in[-1, 1]$ infinite number of values of $y$ are obtained and hence it does not
represent inverse functions. When $y = \arcsin x$ or $\sin^{-1}x$, corresponding to one value of $x\in[-1, 1]$, one value of $y$ is
obtained and hence it represents the inverse trigonometric function.

Hence, for inverse trigonometric functions, consideration of principal values is essential.

\section{Principal Value}
Numerically smallest angle is known as the principal value.

Since inverse trigonometric terms are in fact angles, definitions of principal value of inverse trigonometric term is the same as
the definition of the principal values of angles.

Suppose we have to find the principal value of $\sin^{-1}\frac{1}{2}$. Let $\sin^{-1}\frac{1}{2} = \theta$, then $\sin\theta =
\frac{1}{2} \Rightarrow \theta = \ldots, -\frac{11\pi}{6}, -\frac{7\pi}{6}, \frac{\pi}{6}, \frac{5\pi}{6}, \ldots$. Among all these
angles $\frac{\pi}{6}$ is the numerically smalles angles satisfying $\sin\theta = \frac{1}{2}$ and hence it is the principal value.

\section{Important Formulae}
\starttheorem
  \startitemize[n]
  \item $\sin\sin^{-1}x = x, -1\leq x\leq 1$
  \item $\cos\cos^{-1}x = x, -1\leq x\leq 1$
  \item $\tan\tan^{-1}x = x, -\infty < x\leq\infty$
  \item $\cot\cot^{-1}x = x, -\infty < x\leq\infty$
  \item $\sec\sec^{-1}x = x, x\leq -1$ or $x\geq 1$
  \item ${\rm cosec\,cosec}^{-1}x = x, x\leq -1$ or $x\geq 1$
  \stopitemize
\stoptheorem

\startproof
Let $\sin^{-1}x = \theta$ then $\sin\theta = x$. Putting the value of $\theta$ from first equation in second
$\sin\sin^{-1}x = x$. Other formulae can be proved similarly.
\stopproof

\starttheorem
  \startitemize[n]
  \item $\sin^{-1}x + \cos^{-1}x = \frac{\pi}{2}\,\forall -1\leq x\leq 1$
  \item $\tan^{-1}x + \cot^{-1}x = \frac{\pi}{2}\,\forall x\in\mathbb{R}$
  \item $\sec^{-1}x + {\rm cosec}^{-1}x = \frac{\pi}{2}\,\forall x\leq -1$ or $x\geq 1$
  \stopitemize
\stoptheorem

\startproof
  Let $\sin^{-1}x = \theta$, then $\sin\theta = x \Rightarrow \cos\left(\frac{\pi}{2} - \theta\right) = x \Rightarrow \frac{\pi}{2}
  - \theta = \cos^{-1}x$

  $\Rightarrow \cos^{-1}x + \theta = \frac{\pi}{2} \Rightarrow \sin^{-1}x + \cos^{-1}x = \frac{\pi}{2}$.

  Similarly other results can be proven.
\stopproof

\starttheorem
  \startitemize[n]
  \item $\sin^{-1}x = {\rm cosec}^{-1}\frac{1}{x}, -1\leq x\leq 1$
  \item ${\rm cosec^{-1}x} = \sin^{-1}\frac{1}{x}, x\leq -1$ or $x > 1$
  \item $\cos^{-1}x = \sec^{-1}\frac{1}{x}, -1\leq x\leq 1$
  \item $\sec^{-1}x = \cos^{-1}\frac{1}{x}, x\leq -1$ or $x\geq 1$
  \stopitemize
\stoptheorem

\startproof
  Let $\sin^{-1}x = \theta$ then $\sin\theta = x \Rightarrow \rm cosec\theta = \frac{1}{x}$

  $\Rightarrow \theta = \rm cosec^{-1}\frac{1}{x} \Rightarrow \sin^{-1}x = \rm cosec^{-1}\frac{1}{x}$

  Other results can be proven similarly.
\stopproof

\starttheorem
  \startitemize[n]
    \item $\sin^{-1}x = \cos^{-1}\sqrt{1 - x^2}, \,\forall\,0\leq x\leq 1$

    \item $\sin^{-1}x = -\cos^{-1}\sqrt{1 - x^2}\,\forall\,-1\leq x< 0$
  \stopitemize
\stoptheorem

\startproof
  Let $\sin^{-1}x = \theta$ then $\sin\theta = x$

  $\Rightarrow \cos^2\theta = 1 - x^2 \Rightarrow \cos\theta = \pm\sqrt{1 - x^2}$

  Principal values of $\sin^{-1}x$ lies between $-\frac{\pi}{2}$ and $\frac{\pi}{2}$.

  In this interval $\cos\theta$ is $+$ve.

  $\Rightarrow \sin^{-1}x = \cos^{-1}\sqrt{1 - x^2}$

  For $-1\leq x < 0$ $\sin^{-1}x$ will be negative angle while $\cos^{-1}\sqrt{1 - x^2}$ will be positive
  angle. Hence to balance that we need to use a negative sign for this.
\stopproof

\starttheorem
  \startitemize[n]
    \item $\sin^{-1}(-x) = -\sin^{-1}x$

    \item $\cos^{-1}(-x) = \pi - \cos^{-1}x$

    \item $\tan^{-1}(x) = -\tan^{-1}x$

    \item $\cot^{-1}x = \pi - \cot^{-1}x$
  \stopitemize
\stoptheorem

\startproof
  Let $\cos^{-1}(-x) = \theta$ then $\cos\theta = -x$

  $-\cos\theta = x \Rightarrow \cos(\pi - \theta) = x$

  $\therefore \theta = \pi - \cos^{-1}x$

  Note: $\cos(\pi + \theta)$ is also equal to $-\cos\theta$ but this will make principal value greater than $\pi.$

  Similarly other results can be proven.
\stopproof

\starttheorem
  \startitemize[n]
  \item $\tan^{-1}x + \tan^{-1}y = \tan^{-1}\frac{x + y}{1 - xy}$ where $x, y > 0$ and $xy < 1$

  \item $\tan^{-1}x + \tan^{-1}y = \pi + \tan^{-1}\frac{x + y}{1 - xy}$ where $x, y > 0$ and $xy > 1$

  \item $\tan^{-1}x + \tan^{-1}y = -\pi + \tan^{-1}\frac{x + y}{1 - xy}$ where $x, y , 0$ and $xy > 1$

  \item $\tan^{-1}x - \tan^{-1}y = \tan^{-1}\frac{x - y}{1 + xy}$ where $xy > 1$
  \stopitemize
\stoptheorem

\startproof
  Let $\tan^{-1}x = \alpha$ and $\tan^{-1}y = \beta$ then

  $\tan\alpha = x$ and $\tan\beta = y$

  $\tan(\alpha + \beta) = \frac{\tan\alpha + \tan\beta}{1 - \tan\alpha\tan\beta} = \frac{x + y}{1 - xy}$

  $\Rightarrow \alpha +\beta = \tan^{-1}\frac{x + y}{1 = xy}$

  $\Rightarrow \tan^{-1}x + \tan^{-1}y = \tan^{-1}\frac{x + y}{1 - xy}$

  {\bf Case I.} When $x, y > 0$ and $xy < 1, \tan^{-1}\frac{x + y}{1 - xy} > 0$

  therefore $\tan^{-1}\frac{x + y}{1 - xy}$ will be a positive angle.

  {\bf Case II.} When $x, y >0$ and $xy > 1$ $\tan^{-1}\frac{x + y}{1 - xy}$ will be a negative angle.

  $\therefore \tan^{-1}x + \tan^{-1}y = \pi + \tan^{-1}\frac{x + y}{1 - xy}$

  {\bf Case III.} When $x, y< 0$ and $xy > 1$, $\tan^{-1}x + \tan^{-1}y$ will be a negative angle and
  $\tan^{-1}\frac{x + y}{1 - xy}$ will be a positive angle.

  To balance it we will need to add $-\pi$

  $\therefore \tan^{-1}x + \tan^{-1}y = -\pi + \tan^{-1}\frac{x + y}{1 - xy}$

  Similarly other result can be proven.
\stopproof

$\tan^{-1}x + \tan^{-1}y + \tan^{-1}z = \frac{x + y + z - xyz}{1 - xy - yz - xz}$ can be proven similarly.

\starttheorem
  \startitemize[n]
  \item $\sin^{-1}x + \sin^{-1}y = \sin^{-1}[x\sqrt{1 - y^2} + y\sqrt{1 - x^2}]$ if $-1\leq x, y\leq 1$ and $x^2 +
    y^2\leq 1$ or if $xy < 0$ and $x^2 + y^2 > 1$

  \item $\sin^{-1}x - \sin^{-1}y = \sin^{-1}[x\sqrt{1 - y^2} - y\sqrt{1 - x^2}]$ if $-1\leq x, y\leq 1$ and $x^2 +
    y^2\leq 1$ or if $xy > 0$ and $x^2 + y^2 > 1$
  \stopitemize
 \stoptheorem

 \startproof
   Let $\sin^{-1}x = \alpha$ and $\sin^{-1}y = \beta$ then $\sin\alpha = x, \sin\beta = y.$

   Now $\sin(\alpha + \beta) = \sin\alpha\cos\beta + \sin\beta\cos\alpha$

   $= \sin\alpha\sqrt{1 - \sin^2\beta} + \sin\beta\sqrt{1 - \sin^2\alpha}$

   $= x\sqrt{1 - y^2} + y\sqrt{1 - x^2}$

   $\alpha + \beta = \sin^{-1}[x\sqrt{1 - y^2} + y\sqrt{1 - x^2}]$

   Similarly we can prove that $\sin^{-1}x - \sin^{-1}y = \sin^{-1}[x\sqrt{1 - y^2} - y\sqrt{1 - x^2}]$
 \stopproof

\starttheorem
  \startitemize[n]
  \item $2\tan^{-1}x = \sin^{-1}\frac{2x}{1 + x^2},$ where $|x|< 1$

  \item $2\tan^{-1}x = \cos^{-1}\frac{1 - x^2}{1 + x^2},$ where $x\geq 0$

  \item $2\tan^{-1}x = \tan^{-1}\frac{2x}{1 - x^2},$ where $|x| < 1$
  \stopitemize
\stoptheorem

\startproof
  \startitemize[n]
  \item Let $\tan^{-1}x = \theta$ then $\tan\theta = x$

   $\sin2\theta = \frac{2\tan\theta}{1 + \tan^2x\theta} = \frac{2x}{1 + x^2}$

   $\Rightarrow 2\theta = \sin^{-1}\frac{2x}{1 + x^2}\Rightarrow 2\tan^{-1}x = \sin^{-1}\frac{2x}{1 + x^2}$

   Here, $-\frac{\pi}{2}\leq \sin^{-1} \leq \frac{\pi}{2}$

   $\Rightarrow -\frac{\pi}{2}\leq 2\tan^{-1}x\leq \frac{\pi}{2}$

   $\Rightarrow -\frac{\pi}{4}\leq \tan^{-1}x\leq \frac{\pi}{4}$

   $\Rightarrow -1\leq x\leq 1 \Rightarrow |x| < 1$

 \item $\cos2\theta = \frac{1 - \tan^2\theta}{1 + \tan^2\theta} = \frac{1 - x^2}{1 + x^2}$

   $\Rightarrow 2\theta = \cos^{-1}\left(\frac{1 - x^2}{1 + x^2}\right)$

   $\Rightarrow 2\tan^{-1}x = \cos^{-1}\frac{1 - x^2}{1 + x^2}$

   For $x \geq 0$ both sides will be balanced.

   For $x<0, 2\tan^{-1}x$ will represent a negative angle where R.H.S. will always lie between $0$ and $\pi.$
   Hence two sides cannot be equal.

 \item $\tan2\theta = \frac{2\tan\theta}{1 - \tan^2\theta} = \frac{2x}{1 - x^2}\Rightarrow 2\theta = \tan^{-1}\frac{2x}{1 - x^2}$

   $2\tan^{-1}x = \tan^{-1}\frac{2x}{1 - x^2}$ which holds good for $|x|< 1$
 \stopitemize
\stopproof

\starttheorem
  \startitemize[n]
    \item $2\sin^{-1}x = \sin^{-1}[2x\sqrt{1 - x^2}]$ if $-\frac{1}{\sqrt{2}}\leq x\leq \frac{1}{\sqrt{2}}$

    \item $2\cos^{-1}x = \cos^{-1}(2x^2 - 1)$ where $0\leq x \leq 1$
  \stopitemize
\stoptheorem

These can be proven like $\sin^{-1}x + \sin^{-1}y$

\starttheorem
  \startitemize[n]
  \item $3\sin^{-1}x = \sin^{-1}(3x - 4x^3)$ where $-\frac{1}{2}\leq x\leq \frac{1}{2}$

  \item $3\cos^{-1}x = \cos^{-1}(4x^3 - 3x)$ where $\frac{1}{2}\leq x\leq 1$

  \item $3\tan^{-1}x = \tan^{-1}\frac{3x - x^3}{1 - 3x^2}$ where $-\frac{1}{\sqrt{3}}< x < \frac{1}{\sqrt{3}}$
\stopitemize
\stoptheorem

These can be proven like previous proof.

\section{Graph of Important Inverse Trigonometric Functions}
\startitemize[n]
\item $y = \sin^{-1}x, -1\leq x\leq 1$
  \startplacefigure[title={}]
    \externalfigure[25_1.pdf]
  \stopplacefigure

   From this graph we observer following:
   \startitemize[n]
   \item Domain is $-1\leq x\leq 1$

   \item Range is $-\frac{\pi}{2}\leq y \leq \frac{\pi}{2}$

   \item $\because \sin^{-1}x = -\sin^{-1}x \therefore y = \sin^{-1}x$ is an odd function.

   \item It is a non-periodic function

   \item It passes through origin i.e. when $x = 0, y = 0$
   \stopitemize
 \item $y = \cos^{-1}x, -1\leq x\leq 1$
  \startplacefigure[title={}]
    \externalfigure[25_2.pdf]
  \stopplacefigure

   Follwing points can be observed from the graph:
   \startitemize[n]
   \item Domain is $-1\leq x\leq 1$

   \item Range is $0\leq x\leq \pi$

   \item $\because \cos^{-1}(-x) = \pi - \cos^{-1}x$

     $\Rightarrow y = \cos^{-1}x$ is neither odd nor even.

   \item It is a non-periodic function
   \stopitemize

 \item $y = \tan^{-1}x, -\infty< x <\infty$
  \startplacefigure[title={}]
    \externalfigure[25_3.pdf]
  \stopplacefigure

   From the graph follwing points can be observed:
   \startitemize[n]
   \item Domain is $-\infty < x < \infty$

   \item Range is $-\frac{\pi}{2} < x < \frac{\pi}{2}$

   \item $y = \tan^{-1}x$ is an odd function

   \item It is a non-periodic function.

   \item It passes through origin.
   \stopitemize
 \item $y = \cot^{-1}x, -\infty< x <\infty$
  \startplacefigure[title={}]
    \externalfigure[25_4.pdf]
  \stopplacefigure

   From the graph follwing points can be observed:
   \startitemize[n]
   \item Domain is $-\infty < x < \infty$

   \item Range is $0<y<\pi$

   \item The function is neither odd nor even.

   \item It is a non-periodic function
   \stopitemize
\stopitemize

\section{Problems}
Evaluate the following:
\startitemize[n, 1*broad]
\item $\tan^{-1}(-1)$
\item $\cot^{-1}(-1)$
\item $\sin^{-1}\left(-\frac{\sqrt{3}}{2}\right)$

Find the value of the following:

\item $\sin\left[\frac{\pi}{3} - \sin^{-1}\frac{-1}{2}\right]$
\item $\sin\left[\cos^{-1}\frac{-1}{2}\right]$
\item $\sin\left[\tan^{-1}(-\sqrt{3}) + \cos^{-1}\frac{-\sqrt{3}}{2}\right]$
\item Evaluate $\tan\left[\frac{1}{2}\cos^{-1}\frac{\sqrt{5}}{3}\right]$
\item Find the angle $\sin^{-1}\left(\sin\frac{2\pi}{3}\right)$

Find the value of the following:

\item $\sin^{-1}\frac{\sqrt{3}}{2}$
\item $\tan^{-1}\frac{-1}{\sqrt{3}}$
\item $\cot^{-1}(-\sqrt{3})$
\item $\cot^{-1}\cot\frac{5\pi}{4}$
\item $\tan^{-1}\left(\tan\frac{3\pi}{4}\right)$
\item $\sin^{-1}\frac{1}{2} + \cos^{-1}\frac{1}{2}$
\item $\cos\left[\tan^{-1}\left(\frac{3}{4}\right)\right]$
\item $\cos\left[\cos^{-1}\left(\frac{\sqrt{3}}{2}\right) + \frac{\pi}{6}\right]$

\item Prove that $2\tan^{-1}\frac{1}{3} + \tan^{-1}\frac{1}{7} = \frac{\pi}{4}$
\item Prove that $\tan^{-1}\frac{1}{3} + \tan^{-1}\frac{1}{5} + \tan^{-1}\frac{1}{7} + \tan^{-1}\frac{1}{8} = \frac{\pi}{4}$
\item Prove that $\sin^{-1}\frac{4}{5} + \sin^{-1}\frac{5}{13} + \sin^{-1}\frac{16}{65} = \frac{\pi}{2}$
\item Prove that $4\tan^{-1}\frac{1}{5} - \tan^{-1}\frac{1}{70} + \tan^{-1}\frac{1}{99} = \frac{\pi}{4}$
\item Prove that $\cot^{-1}9 + \rm cosec^{-1}\frac{\sqrt{41}}{4} = \frac{\pi}{4}$
\item Prove that $4(\cot^{-1}3 + \rm cosec^{-1}\sqrt{5}) = \pi$
\item Prove that $\tan^{-1}x = 2\tan^{-1}[\rm cosec\tan^{-1}x - \tan\cot^{-1}x]$
\item Prove that $2\tan^{-1}\left[\sqrt{\frac{a - b}{a + b}}\tan\frac{x}{2}\right] = \cos^{-1}\left[\frac{b + a\cos x}{a +
    b\cos x}\right]$ for $0<b\leq a,$ and $x\geq 0.$
\item Prove that $\tan^{-1}\frac{x - y}{1 + xy} + \tan^{-1}\frac{y - z}{1 + yz} + \tan^{-1}\frac{z - x}{1 + zx} =
    \tan^{-1}\left(\frac{x^2 - y^2}{1 + x^2y^2}\right) + \tan^{-1}\left(\frac{y^2 - z^2}{1 + y^2z^2}\right) +
    \tan^{-1}\left(\frac{z^2 - x^2}{1 + z^2x^2}\right)$
\item Prove that $\sin\cot^{-1}\tan\cos^{-1}x = x$
\item Prove that $\tan^{-1}\left(\frac{1}{2}\tan 2x\right) + \tan^{-1}(\cot x) +\tan^{-1}(\cot^3x) = 0$ if
    $\frac{\pi}{4}< x < \frac{\pi}{2}, = \pi$ if $0<x<\pi$
\item Prove that $\tan^{-1}\frac{1}{2} + \tan^{-1}\frac{1}{3} = \tan^{-1}\frac{3}{5} + \tan^{-1}\frac{1}{4} = \frac{\pi}{4}$
\item Prove that $\tan^{-1}\frac{2a - b}{\sqrt{3}b} + \tan^{-1}\frac{2b - a}{\sqrt{3}a} = \frac{\pi}{3}$
\item Prove that $\tan^{-1}\frac{2}{5} + \tan^{-1}\frac{1}{3} + \tan^{-1}\frac{1}{12} = \frac{\pi}{4}$
\item Prove that $2\tan^{-1}\frac{1}{5} + \tan^{-1}\frac{1}{4} = \tan^{-1}\frac{32}{43}$
\item Prove that $\tan^{-1}1 + \tan^{-1}2 + \tan^{-1}3 = \pi = 2\left(\tan^{-1}1 + \tan^{-1}\frac{1}{2} +
    \tan^{-1}\frac{1}{3}\right)$
\item Prove that $\tan^{-1}x + \cot^{-1}y = \tan^{-1}\frac{xy + 1}{y - x}$
\item Prove that $\tan^{-1}\frac{1}{x + y} + \tan^{-1}\frac{y}{x^2 + xy + 1} = \cot^{-1}x$
\item Prove that $2\cot^{-1}5 + \cot^{-1}7 + 2\cot^{-1}8 = \pi/4$
\item Prove that $\tan^{-1}\frac{a - b}{1 + ab} + \tan^{-1}\frac{b - c}{1 + bc} + \tan^{-1}\frac{c - a}{1 + ca} = 0$
\item Prove that $\tan^{-1}\frac{a^3 - b^3}{1 + a^3b^3} + \tan^{-1}\frac{b^3 - c^3}{1 + b^3c^3} + \tan^{-1}\frac{c^3 - a^3}{1 +
    c^3a^3} = 0$
\item Prove that $\cot^{-1}\frac{xy + 1}{y - x} + \cot^{-1}\frac{yz + 1}{z - y} + \cot^{-1}z = \tan^{-1}\frac{1}{x}$
\item Prove that $\cos^{-1}\left(\frac{\cos\theta + \cos\phi}{1 + \cos\theta\cos\phi}\right) =
    2\tan^{-1}\left(\tan\frac{\theta}{2}\tan\frac{\phi}{2}\right)$
\item Prove that $\sin^{-1}\frac{3}{5} + \sin^{-1}\frac{8}{17} = \sin^{-1}\frac{77}{85}$
\item Prove that $\cos^{-1}\frac{3}{5} + \cos^{-1}\frac{12}{13} + \cos^{-1}\frac{63}{65} = \frac{\pi}{2}$
\item Prove that $\sin^{-1}x + \sin^{-1}y = \cos^{-1}\left(\sqrt{1 - x^2}\sqrt{1 - y^2} - xy\right)$ where $x, y \in[0,
    1]$
\item Prove that $4\left(\sin^{-1}\frac{1}{\sqrt{10}} + \cos^{-1}\frac{2}{\sqrt{5}}\right) =\pi$
\item Prove that $\cos(2\sin^{-1}x) = 1 - 2x^2$
\item Prove that $\frac{1}{2}\cos^{-1}x = \sin^{-1}\sqrt{\frac{1 - x}{2}} = \cos^{-1}\sqrt{\frac{1 + x}{2}} =
    \tan^{-1}\frac{\sqrt{1 - x^2}}{1 + x}$
\item Prove that $\sin^{-1}x + \cos^{-1}y = \tan^{-1}\frac{xy + \sqrt{(1 - x^2)(1 - y^2)}}{y\sqrt{1 - x^2} - x\sqrt{1 - y^2}}$
\item Prove that $\tan^{-1}x + \tan^{-1}y = \frac{1}{2}\sin^{-1}\frac{2(x + y)(1 - xy)}{(1 + x^2)(1 + y^2)}$
\item Prove that $2\tan^{-1}(\rm cosec\tan^{-1}x - \tan\cot^{-1}x) = \tan^{-1}x$
\item Prove that $\cos\tan^{-1}\sin\cot^{-1}x = \sqrt{\frac{x^2 + 1}{x^2 + 2}}$
\item In any $\triangle ABC$ if $A = \tan^{-1}2$ and $B = \tan^{-1}3,$ prove that $C = \frac{\pi}{4}$
\item If $\cos^{-1}x + \cos^{-1}y + \cos^{-1}z = \pi$ then prove that $x^2 + y^2 + z^2 + 2xyz = 1$
\item If $\cos^{-1}\frac{x}{2}+ \cos^{-1}\frac{y}{3} = \theta,$ prove that $9x^2 - 12xy\cos\theta + 4y^2 =
    36\sin^2\theta$
\item If $r = x + y + z$ then prove that $\tan^{-1}\sqrt{\frac{xr}{yz}} + \tan^{-1}\sqrt{\frac{yr}{xz}} +
    \tan^{-1}\sqrt{\frac{zr}{xy}} = \pi$
\item If $u = \cot^{-1}\sqrt{\cos2\theta} - \tan^{-1}\sqrt{\cos2\theta}$ then prove that $\sin u = \tan^2\theta$
\item Solve $\cos^{-1}x\sqrt{3} + \cos^{-1}x = \frac{\pi}{2}$
\item Solve $\sin^{-1}x + \sin^{-1}2x = \frac{\pi}{3}$
\item If $\tan^{-1}x + \tan^{-1}y + \tan^{-1}z= \frac{\pi}{2},$ prove that $xy + yz + zx = 1$
\item If $\tan^{-1}x + \tan^{-1}y + \tan^{-1}z= \pi,$ prove that $x + y + z = xyz$
\item If $\sin^{-1}x + \sin^{-1}y = \frac{\pi}{2},$ prove that $x\sqrt{1 - y^2} + y\sqrt{1 - x^2} = 1$
\item If $\sin^{-1}x + \sin^{-1}y + \sin^{-1}z = \pi,$ prove that $x\sqrt{1 - x^2} + y\sqrt{1 - y^2} + z\sqrt{1 - z^2} =
    2xyz$
\item Establish the relationship between $\tan^{-1}x, \tan^{-1}y, \tan^{-1}z$ are in A.P. and if further $x, y, z$ are
    also in A.P. then prove that $x = y = z.$
\item Solve for $x, \cot^{-1}x + \sin^{-1}\frac{1}{\sqrt{5}} = \frac{\pi}{4}$
\item Solve $\tan^{-1}2x + \tan^{-1}3x = \frac{\pi}{4}$
\item Solve $\tan^{-1} x + \tan^{-1}\frac{2x}{1 - x^2} = \frac{\pi}{3}$
\item Solve $\tan^{-1}\frac{1}{2} = \cot^{-1}x + \tan^{-1}\frac{1}{7}$
\item Solve $\tan^{-1}(x - 1) + \tan^{-1}x + \tan^{-1}(x + 1) = \tan^{-1}3x$
\item Solve $\tan^{-1}\frac{x + 1}{x - 1} + \tan^{-1}\frac{x - 1}{x} = \pi + \tan^{-1}(-7)$
\item Solve $\cot^{-1}(a - 1) = \cot^{-1}x + \cot^{-1}(a^2 - x + 1)$
\item Solve $\sin^{-1}\frac{2\alpha}{1 + \alpha^2} + \sin^{-1}\frac{2\beta}{1 + \beta^2} = 2\tan^{-1}x$
\item Solve $\cos^{-1}\frac{x^2 - 1}{x^2 + 1} + \tan^{-1}\frac{2x}{x^2 - 1} = \frac{2\pi}{3}$
\item Solve $\sin^{-1}\frac{2a}{1 + a^2} + \cos^{-1}\frac{1 - b^2}{1 + b^2} = 2\tan^{-1}x$
\item Solve $\sin^{-1}x + \sin^{-1}(1 - x) = \cos^{-1}x$
\item Solve $\tan^{-1}ax + \frac{1}{2}\sec^{-1}bx = \frac{\pi}{4}$
\item Solve $\tan(\cos^{-1}x) = \sin(\tan^{-1}2)$
\item Solve $\tan\left(\sec^{-1}\frac{1}{x}\right) = \sin\cos^{-1}\frac{1}{\sqrt{5}}$
\item Find the values of $x$ and $y$ satisfying $\sin^{-1}x + \sin^{-1}y = \frac{2\pi}{3}$ and $\cos^{-1}x -
    \cos^{-1}y = \frac{\pi}{3}$
\item Find the angle $\sin^{-1}(\sin10)$
\item Using principal values, express the following as a single angle $3\tan^{-1}\frac{1}{2} + 2\tan^{-1}\frac{1}{5} +
    \sin^{-1}\frac{142}{65\sqrt{5}}$
\item Find the value of $2\cos^{-1}x + \sin^{-1}x$ at $x = \frac{1}{5}$ where $0\leq \cos^{-1}x\leq \pi$ and
    $-\frac{\pi}{2}\leq \sin^{-1}x \leq \frac{\pi}{2}$.
\item Show that $\frac{1}{2}\cos^{-1}\frac{3}{5} = \tan^{-1}\frac{1}{2} = \frac{\pi}{4} - \frac{1}{2}\cos^{-1}\frac{4}{5}$
\item Find the greater angle between $2\tan^{-1}(2\sqrt{2} - 1)$ and $3\sin^{-1}\frac{1}{3} + \sin^{-1}\frac{3}{5}$
\item Prove that $\tan^{-1}\left(\frac{a_1x - y}{x + a_1y}\right) + \tan{-1}\left(\frac{a_2 - a_1}{1 + a_2a_1}\right) +
    \tan^{-1}\left(\frac{a3 - a_2}{1 + a_3a_2}\right) + \ldots + \tan^{-1}\left(\frac{a_n - a_{n - 1}}{1 + a_na_{n - 1}}\right) +
    \tan^{-1}\frac{1}{a_n} = \tan^{-1}\frac{x}{y}$
\item Find the sum $\cot^{-1}2 + \cot^{-1}8 + \cot^{-1}18 + \ldots +$ to $\infty$
\item Show that the function $y = 2\tan^{-1}x + \sin^{-1}\frac{2x}{1 + x^2}$ is constant for $x\geq 1.$ Find the value of
    this constant.
\item Prove the relations $\cos^{-1}x_0 = \frac{\sqrt{1 - x_0^2}}{x_1x_2x_3\ldots\text{~to~}\infty}$ where the successive
    quantities $x_r$ are connected by the relation $x_{r + 1} = \sqrt{\frac{1 + x_r}{2}}$ where $0\leq
    \cos^{-1}x_0\leq \pi$.
\item If $a, b$ are positive quantities and if $a_1 = \frac{a + b}{2}, b_1 = \sqrt{a_1b}, a_2 = \frac{a_1 + b_1}{2}, b_2
    = \sqrt{a_2b_1}$ and so on then show that $\lim_{n\to \infty}a_n\lim_{n\to\infty}b_n = \frac{\sqrt{b^2 - a^2}}{\cos^{-1}\frac{a}{b}}$
\item Using Mathematical Induction prove that $\tan^{-1}\frac{1}{3} + \tan^{-1}\frac{1}{7} + \ldots + \tan^{-1}\frac{1}{n^2 + n
    + 1} = \tan^{-1}\frac{n}{n + 2}$
\item If $x_1, x_2, x_3, x_4$ are the roots of the equation $x^4 - x^3\sin2\beta + x^2\cos2\beta - x\cos\beta - \sin\beta
    = 0$ then prove that $\tan^{-1}x_1 + \tan^{-1}x_2 + \tan^{1}x_3 + \tan^{-1}x_4 = n\pi + \frac{\pi}{2} - \beta$
\item Find theh value of $\cot^{-1}\left(\cot \frac{5\pi}{4}\right)$
\item Find the value of $\sin^{-1}(\sin 5)$
\item Find the value of $\cos^{-1}\cos\frac{5\pi}{4}$
\item Find the value of $\cos^{-1}(\cos 10)$
\item Evaluate $\sin\left(2\tan^{-1}\frac{1}{3}\right) + \cos\tan^{-1}2\sqrt{2}$
\item Evaluate $\cot[\cot^{-1}7 + \cot^{-1}8 + \cot^{-1}18]$
\item Prove that $\sin^{-1}\frac{3}{5} + \cos^{-1}\frac{12}{13} + \cot^{-1}\frac{56}{33} = \frac{\pi}{2}$
\item Prove that $2\cot^{-1}5 + \cot^{-1}7 + 2\cot^{-1}8 = \frac{\pi}{4}$
\item Prove that $\tan^{-1}1 + \tan^{-1}2 + \tan^{-1}3 = 2\left(\tan^{-1}1 + \tan^{-1}\frac{1}{2} +
    \tan^{-1}\frac{1}{3}\right).$
\item If $A = \tan^{-1}\frac{1}{7}$ and $B = \tan^{-1}\frac{1}{3}$ then prove that $\cos 2A = \sin 4B.$
\item Find the sum $\tan^{-1}\frac{x}{1 + 1.2x^2} + \tan^{-1}\frac{x}{1 + 2.3x^2} + \ldots + \tan^{-1}\frac{1}{1 + n(n +
    1)x^2}, x> 0.$
\item Find the sum $\tan^{-1}\frac{d}{1 + a_1a_2} + \tan^{-1}\frac{d}{1 + a_2a_3} + \ldots + \tan^{-1}\frac{d}{1 + a_na_{n +
     1}}$ if $a_1, a_2, \ldots, a_{n + 1}$ form an arithmetic progression with a common difference of $d$ and $d
     > 0, a_i>0$ for $i = 1,2,3,\ldots, n + 1.$
\item For what value of $x,$ the equality $\sin^{-1}(\sin 5) > x^2 - 4x$ holds.
\item If $\tan^{-1}y = 5\tan^{-1}x,$ express $y$ as an algebraic function of $x$ and hence show that
     $18^\circ$ is a root of $5u^4 - 10u^2 + 1 = 0.$
\item If $\cos^{-1}x + \cos^{-1}y + \cos^{-1}z = \pi$ and $x + y + z = \frac{3}{2},$ then prove that $x = y = z.$
\item If $\sin^{-1}x + \sin^{-1}y + \sin^{-1}z = \pi,$ prove that $x^4 + y^4 + z^4 + 4x^2y^2z^2 = 2(x^2y^2 + y^2z^2 +
     z^2x^2).$
\item Prove that $\frac{\alpha^3}{2}\rm cosec^2\left(\frac{1}{2}\tan^{-1}\frac{\alpha}{\beta}\right) +
     \frac{\beta^3}{2}\sec^2\left(\frac{1}{2}\tan^{-1}\frac{\beta}{\alpha}\right) = (\alpha + \beta)(\alpha^2 + \beta^2).$
\item Prove that $2\tan^{-1}\left[\tan\frac{\alpha}{2}\tan\left(\frac{\pi}{4} - \frac{\beta}{2}\right)\right] =
     \tan^{-1}\left[\frac{\sin\alpha\cos\beta}{\sin\beta + \cos\alpha}\right].$
\item Prove that $\tan^{-1}\left[\frac{1}{2}\cos2\alpha\sec2\beta + \frac{1}{2}\cos2\beta\sec2\alpha\right] = \tan^{-1}[\tan^2(\alpha +
     \beta)\tan^2(\alpha -\beta)] + \frac{\pi}{4}.$
\item Express $\cot^{-1}\left(\frac{y}{\sqrt{1 - x^2 - y^2}}\right) = 2\tan^{-1}\sqrt{\frac{3 - 4x^2}{4x^2}} -
     \tan^{-1}\sqrt{\frac{3 - 4x^2}{x^2}}$ as a rational integral equation in $x$ and $y.$
\item If $\frac{m\tan(\alpha - \theta)}{\cos^2\theta} = \frac{n\tan\theta}{\cos^2(\alpha - \theta)}$ then prove that
     $\theta = \frac{1}{2}\left[\alpha - \tan^{-1}\left(\frac{n - m}{n + m}\right)\tan\alpha\right].$
\item If $\sin^{-1}\frac{x}{a} + \sin^{-1}\frac{y}{b} = \sin^{-1}\frac{c^2}{ab}$ then prove that $b^2x^2 +
     2xy\sqrt{a^2b^2 - c^4} = c^4 - a^2y^2.$
\item Prove that $\tan^{-1}t + \tan^{-1}\frac{2t}{1 - t^2} = \tan^{-1}\frac{3t - t^3}{1 - 3t^2},$ if
     $-\frac{1}{\sqrt{3}} < x < \frac{1}{\sqrt{3}}.$
\item Prove that $\cos^{-1}\sqrt{\frac{a - x}{a - b}} = \sin^{-1}\sqrt{\frac{x - b}{a - b}}$ if $a > x> b$ or $a <
     x < b.$
\item Find all values of $p$ and $q$ such that $\cos^{-1}\sqrt{p} + \cos^{-1}\sqrt{1 - p} + \cos^{-1}\sqrt{1 - q}
     = \frac{3\pi}{4}.$
\item Find all positive integral solution of the equation $\tan^{-1}x + \cot^{-1}y = \tan^{-1}3.$
\item Solve $\sin^{-1}\frac{ax}{c} + \sin^{-1}\frac{bx}{c} = \sin^{-1}x$ where $a^2 + b^2 = c^2, c\neq 0.$
\item Convert the trigonometric function $\sin[2\cos^{-1}\{\cot(2\tan^{-1}x)\}]$ into an algebraic function $f(x).$ Then
     from the algebraic function find all the values of $x$ for which $f(x)$ is zero. Express the value of $x$ in
     the form of $a\pm\sqrt{b}$ where $a$ and $b$ are rational numbers.
\item Solve the equation $\theta = \tan^{-1}(2\tan^2\theta) - \frac{1}{2}\sin^{-1}\left(\frac{3\sin2\theta}{5 +
     4\cos2\theta}\right).$
\item Solve $\tan^{-1}2x + \tan^{-1}3x = \frac{\pi}{4}.$
\item If $\sin^{-1}\left(x - \frac{x^2}{2} + \frac{x^3}{4} - \ldots\right) + \cos^{-1}\left(x^2 - \frac{x^4}{2} +
     \frac{x^6}{4} + \ldots\right) = \frac{\pi}{2}~\forall~0<|x|<\sqrt{2}$ then find $x.$
\item Find the number of real solutions for $\tan^{-1}\sqrt{x(x + 1)} + \sin^{-1}\sqrt{x^2 + x + 1} = \frac{\pi}{2}.$
\item Solve $\sin^{-1}\frac{3x}{5} + \cos^{-1}\frac{4x}{5} = \sin^{-1}x.$
\item Solve $\sin^{-1}(1 - x) - 2\sin^{-1}x = \frac{\pi}{2}.$
\item If $k$ be a positive integer, show that the equation $\tan^{-1}x + \tan^{-1}y = \tan^{-1}k$ has no positive
     integral solution.
\item Solve $\tan^{-1}\frac{x + 1}{x - 1} + \tan^{-1}\frac{x - 1}{x} = \tan^{-1}(-7).$
\item Solve $\tan^{-1}\frac{1}{a - 1} = \tan^{-1}\frac{1}{x} + \tan^{-1}\frac{1}{a^2 - x + 1}.$
\item Solve $\cos^{-1}\frac{x^2 - 1}{x^2 + 1} + \tan^{-1}\frac{2x}{x^2 - 1} = \frac{2\pi}{3}.$
\item If $\theta = \tan^{-1}\frac{x\sqrt{3}}{2k - x}$ and $\phi = \tan^{-1}\frac{2x - k}{k\sqrt{3}},$ show that one
     value of $\theta - \phi$ is $\pi/6.$
\item Find all positive integral solutions of the equation $\tan^{-1}x + \cos^{-1}\frac{y}{\sqrt{1 + y^2}} =
     \sin^{-1}\frac{3}{\sqrt{10}}$.

\item Solve the equation $2\cos^{-1}x = \sin^{-1}2x\sqrt{1 - x^2}$
\item Solve $\sin^{-1}\frac{x}{\sqrt{1 + x^2}} - \sin^{-1}\frac{1}{\sqrt{1 + x^2}} = \sin^{-1}\frac{1 + x}{1 + x^2}$
\item Show that the function $y = 2\tan^{-1}\left[\sqrt{\frac{a - b}{a + b}}\tan\frac{x}{2}\right] - \cos^{-1}\left[\frac{b +
     a\cos x}{a + b\cos x}\right]$ is a constant for $0 < b \leq a,$ find the value of this constant for $x\geq 0.$
\item Find the sum $\sum_{i = 1}^n\tan^{-1}\frac{2i}{2 + i^2 + i^4}.$
\item Find the sum of infinite terms of the series $\cot^{-1}\left(1^2 + \frac{3}{4}\right) + \cot^{-1}\left(2^2 +
     \frac{3}{4}\right) + \cot^{-1}\left(3^3 + \frac{3}{4}\right) +\ldots$
\item Solve for $x$ the equation $(\tan^{-1}x)^2 + (\cot^{-1}x)^2 = \frac{5\pi^2}{8}$
\item Show that the greatest and the least values of $(\sin^{-1}x)^3 + (\cos^{-1}x)^3$ are $\frac{7\pi^3}{8}$ and
     $\frac{\pi^2}{32}$ respectively.
\item Obtain the integral values of $p$ for which the following system of equations possesses real solution
     $\cos^{-1}x + (\sin^{-1}y)^2 = \frac{p\pi^2}{4}$ and $(\cos^{-1}x)(\sin^{-1}y)^2 = \frac{\pi^2}{16}.$
\item If $\tan^{-1}x, \tan^{-1}y, \tan^{-1}z$ be in A.P., find the algebraic relation between $x, y$ and $z$.
     If $x, y, z$ be in A.P. prove that $x = y = z$.
\item Show that for $x > 0, \tan^{-1}\frac{x}{1 + 1.2x^2} + \tan^{-1}\frac{x}{1 + 2.3x^2} + \ldots +
     \tan^{-1}\frac{x}{1 + n(n + 1)x^2} = \tan^{-1}\frac{nx}{1 + (n + 1)x^2}$
\stopitemize
