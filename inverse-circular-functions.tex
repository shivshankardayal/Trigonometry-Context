% -*- mode: context; -*-
\chapter{Inverse Circular Functions}
Inverse functions related to trigonometric ratios are called inverse trigonometric functions. The definition of different
inverse trigonometric functions is given below:

If $\sin\theta = x$, then $\theta = \sin^{-1}x$, provided $-1\leq x\leq 1$ and $-\frac{\pi}{2}\leq\theta\leq\frac{\pi}{2}$.

If $\cos\theta = x$, then $\theta = \cos^{-1}x$, provided $-1\leq x\leq 1$ and $0\leq\theta\leq\pi$.

If $\tan\theta = x$, then $\theta = \tan^{-1}x$, provided $-\infty<x<\infty$ and $-\frac{\pi}{2}<\theta<\frac{\pi}{2}$.

If $\cot\theta = x$, then $\theta = \cot^{-1}x$, provided $-\infty<x<\infty$ and $0<\theta<\pi$.

If $\sec\theta = x$, then $\theta = \sec^{-1}x$, provided $x\leq -1$ or $x\geq 1$ and $0\leq \theta\leq
\pi,\theta\neq\frac{\pi}{2}$.

If ${\rm cosec}\theta = x$, then $\theta = {\rm cosec}^{-1}x$, provided $x\leq -1$ or $x\geq 1$ and $-\frac{\pi}{2}\leq\theta\leq
\frac{\pi}{2}, \theta\neq0$.

{\bf Note:} In the above definition, restrictions on $\theta$ are due to the consideration of principal values of inverse
terms. If these restrictions are removed, the terms will represent inverse trigonometric relations and not functions.

{\bf Notations: I.} ${\rm Arc}\sin x$ denotes the sine inverse of $x$ [General value]. $\arcsin x$ denotes the principal value
of sine inverse of $x$.

{\bf II.} $\sin^{-1}x$ denotes the principal value of sine inverse $x$. From the above notations three important results follow;
\startitemize[n]
\item $\sin^{-1}x = \theta \Rightarrow \sin\theta = x$ and $\theta$ is the principal value.
\item $\sin^{-1}x = \arcsin x, \cos^{-1}x = \arccos x$.
\item From the definition of the inverse functions, we know that if $y = f(x)$ is a function then for $f^{-1}$ to be a function,
  $f$ must be one-to-one and onto mapping.
\stopitemize

When we consider $y = {\rm Arc}\sin x$, for any $x\in[-1, 1]$ infinite number of values of $y$ are obtained and hence it does not
represent inverse functions. When $y = \arcsin x$ or $\sin^{-1}x$, corresponding to one value of $x\in[-1, 1]$, one value of $y$ is
obtained and hence it represents the inverse trigonometric function.

Hence, for inverse trigonometric functions, consideration of principal values is essential.

\section{Principal Value}
Numerically smallest angle is known as the principal value.

Since inverse trigonometric terms are in fact angles, definitions of principal value of inverse trigonometric term is the same as
the definition of the principal values of angles.

Suppose we have to find the principal value of $\sin^{-1}\frac{1}{2}$. Let $\sin^{-1}\frac{1}{2} = \theta$, then $\sin\theta =
\frac{1}{2} \Rightarrow \theta = \ldots, -\frac{11\pi}{6}, -\frac{7\pi}{6}, \frac{\pi}{6}, \frac{5\pi}{6}, \ldots$. Among all these
angles $\frac{\pi}{6}$ is the numerically smalles angles satisfying $\sin\theta = \frac{1}{2}$ and hence it is the principal value.

\section{Important Formulae}
\starttheorem
  \startitemize[n]
  \item $\sin\sin^{-1}x = x, -1\leq x\leq 1$
  \item $\cos\cos^{-1}x = x, -1\leq x\leq 1$
  \item $\tan\tan^{-1}x = x, -\infty < x\leq\infty$
  \item $\cot\cot^{-1}x = x, -\infty < x\leq\infty$
  \item $\sec\sec^{-1}x = x, x\leq -1$ or $x\geq 1$
  \item ${\rm cosec\,cosec}^{-1}x = x, x\leq -1$ or $x\geq 1$
  \stopitemize
\stoptheorem

\startproof
Let $\sin^{-1}x = \theta$ then $\sin\theta = x$. Putting the value of $\theta$ from first equation in second
$\sin\sin^{-1}x = x$. Other formulae can be proved similarly.
\stopproof

\starttheorem
  \startitemize[n]
  \item $\sin^{-1}x + \cos^{-1}x = \frac{\pi}{2}\,\forall -1\leq x\leq 1$
  \item $\tan^{-1}x + \cot^{-1}x = \frac{\pi}{2}\,\forall x\in\mathbb{R}$
  \item $\sec^{-1}x + {\rm cosec}^{-1}x = \frac{\pi}{2}\,\forall x\leq -1$ or $x\geq 1$
  \stopitemize
\stoptheorem

\startproof
  Let $\sin^{-1}x = \theta$, then $\sin\theta = x \Rightarrow \cos\left(\frac{\pi}{2} - \theta\right) = x \Rightarrow \frac{\pi}{2}
  - \theta = \cos^{-1}x$

  $\Rightarrow \cos^{-1}x + \theta = \frac{\pi}{2} \Rightarrow \sin^{-1}x + \cos^{-1}x = \frac{\pi}{2}$.

  Similarly other results can be proven.
\stopproof

\starttheorem
  \startitemize[n]
  \item $\sin^{-1}x = {\rm cosec}^{-1}\frac{1}{x}, -1\leq x\leq 1$
  \item ${\rm cosec^{-1}x} = \sin^{-1}\frac{1}{x}, x\leq -1$ or $x > 1$
  \item $\cos^{-1}x = \sec^{-1}\frac{1}{x}, -1\leq x\leq 1$
  \item $\sec^{-1}x = \cos^{-1}\frac{1}{x}, x\leq -1$ or $x\geq 1$
  \stopitemize
\stoptheorem

\startproof
  Let $\sin^{-1}x = \theta$ then $\sin\theta = x \Rightarrow \rm cosec\theta = \frac{1}{x}$

  $\Rightarrow \theta = \rm cosec^{-1}\frac{1}{x} \Rightarrow \sin^{-1}x = \rm cosec^{-1}\frac{1}{x}$

  Other results can be proven similarly.
\stopproof

\starttheorem
  \startitemize[n]
    \item $\sin^{-1}x = \cos^{-1}\sqrt{1 - x^2}, \,\forall\,0\leq x\leq 1$

    \item $\sin^{-1}x = -\cos^{-1}\sqrt{1 - x^2}\,\forall\,-1\leq x< 0$
  \stopitemize
\stoptheorem

\startproof
  Let $\sin^{-1}x = \theta$ then $\sin\theta = x$

  $\Rightarrow \cos^2\theta = 1 - x^2 \Rightarrow \cos\theta = \pm\sqrt{1 - x^2}$

  Principal values of $\sin^{-1}x$ lies between $-\frac{\pi}{2}$ and $\frac{\pi}{2}$.

  In this interval $\cos\theta$ is $+$ve.

  $\Rightarrow \sin^{-1}x = \cos^{-1}\sqrt{1 - x^2}$

  For $-1\leq x < 0$ $\sin^{-1}x$ will be negative angle while $\cos^{-1}\sqrt{1 - x^2}$ will be positive
  angle. Hence to balance that we need to use a negative sign for this.
\stopproof