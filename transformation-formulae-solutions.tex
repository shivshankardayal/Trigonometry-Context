% -*- mode: context; -*-
\chapter{Transformation Formulae}
\startitemize[n, 1*broad]
\item Given, $\frac{\sin 75^\circ - \sin 15^\circ}{\cos 75^\circ + \cos 15^\circ}$

  $= \frac{2\cos \frac{75^\circ + 15^\circ}{2}\sin \frac{75^\circ - 15^\circ}{2}}{2\cos \frac{75^\circ + 15^\circ}{2}\cos
  \frac{75^\circ - 15^\circ}{2}}$

  $= \frac{2\cos 45^\circ\sin30^\circ}{2\cos45^\circ\cos30^\circ} = \tan30^\circ = \frac{1}{\sqrt{3}}$

\item Given, $\frac{(\cos \theta - \cos 2\theta)(\sin 8\theta + \sin 2\theta)}{(\sin 5\theta - \sin\theta)(\cos
  4\theta - \cos 6\theta)} = \frac{2\sin2\theta\sin\theta.2\sin5\theta\cos3\theta}{2\cos3\theta\sin2\theta.2\sin5\theta\sin\theta}$

  $= 1$

\item We have to prove, $\frac{\sin7\theta - \sin5\theta}{\cos7\theta + \cos5\theta} = \tan\theta$

  L.H.S. $= \frac{2\cos6\theta\sin\theta}{2\cos6\theta\cos\theta} = \tan\theta$

\item We have to prove, $\frac{\cos6\theta - \cos4\theta}{\sin6\theta + \sin4\theta} = -\tan\theta$

  L.H.S. $= \frac{2\sin5\theta\sin(-\theta)}{2\sin5\theta\cos\theta} = -\tan\theta$

\item We have to prove, $\frac{\sin A + \sin 3A}{\cos A + \cos 3A} = \tan 2A$

  L.H.S. $= \frac{2\sin2A\cos(-A)}{2\cos2A\cos(-A)} = \tan 2A$

\item We have to prove, $\frac{\sin 7A - \sin A}{\sin 8A - \sin 2A} = \cos 4A\sec 5A$

  L.H.S. $= \frac{2\cos4A\sin3A}{2\cos5A\sin3A} = \cos4A\sec5A$

\item We have to prove, $\frac{\cos 2B + \cos 2A}{\cos 2B - \cos 2A} = \cot(A + B)\cot(A - B)$

  L.H.S. $= \frac{2\cos(A + B)\cos(A - B)}{2\sin(A + B)\sin(A - B)} = \cot(A + B)\cot(A - B)$

\item We have to prove, $\frac{\sin 2A + \sin 2B}{\sin 2A - \sin 2B} = \frac{\tan(A + B)}{\tan(A - B)}$

  L.H.S. $= \frac{2\sin(A + B)\cos(A - B)}{2\cos(A + B)\sin(A - B)}$

  $= \frac{\tan(A + B)}{\tan(A - B)}$

\item We have to prove, $\frac{\sin A + \sin 2A}{\cos A - \cos 2A} = \cot \frac{A}{2}$

  L.H.S. $= \frac{2\sin\frac{3A}{2}\cos\frac{A}{2}}{2\sin\frac{3A}{2}\sin\frac{A}{2}}$

  $= \cot\frac{A}{2}$

\item We have to prove, $\frac{\sin 5A - \sin 3A}{\cos 3A + \cos 5A} = \tan A$

  L.H.S. $= \frac{2\cos4A\sin A}{2\cos4A\cos A} = \tan A$

\item We have to prove, $\frac{\cos 2B - \cos 2A}{\sin 2B + \sin 2A} = \tan(A - B)$

  L.H.S. $= \frac{2\sin(A + B)\sin(A - B)}{2\sin(A + B)\cos(A - B)} = \tan(A - B)$

\item We have to prove, $\cos (A + B) + \sin(A - B) = 2\sin(45^\circ + A)\cos(45^\circ + B)$

  L.H.S. $= \cos A\cos B - \sin A\sin B + \sin A\cos B - \cos A\sin B$

  $= (\sin A + \cos A)(\cos B - \sin B)$

  $= 2(\frac{1}{\sqrt{2}}\sin A + \frac{1}{\sqrt{2}}\cos A)(\frac{1}{\sqrt{2}}\cos B - \frac{1}{\sqrt{2}}\sin B)$

  $= 2(\sin A\cos 45^\circ + \sin 45^\circ\sin A)(\cos 45^\circ\cos B - \sin 45^\circ\sin B)$

  $= 2\sin(45^\circ + A)\cos(45^\circ + B)$

\item We have to prove, $\frac{\cos 3A - \cos A}{\sin 3A - \sin A} + \frac{\cos 2A - \cos 4A}{\sin 4A - \sin 2A} = \frac{\sin A}{\cos
  2A\cos 3A}$

  L.H.S. $= \frac{-2\sin 2A\sin A}{2\cos 2A\sin A} + \frac{2\sin 3A\sin A}{2\cos 3A\sin A}$

  $= \frac{-\sin 2A}{\cos 2A} + \frac{\sin 3A}{\cos 3A}$

  $= \frac{\cos 2A\sin 3A - \sin 2A\cos 3A}{\cos 2A\cos 3A} = \frac{\sin(3A - 2A)}{\cos 2A\cos 3A}$

  $= \frac{\sin A}{\cos 3A\cos 3A}$

\item Given, $\frac{\sin (4A - 2B) + \sin (4B - 2A)}{\cos (4A - 2B) + \cos (4B - 2A)} = \tan(A + B)$
  L.H.S. $= \frac{2\sin(A + B)\cos3(A - B)}{2\cos(A + B)\cos3(A - B)} = \tan(A + B)$

\item We have to prove, $\frac{\tan 5\theta + \tan 3\theta}{\tan 5\theta - \tan 3\theta} = 4\cos 2\theta\cos 4\theta$

  L.H.S. $= \frac{\frac{\sin5\theta}{\cos5\theta} + \frac{\sin4\theta}{\cos3\theta}}{\frac{\sin5\theta}{\cos5\theta} -
    \frac{\sin4\theta}{\cos3\theta}}$

  $= \frac{\sin5\theta\cos3\theta + \sin3\theta\cos5\theta}{\sin5\theta\cos3\theta - \sin3\theta\cos5\theta}$

  $= \frac{\sin8\theta}{\sin2\theta} = \frac{2\sin4\theta\cos\theta}{\sin2\theta}$

  $= \frac{4\sin2\theta\cos2\theta\cos4\theta}{\sin2\theta} = 4\cos2\theta\cos4\theta$

\item We have to prove, $\frac{\cos 3\theta + 2\cos5\theta + \cos 7\theta}{\cos\theta + 2\cos3\theta + \cos 5\theta} = \cos 2\theta - \sin
  2\theta\tan 3\theta$

  Adding first and last terms of numerator and denominator, we have

  L.H.S. $= \frac{2\cos5\theta\cos2\theta + 2\cos5\theta}{2\cos3\theta\cos2\theta + 2\cos3\theta}$

  $= \frac{\cos5\theta(\cos2\theta + 1)}{\cos3\theta(\cos2\theta + 1)}$

  $= \frac{\cos3\theta\cos2\theta - \sin3\theta\sin2\theta}{\cos3\theta}$

  $= \cos2\theta - \sin2\theta\tan3\theta$

\item We have to prove, $\frac{\sin A + \sin 3A + \sin 5A + \sin 7A}{\cos A + \cos 3A + \cos 5A + \cos 7A} = \tan 4A$

  Pairing first and fourth term and second and third term in numerator and denominator, we get

  L.H.S. $= \frac{2\sin4A\cos3A + 2\sin4A\cos A}{2\cos4A\cos3A + 2\cos4A\cos A}$

  $= \frac{2\sin4A(\cos 3A + \cos A)}{2\cos4A(\cos 3A + \cos A)}$

  $= \tan 4A$

\item We have to prove, $\frac{\sin (\theta + \phi) - 2\sin\theta + \sin (\theta - \phi)}{\cos (\theta + \phi) - 2\cos \theta +
  \cos(\theta - \phi)} = \tan\theta$

  Pairing first and last term in both numerator and denominator, we get

  L.H.S. $= \frac{2\sin\theta\cos\phi + 2\sin\theta}{2\cos\theta\cos\phi + 2\cos\theta}$

  $= \frac{2\sin\theta(\cos\phi + 1)}{2\cos\theta(\cos\phi + 1)}$

  $= \tan\theta$

\item We have to prove that, $\frac{\sin A + 2\sin 3A + \sin 5A}{\sin 3A + 2\sin 5A + \sin 7A} = \frac{\sin 3A}{\sin 5A}$

  Pairing first and last term in both numerator and denominator, we get

  L.H.S. $= \frac{2\sin3A\cos2A + 2\sin3A}{2\sin5A\cos2A + 2\sin5A}$

  $= \frac{\sin3A(\cos 2A + 1)}{\sin5A(\cos 2A + 1)}$

  $= \frac{\sin 3A}{\sin 5A}$

\item We have to prove that, $\frac{\sin(A - C) + 2\sin A + \sin(A + C)}{\sin (B - C) + 2\sin B + \sin(B + C)} = \frac{\sin
  A}{\sin B}$
  Pairing first and last term in both numerator and denominator, we get

  L.H.S. $= \frac{2\sin A\cos C + 2\sin A}{2\sin B\cos C + 2\sin B}$

  $= \frac{\sin A(\cos C + 1)}{\sin B(\cos C + 1)}$

  $= \frac{\sin A}{\sin B}$

\item We have to prove that, $\frac{\sin A - \sin 5A + \sin 9A - \sin 13A}{\cos A - \cos 5A - \cos 9A + \cos 13 A} = \cot 4A$

  Pairing first and last term and second and third term in both numerator and denominator, we get

  L.H.S. $= \frac{-2\cos7A\sin6A + 2\cos7A\sin 2A}{3\cos7A\cos6A - 2\cos7A\cos2A}$

  $= \frac{2\cos7A(\sin 2A - \sin 6A)}{2\cos 7A(\cos 6A - \cos 2A)}$

  $= \frac{-2\cos 4A\sin 2A}{-2\sin 4A\sin 2A}$

  $= \cot 4A$

\item We have to prove that, $\frac{\sin A + \sin B}{\sin A - \sin B} = \tan \frac{A + B}{2}\cot \frac{A - B}{2}$

  L.H.S. $= \frac{2\sin\frac{A + B}{2}\cos\frac{A - B }{2}}{2\cos\frac{A + B}{2}\sin\frac{A - B}{2}}$

  $= \tan \frac{A + B}{2}\cot \frac{A - B}{2}$

\item We have to prove that, $\frac{\cos A + \cos B}{\cos B - \cos A} = \cot \frac{A + B}{2}\cot \frac{A - B}{2}$

  L.H.S. $= \frac{2\cos\frac{A + B}{2}\cos\frac{A - B}{2}}{2\sin\frac{A + B}{2}\sin\frac{A - B}{2}}$

  $= \cot\frac{A + B}{2}\cot\frac{A - B}{2}$

\item We have to prove that, $\frac{\sin A + \sin B}{\cos A + \cos B} = \tan \frac{A + B}{2}$

  L.H.S. $= \frac{2\sin\frac{A + B}{2}\cos\frac{A - B}{2}}{2\cos\frac{A + B}{2}\cos\frac{A - B}{2}}$

  $= \tan \frac{A + B}{2}$

\item We have to prove that, $\frac{\sin A - \sin B}{\cos B - \cos A} = \cot \frac{A + B}{2}$

  L.H.S. $= \frac{2\cos\frac{A + B}{2}\sin\frac{A - B}{3}}{2\sin\frac{A + B}{2}\sin\frac{A - B}{2}}$

  $= \cot \frac{A + B}{2}$

\item We have to prove that, $\frac{\cos(A + B + C) + \cos(-A + B + C) + \cos(A - B + C) + \cos(A + B - C)}{\sin(A + B +
  C)+\sin(-A + B + C) - \sin(A - B + C) + \sin(A + B - C)} = \cot B$

  L.H.S. $= \frac{2\cos(B + C)\cos A + 2\cos A\cos(B - C)}{2\sin(B + C)\cos A + 2\sin(B - C)\cos A}$

  $= \frac{\cos(B + C) + \cos(B - C)}{\sin(B + C) + \sin(B - C)}$

  $= \frac{2\cos B\cos C}{2\sin B\cos C} = \cot B$

\item We have to prove that, $\cos 3A + \cos 5A + \cos 7A + \cos 15A = 4 \cos 4A\cos 5A \cos 6A$

  Adding first and last and two middle terms together, we gte

  L.H.S. $= 2\cos9A\cos6A + 2\cos6A\cos A$

  $= 2\cos6A(\cos9A + \cos A)$

  $= 4\cos 4A \cos 5A \cos 6A$

\item We have to prove that, $\cos(-A + B + C) + \cos(A - B + C) + \cos(A + B - C) + \cos(A + B + C) = 4\cos A\cos B\cos C$

  Adding first two and last two, we get

  L.H.S. $= 2\cos C\cos (B - A) + 2\cos (A + B)\cos C$

  $= 2\cos C(\cos (B - A) + \cos (A + B))$

  $= 4\cos A\cos B\cos C$

\item We have to prove that, $\sin 50^\circ - \sin 70^\circ + \sin 10^\circ = 0$

  L.H.S. $= -2\cos 60^\circ \sin 10^\circ + \sin 10^\circ$

  $= -\sin10^\circ + \sin10^\circ = 0$

\item We have to prove that, $\sin 10^\circ + \sin 20^\circ + \sin 40^\circ + \sin 50^\circ = \sin 70^\circ + \sin 80^\circ$

  L.H.S. $= \sin 10^\circ + \sin 50^\circ + \sin 20^\circ + \sin 40^\circ$

  $= 2\sin 30^\circ \cos 20^\circ + 2\sin 30^\circ \cos 10^\circ$

  $= 2\sin30^\circ(\cos 20^\circ + \cos 10^\circ)$

  $= \sin 70^\circ + \sin 80^\circ[\because \cos\theta = \sin(90^\circ - \theta)]$

\item We have to prove that, $\sin\alpha + \sin 2\alpha + \sin 4\alpha + \sin 5\alpha = 4\cos \frac{\alpha}{2}\cos
  \frac{3\alpha}{2}\sin 3\alpha$

  L.H.S. $= \sin\alpha + \sin 5\alpha + \sin2\alpha + \sin4\alpha$

  $= 2\sin3\alpha\cos2\alpha + 2\sin3\alpha\cos\alpha$

  $= 2\sin3\alpha(\cos2\alpha + \cos\alpha)$

  $= 4\cos \frac{\alpha}{2}\cos \frac{3\alpha}{2}\sin 3\alpha$

\item Given, $\cos\left[\theta + \left(n - \frac{3}{2}\right)\phi\right] - \cos\left[\theta + \left(n +
  \frac{3}{2}\right)\phi\right]$

  $= 2\sin\left[\theta + n\phi\right]\sin\left[\frac{\phi}{2}\right]$

\item Given, $\sin\left[\theta + \left(n - \frac{3}{2}\right)\phi\right] + \sin\left[\theta + \left(n +
  \frac{3}{2}\right)\phi\right]$

  $= 2\sin\left[\theta + n\phi\right]\cos\left[\frac{\phi}{2}\right]$

\item Given, $2\sin5\theta\sin7\theta$

  Let the angles are $A$ and $B$ then $\cos C - \cos D = 2\sin \frac{C + D}{2}\cos\frac{D - C}{2}$

  Thus, comparing, we get

  $C + D = 14, D - C = 10$

  $D = 12, C = 2$

  Thus, required expression is $\cos 2\theta - \cos12\theta$

\item Given, $2\cos7\theta\sin5\theta$

  $= \sin 12\theta + \sin 2\theta$

\item Given, $2\cos 11\theta\cos 3\theta$

  $= \cos 14\theta + \cos 8\theta$

\item Given, $2\sin54^\circ\sin66^\circ$

  $=\cos12^\circ - \cos120^\circ$

\item We have to prove that $\sin\frac{\theta}{2}\sin\frac{7\theta}{2} + \sin \frac{3\theta}{2}\sin\frac{11\theta}{2} =\sin
  2\theta\sin 5\theta$

  L.H.S. $= \frac{1}{2}(\cos 3\theta - \cos 4\theta) + \frac{1}{2}(\cos 4\theta - \cos 7\theta)$

  $= \sin2\theta + \sin5\theta$

\item We have to prove that, $\cos 2\theta\cos \frac{\theta}{2} -\cos3\theta\cos\frac{9\theta}{2} =
  \sin5\theta\sin\frac{5\theta}{2}$

  L.H.S. $= \frac{1}{2}\left(\cos\frac{5\theta}{2} + \cos \frac{3\theta}{2}\right) - \frac{1}{2}\left(\cos
  \frac{15\theta}{2} + \cos \frac{3\theta}{2}\right)$

  $= \frac{1}{2}\left(2\sin 5\theta \sin \frac{5\theta}{2}\right)$

  $= \sin5\theta\sin\frac{5\theta}{2}$

\item We have to prove that, $\sin A\sin(A + 2B) - \sin B\sin(B + 2A) = \sin(A - B)\sin(A + B)$

  L.H.S. $= \frac{1}{2}(2\sin A\sin(A + 2B) - 2\sin B\sin(B + 2A))$

  $= \frac{1}{2}(\cos B - \cos(A + B) - \cos A - \cos (A + B))$

  $= \frac{1}{2}2\sin(A - B)\sin(A + B)$

\item We have to prove that, $(\sin 3A + \sin A)\sin A + (\cos 3A - \cos A)\cos A = 0$

  L.H.S. $= 2\sin2A\cos A\sin A - 2\sin2A\sin A\cos A = 0$

\item We have to prove that, $\frac{2\sin(A - C)\cos C - \sin(A - 2C)}{2\sin(B - C)\cos C - \sin(B - 2C)} = \frac{\sin A}{\sin
  B}$

  L.H.S. $= \frac{\sin A + \sin(A - 2C) - \sin(A - 2C)}{\sin B + \sin(B - 2C) - \sin(B - 2C)}$

  $= \frac{\sin A}{\sin B}$

\item We have to prove that, $\frac{\sin A\sin 2A + \sin 3A\sin 6A + \sin4A\sin 13A}{\sin A\cos2A + \sin 3A\cos 6A + \sin
  4A\cos 13A} = \tan 9A$

  L.H.S. $= \frac{\cos A - \cos 3A + \cos 3A - \cos 9A + \cos 9A - \cos 17A}{\sin 3A - \sin A + \sin 9A - \sin 3A + \sin 17A
    - \sin 9A}$

  $= \frac{\cos A - \cos 17A}{\sin 17A - \sin A}$

  $= \frac{2\sin 8A\sin 9A}{2\cos 9A\sin 8A} = \tan 9A$

\item We have to prove that, $\frac{\cos 2A\cos 3A - \cos 2A\cos 7A + \cos A\cos 10A}{\sin 4A\sin 3A - \sin 2A\sin 5A + \sin
  4A\sin 7A} =\cot 6A\cot 5A$

  L.H.S. $= \frac{\cos 5A + \cos A - \cos 9A - \cos 5A + \cos 11A + \cos 9A}{\cos A - \cos 7A - \cos 3A + cos 7A + \cos
    3A - \cos 11A}$

  $= \frac{\cos A + \cos 11A}{\cos A - \cos 11A}$

  $= \frac{\cos 6A\cos 5A}{\sin 6A\sin 5A} = \cot 6A\cot 5A$

\item We have to prove that, $\cos(36^\circ - A)\cos(36^\circ + A) + \cos(54^\circ + A)\cos(54^\circ - A) = \cos 2A$

  L.H.S. $= \frac{1}{2}(\cos (72^\circ)+ \cos 2A) + \frac{1}{2}(\cos 108^\circ + \cos 2A)$

  $= \frac{1}{2}(\sin 18^\circ + - \sin 18^\circ + 2\cos 2A)[\because \sin 18^\circ = \cos(90^\circ - 18^\circ) = \cos
    72^\circ$ and $\cos 108^\circ = \cos(90^\circ + 18^\circ) = -\sin 18^\circ]$

  $= \cos 2A$

\item We have to prove that $\cos A\sin(B - C) + \cos B\sin(C - A) + \cos C\sin(A - B) = 0$

  L.H.S. $= \frac{1}{2}[\sin(A + B - C) - \sin(A - B + C) + \sin(B + C - A) - \sin (B - C + A) + \sin(A - B + C) - \sin(C -
    A + B)] = 0$

\item $\sin(45^\circ + A)\sin(45^\circ - A) = \frac{1}{2}\cos 2A$

  L.H.S. $= \frac{1}{2}(2\sin(45^\circ + A)\sin(45^\circ - A)) = \frac{1}{2}[\cos 2A - \cos 90^\circ] = \frac{1}{2}\cos 2A$

\item We have to prove that, $\sin(\beta - \gamma)\cos(\alpha - \delta) + \sin(\gamma - \alpha)\cos(\beta - \delta) +
  \sin(\alpha - \beta)\cos(\gamma - \delta) = 0$

  L.H.S. $= \frac{1}{2}[\sin(\alpha + \beta - \gamma - \delta) + \sin(\beta + \delta - \gamma - \alpha) + \sin(\gamma
    -\alpha + \beta - \delta) + \sin(\gamma - \alpha - \beta + \delta) + \sin(\alpha - \beta + \gamma - \delta) - \sin(\alpha -
    \beta - \gamma + \delta)]$

  $= 0$

\item We have to prove that, $2\cos\frac{\pi}{13}\cos \frac{9\pi}{13} + \cos \frac{3\pi}{13} + \cos \frac{5\pi}{13} = 0$

  L.H.S. $=\cos \frac{10\pi}{13} + \cos \frac{8\pi}{13} + \cos \left(\pi - \frac{10\pi}{13}\right) + \cos \left(\pi -
  \frac{8\pi}{13}\right)$

  $= \cos \frac{10\pi}{13} + \cos \frac{8\pi}{13} - \cos \frac{10\pi}{13} - \cos \frac{8\pi}{13} = 0$

\item We have to prove that, $\cos 55^\circ + \cos65^\circ + \cos 175^\circ = 0$

  L.H.S. $= 2\cos 60^\circ\cos5^\circ + \cos(180^\circ - 5^\circ)$

  $= 2.\frac{1}{2}.\cos5^\circ - \cos 5^\circ = 0$

\item We have to prove that, $\cos 18^\circ -\sin 18^\circ = \sqrt{2}\sin 27^\circ$

  L.H.S. $= \cos18^\circ - \cos(90^\circ - 72^\circ) = \cos 18^\circ - \cos 72^\circ$

  $= 2\sin 45^\circ\sin 27^\circ = \sqrt{2}\sin27^\circ$

\item We have to prove that, $\frac{\sin A + \sin 2A + \sin 4A + \sin 5A}{\cos A + \cos 2A + \cos 4A + \cos 5A} = \tan 3A$

  L.H.S. $= \frac{(\sin 5A + \sin A) + (\sin 4A + \sin 2A)}{(\cos 5A + \cos A) + (\cos 4A + \cos 2A)}$

  $= \frac{2\sin 3A\cos 2A + 2\sin 3A\cos A}{2\cos3A\cos2A + 2\cos 3A\cos A}$

  $= \frac{\sin 3A(\cos 2A + \cos A)}{\cos 3A(\cos 2A + \cos A)} = \tan 3A$

\item L.H.S $= \left(\frac{2\cos \frac{A + B}{2}\cos \frac{A - B}{2}}{2\cos\frac{A + B}{2}\sin\frac{A - B}{2}}\right)^n +
  \left(\frac{2\sin\frac{A + B}{2}\cos\frac{A - B}{2}}{2\sin\frac{A + B}{2}\sin\frac{B - A}{2}}\right)^n$

  $= \left(\cos\frac{A - B}{2}\right)^n + \left(-\cot\frac{A - B}{2}\right)^n$

  $=\cot^n\frac{A - B}{2}[1 + (-1)^n]$ which is $0$ if $n$ is odd and $2\cos^n\frac{A - B}{2}$ if
  $n$ is even.

\item Given $\alpha, \beta, \gamma$ are in A.P. $\therefore 2\beta = \alpha + \gamma$

  R.H.S. $= \frac{\sin\alpha - \sin\gamma}{\cos\gamma - \cos\alpha} = \frac{2\cos \frac{\alpha + \gamma}{2}\sin\frac{\alpha
      - \gamma}{2}}{2\sin\frac{\gamma + \alpha}{2}\sin\frac{\alpha - \gamma}{2}}$

  $= \cot\frac{\alpha + \gamma}{2} = \cot\beta$

\item Given $\sin\theta + \sin\phi = \sqrt{3}(\cos\phi - \cos\theta)$

  $2\sin\frac{\theta + \phi}{2}\cos\frac{\theta - \phi}{2} = \sqrt{3}.2.\sin\frac{\theta + \phi}{2}\sin\frac{\theta -
  \phi}{2}$

  $= \sin\frac{\theta + \phi}{2}\left[\cos\frac{\theta - \phi}{2} - \sqrt{3}\sin\frac{\theta - \phi}{}\right] = 0$

  $\therefore \sin\frac{\theta + \phi}{2} = 0$ or $\tan\frac{\theta - \phi}{2} = \frac{1}{\sqrt{3}}$

  $\therefore \theta + \phi = 0^\circ$ or $\theta - \phi = 60^\circ$

  Now, $\sin3\theta + \sin3\phi = 2\sin\frac{3(\theta + \phi)}{2}\sin\frac{3(\theta - \phi)}{2} = 0$

  [$\because$ when $\theta + \phi = 0; \sin\frac{3(\theta + \phi)}{2} = 0$ and when $\theta - \phi = 60^\circ;
    \cos\frac{3(\theta - \phi)}{2} = 0$ ]

\item We have to prove that $\sin 65^\circ + cos 65^\circ = \sqrt{2}\cos 20^\circ$

  L.H.S. $= \cos(90^\circ - 65^\circ) + \cos 65^\circ = \cos 25^\circ\cos65^\circ$

  $= 2\cos 45^\circ\cos20^\circ = 2.\frac{1}{\sqrt{2}}\cos20^\circ = \sqrt{2}\cos20^\circ$

\item We have to prove that $\sin 47^\circ + \cos 77^\circ = \cos 17^\circ$

  L.H.S. $= \cos(90^\circ - 47^\circ) + \cos 77^\circ$

  $= \cos 43^\circ + \cos77^\circ$

  $=2\cos 60^\circ \cos 17^\circ$

  $= \cos 17^\circ[\because 2\cos60^\circ = 2.\frac{1}{2} = 1]$

\item We have to prove that, $\frac{\cos 10^\circ - \sin 10^\circ}{\cos 10^\circ + \sin 10^\circ} = \tan 35^\circ$

  L.H.S. $= \frac{\cos10^\circ - \cos(90^\circ - 10^\circ)}{\cos10^\circ + \cos(90^\circ - 10^\circ)}$

  $= \frac{\cos10^\circ - \cos80^\circ}{\cos10^\circ + \cos80^\circ}$

  $= \frac{2\cos45^\circ\cos35^\circ}{2\cos45^\circ\cos35^\circ} = \tan 35^\circ$

\item We have to prove that, $\cos 80^\circ + \cos 40^\circ - cos 20^\circ = 0$

  L.H.S. $= 2\cos60^\circ\cos20^\circ - \cos20^\circ$

  $= \cos20^\circ - \cos20^\circ[\because 2\cos60^\circ = 2.\frac{1}{2} = 1]$

\item We have to prove that $\cos\frac{\pi}{5} + \cos \frac{2\pi}{5} + \cos\frac{6\pi}{5} + \cos \frac{7\pi}{5} = 0$

  L.H.S. $= \cos\frac{7\pi}{5} + \cos\frac{\pi}{5} + \cos\frac{2\pi}{5} + \cos\frac{6\pi}{5}$

  $= 2\cos\frac{4\pi}{5}\cos\frac{3\pi}{5} + 2\cos\frac{4\pi}{5}\cos\frac{2\pi}{5}$

  $= 2\cos\frac{4\pi}{5}\left[\cos\frac{3\pi}{5} + \cos \left(\pi - \frac{3\pi}{5}\right)\right]$

  $= 2\cos\frac{4\pi}{5}\left(\cos\frac{3\pi}{5} - \cos \frac{3\pi}{5}\right) = 0$

\item We have to prove that $\cos\alpha + \cos\beta + \cos\gamma + \cos(\alpha + \beta + \gamma) = 4\cos\frac{\alpha +
  \beta}{2}\cos\frac{\beta + \gamma}{2}\cos \frac{\gamma + \alpha}{2}$

  L.H.S. $= \cos(\alpha + \beta + \gamma) + \cos\alpha + \cos\beta + \cos\gamma$

  $= 2\cos\left(\alpha + \frac{\beta + \gamma}{2}\right)\cos\frac{\beta + \gamma}{2} +
  2\cos\frac{\beta+\gamma}{2}\cos\frac{\beta - \gamma}{2}$

  $= 2\cos\frac{\beta + \gamma}{2}\left[2\cos\left(\alpha +\frac{\beta + \gamma}{2}\right) + \cos\frac{\beta -
      \gamma}{2}\right]$

  $=4\cos\frac{\alpha + \beta}{2}\cos\frac{\beta + \gamma}{2}\cos \frac{\gamma + \alpha}{2}$

\item Given, $\sin\alpha - \sin\beta = \frac{1}{3}$ and $\cos\beta - \cos\alpha = \frac{1}{2}$

  Dividing we get, $\frac{\sin\alpha - \sin\beta}{\cos\beta - \cos\alpha} = \frac{2}{3}$

  $\Rightarrow \frac{2\cos\frac{\alpha + \beta}{2}\sin\frac{\alpha - \beta}{2}}{2\sin\frac{\alpha +
      \beta}{2}\sin\frac{\alpha - \beta}{2}} = \frac{2}{3}$

  $\Rightarrow \cot\frac{\alpha + \beta}{2} = \frac{2}{3}$

\item Given, $\csc A + sec A = \csc B + \sec B$

  $\sec A - \sec B = \csc B - \csc A$

  $\Rightarrow \frac{\cos B - \cos A}{\cos A\cos B} = \frac{\sin A - \sin B}{\sin A\sin B}$

  $\Rightarrow \frac{2\sin\frac{A + B}{2}\sin \frac{A - B}{2}}{\cos A\cos B} = \frac{2\cos\frac{A + B}{2}\sin\frac{A -
    B}{2}}{\sin A\sin B}$

  $\Rightarrow \tan A\tan B = \cot \frac{A + B}{2}$

\item Given, $\sec(\theta + \alpha) + \sec(\theta - \alpha) = 2\sec\theta$

  $\Rightarrow \frac{1}{\cos(\theta + \alpha)} + \frac{1}{\cos(\theta - \alpha)} = \frac{2}{\cos\theta}$

  $\cos\theta[\cos(\theta - \alpha) + \cos(\theta + \alpha)] = 2\cos(\theta + \alpha)\cos(\theta - \alpha)$

  $\cos\theta.2\cos\theta\cos\alpha = \cos2\theta + \cos2\alpha$

  We know that $[\cos(\theta + \theta) = \cos\theta.\cos\theta - \sin\theta\sin\theta = 2\cos^2\theta - 1]$

  Thus, the above equation becomes

  $2\cos^2\theta\cos\alpha = 2\cos^2\theta - 1 + 2\cos^2\alpha - 1$

  $2\cos^2\theta(\cos\alpha - 1) = 2(\cos^2\alpha - 1)$

  $\Rightarrow \cos^2\theta = 1 + \cos\alpha$

\item We have to prove that $\sin50^\circ\cos85^\circ = \frac{1 - \sqrt{2}\sin 35^\circ}{2\sqrt{2}}$

  L.H.S. $= \frac{1}{2}[\sin(85^\circ + \sin50^\circ) - \sin(85^\circ - 50^\circ)]$

  $= \frac{1}{2}[\sin135^\circ - \sin35^\circ]$

  $= \frac{1}{2}\left[\frac{1}{\sqrt{2}} - \sin35^\circ\right]$

  $= \frac{1 - \sqrt{2}\sin 35^\circ}{2\sqrt{2}}$

\item We have to prove that, $\sin 20^\circ \sin 40^\circ\sin 80^\circ = \frac{\sqrt{3}}{8}$

  L.H.S. $= \frac{1}{2}(2\sin80^\circ\sin40^\circ)\sin20^\circ$

  $= \frac{1}{2}[\cos(80^\circ - 40^\circ) - \cos(80^\circ + 40^\circ)]\sin20^\circ$

  $= \frac{1}{2}(\cos40^\circ - \cos120^\circ)\sin 206\circ$

  $= \frac{1}{4}(2\cos40^\circ\sin20^\circ - 2\cos120^\circ\sin20^\circ)$

  $= \frac{1}{4}[\sin(40^\circ + 20^\circ) - \sin(40^\circ - 20^\circ) - 2.-\frac{1}{2}\sin20^\circ]$

  $= \frac{1}{4}[\sin60^\circ - \sin20^\circ + \sin20^\circ] = \frac{1}{4}\sin60^\circ = \frac{\sqrt{3}}{8}$

\item We have to prove that, $\sin A\sin(60^\circ - A)\sin(60^\circ + A) = \frac{1}{4}\sin 3A$

  L.H.S. $= \frac{1}{2}\sin A[2\sin(60^\circ - A)\sin(60^\circ + A)]$

  $= \frac{1}{2}\sin A[\cos(60^\circ + A - 60^\circ + A) - \cos(60^\circ + A + 60^\circ - A)]$

  $= \frac{1}{2}\sin A(\cos 2A - \cos 120^\circ)$

  $= \frac{1}{4}(2\sin A\cos 2A - 2\cos 120^\circ \sin A)$

  $\frac{1}{4}[\sin (2A + A) - \sin(2A - A) - 2.-\frac{1}{2}\sin A]$

  $= \frac{1}{4}(\sin 3A - \sin A + \sin A) = \frac{1}{4}\sin 3A$

\item Let $p = \sin\alpha\sin\beta$

  $= \frac{1}{2}2\sin\alpha\sin\beta$

  $= \frac{1}{2}[\cos(\alpha - \beta) - \cos(\alpha + beta)]$

  $= \frac{1}{2}[\cos(\alpha - \beta) - \cos90^\circ][\because \alpha + \beta = 90^\circ~\text{(given)}]$

  $= \frac{1}{2}\cos(\alpha - \beta)$

  Maximum value of $\cos(\alpha - \beta)$ is $1,$ hence maximum value of $p$ is $\frac{1}{2}$

\item We have to prove that, $\sin 25^\circ\cos 115^\circ = \frac{1}{2}(\sin 40^\circ - 1)$

  L.H.S. $= \sin 25^\circ\cos 115^\circ$

  $= \frac{1}{2}2\sin25^\circ\cos115^\circ = \frac{1}{2}[\sin140^\circ - \sin90^\circ]$

  $= \frac{1}{2}[\cos50^\circ - 1] = \frac{1}{2}[\sin40^\circ - 1]$

\item We have to prove that, $\sin 20^\circ \sin 40^\circ\sin 60^\circ \sin80^\circ = \frac{3}{16}$

  L.H.S. $= \frac{1}{2}[(2\sin 20^\circ\sin80^\circ)(\sin40^\circ\sin60^\circ)]$

  $= \frac{1}{2}(\cos60^\circ - \cos100^\circ)\frac{\sqrt{3}}{2}\sin40^\circ$

  $= \frac{\sqrt{3}}{4}[\cos60^\circ\sin40^\circ + \sin10^\circ\sin40^\circ]$

  $= \frac{\sqrt{3}}{8}[2\cos60^\circ\sin40^\circ + 2\cos80^\circ\sin40^\circ]$

  $= \frac{\sqrt{3}}{8}[\sin100^\circ - \sin 20^\circ + \sin 120^\circ - \sin 40^\circ]$

  $= \frac{\sqrt{3}}{8}[\cos10^\circ - (\sin20^\circ + \sin40^\circ) + \cos30^\circ]$

  $= \frac{\sqrt{3}}{8}[\cos10^\circ - 2\sin30^\circ\cos10^\circ + \frac{\sqrt{3}}{2}]$

  $= \frac{3}{8}[\cos10^\circ - 2.\frac{1}{2}.\cos10^\circ + \frac{3}{2}] = \frac{3}{16}$

\item We have to prove that, $\cos 20^\circ\cos40^\circ\cos80^\circ = \frac{1}{8}$

  L.H.S. $= \frac{1}{2\cos20^\circ\cos80^\circ}\cos40^\circ$

  $= \frac{1}{2}[\cos100^\circ + \cos60^\circ]\cos40^\circ$

  $= \frac{1}{2}[-\sin10^\circ\cos40^\circ + \frac{1}{2}\cos40^\circ]$

  $= \frac{1}{4}[-2\cos40^\circ\sin10^\circ + \cos40^\circ]$

  $= \frac{1}{4}[-\sin50^\circ + sin30^\circ + \cos40^\circ]$

  $= \frac{1}{8}[\because \sin50^\circ = \cos40^\circ~\text{and}~\sin30^\circ = \frac{1}{2}]$

\item We have to prove that, $\tan20^\circ\tan40^\circ\tan60^\circ\tan80^\circ = 3$

  Using results of 70 and 71 it can be solved.

\item We have to prove that, $\cos10^\circ\cos30^\circ\cos50^\circ\cos70^\circ = \frac{3}{16}$

  L.H.S. $= \frac{1}{2}(2\cos10^\circ\cos70^\circ)(\frac{\sqrt{3}}{2}\cos50^\circ)$

  $= \frac{\sqrt{3}}{4}[\cos80^\circ + \cos60^\circ]\cos50^\circ$

  $= \frac{\sqrt{3}}{4}[\cos80^\circ\cos50^\circ + \frac{1}{2}\cos50^\circ]$

  $= \frac{\sqrt{3}}{8}[2\cos80^\circ\cos50^\circ + \cos50^\circ]$

  $= \frac{\sqrt{3}}{8}[\cos 130^\circ + \cos30^\circ + \cos50^\circ]$

  $= \frac{\sqrt{3}}{8}[\cos(180^\circ - 50^\circ) + \cos30^\circ + \cos50^\circ]$

  $= \frac{3}{16}$

\item We have to prove that, $4\cos\theta\cos\left(\frac{\pi}{3} + \theta\right)\cos\left(\frac{\pi}{3} - \theta\right) =
  \cos3\theta$

  L.H.S. $= 2\cos\theta.2\cos\left(\frac{\pi}{3} + \theta\right)\cos\left(\frac{\pi}{3} - \theta\right)$

  $= 2\cos\theta\left[\cos\left(\frac{2\pi}{3}\right) + \cos2\theta\right]$

  $= -\cos\theta + 2\cos\theta\cos2\theta = -\cos\theta + \cos3\theta + \cos\theta = \cos3\theta$

\item We have to prove that $\tan\theta\tan(60^\circ - \theta)\tan(60^\circ + \theta) = \tan3\theta$

  We have just proven, $4\cos\theta\cos\left(\frac{\pi}{3} + \theta\right)\cos\left(\frac{\pi}{3} - \theta\right) =
  \cos3\theta$

  Let us evaluate $\sin\theta\sin(60^\circ - \theta)\sin(60^\circ + \theta)$

  $= \frac{1}{2}\sin\theta.2\sin(60^\circ - \theta)\sin(60^\circ + \theta)$

  $= \frac{1}{2}\sin\theta\left[\cos2\theta - \cos\frac{2\pi}{3}\right]$

  $= \frac{1}{2}\sin\theta\cos2\theta + \frac{1}{4}\sin\theta$

  $= \frac{1}{4}(\sin3\theta - \sin\theta) + \frac{1}{4}\sin\theta$

  $= \frac{1}{4}\sin3\theta$

  Thus, we have the desired result.

\item Let $p = \cos\alpha\cos\beta$

  $= \frac{1}{2}2\cos\alpha\cos\beta = \frac{1}{2}\left[\cos(\alpha + \beta) + \cos(\alpha - \beta)\right]$

  $= \frac{1}{2}\cos(\alpha - \beta)[\because \alpha + \beta = 90^\circ \therefore \cos(\alpha + \beta) = 0]$

  Now maximum value of $\cos(\alpha - \beta)$ is $1$ therefore maximum value of $p$ is $\frac{1}{2}$

\item Since $\cos\alpha = \frac{1}{\sqrt{2}},$ therefore $\alpha$ lies either in first quadrant or fourth quadrant. So
  $\sin\alpha = \pm\frac{1}{\sqrt{2}}$

  We have to compute $\tan\frac{\alpha + \beta}{2}\cot\frac{\alpha - \beta}{2} = 5 + 2\sqrt{6}$

  $= \frac{\sin\frac{\alpha + \beta}{2}\cos\frac{\alpha - \beta}{2}}{\cos\frac{\alpha + \beta}{2}\sin\frac{\alpha -
      \beta}{2}}$

  $= \frac{\sin\alpha + \sin\beta}{\sin\alpha - \sin\beta}$

  Substituting the two pair of values, we get the desired answer.

\item Let $x\cos\theta = y\cos\left(\theta + \frac{2\pi}{3}\right) = z\cos\left(\theta + \frac{4\pi}{3}\right) = k$

  Let $p = \frac{k}{x} + \frac{k}{y} + \frac{k}{z}$

  $= \cos\theta + \cos\left(\theta + \frac{2\pi}{3}\right) + \cos\left(\theta + \frac{4\pi}{3}\right)$

  $= \cos\theta + 2\cos(\theta + \pi)\cos\frac{\pi}{3} = 0[\because \cos\frac{\pi}{3} =
    \frac{1}{2}~\text{and}~\cos(\theta + \pi) = -\cos\theta]$

  Thus, $xy + yz + zx = 0$

\item Given, $\sin\theta = n\sin(\theta + 2\alpha)$

  $\Rightarrow \frac{1}{n} = \frac{\sin(\theta + 2\alpha)}{\sin\theta}$

  Using componendo and dividendo

  $\Rightarrow \frac{1 + n}{1 - n} = \frac{\sin(\theta + 2\alpha) + \sin\theta}{\sin(\theta + 2\alpha) - \sin\theta}$

  $= \frac{\sin(\theta + \alpha)\cos\alpha}{\cos(\theta + \alpha)\sin\alpha}$

  $= \tan(\theta + \alpha)\cot\alpha$

\item Given, $\frac{\sin(\theta + \alpha)}{\cos(\theta - \alpha)} = \frac{1 - m}{1 + m}$

  Using componendo and dividendo

  $\Rightarrow \frac{\sin(\theta + \alpha) + \cos(\theta - \alpha)}{\sin(\theta + \alpha) - \cos(\theta - \alpha)}
  = \frac{1 - m + 1 + m}{1 - m - 1 - m}$

  $\Rightarrow \frac{\sin(\theta + \alpha) + \sin\left(\frac{\pi}{2} - (\theta - \alpha)\right)}{\sin(\theta + \alpha) -
    \sin\left(\frac{\pi}{2} - (\theta - \alpha)\right)} = \frac{2}{-2m} = \frac{-1}{m}$

  $\Rightarrow \frac{\sin\left(\frac{\pi}{4} + \alpha\right)\cos\left(\theta -
    \frac{\pi}{4}\right)}{\cos\left(\frac{\pi}{4} + \alpha\right)\sin\left(\theta - \frac{\pi}{4}\right)} = -\frac{1}{m}$

  $\Rightarrow m = \tan\left(\frac{\pi}{4} - \theta\right)\tan\left(\frac{\pi}{4} - \alpha\right)$

\item Given, $y\sin\phi = x\sin(2\theta + \phi)$

  $\frac{\sin\phi}{\sin(2\theta + \phi)} = \frac{x}{y}$

  By componendo and dividendo

  $\frac{\sin\phi + \sin(2\theta + \phi)}{\sin(2\theta + \phi) - \sin\phi} = \frac{x + y}{y - x}$

  $\Rightarrow \frac{2\sin(\theta + \phi)\cos\theta}{2\cos(\theta + \phi)\sin\theta} = \frac{x + y}{y - x}$

  $\Rightarrow \frac{\cot\theta}{\cot(\theta + \phi)} = \frac{x + y}{y - x}$

  Hence, proven.

\item Given, $\cos(\alpha + \beta)\sin(\gamma + \delta) = \cos(\alpha - beta)\sin(\gamma - \delta)$

  $\Rightarrow \frac{\cos(\alpha + \beta)}{\cos(\alpha - \beta)} = \frac{\sin(\gamma - \delta)}{\sin(\gamma + \delta)}$

  By componendo and dividendo

  $\Rightarrow \frac{\cos(\alpha + \beta) + \cos(\alpha - \beta)}{\cos(\alpha - \beta) - \cos(\alpha + \beta)} =
  \frac{\sin(\gamma - \delta) + \sin(\gamma + \delta)}{\sin(\gamma + \delta) - \sin(\gamma - \delta)}$


  $\Rightarrow \frac{2\cos\alpha\cos\beta}{2\sin\alpha\sin\beta} = \frac{2\sin\gamma\cos\delta}{2\cos\gamma\sin\delta}$

  $\Rightarrow \cot\alpha\cot\beta = \frac{\cot\delta}{\cos\gamma}$

  Hence, proven.

  83, 84 and 85 can be solved by using componendo and dividendo as well.
\stopitemize
