% -*- mode: context; -*-
\chapter{De Moivre's Theorem}
\startitemize[n, 1*broad]
% 1
\item L.H.S. $= \left(\frac{\cos\theta + i\sin\theta}{\sin\theta + i\cos\theta}\right)^4 = \frac{(\cos\theta
    + i\sin\theta)}{[i(\cos\theta - i\sin\theta)]^4} = \frac{(\cos\theta + i\sin\theta)^4}{(\cos\theta -
    i\sin\theta)^4}[\because i^4 = 1]$

  $= \frac{\cos4\theta + i\sin4\theta}{\cos4\theta - i\sin4\theta}.\frac{\cos4\theta +
    i\sin4\theta}{\cos4\theta + i\sin4\theta} = (\cos4\theta + i\sin4\theta)^2$

  $= \cos8\theta + i\sin8\theta =$ R.H.S.
  % 2
\item L.H.S. $= \frac{(\cos\theta + i\sin\theta)^4}{(\sin\theta + i\cos\theta)^5} = \frac{(\cos\theta +
    i\sin\theta)^4}{i^5(\cos\theta - i\sin\theta)^5} = \frac{\cos4\theta + i\sin4\theta}{i(\cos5\theta -
    i\sin5\theta)}$

  $= -i(\cos4\theta + i\sin4\theta)(\cos5\theta - i\sin5\theta)^{-1} = i(\cos4\theta +
  i\sin4\theta)(\cos5\theta + i\sin5\theta)$

  $= -i(\cos9\theta + i\sin9\theta) = \sin9\theta - i\cos9\theta =$ R.H.S.
  % 3
\item Given $\frac{(\cos3\theta + i\sin3\theta)^5(\cos\theta - i\sin\theta)^3}{(\cos5\theta +
  i\sin5\theta)^7(\cos2\theta - i\sin2\theta)^5} = \frac{(\cos15\theta + i\sin15\theta)(\cos3\theta -
  i\sin3\theta)}{(\cos35\theta + i\sin35\theta)(\cos10\theta - i\sin10\theta)}$

$= \frac{\cos12\theta + i\sin12\theta}{\cos25\theta + i\sin25\theta} = \cos13\theta - \sin13\theta$.
%4
\item $(\sin15^\circ - \cos15^\circ) + i(\sin75^\circ - \cos75^\circ) = (\sin15^\circ - \sin75^\circ) +
  i(\cos15^\circ - \cos75^\circ)$

  $= 2\cos45^\circ\sin(-30^\circ) + i.2\sin45^\circ\sin30^\circ = -\cos45^\circ + i\sin45^\circ$

  Thus, $[(\sin15^\circ - \cos15^\circ) + i(\sin75^\circ - \cos75^\circ)]^n = (-1)^n\left[\cos\frac{n\pi}{4}
    - i\sin\frac{n\pi}{4}\right]$

  Similarly, $[(\sin15^\circ - \cos15^\circ - i(\cos75^\circ - \cos75^\circ))]^n =
  (-1)^n\left[\cos\frac{n\pi}{4} + i\sin\frac{n\pi}{4}\right]$

  Thus, given expression is $(-1)^n2\cos\frac{n\pi}{4}$.
  % 5
\item L.H.S.\ $= \left(\frac{1 + \cos\theta + i\sin\theta}{1 + \cos\theta - i\sin\theta}\right)^n =
  \left(\frac{2\cos^2\frac{\theta}{2} + i.2\sin\frac{\theta}{2}\cos\frac{\theta}{2}}{2\cos^2\frac{\theta}{2}
      - i.2\sin\frac{\theta}{2}\cos\frac{\theta}{2}}\right)^n$

  $= \left(\frac{\cos\frac{\theta}{2} + i\sin\frac{\theta}{2}}{\cos\frac{\theta}{2} -
    \sin\frac{\theta}{2}}\right)^n = \left(\cos\frac{n\theta}{2} +
  i\sin\frac{n\theta}{2}\right)\left(\cos\frac{n\theta}{2} + i\sin\frac{n\theta}{2}\right)$

  $= \cos n\theta + i\sin n\theta$.
  %6
\item $(1 + \cos\theta + i\sin\theta)^n = \left(2\cos^2\frac{\theta}{2} +
  i.2\sin\frac{\theta}{2}\cos\frac{\theta}{2}\right)^n$

  $= 2^n\cos^n\frac{\theta}{2}\left(\cos\frac{n\theta}{2} + i\sin\frac{n\theta}{2}\right)$

  Similarly, $(1 + \cos\theta - i\sin\theta)^n = 2^n\cos^n\frac{\theta}{2}\left(\cos\frac{n\theta}{2} -
  i\sin\frac{n\theta}{2}\right)$

  Thus, $(1 + \cos\theta + i\sin\theta)^n(1 + \cos\theta - i\sin\theta)^n = 2^{n +
    1}\cos^n\frac{\theta}{2}\cos\frac{n\theta}{2}$.
  %7
\item L.H.S.\ $= x_1x_2x_3\ldots \infty = \left(\cos\frac{\pi}{2} +
  i\sin\frac{\pi}{2}\right)\left(\cos\frac{\pi}{2^2} + i\sin\frac{\pi}{2^2}\right)\cdots \infty$

  $= \cos\left(\frac{\pi}{2} + \frac{\pi}{2^2} + \cdots \infty\right) + i\sin\left(\frac{\pi}{2}
  + \frac{\pi}{2^2} + \cdots \infty\right)$

  $= \cos\left(\frac{\frac{\pi}{2}}{1 - \frac{1}{2}}\right) + i\sin\left(\frac{\frac{\pi}{2}}{1
    - \frac{1}{2}}\right) = \cos\pi + i\sin\pi = -1$.
  %8
\item $xyz = \cos2(\alpha + \beta + \gamma) + i\sin2(2\alpha + \beta + \gamma)$

  $\sqrt{xyz} = [\cos2(\alpha + \beta + \gamma) + i\sin2(\alpha + \beta + \gamma)]^{1/2} = \cos(\alpha
  + \beta + \gamma) + i\sin(\alpha + \beta + \gamma)$

  $\Rightarrow \frac{1}{\sqrt{xyz}} = \cos(\alpha + \beta + \gamma) - i\sin(\alpha + \beta + \gamma)$

  $\Rightarrow \sqrt{xyz} + \frac{1}{\sqrt{xyz}} = 2\cos(\alpha + \beta + \gamma)$.
  %9
\item $\frac{(\cos\theta + i\sin\theta)^5(\cos2\theta - i\sin2\theta)^2}{(\cos3\theta +
  i\sin3\theta)^{11}(\cos4\theta - i\sin4\theta)^8} = \frac{(\cos5\theta + i\sin5\theta)(\cos4\theta -
  i\sin4\theta)}{(\cos33\theta + i\sin33\theta)(\cos32\theta - i\sin32\theta)}$

  $= \frac{\cos\theta + i\sin\theta}{\cos\theta + i\sin\theta} = 1$.
  %10
\item L.H.S.\ $= [(\cos\theta +\cos\phi) + i(\sin\theta + \sin\phi)]^n + [(\cos\theta + \cos\phi) -
  i(\sin\theta + \sin\phi)]^n$

  $= \left[2\cos\frac{\theta + \phi}{2}\cos\frac{\theta - \phi}{2} + i.2\sin\frac{\theta
    + \phi}{2}\cos\frac{\theta - \phi}{2}\right]^n + \left[2\cos\frac{\theta + \phi}{2}\cos\frac{\theta
    - \phi}{2} - i.2\sin\frac{\theta + \phi}{2}\cos\frac{\theta - \phi}{2}\right]^n$

  $= 2^n\cos^n\frac{\theta - \phi}{2}\left[\cos n\frac{\theta + \phi}{2} + i\sin n\frac{\theta + \phi}{2}
  + \cos n\frac{\theta + \phi}{2} - i\sin n\frac{\theta + \phi}{2}\right]$

  $= 2^{n + 1}\cos^n\frac{\theta - \phi}{2}\cos n\frac{\theta + \phi}{2}$.
  %11
\item L.H.S. $= \left(\frac{1 + \sin\phi + i\cos\phi}{1 + \sin\phi - i\cos\phi}\right)^n$

  $= \left[\frac{\left(\cos\frac{\phi}{2} + \sin\frac{\phi}{2}\right)^2 + i\left(\cos^2\frac{\phi}{2}
    - \sin^2\frac{\phi}{2}\right)}{\left(\cos\frac{\phi}{2} + \sin\frac{\phi}{2}\right)^2 -
    i\left(\cos^2\frac{\phi}{2} - \sin^2\frac{\phi}{2}\right)}\right]^n$

  $= \left[\frac{\left(\cos\frac{\phi}{2} + \sin\frac{\phi}{2}\right) + i\left(\cos\frac{\phi}{2}
    - \sin\frac{\phi}{2}\right)}{\left(\cos\frac{\phi}{2} + \sin\frac{\phi}{2}\right) -
    i\left(\cos\frac{\phi}{2} - \sin\frac{\phi}{2}\right)}\right]^n$

  Rationalizing gives us

  $= \left[\frac{\left(\cos\frac{\phi}{2} + \sin\frac{\phi}{2}\right)^2 - \left(\cos\frac{\phi}{2}
      - \sin\frac{\phi}{2}\right)^2 + 2i\left(\cos^2\frac{\phi}{2}
      - \sin^2\frac{\phi}{2}\right)}{\left(\cos\frac{\phi}{2} + \sin\frac{\phi}{2}\right)^2
      + \left(\cos\frac{\phi}{2} - \sin\frac{\phi}{2}\right)^2}\right]^n$

  $= \left[\frac{2\sin\phi + i.2\cos\phi}{2}\right]^n = (\sin\phi + i\cos\phi)^n =$ R.H.S.
  %12
\item Given $\frac{(\cos\theta + i\sin\theta)(\cos2\theta +
  i\sin\theta)}{(\cos3\theta - i\sin3\theta)} = \frac{\cos3\theta + i\sin3\theta}{\cos3\theta +
  i\sin3\theta}$

  $= \cos6\theta + i\sin6\theta$. Putting $\theta = 15^\circ$ makes the obtained expression as

  $\cos90^\circ + i\sin90^\circ = i$.
  %13
\item Given $\left(\frac{\cos\theta + i\sin\theta}{\sin\theta + i\cos\theta}\right)^5
  = \left(\frac{1}{i}\frac{(\cos\theta + i\sin\theta)}{\cos\theta - i\sin\theta}\right)^5$

  $= \left(\frac{\cos2\theta + i\sin2\theta}{i}\right)^5 = (\sin2\theta -i \cos2\theta)^5
  = \cos\left(10\theta - \frac{5\pi}{2}\right) + i\sin\left(\10\theta - \frac{5\pi}{2}\right)$.
  %14
\item L.H.S. $= [(\cos\theta - \cos\phi) + i(\sin\theta - \sin\phi)]^n + [(\cos\theta - \cos\phi) -
  i(\sin\theta - \sin\phi)]^n$

  $= \left[-2\sin\frac{\theta + \phi}{2}\sin\frac{\theta - \phi}{2} + 2i\cos\frac{\theta
    + \phi}{2}\cos\frac{\theta - \phi}{2}\right]^n + \left[-2\sin\frac{\theta + \phi}{2}\sin\frac{\theta
    - \phi}{2} - 2i\cos\frac{\theta + \phi}{2}\cos\frac{\theta - \phi}{2}\right]^n$

  $= 2^n\sin^n\frac{\theta - \phi}{2}\left[-\sin\frac{\theta + \phi}{2} + i\cos\frac{\theta
    + \phi}{2}\right]^n + 2^n\sin^n\frac{\theta - \phi}{2}\left[-\sin\frac{\theta + \phi}{2} -
  i\cos\frac{\theta + \phi}{2}\right]$

  $= 2^n\sin^n\frac{\theta - \phi}{2}\left[\cos\frac{\pi + \theta + \phi}{2} + i\sin\frac{\pi + \theta
    + \phi}{2}\right]^n + 2^n\sin^n\frac{\theta - \phi}{2}\left[\cos\frac{\pi + \theta + \phi}{2} -
  i\sin\frac{\pi + \theta + \phi}{2}\right]^n$

  $= 2^{n + 1}\sin^n\frac{\theta - \phi}{2}\cos n\left(\frac{\pi + \theta + \phi}{2}\right) =$ R.H.S.
  %15
\item Let $a_n = r_n\cos\theta_n$ and $b_n = r_n\sin\theta_n$ then $r_n = \sqrt{a_n^2 + b_n^2}$ and
  $\theta_n = \tan^{-1}\frac{b_n}{a_n}$

  Putting $n = 1, 2, 3, \ldots$ gives us $a_1 + ib_1 = r_1(\cos\theta_1 + i\sin\theta_1), a_2 =
  r_2(\cos\theta_2 + i\sin\theta_2)$ and so on.

  $\left(a_1 + ib_1\right)\left(a_2 + ib_1\right)\cdots\left(a_n + ib_n\right) =
  (r_1r_2\ldots r_n)\left[\cos\left(\theta_1 + \theta_2 + \cdots + \theta_n\right) + i\sin\left(\theta_1
    + \theta_2 + \cdots + \theta_n\right)\right] = A + iB$

  $\Rightarrow A^2 + B^2 = r_1^2r_2^2\ldots r_n^2 = \left(a_1^2 + b_1^2\right)\left(a_2^2 +
  b_2^2\right)\cdots \left(a_n^2 + b_n^2\right)$

  and $\tan^{-1}\frac{B}{A} = \tan^{-1}\frac{b_1}{a_1} + \tan^{-1}\frac{b_2}{a_2} + \cdots
  + \tan^{-1}\frac{b_n}{a_n}$.
  %16
\item Let $a = r\cos\theta, b =r\sin\theta\Rightarrow r = \sqrt{a^2 + b^2}$ and $theta
  = \tan^{-1}\frac{b}{a}$

  $(a + ib)^{m/n} = r^{m/n}\left[\cos\frac{m\theta}{n} + i\sin\frac{m\theta}{n}\right]$ and $(a - ib)^{m/n}
  = r^{m/n}\left[\cos\frac{m\theta}{n}  i\sin\frac{m\theta}{n}\right]$

  Adding gives us $(a + ib)^{m/n} + (a - ib)^{m/n} = r^{m/n}\left[2\cos\frac{m\theta}{n}\right] =
  2\left(a^2 + b^2\right)^{m/n}\left[\frac{m}{n}\tan^{-1}\frac{b}{a}\right]$.
  %17
\item Let $1 + \sin\phi = r\cos\theta$ and $\cos\phi = r\sin\theta$

  $\Rightarrow \tan\theta = \frac{\cos\phi}{1 + \sin\phi} = \frac{\cos^2\frac{\phi}{2}
    - \sin^2\frac{\phi}{2}}{\left(\cos\frac{\phi}{2} + \sin\frac{\phi}{2}\right)^2}
  = \frac{\cos\frac{\phi}{2} - \sin\frac{\phi}{2}}{\cos\frac{\phi}{2} + \sin\frac{\phi}{2}} = \frac{1
    - \tan\frac{\phi}{2}}{1 + \tan\frac{\phi}{2}}$

  $\Rightarrow \theta = \frac{\pi}{4} - \frac{\phi}{2}$

  Now $\frac{(1 + \sin\phi + i\cos\phi)^n}{(1 + \sin\phi - \cos\phi)^n} = \frac{(r\cos\theta +
    i.r\sin\theta)^n}{(r\cos\theta - i.r\sin\theta)^n}$

  $= \cos2n\theta + i\sin2n\theta = \cos\left(\frac{n\pi}{2} - n\phi\right) + i\sin\left(\frac{n\phi}{2} -
  n\phi\right)$.
  %18
\item Putting $x = i$ in given equation gives us

  $(1 + i)^n = p_0 + p_1i + p_2i^2 + \cdots + p_ni^n = p_0 + p_1i - p_2 - p_3i + p_4 + \cdots$

  $= (p_0 - p_2 + p_4 - \cdots) + i(p_1 - p_3 + p_5 - \cdots)$

  Putting $x = - i$ gives us

  $(1 - i)^n = (p_0 - p_2 + p_4 - \cdots) - i(p_1 - p_3 + p_5 - \cdots)$

  Adding the two equations gives us

  $(1 + i)^n + (1 - i)^n = 2(p_0 - p_2 + p_4 - \cdots)\Rightarrow p_0 - p_2 + p_4 - \cdots = \frac{1}{2}[(1
    + i)^n + (1 - i)^n]$

  Let $1 = r\cos\theta$ and $1 = r\sin\theta \Rightarrow r = \sqrt{2}$ and $theta = \frac{\pi}{4}$

  $\Rightarrow (1 + i)^n = \left[\sqrt{2}\left(\cos\frac{\pi}{4} + i\sin\frac{\pi}{4}\right)\right]^n =
  2^{n/2}\left[\cos\frac{n\pi}{4} + i\sin\frac{n\pi}{4}\right]$

  Similarly $(1 - i)^n = 2^{n/2}\left[\cos\frac{n\pi}{4} - i\sin\frac{n\pi}{4}\right]$

  $\Rightarrow (1 + i)^n + (1 - i)^n = 2^{n/2}.2.\cos\frac{n\pi}{4}$

  $\Rightarrow p_0 - p_2 + p_4 - \cdots = 2^{n/2}\cos\frac{n\pi}{4}$.
  %19
\item This has been proven in last problem.
  %20
\item Let $\sqrt{3} = r\cos\theta$ and $1 = r\sin\theta\Rightarrow r = 2$ and $theta = \frac{\pi}{6}$

  Now $(\sqrt{3} + i)^n = 2^n(\cos\frac{\pi}{6} + i\sin\frac{\pi}{6})^n$ and $(\sqrt{3} - i)^n =
  2^n(\cos\frac{\pi}{6} - i\sin\frac{\pi}{6})^n$

  $\Rightarrow (\sqrt{3} + i)^n + (\sqrt{3} - i)^n = 2^{n + 1}\cos\frac{n\pi}{6}$.
  %21
\item Let $1 = r\cos\theta$ and $1 = r\sin\theta\Rightarrow r = \sqrt{2}$ and $theta = \frac{\pi}{4}$

  Given that $(1 + x)^n = C_0 + C_1x + C_2x^2 + \cdots$

  Putthing $x = i, -i$ gives us

  $(1 + i)^n = 2^{n/2}\left(\cos\frac{n\pi}{4} + i\sin\frac{n\pi}{4}\right) = C_0 + i.C_1 - C_2 -iC_3
  + \cdots$

  and $(1 - i)^n = 2^{n/2}\left(\cos\frac{n\pi}{4} - i\sin\frac{n\pi}{4}\right) = C_0 - iC_1 - C_2 + iC_3
  + \cdots$

  Subtracting gives us $i.2^{n/2 + 1}\sin\frac{n\pi}{4} = 2i(C_1 + C_3 + C_5 + \cdots)$

  $\Rightarrow C_1 - C_3 + C_5 - C_7 + \cdots = 2^{n/2}\sin\frac{n\pi}{4}$

  Putting $x = 1, -1$ gives us

  $2^n = C_0 + C_1 + C_2 + \cdots$ and $0 = C_0 - C_1 + C_2 - \cdots$

  $\Rightarrow 2^n + 0 + (1 + i)^n + (1 - i)^n = 4\left(C_0 + C_4 + C_8 + \cdots\right)$

  $\Rightarrow C_0 + C_4 + C_8 + \cdots = 2^{n/2 - 1}\cos\frac{n\pi}{4} + 2^{n - 2}$
  %22
\item Given that $x + \frac{1}{x} = 2\cos\theta \Rightarrow x^2 - 2x\cos\theta + 1 = 0 \Rightarrow x
  = \cos\theta\pm i\sin\theta$

  $x^7 + \frac{1}{x^7} = (\cos7\theta\pm i\sin7\theta) + (\cos7\theta\mp i\sin7\theta) = 2\cos7\theta$.
  %23
\item Like last problem $x = \cos\alpha\pm i\sin\alpha, y = \cos\beta\pm i\sin\beta$ and $z = \cos\gamma\pm
  i\sin\gamma$

  $\Rightarrow xyz = \cos(\alpha + \beta + \gamma)\pm i\sin(\alpha + \beta + \gamma)$ and $\frac{1}{xyz}
  = \cos(\alpha + \beta + \gamma)\mp i\sin(\alpha + \beta + \gamma)$

  $\Rightarrow xyz + \frac{1}{xyz} = 2\cos(\alpha + \beta + \gamma)$.
  %24
\item Given that $x^2 - 2x\cos\theta + 1 = 0\Rightarrow x = \cos\theta\pm i\sin\theta$

  Let $\alpha = \cos\theta + i\sin\theta$ and $\beta = \cos\theta - i\sin\theta$

  $\alpha^n + \beta^n = 2\cos n\theta$ and $\alpha^n\beta^n = 1$

  The equation whose roots are $\alpha^n$ and $\beta^n$ are $x^2 - 2xn\cos\theta + 1 = 0$.
  %25
\item Given equation $x^2 - 2x + 4 = 0\Rightarrow x = \frac{2\pm\sqrt{4 - 16}}{2} = 1\pm\sqrt{3}i$

  Let $\alpha = 1 + \sqrt{3}i$ and $\beta = 1 - \sqrt{3}i$. Let $1 = r\cos\theta$ and $\sqrt{3} =
  r\sin\theta$

  $\Rightarrow r = 2, \theta = \frac{\pi}{3}$

  $\alpha^n + \beta^n = 2^{n}\left[\cos\frac{\pi}{3} + i\sin\frac{\pi}{3}\right]^n +
  2^{n}\left[\cos\frac{\pi}{3} - i\sin\frac{\pi}{3}\right]^n = 2^{n + 1}\cos\frac{n\pi}{3}$.
  %26
\item $(\cos\theta + i\sin\theta)(\cos2\theta + i\sin2\theta)\cdots(\cos n\theta + i\sin n\theta)
  = \cos\frac{n(n + 1)}{2}\theta + i\sin\frac{n(n + 1)}{2}\theta = 1$

  $\Rightarrow \cos\frac{n(n + 1)}{2}\theta = \cos 0 \Rightarrow \frac{n(n + 1)}{2}\theta = 2k\pi$

  $\Rightarrow theta = \frac{4k\pi}{n(n + 1)}$, where $k = 0, \pm1, \pm2, \ldots$.
  %27
\item Given that $z + \frac{1}{z} = 2\sin\theta \Rightarrow z^2 - 2z\sin\theta + 1 = 0$

  $z = \frac{2\sin\theta\pm\sqrt{4\sin^2\theta - 4}}{2} = \sin\theta\pm i\cos\theta
  = \cos\left(\frac{\pi}{2} - \theta\right) + i\sin\left(\frac{\pi}{2} - \theta\right)$

  $z^{4n} + \frac{1}{z^{4n}} = 2\cos(2n\pi - 4n\theta) = 2\cos4n\theta$.
  %28
\item Given that $2\cos\alpha = x + \frac{1}{x} \Rightarrow x^2 - 2x\cos\alpha + 1 = 0 \Rightarrow x
  = \cos\alpha\pm i\sin\alpha$

  Similarly, $y = \cos\beta\pm i\sin\beta$ and $z = \cos\gamma\pm i\sin\gamma$

  $x^py^qz^r = \cos (p\alpha + q\beta + r\gamma) + i\sin(p\alpha + q\beta + r\gamma)$ and
  $\frac{1}{x^py^qz^r} = \cos(p\alpha + q\beta + r\gamma) - i\sin(p\alpha + q\beta + r\gamma)$

  $\therefore 2\cos(p\alpha + q\beta + r\gamma) = x^py^qz^2 + \frac{1}{x^py^qz^r}$.
  %29
\item Given that $x + \frac{1}{x} = 2\cos\theta \Rightarrow x = \cos\theta\pm i\sin\theta$, and similarly,
  $y = \cos\phi\pm i\sin\phi$

  $\Rightarrow xy = \cos(\theta + \phi)\pm i\sin(\theta + \phi)$ and $\frac{1}{xy} = \cos(\theta + \phi)\mp
  i\sin(\theta + \phi)$

  $\Rightarrow xy + \frac{1}{xy} = 2 \cos(\theta + \phi)$.
  %30
\item Let $y = \frac{1}{2x + \frac{1}{2x + \frac{1}{2x  + \cdots \infty}}}\Rightarrow y = \frac{1}{2x +
  y0\Rightarrow y^2 + 2xy -1 = 0}$

  $y = -x \pm \sqrt{x^2 + 1}$

  Now $x^2 + 1 = \cos^2\theta - \sin^2\theta + 1 + 2i\sin\theta\cos\theta = 2\cos^2\theta +
  2i\sin\theta\cos\theta$

  $\Rightarrow \sqrt{x^2 + 1} = \sqrt{2\cos\theta}\sqrt{\cos\theta + i\sin\theta}$

  Using De Moivre's theorem $\sqrt{\cos\theta + i\sin\theta} = \cos\frac{\theta}{2} + i\sin\frac{\theta}{2}$

  Now $\sqrt{2\cos\theta}\cos\frac{\theta}{2} = \sqrt{\cos\theta + \cos^2\theta}$ and
  $\sqrt{2\cos\frac{\theta}{2}}\sin\frac{\theta}{2} = \sqrt{\cos\theta - \cos^2\theta}$

  And thus, we have proven the required equation.
  %31
\item Given $\sqrt{1 - c^2} = nc - 1\Rightarrow 1 - c^2 = c^2n^2 - 2nc + 1 \Rightarrow c[c(n^2 + 1) - 2n] =
  0$. $c$ cannot be $-1$ because it will make $1 = -1$ from original identity. Thus, $\frac{c(1 + n^2)}{2n}
  = 1$.

  R.H.S.\ $= \frac{c}{2n}\left[1 + n^2 + n\left(x + \frac{1}{x}\right)\right] = \frac{c}{2n}[1 +
    n^2 + n.2\cos\theta] = 1 + c\cos\theta$.
  %32
\item $abcd = \cos(\alpha + \beta + \gamma + \delta) + i\sin(\alpha + \beta + \gamma + \delta)$

  $\Rightarrow \sqrt{abcd} = \cos\frac{\alpha + \beta + \gamma + \delta}{2} + i\sin\frac{\alpha + \beta
  + \gamma + \delta}{2}$

  $\Rightarrow \frac{1}{\sqrt{abcd}} = \cos\frac{\alpha + \beta + \gamma + \delta}{2} - i\sin\frac{\alpha
  + \beta + \gamma + \delta}{2}$

  $\Rightarrow \sqrt{abcd} + \frac{1}{\sqrt{abcd}} = 2\cos\frac{\alpha + \beta + \gamma + \delta}{2}$.
  %33
\item $x^m = \cos m\theta + i\sin m\theta, y^n = \cos n\phi + i\sin n\phi \Rightarrow x^my^n = \cos(m\theta
  + n\phi) + i\sin(m\theta + n\phi)$

  $\Rightarrow \frac{1}{x^my^n} = \cos(m\theta + n\phi) - i\sin(m\theta + n\phi)$

  $\Rightarrow x^my^n + \frac{1}{x^my^n} = 2\cos(m\theta + n\phi)$.
  %34
\item Given that $\alpha$ is a root of $x^2 - 2ax\cos\theta + a^2 = 0\Rightarrow \alpha
  = \frac{2a\cos\theta\pm\sqrt{4a^2\cos^2\theta - 4a^2}}{2} = a(\cos\theta\pm i\sin\theta)$

  Now $\alpha^{2n} - 2a^n\alpha^n\cos n\theta + a^{2n} = a^{2n}(\cos2n\theta\pm i\sin2n\theta) -
  2a^{2n}(\cos n\theta \pm i\sin n\theta)\cos n\theta + a^{2n}$

  Getting rid of $a^{2n}$ we have

  $\cos2n\theta + 1 \pm i\sin2n\theta - 2\cos^2n\theta \mp 2\sin n\theta\cos\theta + 1 = 0$.

  Hence proven.
  %35
\item Given that $x + y + z = 0 \Rightarrow \cos\alpha + \cos\beta + \cos\gamma + i(\sin\alpha + \sin\beta +
  \sin\gamma) = 0$

  Comparing real and imaginary parts we have $\cos\alpha + \cos\beta + \cos\gamma = 0$ and $\sin\alpha
  + \sin\beta + \sin\gamma = 0$

  Now $\frac{1}{x} + \frac{1}{y} + \frac{1}{z} = \cos\alpha - i\sin\alpha + \cos\beta - i\sin\beta
  + \cos\gamma - i\sin\gamma = 0$.
  %36
\item From given conditions we can write $(\cos\alpha + i\sin\alpha) + (\cos\beta + i\sin\beta) +
  (\cos\gamma + i\sin\gamma) = 0$

  Let $x = \cos\alpha + i\sin\alpha, y = \cos\beta + i\sin\beta, z = \cos\gamma + i\sin\gamma \Rightarrow x
  + y + z = 0$

  Thus, $\frac{1}{x} + \frac{1}{y} + \frac{1}{z} = (\cos\alpha - i\sin\alpha) + (\cos\beta - i\sin\beta) +
  (\cos\gamma - i\sin\gamma) = 0$

  $\Rightarrow xy + yz + zx = 0 \Rightarrow (x + y + z)^2 = x^2 + y^2 + z^2 + xy + yz + zx \Rightarrow x^2 +
  y^2 + z^2 = 0 \Rightarrow \cos2\alpha + \cos2\beta + \cos2\gamma = 0$

  and $\sin2\alpha + \sin2\beta + \sin2\gamma = 0$

  From first $1 - 2\sin^2\alpha + 1 - 2\sin^2\beta + 1 - 2\sin^2\gamma = 0 \Rightarrow \sin^2\alpha
  + \sin^2\beta + \sin^2\gamma = \frac{3}{2}$

  $\Rightarrow \cos^2\alpha + \cos^2\beta + \cos^2\gamma = \frac{3}{2}$.
  %37
\item From previous problem we have $x + y + z  = 0 \Rightarrow x^3 + y^3 + z^3 - 3xyz = (x + y +
  z)\left(x^2 + y^2 + z^2 - xy - yz - zx\right) = 0$

  $\Rightarrow x^3 + y^3 + z^3 = 3xyz \Rightarrow \cos3\alpha + \cos3\beta + \cos3\gamma + i(\sin3\alpha
  + \sin3\beta + \sin3\gamma) = 3[\cos(\alpha + \beta + \gamma) + i\sin(\alpha + \beta + \gamma)]$

  Comparing real and imaginary parts gives us desired equations.
  %38
\item Given that $x + y + z = xyz \Rightarrow (\cos\alpha + i\sin\alpha) + (\cos\beta + i\sin\beta) +
  (\cos\gamma + i\sin\gamma) = \cos(\alpha + \beta + \gamma) + i\sin(\alpha + \beta + \gamma)$

  $\Rightarrow \cos\alpha + \cos\beta + \cos\gamma = \cos(\alpha + \beta + \gamma)$ and $\sin\alpha
  + \sin\beta + \sin\gamma = \sin(\alpha + \beta + \gamma)$

  Squaring and adding gives us

  $\Rightarrow (\cos\alpha + \cos\beta + \cos\gamma)^2 + (\sin\alpha + \sin\beta + \sin\gamma)^2 = 1$

  $\Rightarrow \cos^2\alpha + \sin^2\alpha + \cos^2\beta + \sin^2\beta + \cos^2\gamma + \sin^2\gamma +
  2\cos\alpha\cos\beta + 2\cos\beta\cos\gamma + 2\cos\gamma\cos\alpha + 2\sin\alpha\sin\beta +
  2\sin\beta\sin\gamma + 2\sin\gamma\sin\alpha = 1$

  $\Rightarrow 3 + 2[\cos(\alpha - \beta) + \cos(\beta - \gamma) + \cos(\gamma - \alpha)] = 1$

  $\Rightarrow \cos(\alpha - \beta) + \cos(\beta - \gamma) + \cos(\gamma - \alpha) = -1$.
  %39
\item $\frac{x^3}{y^3} + \frac{y^3}{x^3} = \frac{\cos3\theta + i\sin3\theta}{\cos3\phi + i\sin3\phi}
  + \frac{\cos3\phi + i\sin3\phi}{\cos3\theta + i\sin3\theta}$

  $= \cos3(\theta - \phi) + i\sin3(\phi - \theta) + \cos3(\theta - \phi) + i\sin3(\theta - \phi)=
  2\cos3(\theta - \phi)$.
  %40
\item Combining the given identities gives us $\sum e^{i2\alpha} = \sum e^{i(\beta
  + \gamma)}\Rightarrow \sum\left(e^{i2[\alpha - (\beta + \gamma)]}\right) = 0$

  Let $S = \alpha + \beta + \gamma \Rightarrow e^{i2\alpha} - e^{i(\beta + \gamma)} = e^{i(S
    - \alpha)}\left(e^{i(3\alpha - S)} - 1\right)$

  $\Rightarrow \sum e^{i(3\alpha - S)} = 0 \Rightarrow \sum e^{i3\alpha} = 0$.

  Thus, the complex numbers lie on units circle. The sum being zero means that these are vertices of an
  equilateral triangle centered at origin. Thus, $\alpha = \beta = \gamma$.
  %41
\item Let $x = e^{i\alpha}, y = e^{i\beta}, z = e^{i\gamma}\Rightarrow x + y + z = 0$ from given conditions.

  Now $\cos2\alpha + \cos2\beta + \cos2\gamma + 2[\cos(\alpha + \beta) + \cos(\beta + \gamma) + \cos(\gamma
    + \alpha)] = \Re[x^2 + y^2 + z^2 + 2(xy + yz + zx)] = \Re(x + y + z)^2 = 0$.
  %42
\item $-1 = \cos\pi + i\sin\pi = \cos(2k\pi + \pi) + i\sin(2k\pi + \pi)$

  $\Rightarrow (-1)^{1/6} = \cos\left(\frac{2k\pi + \pi}{6}\right) + i\sin\left(\frac{2k\pi +
  \pi}{6}\right)$, where $k = 0, 1, 2, 3, 4, 5$

  Putting $k = 0, 1, 2, 3, 4, 5$ we get desired values as $\cos\frac{\pi}{6} + i\sin\frac{\pi}{6},
  \cos\frac{3\pi}{6} + i\sin\frac{3\pi}{6}, \cos\frac{5\pi}{6} + i\sin\frac{5\pi}{6}, \cos\frac{7\pi}{6} +
  i\sin\frac{7\pi}{6}, \cos\frac{9\pi}{6} + i\sin\frac{9\pi}{6}$ and $\cos\frac{11\pi}{6} +
  i\sin\frac{11\pi}{6}$.
  %43
\item Let $1 = r\cos\theta$ and $1 = r\sin\theta \Rightarrow r = \sqrt{2}, \theta = \frac{\pi}{4}$

  $1 + i = \sqrt{2}\left(\cos\frac{\pi}{4} + i\sin\frac{\pi}{4}\right) = \sqrt{2}\left[\cos\left(2k\pi +
  \frac{\pi}{4}\right) + i\sin\left(2k\pi + \frac{\pi}{4}\right)\right]$

  $(1 + i)^{1/7} = 2^{1/14}\left[\cos\left(\frac{(8k + 1)\pi}{4}\right) + i\sin\left(\frac{(8k +
    1)\pi}{4}\right)\right]$ where $k = 0, 1, 2, 3, 4, 5, 6$

  Putting $k = 0, 1, 2, 3, 4, 5, 6$ we obtain the desired values.
  %44
\item Proceeding like previous problems we obtain $(1 + i)^{1/3} = 2^{1/6}\left[\cos\frac{(8k + 1)\pi}{12}
  + i\sin\frac{(8k + 1)\pi}{12}\right]$, where $k = 0, 1, 2$

  Putting $k = 0, 1, 2$ we obtain following

  $x_1 = 2^{1/6}\left(\cos\frac{\pi}{12} + i\sin\frac{\pi}{12}\right), x_2 =
  2^{1/6}\left(\cos\frac{9\pi}{12} + i\sin\frac{9\pi}{12}\right)$ and $x_3 =
  2^{1/6}\left(\cos\frac{17\pi}{12} + i\sin\frac{17\pi}{12}\right)$

  $\Rightarrow x_1x_2x_3 = \sqrt{2}\left(\cos\frac{27\pi}{12} + i\sin\frac{27\pi}{12}\right)
  = \sqrt{2}\left(\cos\frac{\pi}{4} + i\sin\frac{\pi}{4}\right) = 1 + i$.
  %45
\item We have $1 = \cos0 + i\sin0 = \cos2k\pi + i\sin2k\pi \Rightarrow 1^{1/n} = \cos\frac{2k\pi}{n} +
  i\sin\frac{2k\pi}{n}$, where $k = 0, 1, 2, \ldots, n - 1$

  Putting $k = 0, 1, 2, \ldots, n - 1$ gives us

  $x_1 = \cos0 + i\sin 0 = 1, x_2 = \cos\frac{2\pi}{n} + i\sin\frac{2\pi}{n} = \omega$(let)

  $x_3 = \cos\frac{4\pi}{n} + i\sin\frac{4\pi}{n} = \left(\cos\frac{2\pi}{n} + i\sin\frac{2\pi}{n}\right)^2
  = \omega^2$

  Similarly, $x_4 = \omega^3$ and so on; proving that the roots are in G.P.
  %46
\item Given equation is $x^7 + x^4 + x^3 + 1 = 0 \Rightarrow \left(x^3 + 1\right)\left(x^4 + 1\right) = 0$

  $x^3 + 1 = 0 \Rightarrow x = -1^{1/3}$. Now $-1 = \cos\pi + i\sin\pi = \cos(2k + 1)\pi + i\sin(2k + 1)\pi$

  $\Rightarrow (-1)^{1/3} = \cos\frac{(2k + 1)\pi}{3} + i\sin\frac{(2k + 1)\pi}{3}$, where $k= 0, 1, 2$.

  Putting $k= 0, 1, 2$ gives us

  $x_1 = \cos\frac{\pi}{3} + i\sin\frac{\pi}{3} = \frac{1}{2} + i\frac{\sqrt{3}}{2}$

  $x_2 = \cos\frac{3\pi}{3} + i\sin\frac{3\pi}{3} = -1$

  $x_3 = \cos\frac{5\pi}{3} + i\sin\frac{5\pi}{3} = \frac{1}{2} - i\frac{\sqrt{3}}{2}$

  $x^4 + 1 = 0 \Rightarrow x = (-1)^{1/4} = \cos\frac{(2k + 1)\pi}{4} + i\sin\frac{(2k + 1)\pi}{4}$, where
  $k= 0, 1, 2, 3$.

  Putting $k= 0, 1, 2, 3$ gives us

  $\Rightarrow x_4 = \cos\frac{\pi}{4} + i\sin\frac{\pi}{4} = \frac{1}{\sqrt{2}} + i\frac{1}{\sqrt{2}}$

  $x_5 = \cos\frac{3\pi}{4} + i\sin\frac{3\pi}{4} = -\frac{1}{\sqrt{2}} + i\frac{1}{\sqrt{2}}$

  $x_6 = \cos\frac{5\pi}{4} + i\sin\frac{5\pi}{4} = -\frac{1}{\sqrt{2}} - i\frac{1}{\sqrt{2}}$

  $x_7 = \cos\frac{7\pi}{4} + i\sin\frac{7\pi}{4}= \frac{1}{\sqrt{2}} - i\frac{1}{\sqrt{2}}$.
  %47
\item Given $x^6 + x^5 + x^4 + x^3 + x^2 + x + 1 = 0\Rightarrow (x - 1)\left(\frac{x^7 - 1}{x - 1}\right) =
  0$

  $\Rightarrow x^7 - 1 = 0 \Rightarrow x = 1^{1/7}$

  $1 = \cos0 + i\sin0 \Rightarrow 1 = \cos2k\pi + i\sin2k\pi \Rightarrow x = \cos\frac{2k\pi}{7} +
  i\sin\frac{2k\pi}{7}$, where $k = 0, 1, 2, 3, 4, 5, 6$

  Putting $k = 0, 1, 2, 3, 4, 5, 6$ we obtain the desired roots.
  %48
\item We have obtained the roots in previous problem. Let these roots be $1, z, z^2, \ldots, z^6$.

  Sum of $n$th power of roots is $1 + z^n + z^{2n} + \cdots + z^{6n} = \frac{1 - z^{7n}}{1 - z^n}$

  $z^7 = \left[\cos\frac{2\pi}{7} + i\sin\frac{2\pi}{7}\right]^7  = 1$

  Thus, sum would be zero. Also, $n$ is not a multiple of $7$ so $1 - z^n\neq 0$.
  %49
\item $x^{12} - 1 = 0\Rightarrow x^{12} = 1 = \cos2k\pi + i\sin2k\pi \Rightarrow x = \cos\frac{2k\pi}{12} +
  i\sin\frac{2k\pi}{12}$

  $= \cos\frac{k\pi}{6} + i\sin\frac{k\pi}{6}$, where $k = 0, 1, 2, \ldots, 11$

  Putting the values of $k = 0, 1, 2, \ldots, 11$ we obtain $12$ roots of unity.

  Also given that $x^4 + x^2 + 1 = 0 \Rightarrow (x^2 - 1)(x^4 + x^2 + 1) = 0 \Rightarrow x^6 - 1 = 0$

  $\Rightarrow x = \cos\frac{r\pi}{3} + i\sin\frac{r\pi}{3}$, where $r = 0, 1, 2, \ldots, 5$.

  We can obtain the roots and find that roots of second equation always satisfy the first equation.
  %50
\item Let $x^n = 1 = \cos2k\pi + i\sin2k\pi$ then it will have $n$ roots which will be powers of $\alpha$.

  We have already proven that these roots will form the G.P. Sum of these roots is

  $S = 1 + \alpha + \alpha^2 + \cdots + \alpha^{n- 1} = \frac{1 - \alpha^n}{1 - \alpha}$

  Clearly, $1 - \alpha^n = 0$.
  %51
\item We have shown that sum is zero in previous problem. The product is given by

    $P = \cos\left(0 + \frac{2\pi}{n} + \frac{4\pi}{n} + \cdots + \frac{2(n- 1)\pi}{2}\right) + i\sin \cos\left(0
  + \frac{2\pi}{n} + \frac{4\pi}{n} + \cdots + \frac{2(n- 1)\pi}{2}\right)$

  $= \cos\left[\frac{n - 1}{2}\left(\frac{4\pi}{n} + (n - 2).\frac{2\pi}{n}\right)\right] +
  i\sin\left[\frac{n - 1}{2}\left(\frac{4\pi}{n} + (n - 2).\frac{2\pi}{n}\right)\right] = (-1)^{n - 1}$.
  %52
\item Given that $(x - 1)^n = x^n\Rightarrow \left(1 - \frac{1}{x}\right)^n = 1$. Let $1 - \frac{1}{x} = z$
  then $z_k = \cos\frac{2k\pi}{n} + i\sin\frac{2k\pi}{n} = e^{2i\pi k/n}$

  $1 - \frac{1}{x} = z \Rightarrow x_k = \frac{1}{1 - e^{2i\pi k/n}} = \frac{1}{2}\left[1 +
    i\cot\frac{k\pi}{n}\right]$.
\stopitemize
