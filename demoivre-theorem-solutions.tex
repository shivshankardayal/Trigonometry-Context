% -*- mode: context; -*-
\chapter{De Moivre's Theorem}
\startitemize[n, 1*broad]
% 1
\item L.H.S. $= \left(\frac{\cos\theta + i\sin\theta}{\sin\theta + i\cos\theta}\right)^4 = \frac{(\cos\theta
    + i\sin\theta)}{[i(\cos\theta - i\sin\theta)]^4} = \frac{(\cos\theta + i\sin\theta)^4}{(\cos\theta -
    i\sin\theta)^4}[\because i^4 = 1]$

  $= \frac{\cos4\theta + i\sin4\theta}{\cos4\theta - i\sin4\theta}.\frac{\cos4\theta +
    i\sin4\theta}{\cos4\theta + i\sin4\theta} = (\cos4\theta + i\sin4\theta)^2$

  $= \cos8\theta + i\sin8\theta =$ R.H.S.
  % 2
\item L.H.S. $= \frac{(\cos\theta + i\sin\theta)^4}{(\sin\theta + i\cos\theta)^5} = \frac{(\cos\theta +
    i\sin\theta)^4}{i^5(\cos\theta - i\sin\theta)^5} = \frac{\cos4\theta + i\sin4\theta}{i(\cos5\theta -
    i\sin5\theta)}$

  $= -i(\cos4\theta + i\sin4\theta)(\cos5\theta - i\sin5\theta)^{-1} = i(\cos4\theta +
  i\sin4\theta)(\cos5\theta + i\sin5\theta)$

  $= -i(\cos9\theta + i\sin9\theta) = \sin9\theta - i\cos9\theta =$ R.H.S.
  % 3
\item Given $\frac{(\cos3\theta + i\sin3\theta)^5(\cos\theta - i\sin\theta)^3}{(\cos5\theta +
  i\sin5\theta)^7(\cos2\theta - i\sin2\theta)^5} = \frac{(\cos15\theta + i\sin15\theta)(\cos3\theta -
  i\sin3\theta)}{(\cos35\theta + i\sin35\theta)(\cos10\theta - i\sin10\theta)}$

$= \frac{\cos12\theta + i\sin12\theta}{\cos25\theta + i\sin25\theta} = \cos13\theta - \sin13\theta$.
%4
\item $(\sin15^\circ - \cos15^\circ) + i(\sin75^\circ - \cos75^\circ) = (\sin15^\circ - \sin75^\circ) +
  i(\cos15^\circ - \cos75^\circ)$

  $= 2\cos45^\circ\sin(-30^\circ) + i.2\sin45^\circ\sin30^\circ = -\cos45^\circ + i\sin45^\circ$

  Thus, $[(\sin15^\circ - \cos15^\circ) + i(\sin75^\circ - \cos75^\circ)]^n = (-1)^n\left[\cos\frac{n\pi}{4}
    - i\sin\frac{n\pi}{4}\right]$

  Similarly, $[(\sin15^\circ - \cos15^\circ - i(\cos75^\circ - \cos75^\circ))]^n =
  (-1)^n\left[\cos\frac{n\pi}{4} + i\sin\frac{n\pi}{4}\right]$

  Thus, given expression is $(-1)^n2\cos\frac{n\pi}{4}$.
  % 5
\item L.H.S.\ $= \left(\frac{1 + \cos\theta + i\sin\theta}{1 + \cos\theta - i\sin\theta}\right)^n =
  \left(\frac{2\cos^2\frac{\theta}{2} + i.2\sin\frac{\theta}{2}\cos\frac{\theta}{2}}{2\cos^2\frac{\theta}{2}
      - i.2\sin\frac{\theta}{2}\cos\frac{\theta}{2}}\right)^n$

  $= \left(\frac{\cos\frac{\theta}{2} + i\sin\frac{\theta}{2}}{\cos\frac{\theta}{2} -
    \sin\frac{\theta}{2}}\right)^n = \left(\cos\frac{n\theta}{2} +
  i\sin\frac{n\theta}{2}\right)\left(\cos\frac{n\theta}{2} + i\sin\frac{n\theta}{2}\right)$

  $= \cos n\theta + i\sin n\theta$.
  %6
\item $(1 + \cos\theta + i\sin\theta)^n = \left(2\cos^2\frac{\theta}{2} +
  i.2\sin\frac{\theta}{2}\cos\frac{\theta}{2}\right)^n$

  $= 2^n\cos^n\frac{\theta}{2}\left(\cos\frac{n\theta}{2} + i\sin\frac{n\theta}{2}\right)$

  Similarly, $(1 + \cos\theta - i\sin\theta)^n = 2^n\cos^n\frac{\theta}{2}\left(\cos\frac{n\theta}{2} -
  i\sin\frac{n\theta}{2}\right)$

  Thus, $(1 + \cos\theta + i\sin\theta)^n(1 + \cos\theta - i\sin\theta)^n = 2^{n +
    1}\cos^n\frac{\theta}{2}\cos\frac{n\theta}{2}$.
\stopitemize