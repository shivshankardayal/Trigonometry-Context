% -*- mode: context; -*-
\chapter{Measurement of Angles}
\startitemize[n, 1*broad]
\item $51'' = \left(\frac{51}{50}\right)' = 0.85'$

   $14'51'' = 14.85' = \left(\frac{14.85}{60}\right)^\circ = 0.2475^\circ$

   $63^\circ14'15'' = 63.2475^\circ$

   Now we can use the formula $\frac{D}{180} = \frac{G}{200},$ substituting the value of $D,$ we obtain

   $G = \left(\frac{63.275*200}{180}\right)^g = 70.275^g$

   $0.275^g = 27.5', 0.5' = 50''$

   Thus, angle in centisiaml measure is $70^g27'50''$

\item $10'' = \left(\frac{10}{60\times 60}\right)^\circ , 20' = \left(\frac{20}{60}\right)^\circ, 45^\circ 20'10'' = \left(45 +
   \frac{1}{3} + \frac{1}{360}\right)^\circ = \frac{16321}{360}$

   Using formula formula $\frac{D}{180} = \frac{G}{200},$

   i. $\left(\frac{16321}{360}\right)^\circ = \frac{16321}{360}.\frac{10}{9} = \left(\frac{16321}{324}\right)^g$

      $= 50.3734^g = 50^g37'34''$

   ii. $\left(\frac{16321}{360}\right)^\circ = \frac{16321}{360}.\frac{\pi}{180} = .79$ radians

\item $94^g23'27'' = 0.942327$ right angles $= 0.942327 * 90 = 84.8483^\circ$

   $84.8483^\circ = 84^\circ (0.8683\times 60)' = 84^\circ 48.8898' = 84^\circ48' (0.8898\times 60)'' = 84^\circ 48'53.388''$

\item $(1.2)^c = 1.2 . \frac{180}{\pi} = 68.7272^\circ = 68^\circ (0.7272\times 60)' = 68^\circ (43.63)' =
   68^\circ43'(0.63\times 60)'' = 68^\circ43'37.8''$

\item Since a right angle is $90^\circ \therefore 60^\circ = \frac{60}{90} = \frac{2}{3}$ right angles.

\item $75^\circ15' = 75\left(\frac{15}{60}\right)^\circ = 75.25^\circ = \frac{75.25}{90} = \frac{301}{360}$ right angles.

\item $63^\circ17'25'' = \left(63 + \frac{17}{60} + \frac{25}{60\times 60}\right)^\circ = \left(\frac{45569}{720}\right)^\circ =
   \frac{45569}{720\times90} = \frac{45569}{64800}$ right angles.

\item $130^\circ30' = 130\frac{30}{60} = \left(\frac{261}{2}\right)^\circ = \left(\frac{261}{2\times 90}\right) = \frac{29}{20}$
   right angles.

\item $210^\circ30'30'' = \left(210 + \frac{30}{60} + \frac{30}{60\times 60}\right)^\circ = \frac{25261}{120} =
   \frac{25261}{120\times 90}$ right angles $= \frac{25261}{10800}$ right angles

\item $370^\circ 20'48'' = \left(370 + \frac{20}{60} + \frac{48}{60\times 60}\right) = \left(370 + \frac{1}{3} +
    \frac{1}{75}\right)^\circ$

    $= \left(\frac{27776}{75}\right)^\circ = \frac{27776}{6750}$ right angles

\item $30^\circ = \frac{30}{90}$ right angles $= \frac{1}{3} = 0.333333 = (.333333 \times 100)^g = 33.3333^g =
    33^g(.3333\times 100)'$

    $= 33^g33.33' = 33^g33'(.33\times 100)'' = 33^g33'33''$

\item $81^\circ = \frac{81}{90} = 0.9$ right angles $= 0.9 \times 100 = 90^g$

\item $138^\circ 30' = 138 + \frac{30}{60} = \left(\frac{277}{2}\right)^\circ = \left(\frac{277}{180}\right)$ right angle

    $= 1.5388888$ right angles $= 153^g88'88.8''$

\item $35^\circ47'15'' = \left(35 + \frac{47}{60} + \frac{15}{60\times 60}\right)^\circ = \left(\frac{8589}{240}\right)^\circ$

    $= \left(\frac{8589}{240\times 90}\right)$ right angles $= .3976388$ right angles $=39^g76'38.8''$

\item $235^\circ12'36'' = \left(235 + \frac{12}{60} + \frac{36}{60\times 60}\right)^\circ =
    \left(\frac{23521}{100}\right)^\circ$

    $= \left(\frac{23521}{9000}\right)$ right angles $= 2.6134444$ right angles

    $= 263^g34'44.4''$

\item $475^\circ13'48'' = \left(475 + \frac{13}{60} + \frac{48}{60\times 60}\right)^circ$

    Proceeding like previous problems we obtain the angle as $528^g3'33.3''$

\item $120^g = \frac{120}{100} = 1.2$ right angles $= 1.2\times 90^\circ = 108^\circ$

\item $45^g35'24'' = 45.3524^g = \frac{45.3524}{100} = .453524$ right angles

    $= .453524 \times 90^\circ = 40^\circ19'1.776''$

\item $39^g45'36'' = 39.4536^g = \frac{39.4536}{100}$ right angles $= .394536$ right angles

    $= .394536\times 90^\circ = 35^\circ30'29.664''$

\item $255^g8'9'' = 255.0809^g = \frac{255.0809}{100} = 2.550809$ right angles

    $= 2.550809 \times 90^\circ = 229^\circ34'22.116''$

\item $759^g0'5'' = 7.590005^g = \frac{759.0005}{100} = 7.590005$ right angles

    $= 7.59005\times 90^\circ = 683^\circ6'1.62''$

\item $55^\circ12'36'' = \left(55 + \frac{12}{60} + \frac{36}{60\times 60}\right)^\circ = 55.21^\circ$

    $= \frac{55.21}{90} = .6134444$ right angles $= 61^g34'44.4''$

\item $18^\circ33'45'' = \left(18 + \frac{33}{60} + \frac{45}{60\times60}\right)^\circ = \frac{1485\pi}{80\times180}$ radians

    $= \frac{33\pi}{320}$ radians.

\item $195^g35'24'' = 195.3524^g = 1.953524$ right angles $= 1.953524\times90^\circ = 175^\circ49'1.8''$

\item Minute-hand make $360^\circ$ in $60$ minutes.

    $\therefore$ angle made in $11\frac{1}{9}$ minutes $= \frac{360}{60}.\frac{20}{9} = 66^\circ40'$

    Hour-hand makes $360^\circ$ in $720$ minutes i.e. $12$ times less.

    $\therefore$ angle made by hour-hand $\frac{66^\circ40'}{12} = 5^\circ83'20''$

\item Let the angle in degrees be $x,$ then from problem statement we have

    $x^\circ + x^g = 1$ right angle

    Now we can use the formula $\frac{D}{180} = \frac{G}{200},$

    $x + \frac{9}{10}x = 1$

    $x = 47\frac{7}{19}$

    $\frac{9x}{10} = 42\frac{12}{19}$

\item Let there be $x$ sexagecimal minutes. $x' = \frac{x}{60\times 90}$ right angle

    $x$ centisimal angles $= \frac{x}{100\times100}$ right angles.

    Thus we find that ratio is $27:50$

\item Let there be $x$ seconds in both. Thus, $\left(\frac{x}{1000000}90 + \frac{x}{60\times60}\right)^\circ =
    44^\circ8'$

    Thus, the two parts are $33^\circ20'$ and $10^\circ48'$

\item Let the angles be $3x, 4x, 5x$ in degrees. From Geometry, we know that $3x + 4x + 5x = 180^\circ$ (sum of all
    angles of a triangle is $180^\circ$) $\Rightarrow x = 15^\circ$

    Thus angles are $45^\circ, 60^\circ, 75^\circ$ or $\frac{\pi}{4}, \frac{\pi}{3}, \frac{5\pi}{12}$ radians.

\item At half past $4,$ hour-hand will be at $4\frac{1}{2}$ and minute-hand will be at $6.$

    So the difference is $1\frac{1}{2}$ and $360^\circ$ is divided into $12$ hour parts. So angle for $1$
    hour $= \frac{360}{12} = 30^\circ$

    So for the angle between $4\frac{1}{2}$ and $6$ $= 30\times \frac{3}{2} = 45^\circ = \frac{\pi}{4}$ radians

\item $1$ right angle $= 100^g$

    $1$ radian $= \frac{180^\circ}{\pi}$ or $\pi$ radian $= 180^\circ$

    $180^\circ = 200^g = \pi$ radians

    1. Thus, $\frac{p}{10} = \frac{q}{9} = \frac{20r}{\pi}$

    2. Let $\frac{p}{10} = \frac{q}{9} = \frac{20r}{\pi} = k$

       $p = 10k, q = 9k \Rightarrow p - q = k = \frac{20r}{\pi}$

\item The angles are $72^{\circ}53'51''$ and $41^\circ22'50''.$ Sum of these two angles is $114^\circ16'41''$

    Sum of all angles of a triagle is $180^\circ \therefore$ third angle $= 180^\circ - 114^\circ16'41'' =
    75^\circ43'19''$

    $= 75^\circ43'19'' \frac{\pi}{180}$ radians

\item Let the angles are $a - d, a, a + d$ in degrees $\therefore 3a = 180^\circ \Rightarrow a = 60^\circ$

    Greatest angle in radians $= \frac{(60 + d)\pi}{180}$

    Given that $\frac{(60 + d)\pi}{(60 - d)180} = \frac{\pi}{60}$

    $60 + d = 3(60 - d) \Rightarrow 4d = 120 \Rightarrow d = 30^\circ$

    Thus, the other two angles are $30^\circ$ and $90^\circ.$

\item Let the angles are $a - d, a, a + d$ in degrees $\therefore 3a = 180^\circ \Rightarrow a = 60^\circ$

    Greatest angle in radians $= \frac{(60 + d)\pi}{180}$

    Least angle is $60 - d = (60 - d)\frac{10}{9}$ grades

    Ratio of greatest number of grades in the least to the number of radians in the greatest is $\frac{40}{\pi}$

    $(60 - d)\frac{10}{9}\frac{180}{(60 + d)\pi} = \frac{40}{\pi}$

    $\Rightarrow 5(60 - d) = 60 + d \Rightarrow d = 40^\circ$

    Thus, other two angles are $20^\circ$ and $100^\circ$

\item Let the angles be $\frac{a}{r}, a, ar$ in grades.

    $\frac{a}{r}$ in radians $=\frac{a\pi}{200r}$

    Given that ratio of greatest angle in grades to least angle in radian is $\frac{800}{\pi}$

    $\therefore \frac{ar\times 200r}{a\pi} = \frac{800}{\pi} \Rightarrow r = 2$

    Also given that $\frac{a}{r} + a + ar = 126^\circ = \frac{126\times 10}{9} = 140^g$

    $\frac{a}{2} + a + 2a = 140 \Rightarrow a = 40^g$

    Thus, angles are $20^g, 40^g, 80^g$

\item There are $12$ hours in a clock for an angle of $360^\circ$ therefore each hour subtends an angle of
    $30^\circ.$

    At $4$ o'clock hour-hand will be at $4$ and minute-hand will be at $12$ i.e. a difference of $4$
    hours. Thus angle subtended $= 30\times 4 = 120^\circ = \frac{4\pi}{3}$ radians.

\item At quarter to twelve minute-hand will be at nine and hour hand will be just before twelve. The difference between twelve and
    nine is three hours so angle made will be $3.\frac{360}{12} = 90^\circ.$ For quarter of hour hour-hand will be
    $\frac{30}{4} = 7.5^\circ$ before twleve.

    Thus difference $= 90 - 7.5 = 82.5^\circ$

\item Radius $= \frac{28}{2} = 14$ cm

    Circumference $= 2\pi r = 28\pi$ cm.

    If we take $\pi$ to be $\frac{22}{7}$ distance moved $= 28\frac{22}{7} = 88$ cm.

\item Circumference $= \frac{1760}{5} = 352$ mt.

    Let $r$ be the radius then $2\pi r = 352 \Rightarrow r = \frac{352\times 7}{2\times 22} = 56$ mt.

\item Given $2r = 90$ cm and $3$ revolutions are made per second.

    Circumference $= 2\pi r = \frac{22}{7}90 = 282.86$ cm

    Thus, speed of train $= 3*$ circumference $= 848.57$

\item Total no. of revolutions in an hour $= 60*10 = 600$

    Radius $= 540,$ circumference $= 2\pi r = 1080\pi$ cm

    Distance travelled $= 600\times 1080\pi$ cm $= 20.36$ km/hr

\item Given, radius $= 149,700,000$ km

    Distance travelled in one year $= 2\pi r = 940,600,000$ km.(approximately)

\item Angle subtended in $1$ second $= 9\times 80 = 720^\circ$ i.e. $2$ revolutions.

    Distance travelled by the point on rim per second $= 2\times 2\pi 50$ cm

    Distance travelled by the point on rim per hour $= 3600\times 200\times \pi$ cm $= 23$ km approximately.

\item By Geometry, we know that all the interior angles of any rectilinear figure together with four right angles are equal to twice
    as many right angles as the figure has sides.

    Let the angle of a regular decagon contain $x$ right angles, so that all the angles togethe equal to $10x$ right
    angles.

    The corollary states that

    $10 x + 4 = 20$ so that $x = \frac{8}{5}$ right angles.

    But one right angle $= 90^\circ = 100^g = \frac{\pi}{2}$ radians

    Hence the required angle $= 155^\circ= 160^g = \frac{4\pi}{5}$ radians.

\item $\frac{2}{3}x$ grades $= \frac{2}{3}x\frac{9}{10} = \frac{3}{5}x$ degrees.

    $\frac{\pi x}{75}$ radians $= \frac{\pi x}{75}\frac{180}{\pi} = \frac{12}{5}x$ degrees

    Sum of all angles $= \frac{3}{2}x + \frac{3}{5}x + \frac{12}{5}x = 4.5x = 180^\circ$

    $\Rightarrow x = 45^\circ$

    Thus, angles are $24^\circ, 60^\circ, 96^\circ$

\item Let the third angle be $x$ radians. $x = \pi - \frac{1}{2} - \frac{1}{3} = \frac{6\pi - 5}{6}$ radians

    $= \left(\frac{6\pi - 5}{6}. \frac{180}{\pi}\right)^\circ = 132^\circ14'12.5''$

\item Let the angles are $a - d, a, a + d$ in radians. We know that sum of all angles of a triangle is $\pi$ radians.

    $\Rightarrow 3a = \pi \Rightarrow a = \frac{\pi}{3}$ radians

    Given, the number of radians in the least angle is to the number of degree in the mean angle is $1:120$

    Mean angle in degrees $= \frac{180a}{\pi}$

    $\therefore \frac{(a - d)\pi}{180a} = \frac{1}{120} \Rightarrow \frac{(\frac{\pi}{3} - d)\pi}{3\frac{\pi}{3}} =
    \frac{1}{2}$

    $d = \frac{\pi}{3} - \frac{1}{2}$

    Thus, angles are $\frac{1}{2}, \frac{\pi}{3}, \frac{2\pi}{3}- \frac{1}{2}$ radians.

\item Let us solve these one by one:

    1. We know that if polygon has $n$ sides then sum of angles is $(n - 2)\pi$ radians or $(n - 2)180^\circ$

       For pentagon sum of angles $= 3\pi$ or $540^\circ$

       Measure of one interior angle $= \frac{3\pi}{5}$ or $108^\circ$

    2. Measure of one interior angle $= \frac{5\pi}{7}$ or $\frac{900^\circ}{7}$


    3. Measure of one interior angle $= \frac{3\pi}{4}$ or $135^\circ$

    4. Measure of one interior angle $= \frac{5\pi}{6}$ or $150^\circ$

    5. Measure of one interior angle $= \frac{15\pi}{17}$ or $\frac{2700^\circ}{17}$

\item Let there be $n$ sides in one polygon and $2n$ in another. Angles will be $\frac{(n - 2)\pi}{n}$ and
    $\frac{(2n - 2)\pi}{2n} - \frac{(n - 1)\pi}{n}$

    Given ratio of angles $\frac{3}{2} = \frac{n -1}{n - 2} \Rightarrow 3n - 6 = 2n -2 \Rightarrow n = 4.$ So one polygon is
    a square while the other is an octagon.

\item Let number of sides in one polygon be $n$ then number of sides in another $\frac{5n}{4}.$ The angles will be
    $\frac{(n - 2)}{n}180^\circ, \frac{(\frac{5n}{4} - 2)}{\frac{5n}{4}}180^\circ.$

    Given that difference in angles is $9^\circ$

    Solving for difference we find $n = 8$ and thus the other polygon will have $10$ sides.

\item Let no. of sides in one polygon be $n$ then number of sides in another $\frac{3n}{4}$. The angles will be
    $\frac{(n - 2)}{n}180^\circ, \frac{(\frac{3n}{4} -2)}{\frac{3n}{4}}180^\circ.$

    To convert no. of degrees to no. of grades we multiply the angle with $\frac{10}{9}$ and then comparing ratio to
    $4:5$ we find that $n = 8$ and $\frac{3n}{4}$ i.e. $6.$

\item Let the angles be $a - 3d, a - d, a + d, a + 3d$ in radians. We know that sum of all angles of a quadrilateral $=
    (4 - 2)\pi = 2\pi$ radians.

    $\therefore 4a = 2\pi \Rightarrow a = \frac{\pi}{2}$

    Also given that greatest angle is double of least angle. $\therefore (a + 3d) = 2a - 6d \Rightarrow 9d = a \Rightarrow d
    = \frac{\pi}{18}$ radian.

    Least angle $= \frac{\pi}{2} - \frac{3\pi}{18} = \frac{\pi}{3}$

\item Let us solve these one by one;

    1) At half-past three the difference between hour-hand and minute hand will be two and half hours. Each hour makes an angle of
       $30^\circ$ so two and half hours will make an angle of $2.5\times 30^\circ = 75^\circ = 75.\frac{10}{9} =
       \frac{250^g}{3} = \frac{75\pi}{180}$ radians

    2) At twenty minutes to six the difference between hour and and minute hands will be two hours and twnenty minutes i.e
       $2\frac{1}{3} = \frac{7}{3}$ So angle made will be $\frac{7}{3}\times 30^\circ = 70^\circ$ which you can convert
       in grades and radians.

    3) At quarter part eleven the difference between hour-hand and minute hand will be $3 + \frac{3}{4}$ hours. Now you can
       solve it like previous parts.

\item Let us solve these one by one:

    1. There are two cases here.

       {\bf Case I:} When minute hand is between twelve and four.

       Let the minute hand is at $x$ minute mark. Four makes an angle of $3\times 30^\circ$ with twelve hour as each
       hour makes an angle of $360^\circ/12 = 30^\circ.$ Angle made by minute-hand at $x$ minute $=6x^\circ$
       since each minute make an angle of $360^\circ/60 = 6^\circ.$ Extra angle made by hour hand w.r.t four due to these
       $x$ minute $=x\times 30/60 = x/2$ [Each hour has $60$ minutes and for those it makes angle $x^\circ$
       ]

       Thus, $\frac{x}{2} + 120 - 6x = 78 \Rightarrow \frac{11x}{2} = 42 \Rightarrow x = \frac{84}{11}$ minutes past four.

       {\bf Case II:} When minute hand is between five and twelve.

       Let the minute hand is at $x$ minute mark. Proceeding like previous problem:

       $6x + \frac{x}{2} - 150 = 78 \Rightarrow \frac{13x}{2} = 228 \Rightarrow x = \frac{472}{13}$ minutes past four.

    2. This can be solved like 1.

\item Let there are $n$ sides in the polygon. Sum of all angles $= (n - 2)180^\circ$

    But angles are in A.P. so sum of series $= \frac{n}{2}[240 + (n - 1)5]$

    Equating $(2n - 4)180 = 235n + 5n^2 \Rightarrow 5n^2 - 125n + 720 = 0$

    $\Rightarrow n^2 - 25n + 144 = 0 \Rightarrow n = 9, 16$

    However, if $n = 16,$ greatest angle $= 120 + 18\times 5 = 210^\circ$ which is not possible.

    $\therefore n = 9$

\item Let the angles be $a - 3d, a - d, a + d, a + 3d$ in degrees.

    Sum of all angles of quadrilaters $4a = 360^\circ \Rightarrow a = 90^\circ$

    Given that ration of least angle in grades to greatest angle in radians is $100:\pi$

    $\frac{(a - 3d)10\times 180}{9(a + 3d)} = \frac{100}{\pi} \Rightarrow 2a - 6d = a + 3d \Rightarrow d = \frac{a}{9} = 10^\circ$

    So the angles are $60^\circ, 80^\circ, 100^\circ, 120^\circ$

\item Let there are $n$ side in the polygon. Sum of all angles $= (n - 2)180^\circ$

    Smallest angle $= \frac{5\pi}{12} = 75^\circ$ c.d. $= 10^\circ$

    Sum of all angles in A.P. $= \frac{n}{2}[150 + (n - 1)10]$

    Equating $(2n - 4)18 = 14n + n^2 \Rightarrow n^2 - 22n + 72 = 0, n = 4, 18$

    Largest angle in case of $n = 18,$ is $75 + 170 = 245$ which is not possible. $\therefore n = 4$

\item Angle subtended $\theta = \frac{l}{r} = \frac{1}{3}$ radians

\item $l = \theta . r = 33.25^\circ \times 5 = 33.25\times 5 \times \frac{\pi}{180}$

\item Sun's diameter $= \theta .r = \frac{32}{60}\times 14970000\times \frac{\pi}{180}$

\item Minimum angle needed for the person to be able to read $= 5'$

    1. Height of letters $= \frac{12}{5'} = \frac{12}{\frac{5}{60}\frac{180}{\pi}}$

    2. Can be solved like 1.

\item $\theta = \frac{l}{r} = 0.357\times \frac{180}{\pi}$ degrees

\item Angle subtended in radian $= \frac{15}{25} = \frac{3}{5}$

    Angle subtended in degrees $= \frac{3}{5}\frac{180}{\pi}$

    Angle subtended in grades $= \frac{108}{\pi}\frac{10}{9} = \frac{120}{\pi}$

\item Radius of circle $r = \frac{\theta}{l} = \frac{5}{60}\frac{\pi}{180}.1$ cm

\item $l = \theta \times r = \frac{5}{60}\frac{\pi}{180}36$ cm

\item $r = \frac{l}{\theta} = .5 \frac{10}{1}\frac{180}{\pi}$ cm

\item $\theta = \frac{l}{r} = \frac{100}{6400} = \frac{1}{64}$ radians

\item $r = \frac{l}{\theta} = \frac{139}{2}.\frac{180}{\pi}$ km.

\item $\frac{r_1}{r_2} = \frac{l}{\theta_1}\frac{\theta_2}{l} = \frac{75}{60} = \frac{5}{4}$

\item $r = \frac{l}{\theta} \Rightarrow 4 = \frac{10}{\left(143 + \frac{14}{60} + \frac{22}{60\times
    60}\right)}\frac{180}{\pi}$

    $\pi = 3.1416$

\item Let the parts subtend angles of $a - 2d, a - d, a, a + d, a + 2d$ in radians.

    Total angle subtended $= 2\pi$

    $\Rightarrow 5a = 2\pi \Rightarrow a = \frac{2\pi}{5}$

    Also, given that greatest is six times the least $a + 2d = 6(a - 2d) \Rightarrow 5a = 14d \Rightarrow d = \frac{5a}{14} =
    \frac{\pi}{7}$

    Now angles can be found by simple calculation.

\item Length of semicircle is $\pi r$. Let length of curved part of sector be $l$ then perimeter of sector is $l +
    2r$

    $l = (\pi - 2)r,$ angle subtended by sector $\theta = \frac{l}{r} = \pi - 2$ radians, which can be converted in
    degrees.

\item $r = \frac{\theta}{l} = \frac{10}{60}\frac{\pi}{180}2$ m.

\item $l = r\theta = 5280.\frac{1}{60}\frac{\pi}{180}$

\item $l = r\theta = 38400\frac{31}{60}\frac{\pi}{180}$ km.

\item Distance travelled in $1$ second $= 2\pi r \times 6 = 24\pi$ ft/sec.

\item $r = \frac{l}{\theta} = 1\frac{60}{30}\frac{180}{\pi}$ in.

\item No. of revolutions made in $1$ second $= \frac{30}{60} = \frac{1}{2}$

    Therefore angle subtended $= \frac{2\pi}{2} = \pi$ radians.

\item Length of arc $= 10 \times \frac{36}{3600} = \frac{1}{10}$ miles.

    $d = 2\frac{l}{\theta} = \frac{2}{10}\frac{180}{56\pi}$ miles
\stopitemize
