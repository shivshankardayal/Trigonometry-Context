% -*- mode: context; -*-
\chapter{Compound Angles}
\startitemize[n, 1*broad]
\item Given, $\sin\alpha = \frac{3}{5}$ and $\cos\beta = \frac{9}{41}$

  Therefore, $\cos\alpha = \frac{4}{5}$ and $\sin\beta = \frac{40}{41}$

  $\sin(\alpha - \beta) = \sin\alpha\cos\beta - \cos\alpha\sin\beta$

  $= \frac{3}{5}\frac{9}{41} - \frac{4}{5}\frac{40}{41}$

  $= \frac{27 - 160}{205} = -\frac{133}{205}$

  $\cos(\alpha + \beta) = \cos\alpha\cos\beta - \sin\alpha\sin\beta$

  $= \frac{4}{5}\frac{9}{41} - \frac{3}{5}\frac{40}{41}$

  $= \frac{36 - 120}{205} = -\frac{84}{205}$

\item Given, $\sin\alpha = \frac{45}{53}$ and $\sin\beta = \frac{33}{65}$

  Thus, $\cos\alpha = \frac{28}{53}$ and $\cos\beta = \frac{56}{65}$

  $\sin(\alpha - \beta) = \sin\alpha\cos\beta - \cos\alpha\sin\beta$

  $= \frac{45}{53}\frac{56}{65} - \frac{28}{53}\frac{33}{65}$

  $= \frac{2520 - 924}{3445} = \frac{1596}{3445}$

  $\sin(\alpha + \beta) = \sin\alpha\cos\beta + \cos\alpha\sin\beta$

  $= \frac{45}{53}\frac{56}{65} + \frac{28}{53}\frac{33}{65}$

  $= \frac{2520 + 924}{3445} = \frac{3444}{3445}$

\item Given, $\sin\alpha = \frac{15}{17}$ and $\cos\beta = \frac{12}{13}$

  $\cos\alpha = \frac{8}{17}$ and $\sin\beta = \frac{5}{13}$

  $\tan\alpha = \frac{15}{8}$ and $\tan\beta = \frac{5}{12}$

  $\sin(\alpha + \beta) = \sin\alpha\cos\beta + \cos\alpha\sin\beta$

  $= \frac{15}{17}\frac{12}{13} + \frac{8}{17}\frac{5}{13}$

  $= \frac{220}{221}$

  $\cos(\alpha - \beta) = \cos\alpha\cos\beta + \sin\alpha\sin\beta$

  $= \frac{8}{17}\frac{12}{13} + \frac{15}{17}\frac{5}{13}$

  $= \frac{171}{221}$

  $\tan(\alpha + \beta) = \frac{\tan\alpha + \tan\beta}{1 - \tan\alpha\tan\beta}$

  $= \frac{\frac{15}{8} + \frac{5}{12}}{1 - \frac{15}{8}\frac{5}{12}}$

  $= \frac{220}{21}$

\item L.H.S. $= \cos(45^{\circ} - A)\cos(45^{\circ} - B) - \sin(45^{\circ} - A)\sin(45^{\circ} - B)$

  $= [(\cos 45^\circ\cos A + \sin45^\circ\sin A)(\cos 45^\circ\cos B + \sin45^\circ\sin B) - (\sin45^\circ\cos A -
  \cos45^\circ\sin A)(\sin45^\circ\cos B - \cos45^\circ\sin B)]$

  Substituting valus for $\sin45^\circ$ and $\cos45^\circ$

  $=\left[\left(\frac{\cos A}{\sqrt{2}} + \frac{\sin A}{\sqrt{2}}\right)\left(\frac{\cos B}{\sqrt{2}} + \frac{\sin
      B}{\sqrt{2}}\right)\right] - \left[\left(\frac{\cos A}{\sqrt{2}} - \frac{\sin A}{\sqrt{2}}\right)\left(\frac{\cos B}{\sqrt{2}} -
    \frac{\sin B}{\sqrt{2}}\right)\right]$

  $= \left[\frac{\cos A\cos B}{2} + \frac{\cos A\sin B}{2} + \frac{\sin A\cos B}{2} + \frac{\sin A\sin B}{2}\right] -
  \left[\frac{\cos A\cos B}{2} - \frac{\cos A\sin B}{2} - \frac{\sin A\cos B}{2} + \frac{\sin A\sin B}{2}\right]$

  $= \sin A\cos B + \cos A\sin B = \sin(A + B)$

\item L.H.S. $= \sin(45^{\circ} + A)\cos(45^\circ - B) + \cos(45^{\circ} + A)\sin(45^\circ - B)$

  $= [(\sin45^\circ\cos A + \cos45^\circ\sin A)(\cos45^\circ\cos B + \sin45^\circ\sin B) + (\cos45^\circ\cos A -
  \sin45^\circ\sin A)(\sin45^\circ\cos B - \cos45^\circ)\sin B]$

  $= \left[\left(\frac{\cos A}{\sqrt{2}} + \frac{\sin A}{\sqrt{2}}\right)\left(\frac{\cos B}{\sqrt{2}} + \frac{\sin
    B}{\sqrt{2}}\right) + \left(\frac{\cos A}{\sqrt{2}} - \frac{\sin A}{\sqrt{2}}\right)\left(\frac{\cos B}{\sqrt{2}} - \frac{\sin
    B}{\sqrt{2}}\right)\right]$

  $= \left[\frac{\cos A\cos B}{2} + \frac{\cos A\sin B}{2} + \frac{\sin A\cos B}{2} + \frac{\sin A\sin B}{2} +
  \frac{\cos A\cos B}{2} - \frac{\cos A\sin B}{2} - \frac{\sin A\cos B}{2} + \frac{\sin A\sin B}{2}\right]$

  $= \cos A\cos B + \sin A\sin B = \cos(A - B)$

\item L.H.S. $= \frac{\sin(A - B)}{\cos A\cos B} + \frac{\sin(B - C)}{\cos B\cos C} + \frac{\sin(C - A)}{\cos C\cos A}$

  $= \frac{\sin A\cos B - \cos A\sin B}{\cos A\cos B} + \frac{\sin B\cos C - \cos B\sin C}{\cos B\cos C} + \frac{\sin C\cos
  A - \cos C\sin A}{\cos C\cos A}$

  $= \tan A - \tan B + \tan B - \tan C + \tan C - \tan A = 0 =$ R.H.S.

\item L.H.S. $= \sin 105^\circ + \cos 105^\circ$

  $= \sin(60^\circ + 45^\circ) + \cos(60^\circ + 45^\circ)$

  $=\sin60^\circ\cos45^\circ + \cos60^\circ\sin45^\circ + \cos60^\circ\cos45^\circ - \sin60^\circ\sin45^\circ$

  $=\cos45^\circ(\sin60^\circ + \cos60^\circ + \cos60^\circ - \sin 60^\circ)[\because \sin45^\circ = \cos45^\circ]$

  $= \cos45^\circ =$ R.H.S.

\item Given, $\sin 75^\circ - \sin 15^\circ = \cos 105^\circ + \cos 15^\circ$

  $\Rightarrow \sin75^\circ -\sin15^\circ = \cos(90^\circ + 15^\circ) + \sin(90^\circ - 15^\circ)$

  $\Rightarrow \sin75^\circ -\sin15^\circ = \cos90^\circ\cos15^\circ - \sin90^\circ\sin15^\circ + \sin(90^\circ - 15^\circ)$

  $\Rightarrow \sin75^\circ -\sin15^\circ = -\sin15^\circ + \sin75^\circ [\because \cos90^\circ = 0~\&~\sin90^\circ = 1]$

  Thus, we have proven the equality.

\item L.H.S. $= \cos\alpha\cos(\gamma - \alpha) - \sin\alpha\sin(\gamma - \alpha)$

  $= \cos\alpha(\cos\gamma\cos\alpha + \sin\gamma\sin\alpha) - \sin\alpha(\sin\gamma\cos\alpha - \sin\alpha\cos\gamma)$

  $= \cos^2\alpha\cos\gamma + \sin\gamma\sin\alpha\cos\alpha - \sin\alpha\sin\gamma\cos\alpha - \sin^2\alpha\cos\gamma$

  $= \cos\gamma(\sin^2\alpha + \cos^2\alpha) = \cos\gamma =$ R.H.S.

\item L.H.S. $= \cos(\alpha + \beta)\cos\gamma - \cos(\beta + \gamma)\cos\alpha$

  $= \cos\alpha\cos\beta\cos\gamma - \sin\alpha\sin\beta\cos\gamma - \cos\alpha\cos\beta\cos\gamma +
  \cos\alpha\sin\beta\sin\gamma$

  $= \sin\beta(\cos\alpha\sin\gamma - \sin\alpha\cos\gamma)$

  $= \sin\beta\sin(\gamma - \alpha) =$ R.H.S.

\item L.H.S. $= \sin(n + 1)A\sin(n - 1)A + \cos(n + 1)A\cos(n - 1)A$

  $= \cos(n + 1 - (n - 1))A = \cos 2A =$ R.H.S.

\item L.H.S. $= \sin(n + 1)A\sin(n + 2)A + \cos(n + 1)A\cos(n + 2)A$

  $= \cos(n + 1 - (n + 1)) = \cos A =$ R.H.S.

\item $\cos 15^\circ = \cos(45^\circ - \cos 30^\circ)$

  $= \cos45^\circ\cos30^\circ + \sin45^\circ\sin30^\circ$

  $= \frac{1}{\sqrt{2}}\frac{\sqrt{3}}{2} + \frac{1}{\sqrt{2}}\frac{1}{2}$

  $= \frac{\sqrt{3} + 1}{2\sqrt{2}}$

  $\sin 105^\circ = \sin(60^\circ + 45^\circ)$

  $= \sin60^\circ\cos^45\circ + \cos60^\circ\sin45^\circ$

  $= \frac{\sqrt{3}}{2}\frac{1}{\sqrt{2}} + \frac{1}{2}\frac{1}{\sqrt{2}}$

  $= \frac{\sqrt{3} + 1}{2\sqrt{2}} =$ R.H.S.

\item $\tan 105^\circ = \tan(60^\circ + 45^\circ)$

  $= \frac{\tan60^\circ + \tan 45^\circ}{1 - \tan60^\circ\tan 45^\circ}$

  $= \frac{\sqrt{3} + 1}{1 - \sqrt{3}}$

\item $\frac{\tan 495^\circ}{\cot 855^\circ} = \frac{\tan(360^\circ + 135^\circ)}{\cot(720^\circ + 135^\circ)}$

  $= \frac{\tan 135^\circ}{\cot 135^\circ} = \tan^2135^\circ = (-1)^2 = 1$

\item $\sin(\pi + \theta) = -\sin\theta \therefore \sin(n\pi + \theta) = (1)^n\sin\theta$

  $\sin\left(n\pi + (-1)^n \frac{\pi}{4}\right) = (-1)^n\sin\left((-1)^n\frac{\pi}{4}\right)$

  $= (-1)^n(-1)^n\sin \frac{\pi}{4}~[\because \sin(-\theta) = -\sin\theta]$

  $= (-1)^{2n}\sin \frac{\pi}{4} = \frac{1}{\sqrt{2}}$

\item L.H.S. $= \sin 15^\circ = \sin(60^\circ - 45^\circ)$

  $= \sin60^\circ\cos45^\circ - \cos60^\circ\sin45^\circ$

  $= \frac{\sqrt{3}}{2}\frac{1}{\sqrt{2}} - \frac{1}{2}\frac{1}{\sqrt{2}}$

  $= \frac{\sqrt{3} - 1}{2\sqrt{2}} =$ R.H.S.

\item L.H.S. $= \cos 75^\circ = \cos(45^\circ + 30^\circ)$

  $= \cos45^\circ\cos30^\circ - \sin45^\circ\sin30^\circ$

  $= \frac{1}{\sqrt{2}}\frac{\sqrt{3}}{2} - \frac{1}{\sqrt{2}}\frac{1}{2}$

  $= \frac{\sqrt{3} - 1}{2\sqrt{2}} =$ R.H.S.

\item L.H.S. $= \tan 75^\circ = \tan(45^\circ + 30^\circ)$

  $= \frac{\tan45^\circ + \tan30^\circ}{1 - \tan45^\circ\tan30^\circ}$

  $= \frac{1 + \frac{1}{\sqrt{3}}}{1 - 1.\frac{1}{\sqrt{3}}}$

  $= \frac{\frac{\sqrt{3} + 1}{\sqrt{3}}}{\frac{\sqrt{3} - 1}{\sqrt{3}}}$

  $= \frac{\sqrt{3} + 1}{\sqrt{3} - 1} = \frac{(\sqrt{3} + 1)^2}{3 - 1} = 2 + \sqrt{3} =$ R.H.S.

\item L.H.S. $= \tan 15^\circ = \tan(45^\circ - 30^\circ)$

  $= \frac{\tan45^\circ - \tan30^\circ}{1 + \tan45^\circ\tan30^\circ}$

  $= \frac{1 - \frac{1}{\sqrt{3}}}{1 + 1.\frac{1}{\sqrt{3}}}$

  $= \frac{\frac{\sqrt{3} - 1}{\sqrt{3}}}{\frac{\sqrt{3} + 1}{\sqrt{3}}}$

  $= \frac{\sqrt{3} - 1}{\sqrt{3} + 1} = \frac{(\sqrt{3} - 1)^2}{3 - 1} = 2 - \sqrt{3} =$ R.H.S.

\item $\cos 1395^\circ = \cos(3*360^\circ + 315^\circ) = \cos315^\circ = \cos(270^\circ + 45^\circ)$

  $= \cos45^\circ = \frac{1}{\sqrt{2}}$

\item $\tan(-330^\circ) = -\tan(330^\circ) = -\tan(270^\circ + 60^\circ) = =\cot 60^\circ = \frac{1}{\sqrt{3}}$

\item Given, $\sin 300^\circ \csc 1050^\circ - \tan(-120^\circ)$

  $= \sin (270^\circ + 30^\circ)\csc(720^\circ + 270^\circ + 60^\circ) + \tan(90^\circ + 30^\circ)$

  $= -cos 30^\circ. \frac{1}{-cos 60^\circ} - \cot 30^\circ$

  $= \frac{\sqrt{3}}{2}.\frac{2}{1} - \sqrt{3} = 0$

\item Given, $\tan\left(\frac{11\pi}{12}\right)$

  $= \tan\left(\pi - \frac{\pi}{12}\right) = -\tan15^\circ$

  Using the value computed in 20 for $\tan15^\circ$ we have $\sqrt{3} - 2$ as the answer.

\item We know that $tan(-\theta) = -\tan\theta,$ thus

  $\tan \left((-1)^n\frac{\pi}{4}\right) = (-1)^n\tan\frac{\pi}{4} = (-1)^n$

\item Given, $\cos 18^\circ - \sin 18^\circ = \sqrt{2}\sin 27^\circ$

  $\frac{1}{\sqrt{2}}\cos 18^\circ - \frac{1}{\sqrt{2}}\sin18^\circ = \sin 27^\circ$

  L.H.S. $= \sin45^\circ\cos18^\circ - \cos45^\circ\sin18^\circ$

  $= \sin(45^\circ - 18^\circ) = \sin 27^\circ =$ R.H.S.

\item L.H.S. $=\tan 70^\circ = \tan(50^\circ + 20^\circ)$

  $= \frac{\tan 50^\circ + \tan 20^\circ}{1 - \tan50^\circ\tan20^\circ}$

  $\tan70^\circ - \tan70^\circ\tan50^\circ\tan20^\circ = \tan 50^\circ + \tan 20^\circ$

  $\tan70^\circ = \tan70^\circ\tan50^\circ\tan20^\circ + \tan 50^\circ + \tan 20^\circ$

  $= \tan(90^\circ - 20^\circ)\tan50^\circ\tan20^\circ + \tan 50^\circ + \tan 20^\circ$

  $= \cot20^\circ\tan50^\circ\tan20^\circ + \tan 50^\circ + \tan 20^\circ$

  $= \tan 50^\circ + \tan 50^\circ + \tan 20^\circ = 2\tan50^\circ + \tan20^\circ =$ R.H.S.

\item L.H.S. $= \frac{\cos\left(\frac{\pi}{4} + x\right)\cos\left(\frac{\pi}{4} - x\right)}{\sin\left(\frac{\pi}{4} +
  x\right)\sin\left(\frac{\pi}{4} - x\right)}$

  $= \frac{\cos^2\frac{\pi}{4} - \sin^2x}{\sin^2\frac{\pi}{4} - \sin^2x} = \frac{\frac{1}{2} - \sin^2x}{\frac{1}{2} -
  \sin^2x} = 1 =$ R.H.S.

\item L.H.S. $= \cos(m + n)\theta.\cos(m - n)\theta - \sin(m + n)\theta\sin(m - n)\theta$

  $= \cos(m + n + m - n)\theta = \cos2m\theta =$ R.H.S.

\item L.H.S. $= \frac{\tan(\theta + \phi) + \tan(\theta - \phi)}{1 - \tan(\theta + \phi)\tan(\theta - \phi)}$

  $= \tan(\theta + \phi + \theta - \phi) = \tan 2\theta =$ R.H.S.

\item Given $\cos 9^\circ + \sin 9^\circ = \sqrt{2}\sin 54^\circ$

  $\frac{1}{\sqrt{2}}\cos9^\circ + \frac{1}{\sqrt{2}}\sin9^\circ = \sin54^\circ$

  L.H.S. $= \sin45^\circ\cos9^\circ + \cos45^\circ\sin9^\circ$

  $= \sin(45^\circ + 9^\circ) = \sin 54^\circ =$ R.H.S.

\item L.H.S. $= \frac{\cos 20^\circ - \sin 20^\circ}{\cos 20^\circ + \sin 20^\circ}$

  Dividing both numerator and denominaor with $\cos20^\circ,$ we get

  $= \frac{1 - \tan20^\circ}{1 + \tan20^\circ} = \frac{\tan 45^\circ - \tan20^\circ}{1 -
    \tan45^\circ\tan20^\circ}~[\because \tan45^\circ = 1]$

  $= \tan(45^\circ - 20^\circ) = \tan 25^\circ =$ R.H.S.

\item L.H.S. $= \frac{\tan A + \tan B}{\tan A - \tan B}$

  $= \frac{\frac{\sin A}{\cos A} + \frac{\sin B}{\cos B}}{\frac{\sin A}{\cos A} - \frac{\sin B}{\cos B}}$

  $= \frac{\sin A\cos B + \sin B\cos A}{\sin A\cos B - \sin B\cos A} = \frac{\sin(A + B)}{\sin (A - B)} =$ R.H.S.

\item L.H.S. $= \frac{1}{\tan 3A - \tan A} - \frac{1}{\cot 3A - \cot A}$

  $= \frac{1}{\tan 3A - \tan A} - \frac{1}{\frac{1}{\tan 3A} - \frac{1}{\tan A}}$

  $= \frac{1}{\tan 3A - \tan A} - \frac{\tan A\tan 3A}{\tan A - \tan 3A}$

  $= \frac{1 + \tan A \tan 3A}{\tan 3A - \tan A} = \frac{1}{\tan(3A - A)} = \cot 2A =$ R.H.S.

\item This is similar to previous problema and can be solved likewise.

\item L.H.S. $= \frac{\sin 3\alpha}{\sin\alpha} + \frac{\cos 3\alpha}{cos\alpha}$

  $= \frac{\sin3\alpha\cos\alpha + \cos3\alpha\sin\alpha}{\sin\alpha\cos\alpha}$

  $= \frac{\sin(3\alpha + \alpha)}{\sin\alpha\cos\alpha}$

  $=2\frac{2\sin 4\alpha}{\sin2\alpha}~[\because \sin\alpha\cos\alpha = \frac{1}{2}\sin2\alpha]$

  $= 2\frac{2\sin2\alpha\cos2\alpha}{\sin2\alpha} = 4\cos2\alpha =$ R.H.S.

\item L.H.S. $= \frac{\tan\left(\frac{\pi}{4} + A \right) - \tan\left(\frac{\pi}{4} - A\right)}{\tan\left(\frac{\pi}{4} + A\right) +
  \tan\left(\frac{\pi}{4} - A\right)}$

  $= \frac{\frac{1 + \tan A}{1 - \tan A} - \frac{1 - \tan A}{1 + \tan A}}{\frac{1 + \tan A}{1 - \tan A} + \frac{1 - \tan
    A}{1 + \tan A}}$

  $= \frac{(1 + \tan A)^2 - (1 - \tan A)^2}{(1 + \tan A)^2 + (1 - \tan A)^2}$

  $= \frac{4\tan A}{2 + 2\tan^2A} = \frac{2\tan A}{\sec^2A} = 2\sin A\cos A = \sin 2A$

\item Given, $\tan 40^\circ + 2 \tan 10^\circ = \tan 50^\circ$

  R.H.S. $= \tan 50^\circ = \tan(40^\circ + 10^\circ)$

  $= \frac{\tan 40^\circ + \tan 10^\circ}{1 - \tan 40^\circ\tan 10^\circ}$

  $\tan50^\circ - \tan50^\circ\tan40^\circ\tan10^\circ = \tan 40^\circ + \tan 10^\circ$

  $\tan50^\circ - \cot40^\circ\tan40^\circ\tan10^\circ = tan40^\circ + \tan10^\circ$

  $\tan50^\circ = \tan 40^\circ + 2 \tan 10^\circ$

\item R.H.S. $= \tan(\alpha + \beta)\tan(\alpha - \beta)$

  $= \frac{\sin(\alpha + \beta)}{\cos(\alpha + \beta)}\frac{\sin(\alpha -\beta)}{\cos(\alpha - \beta)}$

  $= \frac{\sin\alpha\cos\beta + \cos\alpha\sin\beta}{\cos\alpha\cos\beta - \sin\alpha\sin\beta}\frac{\sin\alpha\cos\beta -
  \cos\alpha\sin\beta}{\cos\alpha\cos\beta + \sin\alpha\sin\beta}$

  $= \frac{\sin^2\alpha\cos^2\beta - \sin^2\beta\cos^2\alpha}{\cos^2\alpha\cos^2\beta - \sin^2\alpha\sin^2\beta}$

  $= \frac{\sin^2\alpha(1 - \sin^2\beta) - \sin^2\beta(1 - \sin^2\alpha)}{\cos^2\alpha(1 - \sin^2\beta) - \sin^2\beta(1 - \cos^2\alpha)}$

  $= \frac{\sin^2\alpha - \sin^2\beta}{\cos^2\alpha - \sin^2\beta} =$ R.H.S.

\item L.H.S. $= \tan^2\alpha -\tan^2\beta = \frac{\sin^2\alpha}{\cos^2\alpha} - \frac{\sin^2\beta}{\cos^2\beta}$

  $= \frac{\sin^2\alpha\cos^2\beta - \sin^2\beta\cos^2\alpha}{\cos^2\alpha\cos^2\beta}$

  $= \frac{(\sin\alpha\cos\beta + \sin\beta\sin\alpha)(\sin\alpha\cos\beta - \sin\beta\sin\alpha)}{\cos^2\alpha\cos^2\beta}$

  $= \frac{\sin(\alpha + \beta)\sin(\alpha - \beta)}{\cos^2\alpha\cos^2\beta} =$ R.H.S.

\item L.H.S. $= \tan[(2n + 1)\pi + \theta] + \tan[(2n + 1)\pi - \theta]$

  $= \tan(\pi + \theta) + \tan(\pi - \theta)~[\because \tan 2n\pi = 0]$

  $= \tan\theta - \tan\theta = 0 =$ R.H.S.

\item L.H.S. $= \tan\left(\frac{\pi}{4} + \theta\right)\tan\left(\frac{3\pi}{4} + \theta\right) + 1$

  $= \tan\left(\frac{\pi}{4} + \theta\right)\tan\left[\pi - \left(\frac{\pi}{4} - \theta\right)\right] + 1$

  $= -\tan\left(\frac{\pi}{4} + \theta\right)\tan\left(\frac{\pi}{4} - \theta\right) + 1$

  $= -\tan\left(\frac{\pi}{4} + \theta\right)\tan\left[\left(\frac{\pi}{2} - \frac{pi}{4} - \theta\right)\right] + 1$

  $= -\tan\left(\frac{\pi}{4} + \theta\right)\cot\left(\frac{\pi}{4} + \theta\right) + 1 = -1 + 1 = 0 =$ R.H.S.

\item R.H.S. $= \frac{1 - pq}{\sqrt{(1 + p^2)(1 + q^2)}}$

  Substituting for $p$ and $q,$ we get

  $= \frac{1 - \tan\alpha\tan\beta}{\sqrt{(1 + \tan^2\alpha)(1 + \tan^2\beta)}}$

  $= \frac{\frac{\cos\alpha\cos\beta - \sin\alpha\sin\beta}{\cos\alpha\cos\beta}}{\sqrt{\sec\alpha\sec\beta}}$

  $= \cos(\alpha + \beta) =$ R.H.S.

\item Given, $\tan \beta = \frac{2\sin\alpha\sin\gamma}{\sin(\alpha + \gamma)}$

  Inverting, we get

  $2\cot\beta = \frac{\sin(\alpha + \gamma)}{\sin\alpha\sin\gamma} = \frac{\sin\alpha\cos\gamma +
    \sin\gamma\cos\alpha}{\sin\alpha\sin\gamma}$

  $= \cot \alpha + \cot \gamma$

  Thus, $\cot\alpha, \cot\beta, \cot\gamma$ are in A.P.

\item $\tan(\theta + \alpha - (\theta - \alpha)) = \tan2\alpha = \frac{\tan(\theta + \alpha) - \tan(\theta - \alpha)}{1 +
  \tan(\theta + \alpha)\tan(\theta - \alpha)}$

  $= \frac{b - a}{1 + ab}$

\item $\tan\gamma = \tan(\alpha + \beta) = \frac{\tan\alpha + \tan\beta}{1 - \tan\alpha\tan\beta}$

  $a - b = \cot\alpha + \cot\beta - \tan\alpha - \tan\beta$

  $= \frac{\sin(\alpha + \beta)}{\sin\alpha\sin\beta} - \frac{\sin(\alpha + \beta)}{\cos\alpha\beta}$

  $= \sin(\alpha + \beta)\left(\frac{\cos\alpha\cos\beta -
  \sin\alpha\sin\beta}{\sin\alpha\sin\beta\cos\alpha\cos\beta}\right)$

  $= \frac{\sin(\alpha + \beta)\cos(\alpha + \beta)}{\sin\alpha\sin\beta\cos\alpha\cos\beta}$

  $ab = (\tan\alpha + \tan\beta)(\cot\alpha + \cot\beta)$

  $= \frac{\sin^2(\alpha + \beta)}{\sin\alpha\sin\beta\cos\alpha\cos\beta}$

  $\frac{ab}{a - b} = \tan(\alpha + \beta) = \tan\gamma$

\item Given, $A + B = 45^\circ \therefore \tan(A + B) = 1$

  $\frac{\tan A + \tan B}{1 - \tan A\tan B} = 1$

  $1 + \tan A + \tan B + \tan A \tan B = 2$

  $(1 + \tan A)(1 + \tan B) = 2$

\item Given, $\sin\alpha\sin\beta - \cos\alpha\cos\beta + 1 = 0$

  $\Rightarrow \cos\alpha\cos\beta - \sin\alpha\sin\beta = 1$

  $\Rightarrow \cos(\alpha + \beta) = 1$

  $\Rightarrow  \sin(\alpha + \beta) = 0$

  $1 + \cot\alpha\tan\beta = 1 + \frac{\cos\alpha\sin\beta}{\sin\alpha\cos\beta}$

  $= \frac{\sin\alpha\cos\beta + \cos\alpha\sin\beta}{\sin\alpha\cos\beta}$

  $= \frac{\sin(\alpha + \beta)}{\sin\alpha\cos\beta} = \frac{0}{\sin\alpha\cos\beta} = 0$

\item $\tan\beta = \frac{n\sin\alpha\cos\alpha}{1 - n\sin^2\alpha} =
  \frac{\frac{n\sin\alpha\cos\alpha}{\cos^2\alpha}}{\frac{1}{\cos^2\alpha} - n\frac{\sin^2\alpha}{\cos^2\alpha}}$

  $= \frac{n\tan\alpha}{\sec^2\alpha - n\tan^2\alpha} = \frac{n\tan\alpha}{1 + (1 - n)\tan^2\alpha}$

  Now, $\tan(\alpha - \beta) = \frac{\tan\alpha - \frac{n\tan\alpha}{1 + (1 - n)\tan^2\alpha}}{1 -
    \tan\alpha\frac{n\tan\alpha}{1 + (1 - n)\tan^2\alpha}}$

  $= \frac{\tan\alpha + (1 - n)\tan^3\alpha - n\tan\alpha}{1 + (1 - n)\tan^2\alpha + n\tan^2\alpha}$

  $= \frac{(1 - n)\tan\alpha + (1 - n)tan^3\alpha}{1 + \tan^2\alpha}$

  $= \frac{(1 - n)\tan\alpha(1 + \tan^2\alpha)}{ 1 + \tan^2\alpha} = (1 - n)\tan\alpha$

\item Given, $\cos(\beta - \gamma) + \cos(\gamma - \alpha) + \cos(\alpha - \beta) = -\frac{3}{2}$

  $3 + 2\cos(\beta - \gamma) + 2\cos(\gamma - \alpha) + 2\cos(\alpha - \beta) = 0$

  $3 + 2(\cos\beta\cos\gamma + \sin\beta\sin\gamma) + 2(\cos\gamma\cos\alpha + \sin\gamma\sin\alpha) +
  2(\cos\alpha\cos\beta + \sin\alpha\sin\beta) = 0$

  $(\cos^2\alpha + \sin^2\alpha) + (\cos^2\beta + \sin^2\beta) + (\cos^2\gamma + \sin^2\gamma) + 2(\cos\beta\cos\gamma +
  \sin\beta\sin\gamma) + 2(\cos\gamma\cos\alpha + \sin\gamma\sin\alpha) + 2(\cos\alpha\cos\beta + \sin\alpha\sin\beta) = 0$

  $(\cos\alpha + \cos\beta + \cos\gamma)^2 + (\sin\alpha + \sin\beta + \sin\gamma^2) = 0$

  $\cos\alpha + \cos\beta + \cos\gamma = \sin\alpha + \sin\beta + \sin\gamma = 0$

\item $\tan(\alpha + \beta) = \frac{\tan\alpha + \tan\beta}{1 - \tan\alpha\tan\beta}$

  $= \frac{\frac{m}{m + 1} + \frac{1}{2m + 1}}{1 - \frac{m}{m + 1}\frac{1}{2m + 1}}$

  $= \frac{2m^2 + m + m + 1}{2m^2 + 3m + 1 - n} = 1$

  Thus, $\alpha + beta = \frac{\pi}{4}$

\item Given $(\cot A - 1)(\cot B - 1) = 2$

  $\cot A\cot B - 1 - \cot A - \cot B = 0$

  $\cot A\cot B - 1 = \cot A + \cot B \Rightarrow \frac{\cot A\cot B - 1}{\cot A + \cot B} = 1$

  $\cot(A + B) = \cot 45^\circ$

  Thus, $A + B = 45^\circ$

  which we have proved in reverse.

\item Given, $\tan\alpha - \tan\beta = x$ and $\cot\beta - \cot\alpha = y,$ we have to prove that $\cot(\alpha -
  \beta) = \frac{x + y}{xy}$

  Let $\cot(\alpha - \beta) = \frac{x + y}{xy} = \frac{\tan\alpha - \tan\beta + \cot\beta - \cot\alpha}{(\tan\alpha -
    \tan\beta)(\cot\beta - \cot\alpha)}$

  $\tan(\alpha - \beta) = \frac{(\tan\alpha - \tan\beta)(\cot\beta - \cot\alpha)}{\tan\alpha - \tan\beta + \cot\beta -
    \cot\alpha}$

  $= \frac{\tan\alpha - \tan\beta}{1 + \frac{\tan\alpha - \tan\beta}{\cot\beta - \cot \alpha}}$

  $= \frac{\tan\alpha - \tan\beta}{ 1 + \frac{\frac{\sin(\alpha - \beta)}{\cos\alpha\cos\beta}}{\frac{\sin(\alpha -
        \beta)}{\cos\alpha\cos\beta}}}$

  $= \frac{\tan\alpha - \tan\beta}{1 + \tan\alpha\tan\beta} = \tan(\alpha - \beta)$

  Hence proved.

\item Given $\alpha + \beta + \gamma = 90^\circ = \frac{\pi}{2}$

  $\cot \alpha = \cot\left(\frac{\pi}{2} - (\beta + \gamma)\right) = \tan(\beta + \gamma)$

  $= \frac{\tan\beta + \tan\gamma}{1 - \tan\beta\tan\gamma}$

\item We have to prove that $\cot \beta = 2\tan(\alpha - \beta)$

  $\frac{1}{\tan\beta} = 2\frac{\tan\alpha - \tan\beta}{1 + \tan\alpha\tan\beta}$

  $1 + \tan\alpha\tan\beta = 2\tan\alpha\tan\beta - 2\tan^2\beta$

  Dividing both sides by $\tan\beta,$ we get

  $\cot \beta + \tan\alpha = 2\tan\alpha - 2\tan\beta$

  $\cot\beta + 2\tan\beta = =\tan\alpha$

  Hence proved.

\item $\sin A = \frac{a}{c}, \sin B = \frac{b}{c}, \cos A = \frac{b}{c}, \cos B = \frac{a}{c}$

  $\csc(A - B) = \frac{1}{\sin(A - B)} = \frac{1}{\sin A\cos B - \cos A\sin B} = \frac{1}{\frac{a^2}{c^2} -
  \frac{b^2}{c^2}}$

  $= \frac{c^2}{a^2 - b^2} = \frac{a^2 + b^2}{a^2 - b^2}$

  $\sec(A - B) = \frac{1}{\cos(A - B)} = \frac{1}{\cos A\cos B + \sin A\sin B} = \frac{c^2}{2ab}$

\item We have to prove that $A + B = C$ i.e. $\tan(A + B) = \tan C$

  $\frac{\tan A + \tan B}{1 - \tan A\tan B} = \tan C$

  $\frac{\frac{1}{\sqrt{ac}} + \sqrt{\frac{a}{c}}}{1 - \frac{1}{\sqrt{ac}}\sqrt{\frac{a}{c}}} = \sqrt{\frac{c}{a^3}}$

  $= \frac{\frac{1}{\sqrt{ac}} + \sqrt{\frac{a}{c}}}{1 - \frac{1}{c}}$

  $\frac{1 + a}{\sqrt{ac}}.\frac{c}{c - 1} = \sqrt{\frac{c}{a^3}}$

  $\frac{ac + c}{c - 1} = \frac{c}{a}$

  $a^2c + ac = c^2 - c$

  $a^2 + a + 1 = c$ which is given, hence proved.

\item Given $\frac{\tan(A - B)}{\tan A} + \frac{\sin^2C}{\sin^2A} = 1$

  $\frac{\sin^2C}{\sin^2A} = 1 - \frac{\tan(A - B)}{\tan A}$

  $= 1 - \frac{\sin(A - B)\cos A}{\sin A\cos(A - B)} = \frac{\sin (A - A + B)}{\sin A\cos(A - B)}$

  $\sin^2C = \frac{\sin A\sin B}{\cos(A - B)}$

  $\csc^2C = \frac{\cos(A - B)}{\sin A\sin B} = 1 + \cot A\cot B = \cot^2C$

  $\Rightarrow \tan A\tan B= \tan^2 C$

\item Given, $\sin\alpha\sin\beta - \cos\alpha\cos\beta = 1$

  $\cos(\alpha + \beta) = -1 \Rightarrow \alpha + \beta = (2n + 1)\pi$

  $\tan(\alpha + \beta) = 0 \Rightarrow \tan \alpha + \tan \beta = 0$

\item Given, $\sin\theta = 3\sin(\theta + 2\alpha)$

  $\sin(\theta + \alpha - \alpha) = 3\sin(\theta + \alpha + \alpha)$

  $\sin(\theta + \alpha)\cos\alpha - \sin\alpha\cos(\theta + \alpha) = 3\sin(\theta + \alpha)\cos\alpha + 3\cos(\theta +
  \alpha)\sin\alpha$

  $2\sin(\theta + \alpha)\cos\alpha + 4\sin\alpha\cos(\theta + \alpha) = 0$

  Dividingboth sides with $2\cos(\theta + \alpha)\cos\alpha,$ we get

  $\tan(\theta + \alpha) + 2\tan\alpha = 0$

\item Given, $3\tan\theta\tan\phi = 1 \Rightarrow \cot\theta\cot\phi = 3$

  $\frac{\cos\theta\cos\phi}{\sin\theta\sin\phi} = 3$

  Applying componendo and dividendo

  $\frac{\cos\theta\cos\phi + \sin\theta\sin\phi}{\cos\theta\cos\phi - \sin\theta\sin\phi} = \frac{3 + 1}{3 - 1}$

  $\cos(\theta - \phi) = 2\cos(\theta + \phi)$

\item Let $z = \cos\theta + \sin\theta = \sqrt{2}\left(\frac{1}{\sqrt{2}}\cos\theta + \frac{1}{\sqrt{2}}\sin\theta\right)$

  $= \sqrt{2}\cos\left(\theta - \frac{\pi}{4}\right)$

  $= \sqrt{2}\cos 55^\circ$ which has positive sign.

\item Let $z = 5\cos\theta + 3\cos\left(\theta + \frac{\pi}{3}\right) + 3$

  $z = 5\cos\theta + \frac{3}{2}\cos\theta - \frac{3\sqrt{3}}{2}\sin\theta + 3$

  $= \frac{13}{2}\cos\theta - \frac{3\sqrt{3}}{2}\sin\theta + 3$

  $= 7\left(\frac{13}{14}\cos\theta - \frac{3\sqrt{3}}{14}\sin\theta\right) + 3$

  Let $\cos\alpha = \frac{13}{14}$ then $\sin\alpha = \frac{3\sqrt{3}}{14}$

  $y = 7(\cos\alpha\cos\theta - \sin\alpha\sin\theta) + 3$

  $y = 7\cos(\theta + \alpha) + 3$

  Now maximum and minimum values of $\cos(\theta + \alpha)$ are $1$ and $-1.$ Thus, value of $y$ will lie
  between $4$ and $10.$

\item Given, $m\tan(\theta - 30^\circ) = n\tan(\theta + 120^\circ)$

  $\frac{\tan(\theta - 30^\circ)}{\tan(\theta + 120^\circ)} = \frac{n}{m}$

  $\frac{\sin(\theta - 30^\circ)\cos(\theta) + 120^\circ}{\cos(\theta - 30^\circ)\sin(\theta) + 120^\circ} = \frac{n}{m}$

  Applying componendo and dividendo

  $\frac{\sin[(\theta + 120^\circ) + (\theta - 30^\circ)]}{\sin[(\theta + 120^\circ) - (\theta - 30^\circ)} = \frac{m +
      n}{m - n}$

  $\frac{\sin(2\theta + 90^\circ)}{\sin150^\circ} = \frac{m + n}{m - n}$

  $\cos2\theta = \frac{m + n}{2(m - n)}$

\item Given, $\frac{\tan\alpha}{\tan\beta} = \frac{x}{y}$

  Applying componendo and dividendo

  $\frac{\tan\alpha + \tan\beta}{\tan\alpha - \tan\beta} = \frac{x + y}{x - y}$

  $\frac{\sin(\alpha + \beta)}{\sin(\alpha - \beta)} = \frac{x + y}{x - y}$

  $\sin(\alpha - \beta) = \frac{x - y}{x + y}\sin\theta$

\item We have to find the maximum and minimul values of $7\cos\theta + 24\sin\theta = y$ (let)

  $= 25\left(\frac{7}{25}\cos\theta + \frac{24}{25}\sin\theta\right)$

  If $\cos\alpha = \frac{7}{25}$ then $\sin\alpha = \frac{24}{75}$

  $y = 25\cos(\theta - \alpha)$

  Thus, maximum and minimum values of $y$ are $25$ and $-25.$

\item Given expression is $\sin100^\circ - \sin10^\circ = \cos10^\circ - \sin10^\circ = y$ (let)

  $y = \sqrt{2}\left(\frac{1}{\sqrt{2}}\cos^10 - \frac{1}{\sqrt{2}\sin10^\circ}\right)$

  $= \sqrt{2}\cos(45^\circ + 10^\circ)$

  Thus, the sign is positive.
\stopitemize
