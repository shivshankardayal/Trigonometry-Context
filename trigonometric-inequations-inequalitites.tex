% -*- mode: context; -*-
\chapter{Trigonometric Inequations and Inequalities}
Verily, the manifold domains of trigonometric‑inequality endeavours are sundered into sundry and discrete
sortal estates, each bearing its peculiar character and challenge; for by raising a sine or cosine in mortal
coil, or by laying forth a bound upon triangle’s angles, one enters one such estate --- yet another abides
when the inequality stands in symmetric form, or in the guise of a function; thus the whole constellation of
tasks reveals itself as a varied host of kinds, each demanding its own art of proof and reasoning.
In certain passages of mathematical endeavour, one is tasked with the proving of a strict numerical
inequality, a bound which must be upheld by a specific value of a trigonometric function --- or by an
expression wrought from such a value. In such instances the effort lies in demonstrating that the designated
trigonometric datum --- sine, cosine, tangent, or a composite thereof --- indeed conforms to the stipulated
quantitative constraint, under the exacting gaze of reason and the subtle arts of algebra and trigonometric
manipulation. In other kinds of inquiry one must show that an inequality remains valid for every possible
assignment of the arguments of a given trigonometric expression, or at least for all argument‑values
permitted by some auxiliary constraint imposed by the hypothesis. In these circumstances the burden is to
prove universally that no admissible angle or argument escapes the stipulated inequality, that for each
allowable value the trigonometric expression satisfies the required bound under the full rigour of logical
and analytic scrutiny. In either case the resolution of the problem ultimately comes down to an
investigation of the values taken by a trigonometric function over some interval of its domain --- or, in
more general cases, over the entire domain of definition of that function. The crux of the argument lies in
examining how the sine, cosine, tangent (or other) function behaves over that domain (or interval),
determining its extremal, zero or undefined points, and thereby ascertaining whether the inequality holds
throughout, or only on certain subintervals. In simpler incarnations of these investigations one may succeed
in transmuting a trigonometric expression so that it falls into the realm of the elementary and well‑known
boundings — namely those by which one may directly invoke the universal truths $|\sin x| \leq 1$ or $|\cos x|
\leq 1$. In so doing the originally intricate mixture of angles and functions yields to a more tractable form,
allowing one to assert with clarity and certainty that the transformed expression abides by the classical
extremal limits of the sine or cosine function. In other instances the trigonometric expression under
scrutiny may be transformed so that, at the end of the manipulation, it assumes the guise of a new function
$F(z)$, wherein the argument $z$ itself is expressed by some simpler trigonometric function (or by a
trigonometric expression of reduced complexity). Thus the original, perhaps intricate, combination of sines
and cosines gives way to a formulation in which the core object becomes $F(z)$, and the variable $z$ offers
a simpler or more tractable pathway for analysis and proof. In such a scenario the resolution of the problem
becomes equivalent to the investigation of a single function $F(z)$, keeping in mind that the new argument
$z$ may itself be subject to additional conditions dictated by the inherent properties of trigonometric
functions. Thus, the burden of proof shifts from handling a complicated trigonometric expression directly to
analysing the behaviour of the function $F(z)$ over the domain permitted for $z$, ensuring that all
constraints --- both arising from the substitution and from the original hypothesis --- are duly respected.
In such circumstances the problem may be reduced to the analysis of the single function $F(z)$, with the
understanding that the auxiliary variable $z$ is itself bound by certain conditions deriving from the
intrinsic nature of trigonometric functions. Under these additional constraints one may embark upon the
study of $F(z)$ either via differential calculus or by elementary means — for example by invoking the
classical Arithmetic Mean–Geometric Mean inequality (AM–GM) for two non‑negative quantities $a$ and
$b$, which asserts $\frac{a + b}{2}\geq \sqrt{ab}$, and hence under suitable algebraic arrangement yields
the desired bound on $F(z)$ in which the equality sign occurs only for $a = b$. In a distinguished class of
problems the proof of an inequality is reduced to a comparison among the measure of an acute angle, the
value of its sine, and the value of its tangent. In such cases the task is to show that, for a given acute
angle $\alpha$, the relative magnitudes of $\alpha, \sin\alpha$ and $\tan\alpha$ bear the required
ordering (or satisfy the required bounds), thereby establishing the inequality by comparing angle, sine, and
tangent in concert. This method often hinges upon the fundamental relations --- for an acute $\alpha$ ---
that $0 < \sin\alpha < \alpha < \tan\alpha$, and upon exploiting monotonic or functional properties of sine
and tangent to draw the necessary conclusions.

At the outset, our procedure commences with proof of the inequalities $\sin x < x < \tan x$ and $\cos x <
\frac{\sin x}{x} < 1$.

Consider the following diagram:



\section{Problems}
\startitemize[n, 1*broad]
\item Find the solution set of inequation $\sin x > \frac{1}{2}$.
\item Find the solution set of inequation $\cos x \geq -\frac{1}{2}$.
\item Find the solution set of inequation $\tan x > -\sqrt{3}$.
\item Find the set of all $x$ in the interval $[0, \pi]$ for which $2\sin^2x - 3\sin x + 1\geq 0$.
\item Let $A = \left\{\theta: 2\cos^2\theta + \sin\theta \leq 2\right\}$

  $B = \left\{\theta: \frac{\pi}{2}\leq \theta\leq \frac{3\pi}{2}\right\}$, find $A\cap B$.
\item In an acute angled $\triangle ABC$, show that $\tan A\tan B\tan C\geq 3\sqrt{3}$.
\item In a $\triangle ABC$, if $\cos A + \cos B + \cos C = \frac{3}{2}$, show that the triangle is
  equilateral.
\item If $0 < \alpha < \beta < \frac{\pi}{2}$, show that $\alpha - \sin\alpha < \beta - \sin\beta$.
\item If $0 < x < \frac{\pi}{2}$, show that $\cos x > 1 - \frac{x^2}{2}$.
\item Show that $0\leq 3\sin\sqrt{\frac{\pi^2}{16} - x^2}\leq \frac{3}{\sqrt{2}}$.
\item If $\alpha > 0, \beta > 0, \gamma > 0$ and $\alpha + \beta + \gamma < \frac{\pi}{2}$, show that
  $\tan\alpha\tan\beta + \tan\beta\tan\gamma + \tan\gamma\tan\alpha < 1$.
\item If $0 < \alpha_1 < \alpha_2 < \cdots < \alpha_n < \frac{\pi}{2}$, show that $\tan\alpha_1
  < \frac{\sin\alpha_1 + \sin\alpha_2 + \cdots + \sin\alpha_n}{\cos\alpha_1 + \cos\alpha_2 + \cdots
    + \cos\alpha_n} < \tan\alpha_n$.
\item If angles $A, B, C$ of a $\triangle ABC$ are acute, show that $\cos^2A + \cos^2B + \cos^2C < 1$.
\item If one of the angles $A, B, C$ of a tringle is not acute, show that $\cos^2A + \cos^2B + \cos^2C \geq
  1$.
\item If $\alpha > 0, \beta > 0, \gamma > 0$ and $\alpha + \beta + \gamma = \frac{\pi}{2}$, show that
  $\sin^2\alpha + \sin^2\beta + \sin^2\gamma \geq \frac{3}{4}$.
\item Show that $\cos(\alpha + \beta)\cos(\alpha - \beta)\leq \cos^2\alpha$.
\item Show that $0\leq \cos^2\alpha + \cos^2(\alpha + \beta) - 2\cos\alpha\cos\beta\cos(\alpha + \beta) \leq
  1$.
\item Show that $\sin^6\alpha + \cos^6\alpha \geq \frac{1}{4}$.
\item Prove that in a $\triangle ABC, \cos A + \cos B + \cos C\leq \frac{3}{2}$.
\item Prove that in a $\triangle ABC, \sin\frac{A}{2}\sin\frac{B}{2}\sin\frac{C}{2}\leq \frac{1}{8}$.
\item Prove that $\cos\alpha + 3\cos3\alpha + 6\cos6\alpha\geq -\frac{115}{16}$.
\item Prove that $\cos36^\circ > \tan36^\circ$.
\item Prove that $\sin\alpha\sin2\alpha\sin3\alpha < \frac{3}{4}$.
\item If $\alpha, \beta, \gamma$ are angles of a triangle, show that $(\sin\alpha + \sin\beta
  + \sin\gamma)^2\geq 9\sin\alpha\sin\beta\sin\gamma$.
\item Show that $\sin^4\alpha + \cos^4\alpha\geq \frac{1}{2}$.
\item If $0 < \alpha < \beta < \frac{\pi}{2}$, show that $\frac{\sin\alpha}{\alpha}
  > \frac{\sin\beta}{\beta}$.
\item If $0 < \alpha < \frac{\pi}{2}$, show that $\frac{\tan\alpha}{\alpha} > \frac{\alpha}{\sin\alpha}$.
\item If $0 < \alpha < \frac{\pi}{2}$, show that $\tan\frac{\alpha}{2} < \alpha$.
\item If $0 < \alpha < \frac{\pi}{2}$, show that $\alpha - \frac{\alpha^3}{3} < \sin\alpha$.
\item If $-\frac{\pi}{2} < \theta < \frac{\pi}{2}$, show that $\sqrt{\cos\theta} <
  \sqrt{2\cos\frac{\theta}{2}}$.
\item If $0 < x < \frac{\pi}{4}$, show that $\frac{\cos x}{\sin^2x(\cos x - \sin x)} > 8$.
\item Prove the inequality $\sin^8x + \cos^8x \geq \frac{1}{8}$.
\item Prove the inequality $(x + y)(x + y + 2\cos x) + 2\geq 2\sin^2x$.
\item Prove the inequality $-4\leq \cos2x + 3\sin x\leq \frac{17}{8}$.
\item Prove the inequality $0 < \sin^8x + \cos^{14}x \leq 1$.
\stopitemize
