% -*- mode: context; -*-
\chapter{Inverse Circular Functions}
\startitemize[n]
\item Let $\tan^{-1}(-1) = \theta$ then $\tan\theta = -1.$ Also, $-\frac{\pi}{2}< \theta < \frac{\pi}{2}$

  The only value in this range which satisfied this equation is $-\frac{\pi}{4}.$

\item Let $\cot^{-1}(-1) = \theta,$ then $\cot\theta = -1$ and $0<\theta<\pi$

  $\Rightarrow \theta = \frac{3\pi}{4}$

\item Let $\sin^{-1}\left(-\frac{\sqrt{3}}{2}\right) = \theta$ then $\sin\theta = -\frac{\sqrt{3}}{2}$

  Also, $-\frac{\pi}{2}\leq \theta\leq \frac{\pi}{2} \Rightarrow \theta = -\frac{\pi}{3}$

\item $\sin\left[\frac{\pi}{3} -\sin^{-1}\left(-\frac{1}{2}\right)\right]$

  $=\sin\left[\frac{\pi}{3} -\{-\sin^{-1}\frac{1}{2}\}\right]$

  $= \sin\left[\frac{\pi}{3} + \frac{\pi}{6}\right] = 1$

\item $\sin\left[\cos^{-1}\left(-\frac{1}{2}\right)\right] = \sin\left[\pi - \cos^{-1}\frac{1}{2}\right]$

  $= \sin\frac{2\pi}{3} = \frac{\sqrt{3}}{2}$

\item $\sin\left[\tan^{-1}(-\sqrt{3}) + \cos^{-1}\frac{-\sqrt{3}}{2}\right]$

  $= \sin\left[-\frac{\pi}{3} + \pi- \frac{\pi}{6}\right]$

  $= \sin\frac{\pi}{2} = 1$

\item Given expression is $\tan\left[\frac{1}{2}\cos^{-1}\frac{\sqrt{5}}{3}\right]$

  Let $\cos^{-1}\frac{\sqrt{5}}{3} = 2\theta$ then $\cos2\theta = \frac{\sqrt{5}}{3}$

  $\Rightarrow \frac{1 - \tan^2\theta}{1 + \tan^2\theta} = \frac{\sqrt{5}}{3}$

  By componendo and dividendo $\frac{2\tan^2\theta}{2} = \frac{3 - \sqrt{5}}{3 + \sqrt{5}}$

  $\Rightarrow \tan\theta = \pm\left(\frac{3 - \sqrt{5}}{2}\right)$

  Given $0 \leq 2\theta \leq \pi \Rightarrow 0\leq \theta \leq \frac{\pi}{2}$

  i.e. $\theta$ lies in first quadrant. $\Rightarrow \tan\theta = \frac{3 - \sqrt{5}}{2}$

\item Given expression is $\sin^{-1}\left(\sin\frac{2\pi}{3}\right)$

  Let $\sin^{-1}\left(\sin\frac{2\pi}{3}\right) = \theta$

  then $\sin\theta = \sin\frac{2\pi}{3}$ and $-\frac{\pi}{2}\leq\theta\leq \frac{\pi}{2}$

  $=\sin\left(\pi - \frac{\pi}{3}\right) = \sin\frac{\pi}{3} \Rightarrow \theta = \frac{\pi}{3}$

  $\sin^{-1}\left(\sin\frac{2\pi}{3}\right) = \frac{\pi}{3}$

\item Let $\sin^{-1}\frac{\sqrt{3}}{2} = \theta \Rightarrow \sin\theta = \frac{\sqrt{3}}{2}$

  $\Rightarrow \theta = \frac{\pi}{3}$ $-\frac{\pi}{2}\leq\theta\leq \frac{\pi}{2}$

\item Let $\tan^{-1}\frac{-1}{\sqrt{3}} =\theta$

  then $\tan\theta = \frac{-1}{\sqrt{3}} =\tan\left(\frac{-\pi}{3}\right)$

  $\Rightarrow \theta = -\frac{\pi}{3}$

\item Let $\cot^{-1}(-\sqrt{3}) = \theta \Rightarrow \cot\theta = -\sqrt{3}$

  $\cot\theta = \cot\left(\pi - \frac{\pi}{6}\right)$

  $\theta = \frac{5\pi}{6}$

\item Let $\cot^{-1}\cot\frac{5\pi}{4} = \theta \Rightarrow \cot\theta = \cot\left(\pi + \frac{\pi}{4}\right)$

  $\cot\theta = \cot\frac{\pi}{4} \Rightarrow \theta = \frac{\pi}{4}$

\item Let $\tan^{-1}\left(\tan\frac{3\pi}{4}\right) = \theta$

  $\Rightarrow \tan\theta = \tan\frac{3\pi}{4} = \tan\left(\pi - \frac{\pi}{4}\right)$

  $\Rightarrow \theta = -\frac{\pi}{4}$

\item $\sin^{-1}\frac{1}{2} + \cos^{-1}\frac{1}{2} = \frac{\pi}{6} + \frac{\pi}{3} = \frac{\pi}{2}$

\item Let $\tan^{-1} = \theta \Rightarrow \tan\theta = \frac{3}{4} \Rightarrow \cos\theta = \frac{4}{5}$

\item $\cos\left[\cos^{-1}\left(\frac{\sqrt{3}}{2}\right) + \frac{\pi}{6}\right]$

  $=\cos\left[\frac{\pi}{6} + \frac{\pi}{6}\right] = \cos \frac{\pi}{3} = \frac{1}{2}$

\item L.H.S. $= 2\tan^{-1}\frac{1}{3} + \tan^{-1}\frac{1}{7}$

  $= \tan^{-1}\frac{2*\frac{1}{3}}{1 - \frac{1}{9}} + \tan^{-1}\frac{1}{7}$

  $= \tan^{-1}\frac{3}{4} + \tan^{-1}\frac{1}{7}$

  $= \tan^{-1}\frac{\frac{3}{4} + \frac{1}{7}}{1 - \frac{3}{28}} = \tan^{-1}1 = \frac{\pi}{4} =$ R.H.S.

\item L.H.S. $= \tan^{-1}\frac{1}{3} + \tan^{-1}\frac{1}{7} + \tan^{-1}\frac{1}{5} + \tan^{-1}\frac{1}{8}$

  $= \tan^{-1}\frac{\frac{1}{3} + \frac{1}{7}}{1 - \frac{1}{21}} + \tan^{-1}\frac{\frac{1}{5} + \frac{1}{8}}{1 -
  \frac{1}{40}}$

  $= \tan^{-1}\frac{1}{2} + \tan^{-1}\frac{1}{3} = \tan^{-1}1 = \frac{\pi}{4} =$ R.H.S.

\item L.H.S. $= \sin^{-1}\frac{4}{5} + \sin^{-1}\frac{5}{13} + \sin^{-1}\frac{16}{65}$

  $= \sin^{-1}\left(\frac{4}{5}\sqrt{1 - \frac{25}{169}} + \frac{5}{13}\sqrt{1 - \frac{16}{25}}\right) +
  \sin^{-1}\frac{16}{65}$

  $= \sin^{-1}\left(\frac{48}{65} + \frac{15}{65}\right) + \sin^{-1}\frac{16}{65} = \sin^{-1}\frac{64}{65} +
  \sin^{-1}\frac{16}{65}$

  $= \sin^{-1}\left(\frac{63}{65}.\sqrt{1 - \frac{16^2}{65^2}} + \frac{16}{65}\sqrt{1 - \frac{63^2}{65^2}}\right)$

  $\sin^{-1}\frac{63^2 + 16^2}{65^2} = \sin^{-1} = \frac{\pi}{2} =$ R.H.S.

\item We have to prove that $4\tan^{-1}\frac{1}{5} - \tan^{-1}\frac{1}{70} + \tan^{-1}\frac{1}{99} = \frac{\pi}{4}$

  $\Rightarrow 4\tan^{-1}\frac{1}{5} = \frac{\pi}{4} + \tan^{-1}\frac{1}{70} - tan^{-1}\frac{1}{99}$

  L.H.S. $= 4\tan^{-1}\frac{1}{5} = 2\tan^{-1}\frac{2.\frac{1}{5}}{1 - \frac{1}{25}} = 2\tan^{-1}\frac{5}{12}$

  $= \tan^{-1}\frac{2.\frac{5}{12}}{1 - \frac{25}{144}} = \tan^{-1}\frac{120}{119}$

  $\tan^{-1}\frac{1}{70} - \tan^{-1}\frac{1}{99} = \tan^{-1}\frac{\frac{1}{70} - \frac{1}{99}}{1 +
    \frac{1}{70}.\frac{1}{99}}$

  $= \tan^{-1}\frac{1}{239}$

  R.H.S. $= \tan^{-1}1 + \tan^{-1}\frac{1}{239} = \tan^{-1}\frac{120}{119}$

  Thus, L.H.S. = R.H.S.

\item We have to prove that $\cot^{-1}9 + \csc^{-1}\frac{\sqrt{41}}{4} = \frac{\pi}{4}$

  $\cot^{-1}9 = \tan^{-1}\frac{1}{9}$

  Let $\csc^{-1}\frac{\sqrt{41}}{4} = \theta \Rightarrow \csc\theta = \frac{\sqrt{41}}{4}$

  Since we have to consider principal values only $-\frac{\pi}{2}\leq \theta\leq \frac{\pi}{2}$ and $\theta \neq 0$

  As $\csc\theta$ is +ve here, $\theta$ lies between $0$ and $\pi/2,$ hence $\tan\theta$ must be
  positive.

  $\Rightarrow \tan\theta = \frac{4}{5}$

  $\tan^{-1}\frac{1}{9} + \tan^{-1}\frac{4}{5} = \tan^{-1}\frac{\frac{1}{9} + \frac{4}{5}}{1 - \frac{1}{9}\frac{4}{5}}$

  $=\tan^{-1}\frac{41}{45}.\frac{45}{41} = \tan^{-1}1 = \frac{\pi}{4} =$ R.H.S.

\item We have to prove that $4(\cot^{-1}3 + \csc^{-1}\sqrt{5}) = \pi$

  $\cot^{-1}3 = \tan^{-1}\frac{1}{3}$

  $\csc^{-1}\sqrt{5} = \tan^{-1}\frac{1}{2}$

  L.H.S. $=4(\tan^{-1}\frac{1}{3} + \tan^{-1}\frac{1}{2}) = 4\left(\tan^{-1}\frac{\frac{1}{2} + \frac{1}{3}}{1 -
    \frac{1}{2}.\frac{1}{3}}\right)$

  $= 4\tan^{-1}1 = \pi =$ R.H.S.

\item We have to prove that $\tan^{-1}x = 2\tan^{-1}[\csc\tan^{-1}x - \tan\cot^{-1}x]$

  R.H.S. $= 2\tan^{-1}[\csc\csc^{-1}\frac{\sqrt{1 + x^2}}{x} - \tan\tan^{-1}\frac{1}{x}]$

  $= 2\tan^{-1}\left(\frac{\sqrt{1 + x^2} - 1}{x}\right)$

  Let $x = \tan\theta,$ then R.H.S. $= 2\tan^{-1}\left(\frac{sec\theta - 1}{\tan\theta}\right)$

  $=2\tan^{-1}\left(\frac{1 - \cos\theta}{\sin\theta}\right) =
  2\tan^{-1}\left(\frac{2\sin^2\frac{\theta}{2}}{2\sin\frac{\theta}{2}\cos\frac{\theta}{2}}\right)$

  $= 2\tan^{-1}.\tan\frac{\theta}{2} = \theta = tan^{-1}x =$ L.H.S.

\item $\because 0 < b \leq a \therefore \sqrt{\frac{a - b}{a + b}}$ is real.

  L.H.S. $= 2\tan^{-1}\left[\sqrt{\frac{a - b}{a + b}}\tan\frac{x}{2}\right]$

  $= \cos^{-1}\left[\frac{1 - \frac{a- b}{a + b}\tan^2\frac{x}{2}}{1 + \frac{a - b}{a + b}\tan^2\frac{x}{2}}\right]$

  $= \cos^{-1}\left[\frac{a\left(1 - \tan^2\frac{x}{2}\right)+ b\left(1 + \tan^2\frac{x}{2}\right)}{a\left(1 +
      \tan^2\frac{x}{2}\right)+ b\left(1 + \tan^2\frac{x}{2}\right)}\right]$

  $= \cos^{-1}\left[\frac{a\left(\frac{1 - \tan^2\frac{x}{2}}{1 + \tan^2\frac{x}{2} + b}\right)}{a + b\frac{1 -
        \tan^2\frac{x}{2}}{1 + \tan^2\frac{x}{2}}}\right]$

  $= \cos^{-1}\left[\frac{b + a\cos x}{a + b\cos x}\right] =$ R.H.S.

\item L.H.S $= \tan^{-1}\frac{x - y}{1 + xy} + \tan^{-1}\frac{y - z}{1 + yz} + \tan^{-1}\frac{z - x}{1 + zx}$

  $= \tan^{-1}x - \tan^{-1}y + \tan^{-1}y - \tan^{-1}z + \tan^{-1}z - \tan^{-1}x = 0$

  R.H.S. $= \tan^{-1}\left(\frac{x^2 - y^2}{1 + x^2y^2}\right) + \tan^{-1}\left(\frac{y^2 - z^2}{1 + y^2z^2}\right) +
  \tan^{-1}\left(\frac{z^2 - x^2}{1 + z^2x^2}\right)$

  $= \tan^{-1}x^2 - \tan^{-1}y^2 + \tan^{-1}y^2 - \tan^{-1}z^2 + \tan^{-1}z^2 - \tan^{-1}x^2 = 0$

  $\therefore$ L.H.S. = R.H.S.

\item We have to prove that $\sin\cot^{-1}\tan\cos^{-1}x = x$

  L.H.S. $= \sin\cot^{-1}\tan\tan^{-1}\frac{\sqrt{1 - x^2}}{x}$

  $= \sin\cot^{-1}\frac{\sqrt{1 - x^2}}{x}$

  Let $\cot^{-1}\frac{1 - x^2}{x} = \theta$ then $\cot\theta = \frac{\sqrt{1 - x^2}}{x}$

  $\sin\theta = x$

  Thus, L.H.S. = R.H.S.

\item **Case I:** When $\frac{\pi}{4}<x<\frac{\pi}{2}$

  $0<\cot x< 1$ and $0<\cot^3x<1 \therefore 0<\cot x\cot^3x<1$

  $\tan^{-1}\cot x + \tan^{-1}\cot^3x =\tan^{-1}\frac{\cot x + \cot^3x}{1 - \cot x\cot^3x}$

  $=\tan^{-1}\frac{\cot x}{1 - \cot^2x} = \tan^{-1}\frac{\tan x}{\tan^2 - 1}$

  $= -\tan^{-1}\left(\frac{1}{2}\tan 2x\right)$

  $\Rightarrow \tan^{-1}\left(\frac{1}{2}\tan 2x\right) + \tan^{-1}\cot x + \tan^{-1}\cot^3x = 0$

  **Case II:** When $0<x<\frac{\pi}{4}$

  $\cot x> 1$ and $\cot^3x > 1$

  $\Rightarrow \tan^{-1}\cot x + \tan^{-1}\cot^3x = \pi - \tan^{-1}\left(\frac{1}{2}\tan2x\right)$

  $\Rightarrow \tan^{-1}\left(\frac{1}{2}\tan 2x\right) + \tan^{-1}\cot x + \tan^{-1}\cot^3x = \pi$

\item $\tan^{-1}\frac{1}{2} + \tan^{-1}\frac{1}{3} = \tan^{-1}\frac{\frac{1}{2} + \frac{1}{3}}{1 - \frac{1}{2}.\frac{1}{3}}$

  $= \tan^{-1}\frac{5/6}{5/6} = \tan^{-1}1 = \frac{\pi}{4}$

  $\tan^{-1}\frac{3}{5} + \tan^{-1}\frac{1}{4} = \tan^{-1}\frac{\frac{3}{5} + \frac{1}{4}}{1 - \frac{3}{5}.\frac{1}{4}}$

  $=\tan^{-1}\frac{17/20}{17/20} = \tan^{-1}1 = \frac{\pi}{4}$

\item We have to prove that $\tan^{-1}\frac{2a - b}{\sqrt{3}b} + \tan^{-1}\frac{2b - a}{\sqrt{3}a} = \frac{\pi}{3}$

  L.H.S. $= \tan^{-1}\frac{\frac{2a - b}{\sqrt{3}b} + \frac{2b - a}{\sqrt{3}a}}{1 - \frac{(2a - b)(2b - a)}{3ab}}$

  $= \tan^{-1}\frac{\frac{2\sqrt{3}a^2 - \sqrt{3}ab + 2\sqrt{3}b^2 -\sqrt{3}ab}{3ab}}{\frac{3ab - 4ab + 2a^2 +
      2b^2 - ab}{3ab}}$

  $= \tan^{-1}\frac{2\sqrt{a^2} + 2\sqrt{3}b^2 - 2\sqrt{3}ab}{2a^2 + 2b^2 - 2ab} = \tan^{-1}\sqrt{3} = \frac{\pi}{3}$

\item We have to prove that $\tan^{-1}\frac{2}{5} + \tan^{-1}\frac{1}{3} + \tan^{-1}\frac{1}{12} = \frac{\pi}{4}$

  L.H.S. $= \tan^{-1}\frac{2}{5} + \tan^{-1}\frac{\frac{1}{3} + \frac{1}{12}}{1 - \frac{1}{3}.\frac{1}{12}}$

  $= \tan^{-1}\frac{2}{5} + \tan^{-1}\frac{5/12}{35/36} = \tan^{-1}\frac{2}{5} + \tan^{-1}\frac{3}{7}$

  $= \tan^{-1}\frac{\frac{2}{5} + \frac{3}{7}}{1 - \frac{2}{5}.\frac{3}{7}}$

  $= \tan^{-1}\frac{29/35}{29/35} = \tan^{-1}1 = \frac{\pi}{4}$

\item We have to prove that $2\tan^{-1}\frac{1}{5} + \tan^{-1}\frac{1}{4} = \tan^{-1}\frac{32}{43}$

  L.H.S. $= \tan^{-1}\frac{2.\frac{1}{5}}{1 - \frac{1}{5^2}} + \tan^{-1}\frac{1}{4}$

  $= \tan^{-1}\frac{5}{12} + \tan^{-1}\frac{1}{4}$

  $= \tan^{-1}\frac{\frac{5}{12} + \frac{1}{4}}{1 - \frac{5}{12}.\frac{1}{4}}$

  $= \tan^{-1}\frac{2/3}{43/48} = \tan^{-1}\frac{32}{43}$

\item We have to prove that $\tan^{-1}1 + \tan^{-1}2 + \tan^{-1}3 = \pi = 2\left(\tan^{-1}1 + \tan^{-1}\frac{1}{2} +
  \tan^{-1}\frac{1}{3}\right)$

  $\tan^{-1}1 + \tan^{-1}2 + \tan^{-1}3 = \tan^{-1}1 + \tan^{-1}\frac{2 + 3}{1 - 2.3} = \tan^{-1}1 + \tan^{-1}(-1)$

  $= \tan^{-1}\frac{1 - 1}{1 + 1} =\tan^{-1}0 = n\pi$

  $2\left(\tan^{-1}1 + \tan^{-1}\frac{1}{2} + \tan^{-1}\frac{1}{3}\right)$

  $= 2\left(\tan^{-1}1 + \tan^{-1}\frac{\frac{1}{2} + \frac{1}{2}}{1 - \frac{1}{2}.\frac{1}{3}}\right)$

  $= 2\left(\frac{\pi}{4} + \tan^{-1}1\right) = 2.\frac{\pi}{2} = \pi$

  Thus, the above expression will have principal value as $pi.$

\item We have to prove that $\tan^{-1}x + \cot^{-1}y = \tan^{-1}\frac{xy + 1}{y - x}$

  L.H.S. $= \tan^{-1}x + \cot^{-1}y = \tan^{-1}x + \tan^{-1}\frac{1}{y}$

  $= \tan^{-1}\frac{x + \frac{1}{y}}{1 - x.\frac{1}{y}} = \tan^{-1}\frac{xy + 1}{y - x}$

\item We have to prove that $\tan^{-1}\frac{1}{x + y} + \tan^{-1}\frac{y}{x^2 + xy + 1} = \cot^{-1}x$

  L.H.S. $= \tan^{-1}\frac{1}{x + y} + \tan^{-1}\frac{y}{x^2 + xy + 1}$

  $= \tan^{-1}\frac{\frac{1}{x + y} + \frac{y}{x^2 + xy + 1}}{1 - \frac{1}{x + y}.\frac{y}{x^2 + xy + 1}}$

  $= \tan^{-1}\frac{x^2 + 2xy + y^2 + 1}{x^3 + 2x^2y + xy^2 + x} = \tan^{-1}\frac{1}{x} = \cot^{-1}x$

\item We have to prove that $2\cot^{-1}5 + \cot^{-1}7 + 2\cot^{-1}8 = \pi/4$

  We know that $\cot^{-1}x + \cot^{-1}y = \frac{xy - 1}{x + y}$

  $\therefore 2\left(\cot^{-1}5 + \cot^{-1}8\right) = 2\cot^{-1}\frac{39}{13} = 2\cos^{-1}3 = \cot^{-1}\frac{4}{3}$

  $\therefore 2\cot^{-1}5 + \cot^{-1}7 + 2\cot^{-1}8 = \cot^{-1}\frac{4}{3} + \cot^{-1}7$

  $=\cot^{-1}\frac{\frac{28}{3} - 1}{\frac{25}{3}} = \cot^{-1}1 = \pi/4$

\item We have to prove that $\tan^{-1}\frac{a - b}{1 + ab} + \tan^{-1}\frac{b - c}{1 + bc} + \tan^{-1}\frac{c - a}{1 + ca} = 0$

  L.H.S. $= \tan^{-1}a - \tan^{-1}b + \tan^{-1}b - \tan^{-1}c + \tan^{-1}c - \tan^{-1}a = 0$

\item We have to prove that $\tan^{-1}\frac{a^3 - b^3}{1 + a^3b^3} + \tan^{-1}\frac{b^3 - c^3}{1 + b^3c^3} +
  \tan^{-1}\frac{c^3 - a^3}{1 + c^3a^3} = 0$

  L.H.S. $= \tan^{-1}a^3 - \tan^{-1}b^3 + \tan^{-1}b^3 - \tan^{-1}c^3 + \tan^{-1}c^3 - \tan^{-1}a^3 = 0$

\item We have to prove that $\cot^{-1}\frac{xy + 1}{y - x} + \cot^{-1}\frac{yz + 1}{z - y} + \cot^{-1}z = \tan^{-1}\frac{1}{x}$

  L.H.S. $= \cot^{-1}x - \cot^{-1}y + \cot^{-1}y - \cot^{-1}z + \cot^{-1}z= \cot^{-1}x = \tan^{-1}\frac{1}{x}$

\item We have to prove that $\cos^{-1}\left(\frac{\cos\theta + \cos\phi}{1 + \cos\theta\cos\phi}\right) =
  2\tan^{-1}\left(\tan\frac{\theta}{2}\tan\frac{\phi}{2}\right)$

  L.H.S. $= \cos^{-1}\left(\frac{\cos\theta + \cos\phi}{1 + \cos\theta\cos\phi}\right)$

  $= \tan^{-1}\frac{\sqrt{1 + \cos^2\theta\cos^2\phi + 2\cos\theta\cos\phi - \cos^2\theta\cos^2\phi -
      2\cos\theta\cos\phi}}{\cos\theta + \cos\phi}$

  $= \tan^{-1}\frac{\sqrt{(1 - \cos^2\theta)(1 - \cos^2\phi)}}{\cos\theta + \cos\phi} =
  \tan^{-1}\frac{\sin\theta\sin\phi}{\cos\theta + \cos\phi}$

  R.H.S. $= 2\tan^{-1}\left(\tan\frac{\theta}{2}\tan\frac{\phi}{2}\right)$

  $= \tan^{-1}\frac{2\tan\frac{\theta}{2}\tan\frac{\phi}{2}}{1 - \tan^2\frac{\theta}{2}\tan^2\frac{\phi}{2}}$

  $=\tan^{-1}\frac{2\tan\frac{\theta}{2}\tan\frac{\phi}{2}.\cos^2\frac{\theta}{2}\cos^2\frac{\phi}{2}}{\cos^2\frac{\theta}{2}\cos^2\frac{\phi}{2}
    - \sin^2\frac{\theta}{2}\sin^2\frac{\phi}{2}}$

  $= \tan^{-1}\frac{1}{2}.\frac{\sin\theta\sin\phi}{\cos^2\frac{\theta}{2}\cos^2\frac{\phi}{2} - \left(1 -
    \cos^2\frac{\theta}{2}\right)\left(1-\cos^2\frac{\phi}{2}\right)}$

  $= \tan^{-1}\frac{\sin\theta\sin\phi}{\cos\theta + \cos\phi}$

\item We have to prove that $\sin^{-1}\frac{3}{5} + \sin^{-1}\frac{8}{17} = \sin^{-1}\frac{77}{85}$

  L.H.S. $= \sin^{-1}\frac{3}{5} + \sin^{-1}\frac{8}{17}$

  $=\sin^{-1}\left(\frac{3}{5}\sqrt{1 - \frac{8^2}{17^2}} + \frac{8}{17}\sqrt{1 - \frac{3^2}{5^2}}\right)$

  $= \sin^{-1}\left(\frac{3}{5}.\frac{15}{17} + \frac{8}{17}.\frac{4}{5}\right)$

  $\sin^{-1}\left(\frac{45 + 32}{85}\right) = \sin^{-1}\frac{77}{85} =$ R.H.S.

\item We have to prove that $\cos^{-1}\frac{3}{5} + \cos^{-1}\frac{12}{13} + \cos^{-1}\frac{63}{65} = \frac{\pi}{2}$

  We know that $\cos^{-1}x + \cos^{-1}y = xy - \sqrt{(1 - x^2)(1 - y^2)}$

  L.H.S. $= \cos^{-1}\frac{3}{5} + \cos^{-1}\frac{12}{13} + \cos^{-1}\frac{63}{65}$

  $= \cos^{-1}\left(\frac{3}{5}.\frac{12}{13} - \sqrt{\left(1 - \frac{3^2}{5^2}\right)\left(1 -
    \frac{12^2}{13^2}\right)}\right) + \cos^{-1}\frac{63}{65}$

  $= \cos^{-1}\left(\frac{36}{65} - \frac{4}{5}.\frac{5}{13}\right) + \cos^{-1}\frac{63}{65}$

  $= \cos^{-1}\left(\frac{36}{65} - \frac{20}{65}\right) + \cos^{-1}\frac{63}{65}$

  $= \cos^{-1}\frac{16}{65} + \cos^{-1}\frac{63}{65}$

  $= \cos^{-1}\left(\frac{16}{65}.\frac{63}{64} - \sqrt{\left(1 - \frac{16^2}{65^2}\right)\left(1 -
    \frac{63^2}{65^2}\right)}\right)$

  $= \cos^{-1}0 = \frac{\pi}{2} =$ R.H.S.

\item We have to prove that $\sin^{-1}x + \sin^{-1}y = \cos^{-1}\left(\sqrt{1 - x^2}\sqrt{1 - y^2} - xy\right)$

  L.H.S. $= \sin^{-1}x + \sin^{-1}y = \cos^{-1}\sqrt{1 - x^2} + \cos^{-1}\sqrt{1 - y^2}$

  $= \cos^{1-}(\sqrt{1 - x^2}\sqrt{1 - y^2} - \sqrt{[1 - (1 - x^2)][1 - (1 - y^2)]})$

  $= \cos^{-1}\left(\sqrt{1 - x^2}\sqrt{1 - y^2} - xy\right) =$ R.H.S.

\item We have to prove that $4\left(\sin^{-1}\frac{1}{\sqrt{10}} + \cos^{-1}\frac{2}{\sqrt{5}}\right) =\pi$

  or $\sin^{-1}\frac{1}{\sqrt{10}} + \cos^{-1}\frac{2}{\sqrt{5}} =\pi/4$

  L.H.S. $= \sin^{-1}\frac{1}{\sqrt{10}} + \sin^{-1}\frac{1}{\sqrt{5}}$

  $= \sin^{-1}\left(\frac{1}{\sqrt{10}}\sqrt{1 - \frac{1}{5}} + \frac{1}{\sqrt{5}}\sqrt{1 - \frac{1}{10}}\right)$

  $= \sin^{-1}\left(\frac{2}{\sqrt{50}} + \frac{3}{\sqrt{50}}\right) = \sin^{-1}\frac{1}{\sqrt{2}}$

  $= \frac{\pi}{4} =$ R.H.S.

\item We have to prove that $\cos(2\sin^{-1}x) = 1 - 2x^2$

  L.H.S. $= \cos[\sin^{-1}(2x\sqrt{1 - x^2})] = \cos[\cos^{-1}\sqrt{1 - 4x^2(1 - x^2)}] = \cos[\cos^{-1}(1 - 2x^2)]$

  $= 1 - 2x^2 =$ R.H.S.

\item We have to prove that $\frac{1}{2}\cos^{-1}x = \sin^{-1}\sqrt{\frac{1 - x}{2}} = \cos^{-1}\sqrt{\frac{1 + x}{2}} =
  \tan^{-1}\frac{\sqrt{1 - x^2}}{1 + x}$

  or $\cos^{-1}x = 2\sin^{-1}\sqrt{\frac{1 - x}{2}} = 2\cos^{-1}\sqrt{\frac{1 + x}{2}} = 2\tan^{-1}\frac{\sqrt{1 -
      x^2}}{1 + x}$

  $2\sin^{-1}\sqrt{\frac{1 - x}{2}} = \sin^{-1}\left[2.\sqrt{\frac{1 - x}{2}}.\sqrt{1 - \frac{1 - x}{2}}\right]$

  $= \sin^{-1}2.\sqrt{\frac{1 - x}{2}}.\sqrt{\frac{1 + x}{2}} = \sin^{-1}\sqrt{1 - x^2} = \cos^{-1}x$

  $2\cos^{-1}\sqrt{\frac{1 + x}{2}} = \cos^{-1}\left[2.\frac{1 + x}{2} - 1\right][\because 2\cos^{-1}x = \cos^{-1}(2x^2 -
    1)]$

  $= \cos^{-1}x$

  $2\tan^{-1}\frac{\sqrt{1 - x^2}}{1 + x} = \tan^{-1}\frac{2.\frac{\sqrt{1 - x^2}}{(1 + x)}}{1 - \frac{1 - x^2}{(1 + x)^2}}$

  $= \tan^{-1}\frac{\sqrt{1 - x^2}}{x} = \cos^{-1}x$

\item We have to prove that $\sin^{-1}x + \cos^{-1}y = \tan^{-1}\frac{xy + \sqrt{(1 - x^2)(1 - y^2)}}{y\sqrt{1 - x^2} -
  x\sqrt{1 - y^2}}$

  L.H.S. $= \sin^{-1}x + \cos^{-1}y = \sin^{-1}x + \sin^{-1}\sqrt{1 - y^2}$

  $= \sin^{-1}[x\sqrt{1 -(1 - y^2)} + \sqrt{1 - y^2}\sqrt{1 - x^2}]$

  $=\tan^{-1}\frac{xy + \sqrt{1 - x^2}\sqrt{1 - y^2}}{\sqrt{1 - (xy + \sqrt{1 - x^2}(1 - y^2))^2}}$

  $=\tan^{-1}\frac{xy + \sqrt{1 - x^2}\sqrt{1 - y^2}}{\sqrt{1 - x^2y^2 - (1 - x^2)(1 - y^2) - 2xy\sqrt{1 - x^2}\sqrt{1 -
        y^2}}}$

  $= \tan^{-1}\frac{xy + \sqrt{1 - x^2}\sqrt{1 - y^2}}{\sqrt{x^2 + y^2 - 2xy\sqrt{1 - x^2}\sqrt{1 - y62}}}$

  $= \tan^{-1}\frac{xy + \sqrt{(1 - x^2)(1 - y^2)}}{y\sqrt{1 - x^2} - x\sqrt{1 - y^2}}$

\item We have to prove that $\tan^{-1}x + \tan^{-1}y = \frac{1}{2}\sin^{-1}\frac{2(x + y)(1 - xy)}{(1 + x^2)(1 + y^2)}$

  or $2(\tan^{-1}x + \tan^{-1}y = \sin^{-1}\frac{2(x + y)(1 - xy)}{(1 + x^2)(1 + y^2)}$

  L.H.S. $2\tan^{-1}\frac{x + y}{1 - xy} = \tan^{-1}\frac{2.\frac{x + y}{1 - xy}}{1 - \frac{(x + y)^2}{(1 - xy)^2}}$

  $= \tan^{-1}\frac{2(x + y)(1 - xy)}{(1 + x^2y^2 - 2xy - x^2 - Y^2 - 2xy)}$

  $= \sin^{-1}\frac{2(x + y)(1 - xy)}{\sqrt{4(x + y)^2(1 - xy)^2  + (1 + 2x^2y^2 - 4xy - x^2 - y^2)^2}}$

  $= \sin^{-1}\frac{2(x + y)(1 - xy)}{(1 + x^2)(1 + y^2)}$

\item We have to prove that $2\tan^{-1}(\csc\tan^{-1}x - \tan\cot^{-1}x) = \tan^{-1}x$

  L.H.S. $= 2\tan^{-1}(\csc\tan^{-1}x - \tan\cot^{-1}x)$

  $= 2\tan^{-1}\left(\csc\csc^{-1}\frac{\sqrt{1 + x^2}}{x} - \tan\tan^{-1}\frac{1}{x}\right)$

  $= 2\tan^{-1}\left(\frac{\sqrt{1 + x^2}}{x} - \frac{1}{x}\right) = 2\tan^{-1}\frac{\sqrt{1 + x^2} - 1}{x}$

  $= \tan^{-1}\frac{2.\frac{\sqrt{1 + x^2} - 1}{x}}{1 - \left(\frac{\sqrt{1 + x^2} - 1}{x}\right)^2}$

  $= \tan^{-1}x$

\item We have to prove that $\cos\tan^{-1}\sin\cot^{-1}x = \sqrt{\frac{x^2 + 1}{x^2 + 2}}$

  L.H.S. $= \cos\tan^{-1}\sin\cot^{-1}x = \cos\tan^{-1}\sin\sin^{-1}\frac{1}{\sqrt{1 + x^2}}$

  $= \cos\tan^{-1}\frac{1}{\sqrt{1 + x^2}} = \cos\cos^{-1}\frac{\sqrt{1 + x^2}}{\sqrt{x^2 + 2}}$

  $= \sqrt{\frac{x^2 + 1}{x^2 + 2}}$

\item Clearly in a triangle $A + B + C = \pi$ where $A, B, C$ are angled of the triangle.

  Thus, $\pi - C = A + B$

  Given $A + B = \tan^{-1}2 + \tan^{-1}3 = \tan^{-1}\frac{2 + 3}{1 - 2.3} = \tan^{-1}-1 = \frac{3\pi}{4}$

  $C = \pi - 3\pi/4 = \pi/4$

\item Given $\cos^{-1}x + \cos^{-1}y + \cos^{-1}z = \pi$

  $\Rightarrow \cos^{-1}x + \cos^{-1}y = \pi - \cos^{-1}z$

  $\Rightarrow xy - \sqrt{1 - x^2}\sqrt{1 - y^2} = \cos(\pi - \cos^{-1}z) = -z$

  $\Rightarrow xy + z = \sqrt{1 - x^2}\sqrt{1 - y^2}$

  Squaring, we get

  $x^2y^2 + z^2 + 2xyz = 1 - x^2 - y^2 + x^2y^2$

  $\Rightarrow x^2 + y^2 + z^2 + 2xyz = 1$

\item Given $\cos^{-1}\frac{x}{2}+ \cos^{-1}\frac{y}{3} = \theta$

  $\Rightarrow \cos^{-1}\left[\frac{xy}{6} - \frac{\sqrt{(4 - x^2)(9 - y^2)}}{6}\right] = \theta$

  $\Rightarrow xy - 6\cos\theta = \sqrt{(4 - x^2)(9 - y^2)}$

  Squaring, we get

  $x^2y^2 + 36\cos^2\theta - 12xy\cos\theta = 36 -9x^2 - 4y^2 + x^2y^2$

  $\Rightarrow 9x^2 - 12xy\cos\theta + 4y^2 = 36\sin^2\theta$

\item Let $\sqrt{\frac{xr}{yz}} = a, \sqrt{\frac{yr}{zx}} = b$ and $\sqrt{\frac{zr}{xy}} = c$


  Then, L.H.S. $= \tan^{-1}a + \tan^{-1}b + \tan^{-1}c = \frac{a + b + c- abc}{1 - ab - bc - ca}$

  Now, $a + b + c - abc = \frac{x\sqrt{r} + y\sqrt{r} + z\sqrt{r}}{\sqrt{xyz}} - \frac{r\sqrt{r}}{\sqrt{xyz}}$

  $= \frac{\sqrt{r}(x + y + z) - r\sqrt{r}}{\sqrt{xyz}} = 0$

  and, $1 - ab - bc - ca = 1- r\left[\frac{1}{x} + \frac{1}{y} + \frac{1}{z}\right]\neq 0[\because \frac{1}{x} +
    \frac{1}{y} + \frac{1}{z}\neq \frac{1}{r}]$

  $\Rightarrow$ L.H.S. $= 0 = n\pi$ and hence principal value is $\pi$ because sum of three positive angles
  cannot be zero or negative.

\item Given $u = \cot^{-1}\sqrt{\cos2\theta} - \tan^{-1}\sqrt{\cos2\theta}$

  $\sin u = \sin[\cot^{-1}\sqrt{\cos2\theta} - \tan^{-1}\sqrt{\cos2\theta}]$

  $= \sin\left[\tan^{-1}\frac{1}{\sqrt{\cos2\theta}} - \tan^{-1}\sqrt{\cos2\theta}\right]$

  $=\sin\left[\tan^{-1}\frac{\frac{1}{\sqrt{\cos2\theta}} - \sqrt{\cos2\theta}}{1 + 1}\right]$

  $=\sin\left[\tan^{-1}\frac{1}{2}\frac{2\sin^2\theta}{\sqrt{\cos2\theta}}\right]$

  $= \sin\left[\sin^{-1}\frac{\sin^2\theta}{\sqrt{\sin^4\theta + \cos2\theta}}\right]$

  $= \sin\left[\sin^{-1}\frac{\sin^2\theta}{(1 - \sin^2\theta)}\right] = \sin\sin^{-1}\tan^2\theta = \tan^2\theta$

\item Given $\cos^{-1}x\sqrt{3} + \cos^{-1}x = \frac{\pi}{2}$

  $\cos^{-1}x\sqrt{3} = \frac{\pi}{2} - \cos^{-1}x \Rightarrow \cos\cos^{-1}x\sqrt{3} = \cos\left(\frac{\pi}{2} -
  \cos^{-1}x\right)$

  $\Rightarrow x\sqrt{3} = \sin\cos^{-1}x = \sin\sin^{-1}\sqrt{1 - x^2}$

  $\Rightarrow x\sqrt{3} = \sqrt{1  - x^2}$

  $\Rightarrow 3x^2 = 1 - x^2 \Rightarrow x = \pm\frac{1}{2}$

  **Case I:** When $x = \frac{1}{2},$ given equation becomes

  $\cos^{-1}\frac{\sqrt{3}}{2} + \cos^{-1}\frac{1}{2} = \frac{\pi}{6} + \frac{\pi}{3} = \frac{\pi}{2}$

  **Case II:** When $x = -\frac{1}{2},$

  $\cos^{-1}-\frac{\sqrt{3}}{2} + \cos^{-1}-\frac{1}{2} = \pi - \cos^{-1}\frac{\sqrt{3}}{2} + \pi - \cos^{-1}\frac{1}{2}$

  $= \frac{3\pi}{2}\neq \frac{\pi}{2}$

  Thus, $x = \frac{1}{2}$ is the only solution.

\item Given equation is $\sin^{-1}x + \sin^{-1}2x = \frac{\pi}{3}$

  $\Rightarrow \sin^{-1}x + \sin^{-1}2x = \sin^{-1}\frac{\sqrt{3}}{2}$

  $\Rightarrow \sin{-1}x - \sin^{-1}\frac{\sqrt{3}}{2} = -\sin^{-1}2x$

  $\Rightarrow \sin^{-1}\left[\frac{x}{2} - \frac{\sqrt{3}}{2}\sqrt{1 - x^2}\right] = -\sin^{-1}2x$

  $\Rightarrow x - \sqrt{3(1 - x^2)} = -4x \Rightarrow 25x^2 = 3(1 - x^2) \Rightarrow x = \pm\frac{\sqrt{3}}{2\sqrt{7}}$

  Clearly, $x = -\frac{\sqrt{3}}{2\sqrt{7}}$ as angles will become negative and won't satisfy the equality.

\item Given, $\tan^{-1}x + \tan^{-1}y + \tan^{-1}z= \frac{\pi}{2},$ we have to prove that $xy + yz + zx = 1$

  $\tan^{-1}x + \tan^{-1}y + \tan^{-1}z= \frac{\pi}{2}$

  $\Rightarrow \tan^{-1}\frac{x + y + z - xyz}{1 - xy - yz - zx} = \frac{\pi}{2}$

  $\Rightarrow \frac{x + y + z - xyz}{1 - xy - yz - zx} = \infty$

  $\Rightarrow 1 - xy - yz - zx = 0$

  $\Rightarrow x y + yz + zx = 1$

\item Given $\tan^{-1}x + \tan^{-1}y + \tan^{-1}z= \pi,$ we have to prove that $x + y + z = xyz$

  $\tan^{-1}x + \tan^{-1}y + \tan^{-1}z= \pi$

  $\Rightarrow \tan^{-1}\frac{x + y + z - xyz}{1 - xy - yz - zx} = \pi$

  $\Rightarrow \frac{x + y + z - xyz}{1 - xy - yz - zx} = \tan \pi = 0$

  $\Rightarrow x + y + z = xyz$

\item Given $\sin^{-1}x + \sin^{-1}y = \frac{\pi}{2},$ we have to prove that $x\sqrt{1 - y^2} + y\sqrt{1 - x^2} = 1$

  $\Rightarrow \sin^{-1}(x\sqrt{1 - y^2} + y\sqrt{1 - x^2}) = \frac{\pi}{2}$

  $\Rightarrow x\sqrt{1 - y^2} + y\sqrt{1 - x^2} = 1$

\item Give $\sin^{-1}x + \sin^{-1}y + \sin^{-1}z = \pi,$ we have to prove that $x\sqrt{1 - x^2} + y\sqrt{1 - y^2} +
  z\sqrt{1 - z^2} = 2xyz$

  Let $\sin^{-1}x = A, \sin^{-1}y = B$ and $\sin^{-1}z = C$

  Then, $A + B + C = \pi \Rightarrow A + B = \pi - C$

  We have to prove that $\sin A\sqrt{1 - \sin^2A} + B\sqrt{1 - \sin^2B} + z\sqrt{1 - \sin^2C} = 2\sin A\sin B\sin C$

  L.H.S. $= \sin A\cos A + \sin B\cos B + \sin C\cos C$

  $= \frac{1}{2}(\sin 2A + \sin 2B + \sin 2C) = \sin(A + B)\cos(A - B) + \sin C\cos[\pi - (A + B)]$

  $= \sin C[\cos(A - B) - \cos(A + B)][\because \sin(A + B) = \sin(\pi - C) = \sin C]$

  $= 2\sin A\sin B \sin C =$ R.H.S.

\item Form given conditions $2\tan^{-1}y = \tan^{x} + \tan^{-1}z$ and $2y = x + z$

  $\Rightarrow \frac{2y}{1 - y^2} = \frac{x + z}{1 - zx}$

  $\Rightarrow 1 - y^2 = 1 - zx \Rightarrow y^2 = zx$

  i.e. A.M. = G.M which is true only if $x = y = z$

\item Given $\cot^{-1}x + \sin^{-1}\frac{1}{\sqrt{5}} = \frac{\pi}{4}$

  $\Rightarrow \cot^{-1}x + \cot^{-1}\sqrt{(\sqrt{5})^2 - 1} = \frac{\pi}{4}$

  $\Rightarrow \cot^{-1}x + \cot^{-1}2 = \frac{\pi}{4}$

  $\Rightarrow \frac{2x - 1}{x + 2} = \cot\frac{\pi}{4} = 1 \Rightarrow x = 3$

\item We have to solve $\tan^{-1}2x + \tan^{-1}3x = \frac{\pi}{4}$

  $\Rightarrow \tan^{-1}\frac{2x + 3x}{1 - 2x.3x} = \frac{\pi}{4}$

  $\Rightarrow \frac{5x}{1 - 6x^2} = \tan\frac{\pi}{4} =1$

  $\Rightarrow 6x^2 + 5x - 1 = 0 \Rightarrow (6x - 1)(x + 1) = = 0$

  $\Rightarrow x = -1, \frac{1}{6}$

  Clearly, $x = -1$ does not satisfy the equation $\therefore x = \frac{1}{6}$

\item We have to solve $\tan^{-1} x + \tan^{-1}\frac{2x}{1 - x^2} = \frac{\pi}{3}$

  $\Rightarrow \tan^{-1}x + 2\tan^{-1}x = \frac{\pi}{3}$

  $\Rightarrow 3\tan^{-1}x = \frac{\pi}{3}$

  $\Rightarrow x = \tan\frac{\pi}{9}$

\item We have to solve Solve $\tan^{-1}\frac{1}{2} = \cot^{-1}x + \tan^{-1}\frac{1}{7}$

  $\Rightarrow \tan^{-1}\frac{1}{2} - \tan^{-1}\frac{1}{7} = \tan^{-1}\frac{1}{x}$

  $\Rightarrow \tan^{-1}\frac{\frac{1}{2} - \frac{1}{7}}{1 - \frac{1}{2}.\frac{1}{7}} = \tan^{-1}\frac{1}{x}$

  $\Rightarrow \tan^{-1}\frac{5}{13} = \tan^{-1}\frac{1}{x}$

  $\Rightarrow x = \frac{13}{5}$

\item We have to solve $\tan^{-1}(x - 1) + \tan^{-1}x + \tan^{-1}(x + 1) = \tan^{-1}3x$

  $\Rightarrow \tan^{-1}(x - 1) + \tan^{-1}(x + 1) = \tan^{-1}3x - \tan^{-1}x$

  $\Rightarrow \frac{2x}{2 - x^2} = \frac{2x}{1 + 3x^2}$

  $\Rightarrow 2x(4x^2 - 1) = 0$

  $x = 0, \pm\frac{1}{2}$

\item We have to solve $\tan^{-1}\frac{x + 1}{x - 1} + \tan^{-1}\frac{x - 1}{x} = \pi + \tan^{-1}(-7)$

  $\Rightarrow \tan^{-1}\frac{x^2 + x + x^2 -2x + 1}{x - x - 1} = \pi + \tan^{-1}(-7)$

  $\Rightarrow 2x^2 - x + 1 = 7x - 7 \Rightarrow 2x^2 - 8x + 8 = 0$

  $\Rightarrow x^2 - 4x + 4 = 0 \Rightarrow x = 2$

\item We have to solve $\cot^{-1}(a - 1) = \cot^{-1}x + \cot^{-1}(a^2 - x + 1)$

  $\cot^{a - 1} \cot^{-1}\frac{a^2x - x^2 + x - 1}{a^2 + 1}$

  $\Rightarrow a^3 - a^2 + a - 1 = a^2x - x^2 + x - 1$

  $\Rightarrow x^2 - (1 + a^2)x + (a^3 - a^2  + a) = 0$

  $\Rightarrow (x - a)[x - (a^2 - a + 1)] = 0$

  $\Rightarrow x = a, a^2 - a + 1$

\item We have to solve $\sin^{-1}\frac{2\alpha}{1 + \alpha^2} + \sin^{-1}\frac{2\beta}{1 + \beta^2} = 2\tan^{-1}x$

  We know that $2\tan^{-1}x = \sin^{-1}\frac{2x}{1 + x^2}$

  Thus, given equation becomes $2(\tan^{-1}\alpha + \tan^{-1}\beta) = 2\tan^{-1}x$

  $\Rightarrow x = \frac{\alpha + \beta}{1 - \alpha\beta}$

\item We have to solve $\cos^{-1}\frac{x^2 - 1}{x^2 + 1} + \tan^{-1}\frac{2x}{x^2 - 1} = \frac{2\pi}{3}$

  **Case I:** $\Rightarrow \pi - 2\tan^{-1}x - 2\tan^{-1}x = \frac{2\pi}{3}$

  $\Rightarrow \tan^{-1}x = \frac{\pi}{12}$

  $x = 2 - \sqrt{3}$

  **Case II:** $\Rightarrow \pi - 2\tan^{-1}x + \pi - 2\tan^{-1}x = \frac{2\pi}{3}$

  $\tan^{-1}x = \frac{\pi}{3} \Rightarrow x = \sqrt{3}$

\item We have to solve $\sin^{-1}\frac{2a}{1 + a^2} + \cos^{-1}\frac{1 - b^2}{1 + b^2} = 2\tan^{-1}x$

  $\Rightarrow 2\tan^{-1}x + 2\tan^{-1}b = 2\tan^{-1}x$

  $x = \frac{a + b}{1 - ab}$

\item We have to solve $\sin^{-1}x + \sin^{-1}(1 - x) = \cos^{-1}x$

  $\Rightarrow \sin^{-1}\left(x\sqrt{2x - x^2} + (1 - x)\sqrt{1 - x^2}\right) = \sin^{-1}\sqrt{1 - x^2}$

  $\Rightarrow x\sqrt{2x - x^2} + (1 - x)\sqrt{1 - x^2} = \sqrt{1 - x^2}$

  $\Rightarrow x\sqrt{2x - x^2} = x\sqrt{1 - x^2}$

  Squaring, we get

  $\Rightarrow x^2\left(2x - x^2 -1 + x^2\right) = 0$

  $x = 0, \frac{1}{2}$

\item We have to solve $\tan^{-1}ax + \frac{1}{2}\sec^{-1}bx = \frac{\pi}{4}$

  $\Rightarrow 2\tan^{-1}ax + sec^{-1}bx = \frac{\pi}{2}$

  $\Rightarrow \tan^{-1}\frac{2ax}{1 - a^2x^2} + \tan^{-1}\sqrt{1 - b^2x^2} = \frac{\pi}{2}$

  $\Rightarrow \tan^{-1}\frac{\frac{2ax}{1 - a^2x^2} + \sqrt{1 - b^2x^2}}{1 - \frac{2ax}{1 - a^2x^2}\sqrt{1 - b^2x^2}} =
  \frac{\pi}{2}$

  $\Rightarrow 1 - \frac{2ax}{1 - a^2x^2}\sqrt{1 - b^2x^2} = 0$

  $\Rightarrow 1 - a^2x^2 - 2ax\sqrt{1 - b^2x^2} = 0$

  $\Rightarrow 1 - 2a^2x^2 + a^4x^4 = 4a^2x^2(1 - b^2x^2)$

  $\Rightarrow x = \pm \frac{1}{\sqrt{2ab - a^2}}$

\item We have to solve $\tan(\cos^{-1}x) = \sin(\tan^{-1}2)$

  $\Rightarrow \tan\tan^{-1}\frac{\sqrt{1 - x^2}}{x} = \sin\sin^{-1}\frac{2}{\sqrt{4 + 1}}$

  $\Rightarrow \frac{\sqrt{1 - x^2}}{x} = \frac{2}{\sqrt{5}}$

  $\Rightarrow 5(1 - x^2) = 4x^2 \Rightarrow x = \pm\frac{\sqrt{5}}{3}$

\item We have to solve $\tan\left(\sec^{-1}\frac{1}{x}\right) = \sin\cos^{-1}\frac{1}{\sqrt{5}}$

  $\Rightarrow \tan\tan^{-1}\frac{\sqrt{1 - x^2}}{x} = \sin\sin^{-1}\frac{2}{\sqrt{5}}$

  $\Rightarrow \frac{1 - x^2}{x^2} = \frac{4}{5}$

  $\Rightarrow x = \pm\frac{\sqrt{5}}{3}$

\item We have to solve $\sin^{-1}x + \sin^{-1}y = \frac{2\pi}{3}$ and $\cos^{-1}x - \cos^{-1}y = \frac{\pi}{3}$

  $\sin^{-1}x + \sin^{-1}y = \frac{2\pi}{3}$

  $\frac{\pi}{2} - \cos^{-1}x + \frac{\pi}{2} - \cos^{-1} y = \frac{2\pi}{3}$

  $\Rightarrow \cos^{-1}x + \cos^{-1}y = \frac{\pi}{3}$

  Thus, $2\cos^{-1}x = 2\frac{\pi}{3} \Rightarrow x = \cos\frac{\pi}{3} = \frac{1}{2}$

  and $2\cos^{-1}y = 0 \Rightarrow y = 1$

\item Let $\sin^{-1}(\sin10) = \theta \Rightarrow \sin\theta = \sin10 = \sin\frac{35\pi}{11}$

  $\sin\theta = \sin\left(3\pi + \frac{2\pi}{11}\right) = -\sin\frac{2\pi}{11}$

  $= \sin\left(-\frac{2\pi}{11}\right)$

  $\theta = -\frac{2\pi}{11}$

\item $3\tan^{-1}\left(\frac{1}{2}\right) = \tan^{-1}\left[\frac{3.\frac{1}{2} - \left(\frac{1}{2}\right)^3}{1 -
    3\left(\frac{1}{2}\right)^2}\right]\left[\because \tan3\theta = \frac{3\tan\theta - \tan^3\theta}{1 - 3\tan^2\theta}\right]$

  $= \tan^{-1}\left[\frac{\frac{11}{8}}{\frac{1}{4}}\right] = \tan^{-1}\frac{11}{2}$

  $2\tan^{-1}\frac{1}{5} = \tan^{-1}\left[\frac{2.\frac{1}{5}}{1 - \frac{1}{25}}\right] = \tan^{-1}\frac{5}{12}$

  Now $3\tan^{-1}\frac{1}{2} + 2\tan^{-1}\frac{1}{5} = \tan^{-1}\frac{11}{2} + \tan^{-1}\frac{5}{12}$

  $= \pi + \tan^{-1}\left[\frac{\frac{11}{2} + \frac{5}{12}}{1- \frac{11}{2}.\frac{5}{12}}\right] = \pi -
  \tan^{-1}\frac{142}{31}$

  Also, let $\sin^{-1}\frac{142}{65\sqrt{5}} = \theta$

  $\sin\theta = \frac{142}{65\sqrt{5}} \Rightarrow \tan\theta = \frac{142}{31}$

  Thus, $3\tan^{-1}\frac{1}{2} + 2\tan^{-1}\frac{1}{5} + \sin^{-1}\frac{142}{65\sqrt{5}} = \pi - \tan^{-1}\frac{142}{31} +
  \tan^{-1}\frac{142}{31}$

  $= \pi$

\item The given intervals indicate principal values of $\cos^{-1}x$ and $\sin^{-1}x$.

  $\cos[2\cos^{-1}x + \sin^{-1}x] = \cos(\cos^{-1}x + \cos^{-1}x + \sin^{-1}x)$

  $= \cos\left[\frac{\pi}{2} + \cos^{-1}x\right] = -\sin\cos^{-1}x = -\sin\sin^{-1}\sqrt{1 - x^2}$

  $= -\sqrt{1 - x^2} = -\sqrt{1 - \frac{1}{25}} = -\frac{2\sqrt{6}}{5}$.

\item We have to prove that $\frac{1}{2}\cos^{-1}\frac{3}{5} = \tan^{-1}\frac{1}{2} = \frac{\pi}{4} -
  \frac{1}{2}\cos^{-1}\frac{4}{5}$

  Let $\cos^{-1}\frac{3}{5} = \alpha, 2\tan^{-1}\frac{1}{2} = \beta$ and $\frac{\pi}{2} - \cos^{-1}\frac{4}{5} =
  \gamma$

  $\cos\alpha = \cos\cos^{-1}\frac{3}{5} = \frac{3}{5}$

  $\cos\beta = \cos\left[\cos^{-1}\frac{1 - \frac{1}{4}}{1 + \frac{1}{4}}\right] = \cos\cos^{-1}\frac{3}{5} = \frac{3}{5}$

  $\cos\gamma = \cos\left[\frac{\pi}{2} - \cos^{-1}\frac{4}{5}\right] = \sin\cos^{-1}\frac{4}{5} = \frac{3}{5}$

  Thus, $\alpha = \beta = \gamma$

\item Let $A = 2\tan^{-1}(2\sqrt{2} - 1) = 2\tan^{-1}(2\times 1.414 - 1) = 2\tan^{-1}(1.828)$

  $= 2\times (> 60^\circ)[\because \tan60^\circ = \sqrt{3} = 1.732]$

  Let $B = 3\sin^{-1}\frac{1}{3} + \sin^{-1}\frac{3}{5}$

  $= \sin^{-1}\left[3\times\frac{1}{3} - 4\left(\frac{1}{3}\right)^3\right] + \sin^{-1}\frac{3}{5}$

  $= \sin^{-1}\frac{23}{27} + \sin^{-1}\frac{3}{5} = \sin^{-1}0.862 + \sin^{-1}0.6$

  $= <60^\circ + <45^\circ < 105^\circ$

  Thus, $A$ is the greater angle.

\item Whenever you have to sum trigonometric series of inverse terms check if it is possible to write them as difference of two terms
  and add the terms where terms cancel each other. If we look at the terms given in this series then that is possible.

  $\tan^{-1}\left(\frac{a_1x - y}{x + a_1y}\right) = \tan^{-1}\left(\frac{a_1 - \frac{y}{x}}{1 + a_1\frac{y}{x}}\right) =
  \tan^{-1}a_1 - \tan^{-1}\frac{y}{x}$

  $\tan^{-1}\left(\frac{a_1 - a_1}{1 + a_1a_2}\right) = \tan^{-1}a_2 - \tan^{-1}a_1$

  $\ldots$

  $\tan^{-1}\left(\frac{a_n - a_{n - 1}}{1 + a_na_{n - 1}}\right) = \tan^{-1}a_n - \tan^{-1}a_{n - 1}$

  $\tan^{-1}\frac{1}{a_n} = \cot^{-1}a_n$

  Adding these, we get $L.H.S. = \tan^{-1}a_n + \cot^{-1}a_n - \tan^{-1}\frac{y}{x}$

  $= \frac{\pi}{2} - \tan^{-1}\frac{y}{x}\left[\because\tan^{-1}x + \cot^{-1}x = \frac{\pi}{2}\right]$

  $= \cot^{-1}\frac{y}{x} = \tan^{-1}\frac{x}{y} = R.H.S.$

\item Let $t_n$ denote the $n$-th term of the series, then $t_n = \cot^{-1}2n^2 = \cot^{-1}(2n - 1) -
  \cot^{-1}(2n + 1)$

  Putting $n = 1,2,3, ..,$ we get

  $t_1 = \cot^{-1}1 - \cot^{-1}3$

  $t_2 = \cot^{-1}3 - \cot^{-1}5$

  $t_3 = \cot^{-1}5 - \cot^{-1}7$

  $\ldots$

  $t_n = \cot^{-1}(2n - 1) - \cot^{-1}(2n + 1)$

  Adding $S_n = \cot^{-1}1 - \cot^{-1}(2n + 1)$

  As $n\rightarrow \infty, \cot^{-1}(2n + 1)\rightarrow 0$

  Hence, $S_\infty = \cot^{-1}1 = \frac{\pi}{4}$

\item **Case I.** When $x = 1$

  $y = 2\tan^{-1}x + \sin^{-1}\frac{2x}{1 + x^2} = 2.\tan^{-1}1 + \sin^{-1}\frac{2}{1 + 1} = 2.\frac{\pi}{4} +
  \frac{\pi}{2} = \pi$

  **Case II.** When $x > 1$

  $2\tan^{-1}x = \pi - \sin^{-1}\frac{2x}{1 + x^2} \Rightarrow y = \pi$

\item Let $\cos^{-1}x_0 = \theta \Rightarrow \cos\theta = x_0$

  We are also given that $x_{r + 1} = \sqrt{\frac{1 + x_r}{2}}$

  Putting $r = 0,$ we get $x_1 = \sqrt{\frac{1 + x_0}{2}} = \sqrt{\frac{1 + \cos\theta}{2}}$

  $= \sqrt{\cos^2\frac{\theta}{2}} = \left|\cos\frac{\theta}{2}\right| = \cos\frac{\theta}{2}[\because
    0\leq\cos^{-1}x_0\leq \pi]$

  Similarly, $x_2 = \sqrt{\frac{1 + \cos\frac{\theta}{2}}{2}} = \cos\frac{\theta}{2^2}$

  thus, $x_n = \cos\frac{\theta}{2^n}$

  Let $y = x_1x_2x_3\ldots x_n$ then $y = \cos\frac{\theta}{2}\cos\frac{\theta}{2^2}\ldots\cos\frac{\theta}{2^n}$

  $2y\sin\frac{\theta}{2^n} = 2\sin\frac{\theta}{2^n}\cos\frac{\theta}{2^n}\cos\frac{\theta}{2^{n -
      1}}\ldots\cos\frac{\theta}{2}$

  $2^2y\frac{\theta}{2^n} = 2\sin\frac{\theta}{2^{n - 1}}\cos\frac{\theta}{2^{n - 1}}\cos\frac{\theta}{2^{n -
      1}}\ldots\cos\frac{\theta}{2}$

  Proceeding like above, we finally arrive at following

  $2^{n - 1}y\sin\frac{\theta}{2^n} = \sin\frac{\theta}{2}\cos\frac{\theta}{2}$

  $2^ny\sin\frac{\theta}{2^n} = 2\sin\frac{\theta}{2}\cos\frac{\theta}{2} = \sin\theta$

  $y = \frac{1}{2^n}.\frac{\sin\theta}{\sin\frac{\theta}{2^n}}$

  $x_1x_2\ldots$ to $\infty = \lim_{n\to \infty} \frac{1}{2^n}\frac{\sin\theta}{\sin\frac{\theta}{2^n}}$

  $= \lim_{n\to\infty}\frac{1}{2^n}\frac{\sin\theta}{\frac{\sin\frac{\theta}{2^n}}{\frac{\theta}{2^n}}.\frac{\theta}{2^n}}$

  $= \frac{\sin\theta}{\theta}$

  R.H.S. $= \frac{\sqrt{1 - \cos^2\theta}}{\frac{\sin\theta}{\theta}} = \theta = \cos^{-1}x_0 =$ L.H.S.

\item Let $\cos^{-1}\frac{a}{b} = \theta \Rightarrow \cos\theta = \frac{a}{n}\Rightarrow a = b\cos\theta$

  Now, $a_1 = \frac{a + b}{2} = \frac{b\cos\theta + b}{2} = b\cos^2\frac{\theta}{2}$

  $b_1 = \sqrt{a_1b} = \sqrt{b\cos^2\frac{\theta}{2}.b} = b\cos\frac{\theta}{2}$

  $a_2 = \frac{a_1 + b_1}{2} = \frac{b\cos^2\frac{\theta}{2} + b\cos\frac{\theta}{2}}{2} =
  b\cos\frac{\theta}{2}\cos^2\frac{\theta}{2^2}$

  $b_2 = \sqrt{a_2b_1} = \sqrt{b\cos\frac{\theta}{2}.\cos^2\frac{\theta}{2^2}b\cos\frac{\theta}{2}} =
  b\cos\frac{\theta}{2}\cos\frac{\theta}{2^2}$

  Proceeding as above, we get $a_n = b\cos\frac{\theta}{2}\cos\frac{\theta}{2^2}\ldots\cos\frac{\theta}{2^n} =
  b.\frac{1}{2^n}.\frac{\sin\theta}{\sin\frac{\theta}{2^n}}$

  and $b_n = b\cos\frac{\theta}{2}\cos\frac{\theta}{2^2}\ldots\frac{\theta}{2^n}$

  Now, $\lim_{n\to\infty}a_n = \frac{b.\sin\theta}{\theta}$ [like in previous problem]

  $= \frac{b\sqrt{1 - \sin^2\theta}}{\cos^{-1}\frac{a}{b}} = \frac{b\sqrt{1 - \frac{a^2}{b^2}}}{\cos^{-1}\frac{a}{b}} =
  \frac{\sqrt{b^2 - a^2}}{\cos^{-1}\frac{a}{b}}$

  and $\lim_{n\to\infty}b_n = \lim_{n\to\infty}b\cos\frac{\theta}{2}\cos\frac{\theta}{2^2}\ldots\cos\frac{\theta}{2^n} =
  \frac{\sqrt{b^2 - a^2}}{\cos^{-1}\frac{a}{b}}$

\item We have to prove that $\tan^{-1}\frac{1}{3} + \tan^{-1}\frac{1}{7} + \ldots + \tan^{-1}\frac{1}{n^2 + n
  + 1} = \tan^{-1}\frac{n}{n + 2}$

  When $n = 1$, L.H.S. $=\tan^{-1}\frac{1}{3}$ and R.H.S. $= \tan^{-1}\frac{1}{1 + 2} =
  \tan^{-1}\frac{1}{3}$

  We see that it is true for $n = 1$. Let it is true for $n = 1$

  $\Rightarrow \tan^{-1}\frac{1}{3} + \tan^{-1}\frac{1}{7} + \ldots + \tan^{-1}\frac{1}{m^2 + m
    + 1} = \tan^{-1}\frac{m}{m + 2}$

  Adding $\tan^{-1}\frac{1}{(m + 1)^2 + (m + 1) + 1}$ to both sides, we get

  R.H.S. $= \tan^{-1}\frac{m}{m + 2} + \tan^{-1}\frac{1}{(m + 1)^2 + (m + 1) + 1}$

  $= \tan^{-1}\frac{m}{m + 1} + \tan^{-1}\frac{m + 1}{m + 3} - \tan^{-1}\frac{m}{m + 2}$

  $= \tan^{-1}\frac{(m + 1) + 1}{(m + 1) + 2}$

  Thus, it is true for $n = m + 1$ if it is true for $n = m$. Hence, we have proven the result by using mathematical
  induction.

\item Since $x_1, x_2, x_3, x_4$ are the roots of the equation $x^4 - x^3\sin2\beta + x^2\cos2\beta - x\cos\beta -
  \sin\beta = 0$

  $\therefore \sum x_1 = x_1 + x_2 + x_3 + x_4 = -\frac{-\sin2\beta}{1} = \sin2\beta$

  $\sum x_1x_2 = \cos2\beta$

  $\sum x_1x_2x_3 = \cos\beta$

  and $\sum x_1x_2x_3x_4 = -\sin\beta$

  Now $\tan[\tan^{-1}x_1 + \tan^{-1}x_2 + \tan^{-1}x_3 + \tan^{-1}x_4] = \frac{\sum x_1 - \sum x_1x_2x_3}{1 - \sum
    x_1x_2 + x_1x_2x_3x_4}$

  $= \frac{\sin2\beta - \cos\beta}{1 - \cos2\beta - \sin\beta} = \frac{2\sin\beta\cos\beta - \cos\beta}{2\sin^2\beta -
    \sin\beta}$

  $= \cot\beta$

  $\Rightarrow \tan[\tan^{-1}x_1 + \tan^{-1}x_2 + \tan^{-1}x_3 + \tan^{-1}x_4] = \tan\left(\frac{\pi}{2} -
  \beta\right)$

  $\Rightarrow \tan^{-1}x_1 + \tan^{-1}x_2 + \tan^{-1}x_3 + \tan^{-1}x_4 = n\pi + \frac{\pi}{2} - \beta$

\item Let $\cot^{-1}\left(\cot\frac{5\pi}{4}\right) = \theta \Rightarrow \cot\theta = \cot\left(\pi + \frac{\pi}{4}\right)$

  $= \cot\frac{\pi}{4}\Rightarrow \theta = \frac{\pi}{4}$

\item Let $\sin^{-1}(\sin5) = \theta \Rightarrow \sin\theta = \sin5 = \sin\frac{35\pi}{22} = \sin\left(\pi +
  \frac{13\pi}{22}\right)$

  $= -\sin\frac{13\pi}{22} \Rightarrow \sin\theta = -\sin\frac{13\pi}{22} = -\sin\left(\pi - \frac{9\pi}{22}\right)$

  $\theta = -\frac{9\pi}{22} = 5 - 2\pi$

\item Let $\cos^{-1}(\cos\frac{5\pi}{4}) = \theta \Rightarrow \cos\theta = \cos\left(2\pi - \frac{3\pi}{4}\right)$

  $\Rightarrow \cos\theta = \cos\frac{3\pi}{4} \Rightarrow \theta = \frac{3\pi}{4}$

\item Let $\cos^{-1}\cos10 = \theta \Rightarrow \cos\theta = \cos10 = \cos\frac{35\pi}{11} = \cos\left(3\pi +
  \frac{2\pi}{11}\right)$

  $\Rightarrow \cos\theta = -\cos\frac{2\pi}{11} = -\cos\left(\pi + \frac{-9\pi}{11}\right) = \cos\frac{-9\pi}{11}$

  $\Rightarrow \theta = \frac{-9\pi}{11}$

\item Given, $\sin\left(2\tan^{-1}\frac{1}{3}\right) + \cos\tan^{-1}2\sqrt{2}$

  $= \sin\tan^{-1}\frac{2.\frac{1}{3}}{1 - \frac{1}{9}} + \cos\cos^{-1}\frac{1}{3}$

  $= \sin\tan^{-1}\frac{3}{4} + \frac{1}{3} = \sin\sin^{-1}\frac{3}{5} + \frac{1}{3}$

  $= \frac{3}{5} + \frac{1}{3} = \frac{14}{15}$

\item Given, $\cot[\cot^{-1}7 + \cot^{-1}8 + \cot^{-1}18]$

  $\cot^{-1}7 + \cot^{-1}8 + \cot^{-1}18 = \tan^{-1}\frac{1}{7} + \tan^{-1}\frac{1}{8} + \tan^{-1}\frac{1}{18}$

  $= \tan^{-1}\left(\frac{\frac{1}{7} + \frac{1}{8}}{1 - \frac{1}{7}.\frac{1}{8}}\right) + \tan^{-1}\frac{1}{18}=
  \tan^{-1}\frac{15}{55} + \tan^{-1}\frac{1}{18}$

  $= \tan^{-1}\frac{3}{11} + \tan^{-1}\frac{1}{18} = \tan^{-1}\frac{\frac{3}{11} + \frac{1}{18}}{1 -
    \frac{3}{11}.\frac{1}{18}}$

  $= \tan^{-1}.\frac{65}{198}.\frac{198}{195} = \tan^{-1}\frac{1}{3} = \cot^{-1}3$

  $\therefore \cot[\cot^{-1}7 + \cot^{-1}8 + \cot^{-1}18] = 3$

\item We have to prove that that $\sin^{-1}\frac{3}{5} + \cos^{-1}\frac{12}{13} + \cot^{-1}\frac{56}{33} = \frac{\pi}{2}$

  $\sin^{-1}\frac{3}{5} = \tan^{-1}\frac{3}{4}$

  $\cos^{-1}\frac{12}{13} = \tan^{-1}\frac{5}{12}$

  $\cot^{-1}\frac{56}{33} = \tan^{-1}\frac{33}{56}$

  $\therefore \sin^{-1}\frac{3}{5} + \cos^{-1}\frac{12}{13} + \cot^{-1}\frac{56}{33} = \tan^{-1}\frac{3}{4} +
  \tan^{-1}\frac{5}{12} + \tan^{-1}\frac{33}{56}$

  $= \tan^{-1}\frac{\frac{3}{4} + \frac{5}{12}}{1 - \frac{3}{4}.\frac{5}{12}} + \tan^{-1}\frac{33}{56}$

  $= \tan^{-1}\frac{56}{48}.\frac{48}{33} + \tan^{-1}\frac{33}{56} = \tan^{-1}\frac{56}{33} + \tan^{-1}\frac{33}{56}$

  We know that $\tan^{-1}x + \tan^{-1}\frac{1}{x} = \pi/2 \because$ denominator will be zero.

  Hence, $\tan^{-1}\frac{56}{33} + \tan^{-1}\frac{33}{56} = \pi/2$

\item We have to prove that $2\cot^{-1}5 + \cot^{-1}7 + 2\cot^{-1}8 = \frac{\pi}{4}$

  $2\cot^{-1}6 = 2\tan^{-1}\frac{1}{5} = \tan^{-1}\frac{2.\frac{1}{5}}{1 - \frac{1}{25}} = \tan^{-1}\frac{5}{12}$

  $2\cot^{-1}8 = 2\tan^{-1}\frac{1}{8} = \tan^{-1}\frac{2.\frac{1}{8}}{1 - \frac{1}{64}} = \tan^{-1}\frac{16}{63}$

  L.H.S. $= \tan^{-1}\frac{5}{12} + \tan^{-1}\frac{1}{7} + \tan^{-1}\frac{16}{63}$

  $= \tan^{-1}\frac{\frac{5}{12} + \frac{16}{63}}{1 - \frac{5}{12}.\frac{16}{63}} + \tan^{-1}\frac{1}{7} =
  \tan^{-1}\frac{3}{4} + \tan^{-1}\frac{1}{7}$

  $= \tan^{-1}\frac{\frac{3}{4} + \frac{1}{7}}{1 - \frac{3}{4}.\frac{1}{7}} =
  \tan^{-1}1 = \frac{\pi}{4} =$ R.H.S.

\item We have to prove that $\tan^{-1}1 + \tan^{-1}2 + \tan^{-1}3 = 2\left(\tan^{-1}1 + \tan^{-1}\frac{1}{2} +
  \tan^{-1}\frac{1}{3}\right).$

  L.H.S. $= \tan^{-1}1 + \tan^{-1}2 + \tan^{-1}3 = \tan^{-1}\frac{1 + 2}{1 - 2} + \tan^{-1}3$

  $= \tan^{-1}(-3) + \tan^{-1}3 = n\pi$

  $2\tan^{-1}1 = \tan^{-1}\frac{1 + 1}{1 - 1.1} = \tan^{-1}\infty$

  $2\tan^{-1}\frac{1}{2} = \tan^{-1}\frac{2.\frac{1}{2}}{1 - \frac{1}{4}} = \tan^{-1}\frac{4}{3}$

  $2\tan^{-1}\frac{1}{3} = \tan^{-1}\frac{2.\frac{1}{3}}{1 - \frac{1}{9}} = \tan^{-1}\frac{3}{4}$

  Now $\tan^{-1}x + \tan^{-1}\frac{1}{x} = 2n\pi + \frac{\pi}{2}$

  $\therefore$ R.H.S. $= n\pi$

\item Given $A = \tan^{-1}\frac{1}{7}$ and $B = \tan^{-1}\frac{1}{3}$, we have to prove that $\cos 2A = \sin 4B$.

  $\cos A = \cos\tan^{-1}\frac{1}{7} = \cos\cos^{-1}\frac{7}{\sqrt{50}} = \frac{7}{\sqrt{50}}$

  $\cos2A = 2\cos^2A - 1 = 2.\frac{49}{50} - 1 = \frac{48}{50} = \frac{24}{25}$

  $\cos B = \cos\tan^{-1}\frac{1}{3} = \cos\cos^{-1}\frac{3}{\sqrt{10}} \Rightarrow \sin B = \frac{1}{\sqrt{10}}$

  $\sin4B = 4\sin B\cos B(2\cos^2B - 1) = 4.\frac{1}{\sqrt{10}}.\frac{3}{\sqrt{10}}\left(2.\frac{9}{10} - 1\right)$

  $= \frac{12}{10}.\frac{8}{10} = \frac{24}{25}$

  Hence, $\cos2A = \sin4B$.

\item We have to find the sum $\tan^{-1}\frac{x}{1 + 1.2x^2} + \tan^{-1}\frac{x}{1 + 2.3x^2} + \ldots + \tan^{-1}\frac{1}{1 +
  n(n + 1)x^2}, x> 0.$

  $\tan^{-1}\frac{x}{1 + 1.2x^2} = \tan^{-1}2x - \tan^{-1}x$

  $\tan^{-1}\frac{x}{1 + 2.3x^2} = \tan^{-1}3x - \tan^{-1}2x$

  $\ldots$

  $\tan^{-1}\frac{1}{1 + n(n + 1)x^2} = \tan^{-1}(n + 1)x - \tan^{-1}nx$

  Adding, we get

  $\tan^{-1}\frac{x}{1 + 1.2x^2} + \tan^{-1}\frac{x}{1 + 2.3x^2} + \ldots + \tan^{-1}\frac{1}{1 + n(n + 1)x^2} =
  \tan^{-1}(n + 1)x - \tan^{-1}x$

  $= \tan^{-1}\frac{nx}{1 + (n + 1)x^2}$

\item We have to find the sum $\tan^{-1}\frac{d}{1 + a_1a_2} + \tan^{-1}\frac{d}{1 + a_2a_3} + \ldots + \tan^{-1}\frac{d}{1 +
  a_na_{n + 1}},$ where $a_1, a_2, \ldots, a_n, a_{n + 1}$ form an arithmetic progression with common difference
  $d.$

  $\tan^{-1}\frac{d}{1 + a_1a_2} = \tan^{-1}\frac{a_2 - a_1}{1 + a_1a_2} = \tan^{-1}a_2 - \tan^{-1}a_1$

  $\tan^{-1}\frac{d}{1 + a_2a_3} = \tan^{-1}\frac{a_3 - a_1}{1 + a_2a_3} = \tan^{-1}a_3 - \tan^{-1}a_2$

  $\ldots$

  $\tan^{-1}\frac{d}{1 + a_na_{n + 1}} = \tan^{-1}\frac{a_{n + 1} - a_n}{1 + a_na_{n +1}} = \tan^{-1}a_{n + 1} - a_n$

  Adding, we get

  $\tan^{-1}\frac{d}{1 + a_1a_2} + \tan^{-1}\frac{d}{1 + a_2a_3} + \ldots + \tan^{-1}\frac{d}{1 +
    a_na_{n + 1}} = \tan^{-1}a_{n + 1} - \tan^{-1}a_1 = \tan^{-1}\frac{nd}{1 + a_1a_{n + 1}}$

\item We have computed $\sin^{-1}\sin5 = 5-2\pi$, so we can rewrite the inequality as $5 - 2\pi>x^2 - 4x$ or
  $x^2 - 4x + 2\pi - 5 < 0$ which is a quadratic equation having positive coefficient for $x^2$. Thus it will be

  $(x - \alpha)(x - \beta) < 0$ for the above to hold true.

  $\Rightarrow \left[x - \frac{4 - \sqrt{16 - 4(2\pi - 5)}}{2}\right]\left[x - \frac{4 + \sqrt{16 - 4(2\pi - 5)}}{}\right]
  < 0$

  $\Rightarrow x\in (2 - \sqrt{9 - 2\pi}, 2 + \sqrt{9 - 2\pi})$

\item Given, $\tan^{-1}y = 5\tan^{-1}x$, which we can rewrite as $\tan^{-1}y = 2\tan^{-1}x + 3\tan^{-1}x$

  R.H.S. $= \tan^{-1}\frac{2x}{1 - x^2} + \tan^{-1}\frac{3x - x^3}{1 - 3x^2}$

  $= \tan^{-1}\frac{2x(1 - 3x^2) + (1 - x^2)(3x - x^3)}{(1 - x^2)(1 - 3x^2) - 2x(3x - x^3)}$

  $=\tan^{-1}\frac{2x - 6x^3 + 3x - x^3 -3x^3 + x^5}{1 - 4x^2 + 3x^4 - 6x^2 + 2x^4}$

  $\Rightarrow y = \frac{x^5 - 10x^3 + 2x}{5x^4 - 10x^2 + 1}$

  Let $\tan^{-1}x = 18^\circ$ then $\tan^{-1}y = \frac{\pi}{2}\Rightarrow 5x^4 - 10x^2 + 1 = 0$.

\item Let $\cos^{-1}x = \alpha, \cos^{-1}y = \beta, \cos^{-1}z = \gamma$

  $\Rightarrow \cos\alpha = x, \cos\beta = y, \cos\gamma = z$

  Also, given $\alpha + \beta + \gamma = \pi$

  and $x + y + z = \frac{3}{2} \Rightarrow \cos\alpha + \cos\beta + \cos\gamma = \frac{3}{2}$

  Let $z = \cos\alpha + \cos\beta + \cos\gamma$ and angle $\gamma$ be fixed then

  $z = 2\cos\frac{\alpha + \beta}{2}\cos\frac{\alpha - \beta}{2} + \cos\gamma$

  $= 2\sin\frac{\gamma}{2}\cos\frac{\alpha - \beta}{2} + \cos\gamma$

  Since $\gamma$ is fixed, $\cos\gamma$ and $\sin\frac{\gamma}{2}$ are fixed. Only changing term is
  $\cos\frac{\alpha - \beta}{2}$

  Clearly, $z$ will be maximum if $\cos\frac{\alpha - \beta}{2} = 1$ i.e. $\alpha = \beta$

  Similarly, when angle $\beta$ is fixed, $z$ will be maximum if $\gamma = \alpha$

  and when angle $\alpha$ is fixed, $z$ will be maximum if $\beta = \gamma$

  $\Rightarrow z$ will be maximum if $\alpha + \beta + \gamma = 60^\circ$

  $\Rightarrow z_{max} = \cos60^\circ + \cos60^\circ + \cos60^\circ = \frac{3}{2}$

  $\Rightarrow \alpha = \beta = \gamma = 60^\circ \Rightarrow x = y = z$

\item Let $\sin^{-1}x = \alpha, \sin^{-1}y = \beta, \sin^{-1}z = \gamma$

  $\Rightarrow \sin\alpha = x, \sin\beta = y, \sin\gamma = z$

  Also, $\alpha + \beta + \gamma = \pi$

  $\Rightarrow \alpha + \beta = \pi - \gamma$

  $\Rightarrow \cos(\alpha + \beta) = \cos(\pi - \gamma)$

  $\Rightarrow \cos\alpha\cos\beta - \sin\alpha\sin\beta = -\cos\gamma$

  $\Rightarrow \sqrt{1 - x^2}\sqrt{1 - y^2} - xy = -\sqrt{1 - z^2}$

  $\Rightarrow \sqrt{(1 - x^2)(1 - y^2)} = xy - \sqrt{1 - z^2}$

  Squaring, we get

  $(1 - x^2)(1 - y^2) = x^2y^2 + 1 - z^2 -2xy\sqrt{1 - z^2}$

  $\Rightarrow x^2 + y^2 - z^2 = 2xy\sqrt{1 - z^2}$

  Squaring again, we get

  $x^4 + y^4 + z^4 + 4x^2y^2z^2 = 2(x^2y^2 + y^2z^2 + z^2x^2)$

\item Let $\tan^{-1}\frac{\alpha}{\beta} = \theta, \tan^{-1}\frac{\beta}{\alpha} = \phi$

  $\therefore \tan\theta = \frac{\alpha}{\beta}, \tan\phi = \frac{\beta}{\alpha}$

  L.H.S. $= \frac{\alpha^3}{2\sin^2\frac{\theta}{2}} + \frac{\beta^3}{2\cos^2\frac{\phi}{2}}$

  $= \frac{\alpha^3}{1 - \cos\theta} + \frac{\beta^3}{1 + \cos\phi}$

  $= \frac{\alpha^3}{1 - \frac{\beta}{\sqrt{\alpha^2 + \beta^2}}} + \frac{\beta^3}{1 + \frac{\alpha}{\sqrt{\alpha^2 +
        \beta^2}}}$

  $= \sqrt{\alpha^2 + \beta^2}\left[\frac{\alpha^3(\sqrt{\alpha^2 + \beta^2}+ \beta)}{(\alpha^2 + \beta^2) - \beta^2} +
    \frac{\beta^3(\sqrt{\alpha^2 + \beta^2}) - \alpha}{(\alpha^2 + \beta^2) - \alpha^2}\right]$

  $= \sqrt{\alpha^2 + \beta^2}[\alpha(\sqrt{\alpha^2 + \beta^2} + \beta) + \beta(\sqrt{\alpha^2 + \beta^2} - \alpha)]$

  $= (\alpha^2 + \beta^2)(\alpha + \beta)$

\item We have to prove that $2\tan^{-1}\left[\tan\frac{\alpha}{2}\tan\left(\frac{\pi}{4} - \frac{\beta}{2}\right)\right] =
  \tan^{-1}\left[\frac{\sin\alpha\cos\beta}{\sin\beta + \cos\alpha}\right].$

  L.H.S. $= 2\tan^{-1}\left[\tan\frac{\alpha}{2}\tan\left(\frac{\pi}{4} - \frac{\beta}{2}\right)\right]$

  $= 2\tan^{-1}\left[\tan\frac{\alpha}{2}\frac{1 - \tan\frac{\beta}{2}}{1 + \tan\frac{\beta}{2}}\right]$

  $= \tan^{-1}\left[\frac{2\tan\frac{\alpha}{2}\frac{1 - \tan\frac{\beta}{2}}{1 + \tan\frac{\beta}{2}}}{1 -
      \tan^2\frac{\alpha}{2}\frac{\left(1 - \tan\frac{\beta}{2}\right)^2}{\left(1 + \tan\frac{\beta}{2}\right)^2}}\right]$

  Substituting $\tan\frac{\alpha}{2} = \frac{\sin\frac{\alpha}{2}}{\cos\frac{\alpha}{2}}$ and $\tan\frac{\beta}{2} =
  \frac{\sin\frac{\alpha}{2}}{\cos\frac{\beta}{2}}$ and simplifying we arrive at the desired result.

\item R.H.S. $= \tan^{-1}[\tan^2(\alpha + beta)\tan^2(\alpha - \beta)] + \tan^{-1}1$

  $= \tan^{-1}\left[\frac{1 + \tan^2(\alpha + \beta)\tan^{2}(\alpha - beta)}{1 - \tan^2(\alpha + \beta)\tan^{2}(\alpha -
    beta)}\right]$

  $= \tan^{-1}\left[\frac{\cos^2(\alpha + \beta)\cos^2(\alpha - \beta) + \sin^2(\alpha + \beta)\sin^2(\alpha -
    \beta)}{\cos^2(\alpha + \beta)\cos^2(\alpha - \beta) - \sin^2(\alpha + \beta)\sin^2(\alpha - \beta)}\right]$

  $= \tan^{-1}\left[\frac{4\cos^2(\alpha + \beta)\cos^2(\alpha - \beta) + 4\sin^2(\alpha + \beta)\sin^2(\alpha -
    \beta)}{4\cos^2(\alpha + \beta)\cos^2(\alpha - \beta) - 4\sin^2(\alpha + \beta)\sin^2(\alpha - \beta)}\right]$

  $= \tan^{-1}\left[\frac{\{2\cos(\alpha + \beta)\cos(\alpha - \beta)\}^2 + \{2\sin(\alpha + \beta)\sin(\alpha -
    \beta)\}^2}{\{2\cos(\alpha + \beta)\cos(\alpha - \beta)\}^2 - \{2\sin(\alpha + \beta)\sin(\alpha - \beta)\}^2}\right]$

  $=\tan^{-1}\left[\frac{(\cos2\alpha + \cos2\beta)^2 + (\cos2\beta - \cos2\alpha)^2}{\cos2\alpha + \cos2\beta)^2 -
    (\cos2\beta - \cos2\alpha)^2}\right]$

  $= \tan^{-1}\left[\frac{2\cos^2\alpha + 2\cos^2\beta}{4\cos2\alpha\cos\beta}\right]$

  $= \tan^{-1}\left[\frac{1}{2}\cos2\alpha\sec2\beta + \frac{1}{2}\cos\beta\sec\alpha\right] =$ L.H.S.

\item Let $\sqrt{\frac{3 - 4x^2}{x^2}} = t \Rightarrow \sqrt{\frac{3 - 4x^2}{4x^2}} = \frac{t}{2}$

  $\therefore$ R.H.S. $= 2\tan^{-1}\frac{t}{2} - \tan^{-1}t$

  $= \tan{-1}\frac{t}{1 - \frac{t^2}{4}} - \tan^{-1}t = \tan^{-1}\frac{4t}{4 - t^2} - tan^{-1}t$

  $= \tan^{-1}\frac{\frac{4t}{4 - t^2} - t}{1 + \frac{4t^2}{4 - t^2}} = tan^{-1}\frac{t^3}{4 + 3t^2}$

  $\Rightarrow \cot^{-1}\frac{y}{\sqrt{1 - x^2 - y^2}} = \tan^{-1}\frac{t^3}{4 + 3t^2}$

  $\Rightarrow \frac{1 - x^2 - y^2}{y^2} = \frac{t^6}{9t^4 + 24t^2 + 16}$

  $\Rightarrow \frac{1 - x^2}{y^2} - 1 = \frac{t^6}{9t^4 + 24t^2 + 16}$

  $\Rightarrow \frac{1 - x^2}{y^2} = \frac{t^6 + 9t^4 + 24t^2  + 16}{9t^4 + 24t^2 + 16}$

  $\Rightarrow y^2 = \frac{9t^4 + 24t^2 + 16}{t^6 + 9t^4 + 24t^2  + 16}(1 - x^2)$

  We know that $t^2 + 4 = \frac{3}{x^2}$ from our initial equation.

  $\Rightarrow y^2 = \frac{(t^2 + 4)^2 + 8t^4 + 16t^2}{(t^2 + 1)(t^2 + 4)^2}(1 - x^2)$

  Substituting for $t$ and simplifying, we obtain

  $27y^2 = 81x^2 -144x^4 + 64x^6$

\item Given $\frac{m\tan(\alpha - \theta)}{\cos^2\theta} = \frac{n\tan\theta}{\cos^2(\alpha - \theta)}$

  $\Rightarrow \frac{m}{n} = \frac{\sin\theta\cos\theta}{\sin(\alpha - \theta)\cos(\alpha - \theta)} =
  \frac{\sin2\theta}{\sin2(\alpha - \theta)}$

  Doing componendo and dividendo, we get

  $\frac{n - m}{m + n} = \frac{\sin2(\alpha - \theta) - \sin2\theta}{\sin2\theta + \sin2(\alpha - \theta)}$

  $\Rightarrow \frac{n - m}{m + n} = \frac{\cos\alpha\sin(\alpha - 2\theta)}{\sin\alpha\cos(\alpha - 2\theta)}$

  $\Rightarrow \frac{n - m}{m + n} = \frac{\tan(\alpha - 2\theta)}{\tan\alpha}$

  $\Rightarrow \alpha - 2\theta = \tan^{-1}\left(\frac{n - m}{m + n}\right)\tan\alpha$

  $\Rightarrow \theta = \frac{1}{2}\left[\alpha - \tan^{-1}\left(\frac{n - m}{n + m}\right)\tan\alpha\right]$

\item Given, $\sin^{-1}\frac{x}{a} + \sin^{-1}\frac{y}{b} = \sin^{-1}\frac{c^2}{ab}$

  $\Rightarrow \sin^{-1}\frac{x}{a} = \sin^{-1}\frac{c^2}{ab} - \sin^{-1}\frac{y}{b}$

  $\Rightarrow \frac{x}{a}  = \frac{c^2}{ab}\sqrt{1 - \frac{y^2}{b^2}} - \frac{y}{b}\sqrt{1 - \frac{c^4}{a^2b^2}}$

  $\Rightarrow \frac{x}{a} + \frac{y}{b}\sqrt{1 - \frac{c^4}{a^2b^2}} = \frac{c^2}{ab}\sqrt{1 - \frac{y^2}{b^2}}$

  Squaring both sides

  $\Rightarrow \frac{x^2}{a^2} + \frac{y^2}{b^2}\left(1 - \frac{c^4}{a^2b^2}\right) + \frac{2xy}{ab}\sqrt{1 -
    \frac{c^4}{a^2b^2}} = \frac{c^4}{a^2b^2}\left(1 - \frac{y^2}{b^2}\right)$

  $\Rightarrow b^2x^2 + 2xy\sqrt{a^2b^2 - c^4} = c^4 - a^2y^2$

\item We have to prove that $\tan^{-1}t + \tan^{-1}\frac{2t}{1 - t^2} = \tan^{-1}\frac{3t - t^3}{1 - 3t^2}$

  L.H.S. $= \tan^{-1}t + 2\tan^{-1}t = 3\tan^{-1}t = \tan^{-1}\frac{3t - t^3}{1 - 3t^2}$

\item If $a>x>b$ or $a<x<b$ then the fractions under square root are positive and less than one. So the angles are
  defined.

  $\cos^{-1}\sqrt{\frac{a - x}{a - b}} = \sin^{-1}\sqrt{1 - \frac{a - x}{a - b}} = \sin^{-1}\sqrt{\frac{x - b}{a - b}}$

\item Given, $\cos^{-1}\sqrt{p} + \cos^{-1}\sqrt{1 - p} + \cos^{-1}\sqrt{1 - q} = \frac{3\pi}{4}$.

  $\Rightarrow \cos^{-1}\sqrt{p} + \sin^{-1}\sqrt{p} + \cos^{-1}\sqrt{1 - q} = \frac{3\pi}{4}$

  $\Rightarrow \frac{\pi}{2} + \cos^{-1}\sqrt{1 - q} = \frac{3\pi}{4}$

  $\Rightarrow \cos^{-1}\sqrt{1 - q} = \frac{\pi}{4}\Rightarrow 1 - q = \frac{1}{2} \Rightarrow q = \frac{1}{2}$

  For $\cos^{-1}\sqrt{p}$ to be defined $0\leq p\leq 1$ and then $\cos^{-1}\sqrt{1 - p}$ will also be defined.

\item Given, $\tan^{-1}x + \cot^{-1}y = \tan^{-1}3$

  $\Rightarrow \tan^{-1}x + \tan^{-1}\frac{1}{y} = \tan^{-1}3$

  $\Rightarrow \tan^{-1}\frac{x + \frac{1}{y}}{1 - \frac{x}{y}} = \tan^{-1}3$

  $\Rightarrow \frac{xy + 1}{y - x} = 3 \Rightarrow y = \frac{3x + 1}{3 - x}$

  When $x$ is positive, numerator is positive. For denominator to be positive $x = 1, 2$ (considering only integral
  values). Corresponding values of $y = 2, 7$. We can see that both solutions satisfy the original equation.

\item Given $\sin^{-1}\frac{ax}{c} + \sin^{-1}\frac{bx}{c} = \sin^{-1}x$ so we can infer $-1\leq x\leq 1$

  Also given, $a^2 + b^2 = c^2 \Rightarrow \frac{a^2x^2}{c^2} + \frac{b^2x^2}{c^2} = x^2$

  From first equation, $\frac{ax}{c}\sqrt{1 - \frac{b^2x^2}{c^2}} + \frac{bx}{c}\sqrt{1 - \frac{a^2x^2}{c^2}} = x$

  $\Rightarrow x\left[\frac{ax}{c}\sqrt{1 - \frac{b^2x^2}{c^2}} + \frac{bx}{c}\sqrt{1 - \frac{a^2x^2}{c^2}} - 1\right] = 0$

  Either $x = 0$ or $\frac{a}{c}\sqrt{1 - \frac{b^2x^2}{c^2}} + \frac{b}{c}\sqrt{1 - \frac{a^2x^2}{c^2}} = 0$

  $\Rightarrow a\sqrt{c^2 - b^2x^2} + b\sqrt{c^2 - a^2x^2} = c^2$

  $\Rightarrow a\sqrt{c^2 - b^2x^2} = c^2 - b\sqrt{c^2 - a^2x^2}$

  $\Rightarrow a^2c^2 - a^2b^2x^2 = c^4 + b^2c^2 - a^2b^2x^2 - 2bc^2\sqrt{c^2 - a^2x^2}$

  $\Rightarrow a^2c^2 - b^2c^2 - c^4 = 2bc^2\sqrt{c^2 - a^2x^2}$

  $\Rightarrow a^2 - b^2 - c^2 = -2b\sqrt{c^2 - a^2x^2}$

  $\Rightarrow -2b^2 = -2b\sqrt{c^2 - a^2x^2}[\because a^2 + b^2 = c^2]$

  $\Rightarrow b = \sqrt{c^2 - a^2x^2} \Rightarrow a^2x^2 = c^2 - b^2 = a^2 \Rightarrow x = \pm 1$

  Clearly, $x = 0, \pm 1$ satisfy the equation.

\item Let $f(x) = \sin[2\cos^{-1}\{\cot(2\tan^{-1}x)\}]$

  $= \sin\left[2\cos^{-1}\left\{\cot \tan^{-1}\frac{2x}{1 - x^2}\right\}\right]$

  $= \sin\left[2\cos^{-1}\left(\cot\cot^{-1}\frac{1 - x^2}{2x}\right)\right]$

  $= \sin\left[2\cos^{-1}\frac{1 - x^2}{2x}\right]$

  $= \sin\sin^{-1}\left[2.\frac{1 - x^2}{2x}\sqrt{1- \left(\frac{1 - x^2}{2x}\right)^2}\right]$

  $= \frac{1 - x^2}{x}\sqrt{1 - \left(\frac{1 - x^2}{2x}\right)^2}$

  When $f(x) = 0,$ we have $(1 - x^2)\sqrt{1 - \left(\frac{1 - x^2}{2x}\right)^2} = 0$

  $\Rightarrow (1 - x^2)\sqrt{6x^2 - 1 - x^4} = 0$

  Either $1 - x^2 = 0$ or $\sqrt{6x^2 - 1 - x^4} = 0$

  $\Rightarrow x = \pm1$ or $x^4 - 6x^2 + 1 = 0$

  $x^2 = \frac{6\pm \sqrt{36 - 4}}{2} = 3 \pm 2\sqrt{2} = (1 \pm \sqrt{2})^2$

  $x = \pm(1 \pm \sqrt{2})$

  $x = \pm 1, \pm(1 \pm \sqrt{2})$

\item $\frac{3\sin2\theta}{5 + 4\cos2\theta} = \frac{3.\frac{2\tan\theta}{1 + \tan^2\theta}}{5 + 4\frac{1 - \tan^2\theta}{1 +
    \tan^2\theta}}$

  $= \frac{6\tan\theta}{9 + \tan^2\theta} = \frac{2\tan\phi}{1 + \tan^2\phi}$ where $\frac{1}{3}\tan\theta =
  \tan\phi$

  $= \sin2\phi$

  Given equation is $\theta = \tan^{-1}(2\tan^2\theta) - \frac{1}{2}\sin^{-1}\left(\frac{3\sin2\theta}{5 +
    4\cos2\theta}\right)$

  $\Rightarrow \tan^{-1}(2\tan^2\theta) - \frac{1}{2}\sin^{-1}\sin2\phi = \theta$

  $\Rightarrow \tan^{-1}(2\tan^2\theta) - \tan^{-1}\frac{1}{3}\tan\theta = \theta$

  $\Rightarrow \tan^{-1}\left[\frac{2\tan^2\theta - \frac{1}{3}\tan\theta}{1 +
      2\tan^2\theta.\frac{1}{3}\tan\theta}\right] = \theta$

  $\Rightarrow \frac{\tan\theta\left(2\tan\theta - \frac{1}{3}\right)}{1 + \frac{2}{3}\tan^3\theta} = \tan\theta$

  $\Rightarrow \tan\theta\left[\frac{6\tan\theta - 1}{3 + 2\tan^2\theta} - 1\right] = 0$

  If $\tan\theta = 0 \Rightarrow \theta = n\pi$

  If $\frac{6\tan\theta - 1}{3 + 2\tan^3\theta} - 1 = 0$

  $\Rightarrow \tan^3\theta - 3\tan\theta + 2 = 0 \Rightarrow (\tan\theta - 1)^2(\tan\theta + 2) = 0$

  Either $\tan\theta = 1 \Rightarrow \theta = n\pi + \frac{\pi}{4}$

  or $\tan\theta = -2 \Rightarrow \theta = n\pi + \tan^{-1}(-2)$

\item $\tan^{-1}2x + \tan^{-1}3x = \frac{\pi}{4}$

  $\Rightarrow \tan^{-1}3x = \frac{\pi}{4} - \tan^{-1}2x$

  $\Rightarrow 3x = \tan\left(\frac{\pi}{4} - \tan^{-1}2x\right)$

  $3x = \frac{1 - \tan(\tan^{-1}2x)}{1 + \tan(\tan^{-1}2x)} = \frac{1 - 2x}{1 + 2x}$

  $\Rightarrow 6x^2 + 3x = 1 - 2x \Rightarrow 6x^2 + 5x - 1 = 0$

  $\Rightarrow (x + 1)(6x - 1) = 0$

  $\Rightarrow x = -1, \frac{1}{6}$

  When $x = -1$ L.H.S. is negative angle but R.H.S. is positive so it is not a solution. However, for $x =
  \frac{1}{6}$ satisfies both sides are positive and balanced.

\item Given, $\sin^{-1}\left(\frac{x}{1 + \frac{x}{2}}\right) + \cos^{-1}\left(\frac{x^2}{1 + \frac{x^2}{2}}\right) =
  \frac{\pi}{2}$

  $\Rightarrow \sin^{-1}\left(\frac{2x}{x + 2}\right) = \frac{\pi}{2} - \cos^{-1}\left(\frac{2x^2}{2 + x^2}\right) =
  \sin^{-1}\left(\frac{2x^2}{2 + x^2}\right)$

  $\Rightarrow \frac{2x}{2 + x} = \frac{2x^2}{2 + x^2}$

  $\Rightarrow 2x\left[\frac{1}{2 + x} - \frac{x}{2 + x^2}\right] = 0$

  $\Rightarrow x = 0$ or $2 + x^2 = 2x + x^2 \Rightarrow x = 1$

  But $0 < |x| < \sqrt{2} \Rightarrow x = 1$

\item Given, $\tan^{-1}\sqrt{x(x + 1)} + \sin^{-1}\sqrt{x^2 + x + 1} = \frac{\pi}{2}$

  $\Rightarrow \tan^{-1}\sqrt{x(x + 1)} + \tan^{-1}\frac{\sqrt{x^2 + x + 1}}{\sqrt{x(x + 1)}} = \frac{\pi}{2}$

  $\tan^{-1}\left[\frac{\sqrt{x(x + 1)} + \frac{\sqrt{x^2 + x + 1}}{\sqrt{x(x + 1)}}}{1 - \sqrt{x(x + 1)}.\frac{\sqrt{x^2 +
        x + 1}}{\sqrt{x(x + 1)}}}\right] = \frac{\pi}{2}$

  $\Rightarrow 1 - \sqrt{x^2 + x + 1} = 0$

  $\Rightarrow x = 0, -1$

  Clearly both these values of $x$ satisfy the equation.

\item This problem is similar to 115 so $x = -1, 0, 1$.

\item We have to solve $\sin^{-1}(1 - x) - 2\sin^{-1}x = \frac{\pi}{2}$

  Let $x = \sin y$

  $\Rightarrow \sin^{-1}(1 - \sin y) - 2y = \frac{\pi}{2}$

  $\Rightarrow \sin^{-1}(1 - \sin y) =\frac{\pi}{2} + 2y$

  $\Rightarrow 1 - \sin y = \sin\left(\frac{\pi}{2} + 2y\right) = \cos2y = 1 - 2\sin^2y$

  $\Rightarrow 2\sin^2y - \sin y = 0$

  $\Rightarrow 2x^2 - x = 0$

  $x = 0, \frac{1}{2}$

  But $x = \frac{1}{2}$ does not satisfy the equation but $x = 0$ does so it is the required solution.

\item Given equation is $\tan^{-1}x + \tan^{-1}y = \tan^{-1}k$

  $\Rightarrow \tan^{-1}\frac{x + y}{1 - xy} = \tan^{-1}k$

  $\Rightarrow \frac{x + y}{1 - xy} = k$

  For $k > 0, 1 - xy > 0 \Rightarrow xy < 1$ which implies both $x$ and $y$ cannot be positive integers.

\item We have to solve $\tan^{-1}\frac{x + 1}{x - 1} + \tan^{-1}\frac{x - 1}{x} = \tan^{-1}(-7)$

  $\Rightarrow \frac{\frac{x + 1}{x - 1} + \frac{x - 1}{x}}{1 - \frac{x + 1}{x - 1}.\frac{x - 1}{x}} = -7$

  $\Rightarrow \frac{x^2 + x + x^2 - 2x + 1}{x^2 - x - x^2 + 1} = -7$

  $\Rightarrow 2x^2 - x + 1 = 7x - 7$

  $\Rightarrow 2x^2 - 8x + 8 = 0 \Rightarrow x^2 - 4x + 4 = 0 \Rightarrow x = 2$

\item We have to solve $\tan^{-1}\frac{1}{a - 1} = \tan^{-1}\frac{1}{x} + \tan^{-1}\frac{1}{a^2 - x + 1}$

  $\Rightarrow \frac{1}{a - 1} = \frac{a^2 + 1}{a^2x - x^2 + x - 1}$

  $\Rightarrow (a - x)(a^2 - a - x + 1) = 0$

  $x = a, a^2 - a + 1$

\item We have to solve $\cos^{-1}\frac{x^2 - 1}{x^2 + 1} + \tan^{-1}\frac{2x}{x^2 - 1} = \frac{2\pi}{3}$

  **Case I:** When $x > 1$ given equation becomes $\pi - 2\tan^{-1}x + \pi - 2\tan^{-1}x= \frac{2\pi}{3}$

  $\Rightarrow \tan^{-1}x = \frac{\pi}{3} \Rightarrow x = \sqrt{3}$

  **Case II:** When $x < 1$ given equation becomes $\pi - 2\tan^{-1}x - 2\tan^{-1}x = \frac{2\pi}{3}$

  $\Rightarrow x = \tan\frac{\pi}{12} = \frac{\sqrt{3} - 1}{\sqrt{3} + 1} = 2 - \sqrt{3}$

\item Given $\theta = \tan^{-1}\frac{x\sqrt{3}}{2k - x}$ and $\phi = \tan^{-1}\frac{2x - k}{k\sqrt{3}}$.

  $\theta - \phi = \tan^{-1}\frac{\frac{x\sqrt{3}}{2k - x} - \frac{2x - k}{k\sqrt{3}}}{1 + \frac{x}{k}.\frac{2x - k}{2k -
    x}}$

  $= \tan^{-1}\frac{1}{\sqrt{3}}.\frac{2k^2 + 2x^2 - 2kx}{2k^2 + 2x^2 - 2kx}$

  If $2x^2 + 2k^2 - 2kx\neq 0$ then $\theta - \phi = \tan^{-1}\frac{1}{\sqrt{3}} = \frac{\pi}{6}$

\item Given $\tan^{-1}x + \cos^{-1}\frac{y}{\sqrt{1 + y^2}} = \sin^{-1}\frac{3}{\sqrt{10}}$

  $\Rightarrow \tan^{-1}\frac{1}{y} = \tan^{-1}3 - \tan^{-1}x$

  $\Rightarrow \frac{1}{y} = \frac{3 - x}{1 + 3x}$

  $y = \frac{1 + 3x}{3 - x}$

  Clearly, for $x \geq 3$ there can be no solution as $y$ becomes infinity and negative for those values.

  When $x = 1, y = 2$ and when $x = 2, y = 7$.

\item Given equation is an identity except for range. For $\sin^{-1}s\sqrt{1 - x^2}$ range is $\left[-\frac{\pi}{2},
  \frac{\pi}{2}\right]$

  For $2\cos^{-1}x$ range is $[0, 2\pi]$ so common range is $\left[0, \frac{\pi}{2}\right]$

  $\Rightarrow 0 \leq 2\cos^{-1}x \leq \frac{\pi}{2}$

  $\Rightarrow \frac{1}{\sqrt{2}}\leq x \leq 1$, since $\cos$ is a decreasing function, so inequality is reversed.

\item We have to solve $\sin^{-1}\frac{x}{\sqrt{1 + x^2}} - \sin^{-1}\frac{1}{\sqrt{1 + x^2}} = \sin^{-1}\frac{1 + x}{1 +
  x^2}$

  $\Rightarrow \sin^{-1}\frac{x^2 - 1}{1 + x^2} = \sin^{-1}\frac{1 + x}{1 + x^2}$

  $\Rightarrow x^2 - x - 2 = 0 \Rightarrow x = -1, 2$

  However, the equation is not satisfied for $x = -1$ hence $x = 2$ is the required solution.

\item Given equation is $y = 2\tan^{-1}\left[\sqrt{\frac{a - b}{a + b}}\tan\frac{x}{2}\right] - \cos^{-1}\left[\frac{b +
    a\cos x}{a + b\cos x}\right]$

  R.H.S. $= \tan^{-1}\frac{2\sqrt{\frac{a - b}{a + b}}\tan\frac{x}{2}}{1 - \frac{a - b}{a + b}\tan^2\frac{x}{2}} -
  \tan^{-1}\frac{\sqrt{a^2 - b^2}\sin x}{b + a\cos x}$

  $= \tan^{-1}\frac{2\sqrt{a^2 - b^2}\tan\frac{x}{2}}{(a + b) - (a - b)\tan^2\frac{x}{2}} - \tan^{-1}\frac{2\sqrt{a^2 -
      b^2}\tan\frac{x}{2}}{(a + b) - (a - b)\tan^2\frac{x}{2}}$

  $= 0$ which is a constant provided $a \geq b > 0$


\item $tan^{-1}\frac{2i}{2 + i^2 + i^4} = \\tan^{-1}(i^2 + i + 1) - \tan^{-1}(i^2 - i + 1)$

  Subtituting $i = 1,$ R.H.S. $= \tan^{-1}3 - \tan^{-1}1$

  Subtituting $i = 2,$ R.H.S. $= \tan^{-1}7 - \tan^{-1}3$

  Subtituting $i = 3,$ R.H.S. $= \tan^{-1}13 - \tan^{-1}7$

  $\ldots$

  Subtituting $i = n - 1,$ R.H.S. $= \tan^{-1}(n^2 - n + 1) - \tan^{-1}(n^2 - 3n + 3)$

  Subtituting $i = n,$ R.H.S. $= \tan^{-1}(n^2 + n + 1) - \tan^{-1}(n^2 - n + 1)$

  Adding, we get $\sum_{i=1}^{n}\tan^{-1}\frac{2i}{2 + i^2 + i^4} = \tan^{-1}(n^2 + n + 1) - \tan^{-1}1$

  $= \tan^{-1}\frac{n^2 + n}{n^2 + n + 2}$

\item $t_n = \cot^{-1}\left(n^2 + \frac{3}{4}\right) = \cot^{-1}\left(\frac{4n^2 + 3}{4}\right)$

  $= \tan^{-1}\frac{1}{1 + n^4 - \frac{1}{4}} = \frac{\left(n + \frac{1}{2}\right) - \left(n - \frac{1}{2}\right)}{1 +
  \left(n + \frac{1}{2}\right)\left(n - \frac{1}{2}\right)}$

  $= \tan^{-1}\left(n + \frac{1}{2}\right) - \tan^{-1}\left(n - \frac{1}{2}\right)$

  Putting $n = 1, t_1 = \tan^{-1}\frac{3}{2} - \tan^{-1}\frac{1}{2}$

  Putting $n = 2, t_2 = \tan^{-1}\frac{5}{2} - \tan^{-1}\frac{3}{2}$

  $\ldots$

  Putting $n = \infty, t_\infty = \tan^{-1}\left(\infty + \frac{1}{2}\right) - \tan^{-1}\left(\infty - \frac{1}{2}\right)$

  Adding, we get $S = \tan^{-1}\left(\infty + \frac{1}{2}\right) - \tan^{-1}\frac{1}{2}$

  $= \frac{\pi}{2} - \tan^{-1}\frac{1}{2} = \cot^{-1}\frac{1}{2} = \tan^{-1}2$

\item We know that $\tan^{-1}x + \cot^{-1}x = \frac{\pi}{2}$

  So given equation can be written as $(\tan^{-1}x + \cot^{-1}x) - 2\tan^{-1}x\left(\frac{\pi}{2} - \tan^{-1}x\right) =
  \frac{5\pi^2}{8}$

  $\Rightarrow 2(\tan^{-1}x) - \pi\tan^{-1}x - \frac{3\pi^2}{8} = 0$

  $\Rightarrow \tan^{-1}x = -\frac{\pi}{4}, \frac{3\pi}{4} \Rightarrow x = -1$

\item $(\sin^{-1}x + \cos^{-1}x)^3 = (\sin^{-1}x + \cos^{-1}x)[(\sin^{-1}x + \cos^{-1}x)^2 - 3\sin^{-1}x\cos^{-1}x]$

  $= \frac{\pi}{2}\left[\frac{\pi^2}{4} - 3\sin^{-1}x\left(\frac{\pi}{2} - \sin^{-1}x\right)\right]$

  $= \frac{3\pi}{2}\left[\left(\sin^{-1}x - \frac{\pi}{4}\right)^2 + \frac{\pi^2}{48}\right]$

  Least value of $\sin^{-1} - \frac{\pi}{4} = 0$ and greatest value is $\frac{3\pi}{4}$

  Hence greatest and least values of the required expression are $\frac{7\pi^3}{8}$ and $\frac{\pi^2}{32}$.

\item Let $\cos^{-1}x = a \in [0, \pi]$ and $\sin^{-1}y = b \in\left[-\frac{\pi}{2}, \frac{\pi}{2}\right]$

  We have $a + b^2 = \frac{p\pi^2}{4}$ and $ab^2 = \frac{\pi^4}{16}$

  Since $b^2\in [0, \pi^2/4]$, we get $a + b^2 \in [0, \pi + \pi^2/4]$

  $\Rightarrow 0 \leq \frac{p\pi^2}{4}\leq \pi + \frac{\pi^2}{4}$

  $\Rightarrow 0\leq p\leq \frac{4}{\pi} + 1$

  So $p = 0, 1, 2$

  $\Rightarrow a\left(\frac{p\pi^2}{4} - a\right) = \frac{\pi^2}{16}$

  Since $a\in R \Rightarrow D\geq 0$

  $p^2 \geq 4\Rightarrow p = 2$

\item Given $\tan^{-1}x, \tan^{-1}y, \tan^{-1}z$ are in A.P.

  $\Rightarrow 2\tan^{-1}y = \tan^{-1}x + \tan^{-1}z$

  $\Rightarrow \frac{2y}{1 - y^2} = \frac{x + z}{1 - xz}$

  If $x, y, z$ are in A.P. $2y = x + z$

  $\Rightarrow 1 - \frac{(x + z)^2}{4} = 1 - xz$

  $\Rightarrow (x - z)^2 = 0 \Rightarrow x = y = z$

\item $t_n = \tan^{-1}\frac{x}{1 + n(n + 1)x^2} = \tan^{-1}(n + 1)x - \tan^{-1}nx$

  $t_1 = \tan^{-1}2x - \tan^{-1}x$

  $t_2 = \tan^{-1}3x - \tan^{-1}2x$

  $\ldots$

  $t_n = \tan^{-1}(n + 1)x - \tan^{-1}nx$

  Adding, we get $S = \tan^{-1}(n + 1)x - \tan^{-1}x = \tan^{-1}\frac{nx}{1 + (n + 1)x^2} =$ R.H.S.
\stopitemize
