% -*- mode: context; -*-
\chapter{Miscellaneous Problems}
\startitemize[n, 1*broad]
  %1
\item If $\pi < x < 2\pi$, prove that $\frac{\sqrt{1 + \cos x} + \sqrt{1 - \cos x}}{\sqrt{1 + \cos x} -
    \sqrt{1 - \cos x}} = \cot\left(\frac{x}{2} + \frac{\pi}{4}\right)$.
  %2
\item If $\sin(A - B) = \frac{1}{\sqrt{10}}, \cos(A + B) = \frac{2}{\sqrt{29}}$, find the value of $tan2A$,
  where $A$ and $B$ lie between $0$ and $\frac{\pi}{4}$.
  %3
\item Suppose that $\sin^3x\sin3x = \displaystyle\sum_{m = 0}^nc_m\cos mx$ is an identity in $x$, where
  $c_0, c_1, c_2, \ldots, c_n$ are constants and $c_n \neq 0$. Find the value of $n$.
  %4
\item Show that $\sin^4\frac{\pi}{8} + \sin^4\frac{3\pi}{8} + \sin^4\frac{\5pi}{8} + \sin^4\frac{7\pi}{8}
  = \frac{3}{2}$.
  %5
\item Evaluate $\displaystyle\sum_{p = 1}^{32}(3p + 2)\left[\displaystyle\sum_{q = 1
  }^{10}\sin\frac{2q\pi}{11} - i\cos\frac{2q\pi}{11}\right]^p$.
  %6
\item If $\theta = \frac{\pi}{2^n - 1}$, prove that $2^n\cos\theta\cos2\theta\cos2^2\theta\ldots\cos2^{n -
  1}\theta = -1$
  %7
\item Show that $\cos\frac{2\pi}{7}\cos\frac{4\pi}{7}\cos\frac{6\pi}{7} = \frac{1}{8}$.
  %8
\item Prove that $\sin\frac{2\pi}{7} + \sin\frac{4\pi}{7} + \sin\frac{8\pi}{7} = \frac{\sqrt{7}}{2}$.
  %9
\item Show that $\tan^2\frac{\pi}{16} + \tan^2\frac{2\pi}{16} + \cdots + \tan^2\frac{7\pi}{16} = 35$.
  %10
\item Prove that $\tan\frac{\pi}{7}\tan\frac{2\pi}{7}\tan\frac{3\pi}{7} = \sqrt{7}$.
  %11
\item Show that $\left(\tan^2\frac{\pi}{7} + \tan^2\frac{2\pi}{7} + \tan^2\frac{3\pi}{7}\right)
  + \left(\cot^2\frac{\pi}{7} + \cot^2\frac{2\pi}{7} + \cot^2\frac{3\pi}{7}\right) = 105$.
  %12
\item If $0 < x < \frac{\pi}{2}$, prove that $\sqrt{\tan x + \sin x} + \sqrt{\tan x - \sin x} = 2\sqrt{\tan
  x}\cos\left(\frac{\pi}{4} - \frac{x}{2}\right)$.
  %13
\item If $\sin^3x\sin3x = \displaystyle\sum_{n = 0}^6\cos^nx$, where $c_0, c_1, \ldots, c_6$ are constants,
  then find the value of $c_4$.
  %14
\item Find the value of $\sin7^\circ + \sin77^\circ + \sin149^\circ + \cdots + \sin293^\circ$.
  %15
\item Prove that $\cos\frac{\pi}{n} + \cos\frac{2\pi}{n} + \cos\frac{3\pi}{n} + \cdots + \cos\frac{(n -
  1)\pi}{n} = 0$.
  %16
\item Show that $\cos\frac{2\pi}{7} + \cos\frac{4\pi}{7} + \cos\frac{6\pi}{7} = -\frac{1}{2}$.
  %17
\item Show that $3\left[\sin^4\left(\frac{3\pi}{2} - \alpha\right) + \sin^4(3\pi + \alpha)\right] -
  2\left[\sin^6\left(\frac{\pi}{2} + \alpha\right) + \sin^6(5\pi - \alpha)\right] = 1$.
  %18
\item Prove that $\sin36^\circ\sin72^\circ\sin108^\circ\sin144^\circ = \frac{5}{16}$.
  %19
\item Prove that $\sin^212^\circ + \sin^221^\circ + \sin^239^\circ + \sin^248^\circ = 1 + \sin^29^\circ
  + \sin^218^\circ$.
  %20
\item If $270^\circ\leq \alpha\leq 450^\circ$, express $\cos\frac{\alpha}{2}$ and $\sin\frac{\alpha}{2}$ in
  terms of $\sin\alpha$.
  %21
\item Prove that $\tan142^\circ30' = 2 + \sqrt{2} - \sqrt{3} - \sqrt{6}$.
  %22
\item Find $\tan\frac{\alpha}{2}$ if $\sin\alpha + \cos\alpha = \frac{\sqrt{7}}{2}$ and the angle $\alpha$
  lies between $0^\circ$ and $45^\circ$.
  %23
\item Prove that if $\frac{\sin(x - \alpha)}{\sin(x - \beta)} = \frac{a}{b}$ and $\frac{\cos(x
  - \alpha)}{\cos(x - \beta)} = \frac{A}{B}$ and $aB + bA\neq 0$ then $\cos(\alpha - \beta) = \frac{aA +
  bB}{aB + bA}$.
  %24
\item If the sum of three positive numbers $\alpha, \beta, \gamma$ is equal to $\frac{\pi}{2}$. Find the
  product $\cot\alpha\cot\gamma$ if $\cot\alpha, \cot\beta, \cot\gamma$ are in A.P.
  %25
\item Show that $16\cos\frac{2\pi}{15}\cos\frac{4\pi}{15}\cos\frac{8\pi}{15}\cos\frac{16\pi}{15} = 1$.
  %26
\item If $\cos(\alpha + \beta) = \frac{4}{5}, \sin(\alpha - \beta) = \frac{5}{13}$ and $\alpha$ and $\beta$
  lie between $0$ and $\frac{\pi}{4}$, find $\tan2\alpha$.
  %27
\item Sum the series $\tan\alpha\tan(\alpha + \beta) + \tan(\alpha + \beta)\tan(\alpha + 2\beta)
  + \tan(\alpha + 2\beta)\tan(\alpha + 3\beta) + \cdots$ to $n$ terms.
  %28
\item Show that $\tan\alpha + 2\tan2\alpha + 4\tan4\alpha + 8\tan8\alpha + \cdots$ to $n$ terms
  $= \cot\alpha - 2^n\cot2^n\alpha$.
  %29
\item Show that $\frac{\sin x}{\cos3x} + \frac{\sin3x}{\cos9x} + \frac{\sin9x}{\cos27x}
  = \frac{1}{2}[\tan27x - \tan x]$.
  %30
\item Show that $\cot16^\circ.\cot44^\circ + \cot44^\circ.\cot76^\circ - \cot76^\circ.\cot16^\circ = 3$.
  %31
\item If $\theta = \frac{2\pi}{7}$, prove that $\tan\theta\tan2\theta + \tan2\theta\tan4\theta
  + \tan4\theta\tan\theta = -7$.
  %32
\item If $\frac{\sin^4\alpha}{a} + \frac{\cos^4\alpha}{b} = \frac{1}{a + b}$, prove that
  $\frac{\sin^8\alpha}{a^3} + \frac{\cos^8\alpha}{b^3} = \frac{1}{(a + b)^3}$.
  %33
\item If $A + B + C = \pi$, express $S = \sin3A + \sin3B + \sin3C$ as a product of three trigonometric
  ratios. If $S = 0$, show that at least one of the angles is $60^\circ$.
  %34
\item Prove that $\sin x\sin y\sin(x - y) + \sin y\sin z\sin(y - z) + \sin z\sin x\sin(z - x) + \sin(x -
  y)\sin(y - z)\sin(z - x) = 0$.
  %35
\item Prove that $\cot\theta\cot2\theta + \cot2\theta\cot3\theta + 2 = \cot\theta(\cot\theta
  - \cot3\theta)$.
  %36
\item Prove that $\frac{1 + \sin A}{\cos A} + \frac{\cos B}{1 - \sin B} = \frac{2(\sin A - \sin B)}{\sin(A -
  B) + \cos A - \cos B}$.
  %37
\item If $\frac{\cos^4x}{\cos^2y} + \frac{\sin^4x}{\sin^2y} = 1$, prove that $\frac{\cos^4y}{\cos^2x}
  + \frac{\sin^4y}{\sin^2x} = 1$.
  %38
\item If $\theta + \phi + \psi = 2\pi$, prove that $\cos^2\theta + \cos^2\phi + \cos^2\psi -
  2\cos\theta\cos\phi\cos\psi = 1$.
  %39
\item If $A + B + C = \pi$, and if $\cos3A + \cos3B + \cos3C = 1$, then show that one angle must be
  $120^\circ$.
  %40
\item If $A + B + C = \pi$, show that $\sum \sin^3A\sin(B - C) = 0$.
  %41
\item If $u_n = \sin^n\theta + \cos^n\theta$ for all positive integers $n$, prove that $\frac{u_3 -
  u_5}{u_1} = \frac{u_5 - u_7}{u_3}$.
  %42
\item Given that $4\cos(x - y)\cos(y - z)\cos(z - x) = 1$, prove that $1 + 12\cos2(x - y)\cos2(y - z)\cos2(z
  - x) = 4\cos3(x - y)\cos3(y - z)\cos3(z - x)$.
  %43
\item If $x + y + z = xyz$, prove that $\displaystyle\sum \frac{3x - x^3}{1 - 3x^2} = \frac{3\prod x\prod (3 -
  x^2)}{\prod \left(1 - 3x^2\right)}$.
  %44
\item Show that $(2 + \sqrt{3})\sin\theta + 2\cos\theta$ lies between $-(2 + \sqrt{5})$ and $2 + \sqrt{5}$.
  %45
\item Show that $5\cos\theta + 3\cos\left(\theta + \frac{\pi}{3}\right) + 3$ lies between $-4$ and $10$.
  %46
\item Find the maximum and minimum values of the expression $\sin^2\theta + \cos^4\theta$.
  %47
\item Show that the minimum value of $\sin^8x + \cos^8x$ is $\frac{1}{8}$.
  %48
\item Show that $\sin^{2n}x + \cos^{2n}x \leq 1$.
  %49
\item If $0 < \theta < \pi$, prove that $\cot\frac{\theta}{2}\geq 1 + \cot\theta$.
  %50
\item If $\tan\theta = n\tan\phi(n > 0)$, prove that $\tan^2(\theta - \phi)\leq \frac{(n - 1)^2}{4n}$.
  %51
\item Prove that the inequality $|\sin nx|\leq n|\sin x|$ is valid for all positive integeres.
  %52
\item Show that $2^{\sin x} + 2^{\cos x} \geq 2^{1 - \frac{1}{\sqrt{2}}}$ for all real values of $x$.
  %53
\item In a $\triangle ABC$ if angle $C$ is obtuse, prove that $\tan A\tan B < 1$.
  %54
\item If $\frac{1}{\cos\alpha\cos\beta} + \tan\alpha\tan\beta = \tan\gamma$, show that $\cos2\gamma \leq 0$.
  %55
\item If $0 \leq \alpha < \frac{\pi}{2}$ then show that $\tan\alpha + \cot\alpha > \sin\alpha + \cos\alpha$.
  %56
\item If $\alpha, \beta, \gamma > 0$ and $\alpha + \beta + \gamma = \frac{\pi}{2}$ show that $\tan^2\alpha
  + \tan^2\beta + \tan^2\gamma\geq 1$.
  %57
\item Prove that $0\leq \cos^2\alpha + \cos^2(\alpha + \beta) - 2\cos\alpha\cos\beta\cos(\alpha + \beta)\le
  1$.
  %58
\item Prove that $3\left(\tan^2\theta + \cot^2\theta\right) - 8(\tan\theta + \cot\theta) + 10 > 0$.
  %59
\item Prove that inequality $\tan nx > n\tan x$ is valid for $n\geq 2$ if $0 < x < \frac{\pi}{4(n - 1)}$,
  where $n$ is a natural number $\neq 1$.
  %60
\item If $0 < \alpha < \frac{\pi}{2}, 0 < \beta < \frac{\pi}{2}$ and $0 < \gamma < \frac{\pi}{2}$, prove
  that $\sin(\alpha + \beta + \gamma) < \sin\alpha + \sin\beta + \sin\gamma$.
  %61
\item For any positive integer $n$, show by induction that $\frac{2\cos2^n\theta + 1}{2\cos\theta + 1} =
  (2\cos\theta - 1)(2\cos2\theta - 1)\left(2\cos^2\theta - 1\right) \ldots (2\cos2^n\theta - 1)$.
  %62
\item If $\pi < \theta < \frac{3\pi}{2}$, then prove that $\sqrt{4\sin^4\theta + \sin^22\theta} +
  4\cos^2\left(\frac{\pi}{4} - \frac{\theta}{2}\right) = 2$.
  %63
\item If $0 < \theta < \pi, 0 < \phi < \pi$ and $\cos\phi + \cos\theta - \cos(\theta + \phi) = \frac{3}{2}$,
  prove that $\theta = \phi = \frac{\pi}{3}$.
  %64
\item Show that $\sin\frac{\pi}{14}$ is a root of the equation $8x^3 - 4x^2 - 4x + 1 = 0$.
  %65
\item The product of the sines of the angles of a triangle is $p$ and the product of the cosines is $q$,
  show that the tangents if the angles are the roots of the equation $qx^3 - px^2 + (1 + q)x - p = 0$.
  %66
\item Prove that the function $f(x) = \cos^2x + \cos^2\left(\frac{\pi}{3} + x\right) - \cos
  x.\cos\left(\frac{\pi}{3} + x\right)$ is a constant function. Find the value of the constant.
  %67
\item If $0 < x < \frac{\pi}{2}$, show that $\cos(\sin x) > \sin(\cos x)$.
  %68
\item If $\tan\alpha = \frac{p}{1}$, where $\alpha = 6\beta$, $\alpha$ being actute angle, prove that
  $\frac{1}{2}[p\csc2\beta - q\sec2\beta] = \sqrt{p^2 + q^2}$.
  %69
\item If $\cos^2\theta = \frac{m^2 - 1}{3}$ and $\tan^3\frac{\theta}{2} = \tan\alpha$, prove that
  $\cos^{\frac{2}{3}} + \sin^{\frac{2}{3}}\alpha = \left(\frac{2}{m}\right)^{\frac{2}{3}}$.
  %70
\item If $\cos\theta + \cos\phi + \cos\psi = 0$ and $\sin\theta + \sin\phi + \sin\psi = 0$, prove that
  $\cos3\theta + \cos3\phi + \cos3\psi - 3\cos(\theta + \phi + \psi) = 0$.
  %71
\item If $A, B, C$ be the angles of a triangle and the system of linear equations \startformula\startalign
  \NC x\sin A + y\sin B + z\sin C\NC = 0\NR
  \NC x\sin B + y\sin C + z\sin A\NC = 0\NR
  \NC x\sin C + y\sin A + z\sin B\NC = 0\NR
\stopalign\stopformula
has a non-trivial solution, prove that $\sin^2A + \sin^2B + \sin^2C = \cos A + \cos B + \cos C + \cos A\cos
B + \cos B\cos C + \cos C\cos A$.
%72
\item If $\frac{2\sin\alpha}{1 + \cos\alpha + \sin\alpha} = x$, then find $\frac{1 - \cos\alpha
  + \sin\alpha}{1 + \sin\alpha}$ in terms of $x$.
  %73
\item If $k = \sin\frac{\pi}{18}.\sin\frac{5\pi}{18}.\sin\frac{7\pi}{18}$, then find the numerical value of
  $k$.
  %74
\item If $A > , B > 0$ and $A + B = \frac{\pi}{3}$, then find the maximum value of $\tan A\tan B$.
  %75
\item If $a\sin\theta + b\cos\theta = a\csc\theta + b\sec\theta$, show that each expression is equal to
  $\left(a^{\frac{2}{3}} - b^{\frac{2}{3}}\right)\left(a^{\frac{2}{3}} +
  b^{\frac{2}{3}}\right)^{\frac{1}{2}}$.
  %76
\item If $0 < \theta < \pi, 0 < \theta < \phi$ and $\cos\theta\cos\phi\cos(\theta + \phi) = -\frac{1}{8}$,
  prove that $\theta = \phi = \frac{\pi}{3}$.
  %77
\item If $\tan\alpha$ and $\tan\beta$ be the roots of the equation $x^2 + px + q = 0$, find the value of the
  expression $\sin^2(\alpha + \beta) + p\sin(\alpha + \beta)\cos(\alpha + \beta) + q\cos^2(\alpha + \beta)$.
  %78
\item If $\cos(\theta - \alpha) = a, \sin(\theta - \beta) = b$, prove that $a^2 - 2ab\sin(\alpha - \beta) +
  b^2 = \cos^2(\alpha - \beta)$.
  %79
\item If $\tan x\tan y = a$ and $x + y = 2b$, show that $\tan x$ and $\tan y$ are the roots of the equation
  $z^2 - (1 - a)\tan2b.z + a = 0$.
  %80
\item Prove that $4\sin27^\circ = (5 + \sqrt{5})^{1/2} - (3 - \sqrt{5})^{1/2}$.
  %81
\item If $\sin(y + z - x), \sin(z + x - y), \sin(x + y - z)$ are in A.P., prove that $\tan x, \tan y, \tan
  z$ are also in A.P.
  %82
\item If $\alpha + \beta + \gamma = \pi$ and $\tan\frac{1}{4}(\beta + \gamma - \alpha)\tan\frac{1}{4}(\gamma
  + \alpha - \beta)\tan\frac{1}{4}(\alpha + \beta - \gamma) = 1$, prove that $1 + \cos\alpha + \cos\beta
  + \cos\gamma = 0$.
  %83
\item Prove that $\sum\sin(\alpha + \beta)\sin(\alpha - \beta)\sin(\gamma + \delta)\sin(\gamma - delta) =
  0$.
  %84
\item Prove that $x^2 - x\cos(A + B) + 1$ is a factor of $2x^4 + 4x^3\sin A\sin B - x^2(\cos2A + \cos2B) +
  4x\cos A\cos B - 2$.
  %85
\item If $m\sin(\alpha + \beta) = \cos(\alpha - \beta)$, prove that $\frac{1}{1 - m\sin2\alpha} + \frac{1}{1
  - m\sin2\beta} = \frac{2}{1 - m^2}$.
  %86
\item If $\tan\left(\frac{\pi}{4} + \frac{y}{2}\right) = \tan^3\left(\frac{\pi}{4} + \frac{x}{2}\right)$,
  prove that $\sin y = \sin  x.\frac{3 + \sin^2x}{1 + 3\sin^2x}$.
  %87
\item If $x = X\cos\theta - Y\sin\theta$ and $y = X\sin\theta + Y\cos\theta$, then find the smallest
  possible value of $\theta$ for which $x^2 + 4xy + y^2 = AX^2 + BY^2$, $A$ and $B$ being constants.
  %88
\item If $a, b, c$ and $k$ are constant quantities and $\alpha, \beta, \gamma$ are variables subject to the
  relation $a\tan\alpha + b\tan\beta + c\tan\gamma = k$, then find the minimum value of $\tan^2\alpha
  + \tan^2\beta + \tan^2\gamma$.
  %89
\item If $A, B, C$ and $D$ are the angles of a quadrilateral and
  $\sin\frac{A}{2}\sin\frac{B}{2}\sin\frac{C}{2}\sin\frac{D}{2} = \frac{1}{4}$, prove that $A = B = C = D$.
  %90
\item Prove that $\tan\frac{3\pi}{11} + 4\tan\frac{2\pi}{11} = \sqrt{11}$.
  %91
\item If $\alpha$ and $\beta$ ve two distinct solutions of the equation $a\cos x + b\sin x = c$, prove that
  $\cos^2\frac{\alpha - \beta}{2} = \frac{c^2}{a^2 + b^2}$.
  %92
\item If $\alpha, \beta$ be two distinct values of $\theta$ satisfying the equation $a\tan\theta +
  b\sec\theta = 1$. Find $a$ and $b$ in terms of $\alpha$ and $\beta$, and prove that $\sin\alpha
  + \cos\alpha + \sin\beta + \cos\beta = \frac{2b(1 - a)}{1 + a^2}$.
  %93
\item Solve the equation $\sin2x + \cos2x + \sin x + \cos x + 1 = 0$.
  %94
\item If $A + B + C = \pi$ and $A > 0, B > 0, C > 0$, prove that
  $\cos\frac{A}{2}\cos\frac{B}{2}\cos\frac{C}{2}\leq \frac{3\sqrt{3}}{8}$.
  %95
\item If $\cos\theta + \cos\phi = a$ and $\sin\theta + \sin\phi = b$, find $\cos(\theta + \phi)$ and
  $\sin(\theta + \phi)$.
  %96
\item In a $\triangle ABC$, prove that $\sin3A\sin^3(B - C) + \sin3B\sin^3(C - A) + \sin3C\sin^3(A - B) =
  0$.
  %97
\item Find the general solution of the equation $2(\sin x - \cos2x) - \sin2x(1 + 2\sin x) + 2\cos x = 0$.
  %98
\item If $m^2 + {m'}^2 + 2mm'\cos\theta = 1, n^2 + {n'}^2 + 2nn'\cos\theta = 1$ and $mn + m'n' + \left(mn'
  + m'n\right)\cos\theta = 0$, prove that $m^2 + n^2 = \csc^2\theta$.
  %99
\item If $A, B, C$ are the angles of a triangle, prove that $(\sin A + \sin B)(\sin B + \sin C)(\sin C
  + \sin A) > \sin A\sin B\sin C$.
  %100
\item In any $\triangle ABC$, if $\cos\theta = \frac{a}{b + c}, \cos\phi = \frac{b}{a + c}, \cos\psi
  = \frac{c}{a + b}$, where $\theta, \phi, \psi$ lie between $0$ and $\pi$, prove that
  $\tan\frac{\theta}{2}\tan\frac{\phi}{2}\tan\frac{\psi}{2}
  = \tan\frac{A}{2}\tan\frac{B}{2}\tan\frac{C}{2}$.
  %101
\item In any $\triangle ABC$, show that $\left[\cot\frac{A}{2}
  + \cot\frac{B}{2}\right]\left[a^2\sin^2\frac{B}{2} + b\sin^2\frac{A}{2}\right] = c\cot\frac{C}{2}$.
  %102
\item Solve the equations: $\sqrt{3}\sin2A = \sin2B$ and $\sqrt{3}\sin^2A + \sin^2B = \frac{1}{2}(\sqrt{3} -
  1)$.
  %103
\item If $0 < x < \frac{\pi}{2}$, prove that $\sqrt{\tan x + \sin x} + \sqrt{\tan x - \sin x} = 2\sqrt{\tan
  x}\cos\left(\frac{\pi}{4} - \frac{x}{2}\right)$.
  %104
\item In a $\triangle ABC$, show that $s\sec\frac{A}{2}\sec\frac{B}{2}\sec\frac{C}{2} =
  2\sqrt[3]{\frac{abc}{\sin A\sin B\sin C}}$.
  %105
\item If $p\cot^2\theta + q\cot^2\phi = 1, p\cos^2\theta + q\cos^2\phi = 1$ and $p\sin\theta = q\sin\phi$,
  show that $\left(p^2 - q^2\right)^2 = -pq$.
  %106
\item Eliminate $x$ and $y$ from the equation $a\sin^2x + b\cos^2x = c, b\sin^2y + a\cos^2y = d$, and $a\tan
  x = b\tan y$.
  %107
\item Eliminate $\theta$ from the equations $\tan\theta - \cot\theta = a, \cos\theta + \sin\theta = b$.
  %108
\item If $\frac{\sin(\theta + A)}{\sin(\theta + B)} = \sqrt{\frac{\sin2A}{\sin2B}}$, prove that
  $\tan^2\theta = \tan A\tan B$.
  %109
\item If $\frac{x}{\tan(\theta + \alpha)} = \frac{y}{\tan(\theta + \beta)} = \frac{z}{\tan(\theta
  + \gamma)}$, show that $\frac{x + y}{x - y}\sin^2(\alpha - \beta) + \frac{y + z}{y - z}\sin^2(\beta
  - \gamma) + \frac{z + x}{z - x}\sin^2(\gamma - \alpha) = 0$.
  %110
\item Prove that for all real values of $\theta$, the expression $a\sin^2\theta + b\sin\theta\cos\theta +
  c\cos^2\theta$ lies between $\frac{1}{2}(a + c) - \frac{1}{2}\sqrt{b^2 + (a - c)^2}$ and $\frac{1}{2}(a +
  c) + \frac{1}{2}\sqrt{b^2 + (a - c)^2}$.
  %111
\item Solve the equation $\sin^4x + \cos^4x = 1$.
  %112
\item Solve the equation $2\sin^2x + \sin 2x = 2$.
  %113
\item Solve the equation $\sin^8x + \cos^8x = \frac{17}{16}\cos^22x$.
  %114
\item Prove that $\left(1 + \cos\frac{\pi}{10}\right)\left(1 + \cos\frac{3\pi}{10}\right)\left(1
  + \cos\frac{7\pi}{10}\right)\left(1 + \cos\frac{9\pi}{10}\right) = \frac{1}{16}$.
  %115
\item If $p, q, r$ are the perpendiculars from the vertices of a triangle upon the straight line meeting the
  sides externally in $D, E, F$, prove that $a^2(p - q)(p - r) + b^2(q - r)(q - p) + c^2(r - p)(r - q) =
  4\Delta^2$.
  %116
\item If $\theta_1, \theta_2, \theta_3, \theta_4$ be roots of the equation $\sin(\theta + \alpha) =
  k\sin2\theta$, no two of which differ by a multiple of $2\pi$, prove that $\theta_1 + \theta_2 + \theta_3
  + \theta_4 = (2n + 1)\pi$.
  %117
\item Show that the equation $\sec\theta + \csc\theta = c$ has two roots between $0$ and $2\pi$ if $c^2 < 8$
  and four roots if $c^2 > 8$.
  %118
\item If $2\cos n\theta$ be denoted by $u_n$, show that $u_{n + 1} = u_1u_n - u_{n - 1}$. Hence show that
  $2\cos7\theta = u_1^7 - 7u_1^5 + 14u_1^3 - 7u_1$.
  %119
\item Show that the line joining the incenter to the circumcenter of a $\triangle ABC$ is inclined to $BC$
  at an angle $\tan^{-1}\left(\frac{\cos B + \cos C - 1}{\sin B - \sin C}\right)$.
  %120
\item If $x$ be real, prove that $\frac{x^2 - 2\cos\alpha + 1}{x^2 - 2\cos\beta + 1}$ lies between
  $\frac{\sin^2\frac{\alpha}{2}}{\sin^2\frac{\beta}{2}}$ and
  $\frac{\cos^2\frac{\alpha}{2}}{\cos^2\frac{\beta}{2}}$.
  %121
\item If the equation $a_1 + a_2\sin x + a_3\cos x + a_4\sin2x + a_5\cos2x = 0$ holds for all values of $x$,
  where all the constants $a, a_2, \ldots, a_5$ are independent of $x$, the prove that each of the constants
  must be zero.
  %122
\item A ring is $10$ cm in diameter, is suspended from a point $12$ cm above its center by $6$ equal strings
  attached to its circumference at equal intervals. Find the cosine of the angles between consecutive
  strings.
  %123
\item The tangents at $B$ and $C$ to the circumcircle of a triangle $ABC$ meet at $A'$ and $O$ is the
  circumcenter. If $\angle OAA'$ is $\theta$, prove that $2\tan\theta = \cot B - \cot C$ or $\cot C - \cot
  B$.
  %124
\item The area of a regular polygon of $n$ sides inscribed in a circle is to that of the same number of
  sides circumscribing the same circle is $3:4$. Find the value of $n$.
  %125
\item Of two regular polygons of $n$ sides one is circumscribed and the other inscribed in a given
  circle. Prove that the perimeter of the circumscribing polygon, the circle and the inscribed polygon are
  in the ratio $\sec\frac{\pi}{n} : \frac{\pi}{n}\csc\frac{\pi}{n}: 1$.
  %126
\item If $\cos3x = -\frac{3\sqrt{6}}{8}$m show that the three values of $\cos x$ are
  $\frac{1}{2}\sqrt{6}\sin\frac{\pi}{10}, \frac{1}{2}\sqrt{6}\sin\frac{\pi}{6}$ and
  $-\frac{1}{2}\sqrt{6}\sin\frac{3\pi}{10}$.
  %127
\item Find the complete solution of the equations $\tan3\theta + \tan2\phi = 2$ and $\tan\theta + \tan\phi =
  4$.
  %128
\item Show that in general, the equation $A\sin^3x + B\cos^3x + C = 0$ has six distinct roots, no two of
  which differ by $2\pi$ and that the tangents of their semi-sum is $-\frac{A}{B}$.
  %129
\item Find the number of real roots of the equation $x^2\tan x = 1$ between $0$ and $2\pi$.
  %130
\item Find all values of $x$ which satisfy the equation $\tan(x + \beta)\tan(x + \gamma) + \tan(x
  + \gamma)\tan(x + \alpha) + \tan(x + \alpha)\tan(x + \beta) = 1$.
  %131
\item $A, B, C$ are three points on a horizontal plane in the same straight line. $AB$ being $100$ meters
  and $BC$ being $150$ meters. The angle of elevation of a balloon obsewrved simultaneously from $A, B, C$
  are $\alpha, \beta, \gamma$ respectively. Show that the height, $h$, of the balloon in meters is given by
  $h^2\left(3\cot^2\alpha + 2\cot^2\gamma - 5\cot^2\beta\right) = 75000$.
  %132
\item If one angle of a triangle be $60^\circ$, the area $10\sqrt{3}$ sq. cm.\ and the perimeter $20$
  cm. Find the lengths of the sides.
  %133
\item In a triangle the least angle is $45^\circ$ and the tangents of the angles are in A.P. If its area is
  $27$ sq.\ cm., prove that the lengths of the sides are $3\sqrt{5}, 6\sqrt{2}$ and $9$ cm., and that the
  tangents of other angles are respectively $2$ and $3$.
  %134
\item Find the value of $\cos^3\frac{\pi}{8}\cos\frac{3\pi}{8} + \sin^3\frac{\pi}{8}\sin\frac{3\pi}{8}$.
  %135
\item Find the value of $\sin10^\circ\sin30^\circ\sin50^\circ\sin70^\circ$.
\stopitemize
