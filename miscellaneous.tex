% -*- mode: context; -*-
\chapter{Miscellaneous Problems}
\startitemize[n, 1*broad]
  %1
\item If $\pi < x < 2\pi$, prove that $\frac{\sqrt{1 + \cos x} + \sqrt{1 - \cos x}}{\sqrt{1 + \cos x} -
    \sqrt{1 - \cos x}} = \cot\left(\frac{x}{2} + \frac{\pi}{4}\right)$.
  %2
\item If $\sin(A - B) = \frac{1}{\sqrt{10}}, \cos(A + B) = \frac{2}{\sqrt{29}}$, find the value of $tan2A$,
  where $A$ and $B$ lie between $0$ and $\frac{\pi}{4}$.
  %3
\item Suppose that $\sin^3x\sin3x = \displaystyle\sum_{m = 0}^nc_m\cos mx$ is an identity in $x$, where
  $c_0, c_1, c_2, \ldots, c_n$ are constants and $c_n \neq 0$. Find the value of $n$.
  %4
\item Show that $\sin^4\frac{\pi}{8} + \sin^4\frac{3\pi}{8} + \sin^4\frac{\5pi}{8} + \sin^4\frac{7\pi}{8}
  = \frac{3}{2}$.
  %5
\item Evaluate $\displaystyle\sum_{p = 1}^{32}(3p + 2)\left[\displaystyle\sum_{q = 1
  }^{10}\sin\frac{2q\pi}{11} - i\cos\frac{2q\pi}{11}\right]^p$.
  %6
\item If $\theta = \frac{\pi}{2^n - 1}$, prove that $2^n\cos\theta\cos2\theta\cos2^2\theta\ldots\cos2^{n -
  1}\theta = -1$
  %7
\item Show that $\cos\frac{2\pi}{7}\cos\frac{4\pi}{7}\cos\frac{6\pi}{7} = \frac{1}{8}$.
  %8
\item Prove that $\sin\frac{2\pi}{7} + \sin\frac{4\pi}{7} + \sin\frac{8\pi}{7} = \frac{\sqrt{7}}{2}$.
  %9
\item Show that $\tan^2\frac{\pi}{16} + \tan^2\frac{2\pi}{16} + \cdots + \tan^2\frac{7\pi}{16} = 35$.
  %10
\item Prove that $\tan\frac{\pi}{7}\tan\frac{2\pi}{7}\tan\frac{3\pi}{7} = \sqrt{7}$.
  %11
\item Show that $\left(\tan^2\frac{\pi}{7} + \tan^2\frac{2\pi}{7} + \tan^2\frac{3\pi}{7}\right)
  + \left(\cot^2\frac{\pi}{7} + \cot^2\frac{2\pi}{7} + \cot^2\frac{3\pi}{7}\right) = 105$.
  %12
\item If $0 < x < \frac{\pi}{2}$, prove that $\sqrt{\tan x + \sin x} + \sqrt{\tan x - \sin x} = 2\sqrt{\tan
  x}\cos\left(\frac{\pi}{4} - \frac{x}{2}\right)$.
  %13
\item If $\sin^3x\sin3x = \displaystyle\sum_{n = 0}^6\cos^nx$, where $c_0, c_1, \ldots, c_6$ are constants,
  then find the value of $c_4$.
  %14
\item Find the value of $\sin7^\circ + \sin77^\circ + \sin149^\circ + \cdots + \sin293^\circ$.
  %15
\item Prove that $\cos\frac{\pi}{n} + \cos\frac{2\pi}{n} + \cos\frac{3\pi}{n} + \cdots + \cos\frac{(n -
  1)\pi}{n} = 0$.
  %16
\item Show that $\cos\frac{2\pi}{7} + \cos\frac{4\pi}{7} + \cos\frac{6\pi}{7} = -\frac{1}{2}$.
  %17
\item Show that $3\left[\sin^4\left(\frac{3\pi}{2} - \alpha\right) + \sin^4(3\pi + \alpha)\right] -
  2\left[\sin^6\left(\frac{\pi}{2} + \alpha\right) + \sin^6(5\pi - \alpha)\right] = 1$.
  %18
\item Prove that $\sin36^\circ\sin72^\circ\sin108^\circ\sin144^\circ = \frac{5}{16}$.
  %19
\item Prove that $\sin^212^\circ + \sin^221^\circ + \sin^239^\circ + \sin^248^\circ = 1 + \sin^29^\circ
  + \sin^218^\circ$.
  %20
\item If $270^\circ\leq \alpha\leq 450^\circ$, express $\cos\frac{\alpha}{2}$ and $\sin\frac{\alpha}{2}$ in
  terms of $\sin\alpha$.
  %21
\item Prove that $\tan142^\circ30' = 2 + \sqrt{2} - \sqrt{3} - \sqrt{6}$.
  %22
\item Find $\tan\frac{\alpha}{2}$ if $\sin\alpha + \cos\alpha = \frac{\sqrt{7}}{2}$ and the angle $\alpha$
  lies between $0^\circ$ and $45^\circ$.
  %23
\item Prove that if $\frac{\sin(x - \alpha)}{\sin(x - \beta)} = \frac{a}{b}$ and $\frac{\cos(x
  - \alpha)}{\cos(x - \beta)} = \frac{A}{B}$ and $aB + bA\neq 0$ then $\cos(\alpha - \beta) = \frac{aA +
  bB}{aB + bA}$.
  %24
\item If the sum of three positive numbers $\alpha, \beta, \gamma$ is equal to $\frac{\pi}{2}$. Find the
  product $\cot\alpha\cot\gamma$ if $\cot\alpha, \cot\beta, \cot\gamma$ are in A.P.
  %25
\item Show that $16\cos\frac{2\pi}{15}\cos\frac{4\pi}{15}\cos\frac{8\pi}{15}\cos\frac{16\pi}{15} = 1$.
  %26
\item If $\cos(\alpha + \beta) = \frac{4}{5}, \sin(\alpha - \beta) = \frac{5}{13}$ and $\alpha$ and $\beta$
  lie between $0$ and $\frac{\pi}{4}$, find $\tan2\alpha$.
  %27
\item Sum the series $\tan\alpha\tan(\alpha + \beta) + \tan(\alpha + \beta)\tan(\alpha + 2\beta)
  + \tan(\alpha + 2\beta)\tan(\alpha + 3\beta) + \cdots$ to $n$ terms.
  %28
\item Show that $\tan\alpha + 2\tan2\alpha + 4\tan4\alpha + 8\tan8\alpha + \cdots$ to $n$ terms
  $= \cot\alpha - 2^n\cot2^n\alpha$.
  %29
\item Show that $\frac{\sin x}{\cos3x} + \frac{\sin3x}{\cos9x} + \frac{\sin9x}{\cos27x}
  = \frac{1}{2}[\tan27x - \tan x]$.
  %30
\item Show that $\cot16^\circ.\cot44^\circ + \cot44^\circ.\cot76^\circ - \cot76^\circ.\cot16^\circ = 3$.
  %31
\item If $\theta = \frac{2\pi}{7}$, prove that $\tan\theta\tan2\theta + \tan2\theta\tan4\theta
  + \tan4\theta\tan\theta = -7$.
  %32
\item If $\frac{\sin^4\alpha}{a} + \frac{\cos^4\alpha}{b} = \frac{1}{a + b}$, prove that
  $\frac{\sin^8\alpha}{a^3} + \frac{\cos^8\alpha}{b^3} = \frac{1}{(a + b)^3}$.
  %33
\item If $A + B + C = \pi$, express $S = \sin3A + \sin3B + \sin3C$ as a product of three trigonometric
  ratios. If $S = 0$, show that at least one of the angles is $60^\circ$.
  %34
\item Prove that $\sin x\sin y\sin(x - y) + \sin y\sin z\sin(y - z) + \sin z\sin x\sin(z - x) + \sin(x -
  y)\sin(y - z)\sin(z - x) = 0$.
  %35
\item Prove that $\cot\theta\cot2\theta + \cot2\theta\cot3\theta + 2 = \cot\theta(\cot\theta
  - \cot3\theta)$.
  %36
\item Prove that $\frac{1 + \sin A}{\cos A} + \frac{\cos B}{1 - \sin B} = \frac{2(\sin A - \sin B)}{\sin(A -
  B) + \cos A - \cos B}$.
  %37
\item If $\frac{\cos^4x}{\cos^2y} + \frac{\sin^4x}{\sin^2y} = 1$, prove that $\frac{\cos^4y}{\cos^2x}
  + \frac{\sin^4y}{\sin^2x} = 1$.
  %38
\item If $\theta + \phi + \psi = 2\pi$, prove that $\cos^2\theta + \cos^2\phi + \cos^2\psi -
  2\cos\theta\cos\phi\cos\psi = 1$.
  %39
\item If $A + B + C = \pi$, and if $\cos3A + \cos3B + \cos3C = 1$, then show that one angle must be
  $120^\circ$.
  %40
\item If $A + B + C = \pi$, show that $\sum \sin^3A\sin(B - C) = 0$.
  %41
\item If $u_n = \sin^n\theta + \cos^n\theta$ for all positive integers $n$, prove that $\frac{u_3 -
  u_5}{u_1} = \frac{u_5 - u_7}{u_3}$.
  %42
\item Given that $4\cos(x - y)\cos(y - z)\cos(z - x) = 1$, prove that $1 + 12\cos2(x - y)\cos2(y - z)\cos2(z
  - x) = 4\cos3(x - y)\cos3(y - z)\cos3(z - x)$.
  %43
\item If $x + y + z = xyz$, prove that $\displaystyle\sum \frac{3x - x^3}{1 - 3x^2} = \frac{3\prod x\prod (3 -
  x^2)}{\prod \left(1 - 3x^2\right)}$.
  %44
\item Show that $(2 + \sqrt{3})\sin\theta + 2\cos\theta$ lies between $-(2 + \sqrt{5})$ and $2 + \sqrt{5}$.
  %45
\item Show that $5\cos\theta + 3\cos\left(\theta + \frac{\pi}{3}\right) + 3$ lies between $-4$ and $10$.
  %46
\item Find the maximum and minimum values of the expression $\sin^2\theta + \cos^4\theta$.
  %47
\item Show that the minimum value of $\sin^8x + \cos^8x$ is $\frac{1}{8}$.
  %48
\item Show that $\sin^{2n}x + \cos^{2n}x \leq 1$.
  %49
\item If $0 < \theta < \pi$, prove that $\cot\frac{\theta}{2}\geq 1 + \cot\theta$.
  %50
\item If $\tan\theta = n\tan\phi(n > 0)$, prove that $\tan^2(\theta - \phi)\leq \frac{(n - 1)^2}{4n}$.
  %51
\item Prove that the inequality $|\sin nx|\leq n|\sin x|$ is valid for all positive integeres.
  %52
\item Show that $2^{\sin x} + 2^{\cos x} \geq 2^{1 - \frac{1}{\sqrt{2}}}$ for all real values of $x$.
  %53
\item In a $\triangle ABC$ if angle $C$ is obtuse, prove that $\tan A\tan B < 1$.
  %54
\item If $\frac{1}{\cos\alpha\cos\beta} + \tan\alpha\tan\beta = \tan\gamma$, show that $\cos2\gamma \leq 0$.
  %55
\item If $0 \leq \alpha < \frac{\pi}{2}$ then show that $\tan\alpha + \cot\alpha > \sin\alpha + \cos\alpha$.
  %56
\item If $\alpha, \beta, \gamma > 0$ and $\alpha + \beta + \gamma = \frac{\pi}{2}$ show that $\tan^2\alpha
  + \tan^2\beta + \tan^2\gamma\geq 1$.
  %57
\item Prove that $0\leq \cos^2\alpha + \cos^2(\alpha + \beta) - 2\cos\alpha\cos\beta\cos(\alpha + \beta)\le
  1$.
  %58
\item Prove that $3\left(\tan^2\theta + \cot^2\theta\right) - 8(\tan\theta + \cot\theta) + 10 > 0$.
  %59
\item Prove that inequality $\tan nx > n\tan x$ is valid for $n\geq 2$ if $0 < x < \frac{\pi}{4(n - 1)}$,
  where $n$ is a natural number $\neq 1$.
  %60
\item If $0 < \alpha < \frac{\pi}{2}, 0 < \beta < \frac{\pi}{2}$ and $0 < \gamma < \frac{\pi}{2}$, prove
  that $\sin(\alpha + \beta + \gamma) < \sin\alpha + \sin\beta + \sin\gamma$.
  %61
\item For any positive integer $n$, show by induction that $\frac{2\cos2^n\theta + 1}{2\cos\theta + 1} =
  (2\cos\theta - 1)(2\cos2\theta - 1)\left(2\cos^2\theta - 1\right) \ldots (2\cos2^n\theta - 1)$.
  %62
\item If $\pi < \theta < \frac{3\pi}{2}$, then prove that $\sqrt{4\sin^4\theta + \sin^22\theta} +
  4\cos^2\left(\frac{\pi}{4} - \frac{\theta}{2}\right) = 2$.
  %63
\item If $0 < \theta < \pi, 0 < \phi < \pi$ and $\cos\phi + \cos\theta - \cos(\theta + \phi) = \frac{3}{2}$,
  prove that $\theta = \phi = \frac{\pi}{3}$.
  %64
\item Show that $\sin\frac{\pi}{14}$ is a root of the equation $8x^3 - 4x^2 - 4x + 1 = 0$.
  %65
\item The product of the sines of the angles of a triangle is $p$ and the product of the cosines is $q$,
  show that the tangents if the angles are the roots of the equation $qx^3 - px^2 + (1 + q)x - p = 0$.
  %66
\item Prove that the function $f(x) = \cos^2x + \cos^2\left(\frac{\pi}{3} + x\right) - \cos
  x.\cos\left(\frac{\pi}{3} + x\right)$ is a constant function. Find the value of the constant.
  %67
\item If $0 < x < \frac{\pi}{2}$, show that $\cos(\sin x) > \sin(\cos x)$.
  %68
\item If $\tan\alpha = \frac{p}{1}$, where $\alpha = 6\beta$, $\alpha$ being actute angle, prove that
  $\frac{1}{2}[p\csc2\beta - q\sec2\beta] = \sqrt{p^2 + q^2}$.
  %69
\item If $\cos^2\theta = \frac{m^2 - 1}{3}$ and $\tan^3\frac{\theta}{2} = \tan\alpha$, prove that
  $\cos^{\frac{2}{3}} + \sin^{\frac{2}{3}}\alpha = \left(\frac{2}{m}\right)^{\frac{2}{3}}$.
  %70
\item If $\cos\theta + \cos\phi + \cos\psi = 0$ and $\sin\theta + \sin\phi + \sin\psi = 0$, prove that
  $\cos3\theta + \cos3\phi + \cos3\psi - 3\cos(\theta + \phi + \psi) = 0$.
  %71
\item If $A, B, C$ be the angles of a triangle and the system of linear equations \startformula\startalign
  \NC x\sin A + y\sin B + z\sin C\NC = 0\NR
  \NC x\sin B + y\sin C + z\sin A\NC = 0\NR
  \NC x\sin C + y\sin A + z\sin B\NC = 0\NR
\stopalign\stopformula
has a non-trivial solution, prove that $\sin^2A + \sin^2B + \sin^2C = \cos A + \cos B + \cos C + \cos A\cos
B + \cos B\cos C + \cos C\cos A$.
%72
\item If $\frac{2\sin\alpha}{1 + \cos\alpha + \sin\alpha} = x$, then find $\frac{1 - \cos\alpha
  + \sin\alpha}{1 + \sin\alpha}$ in terms of $x$.
  %73
\item If $k = \sin\frac{\pi}{18}.\sin\frac{5\pi}{18}.\sin\frac{7\pi}{18}$, then find the numerical value of
  $k$.
  %74
\item If $A > , B > 0$ and $A + B = \frac{\pi}{3}$, then find the maximum value of $\tan A\tan B$.
  %75
\item If $a\sin\theta + b\cos\theta = a\csc\theta + b\sec\theta$, show that each expression is equal to
  $\left(a^{\frac{2}{3}} - b^{\frac{2}{3}}\right)\left(a^{\frac{2}{3}} +
  b^{\frac{2}{3}}\right)^{\frac{1}{2}}$.
  %76
\item If $0 < \theta < \pi, 0 < \theta < \phi$ and $\cos\theta\cos\phi\cos(\theta + \phi) = -\frac{1}{8}$,
  prove that $\theta = \phi = \frac{\pi}{3}$.
  %77
\item If $\tan\alpha$ and $\tan\beta$ be the roots of the equation $x^2 + px + q = 0$, find the value of the
  expression $\sin^2(\alpha + \beta) + p\sin(\alpha + \beta)\cos(\alpha + \beta) + q\cos^2(\alpha + \beta)$.
  %78
\item If $\cos(\theta - \alpha) = a, \sin(\theta - \beta) = b$, prove that $a^2 - 2ab\sin(\alpha - \beta) +
  b^2 = \cos^2(\alpha - \beta)$.
  %79
\item If $\tan x\tan y = a$ and $x + y = 2b$, show that $\tan x$ and $\tan y$ are the roots of the equation
  $z^2 - (1 - a)\tan2b.z + a = 0$.
  %80
\item Prove that $4\sin27^\circ = (5 + \sqrt{5})^{1/2} - (3 - \sqrt{5})^{1/2}$.
  %81
\item If $\sin(y + z - x), \sin(z + x - y), \sin(x + y - z)$ are in A.P., prove that $\tan x, \tan y, \tan
  z$ are also in A.P.
  %82
\item If $\alpha + \beta + \gamma = \pi$ and $\tan\frac{1}{4}(\beta + \gamma - \alpha)\tan\frac{1}{4}(\gamma
  + \alpha - \beta)\tan\frac{1}{4}(\alpha + \beta - \gamma) = 1$, prove that $1 + \cos\alpha + \cos\beta
  + \cos\gamma = 0$.
  %83
\item Prove that $\sum\sin(\alpha + \beta)\sin(\alpha - \beta)\sin(\gamma + \delta)\sin(\gamma - delta) =
  0$.
  %84
\item Prove that $x^2 - x\cos(A + B) + 1$ is a factor of $2x^4 + 4x^3\sin A\sin B - x^2(\cos2A + \cos2B) +
  4x\cos A\cos B - 2$.
  %85
\item If $m\sin(\alpha + \beta) = \cos(\alpha - \beta)$, prove that $\frac{1}{1 - m\sin2\alpha} + \frac{1}{1
  - m\sin2\beta} = \frac{2}{1 - m^2}$.
  %86
\item If $\tan\left(\frac{\pi}{4} + \frac{y}{2}\right) = \tan^3\left(\frac{\pi}{4} + \frac{x}{2}\right)$,
  prove that $\sin y = \sin  x.\frac{3 + \sin^2x}{1 + 3\sin^2x}$.
  %87
\item If $x = X\cos\theta - Y\sin\theta$ and $y = X\sin\theta + Y\cos\theta$, then find the smallest
  possible value of $\theta$ for which $x^2 + 4xy + y^2 = AX^2 + BY^2$, $A$ and $B$ being constants.
  %88
\item If $a, b, c$ and $k$ are constant quantities and $\alpha, \beta, \gamma$ are variables subject to the
  relation $a\tan\alpha + b\tan\beta + c\tan\gamma = k$, then find the minimum value of $\tan^2\alpha
  + \tan^2\beta + \tan^2\gamma$.
  %89
\item If $A, B, C$ and $D$ are the angles of a quadrilateral and
  $\sin\frac{A}{2}\sin\frac{B}{2}\sin\frac{C}{2}\sin\frac{D}{2} = \frac{1}{4}$, prove that $A = B = C = D$.
  %90
\item Prove that $\tan\frac{3\pi}{11} + 4\tan\frac{2\pi}{11} = \sqrt{11}$.
  %91
\item If $\alpha$ and $\beta$ ve two distinct solutions of the equation $a\cos x + b\sin x = c$, prove that
  $\cos^2\frac{\alpha - \beta}{2} = \frac{c^2}{a^2 + b^2}$.
  %92
\item If $\alpha, \beta$ be two distinct values of $\theta$ satisfying the equation $a\tan\theta +
  b\sec\theta = 1$. Find $a$ and $b$ in terms of $\alpha$ and $\beta$, and prove that $\sin\alpha
  + \cos\alpha + \sin\beta + \cos\beta = \frac{2b(1 - a)}{1 + a^2}$.
  %93
\item Solve the equation $\sin2x + \cos2x + \sin x + \cos x + 1 = 0$.
  %94
\item If $A + B + C = \pi$ and $A > 0, B > 0, C > 0$, prove that
  $\cos\frac{A}{2}\cos\frac{B}{2}\cos\frac{C}{2}\leq \frac{3\sqrt{3}}{8}$.
  %95
\item If $\cos\theta + \cos\phi = a$ and $\sin\theta + \sin\phi = b$, find $\cos(\theta + \phi)$ and
  $\sin(\theta + \phi)$.
  %96
\item In a $\triangle ABC$, prove that $\sin3A\sin^3(B - C) + \sin3B\sin^3(C - A) + \sin3C\sin^3(A - B) =
  0$.
  %97
\item Find the general solution of the equation $2(\sin x - \cos2x) - \sin2x(1 + 2\sin x) + 2\cos x = 0$.
  %98
\item If $m^2 + {m'}^2 + 2mm'\cos\theta = 1, n^2 + {n'}^2 + 2nn'\cos\theta = 1$ and $mn + m'n' + \left(mn'
  + m'n\right)\cos\theta = 0$, prove that $m^2 + n^2 = \csc^2\theta$.
  %99
\item If $A, B, C$ are the angles of a triangle, prove that $(\sin A + \sin B)(\sin B + \sin C)(\sin C
  + \sin A) > \sin A\sin B\sin C$.
  %100
\item In any $\triangle ABC$, if $\cos\theta = \frac{a}{b + c}, \cos\phi = \frac{b}{a + c}, \cos\psi
  = \frac{c}{a + b}$, where $\theta, \phi, \psi$ lie between $0$ and $\pi$, prove that
  $\tan\frac{\theta}{2}\tan\frac{\phi}{2}\tan\frac{\psi}{2}
  = \tan\frac{A}{2}\tan\frac{B}{2}\tan\frac{C}{2}$.
  %101
\item In any $\triangle ABC$, show that $\left[\cot\frac{A}{2}
  + \cot\frac{B}{2}\right]\left[a^2\sin^2\frac{B}{2} + b\sin^2\frac{A}{2}\right] = c\cot\frac{C}{2}$.
  %102
\item Solve the equations: $\sqrt{3}\sin2A = \sin2B$ and $\sqrt{3}\sin^2A + \sin^2B = \frac{1}{2}(\sqrt{3} -
  1)$.
  %103
\item If $0 < x < \frac{\pi}{2}$, prove that $\sqrt{\tan x + \sin x} + \sqrt{\tan x - \sin x} = 2\sqrt{\tan
  x}\cos\left(\frac{\pi}{4} - \frac{x}{2}\right)$.
  %104
\item In a $\triangle ABC$, show that $s\sec\frac{A}{2}\sec\frac{B}{2}\sec\frac{C}{2} =
  2\sqrt[3]{\frac{abc}{\sin A\sin B\sin C}}$.
  %105
\item If $p\cot^2\theta + q\cot^2\phi = 1, p\cos^2\theta + q\cos^2\phi = 1$ and $p\sin\theta = q\sin\phi$,
  show that $\left(p^2 - q^2\right)^2 = -pq$.
  %106
\item Eliminate $x$ and $y$ from the equation $a\sin^2x + b\cos^2x = c, b\sin^2y + a\cos^2y = d$, and $a\tan
  x = b\tan y$.
  %107
\item Eliminate $\theta$ from the equations $\tan\theta - \cot\theta = a, \cos\theta + \sin\theta = b$.
  %108
\item If $\frac{\sin(\theta + A)}{\sin(\theta + B)} = \sqrt{\frac{\sin2A}{\sin2B}}$, prove that
  $\tan^2\theta = \tan A\tan B$.
  %109
\item If $\frac{x}{\tan(\theta + \alpha)} = \frac{y}{\tan(\theta + \beta)} = \frac{z}{\tan(\theta
  + \gamma)}$, show that $\frac{x + y}{x - y}\sin^2(\alpha - \beta) + \frac{y + z}{y - z}\sin^2(\beta
  - \gamma) + \frac{z + x}{z - x}\sin^2(\gamma - \alpha) = 0$.
  %110
\item Prove that for all real values of $\theta$, the expression $a\sin^2\theta + b\sin\theta\cos\theta +
  c\cos^2\theta$ lies between $\frac{1}{2}(a + c) - \frac{1}{2}\sqrt{b^2 + (a - c)^2}$ and $\frac{1}{2}(a +
  c) + \frac{1}{2}\sqrt{b^2 + (a - c)^2}$.
  %111
\item Solve the equation $\sin^4x + \cos^4x = 1$.
  %112
\item Solve the equation $2\sin^2x + \sin 2x = 2$.
  %113
\item Solve the equation $\sin^8x + \cos^8x = \frac{17}{16}\cos^22x$.
  %114
\item Prove that $\left(1 + \cos\frac{\pi}{10}\right)\left(1 + \cos\frac{3\pi}{10}\right)\left(1
  + \cos\frac{7\pi}{10}\right)\left(1 + \cos\frac{9\pi}{10}\right) = \frac{1}{16}$.
  %115
\item If $p, q, r$ are the perpendiculars from the vertices of a triangle upon the straight line meeting the
  sides externally in $D, E, F$, prove that $a^2(p - q)(p - r) + b^2(q - r)(q - p) + c^2(r - p)(r - q) =
  4\Delta^2$.
  %116
\item If $\theta_1, \theta_2, \theta_3, \theta_4$ be roots of the equation $\sin(\theta + \alpha) =
  k\sin2\theta$, no two of which differ by a multiple of $2\pi$, prove that $\theta_1 + \theta_2 + \theta_3
  + \theta_4 = (2n + 1)\pi$.
  %117
\item Show that the equation $\sec\theta + \csc\theta = c$ has two roots between $0$ and $2\pi$ if $c^2 < 8$
  and four roots if $c^2 > 8$.
  %118
\item If $2\cos n\theta$ be denoted by $u_n$, show that $u_{n + 1} = u_1u_n - u_{n - 1}$. Hence show that
  $2\cos7\theta = u_1^7 - 7u_1^5 + 14u_1^3 - 7u_1$.
  %119
\item Show that the line joining the incenter to the circumcenter of a $\triangle ABC$ is inclined to $BC$
  at an angle $\tan^{-1}\left(\frac{\cos B + \cos C - 1}{\sin B - \sin C}\right)$.
  %120
\item If $x$ be real, prove that $\frac{x^2 - 2\cos\alpha + 1}{x^2 - 2\cos\beta + 1}$ lies between
  $\frac{\sin^2\frac{\alpha}{2}}{\sin^2\frac{\beta}{2}}$ and
  $\frac{\cos^2\frac{\alpha}{2}}{\cos^2\frac{\beta}{2}}$.
  %121
\item If the equation $a_1 + a_2\sin x + a_3\cos x + a_4\sin2x + a_5\cos2x = 0$ holds for all values of $x$,
  where all the constants $a, a_2, \ldots, a_5$ are independent of $x$, the prove that each of the constants
  must be zero.
  %122
\item A ring is $10$ cm in diameter, is suspended from a point $12$ cm above its center by $6$ equal strings
  attached to its circumference at equal intervals. Find the cosine of the angles between consecutive
  strings.
  %123
\item The tangents at $B$ and $C$ to the circumcircle of a triangle $ABC$ meet at $A'$ and $O$ is the
  circumcenter. If $\angle OAA'$ is $\theta$, prove that $2\tan\theta = \cot B - \cot C$ or $\cot C - \cot
  B$.
  %124
\item The area of a regular polygon of $n$ sides inscribed in a circle is to that of the same number of
  sides circumscribing the same circle is $3:4$. Find the value of $n$.
  %125
\item Of two regular polygons of $n$ sides one is circumscribed and the other inscribed in a given
  circle. Prove that the perimeter of the circumscribing polygon, the circle and the inscribed polygon are
  in the ratio $\sec\frac{\pi}{n} : \frac{\pi}{n}\csc\frac{\pi}{n}: 1$.
  %126
\item If $\cos3x = -\frac{3\sqrt{6}}{8}$m show that the three values of $\cos x$ are
  $\frac{1}{2}\sqrt{6}\sin\frac{\pi}{10}, \frac{1}{2}\sqrt{6}\sin\frac{\pi}{6}$ and
  $-\frac{1}{2}\sqrt{6}\sin\frac{3\pi}{10}$.
  %127
\item Find the complete solution of the equations $\tan3\theta + \tan2\phi = 2$ and $\tan\theta + \tan\phi =
  4$.
  %128
\item Show that in general, the equation $A\sin^3x + B\cos^3x + C = 0$ has six distinct roots, no two of
  which differ by $2\pi$ and that the tangents of their semi-sum is $-\frac{A}{B}$.
  %129
\item Find the number of real roots of the equation $x^2\tan x = 1$ between $0$ and $2\pi$.
  %130
\item Find all values of $x$ which satisfy the equation $\tan(x + \beta)\tan(x + \gamma) + \tan(x
  + \gamma)\tan(x + \alpha) + \tan(x + \alpha)\tan(x + \beta) = 1$.
  %131
\item $A, B, C$ are three points on a horizontal plane in the same straight line. $AB$ being $100$ meters
  and $BC$ being $150$ meters. The angle of elevation of a balloon obsewrved simultaneously from $A, B, C$
  are $\alpha, \beta, \gamma$ respectively. Show that the height, $h$, of the balloon in meters is given by
  $h^2\left(3\cot^2\alpha + 2\cot^2\gamma - 5\cot^2\beta\right) = 75000$.
  %132
\item If one angle of a triangle be $60^\circ$, the area $10\sqrt{3}$ sq. cm.\ and the perimeter $20$
  cm. Find the lengths of the sides.
  %133
\item In a triangle the least angle is $45^\circ$ and the tangents of the angles are in A.P. If its area is
  $27$ sq.\ cm., prove that the lengths of the sides are $3\sqrt{5}, 6\sqrt{2}$ and $9$ cm., and that the
  tangents of other angles are respectively $2$ and $3$.
  %134
\item Find the value of $\cos^3\frac{\pi}{8}\cos\frac{3\pi}{8} + \sin^3\frac{\pi}{8}\sin\frac{3\pi}{8}$.
  %135
\item Find the value of $\sin10^\circ\sin30^\circ\sin50^\circ\sin70^\circ$.
  %136
\item Find the value of $\cos^210^\circ - \cos10^\circ\cos50^\circ + \cos^250^\circ$.
  %137
\item If the lengths of the sides of a triangle are in A.P.\ and the greatest angle is double the smallest
  then find the ratio of lengths of the sides of the triangle.
  %138
\item If $\cos(\alpha + \beta) = \frac{3}{5}, \sin(\alpha - \beta) = \frac{5}{13}$ and $0 < \alpha, \beta
  < \frac{\pi}{4}$, then find $\tan2\alpha$.
  %139
\item Let $f_k(x) = \frac{1}{k}\left(\sin^kx + \cos^kx\right)$ for $k = 1, 2, 3, \ldots\forall
  x\in\mathbb{R}$. Find the value of $f_4(x) - f_6(x)$.
  %140
\item Find the value of
  $\cos\frac{\pi}{2^2}\cos\frac{\pi}{2^3}\cdots\cos\frac{\pi}{2^{10}}\sin\frac{\pi}{2^{10}}$.
  %141
\item For $\theta \in\left(\frac{\pi}{4}, \frac{\pi}{2}\right)$, find the value of $3(\sin\theta
  - \cos\theta)^4 + 6(\sin\theta + \cos\theta)^2 + 4\sin^6\theta$.
  %142
\item Prove that $\frac{\tan A}{1 - \cot A} + \frac{\cot A}{1 - \tan A} = 1 + \sec A\csc A$.
  %143
\item Find the number of ordered pairs $(\alpha, \beta)$, where $\alpha, \beta\in(-\pi, \pi)$ satisfying
  $\cos^2(\alpha - \beta) = 1$ and $\cos{\alpha + \beta} = \frac{1}{e}$.
  %144
\item Given both $\theta$ and $\phi$ are acute angles and $\sin\theta = \frac{1}{2}, \cos\phi
  = \frac{1}{3}$, then find the range to which $\theta + \phi$ belongs.
  %145
\item Which among the following is rational? $\sin15^\circ, \cos15^\circ, \sin15^\circ\cos15^\circ$, and
  $\sin15^\circ\cos75^\circ$.
  %146
\item Find the value of $3(\sin x - \cos x)^4 + 6(\sin x + \cos x)^2 + 4\left(\sin^6x + \cos^6x\right)$.
  %147
\item Find the value of $\sqrt{3}\csc20^\circ - \sec20^\circ$.
  %148
\item Find the value of the expression $3\left[\sin^4\left(\frac{3\pi}{2} - \alpha\right) + \sin^4(3\pi
  + \alpha)\right] - 2\left[\sin^6\left(\frac{\pi}{2} + \alpha\right) + \sin^6(5\pi - \alpha)\right]$.
  %149
\item Let $f: (-1, 1)\rightarrow \mathbb{R}$ be such that $f(\cos4\theta) = \frac{2}{2 - \sec^2\theta}$ for
  $\theta\in \left(0, \frac{\pi}{4}\right)\cup \left(\frac{\pi}{4}, \frac{\pi}{2}\right)$. Find the value of
  $f\left(\frac{1}{3}\right)$.
  %150
\item For $0 < \theta < \frac{\pi}{2}$, find the solutions of $\displaystyle\sum_{m =
  1}^6\csc\left(\theta + \frac{(m - 1)\pi}{4}\right)\csc\left(\theta + \frac{m\pi}{4}\right) =
  4\sqrt{2}$.
  %151
\item If $\frac{\sin^4x}{2} + \frac{\cos^4x}{3} = \frac{1}{5}$, then prove that $\frac{\sin^8x}{8}
  + \frac{\cos^8x}{27} = \frac{1}{125}$.
  %152
\item For a positive integer $n$, let $f_n(\theta) = \tan\frac{\theta}{2}(1 + \sec\theta) + (1
  + \sec2\theta)\left(1 + \sec^2\theta\right)\ldots\left(1 + \sec2^n\theta\right)$, then find
  $f_5(\frac{\pi}{128})$.
  %153
\item If $\frac{\sqrt{2}\sin\alpha}{\sqrt{1 + \cos2\alpha}} = \frac{1}{7}$ and $\sqrt{\frac{1
    - \cos2\beta}{2}} = \frac{1}{\sqrt{10}}, (\alpha, \beta)\in\left(0, \frac{\pi}{2}\right)$, then find
  $\tan(\alpha + 2\beta)$.
  %154
\item Find the number of all possible values of $\theta$, where $0 < \theta < \pi$, for which the system of
  equations $(y + z)\cos3\theta = xyz\sin3\theta, x\sin\theta = \frac{2\cos3\theta}{y}
  + \frac{2\sin3\theta}{z}$ and $xyz\sin3\theta = (y + 2z)\cos3\theta + y\sin3\theta$ have a solution $(x_0,
  y_0, z_0)$ with $y_oz_0\neq 0$.
  %155
\item If $\alpha + \beta = \frac{\pi}{2}$ and $\beta + \gamma = \alpha$, then prove that $\tan\alpha
  = \tan\beta + 2\tan\gamma$.
  %156
\item If $\tan A = \frac{1 - \cos B}{\sin B}$, then prove that $\tan2A = \tan B$.
  %157
\item $\triangle ABC$ is such a triangle that $\sin(2A + B) = \sin(C - A) = -\sin(B + 2C) = 1/2$. If $A, B,
  C$ are in A.P.\ determine the values of $A, B$ and $C$.
  %158
\item Find the maximum value of $3\cos\theta + 5\sin\left(\theta - \frac{\pi}{6}\right)$ for any real value
  of $\theta$.
  %159
\item Prove that the value of the function $\frac{\sin x\cos3x}{\sin3x\cos x}$ do not lie between $3$ and
  $\frac{1}{3}$ for any value of $x$.
  %160
\item A $2$ m ladder leans  against a verrtical wall. If the top of the ladder begins to slide down the wall
  at  the rate of $25$ cm/s, then find the rate at which the bottom of the ladder slides away from the wall
  on the horizontal ground when the top of the ladder is $1$ m above the ground.
  %161
\item $ABC$ is a triangular park with $AB = AC = 100$ m. A vertical tower is situated at mid-point of
  $BC$. If the angle of elevation of the top of tower at $A$ and $B$ are $\cot^{-1}(3\sqrt{2})$ and
  $\csc^{-1}(2\sqrt{2})$ respectively, find the height of the tower.
  %162
\item Two poles standing on a horizontal ground are of heights $5$ m and $10$ m, respectively. The line
  joining their tops makes an angle of $15^\circ$ with the ground. Find the distance between the poles.
  %163
\item Two verticle poles of heights $20$ m and $80$ m stand apart on a horizontal plane. Find the height of
  the point of intersection of the lines joining the top of each pole to the foot of other.
  %164
\item If the equation $\cos^4\theta + \sin^4\theta + \lambda = 0$ has real solutions for $\theta$, then find
  the range in which $\lambda$ lies.
  %165
\item Let $S$ be the set of all $\alpha\in\mathbb{R}$ such that the equation $\cos2x + \alpha\sin x =
  2\alpha - 7$ has a solution. Find $S$.
  %166
\item Find the number of solutions of the equation $1 + \sin^4x = \cos^23x,
  x\in\left[-\frac{5\pi}{2}, \frac{5\pi}{2}\right]$.
  %167
\item If $S = \left\{\theta \in [-2\pi, 2\pi]: 2\cos^2\theta + 3\sin\theta = 0\right\}$, then find the sum of
  the elements of $S$.
  %168
\item If $\sin^4\alpha + \cos^4\beta + 2 = 4\sqrt{2}\sin\alpha\cos\beta, \alpha, \beta \in[0,\pi]$, then
  find $\cos(\alpha + \beta) - \cos(\alpha - \beta)$.
  %169
\item Let $\alpha$ and $\beta$ be the roots of the quadratic equation $x^2\sin\theta - x(\sin\theta
  + \cos\theta + 1) + \cos\theta = 0\left(0 < \theta < 45^\circ\right)$ and $\alpha < \beta$. Find
  $\displaystyle\sum_{n = 0}^\infty\left(\alpha^n + \frac{(-1)^n}{\beta^n}\right)$.
  %170
\item Find the sum of all values of $\theta\in\left(0, \frac{\pi}{2}\right)$ satisfying $\sin^22\theta
  + \cos^42\theta = \frac{3}{4}$.
  %171
\item If $0\leq x <\frac{\pi}{2}$, then find the number of values of $x$ for which $\sin x - \sin2x + \sin3x
  = 0$.
  %172
\item If sum of all the solutions of the equation $8\cos x\left[\cos\left(\frac{\pi}{6} +
  x\right)\cos\left(\frac{\pi}{6} - x\right) - \frac{1}{2}\right] = 1$ in $[0, \pi]$ is $k\pi$, then find
  $k$.
  %173
\item If $5\left(\tan^2x - \cos^2x\right) = 2\cos2x + 9$, then find the value of $\cos4x$.
  %174
\item If $0\le x < 2\pi$, then find the number of real values of $x$, which satisfy the equation $\cos x
  + \cos2x + \cos3x + \cos4x = 0$.
  %175
\item Let $S = \left\{x\in(-\pi, \pi): x\ne 0, \pm\frac{\pi}{2}\right\}$. Find the sum of all distinct
  solutions of $\sqrt{3}\sec x + \csc x + 2(\tan x - \cot x) = 0$ in the set $S$.
  %176
\item If $P = \left\{\theta: \sin\theta - \cos\theta = \sqrt{2}\cos\theta\right\}$ and $Q
  = \left\{\sin\theta + \cos\theta = \sqrt{2}\sin\theta\right\}$ then prove that $P = Q$.
  %177
\item Find the general value of $\theta$ satisfying the equation $2\sin^2\theta - 3\sin\theta - 2 = 0$.
  %178
\item In a $\triangle ABC$, angle $A$ is greater than angle $B$. If the measure of the angles $A$ and $B$
  satisfy the equation $3\sin x - 4\sin^3x - k = 0, 0 < k < 1$, then find the angle $C$.
  %179
\item Find the general solution of $\sin x - 3\sin2x + \sin3x = \cos x - 3\cos2x + \cos3x$.
  %180
\item Find the value of $\theta\in\left[0, \frac{\pi}{2}\right]$ satisfying the equation
  $\startdeterminant\NC 1 + \sin^2\theta\NC \cos^2\theta\NC 4\sin4\theta\NR\NC \sin^2\theta\NC 1
  + \cos^2\theta\NC 4\sin4\theta\NR\NC \sin^2\theta\NC \cos^2\theta\NC 1 + 4\sin4\theta\NR\stopdeterminant =
  0$.
  %181
\item Let $a, b, c$ be three non-zero real numbers such that the equation $\sqrt{3}a\cos x + 2b\sin x = c,
  x\in\left[-\frac{\pi}{2}, \frac{\pi}{2}\right]$ has two distinct real roots $\alpha$ and $\beta$ such that
  $\alpha + \beta = \frac{\pi}{3}$. Find the value of $\frac{b}{a}$.
  %182
\item Find the number of distinct solutions of the equation $\frac{5}{4}\cos^22x + \cos^4x + \sin^4x
  + \cos^6x + \sin^6x = 2$ in the interval $[0, 2\pi]$.
  %183
\item Find the general value of $\theta$ satisfying the equation $\tan^2\theta + \sec2\theta = 1$.
  %184
\item Determine the smallest positive value of $x$ for which $\tan\left(x + 100^\circ\right) = \tan\left(x
  + 50^\circ\right)\tan\left(x - 50^\circ\right)$.\
  %185
\item Consider the system of equations in $x, y$ and $z$
  \startformula\startalign
    \NC (\sin3\theta)x - y + z\NC = 0\NR
    \NC (\cos2\theta)x + 4y + 3z\NC = 0\NR
    \NC 2x + 7y + 7z\NC = 0.\NR
  \stopalign\stopformula

  Find the values of $\theta$ for which this system has non-trivial solutions.
  %186
\item If $\exp\left\{\left(\sin^2x + \sin^4x + \sin^6x + \cdots \infty\right)\log_e2\right\}$, satisfies the
  equation $x^2 - 9x + 8 = 0$, find the value of $\frac{\cos nx}{\cos x + \sin x}, o < x < \frac{\pi}{2}$.
  %187
\item Solve $2(\cos x + \cos2x) + (1 + 2\cos x)\sin2x = 2\sin x, -\pi\le x\le \pi$.
  %188
\item Find all $x$ satisfying the inequality $\left(\cot^{-1}x\right)^2 - 7\cot^{-1}x + 10 > 0$.
  %189
\item Find the number of solutions of $\sin x + 2\sin2x - \sin3x = 3$ for $x\in(0, \pi)$.
  %190
\item Find the number of solutions of the pair of equations $2\sin^2\theta - \cos2\theta = 0$ and
  $2\cos^2\theta - 3\sin\theta = 0$ in the interval $[0, 2\pi]$.
  %191
\item Find the number of integral values of $k$ for which $7\cos x + 5\sin x = 2k + 1$ has a solution.
  %192
\item Find the number of values of $x$ in the interval $[0, 5\pi]$ satisfying the equation $3\sin^2x - 7\sin
  x + 2 = 0$.
  %193
\item Find the number of solutions of the equation $\tan x + \sec x = 2\cos x$ lying in the interval $[0,
  2\pi]$.
  %194
\item Find the number of solutions of $\sin\left(e^x\right) = 5^x + 5^{-x}$.
  %195
\item The equation $(\cos p - 1)x^2 + \cos p.x + \sin p = 0$ in the variable $x$, has real roots. Find the
  interval in which $p$ can take any value.
  %196
\item Find the number of all triplets $(a_1, a_2, a_3)$ such that $a_1 + a_2\cos2x + a_3\sin^2x = 0\forall
  x$.
  %197
\item Let $f: [0, 2] \rightarrow\mathbb{R}$ be the function defined by $f(x) = (3 - \sin 2\pi
  x)\sin\left(\pi x - \frac{\pi}{4}\right) - \sin\left(3\pi x + \frac{\pi}{4}\right)$. If
  $\alpha, \beta\in[0, 2]$ are such that $\left\{x\in[0, 2]: f(x)\ge 0\right\}
  = \left\{\alpha, \beta\right\}$, then find the value of $\beta - \alpha$.
  %198
\item Find the integer $n > 3$ satisfying the equation $\frac{1}{\sin\frac{\pi}{n}}
  = \frac{1}{\sin\frac{2\pi}{n}} + \frac{1}{\sin\frac{3\pi}{n}}$
  %199
\item Find the values of $\theta$ in the interval $\left(-\frac{\pi}{2}, \frac{\pi}{2}\right)$, which
  satisfies $\tan\theta = \cot5\theta$ and $\sin2\theta = \cos4\theta$, where $\theta\ne \frac{n\pi}{5}$ for
  $n = 0, \pm1, \pm2$.
  %200
\item For real $x, y$ find the solution of the equations $x + y = \frac{2\pi}{3}$ and $\cos x + \cos y
  = \frac{3}{2}$.
  %201
\item Find the larger of $\cos(\log\theta)$ and $\log(\cos\theta)$.
  %202
\item Find the smallest positive number $p$ for which the equation $\cos(p\sin x) = \sin(p\cos x)$ has a
  solution in $x\in[0, 2\pi]$.
  %203
\item The domain of the function $f(x) = \sin^{-1}\left(\frac{|x| + 5}{x^2 + 1}\right)$ is $(-\infty,
  -a]\cup [a, \infty)$. Find $a$.
  %204
\item Find the number of real solutions of $\tan^{-1}\sqrt{x(x + 1)} + \sin^{-1}\sqrt{x^2 + x + 1}
  = \frac{\pi}{2}$.
  %205
\item Find the greater of the two angles $A = 2\tan^{-1}\left(2\sqrt{2} - 1\right)$ and $B =
  3\sin^{-1}\frac{1}{3} + \sin^{-1}\frac{3}{5}$.
  %206
\item For non-negative integers $n$, let $\displaystyle f(n) = \frac{\displaystyle \sum_{k =
    0}^n\sin\left(\frac{k + 1}{n + 2}\pi\sin\frac{k + 2}{n + 2}\pi\right)}{\displaystyle\sum_{k =
    0}^n\sin^2\left(\frac{k + 1}{n + 2}\pi\right)}$. Assuming $\cos^{-1}x$ takes value in $[0, \pi]$, prove
  that if $\alpha = \tan\left(cos^{-1}f(6)\right)$, then $\alpha^2 + 2\alpha - 1 = 0$, $f(4)
  = \frac{\sqrt{3}}{2}$, and $\sin\left(7\cos^{-1}f(5)\right) = 0$.
  %207
\item Let $f(x) = \log_e(\sin x), (0 < x < \pi)$ and $g(x) = \sin^{-1}\left(e^{-x}\right), (x\ge 0)$. If
  $\alpha$ is a positive real number such that $a = (fog)'(\alpha)$ and $b = (fog)(\alpha)$, then prove that
  $a\alpha^2 - b\alpha - a = 1$.
  %208
\item Find the value of $\displaystyle\cot\left[\sum_{n = 1}^{19}\cot^{-1}\left(1 + \sum_{p =
    1}^n2p\right)\right]$.
  %209
\item If $\sin^{-1}\left(x - \frac{x^2}{2} + \frac{x^3}{4}- \cdots\right) + \cos^{-1}\left(x^2
  - \frac{x^4}{2} + \frac{x^6}{4} - \cdots\right) = \frac{\pi}{2}$ for $0 < |x| < \sqrt{2}$, then find $x$.
  %210
\item Find the value of $\displaystyle\sec^{-1}\left[\frac{1}{4}\sum_{k = 0}^{10}\sec\left(\frac{7\pi}{12}
  + \frac{k\pi}{2}\right)\sec\left(\frac{7\pi}{12} + \frac{(k + 1)\pi}{2}\right)\right]$ in the interval
  $\left[-\frac{\pi}{4}, \frac{3\pi}{4}\right]$.
  %211
\item Find the number of real solutions of the equation $\sin^{-1}\left[\displaystyle\sum_{i = 1}^\infty
  x^{i + 1} - x\sum_{i = 1}^\infty\left(\frac{x}{2}\right)^i\right] = \frac{\pi}{2} - \cos^{-1}\left[\sum_{i
  = 1}^\infty\left(-\frac{x}{2}\right)^i - \sum_{i = 1}^\infty(-x)^i\right]$ lying in the interval
  $\left(-\frac{1}{2}, \frac{1}{2}\right)$.
  %212
\item If $f: [0, 4\pi]\rightarrow [0, \pi]$ be defined by $f(x) = \cos^{-1}(\cos x).9$. Then find the number
  of points $x\in[0, 4\pi]$ satisfying the equation $f(x) = \frac{10 - x}{10}$.
  %213
\item If $\cos^{-1}x - \cos^{-1}\frac{y}{2} = \alpha$, where $-1\le x\le 1, -2\le y\le 2, x\le\frac{y}{2}$,
  then for all $x$, find $4x^2 - 4xy\cos\alpha + y^2$.
  %214
\item If $a, b, c$ are positive real numbers $\theta = \tan^{-1}\sqrt{\frac{a(a + b + c)}{bc}}
  + \tan^{-1}\sqrt{\frac{b(a + b + c}{ca})} + \tan^{-1}\sqrt{\frac{c(a + b + c)}{ab}}$, then find
  $\tan\theta$.
  %215
\item The angles $A, B$ and $C$ of a $\triangle ABC$ are in A.P.\ and $a:b = 1:\sqrt{3}$. If $c = 4$ cm,
  then find the area of the triangle.
  %216
\item Given, $\frac{b + c}{11} = \frac{c + a}{12} = \frac{a + b}{13}$ for a $\triangle ABC$ with usual
  notations. If $\frac{\cos A}{\alpha} = \frac{\cos B}{\beta} = \frac{\cos C}{\gamma}$, then find the
  ordered triplet $(\alpha, \beta, \gamma)$.
  %217
\item In a triangle, the sum of lengths of two sides is $x$ and the product of lengths of the same sides is
  $y$. If $x^2 - c^2 = y$, where $c$ is the length of the third side of the triangle, then find the
  circumradius of the triangle.
  %218
\item $ABCD$ is a trapezium such that $AB$ and $CD$ are parallel and $BC\perp CD$, if $\angle ADB = \theta,
  BC = p, CD = q$, then find $AB$.
  %219
\item If the angles of a $\triangle ABC$ are in an A.P.\ and if $a, b$, and $c$ are the lengths of sides,
  then find $\frac{a}{c}\sin2C + \frac{c}{a}\sin2A$.
  %220
\item Prove that in a $\triangle ABC, (b - c)\cos\frac{A}{2} = a\sin\left(\frac{B - C}{2}\right)$.
  %221
\item If the angles of a triangle are in the ratio $4:1:1$, then find the ratio of the longest side to
  perimeter.
  %222
\item In a $\triangle ABC$, prove that $2ac\sin\left(\frac{1}{2}(A - B + C)\right) = a^2 + c^2 - b^2$.
  %223
\item In a $\triangle PQR, \angle R = \frac{\pi}{2}$, if $\tan\frac{P}{2}$ and $\tan\frac{Q}{2}$ are the
  roots of the equation $ax^2 + bx + c = 0(a\ne 0)$, then prove that $a + b = c$.
  %224
\item In a $\triangle ABC, \sin A, \sin B,\sin C$ are in A.P., then prove that the altitudes are in H.P.
  %225
\item In a $\triangle ABC, \angle B = \frac{\pi}{3}$ and $\angle C = \frac{\pi}{4}$. Let $D$ divide $BC$
  internally in the ratio $1:3$ then find $\frac{\sin\angle BAD}{\sin\angle CAD}$.
  %226
\item In a $\triangle PQR, P$ is the largest angle and $\cos P = \frac{1}{3}$. Further in the circle of the
  triangle touches the sides $PQ, QR$ and $RP$ at $N, L$ and $M$ such that the lengths $PN, QL$ and $RM$
  are consecutive even integers. Find the possible lengths of the sides.
  %227
\item Let $ABC$ be a triangle such that $\angle ACB = \frac{\pi}{6}$. If $a, b$ and $c$ denote the lengths
  of sides opposite to $A, B$ and $C$ respectively. Find the values of $x$ for which $a = x^2 + x + 1, b =
  x^2 - 1$ and $c = 2x + 1$.
  %228
\item In a $\triangle ABC$ with fixed base $BC$, the vertex $A$ moves such that $\cos B + \cos C =
  4\sin^2\frac{A}{2}$. Prove that $b + c = 2a$ or locus of point $A$ is an ellipse.
  %229
\item Internal bisector of $\angle A$ of $\triangle ABC$ meets side $BC$ at $D$. A line drawn through $D$
  perpendicular to $AD$ intersects the side $AC$ at $E$ and side $AB$ at $F$. If $a, b, c$ represent the
  sides of $\triangle ABC$, then prove that (a) $AE$ is H.M.\ of $b$ and $c$, (b) $AD = \frac{2bc}{b +
    c}\cos\frac{A}{2}$, (c) $EF = \frac{4bc}{b + c}\sin\frac{A}{2}$, and (d) $\triangle AEF$ is isosceles.
  %230
\item Prove that if $b\sin A = a, A < \frac{\pi}{2}$ or if $b\sin A < a, A < \frac{A}{2}, b > a$ then a
  triangle is possible.
  %231
\item In a $\triangle ABC, AD$ si the altitude from $A$. Given $b > c, \angle C = 23^\circ$ and $AD
  = \frac{abc}{b^2 - c^2}$, then find $\angle B$.
  %232
\item If in a $\triangle ABC, \frac{2\cos A}{a} + \frac{\cos B}{b} + \frac{2\cos C}{c} = \frac{a}{bc}
  + \frac{b}{ca}$, then find the value of $\angle A$.
  %233
\item The sides of a triangle are three consecutive natural numbers and its largest angle is twice the
  smallest one. Determine the sides of the triangle.
  %234
\item In a $\triangle ABC$, the median to the side $BC$ is of length $\frac{1}{\sqrt{11 - 6\sqrt{3}}}$ and
  it divides the angle $\angle A$ into angles $30^\circ$ and $45^\circ$. Find the length of the side $BC$.
  %235
\item In a $\triangle ABC$, if $\angle A + \angle B = 120^\circ, a = \sqrt{3} + 1$ and $b = \sqrt{3} - 1$,
  then find $\angle A:\angle B$.
  %236
\item If $PQR$ is a triangle of area $\Delta$ with $a = 2, b = \frac{7}{2}$ and $c = \frac{5}{2}$, where
  $a, b$ and $c$ are the lengths of the sides of the triangle opposite to $P, Q$ and $R$, respectively. Find
  $\frac{2\sin P - \sin 2P}{2\sin P + \sin2P}$.
  %237
\item Inradius of a circle inscribed in an isosceles triangle one of whose angle is $\frac{2\pi}{3}$ is
  $\sqrt{3}$, then find the area of the triangle.
  %238
\item The sides of a triangle are in the ratio $1:\sqrt{3}:2$, then find the ratio of the angles.
  %239
\item Let $x, y$ and $z$ be positive real numbers. Suppose $x, y$ and $z$ are the lengths of the sides of a
  triangle opposite to angles $X, Y$ and $Z$ respectively. If $\tan\frac{X}{2} + \tan \frac{Z}{2}
  = \frac{2y}{x + y + z}$, then prove that $\tan\frac{X}{2} = \frac{x}{y + z}$.
  %240
\item If the angles of a triangle are $30^\circ$ and $45^\circ$ and the included side is $\sqrt{3} + 1$ cm,
  then find the area of the triangle.
  %241
\item Find the set of real numbers $a$ such that $a^2 + 2a, 2a + 3$ and $a^2 + 3a + 8$ are the sides of a
  triangle.
  %242
\item If $\Delta$ is the area of the triangle with side lengths $a, b, c$, then show that
  $\Delta\le \frac{1}{4}\sqrt{(a + b + c)abc}$. Also show that the equality occurs if and only if $a = b =
  c$.
  %243
\item Show that for any triangle with sides $a, b, c$ we have $3(ab + bc + ca)\le(a + b + c)^2\le 4(ab + bc
  + ca)$.
  %244
\item Let $A, B, C$ be three angles such that $A = \frac{\pi}{4}$ and $\tan B\tan C = p$. Find all positive
  values of $p$ such that $A, B, C$ are angles of a triangle.
  %245
\item Consider the following statements about a $\triangle ABC$
  \startitemize[i, 1*broad]
  \item The sides $a, b, c$ and area of the triangle are rational.
  \item $a, \tan\frac{B}{2}, \tan\frac{C}{2}$ are rational.
  \item $a, \sin A, \sin B$ are rational.
  \stopitemize

  Prove that $(i)\Rightarrow (ii)\Rightarrow (iii)\Rightarrow (i)$
  %246
\item If $p_1, p_2, p_3$ are the perpendiculars from the vertices of a triangle to opposite sides, then
  prove that $p_1p_2p_3 = \frac{a^2b^2c^2}{8R^3}$.
  %247
\item Let $ABC$ and $ABC'$ be two non-congruent triangles with sides $AB = 4, AC = AC' = 2\sqrt{2}$ and
  $\angle B = 30^\circ$. Find the absolute value of difference between the areas of the triangles.
  %248
\item Two vertices of a triangle are $(0, 2)$ and $(4, 3)$. If its orthocenter is origin, then find the
  quadrant in which the third vertes lie.
  %249
\item Let the equations of two sides of a triangle be $3x - 2y + 6 = 0$ and $4x + 5y - 20 = 0$. If the
  orthocenter is $(1, 1)$, then find the equation of the third side.
  %250
\item In a triangle sum of two sides is $x$ and the product of the same two sides is $y$. If $x^2 - c^2 =
  y$, where $c$ is the third side of the triangle, then find the ratio of inradius to circumradius.
  %251
\item In a $\triangle ABC$, let $\angle C = \frac{\pi}{2}$. If $r$ is the inradius and $R$ is the
  circumradius, then prove that $2(r + R) = a + b$.
  %252
\item Consider the circle $x^2 + y^2 = 9$ and the parabola $y^2 = 8x$. They intersect at $P$ and $Q$ in the
  first and fourth quadrants, respectively. Tangents to the circle at $P$ and $Q$ intersect the $X$-axis at
  $R$ and tangents to the parabola at $P$ and $Q$ intersect the $X$-axis at $S$. Find the inradius of
  $\triangle PQR$, circumradius of $\triangle PRS$, and the ratio of areas of $\triangle PQA$ and $\triangle
  PQR$.
  %253
\item In a non-right-angled $\triangle PQR$, let $p, q, r$ denote the lenghts of the sides opposite to the
  angles at $P, Q, R$ respectively. The median from $R$ meets the side $PQ$ at $S$, the perpendicular from
  $P$ meets the side $QR$ at $E$, and $RS$ and $PE$ intersect at $O$. If $p = \sqrt{3}, q = 1$, and the
  radius of the circumcircle of the $\triangle PQR$ equals $1$. Prove that $OE = \frac{1}{6}, RS
  = \frac{\sqrt{7}}{2}$, and $r = \frac{\sqrt{3}}{2}\left(2 - \sqrt{3}\right)$.
  %254
\item In a $\triangle PQR$, let $\angle PQR = 30^\circ$ and the sides $PQ$ and $QR$ have lengths
  $10\sqrt{3}$ and $10$, respectively. Prove that the area of $\triangle PQR$ is $25\sqrt{3}, \angle QRP =
  120^\circ, r = 10\sqrt{3} - 15$ and the area of the circumcircle of the $\triangle PQR$ is $100\pi$.
  %255
\item In a $\triangle XYZ$, let $x, y, z$ be the lengths of sides opposite to the angles $X, Y, Z$
  respectively and $2s = x + y + z$. If $\frac{s - x}{4} = \frac{s - y}{3} = \frac{s - z}{2}$ and area of
  incircle of $\triangle XYZ$ is $\frac{8\pi}{3}$, then prove that $\sin^2\frac{X + Y}{2} = \frac{3}{5}$.
  %256
\item A straight line through the vertex $P$ of a $\triangle PQR$ intersects the side $QR$ at the point $S$
  and the circumcircle of the the triangle at point $T$. If $S$ is not the circumcenter then prove that
  $\frac{1}{PS} + \frac{1}{ST} > \frac{2}{\sqrt{QS\times SR}}$. Also, prove that $\frac{1}{PS}
  + \frac{1}{ST} > \frac{4}{QR}$.
  %257
\item Circle with radii $3, 4$ and $5$ touch each other externally, if $P$ is the point of intersection of
  tangents of these circles at their point of contact. Find the distance of $P$ from the point of contact.
  %258
\item $I_n$ is the area of $n$-sided regular polygon inscribed in a circle of unit radius and $O_n$ be the
  area of the polygon circumscribing the given circle. Prove that $I_n = \frac{O_n}{2}\left[1 + \sqrt{1
      - \left(\frac{2I_n}{n}\right)^2}\right]$.
  %259
\item Consider a $\triangle ABC$ and let $a, b$ and $c$ denote the lengths of the sides opposite to the
  vertices $A, B$ and $C$, respectively. $a = 6, b = 10$ and the area of the triangle is $15\sqrt{3}$. If
  $\angle ACB$ is obtuse and if $r$ denotes the inradius of the triangle then find $r^2$.
  %260
\item For any positive integer $n$, let $S_n: (0, \infty)\rightarrow\mathbb{R}$ be defined by $S_n(x)
  = \displaystyle\sum_{k = 1}^n\cot^{-1}\left[\frac{1 + k(k + 1)x^2}{x}\right]$, where for any
  $x\in\mathbb{R}, \cot^{-1}x\in(0, \pi)$ and
  $\tan^{-1}x\in\left(-\frac{\pi}{2}, \frac{\pi}{2}\right)$. Then which of the following statements is/are
  true?
  \startitemize[a, 1*broad]
  \item $S_{10}x = \frac{\pi}{2} - \tan^{-1}\frac{1 + 11x^2}{10x}, \forall x > 0$
  \item $\displaystyle\lim_{n\to\infty}\cot\left(S_n(x)\right) = x, \forall x > 0$
  \item The equation $S_3(x) = \frac{\pi}{4}$ has a root in $(0, \infty)$
  \item $\tan\left(S_n(x)\right)\le \frac{1}{2}\forall n\ge 1$ and $x > 0$
  \stopitemize
  %261
\item In a $\triangle ABC$, let $AB = \sqrt{23}, BC = 3$ and $CA = 4$. Then find the value of $\frac{\cot A
  + \cot C}{\cot B}$.
  %262
\item Consider a $\triangle PQR$ having sides of lengths $p, q$ and $r$ opposite to angles $P, Q$ and $R$,
  respectively. Then which of the following statements is/are true?
  \startitemize[a, 1*broad]
  \item $\cos P > 1 - \frac{p^2}{2qr}$
  \item $\cos R\ge \frac{q - r}{p + q}\cos P + \frac{p - r}{p + q}\cos Q$
  \item $\frac{q + r}{p} < 2\frac{\sqrt{\sin Q\sin R}}{\sin P}$
  \item If $p < q$ and $p < r$, then $\cos Q > \frac{p}{r}$ and $\cos R > \frac{p}{q}$.
  \stopitemize
  %263
\item A tower $PQ$ stands on a horizontal ground with base $Q$ on the ground. The point $R$ divides the two
  parts such that $QR = 15$ m. If from a point $A$ on the ground the angle of elevation of $R$ is $60^\circ$
  and the part $PR$ of the tower subtends an angle $15^\circ$ at $A$, then find the height of the tower.
  %264
\item Find the number of $\theta\in(0, 4\pi)$ for which the system of equations
  \startformula\startalign
    \NC3(\sin3\theta)x - y + z\NC = 2\NR
    \NC3(\cos2\theta)x + 4y + 3z\NC = 3\NR
    \NC 6x + 7y + 7z\NC = 9\NR
  \stopalign\stopformula

  has no solution.
  %265
\item Find the number of values of $x$ in the interval $\left(\frac{\pi}{4}, \frac{7\pi}{4}\right)$ for
  which $14\csc^{2}x - 2\sin^{2}x = 21 - 4\cos^{2}x$ holds.
  %266
\item Find the value of $2\sin12^\circ - \sin72^\circ$.
  %267
\item Let $S = \left\{\theta\in[0, 2\pi]: 8^{2\sin^{2}\theta} + 8^{2\cos^{2}\theta} = 16\right\}$. Find $n(S)
  + \displaystyle\sum_{\theta\in S}\left[\sec\left(\frac{\pi}{4} + 2\theta\right)\csc\left(\frac{\pi}{4} +
    2\theta\right)\right]$.
  %268
\item If $0 < x < \frac{1}{\sqrt{2}}$ and $\frac{\sin^{-1}x}{\alpha} = \frac{\cos^{-1}x}{\beta}$, then find
  a value of $\sin\frac{2\pi\alpha}{\alpha + \beta}$.
  %269
\item If the sum of solutions of the system of equations $2\sin^{2}\theta - \cos2\theta = 0$ and
  $2\cos^{2}\theta + 3\sin\theta = 0$ in the interval $[0, 2\pi]$ is $k\pi$, then find $k$.
  %270
\item If $\sin^{2}10^\circ\sin20^\circ\sin40^\circ\sin50^\circ\sin70^\circ = \alpha
  - \frac{1}{16}\sin10^\circ$, then find $16 + \alpha^{-1}$.
  %271
\item The angle of elevation of the top $P$ of a vertical tower $PQ$ of height $10$ from a point on the
  ground is $45^\circ$. Let $R$ be a point on $AQ$ and from a point $B$, vertically above $R$, the angle of
  elevation of $P$ is $60^\circ$. If $\angle BAQ = 30^\circ, AB = d$ and the area of the trapezium $PQRB$ is
  $\alpha$, then find the ordered pair $(d, \alpha)$.
  %272
\item Let $S = \left\{\theta\in\left(0, \frac{\pi}{2}\right): \displaystyle\sum_{m = 1}^9\sec\left(\theta +
  (m - 1)\frac{\pi}{6}\right) \sec\left(\theta + \frac{m\pi}{6}\right) = -\frac{8}{\sqrt{3}}\right\}$. Prove
  that $\displaystyle\sum_{\theta\in S}\theta = \frac{\pi}{2}$.
  %273
\item Find the value of $\cot\left[\displaystyle\sum{n = 1}^{50}\tan^{-1}\frac{1}{1 + n + n^{2}}\right]$.
  %274
\item Find the equation one of whose roots is $\sin36^\circ = \alpha$.
  %275
\item Considering the principal values of the inverse trigonometric functions, find the sum of all the
  solutions of the equation $\cos^{-1}x - 2\sin^{-1}x = \cos^{-1}2x$.
  %276
\item Let $S = \left\{\theta\in(0, 2\pi): 7\cos^{2}\theta - 3\sin^{2}\theta - 2\cos^{2}2\theta =
  2\right\}$. Find the sum of roots of all the equations $x^{2} - 2\left(\tan^{2}\theta
  + \cot^{2}\theta\right)x + 6\sin^{2}\theta = 0, \theta\in S$.
  %277
\item Find the number of solutions in the set $S= \left\{x\in\mathbb{R}: 2\cos\left(\frac{x^{2} + x}{6} = 4^x
  + 4^{-x}\right)\right\}$.
  %278
\item Find the number of solutions in the set $S = \left\{\theta\in[-4\pi, 4\pi]: 3\cos^{2}2\theta +
  6\cos2\theta - 10\cos^{2}\theta + 5 = 0\right\}$.
  %279
\item Find the number of solutions of the equation $2\theta - \cos^{2}\theta + \sqrt{2} = 0$.
  %280
\item Considering only the principal values of the inverse trigonometic functions, find the value of
  $\frac{3}{2}\cos^{-1}\sqrt{\frac{2}{2 + \pi^{2}}} + \frac{1}{4}\sin^{-1}\frac{2\sqrt{2}\pi}{2 + \pi^{2}} +
  \tan^{-1}\frac{\sqrt{2}}{\pi}$.
  %281
\item Let $ABC$ be the triangle with $AB = 1, AC = 3$ and $\angle BAC = \frac{\pi}{2}$. If a circle of
  radius $r$ touches the sides $AB$ and $AC$ and also touches the circumcircle of the triangle $ABC$, then
  find the value of $r$.
  %282
\item Let $\alpha, \beta$ be real numbers such that $-\frac{\pi}{4} < \beta < 0 < \alpha
  < \frac{\pi}{4}$. If $\sin(\alpha + \beta) = \frac{1}{3}$ and $\cos(\alpha - \beta) = \frac{2}{3}$, then
  find the greatest integer less than or equal to $\left(\frac{\sin\alpha}{\cos\beta}
  + \frac{\cos\beta}{\sin\alpha} + \frac{\cos\alpha}{\sin\beta} + \frac{\sin\beta}{\cos\alpha}\right)^{2}$.
  %283
\item Let $PQRS$ be a quadrilateral in a plane, where $QR = 1, \angle PQR = \angle QRS = 70^\circ, \angle
  PQS = 15^\circ$ and $\angle PRS = 40^{\circ}$. If $\angle RPS = \theta, PQ = \alpha$ and $PS = \beta$,
  find the value/range of $4\alpha\beta\sin\theta$.
  %284
\item Let $G$ be a circle of radius $R$. Let $G_1, G_2, \ldots, G_n$ be $n$ circles of equal radius
  $r$. Suppose each of the $n$ circles touches the circle $G$ externally. Also, for $i = 1, 2, \ldots, n -
  1$ the circle $G_i$ touches $G_{i + 1}$ externally, and $G_n$ touches $G_1$ externally. Then which of the
  following statements are true?
  \startitemize[a, 1*broad]
  \item If $n = 4$, then $\left(\sqrt{2} - 1\right)r < R$
  \item If $n = 5$, then $r < R$
  \item If $n = 8$, then $\left(\sqrt{2} - 1\right)r < R$
  \item If $n = 12$, then $\sqrt{2}\left(\sqrt{3} + 1\right)r > R$
  \stopitemize
  %285
\item For a $\triangle ABC$, the value of $\cos2A + \cos2B + \cos2C$ is least. If its inradius is $3$, then
  which of the following is not ccorrect?
  \startitemize[a, 1*broad]
  \item Perimeter of $\triangle ABC$ is $18\sqrt{3}$
  \item $\sin2A + \sin2B + \sin2C = \sin A + \sin B + \sin C$
  \item $MA.MB = -18$
  \item $\Delta ABC = \frac{27\sqrt{3}}{2}$
  \stopitemize
  %286
\item Let $S$ be the set of all solutions of the equation $\cos^{-1}2x - 2\cos^{-1}\sqrt{1 - x^2} = \pi,
  x\in\left[-\frac{1}{2}, \frac{1}{2}\right]$. Find $\displaystyle\sum_{x\in S}2\sin^{-1}\left(x^2 -
  1\right)$.
  %287
\item Let $f(x) = \startdeterminant\NC 1+ \sin^2x\NC \cos^2x\NC\sin2x\NR\NC \sin^2x\NC 1
  + \cos^2x\NC\sin2x \NR\NC \sin^2x\NC \cos^2x\NC 1 + \sin2x\NR\stopdeterminant,
  x\in\left[\frac{\pi}{6}, \frac{\pi}{3}\right]$. If $\alpha, \beta$ respectively are the maximum and
  minimum values of $f$ then find them.
  %288
\item Let $S = \left\{x\in\mathbb{R}: 0 < x < 1, 2\tan^{-1}\left(\frac{1 - x}{1 + x}\right)=
  cos^{-1}\left(\frac{1 - x^2}{1 + x^2}\right)\right\}$. Show that there is only one element in $S$ and it
  is less than $\frac{1}{2}$.
  %289
\item A line segment $AB$ of length $\lambda$ moves such that the points $A$ and $B$ remain on the periphery
  of a circle of radius $\lambda$. Then the locus of the point, that divides the line segment $AB$ in the
  ratio $2 : 3$ is a circle of what radius?
  %290
\item Find the value of
  $96\cos\frac{\pi}{33}\cos\frac{2\pi}{33}\cos\frac{4\pi}{33}\cos\frac{8\pi}{33}\cos\frac{16\pi}{33}$.
  %291
\item Let $S = \left\{x\in\left(-\frac{\pi}{2}, \frac{\pi}{2}\right): 9^{1 - \tan^2x} + p^{\tan^2x} =
  10\right\}$ and $b = \displaystyle\sum_{x\in S}\tan^2\left(\frac{x}{3}\right)$ then find $\frac{1}{6}(b -
  14)^2$.
  %292
\item In the figure $\theta_1 + \theta_2 = \frac{\pi}{2}$ and $\sqrt{3}BE = 4AB$. If the area of $\triangle
  ABC$ is $2\sqrt{3} - 3$ sq.\ units, when $\frac{\theta_2}{\theta_1}$ is the largest, then find the
  perimeter of $\triangle CDE$.

  \startplacefigure
    \externalfigure[17_32.pdf]
  \stopplacefigure
  %293
\item Find the number of elements in the set $S = \left\{\theta\in[0, 2\pi]: 4\cos^4\theta - 5\cos^{2}\theta -
  2\sin^6\theta + 2 = 0\right\}$.
  %294
\item In a $\triangle ABC$, if $\cos A + 2\cos B + \cos C = 2$ and the lengths of sides opposite to the
  angles $A$ and $C$ are $3$ and $7$ respectively, then find $\cos A - \cos C$.
  %295
\item Let $S = \left\{\theta\in[0, 2\pi]: \tan(\pi\cos\theta) + \tan(\pi\sin\theta) = 0\right\}$. Then find
  $\displaystyle\sum_{\theta\in S}\sin^{2}\left(\theta + \frac{\pi}{4}\right)$.
  %296
\item Find the set of all values of $k$ for which the equation $\cos^{2}2x - 2\sin^4x - 2\cos^{2}x = k$
  holds.
  %297
\item If the solution of the equation $\log_{\cos x}\cot x + 4\log_{\sin x}\tan x = 1,
  x\in\left(0, \frac{\pi}{2}\right)$ is $\sin^{-1}\frac{\alpha + \sqrt{\beta}}{2}$, where $\alpha, \beta$
  are integers, them find $\alpha + \beta$.
  %298
\item Let $(a, b)\subset(0, 2\pi)$ be the largest interval for which $\sin^{-1}(\sin\theta)
  - \cos^{-1}(\sin\theta) > 0, \theta\in(0, 2\pi)$ holds. If $ax^{2} + \beta x + \sin^{-1}\left(x^{2} - 6x
  + 10\right) + \cos^{-1}\left(x^{2} - 6x + 10\right) = 0$ and $\alpha - \beta = b - a$, then find $\alpha$.
  %299
\item Let $\tan^{-1}x\in\left(-\frac{\pi}{2}, \frac{\pi}{2}\right)$ for $x\in\mathbb{R}$. Find the number of
  real solutions of the equaiton $\sqrt{1 + \cos2x} = \sqrt{2}\tan^{-1}(\tan x)$ in the set
  $\left(-\frac{3\pi}{2},
  -\frac{\pi}{2}\right)\cup\left(-\frac{\pi}{2}, \frac{\pi}{2}\right)\cup \left(\frac{\pi}{2},
  \frac{3\pi}{2}\right)$.
  %300
\item Let $A_1, A_2, A-3, \ldots, A_8$ be be the vertices of a regular octagon that lie on a circle of
  radius $2$. Let $P$ be a point on the circle and let $PA_i$ denote the distance between the points $P$ and
  $A_i$ or $i = 1,2,\ldots,8$. If $P$ varies over the circle, then find the maximum value of the product
  $PA_1.PA_2.\ldots.PA_8$.
  %301
\item Consider an obtuse angled triangle $ABC$ in which the difference between the largest and the smallest
  angles is $\frac{\pi}{2}$ and whose sides are in A.P. Suppose that the vertices of the triangle lie on the
  circle of radius $1$. Let $a$ be the area of the triangle. Find $(64a)^{2}$. Also find the inradius of the
  incircle.
  %302
\item Let $S = \left\{\sin^{2}2\theta: \left(\sin^4\theta + \cos^4\theta\right)x^{2} + (\sin2\theta)x
  + \left(\sin^6\theta + \cos^6\theta\right) = 0 \text{ has real root}\right\}$. If $\alpha$ and $\beta$ be
  the smallest and largest elements of the set $S$, respectively, then find them.
  %303
\item Let a rectangle $ABCD$ of sides $2$ and $4$ inscribed in another rectangle $PQRS$ such that the
  vertices of the rectangle $ABCD$ lie on the sides of the rectangle $PQRS$. Let $a$ and $b$ be the sides of
  the rectangle $PQRS$ when its area is maximum. Then find $(a + b)^{2}$.
\stopitemize
