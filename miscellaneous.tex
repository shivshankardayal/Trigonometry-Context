% -*- mode: context; -*-
\chapter{Miscellaneous Problems}
\startitemize[n, 1*broad]
  %1
\item If $\pi < x < 2\pi$, prove that $\frac{\sqrt{1 + \cos x} + \sqrt{1 - \cos x}}{\sqrt{1 + \cos x} -
    \sqrt{1 - \cos x}} = \cot\left(\frac{x}{2} + \frac{\pi}{4}\right)$.
  %2
\item If $\sin(A - B) = \frac{1}{\sqrt{10}}, \cos(A + B) = \frac{2}{\sqrt{29}}$, find the value of $tan2A$,
  where $A$ and $B$ lie between $0$ and $\frac{\pi}{4}$.
  %3
\item Suppose that $\sin^3x\sin3x = \displaystyle\sum_{m = 0}^nc_m\cos mx$ is an identity in $x$, where
  $c_0, c_1, c_2, \ldots, c_n$ are constants and $c_n \neq 0$. Find the value of $n$.
  %4
\item Show that $\sin^4\frac{\pi}{8} + \sin^4\frac{3\pi}{8} + \sin^4\frac{\5pi}{8} + \sin^4\frac{7\pi}{8}
  = \frac{3}{2}$.
  %5
\item Evaluate $\displaystyle\sum_{p = 1}^{32}(3p + 2)\left[\displaystyle\sum_{q = 1
  }^{10}\sin\frac{2q\pi}{11} - i\cos\frac{2q\pi}{11}\right]^p$.
  %6
\item If $\theta = \frac{\pi}{2^n - 1}$, prove that $2^n\cos\theta\cos2\theta\cos2^2\theta\ldots\cos2^{n -
  1}\theta = -1$
  %7
\item Show that $\cos\frac{2\pi}{7}\cos\frac{4\pi}{7}\cos\frac{6\pi}{7} = \frac{1}{8}$.
  %8
\item Prove that $\sin\frac{2\pi}{7} + \sin\frac{4\pi}{7} + \sin\frac{8\pi}{7} = \frac{\sqrt{7}}{2}$.
  %9
\item Show that $\tan^2\frac{\pi}{16} + \tan^2\frac{2\pi}{16} + \cdots + \tan^2\frac{7\pi}{16} = 35$.
  %10
\item Prove that $\tan\frac{\pi}{7}\tan\frac{2\pi}{7}\tan\frac{3\pi}{7} = \sqrt{7}$.
  %11
\item Show that $\left(\tan^2\frac{\pi}{7} + \tan^2\frac{2\pi}{7} + \tan^2\frac{3\pi}{7}\right)
  + \left(\cot^2\frac{\pi}{7} + \cot^2\frac{2\pi}{7} + \cot^2\frac{3\pi}{7}\right) = 105$.
  %12
\item If $0 < x < \frac{\pi}{2}$, prove that $\sqrt{\tan x + \sin x} + \sqrt{\tan x - \sin x} = 2\sqrt{\tan
  x}\cos\left(\frac{\pi}{4} - \frac{x}{2}\right)$.
  %13
\item If $\sin^3x\sin3x = \displaystyle\sum_{n = 0}^6\cos^nx$, where $c_0, c_1, \ldots, c_6$ are constants,
  then find the value of $c_4$.
  %14
\item Find the value of $\sin7^\circ + \sin77^\circ + \sin149^\circ + \cdots + \sin293^\circ$.
  %15
\item Prove that $\cos\frac{\pi}{n} + \cos\frac{2\pi}{n} + \cos\frac{3\pi}{n} + \cdots + \cos\frac{(n -
  1)\pi}{n} = 0$.
  %16
\item Show that $\cos\frac{2\pi}{7} + \cos\frac{4\pi}{7} + \cos\frac{6\pi}{7} = -\frac{1}{2}$.
  %17
\item Show that $3\left[\sin^4\left(\frac{3\pi}{2} - \alpha\right) + \sin^4(3\pi + \alpha)\right] -
  2\left[\sin^6\left(\frac{\pi}{2} + \alpha\right) + \sin^6(5\pi - \alpha)\right] = 1$.
  %18
\item Prove that $\sin36^\circ\sin72^\circ\sin108^\circ\sin144^\circ = \frac{5}{16}$.
  %19
\item Prove that $\sin^212^\circ + \sin^221^\circ + \sin^239^\circ + \sin^248^\circ = 1 + \sin^29^\circ
  + \sin^218^\circ$.
  %20
\item If $270^\circ\leq \alpha\leq 450^\circ$, express $\cos\frac{\alpha}{2}$ and $\sin\frac{\alpha}{2}$ in
  terms of $\sin\alpha$.
  %21
\item Prove that $\tan142^\circ30' = 2 + \sqrt{2} - \sqrt{3} - \sqrt{6}$.
  %22
\item Find $\tan\frac{\alpha}{2}$ if $\sin\alpha + \cos\alpha = \frac{\sqrt{7}}{2}$ and the angle $\alpha$
  lies between $0^\circ$ and $45^\circ$.
  %23
\item Prove that if $\frac{\sin(x - \alpha)}{\sin(x - \beta)} = \frac{a}{b}$ and $\frac{\cos(x
  - \alpha)}{\cos(x - \beta)} = \frac{A}{B}$ and $aB + bA\neq 0$ then $\cos(\alpha - \beta) = \frac{aA +
  bB}{aB + bA}$.
  %24
\item If the sum of three positive numbers $\alpha, \beta, \gamma$ is equal to $\frac{\pi}{2}$. Find the
  product $\cot\alpha\cot\gamma$ if $\cot\alpha, \cot\beta, \cot\gamma$ are in A.P.
  %25
\item Show that $16\cos\frac{2\pi}{15}\cos\frac{4\pi}{15}\cos\frac{8\pi}{15}\cos\frac{16\pi}{15} = 1$.
  %26
\item If $\cos(\alpha + \beta) = \frac{4}{5}, \sin(\alpha - \beta) = \frac{5}{13}$ and $\alpha$ and $\beta$
  lie between $0$ and $\frac{\pi}{4}$, find $\tan2\alpha$.
  %27
\item Sum the series $\tan\alpha\tan(\alpha + \beta) + \tan(\alpha + \beta)\tan(\alpha + 2\beta)
  + \tan(\alpha + 2\beta)\tan(\alpha + 3\beta) + \cdots$ to $n$ terms.
\stopitemize
